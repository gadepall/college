\renewcommand{\theequation}{\theenumi}
\renewcommand{\thefigure}{\theenumi}
\begin{enumerate}[label=\thesection.\arabic*.,ref=\thesection.\theenumi]
\numberwithin{equation}{enumi}
\numberwithin{figure}{enumi}

\item Consider the subspaces $W_1$ and $W_2$ of $\mathbb{R}^3$ given by
\begin{align}
W_1 &= \cbrak{\vec{x} \in \mathbb{R}^3: \myvec{1 & 1 & 1}\vec{x} = 0}
\\
W_2 &= \cbrak{\vec{x} \in \mathbb{R}^3: \myvec{1 & -1 & 1}\vec{x} = 0}.
\end{align}
If $W \subseteq \mathbb{R}^3$, such that 
\begin{enumerate}
\item $W \cap W_2 =$ span $\cbrak{\myvec{0\\1\\1}}$
\item $\cbrak{W \cap W_1} \perp \cbrak{W \cap W_2}$, 
\end{enumerate}
then 
\begin{enumerate}
\item $W =$ span $\cbrak{\myvec{0\\1\\-1},\myvec{0\\1\\1}}$
\item $W =$ span $\cbrak{\myvec{1\\0\\-1},\myvec{0\\1\\-1}}$
\item $W =$ span $\cbrak{\myvec{1\\0\\-1},\myvec{0\\1\\1}}$
\item $W =$ span $\cbrak{\myvec{1\\0\\-1},\myvec{1\\0\\1}}$
\end{enumerate}
\item Let
\begin{align}
C = \cbrak{\myvec{1 \\ 2},\myvec{2 \\ 1}}
\end{align}
be a basis of $\mathbb{R}^2$ and 
\begin{align}
T\myvec{x\\y} = \myvec{x+y \\ x-2y}.
\end{align}
If $T\sbrak{C}$ represents the matrix of $T$ with respect to the basis C then
which among the following is true?
\begin{enumerate}
\item $T\sbrak{C} = \myvec{-3 & -2\\3 & 1}$
\item $T\sbrak{C} = \myvec{3 & -2\\-3 & 1}$
\item $T\sbrak{C} = \myvec{-3 & -1\\3 & 2}$
\item $T\sbrak{C} = \myvec{3 & -1\\-3 & 2}$
\end{enumerate}
\item Let $W_1 = \cbrak{\vec{x} \in \mathbb{R}^4:}$
\begin{align}
 \myvec{1 & 1 & 1 & 0}\vec{x} = 0
\\
 \myvec{0 & 2 & 0 & 1}\vec{x} = 0
\\
 \myvec{2 & 0 & 2 & -1}\vec{x} = 0
\end{align}
and
$W_2 = \cbrak{\vec{x} \in \mathbb{R}^4:}$
\begin{align}
 \myvec{1 & 1 & 0 & 1}\vec{x} &= 0
\\
 \myvec{1 & 0 & 1 & -2}\vec{x} &= 0
\\
 \myvec{0 & 1 & 0 & -1}\vec{x} &= 0.
\end{align}
Then which among the following is true?
\begin{enumerate}
\item $\text{dim}\brak{W_1} = 1$
\item $\text{dim}\brak{W_2} = 2$
\item $\text{dim}\brak{W_1 \cap W_2} = 1$
\item $\text{dim}\brak{W_1+W_2} = 3$
\end{enumerate}
%
\item Let $A$ be an $n \times n$ complex matrix.  Assume that $A$ is self-adjoint and let $B$ denote the inverse of $A + \j I$. Then all eigenvalues of $\brak{A-\j I}B$ are 
\begin{enumerate}
\item purely imaginary
\item of modulus one
\item real
\item of modulus less than one
\end{enumerate}  
%
\item Let $\cbrak{u_1,u_2,\dots, u_n}$ be an orthonormal basis of $\mathbb{C}^n$ as column vectors.Let 
\begin{align}
\vec{M} &= \myvec{\vec{u}_1 & \vec{u}_2 & \dots & \vec{u}_k},
\\
\vec{N} &= \myvec{\vec{u}_{k+1} & \vec{u}_{k+2} & \dots & \vec{u}_n}
\end{align}
%
and $\vec{P}$ be the diagonal $k \times k$ matrix with diagonal entries $\alpha_1,\alpha_2, \dots, \alpha_k \in \mathbb{R}$.  Then which of the following is true?
\begin{enumerate}
\item rank$\brak{\vec{M}\vec{P}\vec{M}^*} = k$ whenever $\alpha_i \ne \alpha_j, 1 \le i, j \le k$.
\item tr$\brak{\vec{M}\vec{P}\vec{M}^*} = \sum_{i=1}^{k}\alpha_i$
\item rank$\brak{\vec{M}^*\vec{N}} = \min\brak{k,n-k}$
\item rank$\brak{\vec{M}\vec{M}^*+\vec{N}\vec{N}^*}  < n$.
\end{enumerate}  
%
\item Let $B: \mathbb{R} \times \mathbb{R} \to \mathbb{R}$ be the function
\begin{align}
B(a,b) = ab
\end{align}
Which of the following is true?
\begin{enumerate}
\item $B$ is a linear transformation
\item $B$ is a positive definite bilinear form
\item $B$ is symmetric but not positive definite
\item $B$ is neither linear nor bilinear
\end{enumerate}  

%\item Consider a Markov Chain with state space $\cbrak{0,1,2}$ and transition matrix
%\begin{align}
%P = 
%\begin{blockarray}{c@{\hspace{1pt}}rrr@{\hspace{3pt}}}
%         & 0   & 1   & 2 \\
%        \begin{block}{r@{\hspace{3pt}}@{\hspace{1pt}}
%    (@{\hspace{1pt}}rrr@{\hspace{1pt}}@{\hspace{1pt}})}
%        0 & \frac{1}{4} & \frac{5}{8} & \frac{1}{8}  \\
%        1 & \frac{1}{4} & 0 & \frac{3}{4}  \\
%%
%        2 & 0 & \frac{1}{2} & \frac{3}{8} & \frac{1}{8}  \\
%        \end{block}
%    \end{blockarray}
%\end{align}
%Then which of the following are true?
\end{enumerate}
