\renewcommand{\theequation}{\theenumi}
\renewcommand{\thefigure}{\theenumi}
\begin{enumerate}[label=\thesection.\arabic*.,ref=\thesection.\theenumi]
\numberwithin{equation}{enumi}
\numberwithin{figure}{enumi}

\item Let $\vec{A}$ be a $\brak{m \times n}$ matrix 
and $\vec{B}$ be a $\brak{n \times m}$ matrix over real numbers with $m < n$.  Then
\begin{enumerate}
\item $\vec{A}\vec{B}$ is always nonsingular.
\item $\vec{A}\vec{B}$ is always singular.
\item $\vec{B}\vec{A}$ is always nonsingular.
\item $\vec{B}\vec{A}$ is always singular.
\end{enumerate}
%
\item If $\vec{A}$ is a $\brak{2\times 2}$ matrix over $\mathbb{R}$ with $det\brak{\vec{A}+\vec{I}} 
= 1 + det\brak{\vec{A}}$.  Then we can conclude that
\begin{enumerate}
\item $det\brak{\vec{A}} = 0$.
\item $\vec{A} = 0$.
\item $tr\brak{\vec{A}} = 0$.
\item $\vec{A}$ is nonsingular.
\end{enumerate}
%
\item The system of equations
\begin{align}
x+2x^2+3xy = 6 \\
x+x^2+3xy + y = 5 \\
x-x^2+y = 7
\end{align}
\begin{enumerate}
\item has solutions in rational numbers.
\item has solutions in real numbers.
\item has solutions in complex numbers.
\item has no solutions.
\end{enumerate}
%
\item The trace of the matrix
\begin{align}
\myvec
{
2 & 1 & 0
\\
0 & 2 & 0
\\
0 & 0 & 3
}^{20}
\end{align}
is
\begin{enumerate}
\item $7^{20}$.
\item $2^{20}+3^{20}$.
\item $2^{21}+3^{20}$.
\item $2^{20}+3^{20}+1$.
\end{enumerate}
%
\item Given that there are real constants $a,b,c,d$ such that the identity
\begin{multline}
\lambda x^2 + 2xy + y^2 = \brak{ax+by}^2 + \brak{cx+dy}^2, 
\\
 \forall x,y \in \mathbb{R}
\end{multline}
This implies that
\begin{enumerate}
\item $\lambda = -5$
\item $\lambda \ge 1$
\item $0 < \lambda < 1$
\item There is no such $\lambda \in \mathbb{R}$
\end{enumerate}
%
\item Let $\mathbb{R}, n \ge 2$, be equipped with the standard inner product.  Let
$\vec{v}_1,\vec{v}_2,\dots,\vec{v}_n$ be $n$ column vectors forming an orthonormal
basis of $\mathbb{R}^n$.  Let $A$ be the $n \times n$ matrix formed by the column vectors
$\vec{v}_1,\vec{v}_2,\dots,\vec{v}_n$.  Then 
\begin{enumerate}
\begin{multicols}{2}
\item $\vec{A}=\vec{A}^{-1}$
\item $\vec{A}=\vec{A}^{\top}$
\item $\vec{A}^{-1}=\vec{A}^{\top}$
\item $det\brak{\vec{A}}=1$
\end{multicols}
\end{enumerate}

\item Consider a Markov Chain with state space $\cbrak{0,1,2}$ and transition matrix
\begin{align}
P = 
\begin{blockarray}{c@{\hspace{1pt}}rrr@{\hspace{3pt}}}
         & 0   & 1   & 2 \\
        \begin{block}{r@{\hspace{3pt}}@{\hspace{1pt}}
    (@{\hspace{1pt}}rrr@{\hspace{1pt}}@{\hspace{1pt}})}
        0 & \frac{1}{2} & \frac{1}{2} & 0  \\
        1 & 0 &\frac{1}{2}  & \frac{3}{4}  \\
%
        2 &  \frac{1}{3} & \frac{1}{3} & \frac{1}{3}  \\
        \end{block}
    \end{blockarray}
\end{align}
For any two states $i$ and $j$, let $p_{ij}^{(n)}$ denote the $n$-step transition probability of going from $i$ to $j$.  Identify correct statements.
\begin{enumerate}
\item $\lim_{n \to \infty} p_{11}^{(n)} = \frac{2}{9}$
\item $\lim_{n \to \infty} p_{21}^{(n)} = 0$
\item $\lim_{n \to \infty} p_{32}^{(n)} = \frac{1}{3}$
\item $\lim_{n \to \infty} p_{13}^{(n)} = \frac{1}{3}$
\end{enumerate}

\end{enumerate}
