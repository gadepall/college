\documentclass[journal,12pt,twocolumn]{IEEEtran}
%
\usepackage{setspace}
\usepackage{gensymb}
%\doublespacing
\singlespacing

%\usepackage{graphicx}
%\usepackage{amssymb}
%\usepackage{relsize}
\usepackage[cmex10]{amsmath}
\usepackage{siunitx}
%\usepackage{amsthm}
%\interdisplaylinepenalty=2500
%\savesymbol{iint}
%\usepackage{txfonts}
%\restoresymbol{TXF}{iint}
%\usepackage{wasysym}
\usepackage{amsthm}
\usepackage{iithtlc}
\usepackage{mathrsfs}
\usepackage{txfonts}
\usepackage{stfloats}
\usepackage{steinmetz}
\usepackage{supertabular}
%\usepackage{bm}
\usepackage{cite}
\usepackage{cases}
\usepackage{subfig}
%\usepackage{xtab}
\usepackage{longtable}
\usepackage{multirow}
%\usepackage{algorithm}
%\usepackage{algpseudocode}
\usepackage{enumitem}
\usepackage{mathtools}
\usepackage{tikz}
\usepackage{circuitikz}
\usepackage{verbatim}
\usepackage{tfrupee}
\usepackage[breaklinks=true]{hyperref}
%\usepackage{stmaryrd}
\usepackage{tkz-euclide} % loads  TikZ and tkz-base
%\usetkzobj{all}
\usetikzlibrary{calc,math}
\usetikzlibrary{fadings}
\usepackage{listings}
    \usepackage{color}                                            %%
    \usepackage{array}                                            %%
    \usepackage{longtable}                                        %%
    \usepackage{calc}                                             %%
    \usepackage{multirow}                                         %%
    \usepackage{hhline}                                           %%
    \usepackage{ifthen}                                           %%
  %optionally (for landscape tables embedded in another document): %%
    \usepackage{lscape}     
\usepackage{multicol}
\usepackage{chngcntr}
\usepackage{blkarray}
%\usepackage{enumerate}

%\usepackage{wasysym}
%\newcounter{MYtempeqncnt}
\DeclareMathOperator*{\Res}{Res}
%\renewcommand{\baselinestretch}{2}
\renewcommand\thesection{\arabic{section}}
\renewcommand\thesubsection{\thesection.\arabic{subsection}}
\renewcommand\thesubsubsection{\thesubsection.\arabic{subsubsection}}

\renewcommand\thesectiondis{\arabic{section}}
\renewcommand\thesubsectiondis{\thesectiondis.\arabic{subsection}}
\renewcommand\thesubsubsectiondis{\thesubsectiondis.\arabic{subsubsection}}

% correct bad hyphenation here
\hyphenation{op-tical net-works semi-conduc-tor}
\def\inputGnumericTable{}                                 %%

\lstset{
%language=C,
frame=single, 
breaklines=true,
columns=fullflexible
}
%\lstset{
%language=tex,
%frame=single, 
%breaklines=true
%}

\begin{document}
%


\newtheorem{theorem}{Theorem}[section]
\newtheorem{problem}{Problem}
\newtheorem{proposition}{Proposition}[section]
\newtheorem{lemma}{Lemma}[section]
\newtheorem{corollary}[theorem]{Corollary}
\newtheorem{example}{Example}[section]
\newtheorem{definition}[problem]{Definition}
%\newtheorem{thm}{Theorem}[section] 
%\newtheorem{defn}[thm]{Definition}
%\newtheorem{algorithm}{Algorithm}[section]
%\newtheorem{cor}{Corollary}
\newcommand{\BEQA}{\begin{eqnarray}}
\newcommand{\EEQA}{\end{eqnarray}}
\newcommand{\define}{\stackrel{\triangle}{=}}

\bibliographystyle{IEEEtran}
%\bibliographystyle{ieeetr}


\providecommand{\mbf}{\mathbf}
\providecommand{\pr}[1]{\ensuremath{\Pr\left(#1\right)}}
\providecommand{\qfunc}[1]{\ensuremath{Q\left(#1\right)}}
\providecommand{\sbrak}[1]{\ensuremath{{}\left[#1\right]}}
\providecommand{\lsbrak}[1]{\ensuremath{{}\left[#1\right.}}
\providecommand{\rsbrak}[1]{\ensuremath{{}\left.#1\right]}}
\providecommand{\brak}[1]{\ensuremath{\left(#1\right)}}
\providecommand{\lbrak}[1]{\ensuremath{\left(#1\right.}}
\providecommand{\rbrak}[1]{\ensuremath{\left.#1\right)}}
\providecommand{\cbrak}[1]{\ensuremath{\left\{#1\right\}}}
\providecommand{\lcbrak}[1]{\ensuremath{\left\{#1\right.}}
\providecommand{\rcbrak}[1]{\ensuremath{\left.#1\right\}}}
\theoremstyle{remark}
\newtheorem{rem}{Remark}
\newcommand{\sgn}{\mathop{\mathrm{sgn}}}
\providecommand{\abs}[1]{\left\vert#1\right\vert}
\providecommand{\res}[1]{\Res\displaylimits_{#1}} 
\providecommand{\norm}[1]{\left\lVert#1\right\rVert}
%\providecommand{\norm}[1]{\lVert#1\rVert}
\providecommand{\mtx}[1]{\mathbf{#1}}
\providecommand{\mean}[1]{E\left[ #1 \right]}
\providecommand{\fourier}{\overset{\mathcal{F}}{ \rightleftharpoons}}
%\providecommand{\hilbert}{\overset{\mathcal{H}}{ \rightleftharpoons}}
\providecommand{\system}{\overset{\mathcal{H}}{ \longleftrightarrow}}
	%\newcommand{\solution}[2]{\textbf{Solution:}{#1}}
\newcommand{\solution}{\noindent \textbf{Solution: }}
\newcommand{\cosec}{\,\text{cosec}\,}
\providecommand{\dec}[2]{\ensuremath{\overset{#1}{\underset{#2}{\gtrless}}}}
\newcommand{\myvec}[1]{\ensuremath{\begin{pmatrix}#1\end{pmatrix}}}
\newcommand{\mydet}[1]{\ensuremath{\begin{vmatrix}#1\end{vmatrix}}}
%\numberwithin{equation}{section}
\numberwithin{equation}{subsection}
%\numberwithin{problem}{section}
%\numberwithin{definition}{section}
\makeatletter
\@addtoreset{figure}{problem}
\makeatother

\let\StandardTheFigure\thefigure
\let\vec\mathbf
%\renewcommand{\thefigure}{\theproblem.\arabic{figure}}
\renewcommand{\thefigure}{\theproblem}
%\setlist[enumerate,1]{before=\renewcommand\theequation{\theenumi.\arabic{equation}}
%\counterwithin{equation}{enumi}


%\renewcommand{\theequation}{\arabic{subsection}.\arabic{equation}}

\def\putbox#1#2#3{\makebox[0in][l]{\makebox[#1][l]{}\raisebox{\baselineskip}[0in][0in]{\raisebox{#2}[0in][0in]{#3}}}}
     \def\rightbox#1{\makebox[0in][r]{#1}}
     \def\centbox#1{\makebox[0in]{#1}}
     \def\topbox#1{\raisebox{-\baselineskip}[0in][0in]{#1}}
     \def\midbox#1{\raisebox{-0.5\baselineskip}[0in][0in]{#1}}

\vspace{3cm}

\title{
	\logo{
Linear Algebra
	}
}
\author{ G V V Sharma$^{*}$% <-this % stops a space
	\thanks{*The author is with the Department
		of Electrical Engineering, Indian Institute of Technology, Hyderabad
		502285 India e-mail:  gadepall@iith.ac.in. All content in this manual is released under GNU GPL.  Free and open source.}
	
}	
%\title{
%	\logo{Matrix Analysis through Octave}{\begin{center}\includegraphics[scale=.24]{tlc}\end{center}}{}{HAMDSP}
%}


% paper title
% can use linebreaks \\ within to get better formatting as desired
%\title{Matrix Analysis through Octave}
%
%
% author names and IEEE memberships
% note positions of commas and nonbreaking spaces ( ~ ) LaTeX will not break
% a structure at a ~ so this keeps an author's name from being broken across
% two lines.
% use \thanks{} to gain access to the first footnote area
% a separate \thanks must be used for each paragraph as LaTeX2e's \thanks
% was not built to handle multiple paragraphs
%

%\author{<-this % stops a space
%\thanks{}}
%}
% note the % following the last \IEEEmembership and also \thanks - 
% these prevent an unwanted space from occurring between the last author name
% and the end of the author line. i.e., if you had this:
% 
% \author{....lastname \thanks{...} \thanks{...} }
%                     ^------------^------------^----Do not want these spaces!
%
% a space would be appended to the last name and could cause every name on that
% line to be shifted left slightly. This is one of those "LaTeX things". For
% instance, "\textbf{A} \textbf{B}" will typeset as "A B" not "AB". To get
% "AB" then you have to do: "\textbf{A}\textbf{B}"
% \thanks is no different in this regard, so shield the last } of each \thanks
% that ends a line with a % and do not let a space in before the next \thanks.
% Spaces after \IEEEmembership other than the last one are OK (and needed) as
% you are supposed to have spaces between the names. For what it is worth,
% this is a minor point as most people would not even notice if the said evil
% space somehow managed to creep in.



% The paper headers
%\markboth{Journal of \LaTeX\ Class Files,~Vol.~6, No.~1, January~2007}%
%{Shell \MakeLowercase{\textit{et al.}}: Bare Demo of IEEEtran.cls for Journals}
% The only time the second header will appear is for the odd numbered pages
% after the title page when using the twoside option.
% 
% *** Note that you probably will NOT want to include the author's ***
% *** name in the headers of peer review papers.                   ***
% You can use \ifCLASSOPTIONpeerreview for conditional compilation here if
% you desire.




% If you want to put a publisher's ID mark on the page you can do it like
% this:
%\IEEEpubid{0000--0000/00\$00.00~\copyright~2007 IEEE}
% Remember, if you use this you must call \IEEEpubidadjcol in the second
% column for its text to clear the IEEEpubid mark.



% make the title area
\maketitle

\newpage

\tableofcontents

\bigskip

\renewcommand{\thefigure}{\theenumi}
\renewcommand{\thetable}{\theenumi}
%\renewcommand{\theequation}{\theenumi}

%\begin{abstract}
%%\boldmath
%In this letter, an algorithm for evaluating the exact analytical bit error rate  (BER)  for the piecewise linear (PL) combiner for  multiple relays is presented. Previous results were available only for upto three relays. The algorithm is unique in the sense that  the actual mathematical expressions, that are prohibitively large, need not be explicitly obtained. The diversity gain due to multiple relays is shown through plots of the analytical BER, well supported by simulations. 
%
%\end{abstract}
% IEEEtran.cls defaults to using nonbold math in the Abstract.
% This preserves the distinction between vectors and scalars. However,
% if the journal you are submitting to favors bold math in the abstract,
% then you can use LaTeX's standard command \boldmath at the very start
% of the abstract to achieve this. Many IEEE journals frown on math
% in the abstract anyway.

% Note that keywords are not normally used for peerreview papers.
%\begin{IEEEkeywords}
%Cooperative diversity, decode and forward, piecewise linear
%\end{IEEEkeywords}



% For peer review papers, you can put extra information on the cover
% page as needed:
% \ifCLASSOPTIONpeerreview
% \begin{center} \bfseries EDICS Category: 3-BBND \end{center}
% \fi
%
% For peerreview papers, this IEEEtran command inserts a page break and
% creates the second title. It will be ignored for other modes.
%\IEEEpeerreviewmaketitle

\begin{abstract}
This book provides solved examples on Linear Algebra.
\end{abstract}


\section{June 2019}
\renewcommand{\theequation}{\theenumi}
\renewcommand{\thefigure}{\theenumi}
\begin{enumerate}[label=\thesection.\arabic*.,ref=\thesection.\theenumi]
\numberwithin{equation}{enumi}
\numberwithin{figure}{enumi}


\item Consider the vector space $\mathbb{P}_n$  of real polynomials in $x$ of degree  $\le n$. Define
\begin{align}
T: \mathbb{P}_2 \to \mathbb{P}_3
\end{align}
by
\begin{align}
\brak{Tf}(x) = \int_{0}^{x}f(t)\,dt + f^{\prime}(x).
\end{align}
%
Then find the matrix representation of $T$ with respect to the bases
\begin{align}
\cbrak{1,x,x^2} \text{ and } \cbrak{1,x,x^2,x^3}
\end{align}
%
\item Let $P_A(x)$ denote the characteristic polynomial of a matrix $A$.  Then for which of the following matrices is
\begin{align}
P_A(x)-P_{A^{-1}}(x)
\end{align}
a constant?
\begin{enumerate}
\begin{multicols}{2}
\item $\myvec{3 & 3 \\ 2 & 4}$
\item $\myvec{4 & 3 \\ 2 & 3}$
\item $\myvec{3 & 2 \\ 4 & 3}$
\item $\myvec{2 & 3 \\ 3 & 4}$
\end{multicols}
\end{enumerate}
\item Which of the following matrices is not diagonalizable over $\mathbb{R}$?
\begin{enumerate}
\begin{multicols}{2}
\item $\myvec{2 & 0 & 1 \\ 0 & 3 & 0 \\ 0 & 0 & 2}$
\item $\myvec{1 & 1 \\ 1 & 1}$
\item $\myvec{2 & 0 & 1 \\ 0 & 3 & 0 \\ 0 & 0 & 3}$
\item $\myvec{1 & -1 \\ 2 & 4}$
\end{multicols}
\end{enumerate}
\item What is the rank of the following matrix?
\begin{align}
\myvec
{
1 & 1 & 1 & 1 & 1 \\
1 & 2 & 2 & 2 & 2 \\
1 & 2 & 3 & 3 & 3 \\
1 & 2 & 3 & 4 & 4 \\
1 & 2 & 3 & 4 & 5 
}
\end{align}
\item Let $V$ denote the vector space of real valued continuous functions on the close interval $\sbrak{0,1}$.  Let $W$ be the subspace of $V$ spanned by $\cbrak{\sin x, \cos x, \tan x}$.  Find the dimension of $W$ over $\mathbb{R}$.
\item Let $V$ be the vector space of polynomials in the variable $t$ of degree at most 2 over $\mathbb{R}$.  An inner product on $V$ is defined by
\begin{align}
f^Tg = \int_{0}^{1}f(t)g(t)\,dt, \quad f,g \in V.
\end{align}
Let
\begin{align}
W = span\cbrak{1-t^2,1+t^2}
\end{align}
and $W^{\perp}$  be the orthogonal complement of $W$ in $V$.  Which of the following conditions is satisfied for all $h \in W^{\perp}$?
\begin{enumerate}
\item $h$ is an even function
\item $h$ is an odd function
\item $h(t) =0$ has a real solution
\item $h(0) = 0$
\end{enumerate}
\item Consider solving the following system by Jacobi iteration scheme
\begin{align}
\myvec
{
1 & 2m & -2m \\
n & 1 & n \\
2m & 2m & 1
}
\myvec{x}
= \myvec{ 1 \\ 2 \\1}
\end{align}
where $m,n \in \mathbb{Z}$.  With any initial vector, the scheme converges provided $m,n$ satisfy
\begin{enumerate}
\begin{multicols}{2}
\item $m+n = 3$
\item $m > n$
\item $m < n$
\item $m = n$
\end{multicols}
\end{enumerate}
\item Consider a Markov Chain with state space $\cbrak{0,1,2,3,4}$ and transition matrix
\begin{align}
P = 
%\begin{blockarray}{c@{\hspace{1pt}}rrr@{\hspace{3pt}}}
%         & 0   & 1   & 2 \\
%        \begin{block}{r@{\hspace{3pt}}@{\hspace{1pt}}
%    (@{\hspace{1pt}}rrr@{\hspace{1pt}}@{\hspace{1pt}})}
%        0 & 0.7 & 0.2 & 0.1 \\
%        1 & 0.3 & 0.5 & 0.2 \\
%        2 & 0   & 0   & 0   \\
%        \end{block}
%    \end{blockarray}
\begin{blockarray}{c@{\hspace{1pt}}rrrrr@{\hspace{3pt}}}
         & 0   & 1   & 2 &3 & 4\\
        \begin{block}{r@{\hspace{3pt}}@{\hspace{1pt}}
    (@{\hspace{1pt}}rrrrr@{\hspace{1pt}}@{\hspace{1pt}})}
        0 & 1 & 0 & 0 & 0 & 0 \\
        1 & \frac{1}{3} & \frac{1}{3} & \frac{1}{3} & 0 & 0 \\
%
        2 & 0 & \frac{1}{3} & \frac{1}{3} & \frac{1}{3} & 0  \\
        3 & 0 & 0 & \frac{1}{3} & \frac{1}{3} & \frac{1}{3}   \\
        4 & 0 & 0 & 0 & 0 & 1 \\
        \end{block}
    \end{blockarray}
\end{align}
Then find 
\begin{align}
\lim_{n \to \infty} p_{23}^{\brak{n}}
\end{align}
\item Let $L\brak{\mathbb{R}}^n$ be the space of $\mathbb{R}-$linear maps from $\mathbb{R}^n$ to $\mathbb{R}^n$.  If $Ker(T)$ denotes the kernel of $T$ then which of the following are true?
\begin{enumerate}
\item There exists $T \in L\brak{\mathbb{R}^5}\ \cbrak{0}$ such that $Range(T) = Ker(T)$
\item There does not exist $T \in L\brak{\mathbb{R}^5}\ \cbrak{0}$ such that $Range(T) = Ker(T)$
\item There exists $T \in L\brak{\mathbb{R}^6}\ \cbrak{0}$ such that $Range(T) = Ker(T)$
\item There does not exist $T \in L\brak{\mathbb{R}^6}\ \cbrak{0}$ such that $Range(T) = Ker(T)$
\end{enumerate}
\item Let $V$ be a finite dimensional vector space over $\mathbb{R}$ and $T:V\to V$ be a linear map.   Can you always write $T = T_2 \circ T_1$ for some linear maps
\begin{align}
T_1:V\to W, T:W\to V,
\end{align}
where $W$ is some finite dimensional vector space such that\begin{enumerate}
\item both $T_1$ and $T_2$ are onto
\item both $T_1$ and $T_2$ are one to one
\item $T_1$ is onto, $T_2$ is one to one
\item $T_1$ is one to one, $T_2$ is onto
\end{enumerate}
\item Let $A = \sbrak{a_{ij}}$ be a $3\times 3$ complex matrix.  Identify the correct statements
\begin{enumerate}
\item $det\sbrak{\brak{-1}^{i+j}a_{ij}} = det(A)$
\item $det\sbrak{\brak{-1}^{i+j}a_{ij}} = -det(A)$
\item $det\sbrak{\brak{\sqrt{-1}}^{i+j}a_{ij}} = det(A)$
\item $det\sbrak{\brak{\sqrt{-1}}^{i+j}a_{ij}} = -det(A)$
\end{enumerate}
\item Let 
\begin{align}
p(x) = a_0 + a_1x + \dots + a_n x^n 
\end{align}
be a non-constant polynomial of degree $n \ge 1$.  Consider the polynomial
\begin{align}
q(x) = \int_{0}^{x}p(t)\, dt,
r(x) = \frac{d}{dx}p(x)
\end{align}
Let $V$ denote the real vector space of all polynomials in $x$.  Then which of the following are true?
\begin{enumerate}
\item $q$ and $r$ are linearly independent in $V$
\item $q$ and $r$ are linearly dependent in $V$
\item $x^n$ belongs to the linear span of $q$ and $r$
\item $x^{n+1}$ belongs to the linear span of $q$ and $r$.
\end{enumerate}
\item Let $M_n\brak{\mathbb{R}}$ be the ring of $n \times n$ matrices over $\mathbb{R}$.  Which of the following are true for every $n \ge 2$?
\begin{enumerate}
\item there exist matrices $A,B \in M_n\brak{\mathbb{R}}$ such that $AB- BA = I_n$, where $I_n$ denotes the identity matrix.
\item If $A,B \in M_n\brak{\mathbb{R}}$ and $AB = BA$, then $A$ is diagonalisable over $\mathbb{R}$ if and only if $B$ is diagonalisable over $\mathbb{R}$.
\item If $A,B \in M_n\brak{\mathbb{R}}$, then $AB$ and $BA$ have the same minimal polynomial.
\item If $A,B \in M_n\brak{\mathbb{R}}$, then $AB$ and $BA$ have the same eigenvalues in $\mathbb{R}$.
\end{enumerate}
\item Consider a matrix 
\begin{align}
A = \sbrak{a_{ij}}, 1 \le i,j \le 5
\end{align}
such that
\begin{align}
a_{ij} = \frac{1}{n_i+n_j+1}, \quad n_i,n_j \in \mathbb{N}
\end{align}
Then in which of the following cases $A$ is a positive definite matrix?
\begin{enumerate}
\item $n_i = 1 \forall i = 1,2,3,4,5$.
\item $n_1 < n_2 < \dots < n_5$.
\item $n_1 = n_2 = \dots = n_5$.
\item $n_1 > n_2 > \dots > n_5$.
\end{enumerate}
\item For a nonzero $w \in \mathbb{R}^n$, define
\begin{align}
T_w: \mathbb{R}^n \to \mathbb{R}^n
\end{align}
by
\begin{align}
T_w = v - \frac{2v^Tw}{w^Tw}w, \quad v \in \mathbb{R}^n
\end{align}
Which of the following are true?
\begin{enumerate}
\item $det\brak{T_w} = 1$
\item $T_w(v_1)^T_w(v_2) = v_1^Tv_2 \forall v_1, v_2 \in \mathbb{R}^n$
\item  $T_w = T_w ^{-1}$
\item  $T_{2w} = 2T_{w}$
\end{enumerate}
\item Consider the matrix 
\begin{align}
A = \myvec{0 & 1 \\ 1 & 0}
\end{align}
over the field $\mathbb{Q}$ of rationals.  Which of the following matrices are of the form $P^TAP$ for suitable $2\times 2$ invertible matrix $P$ over $\mathbb{Q}$?
\begin{enumerate}
\begin{multicols}{2}
\item $\myvec{2 & 0 \\ 0 & -2}$
\item $\myvec{2 & 0 \\ 0 & 2}$
\item $\myvec{1 & 0 \\ 0 & -1}$
\item $\myvec{3 & 4 \\ 4 & 5}$
\end{multicols}
\end{enumerate}
\item Consider a Markov Chain with state space $\cbrak{0,1,2}$ and transition matrix
\begin{align}
P = 
\begin{blockarray}{c@{\hspace{1pt}}rrr@{\hspace{3pt}}}
         & 0   & 1   & 2 \\
        \begin{block}{r@{\hspace{3pt}}@{\hspace{1pt}}
    (@{\hspace{1pt}}rrr@{\hspace{1pt}}@{\hspace{1pt}})}
        0 & \frac{1}{4} & \frac{5}{8} & \frac{1}{8}  \\
        1 & \frac{1}{4} & 0 & \frac{3}{4}  \\
%
        2 & 0 & \frac{1}{2} & \frac{3}{8} & \frac{1}{8}  \\
        \end{block}
    \end{blockarray}
\end{align}
Then which of the following are true?
\begin{enumerate}
\item $\lim_{n \to \infty} p_{12}^(n) = 0$
\item $\lim_{n \to \infty} p_{12}^(n) = \lim_{n \to \infty} p_{21}^(n)$
\item $\lim_{n \to \infty} p_{22}^(n) = \frac{1}{8}$
\item $\lim_{n \to \infty} p_{21}^(n) = \frac{1}{3}$
\end{enumerate}
\end{enumerate}

\section{December 2018}
\renewcommand{\theequation}{\theenumi}
\renewcommand{\thefigure}{\theenumi}
\renewcommand{\thetable}{\theenumi}
\begin{enumerate}[label=\thesection.\arabic*.,ref=\thesection.\theenumi]
\numberwithin{equation}{enumi}
\numberwithin{figure}{enumi}
\numberwithin{table}{enumi}

\item %
Let $\vec{A}=\myvec{1 & 1\\1 & 0}$ and let $\alpha_n$ and $\beta_n$ denote the two eigenvalues of $\vec{A}^n$ such that $\abs{\alpha_n}\geq\abs{\beta_n}$.\\
Then
\begin{enumerate}
    \item $\alpha_n\rightarrow \infty$ as $n\rightarrow \infty$
    \item $\beta_n\rightarrow 0$ as $n\rightarrow \infty$
    \item $\beta_n$ is positive if n is even.
    \item $\beta_n$ is negative if n is odd.
\end{enumerate}
%
\solution
See Table \ref{eq:solutions/6/2/11/table:1}
\onecolumn
\begin{longtable}{|l|l|}
\hline
\multirow{3}{*}{} & \\
Statement&Solution\\
\hline
&\parbox{6cm}{\begin{align}
\mbox{Let }\vec{N}&=\myvec{a&b\\c&d}\\
\mbox{Since }\vec{N}^2&=0
\end{align}}\\
&If $\myvec{a\\c},\myvec{b\\d}$ are linearly independent then $\vec{N}$ is diagonalizable to $\myvec{0&0\\0&0}$.\\
&\parbox{6cm}{\begin{align}
\mbox{If }\vec{P}\vec{N}\vec{P}^{-1}=0\\
\mbox{then }\vec{N}=\vec{P}^{-1}\vec{0}\vec{P}=0
\end{align}}\\
Proof that&So in this case $\vec{N}$ itself is the zero matrix.\\
$\vec{N}=0$&This contradicts the assumption that $\myvec{a\\c},\myvec{b\\d}$ are linearly independent.\\
&$\therefore$ we can assume that $\myvec{a\\c},\myvec{b\\d}$ are linearly dependent if both are\\
&equal to the zero vector\\
&\parbox{6cm}
{\begin{align}
   \mbox{then } \vec{N} &= 0.
\end{align}}\\
%\hline
%\pagebreak
\hline
&\\
&Therefore we can assume at least one vector is non-zero.\\
Assuming $\myvec{b\\d}$ as&Therefore
$\vec{N}=\myvec{a&0\\c&0}$\\
the zero vector&\\
&\parbox{6cm}{\begin{align}
\mbox{So }\vec{N}^2&=0\\
\implies a^2&=0\\
\therefore a&=0\\
\mbox{Thus }\vec{N}&=\myvec{a&0\\c&0}
\end{align}}\\
&In this case $\vec{N}$ is similar to $\vec{N}=\myvec{0&0\\1&0}$ via the matrix $\vec{P}=\myvec{c&0\\0&1}$\\
&\\
\hline
&\\
Assuming $\myvec{a\\c}$ as&Therefore
$\vec{N}=\myvec{0&b\\0&d}$\\
the zero vector&\\
&\parbox{6cm}{\begin{align}
\mbox{Then }\vec{N}^2&=0\\
\implies d^2&=0\\
\therefore d&=0\\
\mbox{Thus }\vec{N}&=\myvec{0&b\\0&0}
\end{align}}\\
&In this case $\vec{N}$ is similar to $\vec{N}=\myvec{0&0\\b&0}$ via the matrix $\vec{P}=\myvec{0&1\\1&0}$,\\
&which is similar to $\myvec{0&0\\1&0}$ as above.\\
&\\
\hline
&\\
Hence& we can assume neither $\myvec{a\\c}$ or $\myvec{b\\d}$ is the zero vector.\\
&\\
%\hline
%\pagebreak
\hline
&\\
Consequences of &Since they are linearly dependent we can assume,\\
linear&\\
independence&\\
&\parbox{6cm}{\begin{align}
\myvec{b\\d}&=x\myvec{a\\c}\\
\therefore \vec{N}&=\myvec{a&ax\\c&cx}\\
\therefore \vec{N}^2&=0\\
\implies a(a+cx)&=0\\
c(a+cx)&=0\\
ax(a+cx)&=0\\
cx(a+cx)&=0
\end{align}}\\
\hline
&\\
Proof that $\vec{N}$ is&We know that at least one of a or c is not zero.\\
similar over $\mathbb{C}$ to&If a = 0 then c $\neq$ 0, it must be that x = 0.\\
$\myvec{0&0\\1&0}$&So in this case $\vec{N}=\myvec{0&0\\c&0}$ which is similar to $\myvec{0&0\\1&0}$ as before.\\
&\\
&\parbox{6cm}{\begin{align}
\mbox{If } a &\neq0\\
\mbox{then }x &\neq0\\
\mbox{else }a(a+cx)&=0\\
\implies a&=0\\
\mbox{Thus }a+cx&=0\\
\mbox{Hence }\vec{N}&=\myvec{a&ax\\ \frac{-a}{x}&-a}
\end{align}}\\
 &This is similar to $\myvec{a&a\\-a&-a}$ via $\vec{P}=\myvec{\sqrt{x}&0\\0&\frac{1}{\sqrt{x}}}$.\\
 &And $\myvec{a&a\\-a&-a}$ is similar to $\myvec{0&0\\-a&0}$ via $\vec{P}=\myvec{-1&-1\\1&0}$\\
 &And this finally is similar to $\myvec{0&0\\1&0}$ as before.\\
 &\\
\hline
&\\
Conclusion &Thus either $\vec{N}$ = 0 or $\vec{N}$ is similar over $\mathbb{C}$ to $\myvec{0&0\\1&0}$.\\
&\\
\hline
\caption{Solution summary}
\label{eq:solutions/6/2/11/table:1}
\end{longtable}

\item Let $M_n$ denote the vector space of all $n\times n$ real matrices. Which of the following is a linear subspaces of $M_n$ :-
\begin{enumerate}
\item $ V_1 = \{  A \in M_n : \text{ A is nonsingular} \}$
\item $ V_2 = \{  A \in M_n : det(A) = 0 \}$
\item $ V_3 = \{  A \in M_n : trace(A) = 0 \}$
\item $ V_4 = \{  BA : A \in M_n\},$ where $ B$ is some fixed matrix in $ M_n$
\end{enumerate}
%
\solution
See Table \ref{eq:solutions/6/2/11/table:1}
\onecolumn
\begin{longtable}{|l|l|}
\hline
\multirow{3}{*}{} & \\
Statement&Solution\\
\hline
&\parbox{6cm}{\begin{align}
\mbox{Let }\vec{N}&=\myvec{a&b\\c&d}\\
\mbox{Since }\vec{N}^2&=0
\end{align}}\\
&If $\myvec{a\\c},\myvec{b\\d}$ are linearly independent then $\vec{N}$ is diagonalizable to $\myvec{0&0\\0&0}$.\\
&\parbox{6cm}{\begin{align}
\mbox{If }\vec{P}\vec{N}\vec{P}^{-1}=0\\
\mbox{then }\vec{N}=\vec{P}^{-1}\vec{0}\vec{P}=0
\end{align}}\\
Proof that&So in this case $\vec{N}$ itself is the zero matrix.\\
$\vec{N}=0$&This contradicts the assumption that $\myvec{a\\c},\myvec{b\\d}$ are linearly independent.\\
&$\therefore$ we can assume that $\myvec{a\\c},\myvec{b\\d}$ are linearly dependent if both are\\
&equal to the zero vector\\
&\parbox{6cm}
{\begin{align}
   \mbox{then } \vec{N} &= 0.
\end{align}}\\
%\hline
%\pagebreak
\hline
&\\
&Therefore we can assume at least one vector is non-zero.\\
Assuming $\myvec{b\\d}$ as&Therefore
$\vec{N}=\myvec{a&0\\c&0}$\\
the zero vector&\\
&\parbox{6cm}{\begin{align}
\mbox{So }\vec{N}^2&=0\\
\implies a^2&=0\\
\therefore a&=0\\
\mbox{Thus }\vec{N}&=\myvec{a&0\\c&0}
\end{align}}\\
&In this case $\vec{N}$ is similar to $\vec{N}=\myvec{0&0\\1&0}$ via the matrix $\vec{P}=\myvec{c&0\\0&1}$\\
&\\
\hline
&\\
Assuming $\myvec{a\\c}$ as&Therefore
$\vec{N}=\myvec{0&b\\0&d}$\\
the zero vector&\\
&\parbox{6cm}{\begin{align}
\mbox{Then }\vec{N}^2&=0\\
\implies d^2&=0\\
\therefore d&=0\\
\mbox{Thus }\vec{N}&=\myvec{0&b\\0&0}
\end{align}}\\
&In this case $\vec{N}$ is similar to $\vec{N}=\myvec{0&0\\b&0}$ via the matrix $\vec{P}=\myvec{0&1\\1&0}$,\\
&which is similar to $\myvec{0&0\\1&0}$ as above.\\
&\\
\hline
&\\
Hence& we can assume neither $\myvec{a\\c}$ or $\myvec{b\\d}$ is the zero vector.\\
&\\
%\hline
%\pagebreak
\hline
&\\
Consequences of &Since they are linearly dependent we can assume,\\
linear&\\
independence&\\
&\parbox{6cm}{\begin{align}
\myvec{b\\d}&=x\myvec{a\\c}\\
\therefore \vec{N}&=\myvec{a&ax\\c&cx}\\
\therefore \vec{N}^2&=0\\
\implies a(a+cx)&=0\\
c(a+cx)&=0\\
ax(a+cx)&=0\\
cx(a+cx)&=0
\end{align}}\\
\hline
&\\
Proof that $\vec{N}$ is&We know that at least one of a or c is not zero.\\
similar over $\mathbb{C}$ to&If a = 0 then c $\neq$ 0, it must be that x = 0.\\
$\myvec{0&0\\1&0}$&So in this case $\vec{N}=\myvec{0&0\\c&0}$ which is similar to $\myvec{0&0\\1&0}$ as before.\\
&\\
&\parbox{6cm}{\begin{align}
\mbox{If } a &\neq0\\
\mbox{then }x &\neq0\\
\mbox{else }a(a+cx)&=0\\
\implies a&=0\\
\mbox{Thus }a+cx&=0\\
\mbox{Hence }\vec{N}&=\myvec{a&ax\\ \frac{-a}{x}&-a}
\end{align}}\\
 &This is similar to $\myvec{a&a\\-a&-a}$ via $\vec{P}=\myvec{\sqrt{x}&0\\0&\frac{1}{\sqrt{x}}}$.\\
 &And $\myvec{a&a\\-a&-a}$ is similar to $\myvec{0&0\\-a&0}$ via $\vec{P}=\myvec{-1&-1\\1&0}$\\
 &And this finally is similar to $\myvec{0&0\\1&0}$ as before.\\
 &\\
\hline
&\\
Conclusion &Thus either $\vec{N}$ = 0 or $\vec{N}$ is similar over $\mathbb{C}$ to $\myvec{0&0\\1&0}$.\\
&\\
\hline
\caption{Solution summary}
\label{eq:solutions/6/2/11/table:1}
\end{longtable}


\item If $\vec{P}$ and $\vec{Q}$ are invertible matrices such that   
$\vec{P}\vec{Q} = -\vec{Q}\vec{P}$,then we can conclude that
\begin{enumerate}
\item  $Tr(\vec{P})=Tr(\vec{Q})=0$ \label{eq:solutions/2016/dec/75/1}
\item  $Tr(\vec{P})=Tr(\vec{Q})=1$ \label{eq:solutions/2016/dec/75/2}
\item  $Tr(\vec{P})=-Tr(\vec{Q})$ \label{eq:solutions/2016/dec/75/3}
\item  $Tr(\vec{P}) \neq Tr(\vec{Q})$ \label{eq:solutions/2016/dec/75/4}
\end{enumerate}
%
%
\solution
See Table \ref{eq:solutions/6/2/11/table:1}
\onecolumn
\begin{longtable}{|l|l|}
\hline
\multirow{3}{*}{} & \\
Statement&Solution\\
\hline
&\parbox{6cm}{\begin{align}
\mbox{Let }\vec{N}&=\myvec{a&b\\c&d}\\
\mbox{Since }\vec{N}^2&=0
\end{align}}\\
&If $\myvec{a\\c},\myvec{b\\d}$ are linearly independent then $\vec{N}$ is diagonalizable to $\myvec{0&0\\0&0}$.\\
&\parbox{6cm}{\begin{align}
\mbox{If }\vec{P}\vec{N}\vec{P}^{-1}=0\\
\mbox{then }\vec{N}=\vec{P}^{-1}\vec{0}\vec{P}=0
\end{align}}\\
Proof that&So in this case $\vec{N}$ itself is the zero matrix.\\
$\vec{N}=0$&This contradicts the assumption that $\myvec{a\\c},\myvec{b\\d}$ are linearly independent.\\
&$\therefore$ we can assume that $\myvec{a\\c},\myvec{b\\d}$ are linearly dependent if both are\\
&equal to the zero vector\\
&\parbox{6cm}
{\begin{align}
   \mbox{then } \vec{N} &= 0.
\end{align}}\\
%\hline
%\pagebreak
\hline
&\\
&Therefore we can assume at least one vector is non-zero.\\
Assuming $\myvec{b\\d}$ as&Therefore
$\vec{N}=\myvec{a&0\\c&0}$\\
the zero vector&\\
&\parbox{6cm}{\begin{align}
\mbox{So }\vec{N}^2&=0\\
\implies a^2&=0\\
\therefore a&=0\\
\mbox{Thus }\vec{N}&=\myvec{a&0\\c&0}
\end{align}}\\
&In this case $\vec{N}$ is similar to $\vec{N}=\myvec{0&0\\1&0}$ via the matrix $\vec{P}=\myvec{c&0\\0&1}$\\
&\\
\hline
&\\
Assuming $\myvec{a\\c}$ as&Therefore
$\vec{N}=\myvec{0&b\\0&d}$\\
the zero vector&\\
&\parbox{6cm}{\begin{align}
\mbox{Then }\vec{N}^2&=0\\
\implies d^2&=0\\
\therefore d&=0\\
\mbox{Thus }\vec{N}&=\myvec{0&b\\0&0}
\end{align}}\\
&In this case $\vec{N}$ is similar to $\vec{N}=\myvec{0&0\\b&0}$ via the matrix $\vec{P}=\myvec{0&1\\1&0}$,\\
&which is similar to $\myvec{0&0\\1&0}$ as above.\\
&\\
\hline
&\\
Hence& we can assume neither $\myvec{a\\c}$ or $\myvec{b\\d}$ is the zero vector.\\
&\\
%\hline
%\pagebreak
\hline
&\\
Consequences of &Since they are linearly dependent we can assume,\\
linear&\\
independence&\\
&\parbox{6cm}{\begin{align}
\myvec{b\\d}&=x\myvec{a\\c}\\
\therefore \vec{N}&=\myvec{a&ax\\c&cx}\\
\therefore \vec{N}^2&=0\\
\implies a(a+cx)&=0\\
c(a+cx)&=0\\
ax(a+cx)&=0\\
cx(a+cx)&=0
\end{align}}\\
\hline
&\\
Proof that $\vec{N}$ is&We know that at least one of a or c is not zero.\\
similar over $\mathbb{C}$ to&If a = 0 then c $\neq$ 0, it must be that x = 0.\\
$\myvec{0&0\\1&0}$&So in this case $\vec{N}=\myvec{0&0\\c&0}$ which is similar to $\myvec{0&0\\1&0}$ as before.\\
&\\
&\parbox{6cm}{\begin{align}
\mbox{If } a &\neq0\\
\mbox{then }x &\neq0\\
\mbox{else }a(a+cx)&=0\\
\implies a&=0\\
\mbox{Thus }a+cx&=0\\
\mbox{Hence }\vec{N}&=\myvec{a&ax\\ \frac{-a}{x}&-a}
\end{align}}\\
 &This is similar to $\myvec{a&a\\-a&-a}$ via $\vec{P}=\myvec{\sqrt{x}&0\\0&\frac{1}{\sqrt{x}}}$.\\
 &And $\myvec{a&a\\-a&-a}$ is similar to $\myvec{0&0\\-a&0}$ via $\vec{P}=\myvec{-1&-1\\1&0}$\\
 &And this finally is similar to $\myvec{0&0\\1&0}$ as before.\\
 &\\
\hline
&\\
Conclusion &Thus either $\vec{N}$ = 0 or $\vec{N}$ is similar over $\mathbb{C}$ to $\myvec{0&0\\1&0}$.\\
&\\
\hline
\caption{Solution summary}
\label{eq:solutions/6/2/11/table:1}
\end{longtable}

\twocolumn
\item Let $\vec{W_1}$, $\vec{W_2}$, $\vec{W_3}$ be 3 distinct subspaces of $\vec{R}^{10}$ such that each $\vec{W_i}$ has dimension of 9. Let $\vec{W} = \vec{W_1} \cap \vec{W_2} \cap \vec{W_3}$. Then we can conclude that\\
\begin{enumerate}
\item $\vec{W}$ may not be a subspace of $\vec{R}^{10}$\\
\item dim $\vec{W} \leq 8$\\
\item dim $\vec{W} \geq 7$\\
\item  dim $\vec{W} \leq 3$\\
\end{enumerate}
 %
\solution
See Table \ref{eq:solutions/6/2/11/table:1}
\onecolumn
\begin{longtable}{|l|l|}
\hline
\multirow{3}{*}{} & \\
Statement&Solution\\
\hline
&\parbox{6cm}{\begin{align}
\mbox{Let }\vec{N}&=\myvec{a&b\\c&d}\\
\mbox{Since }\vec{N}^2&=0
\end{align}}\\
&If $\myvec{a\\c},\myvec{b\\d}$ are linearly independent then $\vec{N}$ is diagonalizable to $\myvec{0&0\\0&0}$.\\
&\parbox{6cm}{\begin{align}
\mbox{If }\vec{P}\vec{N}\vec{P}^{-1}=0\\
\mbox{then }\vec{N}=\vec{P}^{-1}\vec{0}\vec{P}=0
\end{align}}\\
Proof that&So in this case $\vec{N}$ itself is the zero matrix.\\
$\vec{N}=0$&This contradicts the assumption that $\myvec{a\\c},\myvec{b\\d}$ are linearly independent.\\
&$\therefore$ we can assume that $\myvec{a\\c},\myvec{b\\d}$ are linearly dependent if both are\\
&equal to the zero vector\\
&\parbox{6cm}
{\begin{align}
   \mbox{then } \vec{N} &= 0.
\end{align}}\\
%\hline
%\pagebreak
\hline
&\\
&Therefore we can assume at least one vector is non-zero.\\
Assuming $\myvec{b\\d}$ as&Therefore
$\vec{N}=\myvec{a&0\\c&0}$\\
the zero vector&\\
&\parbox{6cm}{\begin{align}
\mbox{So }\vec{N}^2&=0\\
\implies a^2&=0\\
\therefore a&=0\\
\mbox{Thus }\vec{N}&=\myvec{a&0\\c&0}
\end{align}}\\
&In this case $\vec{N}$ is similar to $\vec{N}=\myvec{0&0\\1&0}$ via the matrix $\vec{P}=\myvec{c&0\\0&1}$\\
&\\
\hline
&\\
Assuming $\myvec{a\\c}$ as&Therefore
$\vec{N}=\myvec{0&b\\0&d}$\\
the zero vector&\\
&\parbox{6cm}{\begin{align}
\mbox{Then }\vec{N}^2&=0\\
\implies d^2&=0\\
\therefore d&=0\\
\mbox{Thus }\vec{N}&=\myvec{0&b\\0&0}
\end{align}}\\
&In this case $\vec{N}$ is similar to $\vec{N}=\myvec{0&0\\b&0}$ via the matrix $\vec{P}=\myvec{0&1\\1&0}$,\\
&which is similar to $\myvec{0&0\\1&0}$ as above.\\
&\\
\hline
&\\
Hence& we can assume neither $\myvec{a\\c}$ or $\myvec{b\\d}$ is the zero vector.\\
&\\
%\hline
%\pagebreak
\hline
&\\
Consequences of &Since they are linearly dependent we can assume,\\
linear&\\
independence&\\
&\parbox{6cm}{\begin{align}
\myvec{b\\d}&=x\myvec{a\\c}\\
\therefore \vec{N}&=\myvec{a&ax\\c&cx}\\
\therefore \vec{N}^2&=0\\
\implies a(a+cx)&=0\\
c(a+cx)&=0\\
ax(a+cx)&=0\\
cx(a+cx)&=0
\end{align}}\\
\hline
&\\
Proof that $\vec{N}$ is&We know that at least one of a or c is not zero.\\
similar over $\mathbb{C}$ to&If a = 0 then c $\neq$ 0, it must be that x = 0.\\
$\myvec{0&0\\1&0}$&So in this case $\vec{N}=\myvec{0&0\\c&0}$ which is similar to $\myvec{0&0\\1&0}$ as before.\\
&\\
&\parbox{6cm}{\begin{align}
\mbox{If } a &\neq0\\
\mbox{then }x &\neq0\\
\mbox{else }a(a+cx)&=0\\
\implies a&=0\\
\mbox{Thus }a+cx&=0\\
\mbox{Hence }\vec{N}&=\myvec{a&ax\\ \frac{-a}{x}&-a}
\end{align}}\\
 &This is similar to $\myvec{a&a\\-a&-a}$ via $\vec{P}=\myvec{\sqrt{x}&0\\0&\frac{1}{\sqrt{x}}}$.\\
 &And $\myvec{a&a\\-a&-a}$ is similar to $\myvec{0&0\\-a&0}$ via $\vec{P}=\myvec{-1&-1\\1&0}$\\
 &And this finally is similar to $\myvec{0&0\\1&0}$ as before.\\
 &\\
\hline
&\\
Conclusion &Thus either $\vec{N}$ = 0 or $\vec{N}$ is similar over $\mathbb{C}$ to $\myvec{0&0\\1&0}$.\\
&\\
\hline
\caption{Solution summary}
\label{eq:solutions/6/2/11/table:1}
\end{longtable}

\twocolumn


%\item Consider a Markov Chain with state space $\cbrak{0,1,2}$ and transition matrix
%\begin{align}
%P = 
%\begin{blockarray}{c@{\hspace{1pt}}rrr@{\hspace{3pt}}}
%         & 0   & 1   & 2 \\
%        \begin{block}{r@{\hspace{3pt}}@{\hspace{1pt}}
%    (@{\hspace{1pt}}rrr@{\hspace{1pt}}@{\hspace{1pt}})}
%        0 & \frac{1}{2} & \frac{1}{2} & 0  \\
%        1 & 0 &\frac{1}{2}  & \frac{3}{4}  \\
%%
%        2 &  \frac{1}{3} & \frac{1}{3} & \frac{1}{3}  \\
%        \end{block}
%    \end{blockarray}
%\end{align}
%For any two states $i$ and $j$, let $p_{ij}^{(n)}$ denote the $n$-step transition probability of going from $i$ to $j$.  Identify correct statements.
%\begin{enumerate}
%\item $\lim_{n \to \infty} p_{11}^{(n)} = \frac{2}{9}$
%\item $\lim_{n \to \infty} p_{21}^{(n)} = 0$
%\item $\lim_{n \to \infty} p_{32}^{(n)} = \frac{1}{3}$
%\item $\lim_{n \to \infty} p_{13}^{(n)} = \frac{1}{3}$
%\end{enumerate}

\end{enumerate}

\section{June 2018}
\renewcommand{\theequation}{\theenumi}
\renewcommand{\thefigure}{\theenumi}
\begin{enumerate}[label=\thesection.\arabic*.,ref=\thesection.\theenumi]
\numberwithin{equation}{enumi}
\numberwithin{figure}{enumi}

\item Let $\vec{A}$ be an n $\times$ n self-adjoint matrix with eigenvalues $\lambda_1, \cdots, \lambda_2$.
Let,\begin{align} \norm{\vec{X}}_2=\sqrt{\vert\vec{X}_{1}^{2}\vert+\cdots+\vert\vec{X}_{n}^{2}\vert}\end{align} for $\vec{X}$=$(\vec{X}_{1},\cdots,\vec{X}_{n})\in \mathbb{C}^n$. If \begin{align}
p(\vec{A})=a_0\vec{I}+a_1\vec{A}+\cdots+a_n\vec{A}^n
\end{align}
then $sup_{\norm{\vec{X}}_{2}=1}\norm{p(\vec{A})\vec{X}}_2$ is equal to
%
\\
\solution
See Table \ref{eq:solutions/6/2/11/table:1}
\onecolumn
\begin{longtable}{|l|l|}
\hline
\multirow{3}{*}{} & \\
Statement&Solution\\
\hline
&\parbox{6cm}{\begin{align}
\mbox{Let }\vec{N}&=\myvec{a&b\\c&d}\\
\mbox{Since }\vec{N}^2&=0
\end{align}}\\
&If $\myvec{a\\c},\myvec{b\\d}$ are linearly independent then $\vec{N}$ is diagonalizable to $\myvec{0&0\\0&0}$.\\
&\parbox{6cm}{\begin{align}
\mbox{If }\vec{P}\vec{N}\vec{P}^{-1}=0\\
\mbox{then }\vec{N}=\vec{P}^{-1}\vec{0}\vec{P}=0
\end{align}}\\
Proof that&So in this case $\vec{N}$ itself is the zero matrix.\\
$\vec{N}=0$&This contradicts the assumption that $\myvec{a\\c},\myvec{b\\d}$ are linearly independent.\\
&$\therefore$ we can assume that $\myvec{a\\c},\myvec{b\\d}$ are linearly dependent if both are\\
&equal to the zero vector\\
&\parbox{6cm}
{\begin{align}
   \mbox{then } \vec{N} &= 0.
\end{align}}\\
%\hline
%\pagebreak
\hline
&\\
&Therefore we can assume at least one vector is non-zero.\\
Assuming $\myvec{b\\d}$ as&Therefore
$\vec{N}=\myvec{a&0\\c&0}$\\
the zero vector&\\
&\parbox{6cm}{\begin{align}
\mbox{So }\vec{N}^2&=0\\
\implies a^2&=0\\
\therefore a&=0\\
\mbox{Thus }\vec{N}&=\myvec{a&0\\c&0}
\end{align}}\\
&In this case $\vec{N}$ is similar to $\vec{N}=\myvec{0&0\\1&0}$ via the matrix $\vec{P}=\myvec{c&0\\0&1}$\\
&\\
\hline
&\\
Assuming $\myvec{a\\c}$ as&Therefore
$\vec{N}=\myvec{0&b\\0&d}$\\
the zero vector&\\
&\parbox{6cm}{\begin{align}
\mbox{Then }\vec{N}^2&=0\\
\implies d^2&=0\\
\therefore d&=0\\
\mbox{Thus }\vec{N}&=\myvec{0&b\\0&0}
\end{align}}\\
&In this case $\vec{N}$ is similar to $\vec{N}=\myvec{0&0\\b&0}$ via the matrix $\vec{P}=\myvec{0&1\\1&0}$,\\
&which is similar to $\myvec{0&0\\1&0}$ as above.\\
&\\
\hline
&\\
Hence& we can assume neither $\myvec{a\\c}$ or $\myvec{b\\d}$ is the zero vector.\\
&\\
%\hline
%\pagebreak
\hline
&\\
Consequences of &Since they are linearly dependent we can assume,\\
linear&\\
independence&\\
&\parbox{6cm}{\begin{align}
\myvec{b\\d}&=x\myvec{a\\c}\\
\therefore \vec{N}&=\myvec{a&ax\\c&cx}\\
\therefore \vec{N}^2&=0\\
\implies a(a+cx)&=0\\
c(a+cx)&=0\\
ax(a+cx)&=0\\
cx(a+cx)&=0
\end{align}}\\
\hline
&\\
Proof that $\vec{N}$ is&We know that at least one of a or c is not zero.\\
similar over $\mathbb{C}$ to&If a = 0 then c $\neq$ 0, it must be that x = 0.\\
$\myvec{0&0\\1&0}$&So in this case $\vec{N}=\myvec{0&0\\c&0}$ which is similar to $\myvec{0&0\\1&0}$ as before.\\
&\\
&\parbox{6cm}{\begin{align}
\mbox{If } a &\neq0\\
\mbox{then }x &\neq0\\
\mbox{else }a(a+cx)&=0\\
\implies a&=0\\
\mbox{Thus }a+cx&=0\\
\mbox{Hence }\vec{N}&=\myvec{a&ax\\ \frac{-a}{x}&-a}
\end{align}}\\
 &This is similar to $\myvec{a&a\\-a&-a}$ via $\vec{P}=\myvec{\sqrt{x}&0\\0&\frac{1}{\sqrt{x}}}$.\\
 &And $\myvec{a&a\\-a&-a}$ is similar to $\myvec{0&0\\-a&0}$ via $\vec{P}=\myvec{-1&-1\\1&0}$\\
 &And this finally is similar to $\myvec{0&0\\1&0}$ as before.\\
 &\\
\hline
&\\
Conclusion &Thus either $\vec{N}$ = 0 or $\vec{N}$ is similar over $\mathbb{C}$ to $\myvec{0&0\\1&0}$.\\
&\\
\hline
\caption{Solution summary}
\label{eq:solutions/6/2/11/table:1}
\end{longtable}

%
\item Find the Eigenvalues of the matrix,
\begin{align}
\vec{A} = \myvec{1 & 1 & 2 \\ 1 & -2 & 5 \\ 2 & 5 & -3 }\label{eq:solutions/2017/june/32/eq:1}
\end{align}
\begin{enumerate}
\item -4, 3, -3
\item 4, 3, 1
\item 4, -4$\pm\sqrt{13}$
\item 4, -2$\pm\sqrt{7}$
\end{enumerate}
%
%
\solution
See Table \ref{eq:solutions/6/2/11/table:1}
\onecolumn
\begin{longtable}{|l|l|}
\hline
\multirow{3}{*}{} & \\
Statement&Solution\\
\hline
&\parbox{6cm}{\begin{align}
\mbox{Let }\vec{N}&=\myvec{a&b\\c&d}\\
\mbox{Since }\vec{N}^2&=0
\end{align}}\\
&If $\myvec{a\\c},\myvec{b\\d}$ are linearly independent then $\vec{N}$ is diagonalizable to $\myvec{0&0\\0&0}$.\\
&\parbox{6cm}{\begin{align}
\mbox{If }\vec{P}\vec{N}\vec{P}^{-1}=0\\
\mbox{then }\vec{N}=\vec{P}^{-1}\vec{0}\vec{P}=0
\end{align}}\\
Proof that&So in this case $\vec{N}$ itself is the zero matrix.\\
$\vec{N}=0$&This contradicts the assumption that $\myvec{a\\c},\myvec{b\\d}$ are linearly independent.\\
&$\therefore$ we can assume that $\myvec{a\\c},\myvec{b\\d}$ are linearly dependent if both are\\
&equal to the zero vector\\
&\parbox{6cm}
{\begin{align}
   \mbox{then } \vec{N} &= 0.
\end{align}}\\
%\hline
%\pagebreak
\hline
&\\
&Therefore we can assume at least one vector is non-zero.\\
Assuming $\myvec{b\\d}$ as&Therefore
$\vec{N}=\myvec{a&0\\c&0}$\\
the zero vector&\\
&\parbox{6cm}{\begin{align}
\mbox{So }\vec{N}^2&=0\\
\implies a^2&=0\\
\therefore a&=0\\
\mbox{Thus }\vec{N}&=\myvec{a&0\\c&0}
\end{align}}\\
&In this case $\vec{N}$ is similar to $\vec{N}=\myvec{0&0\\1&0}$ via the matrix $\vec{P}=\myvec{c&0\\0&1}$\\
&\\
\hline
&\\
Assuming $\myvec{a\\c}$ as&Therefore
$\vec{N}=\myvec{0&b\\0&d}$\\
the zero vector&\\
&\parbox{6cm}{\begin{align}
\mbox{Then }\vec{N}^2&=0\\
\implies d^2&=0\\
\therefore d&=0\\
\mbox{Thus }\vec{N}&=\myvec{0&b\\0&0}
\end{align}}\\
&In this case $\vec{N}$ is similar to $\vec{N}=\myvec{0&0\\b&0}$ via the matrix $\vec{P}=\myvec{0&1\\1&0}$,\\
&which is similar to $\myvec{0&0\\1&0}$ as above.\\
&\\
\hline
&\\
Hence& we can assume neither $\myvec{a\\c}$ or $\myvec{b\\d}$ is the zero vector.\\
&\\
%\hline
%\pagebreak
\hline
&\\
Consequences of &Since they are linearly dependent we can assume,\\
linear&\\
independence&\\
&\parbox{6cm}{\begin{align}
\myvec{b\\d}&=x\myvec{a\\c}\\
\therefore \vec{N}&=\myvec{a&ax\\c&cx}\\
\therefore \vec{N}^2&=0\\
\implies a(a+cx)&=0\\
c(a+cx)&=0\\
ax(a+cx)&=0\\
cx(a+cx)&=0
\end{align}}\\
\hline
&\\
Proof that $\vec{N}$ is&We know that at least one of a or c is not zero.\\
similar over $\mathbb{C}$ to&If a = 0 then c $\neq$ 0, it must be that x = 0.\\
$\myvec{0&0\\1&0}$&So in this case $\vec{N}=\myvec{0&0\\c&0}$ which is similar to $\myvec{0&0\\1&0}$ as before.\\
&\\
&\parbox{6cm}{\begin{align}
\mbox{If } a &\neq0\\
\mbox{then }x &\neq0\\
\mbox{else }a(a+cx)&=0\\
\implies a&=0\\
\mbox{Thus }a+cx&=0\\
\mbox{Hence }\vec{N}&=\myvec{a&ax\\ \frac{-a}{x}&-a}
\end{align}}\\
 &This is similar to $\myvec{a&a\\-a&-a}$ via $\vec{P}=\myvec{\sqrt{x}&0\\0&\frac{1}{\sqrt{x}}}$.\\
 &And $\myvec{a&a\\-a&-a}$ is similar to $\myvec{0&0\\-a&0}$ via $\vec{P}=\myvec{-1&-1\\1&0}$\\
 &And this finally is similar to $\myvec{0&0\\1&0}$ as before.\\
 &\\
\hline
&\\
Conclusion &Thus either $\vec{N}$ = 0 or $\vec{N}$ is similar over $\mathbb{C}$ to $\myvec{0&0\\1&0}$.\\
&\\
\hline
\caption{Solution summary}
\label{eq:solutions/6/2/11/table:1}
\end{longtable}


\item Let \textbf{V} be the vector space of polynomials of degree at most 3 in a variable x with coefficients in $\mathbb{R}$. Let \textbf{T}=d/dx be the linear transformation of \textbf{V} to itself given by differentiation.\\

Which of the following are correct?\\
\begin{enumerate}
\item $\vec{T}$ is invertible
\item 0 is an eigenvalue of $\vec{T}$
\item There is a basis with respect to which the matrix of \textbf{T} is nilpotent.
\item The matrix of \textbf{T} with respect to the basis \myvec{1,1+x,1+x+x^2,1+x+x^2+x^3} is diagonal.
\end{enumerate}
\solution
See Table \ref{eq:solutions/6/2/11/table:1}
\onecolumn
\begin{longtable}{|l|l|}
\hline
\multirow{3}{*}{} & \\
Statement&Solution\\
\hline
&\parbox{6cm}{\begin{align}
\mbox{Let }\vec{N}&=\myvec{a&b\\c&d}\\
\mbox{Since }\vec{N}^2&=0
\end{align}}\\
&If $\myvec{a\\c},\myvec{b\\d}$ are linearly independent then $\vec{N}$ is diagonalizable to $\myvec{0&0\\0&0}$.\\
&\parbox{6cm}{\begin{align}
\mbox{If }\vec{P}\vec{N}\vec{P}^{-1}=0\\
\mbox{then }\vec{N}=\vec{P}^{-1}\vec{0}\vec{P}=0
\end{align}}\\
Proof that&So in this case $\vec{N}$ itself is the zero matrix.\\
$\vec{N}=0$&This contradicts the assumption that $\myvec{a\\c},\myvec{b\\d}$ are linearly independent.\\
&$\therefore$ we can assume that $\myvec{a\\c},\myvec{b\\d}$ are linearly dependent if both are\\
&equal to the zero vector\\
&\parbox{6cm}
{\begin{align}
   \mbox{then } \vec{N} &= 0.
\end{align}}\\
%\hline
%\pagebreak
\hline
&\\
&Therefore we can assume at least one vector is non-zero.\\
Assuming $\myvec{b\\d}$ as&Therefore
$\vec{N}=\myvec{a&0\\c&0}$\\
the zero vector&\\
&\parbox{6cm}{\begin{align}
\mbox{So }\vec{N}^2&=0\\
\implies a^2&=0\\
\therefore a&=0\\
\mbox{Thus }\vec{N}&=\myvec{a&0\\c&0}
\end{align}}\\
&In this case $\vec{N}$ is similar to $\vec{N}=\myvec{0&0\\1&0}$ via the matrix $\vec{P}=\myvec{c&0\\0&1}$\\
&\\
\hline
&\\
Assuming $\myvec{a\\c}$ as&Therefore
$\vec{N}=\myvec{0&b\\0&d}$\\
the zero vector&\\
&\parbox{6cm}{\begin{align}
\mbox{Then }\vec{N}^2&=0\\
\implies d^2&=0\\
\therefore d&=0\\
\mbox{Thus }\vec{N}&=\myvec{0&b\\0&0}
\end{align}}\\
&In this case $\vec{N}$ is similar to $\vec{N}=\myvec{0&0\\b&0}$ via the matrix $\vec{P}=\myvec{0&1\\1&0}$,\\
&which is similar to $\myvec{0&0\\1&0}$ as above.\\
&\\
\hline
&\\
Hence& we can assume neither $\myvec{a\\c}$ or $\myvec{b\\d}$ is the zero vector.\\
&\\
%\hline
%\pagebreak
\hline
&\\
Consequences of &Since they are linearly dependent we can assume,\\
linear&\\
independence&\\
&\parbox{6cm}{\begin{align}
\myvec{b\\d}&=x\myvec{a\\c}\\
\therefore \vec{N}&=\myvec{a&ax\\c&cx}\\
\therefore \vec{N}^2&=0\\
\implies a(a+cx)&=0\\
c(a+cx)&=0\\
ax(a+cx)&=0\\
cx(a+cx)&=0
\end{align}}\\
\hline
&\\
Proof that $\vec{N}$ is&We know that at least one of a or c is not zero.\\
similar over $\mathbb{C}$ to&If a = 0 then c $\neq$ 0, it must be that x = 0.\\
$\myvec{0&0\\1&0}$&So in this case $\vec{N}=\myvec{0&0\\c&0}$ which is similar to $\myvec{0&0\\1&0}$ as before.\\
&\\
&\parbox{6cm}{\begin{align}
\mbox{If } a &\neq0\\
\mbox{then }x &\neq0\\
\mbox{else }a(a+cx)&=0\\
\implies a&=0\\
\mbox{Thus }a+cx&=0\\
\mbox{Hence }\vec{N}&=\myvec{a&ax\\ \frac{-a}{x}&-a}
\end{align}}\\
 &This is similar to $\myvec{a&a\\-a&-a}$ via $\vec{P}=\myvec{\sqrt{x}&0\\0&\frac{1}{\sqrt{x}}}$.\\
 &And $\myvec{a&a\\-a&-a}$ is similar to $\myvec{0&0\\-a&0}$ via $\vec{P}=\myvec{-1&-1\\1&0}$\\
 &And this finally is similar to $\myvec{0&0\\1&0}$ as before.\\
 &\\
\hline
&\\
Conclusion &Thus either $\vec{N}$ = 0 or $\vec{N}$ is similar over $\mathbb{C}$ to $\myvec{0&0\\1&0}$.\\
&\\
\hline
\caption{Solution summary}
\label{eq:solutions/6/2/11/table:1}
\end{longtable}

\item For any $n\times n$ matrix $B$, let $N(B)=\{X\in \mathbb{R}^n:BX=0\}$ be the null space of $B$. Let $A$ be a $4\times 4$ matrix with $dim(N(A-4I))=2, dim(N(A-2I))=1$ and $rank(A)=3$
Which of the following are true?
\begin{enumerate}
\item 0,2 and 4 are eigenvalues of A
\item determinant(A)=0
\item A is not diagonalizable
\item trace(A)=8
\end{enumerate}
%
\solution
See Table \ref{eq:solutions/6/2/11/table:1}
\onecolumn
\begin{longtable}{|l|l|}
\hline
\multirow{3}{*}{} & \\
Statement&Solution\\
\hline
&\parbox{6cm}{\begin{align}
\mbox{Let }\vec{N}&=\myvec{a&b\\c&d}\\
\mbox{Since }\vec{N}^2&=0
\end{align}}\\
&If $\myvec{a\\c},\myvec{b\\d}$ are linearly independent then $\vec{N}$ is diagonalizable to $\myvec{0&0\\0&0}$.\\
&\parbox{6cm}{\begin{align}
\mbox{If }\vec{P}\vec{N}\vec{P}^{-1}=0\\
\mbox{then }\vec{N}=\vec{P}^{-1}\vec{0}\vec{P}=0
\end{align}}\\
Proof that&So in this case $\vec{N}$ itself is the zero matrix.\\
$\vec{N}=0$&This contradicts the assumption that $\myvec{a\\c},\myvec{b\\d}$ are linearly independent.\\
&$\therefore$ we can assume that $\myvec{a\\c},\myvec{b\\d}$ are linearly dependent if both are\\
&equal to the zero vector\\
&\parbox{6cm}
{\begin{align}
   \mbox{then } \vec{N} &= 0.
\end{align}}\\
%\hline
%\pagebreak
\hline
&\\
&Therefore we can assume at least one vector is non-zero.\\
Assuming $\myvec{b\\d}$ as&Therefore
$\vec{N}=\myvec{a&0\\c&0}$\\
the zero vector&\\
&\parbox{6cm}{\begin{align}
\mbox{So }\vec{N}^2&=0\\
\implies a^2&=0\\
\therefore a&=0\\
\mbox{Thus }\vec{N}&=\myvec{a&0\\c&0}
\end{align}}\\
&In this case $\vec{N}$ is similar to $\vec{N}=\myvec{0&0\\1&0}$ via the matrix $\vec{P}=\myvec{c&0\\0&1}$\\
&\\
\hline
&\\
Assuming $\myvec{a\\c}$ as&Therefore
$\vec{N}=\myvec{0&b\\0&d}$\\
the zero vector&\\
&\parbox{6cm}{\begin{align}
\mbox{Then }\vec{N}^2&=0\\
\implies d^2&=0\\
\therefore d&=0\\
\mbox{Thus }\vec{N}&=\myvec{0&b\\0&0}
\end{align}}\\
&In this case $\vec{N}$ is similar to $\vec{N}=\myvec{0&0\\b&0}$ via the matrix $\vec{P}=\myvec{0&1\\1&0}$,\\
&which is similar to $\myvec{0&0\\1&0}$ as above.\\
&\\
\hline
&\\
Hence& we can assume neither $\myvec{a\\c}$ or $\myvec{b\\d}$ is the zero vector.\\
&\\
%\hline
%\pagebreak
\hline
&\\
Consequences of &Since they are linearly dependent we can assume,\\
linear&\\
independence&\\
&\parbox{6cm}{\begin{align}
\myvec{b\\d}&=x\myvec{a\\c}\\
\therefore \vec{N}&=\myvec{a&ax\\c&cx}\\
\therefore \vec{N}^2&=0\\
\implies a(a+cx)&=0\\
c(a+cx)&=0\\
ax(a+cx)&=0\\
cx(a+cx)&=0
\end{align}}\\
\hline
&\\
Proof that $\vec{N}$ is&We know that at least one of a or c is not zero.\\
similar over $\mathbb{C}$ to&If a = 0 then c $\neq$ 0, it must be that x = 0.\\
$\myvec{0&0\\1&0}$&So in this case $\vec{N}=\myvec{0&0\\c&0}$ which is similar to $\myvec{0&0\\1&0}$ as before.\\
&\\
&\parbox{6cm}{\begin{align}
\mbox{If } a &\neq0\\
\mbox{then }x &\neq0\\
\mbox{else }a(a+cx)&=0\\
\implies a&=0\\
\mbox{Thus }a+cx&=0\\
\mbox{Hence }\vec{N}&=\myvec{a&ax\\ \frac{-a}{x}&-a}
\end{align}}\\
 &This is similar to $\myvec{a&a\\-a&-a}$ via $\vec{P}=\myvec{\sqrt{x}&0\\0&\frac{1}{\sqrt{x}}}$.\\
 &And $\myvec{a&a\\-a&-a}$ is similar to $\myvec{0&0\\-a&0}$ via $\vec{P}=\myvec{-1&-1\\1&0}$\\
 &And this finally is similar to $\myvec{0&0\\1&0}$ as before.\\
 &\\
\hline
&\\
Conclusion &Thus either $\vec{N}$ = 0 or $\vec{N}$ is similar over $\mathbb{C}$ to $\myvec{0&0\\1&0}$.\\
&\\
\hline
\caption{Solution summary}
\label{eq:solutions/6/2/11/table:1}
\end{longtable}

\item Which of the following 3x3 matrices are diagonizable over $\mathbb{R}?$\\
\begin{enumerate}
    \item \myvec{1&2&3\\0&4&5\\0&0&6}
    \item \myvec{0&1&0\\-1&0&0\\0&0&1}
    \item \myvec{1&2&3\\2&1&4\\3&4&1}
    \item \myvec{0&1&2\\0&0&1\\0&0&0}
\end{enumerate}
%
\solution
See Table \ref{eq:solutions/6/2/11/table:1}
\onecolumn
\begin{longtable}{|l|l|}
\hline
\multirow{3}{*}{} & \\
Statement&Solution\\
\hline
&\parbox{6cm}{\begin{align}
\mbox{Let }\vec{N}&=\myvec{a&b\\c&d}\\
\mbox{Since }\vec{N}^2&=0
\end{align}}\\
&If $\myvec{a\\c},\myvec{b\\d}$ are linearly independent then $\vec{N}$ is diagonalizable to $\myvec{0&0\\0&0}$.\\
&\parbox{6cm}{\begin{align}
\mbox{If }\vec{P}\vec{N}\vec{P}^{-1}=0\\
\mbox{then }\vec{N}=\vec{P}^{-1}\vec{0}\vec{P}=0
\end{align}}\\
Proof that&So in this case $\vec{N}$ itself is the zero matrix.\\
$\vec{N}=0$&This contradicts the assumption that $\myvec{a\\c},\myvec{b\\d}$ are linearly independent.\\
&$\therefore$ we can assume that $\myvec{a\\c},\myvec{b\\d}$ are linearly dependent if both are\\
&equal to the zero vector\\
&\parbox{6cm}
{\begin{align}
   \mbox{then } \vec{N} &= 0.
\end{align}}\\
%\hline
%\pagebreak
\hline
&\\
&Therefore we can assume at least one vector is non-zero.\\
Assuming $\myvec{b\\d}$ as&Therefore
$\vec{N}=\myvec{a&0\\c&0}$\\
the zero vector&\\
&\parbox{6cm}{\begin{align}
\mbox{So }\vec{N}^2&=0\\
\implies a^2&=0\\
\therefore a&=0\\
\mbox{Thus }\vec{N}&=\myvec{a&0\\c&0}
\end{align}}\\
&In this case $\vec{N}$ is similar to $\vec{N}=\myvec{0&0\\1&0}$ via the matrix $\vec{P}=\myvec{c&0\\0&1}$\\
&\\
\hline
&\\
Assuming $\myvec{a\\c}$ as&Therefore
$\vec{N}=\myvec{0&b\\0&d}$\\
the zero vector&\\
&\parbox{6cm}{\begin{align}
\mbox{Then }\vec{N}^2&=0\\
\implies d^2&=0\\
\therefore d&=0\\
\mbox{Thus }\vec{N}&=\myvec{0&b\\0&0}
\end{align}}\\
&In this case $\vec{N}$ is similar to $\vec{N}=\myvec{0&0\\b&0}$ via the matrix $\vec{P}=\myvec{0&1\\1&0}$,\\
&which is similar to $\myvec{0&0\\1&0}$ as above.\\
&\\
\hline
&\\
Hence& we can assume neither $\myvec{a\\c}$ or $\myvec{b\\d}$ is the zero vector.\\
&\\
%\hline
%\pagebreak
\hline
&\\
Consequences of &Since they are linearly dependent we can assume,\\
linear&\\
independence&\\
&\parbox{6cm}{\begin{align}
\myvec{b\\d}&=x\myvec{a\\c}\\
\therefore \vec{N}&=\myvec{a&ax\\c&cx}\\
\therefore \vec{N}^2&=0\\
\implies a(a+cx)&=0\\
c(a+cx)&=0\\
ax(a+cx)&=0\\
cx(a+cx)&=0
\end{align}}\\
\hline
&\\
Proof that $\vec{N}$ is&We know that at least one of a or c is not zero.\\
similar over $\mathbb{C}$ to&If a = 0 then c $\neq$ 0, it must be that x = 0.\\
$\myvec{0&0\\1&0}$&So in this case $\vec{N}=\myvec{0&0\\c&0}$ which is similar to $\myvec{0&0\\1&0}$ as before.\\
&\\
&\parbox{6cm}{\begin{align}
\mbox{If } a &\neq0\\
\mbox{then }x &\neq0\\
\mbox{else }a(a+cx)&=0\\
\implies a&=0\\
\mbox{Thus }a+cx&=0\\
\mbox{Hence }\vec{N}&=\myvec{a&ax\\ \frac{-a}{x}&-a}
\end{align}}\\
 &This is similar to $\myvec{a&a\\-a&-a}$ via $\vec{P}=\myvec{\sqrt{x}&0\\0&\frac{1}{\sqrt{x}}}$.\\
 &And $\myvec{a&a\\-a&-a}$ is similar to $\myvec{0&0\\-a&0}$ via $\vec{P}=\myvec{-1&-1\\1&0}$\\
 &And this finally is similar to $\myvec{0&0\\1&0}$ as before.\\
 &\\
\hline
&\\
Conclusion &Thus either $\vec{N}$ = 0 or $\vec{N}$ is similar over $\mathbb{C}$ to $\myvec{0&0\\1&0}$.\\
&\\
\hline
\caption{Solution summary}
\label{eq:solutions/6/2/11/table:1}
\end{longtable}

\twocolumn
\item Let $\vec{A} = \myvec{3 & 1 & 2 \\ 1 & 2 & 3 \\ 2 & 3 & 1  }$ and $\vec{Q(X) = X^TAX}$ for $\vec{X} \in \mathbb{R}^{3}$. Then
\begin{enumerate}
	\item $\vec{A}$ has exactly two positive eigen values.
	\item all the eigen values of $\vec{A}$ are positive.
	\item $\vec{Q(X)} \geq 0 $ $\forall$ $\vec{X}$ $\in$ $\mathbb{R}^3$
	\item $\vec{Q(X)} < 0 $ for some $\vec{X}$ $\in$ $\mathbb{R}^3$
\end{enumerate}
%
%
\solution
See Table \ref{eq:solutions/6/2/11/table:1}
\onecolumn
\begin{longtable}{|l|l|}
\hline
\multirow{3}{*}{} & \\
Statement&Solution\\
\hline
&\parbox{6cm}{\begin{align}
\mbox{Let }\vec{N}&=\myvec{a&b\\c&d}\\
\mbox{Since }\vec{N}^2&=0
\end{align}}\\
&If $\myvec{a\\c},\myvec{b\\d}$ are linearly independent then $\vec{N}$ is diagonalizable to $\myvec{0&0\\0&0}$.\\
&\parbox{6cm}{\begin{align}
\mbox{If }\vec{P}\vec{N}\vec{P}^{-1}=0\\
\mbox{then }\vec{N}=\vec{P}^{-1}\vec{0}\vec{P}=0
\end{align}}\\
Proof that&So in this case $\vec{N}$ itself is the zero matrix.\\
$\vec{N}=0$&This contradicts the assumption that $\myvec{a\\c},\myvec{b\\d}$ are linearly independent.\\
&$\therefore$ we can assume that $\myvec{a\\c},\myvec{b\\d}$ are linearly dependent if both are\\
&equal to the zero vector\\
&\parbox{6cm}
{\begin{align}
   \mbox{then } \vec{N} &= 0.
\end{align}}\\
%\hline
%\pagebreak
\hline
&\\
&Therefore we can assume at least one vector is non-zero.\\
Assuming $\myvec{b\\d}$ as&Therefore
$\vec{N}=\myvec{a&0\\c&0}$\\
the zero vector&\\
&\parbox{6cm}{\begin{align}
\mbox{So }\vec{N}^2&=0\\
\implies a^2&=0\\
\therefore a&=0\\
\mbox{Thus }\vec{N}&=\myvec{a&0\\c&0}
\end{align}}\\
&In this case $\vec{N}$ is similar to $\vec{N}=\myvec{0&0\\1&0}$ via the matrix $\vec{P}=\myvec{c&0\\0&1}$\\
&\\
\hline
&\\
Assuming $\myvec{a\\c}$ as&Therefore
$\vec{N}=\myvec{0&b\\0&d}$\\
the zero vector&\\
&\parbox{6cm}{\begin{align}
\mbox{Then }\vec{N}^2&=0\\
\implies d^2&=0\\
\therefore d&=0\\
\mbox{Thus }\vec{N}&=\myvec{0&b\\0&0}
\end{align}}\\
&In this case $\vec{N}$ is similar to $\vec{N}=\myvec{0&0\\b&0}$ via the matrix $\vec{P}=\myvec{0&1\\1&0}$,\\
&which is similar to $\myvec{0&0\\1&0}$ as above.\\
&\\
\hline
&\\
Hence& we can assume neither $\myvec{a\\c}$ or $\myvec{b\\d}$ is the zero vector.\\
&\\
%\hline
%\pagebreak
\hline
&\\
Consequences of &Since they are linearly dependent we can assume,\\
linear&\\
independence&\\
&\parbox{6cm}{\begin{align}
\myvec{b\\d}&=x\myvec{a\\c}\\
\therefore \vec{N}&=\myvec{a&ax\\c&cx}\\
\therefore \vec{N}^2&=0\\
\implies a(a+cx)&=0\\
c(a+cx)&=0\\
ax(a+cx)&=0\\
cx(a+cx)&=0
\end{align}}\\
\hline
&\\
Proof that $\vec{N}$ is&We know that at least one of a or c is not zero.\\
similar over $\mathbb{C}$ to&If a = 0 then c $\neq$ 0, it must be that x = 0.\\
$\myvec{0&0\\1&0}$&So in this case $\vec{N}=\myvec{0&0\\c&0}$ which is similar to $\myvec{0&0\\1&0}$ as before.\\
&\\
&\parbox{6cm}{\begin{align}
\mbox{If } a &\neq0\\
\mbox{then }x &\neq0\\
\mbox{else }a(a+cx)&=0\\
\implies a&=0\\
\mbox{Thus }a+cx&=0\\
\mbox{Hence }\vec{N}&=\myvec{a&ax\\ \frac{-a}{x}&-a}
\end{align}}\\
 &This is similar to $\myvec{a&a\\-a&-a}$ via $\vec{P}=\myvec{\sqrt{x}&0\\0&\frac{1}{\sqrt{x}}}$.\\
 &And $\myvec{a&a\\-a&-a}$ is similar to $\myvec{0&0\\-a&0}$ via $\vec{P}=\myvec{-1&-1\\1&0}$\\
 &And this finally is similar to $\myvec{0&0\\1&0}$ as before.\\
 &\\
\hline
&\\
Conclusion &Thus either $\vec{N}$ = 0 or $\vec{N}$ is similar over $\mathbb{C}$ to $\myvec{0&0\\1&0}$.\\
&\\
\hline
\caption{Solution summary}
\label{eq:solutions/6/2/11/table:1}
\end{longtable}

\item Consider the matrix
\begin{align}
A(x) = \myvec{1+x^2&7&11\\3x&2x&4\\8x&17&13} & ;x\in \vec{R}.
\end{align}
Then,
\begin{enumerate}
\item A(x) has eigenvalue 0 for some $x\in \vec{R}$.
\item 0 is not an eigenvalue of A(x) for any $x\in \vec{R}$.
\item A(x) has eigenvalue 0 $\forall x\in \vec{R}$.
\item A(x) is invertible $\forall x\in \vec{R}$.
\end{enumerate}
%
\solution
See Table \ref{eq:solutions/6/2/11/table:1}
\onecolumn
\begin{longtable}{|l|l|}
\hline
\multirow{3}{*}{} & \\
Statement&Solution\\
\hline
&\parbox{6cm}{\begin{align}
\mbox{Let }\vec{N}&=\myvec{a&b\\c&d}\\
\mbox{Since }\vec{N}^2&=0
\end{align}}\\
&If $\myvec{a\\c},\myvec{b\\d}$ are linearly independent then $\vec{N}$ is diagonalizable to $\myvec{0&0\\0&0}$.\\
&\parbox{6cm}{\begin{align}
\mbox{If }\vec{P}\vec{N}\vec{P}^{-1}=0\\
\mbox{then }\vec{N}=\vec{P}^{-1}\vec{0}\vec{P}=0
\end{align}}\\
Proof that&So in this case $\vec{N}$ itself is the zero matrix.\\
$\vec{N}=0$&This contradicts the assumption that $\myvec{a\\c},\myvec{b\\d}$ are linearly independent.\\
&$\therefore$ we can assume that $\myvec{a\\c},\myvec{b\\d}$ are linearly dependent if both are\\
&equal to the zero vector\\
&\parbox{6cm}
{\begin{align}
   \mbox{then } \vec{N} &= 0.
\end{align}}\\
%\hline
%\pagebreak
\hline
&\\
&Therefore we can assume at least one vector is non-zero.\\
Assuming $\myvec{b\\d}$ as&Therefore
$\vec{N}=\myvec{a&0\\c&0}$\\
the zero vector&\\
&\parbox{6cm}{\begin{align}
\mbox{So }\vec{N}^2&=0\\
\implies a^2&=0\\
\therefore a&=0\\
\mbox{Thus }\vec{N}&=\myvec{a&0\\c&0}
\end{align}}\\
&In this case $\vec{N}$ is similar to $\vec{N}=\myvec{0&0\\1&0}$ via the matrix $\vec{P}=\myvec{c&0\\0&1}$\\
&\\
\hline
&\\
Assuming $\myvec{a\\c}$ as&Therefore
$\vec{N}=\myvec{0&b\\0&d}$\\
the zero vector&\\
&\parbox{6cm}{\begin{align}
\mbox{Then }\vec{N}^2&=0\\
\implies d^2&=0\\
\therefore d&=0\\
\mbox{Thus }\vec{N}&=\myvec{0&b\\0&0}
\end{align}}\\
&In this case $\vec{N}$ is similar to $\vec{N}=\myvec{0&0\\b&0}$ via the matrix $\vec{P}=\myvec{0&1\\1&0}$,\\
&which is similar to $\myvec{0&0\\1&0}$ as above.\\
&\\
\hline
&\\
Hence& we can assume neither $\myvec{a\\c}$ or $\myvec{b\\d}$ is the zero vector.\\
&\\
%\hline
%\pagebreak
\hline
&\\
Consequences of &Since they are linearly dependent we can assume,\\
linear&\\
independence&\\
&\parbox{6cm}{\begin{align}
\myvec{b\\d}&=x\myvec{a\\c}\\
\therefore \vec{N}&=\myvec{a&ax\\c&cx}\\
\therefore \vec{N}^2&=0\\
\implies a(a+cx)&=0\\
c(a+cx)&=0\\
ax(a+cx)&=0\\
cx(a+cx)&=0
\end{align}}\\
\hline
&\\
Proof that $\vec{N}$ is&We know that at least one of a or c is not zero.\\
similar over $\mathbb{C}$ to&If a = 0 then c $\neq$ 0, it must be that x = 0.\\
$\myvec{0&0\\1&0}$&So in this case $\vec{N}=\myvec{0&0\\c&0}$ which is similar to $\myvec{0&0\\1&0}$ as before.\\
&\\
&\parbox{6cm}{\begin{align}
\mbox{If } a &\neq0\\
\mbox{then }x &\neq0\\
\mbox{else }a(a+cx)&=0\\
\implies a&=0\\
\mbox{Thus }a+cx&=0\\
\mbox{Hence }\vec{N}&=\myvec{a&ax\\ \frac{-a}{x}&-a}
\end{align}}\\
 &This is similar to $\myvec{a&a\\-a&-a}$ via $\vec{P}=\myvec{\sqrt{x}&0\\0&\frac{1}{\sqrt{x}}}$.\\
 &And $\myvec{a&a\\-a&-a}$ is similar to $\myvec{0&0\\-a&0}$ via $\vec{P}=\myvec{-1&-1\\1&0}$\\
 &And this finally is similar to $\myvec{0&0\\1&0}$ as before.\\
 &\\
\hline
&\\
Conclusion &Thus either $\vec{N}$ = 0 or $\vec{N}$ is similar over $\mathbb{C}$ to $\myvec{0&0\\1&0}$.\\
&\\
\hline
\caption{Solution summary}
\label{eq:solutions/6/2/11/table:1}
\end{longtable}

\end{enumerate}

\section{December 2017}
\renewcommand{\theequation}{\theenumi}
\renewcommand{\thefigure}{\theenumi}
\renewcommand{\thetable}{\theenumi}
\begin{enumerate}[label=\thesection.\arabic*.,ref=\thesection.\theenumi]
\numberwithin{equation}{enumi}
\numberwithin{figure}{enumi}
\numberwithin{table}{enumi}

\item %
Let $\vec{A}=\myvec{1 & 1\\1 & 0}$ and let $\alpha_n$ and $\beta_n$ denote the two eigenvalues of $\vec{A}^n$ such that $\abs{\alpha_n}\geq\abs{\beta_n}$.\\
Then
\begin{enumerate}
    \item $\alpha_n\rightarrow \infty$ as $n\rightarrow \infty$
    \item $\beta_n\rightarrow 0$ as $n\rightarrow \infty$
    \item $\beta_n$ is positive if n is even.
    \item $\beta_n$ is negative if n is odd.
\end{enumerate}
%
\solution
See Table \ref{eq:solutions/6/2/11/table:1}
\onecolumn
\begin{longtable}{|l|l|}
\hline
\multirow{3}{*}{} & \\
Statement&Solution\\
\hline
&\parbox{6cm}{\begin{align}
\mbox{Let }\vec{N}&=\myvec{a&b\\c&d}\\
\mbox{Since }\vec{N}^2&=0
\end{align}}\\
&If $\myvec{a\\c},\myvec{b\\d}$ are linearly independent then $\vec{N}$ is diagonalizable to $\myvec{0&0\\0&0}$.\\
&\parbox{6cm}{\begin{align}
\mbox{If }\vec{P}\vec{N}\vec{P}^{-1}=0\\
\mbox{then }\vec{N}=\vec{P}^{-1}\vec{0}\vec{P}=0
\end{align}}\\
Proof that&So in this case $\vec{N}$ itself is the zero matrix.\\
$\vec{N}=0$&This contradicts the assumption that $\myvec{a\\c},\myvec{b\\d}$ are linearly independent.\\
&$\therefore$ we can assume that $\myvec{a\\c},\myvec{b\\d}$ are linearly dependent if both are\\
&equal to the zero vector\\
&\parbox{6cm}
{\begin{align}
   \mbox{then } \vec{N} &= 0.
\end{align}}\\
%\hline
%\pagebreak
\hline
&\\
&Therefore we can assume at least one vector is non-zero.\\
Assuming $\myvec{b\\d}$ as&Therefore
$\vec{N}=\myvec{a&0\\c&0}$\\
the zero vector&\\
&\parbox{6cm}{\begin{align}
\mbox{So }\vec{N}^2&=0\\
\implies a^2&=0\\
\therefore a&=0\\
\mbox{Thus }\vec{N}&=\myvec{a&0\\c&0}
\end{align}}\\
&In this case $\vec{N}$ is similar to $\vec{N}=\myvec{0&0\\1&0}$ via the matrix $\vec{P}=\myvec{c&0\\0&1}$\\
&\\
\hline
&\\
Assuming $\myvec{a\\c}$ as&Therefore
$\vec{N}=\myvec{0&b\\0&d}$\\
the zero vector&\\
&\parbox{6cm}{\begin{align}
\mbox{Then }\vec{N}^2&=0\\
\implies d^2&=0\\
\therefore d&=0\\
\mbox{Thus }\vec{N}&=\myvec{0&b\\0&0}
\end{align}}\\
&In this case $\vec{N}$ is similar to $\vec{N}=\myvec{0&0\\b&0}$ via the matrix $\vec{P}=\myvec{0&1\\1&0}$,\\
&which is similar to $\myvec{0&0\\1&0}$ as above.\\
&\\
\hline
&\\
Hence& we can assume neither $\myvec{a\\c}$ or $\myvec{b\\d}$ is the zero vector.\\
&\\
%\hline
%\pagebreak
\hline
&\\
Consequences of &Since they are linearly dependent we can assume,\\
linear&\\
independence&\\
&\parbox{6cm}{\begin{align}
\myvec{b\\d}&=x\myvec{a\\c}\\
\therefore \vec{N}&=\myvec{a&ax\\c&cx}\\
\therefore \vec{N}^2&=0\\
\implies a(a+cx)&=0\\
c(a+cx)&=0\\
ax(a+cx)&=0\\
cx(a+cx)&=0
\end{align}}\\
\hline
&\\
Proof that $\vec{N}$ is&We know that at least one of a or c is not zero.\\
similar over $\mathbb{C}$ to&If a = 0 then c $\neq$ 0, it must be that x = 0.\\
$\myvec{0&0\\1&0}$&So in this case $\vec{N}=\myvec{0&0\\c&0}$ which is similar to $\myvec{0&0\\1&0}$ as before.\\
&\\
&\parbox{6cm}{\begin{align}
\mbox{If } a &\neq0\\
\mbox{then }x &\neq0\\
\mbox{else }a(a+cx)&=0\\
\implies a&=0\\
\mbox{Thus }a+cx&=0\\
\mbox{Hence }\vec{N}&=\myvec{a&ax\\ \frac{-a}{x}&-a}
\end{align}}\\
 &This is similar to $\myvec{a&a\\-a&-a}$ via $\vec{P}=\myvec{\sqrt{x}&0\\0&\frac{1}{\sqrt{x}}}$.\\
 &And $\myvec{a&a\\-a&-a}$ is similar to $\myvec{0&0\\-a&0}$ via $\vec{P}=\myvec{-1&-1\\1&0}$\\
 &And this finally is similar to $\myvec{0&0\\1&0}$ as before.\\
 &\\
\hline
&\\
Conclusion &Thus either $\vec{N}$ = 0 or $\vec{N}$ is similar over $\mathbb{C}$ to $\myvec{0&0\\1&0}$.\\
&\\
\hline
\caption{Solution summary}
\label{eq:solutions/6/2/11/table:1}
\end{longtable}

\item Let $M_n$ denote the vector space of all $n\times n$ real matrices. Which of the following is a linear subspaces of $M_n$ :-
\begin{enumerate}
\item $ V_1 = \{  A \in M_n : \text{ A is nonsingular} \}$
\item $ V_2 = \{  A \in M_n : det(A) = 0 \}$
\item $ V_3 = \{  A \in M_n : trace(A) = 0 \}$
\item $ V_4 = \{  BA : A \in M_n\},$ where $ B$ is some fixed matrix in $ M_n$
\end{enumerate}
%
\solution
See Table \ref{eq:solutions/6/2/11/table:1}
\onecolumn
\begin{longtable}{|l|l|}
\hline
\multirow{3}{*}{} & \\
Statement&Solution\\
\hline
&\parbox{6cm}{\begin{align}
\mbox{Let }\vec{N}&=\myvec{a&b\\c&d}\\
\mbox{Since }\vec{N}^2&=0
\end{align}}\\
&If $\myvec{a\\c},\myvec{b\\d}$ are linearly independent then $\vec{N}$ is diagonalizable to $\myvec{0&0\\0&0}$.\\
&\parbox{6cm}{\begin{align}
\mbox{If }\vec{P}\vec{N}\vec{P}^{-1}=0\\
\mbox{then }\vec{N}=\vec{P}^{-1}\vec{0}\vec{P}=0
\end{align}}\\
Proof that&So in this case $\vec{N}$ itself is the zero matrix.\\
$\vec{N}=0$&This contradicts the assumption that $\myvec{a\\c},\myvec{b\\d}$ are linearly independent.\\
&$\therefore$ we can assume that $\myvec{a\\c},\myvec{b\\d}$ are linearly dependent if both are\\
&equal to the zero vector\\
&\parbox{6cm}
{\begin{align}
   \mbox{then } \vec{N} &= 0.
\end{align}}\\
%\hline
%\pagebreak
\hline
&\\
&Therefore we can assume at least one vector is non-zero.\\
Assuming $\myvec{b\\d}$ as&Therefore
$\vec{N}=\myvec{a&0\\c&0}$\\
the zero vector&\\
&\parbox{6cm}{\begin{align}
\mbox{So }\vec{N}^2&=0\\
\implies a^2&=0\\
\therefore a&=0\\
\mbox{Thus }\vec{N}&=\myvec{a&0\\c&0}
\end{align}}\\
&In this case $\vec{N}$ is similar to $\vec{N}=\myvec{0&0\\1&0}$ via the matrix $\vec{P}=\myvec{c&0\\0&1}$\\
&\\
\hline
&\\
Assuming $\myvec{a\\c}$ as&Therefore
$\vec{N}=\myvec{0&b\\0&d}$\\
the zero vector&\\
&\parbox{6cm}{\begin{align}
\mbox{Then }\vec{N}^2&=0\\
\implies d^2&=0\\
\therefore d&=0\\
\mbox{Thus }\vec{N}&=\myvec{0&b\\0&0}
\end{align}}\\
&In this case $\vec{N}$ is similar to $\vec{N}=\myvec{0&0\\b&0}$ via the matrix $\vec{P}=\myvec{0&1\\1&0}$,\\
&which is similar to $\myvec{0&0\\1&0}$ as above.\\
&\\
\hline
&\\
Hence& we can assume neither $\myvec{a\\c}$ or $\myvec{b\\d}$ is the zero vector.\\
&\\
%\hline
%\pagebreak
\hline
&\\
Consequences of &Since they are linearly dependent we can assume,\\
linear&\\
independence&\\
&\parbox{6cm}{\begin{align}
\myvec{b\\d}&=x\myvec{a\\c}\\
\therefore \vec{N}&=\myvec{a&ax\\c&cx}\\
\therefore \vec{N}^2&=0\\
\implies a(a+cx)&=0\\
c(a+cx)&=0\\
ax(a+cx)&=0\\
cx(a+cx)&=0
\end{align}}\\
\hline
&\\
Proof that $\vec{N}$ is&We know that at least one of a or c is not zero.\\
similar over $\mathbb{C}$ to&If a = 0 then c $\neq$ 0, it must be that x = 0.\\
$\myvec{0&0\\1&0}$&So in this case $\vec{N}=\myvec{0&0\\c&0}$ which is similar to $\myvec{0&0\\1&0}$ as before.\\
&\\
&\parbox{6cm}{\begin{align}
\mbox{If } a &\neq0\\
\mbox{then }x &\neq0\\
\mbox{else }a(a+cx)&=0\\
\implies a&=0\\
\mbox{Thus }a+cx&=0\\
\mbox{Hence }\vec{N}&=\myvec{a&ax\\ \frac{-a}{x}&-a}
\end{align}}\\
 &This is similar to $\myvec{a&a\\-a&-a}$ via $\vec{P}=\myvec{\sqrt{x}&0\\0&\frac{1}{\sqrt{x}}}$.\\
 &And $\myvec{a&a\\-a&-a}$ is similar to $\myvec{0&0\\-a&0}$ via $\vec{P}=\myvec{-1&-1\\1&0}$\\
 &And this finally is similar to $\myvec{0&0\\1&0}$ as before.\\
 &\\
\hline
&\\
Conclusion &Thus either $\vec{N}$ = 0 or $\vec{N}$ is similar over $\mathbb{C}$ to $\myvec{0&0\\1&0}$.\\
&\\
\hline
\caption{Solution summary}
\label{eq:solutions/6/2/11/table:1}
\end{longtable}


\item If $\vec{P}$ and $\vec{Q}$ are invertible matrices such that   
$\vec{P}\vec{Q} = -\vec{Q}\vec{P}$,then we can conclude that
\begin{enumerate}
\item  $Tr(\vec{P})=Tr(\vec{Q})=0$ \label{eq:solutions/2016/dec/75/1}
\item  $Tr(\vec{P})=Tr(\vec{Q})=1$ \label{eq:solutions/2016/dec/75/2}
\item  $Tr(\vec{P})=-Tr(\vec{Q})$ \label{eq:solutions/2016/dec/75/3}
\item  $Tr(\vec{P}) \neq Tr(\vec{Q})$ \label{eq:solutions/2016/dec/75/4}
\end{enumerate}
%
%
\solution
See Table \ref{eq:solutions/6/2/11/table:1}
\onecolumn
\begin{longtable}{|l|l|}
\hline
\multirow{3}{*}{} & \\
Statement&Solution\\
\hline
&\parbox{6cm}{\begin{align}
\mbox{Let }\vec{N}&=\myvec{a&b\\c&d}\\
\mbox{Since }\vec{N}^2&=0
\end{align}}\\
&If $\myvec{a\\c},\myvec{b\\d}$ are linearly independent then $\vec{N}$ is diagonalizable to $\myvec{0&0\\0&0}$.\\
&\parbox{6cm}{\begin{align}
\mbox{If }\vec{P}\vec{N}\vec{P}^{-1}=0\\
\mbox{then }\vec{N}=\vec{P}^{-1}\vec{0}\vec{P}=0
\end{align}}\\
Proof that&So in this case $\vec{N}$ itself is the zero matrix.\\
$\vec{N}=0$&This contradicts the assumption that $\myvec{a\\c},\myvec{b\\d}$ are linearly independent.\\
&$\therefore$ we can assume that $\myvec{a\\c},\myvec{b\\d}$ are linearly dependent if both are\\
&equal to the zero vector\\
&\parbox{6cm}
{\begin{align}
   \mbox{then } \vec{N} &= 0.
\end{align}}\\
%\hline
%\pagebreak
\hline
&\\
&Therefore we can assume at least one vector is non-zero.\\
Assuming $\myvec{b\\d}$ as&Therefore
$\vec{N}=\myvec{a&0\\c&0}$\\
the zero vector&\\
&\parbox{6cm}{\begin{align}
\mbox{So }\vec{N}^2&=0\\
\implies a^2&=0\\
\therefore a&=0\\
\mbox{Thus }\vec{N}&=\myvec{a&0\\c&0}
\end{align}}\\
&In this case $\vec{N}$ is similar to $\vec{N}=\myvec{0&0\\1&0}$ via the matrix $\vec{P}=\myvec{c&0\\0&1}$\\
&\\
\hline
&\\
Assuming $\myvec{a\\c}$ as&Therefore
$\vec{N}=\myvec{0&b\\0&d}$\\
the zero vector&\\
&\parbox{6cm}{\begin{align}
\mbox{Then }\vec{N}^2&=0\\
\implies d^2&=0\\
\therefore d&=0\\
\mbox{Thus }\vec{N}&=\myvec{0&b\\0&0}
\end{align}}\\
&In this case $\vec{N}$ is similar to $\vec{N}=\myvec{0&0\\b&0}$ via the matrix $\vec{P}=\myvec{0&1\\1&0}$,\\
&which is similar to $\myvec{0&0\\1&0}$ as above.\\
&\\
\hline
&\\
Hence& we can assume neither $\myvec{a\\c}$ or $\myvec{b\\d}$ is the zero vector.\\
&\\
%\hline
%\pagebreak
\hline
&\\
Consequences of &Since they are linearly dependent we can assume,\\
linear&\\
independence&\\
&\parbox{6cm}{\begin{align}
\myvec{b\\d}&=x\myvec{a\\c}\\
\therefore \vec{N}&=\myvec{a&ax\\c&cx}\\
\therefore \vec{N}^2&=0\\
\implies a(a+cx)&=0\\
c(a+cx)&=0\\
ax(a+cx)&=0\\
cx(a+cx)&=0
\end{align}}\\
\hline
&\\
Proof that $\vec{N}$ is&We know that at least one of a or c is not zero.\\
similar over $\mathbb{C}$ to&If a = 0 then c $\neq$ 0, it must be that x = 0.\\
$\myvec{0&0\\1&0}$&So in this case $\vec{N}=\myvec{0&0\\c&0}$ which is similar to $\myvec{0&0\\1&0}$ as before.\\
&\\
&\parbox{6cm}{\begin{align}
\mbox{If } a &\neq0\\
\mbox{then }x &\neq0\\
\mbox{else }a(a+cx)&=0\\
\implies a&=0\\
\mbox{Thus }a+cx&=0\\
\mbox{Hence }\vec{N}&=\myvec{a&ax\\ \frac{-a}{x}&-a}
\end{align}}\\
 &This is similar to $\myvec{a&a\\-a&-a}$ via $\vec{P}=\myvec{\sqrt{x}&0\\0&\frac{1}{\sqrt{x}}}$.\\
 &And $\myvec{a&a\\-a&-a}$ is similar to $\myvec{0&0\\-a&0}$ via $\vec{P}=\myvec{-1&-1\\1&0}$\\
 &And this finally is similar to $\myvec{0&0\\1&0}$ as before.\\
 &\\
\hline
&\\
Conclusion &Thus either $\vec{N}$ = 0 or $\vec{N}$ is similar over $\mathbb{C}$ to $\myvec{0&0\\1&0}$.\\
&\\
\hline
\caption{Solution summary}
\label{eq:solutions/6/2/11/table:1}
\end{longtable}

\twocolumn
\item Let $\vec{W_1}$, $\vec{W_2}$, $\vec{W_3}$ be 3 distinct subspaces of $\vec{R}^{10}$ such that each $\vec{W_i}$ has dimension of 9. Let $\vec{W} = \vec{W_1} \cap \vec{W_2} \cap \vec{W_3}$. Then we can conclude that\\
\begin{enumerate}
\item $\vec{W}$ may not be a subspace of $\vec{R}^{10}$\\
\item dim $\vec{W} \leq 8$\\
\item dim $\vec{W} \geq 7$\\
\item  dim $\vec{W} \leq 3$\\
\end{enumerate}
 %
\solution
See Table \ref{eq:solutions/6/2/11/table:1}
\onecolumn
\begin{longtable}{|l|l|}
\hline
\multirow{3}{*}{} & \\
Statement&Solution\\
\hline
&\parbox{6cm}{\begin{align}
\mbox{Let }\vec{N}&=\myvec{a&b\\c&d}\\
\mbox{Since }\vec{N}^2&=0
\end{align}}\\
&If $\myvec{a\\c},\myvec{b\\d}$ are linearly independent then $\vec{N}$ is diagonalizable to $\myvec{0&0\\0&0}$.\\
&\parbox{6cm}{\begin{align}
\mbox{If }\vec{P}\vec{N}\vec{P}^{-1}=0\\
\mbox{then }\vec{N}=\vec{P}^{-1}\vec{0}\vec{P}=0
\end{align}}\\
Proof that&So in this case $\vec{N}$ itself is the zero matrix.\\
$\vec{N}=0$&This contradicts the assumption that $\myvec{a\\c},\myvec{b\\d}$ are linearly independent.\\
&$\therefore$ we can assume that $\myvec{a\\c},\myvec{b\\d}$ are linearly dependent if both are\\
&equal to the zero vector\\
&\parbox{6cm}
{\begin{align}
   \mbox{then } \vec{N} &= 0.
\end{align}}\\
%\hline
%\pagebreak
\hline
&\\
&Therefore we can assume at least one vector is non-zero.\\
Assuming $\myvec{b\\d}$ as&Therefore
$\vec{N}=\myvec{a&0\\c&0}$\\
the zero vector&\\
&\parbox{6cm}{\begin{align}
\mbox{So }\vec{N}^2&=0\\
\implies a^2&=0\\
\therefore a&=0\\
\mbox{Thus }\vec{N}&=\myvec{a&0\\c&0}
\end{align}}\\
&In this case $\vec{N}$ is similar to $\vec{N}=\myvec{0&0\\1&0}$ via the matrix $\vec{P}=\myvec{c&0\\0&1}$\\
&\\
\hline
&\\
Assuming $\myvec{a\\c}$ as&Therefore
$\vec{N}=\myvec{0&b\\0&d}$\\
the zero vector&\\
&\parbox{6cm}{\begin{align}
\mbox{Then }\vec{N}^2&=0\\
\implies d^2&=0\\
\therefore d&=0\\
\mbox{Thus }\vec{N}&=\myvec{0&b\\0&0}
\end{align}}\\
&In this case $\vec{N}$ is similar to $\vec{N}=\myvec{0&0\\b&0}$ via the matrix $\vec{P}=\myvec{0&1\\1&0}$,\\
&which is similar to $\myvec{0&0\\1&0}$ as above.\\
&\\
\hline
&\\
Hence& we can assume neither $\myvec{a\\c}$ or $\myvec{b\\d}$ is the zero vector.\\
&\\
%\hline
%\pagebreak
\hline
&\\
Consequences of &Since they are linearly dependent we can assume,\\
linear&\\
independence&\\
&\parbox{6cm}{\begin{align}
\myvec{b\\d}&=x\myvec{a\\c}\\
\therefore \vec{N}&=\myvec{a&ax\\c&cx}\\
\therefore \vec{N}^2&=0\\
\implies a(a+cx)&=0\\
c(a+cx)&=0\\
ax(a+cx)&=0\\
cx(a+cx)&=0
\end{align}}\\
\hline
&\\
Proof that $\vec{N}$ is&We know that at least one of a or c is not zero.\\
similar over $\mathbb{C}$ to&If a = 0 then c $\neq$ 0, it must be that x = 0.\\
$\myvec{0&0\\1&0}$&So in this case $\vec{N}=\myvec{0&0\\c&0}$ which is similar to $\myvec{0&0\\1&0}$ as before.\\
&\\
&\parbox{6cm}{\begin{align}
\mbox{If } a &\neq0\\
\mbox{then }x &\neq0\\
\mbox{else }a(a+cx)&=0\\
\implies a&=0\\
\mbox{Thus }a+cx&=0\\
\mbox{Hence }\vec{N}&=\myvec{a&ax\\ \frac{-a}{x}&-a}
\end{align}}\\
 &This is similar to $\myvec{a&a\\-a&-a}$ via $\vec{P}=\myvec{\sqrt{x}&0\\0&\frac{1}{\sqrt{x}}}$.\\
 &And $\myvec{a&a\\-a&-a}$ is similar to $\myvec{0&0\\-a&0}$ via $\vec{P}=\myvec{-1&-1\\1&0}$\\
 &And this finally is similar to $\myvec{0&0\\1&0}$ as before.\\
 &\\
\hline
&\\
Conclusion &Thus either $\vec{N}$ = 0 or $\vec{N}$ is similar over $\mathbb{C}$ to $\myvec{0&0\\1&0}$.\\
&\\
\hline
\caption{Solution summary}
\label{eq:solutions/6/2/11/table:1}
\end{longtable}

\twocolumn


%\item Consider a Markov Chain with state space $\cbrak{0,1,2}$ and transition matrix
%\begin{align}
%P = 
%\begin{blockarray}{c@{\hspace{1pt}}rrr@{\hspace{3pt}}}
%         & 0   & 1   & 2 \\
%        \begin{block}{r@{\hspace{3pt}}@{\hspace{1pt}}
%    (@{\hspace{1pt}}rrr@{\hspace{1pt}}@{\hspace{1pt}})}
%        0 & \frac{1}{2} & \frac{1}{2} & 0  \\
%        1 & 0 &\frac{1}{2}  & \frac{3}{4}  \\
%%
%        2 &  \frac{1}{3} & \frac{1}{3} & \frac{1}{3}  \\
%        \end{block}
%    \end{blockarray}
%\end{align}
%For any two states $i$ and $j$, let $p_{ij}^{(n)}$ denote the $n$-step transition probability of going from $i$ to $j$.  Identify correct statements.
%\begin{enumerate}
%\item $\lim_{n \to \infty} p_{11}^{(n)} = \frac{2}{9}$
%\item $\lim_{n \to \infty} p_{21}^{(n)} = 0$
%\item $\lim_{n \to \infty} p_{32}^{(n)} = \frac{1}{3}$
%\item $\lim_{n \to \infty} p_{13}^{(n)} = \frac{1}{3}$
%\end{enumerate}

\end{enumerate}

\twocolumn
\section{June 2017}
\renewcommand{\theequation}{\theenumi}
\renewcommand{\thefigure}{\theenumi}
\begin{enumerate}[label=\thesection.\arabic*.,ref=\thesection.\theenumi]
\numberwithin{equation}{enumi}
\numberwithin{figure}{enumi}

\item Let $\vec{A}$ be an n $\times$ n self-adjoint matrix with eigenvalues $\lambda_1, \cdots, \lambda_2$.
Let,\begin{align} \norm{\vec{X}}_2=\sqrt{\vert\vec{X}_{1}^{2}\vert+\cdots+\vert\vec{X}_{n}^{2}\vert}\end{align} for $\vec{X}$=$(\vec{X}_{1},\cdots,\vec{X}_{n})\in \mathbb{C}^n$. If \begin{align}
p(\vec{A})=a_0\vec{I}+a_1\vec{A}+\cdots+a_n\vec{A}^n
\end{align}
then $sup_{\norm{\vec{X}}_{2}=1}\norm{p(\vec{A})\vec{X}}_2$ is equal to
%
\\
\solution
See Table \ref{eq:solutions/6/2/11/table:1}
\onecolumn
\begin{longtable}{|l|l|}
\hline
\multirow{3}{*}{} & \\
Statement&Solution\\
\hline
&\parbox{6cm}{\begin{align}
\mbox{Let }\vec{N}&=\myvec{a&b\\c&d}\\
\mbox{Since }\vec{N}^2&=0
\end{align}}\\
&If $\myvec{a\\c},\myvec{b\\d}$ are linearly independent then $\vec{N}$ is diagonalizable to $\myvec{0&0\\0&0}$.\\
&\parbox{6cm}{\begin{align}
\mbox{If }\vec{P}\vec{N}\vec{P}^{-1}=0\\
\mbox{then }\vec{N}=\vec{P}^{-1}\vec{0}\vec{P}=0
\end{align}}\\
Proof that&So in this case $\vec{N}$ itself is the zero matrix.\\
$\vec{N}=0$&This contradicts the assumption that $\myvec{a\\c},\myvec{b\\d}$ are linearly independent.\\
&$\therefore$ we can assume that $\myvec{a\\c},\myvec{b\\d}$ are linearly dependent if both are\\
&equal to the zero vector\\
&\parbox{6cm}
{\begin{align}
   \mbox{then } \vec{N} &= 0.
\end{align}}\\
%\hline
%\pagebreak
\hline
&\\
&Therefore we can assume at least one vector is non-zero.\\
Assuming $\myvec{b\\d}$ as&Therefore
$\vec{N}=\myvec{a&0\\c&0}$\\
the zero vector&\\
&\parbox{6cm}{\begin{align}
\mbox{So }\vec{N}^2&=0\\
\implies a^2&=0\\
\therefore a&=0\\
\mbox{Thus }\vec{N}&=\myvec{a&0\\c&0}
\end{align}}\\
&In this case $\vec{N}$ is similar to $\vec{N}=\myvec{0&0\\1&0}$ via the matrix $\vec{P}=\myvec{c&0\\0&1}$\\
&\\
\hline
&\\
Assuming $\myvec{a\\c}$ as&Therefore
$\vec{N}=\myvec{0&b\\0&d}$\\
the zero vector&\\
&\parbox{6cm}{\begin{align}
\mbox{Then }\vec{N}^2&=0\\
\implies d^2&=0\\
\therefore d&=0\\
\mbox{Thus }\vec{N}&=\myvec{0&b\\0&0}
\end{align}}\\
&In this case $\vec{N}$ is similar to $\vec{N}=\myvec{0&0\\b&0}$ via the matrix $\vec{P}=\myvec{0&1\\1&0}$,\\
&which is similar to $\myvec{0&0\\1&0}$ as above.\\
&\\
\hline
&\\
Hence& we can assume neither $\myvec{a\\c}$ or $\myvec{b\\d}$ is the zero vector.\\
&\\
%\hline
%\pagebreak
\hline
&\\
Consequences of &Since they are linearly dependent we can assume,\\
linear&\\
independence&\\
&\parbox{6cm}{\begin{align}
\myvec{b\\d}&=x\myvec{a\\c}\\
\therefore \vec{N}&=\myvec{a&ax\\c&cx}\\
\therefore \vec{N}^2&=0\\
\implies a(a+cx)&=0\\
c(a+cx)&=0\\
ax(a+cx)&=0\\
cx(a+cx)&=0
\end{align}}\\
\hline
&\\
Proof that $\vec{N}$ is&We know that at least one of a or c is not zero.\\
similar over $\mathbb{C}$ to&If a = 0 then c $\neq$ 0, it must be that x = 0.\\
$\myvec{0&0\\1&0}$&So in this case $\vec{N}=\myvec{0&0\\c&0}$ which is similar to $\myvec{0&0\\1&0}$ as before.\\
&\\
&\parbox{6cm}{\begin{align}
\mbox{If } a &\neq0\\
\mbox{then }x &\neq0\\
\mbox{else }a(a+cx)&=0\\
\implies a&=0\\
\mbox{Thus }a+cx&=0\\
\mbox{Hence }\vec{N}&=\myvec{a&ax\\ \frac{-a}{x}&-a}
\end{align}}\\
 &This is similar to $\myvec{a&a\\-a&-a}$ via $\vec{P}=\myvec{\sqrt{x}&0\\0&\frac{1}{\sqrt{x}}}$.\\
 &And $\myvec{a&a\\-a&-a}$ is similar to $\myvec{0&0\\-a&0}$ via $\vec{P}=\myvec{-1&-1\\1&0}$\\
 &And this finally is similar to $\myvec{0&0\\1&0}$ as before.\\
 &\\
\hline
&\\
Conclusion &Thus either $\vec{N}$ = 0 or $\vec{N}$ is similar over $\mathbb{C}$ to $\myvec{0&0\\1&0}$.\\
&\\
\hline
\caption{Solution summary}
\label{eq:solutions/6/2/11/table:1}
\end{longtable}

%
\item Find the Eigenvalues of the matrix,
\begin{align}
\vec{A} = \myvec{1 & 1 & 2 \\ 1 & -2 & 5 \\ 2 & 5 & -3 }\label{eq:solutions/2017/june/32/eq:1}
\end{align}
\begin{enumerate}
\item -4, 3, -3
\item 4, 3, 1
\item 4, -4$\pm\sqrt{13}$
\item 4, -2$\pm\sqrt{7}$
\end{enumerate}
%
%
\solution
See Table \ref{eq:solutions/6/2/11/table:1}
\onecolumn
\begin{longtable}{|l|l|}
\hline
\multirow{3}{*}{} & \\
Statement&Solution\\
\hline
&\parbox{6cm}{\begin{align}
\mbox{Let }\vec{N}&=\myvec{a&b\\c&d}\\
\mbox{Since }\vec{N}^2&=0
\end{align}}\\
&If $\myvec{a\\c},\myvec{b\\d}$ are linearly independent then $\vec{N}$ is diagonalizable to $\myvec{0&0\\0&0}$.\\
&\parbox{6cm}{\begin{align}
\mbox{If }\vec{P}\vec{N}\vec{P}^{-1}=0\\
\mbox{then }\vec{N}=\vec{P}^{-1}\vec{0}\vec{P}=0
\end{align}}\\
Proof that&So in this case $\vec{N}$ itself is the zero matrix.\\
$\vec{N}=0$&This contradicts the assumption that $\myvec{a\\c},\myvec{b\\d}$ are linearly independent.\\
&$\therefore$ we can assume that $\myvec{a\\c},\myvec{b\\d}$ are linearly dependent if both are\\
&equal to the zero vector\\
&\parbox{6cm}
{\begin{align}
   \mbox{then } \vec{N} &= 0.
\end{align}}\\
%\hline
%\pagebreak
\hline
&\\
&Therefore we can assume at least one vector is non-zero.\\
Assuming $\myvec{b\\d}$ as&Therefore
$\vec{N}=\myvec{a&0\\c&0}$\\
the zero vector&\\
&\parbox{6cm}{\begin{align}
\mbox{So }\vec{N}^2&=0\\
\implies a^2&=0\\
\therefore a&=0\\
\mbox{Thus }\vec{N}&=\myvec{a&0\\c&0}
\end{align}}\\
&In this case $\vec{N}$ is similar to $\vec{N}=\myvec{0&0\\1&0}$ via the matrix $\vec{P}=\myvec{c&0\\0&1}$\\
&\\
\hline
&\\
Assuming $\myvec{a\\c}$ as&Therefore
$\vec{N}=\myvec{0&b\\0&d}$\\
the zero vector&\\
&\parbox{6cm}{\begin{align}
\mbox{Then }\vec{N}^2&=0\\
\implies d^2&=0\\
\therefore d&=0\\
\mbox{Thus }\vec{N}&=\myvec{0&b\\0&0}
\end{align}}\\
&In this case $\vec{N}$ is similar to $\vec{N}=\myvec{0&0\\b&0}$ via the matrix $\vec{P}=\myvec{0&1\\1&0}$,\\
&which is similar to $\myvec{0&0\\1&0}$ as above.\\
&\\
\hline
&\\
Hence& we can assume neither $\myvec{a\\c}$ or $\myvec{b\\d}$ is the zero vector.\\
&\\
%\hline
%\pagebreak
\hline
&\\
Consequences of &Since they are linearly dependent we can assume,\\
linear&\\
independence&\\
&\parbox{6cm}{\begin{align}
\myvec{b\\d}&=x\myvec{a\\c}\\
\therefore \vec{N}&=\myvec{a&ax\\c&cx}\\
\therefore \vec{N}^2&=0\\
\implies a(a+cx)&=0\\
c(a+cx)&=0\\
ax(a+cx)&=0\\
cx(a+cx)&=0
\end{align}}\\
\hline
&\\
Proof that $\vec{N}$ is&We know that at least one of a or c is not zero.\\
similar over $\mathbb{C}$ to&If a = 0 then c $\neq$ 0, it must be that x = 0.\\
$\myvec{0&0\\1&0}$&So in this case $\vec{N}=\myvec{0&0\\c&0}$ which is similar to $\myvec{0&0\\1&0}$ as before.\\
&\\
&\parbox{6cm}{\begin{align}
\mbox{If } a &\neq0\\
\mbox{then }x &\neq0\\
\mbox{else }a(a+cx)&=0\\
\implies a&=0\\
\mbox{Thus }a+cx&=0\\
\mbox{Hence }\vec{N}&=\myvec{a&ax\\ \frac{-a}{x}&-a}
\end{align}}\\
 &This is similar to $\myvec{a&a\\-a&-a}$ via $\vec{P}=\myvec{\sqrt{x}&0\\0&\frac{1}{\sqrt{x}}}$.\\
 &And $\myvec{a&a\\-a&-a}$ is similar to $\myvec{0&0\\-a&0}$ via $\vec{P}=\myvec{-1&-1\\1&0}$\\
 &And this finally is similar to $\myvec{0&0\\1&0}$ as before.\\
 &\\
\hline
&\\
Conclusion &Thus either $\vec{N}$ = 0 or $\vec{N}$ is similar over $\mathbb{C}$ to $\myvec{0&0\\1&0}$.\\
&\\
\hline
\caption{Solution summary}
\label{eq:solutions/6/2/11/table:1}
\end{longtable}


\item Let \textbf{V} be the vector space of polynomials of degree at most 3 in a variable x with coefficients in $\mathbb{R}$. Let \textbf{T}=d/dx be the linear transformation of \textbf{V} to itself given by differentiation.\\

Which of the following are correct?\\
\begin{enumerate}
\item $\vec{T}$ is invertible
\item 0 is an eigenvalue of $\vec{T}$
\item There is a basis with respect to which the matrix of \textbf{T} is nilpotent.
\item The matrix of \textbf{T} with respect to the basis \myvec{1,1+x,1+x+x^2,1+x+x^2+x^3} is diagonal.
\end{enumerate}
\solution
See Table \ref{eq:solutions/6/2/11/table:1}
\onecolumn
\begin{longtable}{|l|l|}
\hline
\multirow{3}{*}{} & \\
Statement&Solution\\
\hline
&\parbox{6cm}{\begin{align}
\mbox{Let }\vec{N}&=\myvec{a&b\\c&d}\\
\mbox{Since }\vec{N}^2&=0
\end{align}}\\
&If $\myvec{a\\c},\myvec{b\\d}$ are linearly independent then $\vec{N}$ is diagonalizable to $\myvec{0&0\\0&0}$.\\
&\parbox{6cm}{\begin{align}
\mbox{If }\vec{P}\vec{N}\vec{P}^{-1}=0\\
\mbox{then }\vec{N}=\vec{P}^{-1}\vec{0}\vec{P}=0
\end{align}}\\
Proof that&So in this case $\vec{N}$ itself is the zero matrix.\\
$\vec{N}=0$&This contradicts the assumption that $\myvec{a\\c},\myvec{b\\d}$ are linearly independent.\\
&$\therefore$ we can assume that $\myvec{a\\c},\myvec{b\\d}$ are linearly dependent if both are\\
&equal to the zero vector\\
&\parbox{6cm}
{\begin{align}
   \mbox{then } \vec{N} &= 0.
\end{align}}\\
%\hline
%\pagebreak
\hline
&\\
&Therefore we can assume at least one vector is non-zero.\\
Assuming $\myvec{b\\d}$ as&Therefore
$\vec{N}=\myvec{a&0\\c&0}$\\
the zero vector&\\
&\parbox{6cm}{\begin{align}
\mbox{So }\vec{N}^2&=0\\
\implies a^2&=0\\
\therefore a&=0\\
\mbox{Thus }\vec{N}&=\myvec{a&0\\c&0}
\end{align}}\\
&In this case $\vec{N}$ is similar to $\vec{N}=\myvec{0&0\\1&0}$ via the matrix $\vec{P}=\myvec{c&0\\0&1}$\\
&\\
\hline
&\\
Assuming $\myvec{a\\c}$ as&Therefore
$\vec{N}=\myvec{0&b\\0&d}$\\
the zero vector&\\
&\parbox{6cm}{\begin{align}
\mbox{Then }\vec{N}^2&=0\\
\implies d^2&=0\\
\therefore d&=0\\
\mbox{Thus }\vec{N}&=\myvec{0&b\\0&0}
\end{align}}\\
&In this case $\vec{N}$ is similar to $\vec{N}=\myvec{0&0\\b&0}$ via the matrix $\vec{P}=\myvec{0&1\\1&0}$,\\
&which is similar to $\myvec{0&0\\1&0}$ as above.\\
&\\
\hline
&\\
Hence& we can assume neither $\myvec{a\\c}$ or $\myvec{b\\d}$ is the zero vector.\\
&\\
%\hline
%\pagebreak
\hline
&\\
Consequences of &Since they are linearly dependent we can assume,\\
linear&\\
independence&\\
&\parbox{6cm}{\begin{align}
\myvec{b\\d}&=x\myvec{a\\c}\\
\therefore \vec{N}&=\myvec{a&ax\\c&cx}\\
\therefore \vec{N}^2&=0\\
\implies a(a+cx)&=0\\
c(a+cx)&=0\\
ax(a+cx)&=0\\
cx(a+cx)&=0
\end{align}}\\
\hline
&\\
Proof that $\vec{N}$ is&We know that at least one of a or c is not zero.\\
similar over $\mathbb{C}$ to&If a = 0 then c $\neq$ 0, it must be that x = 0.\\
$\myvec{0&0\\1&0}$&So in this case $\vec{N}=\myvec{0&0\\c&0}$ which is similar to $\myvec{0&0\\1&0}$ as before.\\
&\\
&\parbox{6cm}{\begin{align}
\mbox{If } a &\neq0\\
\mbox{then }x &\neq0\\
\mbox{else }a(a+cx)&=0\\
\implies a&=0\\
\mbox{Thus }a+cx&=0\\
\mbox{Hence }\vec{N}&=\myvec{a&ax\\ \frac{-a}{x}&-a}
\end{align}}\\
 &This is similar to $\myvec{a&a\\-a&-a}$ via $\vec{P}=\myvec{\sqrt{x}&0\\0&\frac{1}{\sqrt{x}}}$.\\
 &And $\myvec{a&a\\-a&-a}$ is similar to $\myvec{0&0\\-a&0}$ via $\vec{P}=\myvec{-1&-1\\1&0}$\\
 &And this finally is similar to $\myvec{0&0\\1&0}$ as before.\\
 &\\
\hline
&\\
Conclusion &Thus either $\vec{N}$ = 0 or $\vec{N}$ is similar over $\mathbb{C}$ to $\myvec{0&0\\1&0}$.\\
&\\
\hline
\caption{Solution summary}
\label{eq:solutions/6/2/11/table:1}
\end{longtable}

\item For any $n\times n$ matrix $B$, let $N(B)=\{X\in \mathbb{R}^n:BX=0\}$ be the null space of $B$. Let $A$ be a $4\times 4$ matrix with $dim(N(A-4I))=2, dim(N(A-2I))=1$ and $rank(A)=3$
Which of the following are true?
\begin{enumerate}
\item 0,2 and 4 are eigenvalues of A
\item determinant(A)=0
\item A is not diagonalizable
\item trace(A)=8
\end{enumerate}
%
\solution
See Table \ref{eq:solutions/6/2/11/table:1}
\onecolumn
\begin{longtable}{|l|l|}
\hline
\multirow{3}{*}{} & \\
Statement&Solution\\
\hline
&\parbox{6cm}{\begin{align}
\mbox{Let }\vec{N}&=\myvec{a&b\\c&d}\\
\mbox{Since }\vec{N}^2&=0
\end{align}}\\
&If $\myvec{a\\c},\myvec{b\\d}$ are linearly independent then $\vec{N}$ is diagonalizable to $\myvec{0&0\\0&0}$.\\
&\parbox{6cm}{\begin{align}
\mbox{If }\vec{P}\vec{N}\vec{P}^{-1}=0\\
\mbox{then }\vec{N}=\vec{P}^{-1}\vec{0}\vec{P}=0
\end{align}}\\
Proof that&So in this case $\vec{N}$ itself is the zero matrix.\\
$\vec{N}=0$&This contradicts the assumption that $\myvec{a\\c},\myvec{b\\d}$ are linearly independent.\\
&$\therefore$ we can assume that $\myvec{a\\c},\myvec{b\\d}$ are linearly dependent if both are\\
&equal to the zero vector\\
&\parbox{6cm}
{\begin{align}
   \mbox{then } \vec{N} &= 0.
\end{align}}\\
%\hline
%\pagebreak
\hline
&\\
&Therefore we can assume at least one vector is non-zero.\\
Assuming $\myvec{b\\d}$ as&Therefore
$\vec{N}=\myvec{a&0\\c&0}$\\
the zero vector&\\
&\parbox{6cm}{\begin{align}
\mbox{So }\vec{N}^2&=0\\
\implies a^2&=0\\
\therefore a&=0\\
\mbox{Thus }\vec{N}&=\myvec{a&0\\c&0}
\end{align}}\\
&In this case $\vec{N}$ is similar to $\vec{N}=\myvec{0&0\\1&0}$ via the matrix $\vec{P}=\myvec{c&0\\0&1}$\\
&\\
\hline
&\\
Assuming $\myvec{a\\c}$ as&Therefore
$\vec{N}=\myvec{0&b\\0&d}$\\
the zero vector&\\
&\parbox{6cm}{\begin{align}
\mbox{Then }\vec{N}^2&=0\\
\implies d^2&=0\\
\therefore d&=0\\
\mbox{Thus }\vec{N}&=\myvec{0&b\\0&0}
\end{align}}\\
&In this case $\vec{N}$ is similar to $\vec{N}=\myvec{0&0\\b&0}$ via the matrix $\vec{P}=\myvec{0&1\\1&0}$,\\
&which is similar to $\myvec{0&0\\1&0}$ as above.\\
&\\
\hline
&\\
Hence& we can assume neither $\myvec{a\\c}$ or $\myvec{b\\d}$ is the zero vector.\\
&\\
%\hline
%\pagebreak
\hline
&\\
Consequences of &Since they are linearly dependent we can assume,\\
linear&\\
independence&\\
&\parbox{6cm}{\begin{align}
\myvec{b\\d}&=x\myvec{a\\c}\\
\therefore \vec{N}&=\myvec{a&ax\\c&cx}\\
\therefore \vec{N}^2&=0\\
\implies a(a+cx)&=0\\
c(a+cx)&=0\\
ax(a+cx)&=0\\
cx(a+cx)&=0
\end{align}}\\
\hline
&\\
Proof that $\vec{N}$ is&We know that at least one of a or c is not zero.\\
similar over $\mathbb{C}$ to&If a = 0 then c $\neq$ 0, it must be that x = 0.\\
$\myvec{0&0\\1&0}$&So in this case $\vec{N}=\myvec{0&0\\c&0}$ which is similar to $\myvec{0&0\\1&0}$ as before.\\
&\\
&\parbox{6cm}{\begin{align}
\mbox{If } a &\neq0\\
\mbox{then }x &\neq0\\
\mbox{else }a(a+cx)&=0\\
\implies a&=0\\
\mbox{Thus }a+cx&=0\\
\mbox{Hence }\vec{N}&=\myvec{a&ax\\ \frac{-a}{x}&-a}
\end{align}}\\
 &This is similar to $\myvec{a&a\\-a&-a}$ via $\vec{P}=\myvec{\sqrt{x}&0\\0&\frac{1}{\sqrt{x}}}$.\\
 &And $\myvec{a&a\\-a&-a}$ is similar to $\myvec{0&0\\-a&0}$ via $\vec{P}=\myvec{-1&-1\\1&0}$\\
 &And this finally is similar to $\myvec{0&0\\1&0}$ as before.\\
 &\\
\hline
&\\
Conclusion &Thus either $\vec{N}$ = 0 or $\vec{N}$ is similar over $\mathbb{C}$ to $\myvec{0&0\\1&0}$.\\
&\\
\hline
\caption{Solution summary}
\label{eq:solutions/6/2/11/table:1}
\end{longtable}

\item Which of the following 3x3 matrices are diagonizable over $\mathbb{R}?$\\
\begin{enumerate}
    \item \myvec{1&2&3\\0&4&5\\0&0&6}
    \item \myvec{0&1&0\\-1&0&0\\0&0&1}
    \item \myvec{1&2&3\\2&1&4\\3&4&1}
    \item \myvec{0&1&2\\0&0&1\\0&0&0}
\end{enumerate}
%
\solution
See Table \ref{eq:solutions/6/2/11/table:1}
\onecolumn
\begin{longtable}{|l|l|}
\hline
\multirow{3}{*}{} & \\
Statement&Solution\\
\hline
&\parbox{6cm}{\begin{align}
\mbox{Let }\vec{N}&=\myvec{a&b\\c&d}\\
\mbox{Since }\vec{N}^2&=0
\end{align}}\\
&If $\myvec{a\\c},\myvec{b\\d}$ are linearly independent then $\vec{N}$ is diagonalizable to $\myvec{0&0\\0&0}$.\\
&\parbox{6cm}{\begin{align}
\mbox{If }\vec{P}\vec{N}\vec{P}^{-1}=0\\
\mbox{then }\vec{N}=\vec{P}^{-1}\vec{0}\vec{P}=0
\end{align}}\\
Proof that&So in this case $\vec{N}$ itself is the zero matrix.\\
$\vec{N}=0$&This contradicts the assumption that $\myvec{a\\c},\myvec{b\\d}$ are linearly independent.\\
&$\therefore$ we can assume that $\myvec{a\\c},\myvec{b\\d}$ are linearly dependent if both are\\
&equal to the zero vector\\
&\parbox{6cm}
{\begin{align}
   \mbox{then } \vec{N} &= 0.
\end{align}}\\
%\hline
%\pagebreak
\hline
&\\
&Therefore we can assume at least one vector is non-zero.\\
Assuming $\myvec{b\\d}$ as&Therefore
$\vec{N}=\myvec{a&0\\c&0}$\\
the zero vector&\\
&\parbox{6cm}{\begin{align}
\mbox{So }\vec{N}^2&=0\\
\implies a^2&=0\\
\therefore a&=0\\
\mbox{Thus }\vec{N}&=\myvec{a&0\\c&0}
\end{align}}\\
&In this case $\vec{N}$ is similar to $\vec{N}=\myvec{0&0\\1&0}$ via the matrix $\vec{P}=\myvec{c&0\\0&1}$\\
&\\
\hline
&\\
Assuming $\myvec{a\\c}$ as&Therefore
$\vec{N}=\myvec{0&b\\0&d}$\\
the zero vector&\\
&\parbox{6cm}{\begin{align}
\mbox{Then }\vec{N}^2&=0\\
\implies d^2&=0\\
\therefore d&=0\\
\mbox{Thus }\vec{N}&=\myvec{0&b\\0&0}
\end{align}}\\
&In this case $\vec{N}$ is similar to $\vec{N}=\myvec{0&0\\b&0}$ via the matrix $\vec{P}=\myvec{0&1\\1&0}$,\\
&which is similar to $\myvec{0&0\\1&0}$ as above.\\
&\\
\hline
&\\
Hence& we can assume neither $\myvec{a\\c}$ or $\myvec{b\\d}$ is the zero vector.\\
&\\
%\hline
%\pagebreak
\hline
&\\
Consequences of &Since they are linearly dependent we can assume,\\
linear&\\
independence&\\
&\parbox{6cm}{\begin{align}
\myvec{b\\d}&=x\myvec{a\\c}\\
\therefore \vec{N}&=\myvec{a&ax\\c&cx}\\
\therefore \vec{N}^2&=0\\
\implies a(a+cx)&=0\\
c(a+cx)&=0\\
ax(a+cx)&=0\\
cx(a+cx)&=0
\end{align}}\\
\hline
&\\
Proof that $\vec{N}$ is&We know that at least one of a or c is not zero.\\
similar over $\mathbb{C}$ to&If a = 0 then c $\neq$ 0, it must be that x = 0.\\
$\myvec{0&0\\1&0}$&So in this case $\vec{N}=\myvec{0&0\\c&0}$ which is similar to $\myvec{0&0\\1&0}$ as before.\\
&\\
&\parbox{6cm}{\begin{align}
\mbox{If } a &\neq0\\
\mbox{then }x &\neq0\\
\mbox{else }a(a+cx)&=0\\
\implies a&=0\\
\mbox{Thus }a+cx&=0\\
\mbox{Hence }\vec{N}&=\myvec{a&ax\\ \frac{-a}{x}&-a}
\end{align}}\\
 &This is similar to $\myvec{a&a\\-a&-a}$ via $\vec{P}=\myvec{\sqrt{x}&0\\0&\frac{1}{\sqrt{x}}}$.\\
 &And $\myvec{a&a\\-a&-a}$ is similar to $\myvec{0&0\\-a&0}$ via $\vec{P}=\myvec{-1&-1\\1&0}$\\
 &And this finally is similar to $\myvec{0&0\\1&0}$ as before.\\
 &\\
\hline
&\\
Conclusion &Thus either $\vec{N}$ = 0 or $\vec{N}$ is similar over $\mathbb{C}$ to $\myvec{0&0\\1&0}$.\\
&\\
\hline
\caption{Solution summary}
\label{eq:solutions/6/2/11/table:1}
\end{longtable}

\twocolumn
\item Let $\vec{A} = \myvec{3 & 1 & 2 \\ 1 & 2 & 3 \\ 2 & 3 & 1  }$ and $\vec{Q(X) = X^TAX}$ for $\vec{X} \in \mathbb{R}^{3}$. Then
\begin{enumerate}
	\item $\vec{A}$ has exactly two positive eigen values.
	\item all the eigen values of $\vec{A}$ are positive.
	\item $\vec{Q(X)} \geq 0 $ $\forall$ $\vec{X}$ $\in$ $\mathbb{R}^3$
	\item $\vec{Q(X)} < 0 $ for some $\vec{X}$ $\in$ $\mathbb{R}^3$
\end{enumerate}
%
%
\solution
See Table \ref{eq:solutions/6/2/11/table:1}
\onecolumn
\begin{longtable}{|l|l|}
\hline
\multirow{3}{*}{} & \\
Statement&Solution\\
\hline
&\parbox{6cm}{\begin{align}
\mbox{Let }\vec{N}&=\myvec{a&b\\c&d}\\
\mbox{Since }\vec{N}^2&=0
\end{align}}\\
&If $\myvec{a\\c},\myvec{b\\d}$ are linearly independent then $\vec{N}$ is diagonalizable to $\myvec{0&0\\0&0}$.\\
&\parbox{6cm}{\begin{align}
\mbox{If }\vec{P}\vec{N}\vec{P}^{-1}=0\\
\mbox{then }\vec{N}=\vec{P}^{-1}\vec{0}\vec{P}=0
\end{align}}\\
Proof that&So in this case $\vec{N}$ itself is the zero matrix.\\
$\vec{N}=0$&This contradicts the assumption that $\myvec{a\\c},\myvec{b\\d}$ are linearly independent.\\
&$\therefore$ we can assume that $\myvec{a\\c},\myvec{b\\d}$ are linearly dependent if both are\\
&equal to the zero vector\\
&\parbox{6cm}
{\begin{align}
   \mbox{then } \vec{N} &= 0.
\end{align}}\\
%\hline
%\pagebreak
\hline
&\\
&Therefore we can assume at least one vector is non-zero.\\
Assuming $\myvec{b\\d}$ as&Therefore
$\vec{N}=\myvec{a&0\\c&0}$\\
the zero vector&\\
&\parbox{6cm}{\begin{align}
\mbox{So }\vec{N}^2&=0\\
\implies a^2&=0\\
\therefore a&=0\\
\mbox{Thus }\vec{N}&=\myvec{a&0\\c&0}
\end{align}}\\
&In this case $\vec{N}$ is similar to $\vec{N}=\myvec{0&0\\1&0}$ via the matrix $\vec{P}=\myvec{c&0\\0&1}$\\
&\\
\hline
&\\
Assuming $\myvec{a\\c}$ as&Therefore
$\vec{N}=\myvec{0&b\\0&d}$\\
the zero vector&\\
&\parbox{6cm}{\begin{align}
\mbox{Then }\vec{N}^2&=0\\
\implies d^2&=0\\
\therefore d&=0\\
\mbox{Thus }\vec{N}&=\myvec{0&b\\0&0}
\end{align}}\\
&In this case $\vec{N}$ is similar to $\vec{N}=\myvec{0&0\\b&0}$ via the matrix $\vec{P}=\myvec{0&1\\1&0}$,\\
&which is similar to $\myvec{0&0\\1&0}$ as above.\\
&\\
\hline
&\\
Hence& we can assume neither $\myvec{a\\c}$ or $\myvec{b\\d}$ is the zero vector.\\
&\\
%\hline
%\pagebreak
\hline
&\\
Consequences of &Since they are linearly dependent we can assume,\\
linear&\\
independence&\\
&\parbox{6cm}{\begin{align}
\myvec{b\\d}&=x\myvec{a\\c}\\
\therefore \vec{N}&=\myvec{a&ax\\c&cx}\\
\therefore \vec{N}^2&=0\\
\implies a(a+cx)&=0\\
c(a+cx)&=0\\
ax(a+cx)&=0\\
cx(a+cx)&=0
\end{align}}\\
\hline
&\\
Proof that $\vec{N}$ is&We know that at least one of a or c is not zero.\\
similar over $\mathbb{C}$ to&If a = 0 then c $\neq$ 0, it must be that x = 0.\\
$\myvec{0&0\\1&0}$&So in this case $\vec{N}=\myvec{0&0\\c&0}$ which is similar to $\myvec{0&0\\1&0}$ as before.\\
&\\
&\parbox{6cm}{\begin{align}
\mbox{If } a &\neq0\\
\mbox{then }x &\neq0\\
\mbox{else }a(a+cx)&=0\\
\implies a&=0\\
\mbox{Thus }a+cx&=0\\
\mbox{Hence }\vec{N}&=\myvec{a&ax\\ \frac{-a}{x}&-a}
\end{align}}\\
 &This is similar to $\myvec{a&a\\-a&-a}$ via $\vec{P}=\myvec{\sqrt{x}&0\\0&\frac{1}{\sqrt{x}}}$.\\
 &And $\myvec{a&a\\-a&-a}$ is similar to $\myvec{0&0\\-a&0}$ via $\vec{P}=\myvec{-1&-1\\1&0}$\\
 &And this finally is similar to $\myvec{0&0\\1&0}$ as before.\\
 &\\
\hline
&\\
Conclusion &Thus either $\vec{N}$ = 0 or $\vec{N}$ is similar over $\mathbb{C}$ to $\myvec{0&0\\1&0}$.\\
&\\
\hline
\caption{Solution summary}
\label{eq:solutions/6/2/11/table:1}
\end{longtable}

\item Consider the matrix
\begin{align}
A(x) = \myvec{1+x^2&7&11\\3x&2x&4\\8x&17&13} & ;x\in \vec{R}.
\end{align}
Then,
\begin{enumerate}
\item A(x) has eigenvalue 0 for some $x\in \vec{R}$.
\item 0 is not an eigenvalue of A(x) for any $x\in \vec{R}$.
\item A(x) has eigenvalue 0 $\forall x\in \vec{R}$.
\item A(x) is invertible $\forall x\in \vec{R}$.
\end{enumerate}
%
\solution
See Table \ref{eq:solutions/6/2/11/table:1}
\onecolumn
\begin{longtable}{|l|l|}
\hline
\multirow{3}{*}{} & \\
Statement&Solution\\
\hline
&\parbox{6cm}{\begin{align}
\mbox{Let }\vec{N}&=\myvec{a&b\\c&d}\\
\mbox{Since }\vec{N}^2&=0
\end{align}}\\
&If $\myvec{a\\c},\myvec{b\\d}$ are linearly independent then $\vec{N}$ is diagonalizable to $\myvec{0&0\\0&0}$.\\
&\parbox{6cm}{\begin{align}
\mbox{If }\vec{P}\vec{N}\vec{P}^{-1}=0\\
\mbox{then }\vec{N}=\vec{P}^{-1}\vec{0}\vec{P}=0
\end{align}}\\
Proof that&So in this case $\vec{N}$ itself is the zero matrix.\\
$\vec{N}=0$&This contradicts the assumption that $\myvec{a\\c},\myvec{b\\d}$ are linearly independent.\\
&$\therefore$ we can assume that $\myvec{a\\c},\myvec{b\\d}$ are linearly dependent if both are\\
&equal to the zero vector\\
&\parbox{6cm}
{\begin{align}
   \mbox{then } \vec{N} &= 0.
\end{align}}\\
%\hline
%\pagebreak
\hline
&\\
&Therefore we can assume at least one vector is non-zero.\\
Assuming $\myvec{b\\d}$ as&Therefore
$\vec{N}=\myvec{a&0\\c&0}$\\
the zero vector&\\
&\parbox{6cm}{\begin{align}
\mbox{So }\vec{N}^2&=0\\
\implies a^2&=0\\
\therefore a&=0\\
\mbox{Thus }\vec{N}&=\myvec{a&0\\c&0}
\end{align}}\\
&In this case $\vec{N}$ is similar to $\vec{N}=\myvec{0&0\\1&0}$ via the matrix $\vec{P}=\myvec{c&0\\0&1}$\\
&\\
\hline
&\\
Assuming $\myvec{a\\c}$ as&Therefore
$\vec{N}=\myvec{0&b\\0&d}$\\
the zero vector&\\
&\parbox{6cm}{\begin{align}
\mbox{Then }\vec{N}^2&=0\\
\implies d^2&=0\\
\therefore d&=0\\
\mbox{Thus }\vec{N}&=\myvec{0&b\\0&0}
\end{align}}\\
&In this case $\vec{N}$ is similar to $\vec{N}=\myvec{0&0\\b&0}$ via the matrix $\vec{P}=\myvec{0&1\\1&0}$,\\
&which is similar to $\myvec{0&0\\1&0}$ as above.\\
&\\
\hline
&\\
Hence& we can assume neither $\myvec{a\\c}$ or $\myvec{b\\d}$ is the zero vector.\\
&\\
%\hline
%\pagebreak
\hline
&\\
Consequences of &Since they are linearly dependent we can assume,\\
linear&\\
independence&\\
&\parbox{6cm}{\begin{align}
\myvec{b\\d}&=x\myvec{a\\c}\\
\therefore \vec{N}&=\myvec{a&ax\\c&cx}\\
\therefore \vec{N}^2&=0\\
\implies a(a+cx)&=0\\
c(a+cx)&=0\\
ax(a+cx)&=0\\
cx(a+cx)&=0
\end{align}}\\
\hline
&\\
Proof that $\vec{N}$ is&We know that at least one of a or c is not zero.\\
similar over $\mathbb{C}$ to&If a = 0 then c $\neq$ 0, it must be that x = 0.\\
$\myvec{0&0\\1&0}$&So in this case $\vec{N}=\myvec{0&0\\c&0}$ which is similar to $\myvec{0&0\\1&0}$ as before.\\
&\\
&\parbox{6cm}{\begin{align}
\mbox{If } a &\neq0\\
\mbox{then }x &\neq0\\
\mbox{else }a(a+cx)&=0\\
\implies a&=0\\
\mbox{Thus }a+cx&=0\\
\mbox{Hence }\vec{N}&=\myvec{a&ax\\ \frac{-a}{x}&-a}
\end{align}}\\
 &This is similar to $\myvec{a&a\\-a&-a}$ via $\vec{P}=\myvec{\sqrt{x}&0\\0&\frac{1}{\sqrt{x}}}$.\\
 &And $\myvec{a&a\\-a&-a}$ is similar to $\myvec{0&0\\-a&0}$ via $\vec{P}=\myvec{-1&-1\\1&0}$\\
 &And this finally is similar to $\myvec{0&0\\1&0}$ as before.\\
 &\\
\hline
&\\
Conclusion &Thus either $\vec{N}$ = 0 or $\vec{N}$ is similar over $\mathbb{C}$ to $\myvec{0&0\\1&0}$.\\
&\\
\hline
\caption{Solution summary}
\label{eq:solutions/6/2/11/table:1}
\end{longtable}

\end{enumerate}

\section{December 2016}
\renewcommand{\theequation}{\theenumi}
\renewcommand{\thefigure}{\theenumi}
\renewcommand{\thetable}{\theenumi}
\begin{enumerate}[label=\thesection.\arabic*.,ref=\thesection.\theenumi]
\numberwithin{equation}{enumi}
\numberwithin{figure}{enumi}
\numberwithin{table}{enumi}

\item %
Let $\vec{A}=\myvec{1 & 1\\1 & 0}$ and let $\alpha_n$ and $\beta_n$ denote the two eigenvalues of $\vec{A}^n$ such that $\abs{\alpha_n}\geq\abs{\beta_n}$.\\
Then
\begin{enumerate}
    \item $\alpha_n\rightarrow \infty$ as $n\rightarrow \infty$
    \item $\beta_n\rightarrow 0$ as $n\rightarrow \infty$
    \item $\beta_n$ is positive if n is even.
    \item $\beta_n$ is negative if n is odd.
\end{enumerate}
%
\solution
See Table \ref{eq:solutions/6/2/11/table:1}
\onecolumn
\begin{longtable}{|l|l|}
\hline
\multirow{3}{*}{} & \\
Statement&Solution\\
\hline
&\parbox{6cm}{\begin{align}
\mbox{Let }\vec{N}&=\myvec{a&b\\c&d}\\
\mbox{Since }\vec{N}^2&=0
\end{align}}\\
&If $\myvec{a\\c},\myvec{b\\d}$ are linearly independent then $\vec{N}$ is diagonalizable to $\myvec{0&0\\0&0}$.\\
&\parbox{6cm}{\begin{align}
\mbox{If }\vec{P}\vec{N}\vec{P}^{-1}=0\\
\mbox{then }\vec{N}=\vec{P}^{-1}\vec{0}\vec{P}=0
\end{align}}\\
Proof that&So in this case $\vec{N}$ itself is the zero matrix.\\
$\vec{N}=0$&This contradicts the assumption that $\myvec{a\\c},\myvec{b\\d}$ are linearly independent.\\
&$\therefore$ we can assume that $\myvec{a\\c},\myvec{b\\d}$ are linearly dependent if both are\\
&equal to the zero vector\\
&\parbox{6cm}
{\begin{align}
   \mbox{then } \vec{N} &= 0.
\end{align}}\\
%\hline
%\pagebreak
\hline
&\\
&Therefore we can assume at least one vector is non-zero.\\
Assuming $\myvec{b\\d}$ as&Therefore
$\vec{N}=\myvec{a&0\\c&0}$\\
the zero vector&\\
&\parbox{6cm}{\begin{align}
\mbox{So }\vec{N}^2&=0\\
\implies a^2&=0\\
\therefore a&=0\\
\mbox{Thus }\vec{N}&=\myvec{a&0\\c&0}
\end{align}}\\
&In this case $\vec{N}$ is similar to $\vec{N}=\myvec{0&0\\1&0}$ via the matrix $\vec{P}=\myvec{c&0\\0&1}$\\
&\\
\hline
&\\
Assuming $\myvec{a\\c}$ as&Therefore
$\vec{N}=\myvec{0&b\\0&d}$\\
the zero vector&\\
&\parbox{6cm}{\begin{align}
\mbox{Then }\vec{N}^2&=0\\
\implies d^2&=0\\
\therefore d&=0\\
\mbox{Thus }\vec{N}&=\myvec{0&b\\0&0}
\end{align}}\\
&In this case $\vec{N}$ is similar to $\vec{N}=\myvec{0&0\\b&0}$ via the matrix $\vec{P}=\myvec{0&1\\1&0}$,\\
&which is similar to $\myvec{0&0\\1&0}$ as above.\\
&\\
\hline
&\\
Hence& we can assume neither $\myvec{a\\c}$ or $\myvec{b\\d}$ is the zero vector.\\
&\\
%\hline
%\pagebreak
\hline
&\\
Consequences of &Since they are linearly dependent we can assume,\\
linear&\\
independence&\\
&\parbox{6cm}{\begin{align}
\myvec{b\\d}&=x\myvec{a\\c}\\
\therefore \vec{N}&=\myvec{a&ax\\c&cx}\\
\therefore \vec{N}^2&=0\\
\implies a(a+cx)&=0\\
c(a+cx)&=0\\
ax(a+cx)&=0\\
cx(a+cx)&=0
\end{align}}\\
\hline
&\\
Proof that $\vec{N}$ is&We know that at least one of a or c is not zero.\\
similar over $\mathbb{C}$ to&If a = 0 then c $\neq$ 0, it must be that x = 0.\\
$\myvec{0&0\\1&0}$&So in this case $\vec{N}=\myvec{0&0\\c&0}$ which is similar to $\myvec{0&0\\1&0}$ as before.\\
&\\
&\parbox{6cm}{\begin{align}
\mbox{If } a &\neq0\\
\mbox{then }x &\neq0\\
\mbox{else }a(a+cx)&=0\\
\implies a&=0\\
\mbox{Thus }a+cx&=0\\
\mbox{Hence }\vec{N}&=\myvec{a&ax\\ \frac{-a}{x}&-a}
\end{align}}\\
 &This is similar to $\myvec{a&a\\-a&-a}$ via $\vec{P}=\myvec{\sqrt{x}&0\\0&\frac{1}{\sqrt{x}}}$.\\
 &And $\myvec{a&a\\-a&-a}$ is similar to $\myvec{0&0\\-a&0}$ via $\vec{P}=\myvec{-1&-1\\1&0}$\\
 &And this finally is similar to $\myvec{0&0\\1&0}$ as before.\\
 &\\
\hline
&\\
Conclusion &Thus either $\vec{N}$ = 0 or $\vec{N}$ is similar over $\mathbb{C}$ to $\myvec{0&0\\1&0}$.\\
&\\
\hline
\caption{Solution summary}
\label{eq:solutions/6/2/11/table:1}
\end{longtable}

\item Let $M_n$ denote the vector space of all $n\times n$ real matrices. Which of the following is a linear subspaces of $M_n$ :-
\begin{enumerate}
\item $ V_1 = \{  A \in M_n : \text{ A is nonsingular} \}$
\item $ V_2 = \{  A \in M_n : det(A) = 0 \}$
\item $ V_3 = \{  A \in M_n : trace(A) = 0 \}$
\item $ V_4 = \{  BA : A \in M_n\},$ where $ B$ is some fixed matrix in $ M_n$
\end{enumerate}
%
\solution
See Table \ref{eq:solutions/6/2/11/table:1}
\onecolumn
\begin{longtable}{|l|l|}
\hline
\multirow{3}{*}{} & \\
Statement&Solution\\
\hline
&\parbox{6cm}{\begin{align}
\mbox{Let }\vec{N}&=\myvec{a&b\\c&d}\\
\mbox{Since }\vec{N}^2&=0
\end{align}}\\
&If $\myvec{a\\c},\myvec{b\\d}$ are linearly independent then $\vec{N}$ is diagonalizable to $\myvec{0&0\\0&0}$.\\
&\parbox{6cm}{\begin{align}
\mbox{If }\vec{P}\vec{N}\vec{P}^{-1}=0\\
\mbox{then }\vec{N}=\vec{P}^{-1}\vec{0}\vec{P}=0
\end{align}}\\
Proof that&So in this case $\vec{N}$ itself is the zero matrix.\\
$\vec{N}=0$&This contradicts the assumption that $\myvec{a\\c},\myvec{b\\d}$ are linearly independent.\\
&$\therefore$ we can assume that $\myvec{a\\c},\myvec{b\\d}$ are linearly dependent if both are\\
&equal to the zero vector\\
&\parbox{6cm}
{\begin{align}
   \mbox{then } \vec{N} &= 0.
\end{align}}\\
%\hline
%\pagebreak
\hline
&\\
&Therefore we can assume at least one vector is non-zero.\\
Assuming $\myvec{b\\d}$ as&Therefore
$\vec{N}=\myvec{a&0\\c&0}$\\
the zero vector&\\
&\parbox{6cm}{\begin{align}
\mbox{So }\vec{N}^2&=0\\
\implies a^2&=0\\
\therefore a&=0\\
\mbox{Thus }\vec{N}&=\myvec{a&0\\c&0}
\end{align}}\\
&In this case $\vec{N}$ is similar to $\vec{N}=\myvec{0&0\\1&0}$ via the matrix $\vec{P}=\myvec{c&0\\0&1}$\\
&\\
\hline
&\\
Assuming $\myvec{a\\c}$ as&Therefore
$\vec{N}=\myvec{0&b\\0&d}$\\
the zero vector&\\
&\parbox{6cm}{\begin{align}
\mbox{Then }\vec{N}^2&=0\\
\implies d^2&=0\\
\therefore d&=0\\
\mbox{Thus }\vec{N}&=\myvec{0&b\\0&0}
\end{align}}\\
&In this case $\vec{N}$ is similar to $\vec{N}=\myvec{0&0\\b&0}$ via the matrix $\vec{P}=\myvec{0&1\\1&0}$,\\
&which is similar to $\myvec{0&0\\1&0}$ as above.\\
&\\
\hline
&\\
Hence& we can assume neither $\myvec{a\\c}$ or $\myvec{b\\d}$ is the zero vector.\\
&\\
%\hline
%\pagebreak
\hline
&\\
Consequences of &Since they are linearly dependent we can assume,\\
linear&\\
independence&\\
&\parbox{6cm}{\begin{align}
\myvec{b\\d}&=x\myvec{a\\c}\\
\therefore \vec{N}&=\myvec{a&ax\\c&cx}\\
\therefore \vec{N}^2&=0\\
\implies a(a+cx)&=0\\
c(a+cx)&=0\\
ax(a+cx)&=0\\
cx(a+cx)&=0
\end{align}}\\
\hline
&\\
Proof that $\vec{N}$ is&We know that at least one of a or c is not zero.\\
similar over $\mathbb{C}$ to&If a = 0 then c $\neq$ 0, it must be that x = 0.\\
$\myvec{0&0\\1&0}$&So in this case $\vec{N}=\myvec{0&0\\c&0}$ which is similar to $\myvec{0&0\\1&0}$ as before.\\
&\\
&\parbox{6cm}{\begin{align}
\mbox{If } a &\neq0\\
\mbox{then }x &\neq0\\
\mbox{else }a(a+cx)&=0\\
\implies a&=0\\
\mbox{Thus }a+cx&=0\\
\mbox{Hence }\vec{N}&=\myvec{a&ax\\ \frac{-a}{x}&-a}
\end{align}}\\
 &This is similar to $\myvec{a&a\\-a&-a}$ via $\vec{P}=\myvec{\sqrt{x}&0\\0&\frac{1}{\sqrt{x}}}$.\\
 &And $\myvec{a&a\\-a&-a}$ is similar to $\myvec{0&0\\-a&0}$ via $\vec{P}=\myvec{-1&-1\\1&0}$\\
 &And this finally is similar to $\myvec{0&0\\1&0}$ as before.\\
 &\\
\hline
&\\
Conclusion &Thus either $\vec{N}$ = 0 or $\vec{N}$ is similar over $\mathbb{C}$ to $\myvec{0&0\\1&0}$.\\
&\\
\hline
\caption{Solution summary}
\label{eq:solutions/6/2/11/table:1}
\end{longtable}


\item If $\vec{P}$ and $\vec{Q}$ are invertible matrices such that   
$\vec{P}\vec{Q} = -\vec{Q}\vec{P}$,then we can conclude that
\begin{enumerate}
\item  $Tr(\vec{P})=Tr(\vec{Q})=0$ \label{eq:solutions/2016/dec/75/1}
\item  $Tr(\vec{P})=Tr(\vec{Q})=1$ \label{eq:solutions/2016/dec/75/2}
\item  $Tr(\vec{P})=-Tr(\vec{Q})$ \label{eq:solutions/2016/dec/75/3}
\item  $Tr(\vec{P}) \neq Tr(\vec{Q})$ \label{eq:solutions/2016/dec/75/4}
\end{enumerate}
%
%
\solution
See Table \ref{eq:solutions/6/2/11/table:1}
\onecolumn
\begin{longtable}{|l|l|}
\hline
\multirow{3}{*}{} & \\
Statement&Solution\\
\hline
&\parbox{6cm}{\begin{align}
\mbox{Let }\vec{N}&=\myvec{a&b\\c&d}\\
\mbox{Since }\vec{N}^2&=0
\end{align}}\\
&If $\myvec{a\\c},\myvec{b\\d}$ are linearly independent then $\vec{N}$ is diagonalizable to $\myvec{0&0\\0&0}$.\\
&\parbox{6cm}{\begin{align}
\mbox{If }\vec{P}\vec{N}\vec{P}^{-1}=0\\
\mbox{then }\vec{N}=\vec{P}^{-1}\vec{0}\vec{P}=0
\end{align}}\\
Proof that&So in this case $\vec{N}$ itself is the zero matrix.\\
$\vec{N}=0$&This contradicts the assumption that $\myvec{a\\c},\myvec{b\\d}$ are linearly independent.\\
&$\therefore$ we can assume that $\myvec{a\\c},\myvec{b\\d}$ are linearly dependent if both are\\
&equal to the zero vector\\
&\parbox{6cm}
{\begin{align}
   \mbox{then } \vec{N} &= 0.
\end{align}}\\
%\hline
%\pagebreak
\hline
&\\
&Therefore we can assume at least one vector is non-zero.\\
Assuming $\myvec{b\\d}$ as&Therefore
$\vec{N}=\myvec{a&0\\c&0}$\\
the zero vector&\\
&\parbox{6cm}{\begin{align}
\mbox{So }\vec{N}^2&=0\\
\implies a^2&=0\\
\therefore a&=0\\
\mbox{Thus }\vec{N}&=\myvec{a&0\\c&0}
\end{align}}\\
&In this case $\vec{N}$ is similar to $\vec{N}=\myvec{0&0\\1&0}$ via the matrix $\vec{P}=\myvec{c&0\\0&1}$\\
&\\
\hline
&\\
Assuming $\myvec{a\\c}$ as&Therefore
$\vec{N}=\myvec{0&b\\0&d}$\\
the zero vector&\\
&\parbox{6cm}{\begin{align}
\mbox{Then }\vec{N}^2&=0\\
\implies d^2&=0\\
\therefore d&=0\\
\mbox{Thus }\vec{N}&=\myvec{0&b\\0&0}
\end{align}}\\
&In this case $\vec{N}$ is similar to $\vec{N}=\myvec{0&0\\b&0}$ via the matrix $\vec{P}=\myvec{0&1\\1&0}$,\\
&which is similar to $\myvec{0&0\\1&0}$ as above.\\
&\\
\hline
&\\
Hence& we can assume neither $\myvec{a\\c}$ or $\myvec{b\\d}$ is the zero vector.\\
&\\
%\hline
%\pagebreak
\hline
&\\
Consequences of &Since they are linearly dependent we can assume,\\
linear&\\
independence&\\
&\parbox{6cm}{\begin{align}
\myvec{b\\d}&=x\myvec{a\\c}\\
\therefore \vec{N}&=\myvec{a&ax\\c&cx}\\
\therefore \vec{N}^2&=0\\
\implies a(a+cx)&=0\\
c(a+cx)&=0\\
ax(a+cx)&=0\\
cx(a+cx)&=0
\end{align}}\\
\hline
&\\
Proof that $\vec{N}$ is&We know that at least one of a or c is not zero.\\
similar over $\mathbb{C}$ to&If a = 0 then c $\neq$ 0, it must be that x = 0.\\
$\myvec{0&0\\1&0}$&So in this case $\vec{N}=\myvec{0&0\\c&0}$ which is similar to $\myvec{0&0\\1&0}$ as before.\\
&\\
&\parbox{6cm}{\begin{align}
\mbox{If } a &\neq0\\
\mbox{then }x &\neq0\\
\mbox{else }a(a+cx)&=0\\
\implies a&=0\\
\mbox{Thus }a+cx&=0\\
\mbox{Hence }\vec{N}&=\myvec{a&ax\\ \frac{-a}{x}&-a}
\end{align}}\\
 &This is similar to $\myvec{a&a\\-a&-a}$ via $\vec{P}=\myvec{\sqrt{x}&0\\0&\frac{1}{\sqrt{x}}}$.\\
 &And $\myvec{a&a\\-a&-a}$ is similar to $\myvec{0&0\\-a&0}$ via $\vec{P}=\myvec{-1&-1\\1&0}$\\
 &And this finally is similar to $\myvec{0&0\\1&0}$ as before.\\
 &\\
\hline
&\\
Conclusion &Thus either $\vec{N}$ = 0 or $\vec{N}$ is similar over $\mathbb{C}$ to $\myvec{0&0\\1&0}$.\\
&\\
\hline
\caption{Solution summary}
\label{eq:solutions/6/2/11/table:1}
\end{longtable}

\twocolumn
\item Let $\vec{W_1}$, $\vec{W_2}$, $\vec{W_3}$ be 3 distinct subspaces of $\vec{R}^{10}$ such that each $\vec{W_i}$ has dimension of 9. Let $\vec{W} = \vec{W_1} \cap \vec{W_2} \cap \vec{W_3}$. Then we can conclude that\\
\begin{enumerate}
\item $\vec{W}$ may not be a subspace of $\vec{R}^{10}$\\
\item dim $\vec{W} \leq 8$\\
\item dim $\vec{W} \geq 7$\\
\item  dim $\vec{W} \leq 3$\\
\end{enumerate}
 %
\solution
See Table \ref{eq:solutions/6/2/11/table:1}
\onecolumn
\begin{longtable}{|l|l|}
\hline
\multirow{3}{*}{} & \\
Statement&Solution\\
\hline
&\parbox{6cm}{\begin{align}
\mbox{Let }\vec{N}&=\myvec{a&b\\c&d}\\
\mbox{Since }\vec{N}^2&=0
\end{align}}\\
&If $\myvec{a\\c},\myvec{b\\d}$ are linearly independent then $\vec{N}$ is diagonalizable to $\myvec{0&0\\0&0}$.\\
&\parbox{6cm}{\begin{align}
\mbox{If }\vec{P}\vec{N}\vec{P}^{-1}=0\\
\mbox{then }\vec{N}=\vec{P}^{-1}\vec{0}\vec{P}=0
\end{align}}\\
Proof that&So in this case $\vec{N}$ itself is the zero matrix.\\
$\vec{N}=0$&This contradicts the assumption that $\myvec{a\\c},\myvec{b\\d}$ are linearly independent.\\
&$\therefore$ we can assume that $\myvec{a\\c},\myvec{b\\d}$ are linearly dependent if both are\\
&equal to the zero vector\\
&\parbox{6cm}
{\begin{align}
   \mbox{then } \vec{N} &= 0.
\end{align}}\\
%\hline
%\pagebreak
\hline
&\\
&Therefore we can assume at least one vector is non-zero.\\
Assuming $\myvec{b\\d}$ as&Therefore
$\vec{N}=\myvec{a&0\\c&0}$\\
the zero vector&\\
&\parbox{6cm}{\begin{align}
\mbox{So }\vec{N}^2&=0\\
\implies a^2&=0\\
\therefore a&=0\\
\mbox{Thus }\vec{N}&=\myvec{a&0\\c&0}
\end{align}}\\
&In this case $\vec{N}$ is similar to $\vec{N}=\myvec{0&0\\1&0}$ via the matrix $\vec{P}=\myvec{c&0\\0&1}$\\
&\\
\hline
&\\
Assuming $\myvec{a\\c}$ as&Therefore
$\vec{N}=\myvec{0&b\\0&d}$\\
the zero vector&\\
&\parbox{6cm}{\begin{align}
\mbox{Then }\vec{N}^2&=0\\
\implies d^2&=0\\
\therefore d&=0\\
\mbox{Thus }\vec{N}&=\myvec{0&b\\0&0}
\end{align}}\\
&In this case $\vec{N}$ is similar to $\vec{N}=\myvec{0&0\\b&0}$ via the matrix $\vec{P}=\myvec{0&1\\1&0}$,\\
&which is similar to $\myvec{0&0\\1&0}$ as above.\\
&\\
\hline
&\\
Hence& we can assume neither $\myvec{a\\c}$ or $\myvec{b\\d}$ is the zero vector.\\
&\\
%\hline
%\pagebreak
\hline
&\\
Consequences of &Since they are linearly dependent we can assume,\\
linear&\\
independence&\\
&\parbox{6cm}{\begin{align}
\myvec{b\\d}&=x\myvec{a\\c}\\
\therefore \vec{N}&=\myvec{a&ax\\c&cx}\\
\therefore \vec{N}^2&=0\\
\implies a(a+cx)&=0\\
c(a+cx)&=0\\
ax(a+cx)&=0\\
cx(a+cx)&=0
\end{align}}\\
\hline
&\\
Proof that $\vec{N}$ is&We know that at least one of a or c is not zero.\\
similar over $\mathbb{C}$ to&If a = 0 then c $\neq$ 0, it must be that x = 0.\\
$\myvec{0&0\\1&0}$&So in this case $\vec{N}=\myvec{0&0\\c&0}$ which is similar to $\myvec{0&0\\1&0}$ as before.\\
&\\
&\parbox{6cm}{\begin{align}
\mbox{If } a &\neq0\\
\mbox{then }x &\neq0\\
\mbox{else }a(a+cx)&=0\\
\implies a&=0\\
\mbox{Thus }a+cx&=0\\
\mbox{Hence }\vec{N}&=\myvec{a&ax\\ \frac{-a}{x}&-a}
\end{align}}\\
 &This is similar to $\myvec{a&a\\-a&-a}$ via $\vec{P}=\myvec{\sqrt{x}&0\\0&\frac{1}{\sqrt{x}}}$.\\
 &And $\myvec{a&a\\-a&-a}$ is similar to $\myvec{0&0\\-a&0}$ via $\vec{P}=\myvec{-1&-1\\1&0}$\\
 &And this finally is similar to $\myvec{0&0\\1&0}$ as before.\\
 &\\
\hline
&\\
Conclusion &Thus either $\vec{N}$ = 0 or $\vec{N}$ is similar over $\mathbb{C}$ to $\myvec{0&0\\1&0}$.\\
&\\
\hline
\caption{Solution summary}
\label{eq:solutions/6/2/11/table:1}
\end{longtable}

\twocolumn


%\item Consider a Markov Chain with state space $\cbrak{0,1,2}$ and transition matrix
%\begin{align}
%P = 
%\begin{blockarray}{c@{\hspace{1pt}}rrr@{\hspace{3pt}}}
%         & 0   & 1   & 2 \\
%        \begin{block}{r@{\hspace{3pt}}@{\hspace{1pt}}
%    (@{\hspace{1pt}}rrr@{\hspace{1pt}}@{\hspace{1pt}})}
%        0 & \frac{1}{2} & \frac{1}{2} & 0  \\
%        1 & 0 &\frac{1}{2}  & \frac{3}{4}  \\
%%
%        2 &  \frac{1}{3} & \frac{1}{3} & \frac{1}{3}  \\
%        \end{block}
%    \end{blockarray}
%\end{align}
%For any two states $i$ and $j$, let $p_{ij}^{(n)}$ denote the $n$-step transition probability of going from $i$ to $j$.  Identify correct statements.
%\begin{enumerate}
%\item $\lim_{n \to \infty} p_{11}^{(n)} = \frac{2}{9}$
%\item $\lim_{n \to \infty} p_{21}^{(n)} = 0$
%\item $\lim_{n \to \infty} p_{32}^{(n)} = \frac{1}{3}$
%\item $\lim_{n \to \infty} p_{13}^{(n)} = \frac{1}{3}$
%\end{enumerate}

\end{enumerate}



\end{document}


