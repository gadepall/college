%
A matrix is real symmetric implies its eigen values are real and eigen vectors are orthogonal,that is its eigen value decomposition is
\begin{align}
 \vec{A}=\vec{P}\vec{D}\vec{P}^T
\end{align}
$\vec{D}$ is the diagonal matrix containing the real eigen values of $\vec{A}$\\
$\vec{P}$ has the corresponding eigen vectors
\begin{align}
    \vec{P}\vec{P}^T=\vec{P}^T\vec{P}=\vec{I}
\end{align}
A real matrix is positive definite if 
\begin{align}
    \vec{x}^T\vec{A}\vec{x}>0\\
    \implies  \vec{x}^T\lambda\vec{x}>0\\
    \implies \lambda \vec{x}^T\vec{x}>0\\
    \implies \lambda>0
\end{align}
In other words, all the eigen values of $A$ are positive
See Table \ref{eq:solutions/2017/dec/77/table:0}

Let $\vec{A}$ be
\begin{align}
    \vec{A}=\vec{P}\vec{D}\vec{P}^T\\
    \vec{D}=\myvec{\lambda_1&0&0&0\\0&\lambda_2&0&0\\0&0&\lambda_3&0\\0&0&0&\lambda_4}
\end{align}



\begin{table*}[!t]
    \centering
    \begin{tabular}{|l|l|}
    \hline

    \textbf{OPTIONS} & \textbf{DERIVATIONS}\\
    \hline
     Choice 1 & 
      \parbox{12cm}{\begin{align}
         p\vec{I}+\vec{A}=\vec{P}(p\vec{I})\vec{P}^T+\vec{P}\vec{D}\vec{P}^T\\
   = \vec{P}\vec{D}_1\vec{P}^T\\
    \vec{D}_1=\myvec{\lambda_1+p&0&0&0\\0&\lambda_2+p&0&0\\0&0&\lambda_3+p&0\\0&0&0&\lambda_4+p}
    \end{align}}\\
     & Some of the eigen values of $\vec{A}$ may be negative.\\
     & All the eigen values in $\vec{D}_1$ are positive only if \\
     & \parbox{12cm}{\begin{align}
          p>|\lambda_i|\text{  } \forall i \in [1,4]
     \end{align}}\\
      
      \hline
      Choice 2 & 
       \parbox{12cm}{\begin{align}
          \vec{A}^2=\vec{A}\vec{A}\\
    =(\vec{P}\vec{D}\vec{P}^T)(\vec{P}\vec{D}\vec{P}^T)\\
    =\vec{P}\vec{D}^2\vec{P}^T\\
    \text{Similarly, }\vec{A}^p=\vec{P}\vec{D}^p\vec{P}^T\\
    \vec{D}^p=\myvec{\lambda_1^p&0&0&0\\0&\lambda_2^p&0&0\\0&0&\lambda_3^p&0\\0&0&0&\lambda_4^p}
       \end{align}}\\
       & $\vec{A}^{p}$ is positive definite only if $p$ is even.\\
       \hline
      Choice 3& 
        \parbox{12cm}{\begin{align}
           \vec{A}^{-p}=\vec{P}\vec{D}^{-p}\vec{P}^T\\
    \vec{D}^{-p}=\myvec{\lambda_1^{-p}&0&0&0\\0&\lambda_2^{-p}&0&0\\0&0&\lambda_3^{-p}&0\\0&0&0&\lambda_4^{-p}}
       \end{align}}\\
       & $\vec{A}^{-p}$ is positive definite only if $p$ is even.\\
       \hline
      Choice 4 &
        \parbox{12cm}{\begin{align}
          \text{exp}(p\vec{A})=\sum_{k=0}^\infty \frac{(p\vec{A})^k}{k!}\\
     \implies  \text{exp}(p\vec{A})-\vec{I}=\vec{P} \text{exp}(p\vec{D})\vec{P}^T-\vec{P}\vec{I}\vec{P}^T\\
     =\vec{P}( \text{exp}(p\vec{D})-\vec{I}) \vec{P}^T\\
     \text{exp}(p\vec{D})-\vec{I}=\myvec{e^{\lambda_1}-1&0&0&0\\0&e^{\lambda_2}-1&0&0\\0&0&e^{\lambda_3}-1&0\\0&0&0&e^{\lambda_4}-1}
       \end{align}}\\
       & $\vec{A}$ is non-singular\\
        & \parbox{12cm}{\begin{align}
   \implies \forall i \in [1,4], \lambda_i\neq0\\
   e^{\lambda_i}<1
\end{align}}\\
 & So, exp$(p\vec{A})-\vec{I}$ is not positive definite.\\
       \hline
    \end{tabular}
    \caption{Solution}
\label{eq:solutions/2017/dec/77/table:0}
\end{table*}
 From the table,the choices would be option 1,2,3




