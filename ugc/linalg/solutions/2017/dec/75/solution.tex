See Table \ref{eq:solutions/2017/dec/75/table:1}.

\onecolumn
\begin{longtable}{|l|l|}
\hline
\multirow{3}{*}{} & \\
Statement 1. & A  is necessarily diagonalizable over $\vec{R}$\\
\hline
& \\
False statement& Matrix A is diagonalizable if and only if there is a basis of $\vec{R}^3 $consisting of\\
& eigenvectors of A.\\
Example:&Consider a matrix\\&\parbox{12cm}{\begin{align}
 \myvec{
1 &1 &0\\
0&1&1\\
0&0&4}\end{align}}\\
&Eigen values are:\\
&\parbox{12cm}{\begin{align}
 \myvec{
1 -\lambda &1 &0\\
0&1-\lambda&1\\
0&0&4-\lambda}=0.
\implies\lambda_1=1,\lambda_2=4\end{align}}\\
&\parbox{12cm}{\begin{align}
  \lambda_1=1\text { has eigen vector}
 \myvec{1\\0\\0} \text{and} 
  \lambda_2=4 \text{ has eigen vector}
\myvec{1\\3\\9}
\end{align}}\\
 & We have found only two linearly independent eigenvectors for A,not diagonalisable
\\
\hline
\multirow{3}{*}&\\
Statement 2. & If A has distinct real  eigen values
 than  it is diagonalizable over$\vec{R}$\\
\hline
&\\
True statement& Distinct real eigenvalues implies linearly independent eigenvectors .\\
& and if a matrix has n linearly independent vectors than it is  diagonalizable.\\
\hline
Proof  1:& \textbf{Distinct eigen values implies linearly independent vectors that spans entire space.}\\&Consider 2 eigen vectors $\vec{v}$,$\vec{w}$  with eigen values $\lambda$,$\mu$ respectively.\\
& such that $\lambda\neq\mu$\\
 &\parbox{12cm}{\begin{align}
    \alpha(\vec{v})+\beta(\vec{w})=0\label{eq:solutions/2017/dec/75/eq2}\\
     \alpha A(\vec{v})+\beta A(\vec{w})=0\\
     \alpha \lambda\vec{v}+\beta\mu\vec{w}=0\label{eq:solutions/2017/dec/75/eq3}
     \end{align}}\\
     & Multiplying $\eqref{eq:solutions/2017/dec/75/eq2}$with -$\lambda$ and subtracting from $\eqref{eq:solutions/2017/dec/75/eq3}$ we have,\\
   & \parbox{12cm}{\begin{align}  
  \beta(\mu-\lambda)\vec{w}=0 \label{eq:solutions/2017/dec/75/eq1}
  \end{align}}\\
  & eigen values are distinct $(\mu-\lambda)\neq 0$ .
  From equation$\eqref{eq:solutions/2017/dec/75/eq1}$ we have, $\beta=0$\\
  & substituting $\beta=0$ in  equation $\eqref{eq:solutions/2017/dec/75/eq2}$we have,$\alpha=0$.As, $\vec{v}\neq 0$\\
  & \textbf{which proves that vectors are linearly independent}.\\
  
  Proof 2:
 & \textbf{If a matrix has n linearly independent vectors than it is diagonalizable}\\
 & If\myvec{
\vec{p_1} &\vec{p_2}&\cdots&\vec{p_n} 
}are n independent eigen vectors then,
 $A\vec{p_1}=\lambda \vec{p_1},\cdots ,A\vec{p_n}=\lambda \vec{p_n}$\\
&\parbox{12cm}{\begin{align}{D}=\myvec{\lambda_1&0&\cdots&0\\
0&\lambda_2&\cdots&0\\
\vdots&\vdots&\vdots&\vdots\\
0&0&\cdots&\lambda_n}
 \and{P}=\myvec{
\vec{P_1}& \vec{P_2}&\cdots& \vec{P_n}
}\end{align}}\\
& Now, $A\vec{P_i}=\lambda_i\vec{P_i}$ $\implies$ ${A}{P}={P}{D}$\\
& so,${P^{-1}}{AP}={D}$ is a diagonal matrix.\\
    \hline
\multirow{3}{*} & \\ 
Statement 3.  & If A has distinct real  eigen values
 than  it is diagonalizable over$\vec{C}$\\
\hline
&\\
True statement& If A is an $N \times N$ complex matrix with n distinct eigenvalues, then any set of\\
& n corresponding eigenvectors form a basis for $\vec{C}^n$\\ .
&\\
Proof: &It is sufficient to prove that the set of eigenvectors is linearly independent \\
&which is proved in statement 2.\\
 Example:&\parbox{12cm}{\begin{align}A=\myvec{
4& 0 &-2\\
2& 5 &4\\
0& 0 &5
}\end{align}}\\
& Eigen values of A are:\\
& \parbox{12cm}{\begin{align}\lambda_1=2,\lambda_2=3 , \lambda_3=6\end{align}}\\
\hline
& Eigen vectors are:\\&\parbox{12cm}{\begin{align}x_1=\myvec{
-1\\1\\0
},
x_ 2=\myvec{
1\\1\\1
},
x_3=\myvec{
-1\\ - 1\\ 2
}\end{align}}
\\
 & Matrix A is diagonalizable because there is a basis of $\vec{C}^3 $consisting of\\
& eigenvectors of A.\\
\hline
\multirow{3}{*}&\\
Statement 4. & If all eigen values are non zero than it is diagonalizable over $\vec{C}$\\
\hline
& \\
False Statement:& Matrix would be diagonalizable if and only if it has linearly independent eigenvectors .\\
\hline
Example:&Consider a matrix\\&\parbox{12cm}{\begin{align}
 \myvec{
1 &1 &0\\
0&1&1\\
0&0&4}\end{align}}\\
&Eigen values are:\\
&\parbox{12cm}{\begin{align}
 \myvec{
1 -\lambda &1 &0\\
0&1-\lambda&1\\
0&0&4-\lambda}=0.
\implies\lambda_1=1,\lambda_2=4\neq 0\end{align}}\\
&\parbox{12cm}{\begin{align}
  \lambda_1=1\text { has eigen vector}
 \myvec{1\\0\\0} \text{and} 
  \lambda_2=4 \text{ has eigen vector}
\myvec{1\\3\\9}
\end{align}}\\
 & We have found only two linearly independent eigenvectors for A,not diagonalisable.\\
 &\\
\hline
\caption{Solution summary}
\label{eq:solutions/2017/dec/75/table:1}
\end{longtable}

   

