Let S be a set of functions.
Let $f_1,f_2$ $\in$ $S$ and $\alpha,\beta$ $\in$ $\Re$\\
 For a set of functions to be considered as a vector space:
\begin{enumerate}
\item The linear combination of $f_1$ and $f_2$ should be in S.\\
i.e. $\alpha f_1(x) + \beta f_2(x)$ $\in$ $S$
\item The $\vec{0}$ should belong to S\\
i.e. $\vec{0}$ $\in$ $S$
\end{enumerate}
Case1: Test for $S_1$
\begin{enumerate}
\item Let $f_1,f_2$ $\in$ $S_1$ and $\alpha,\beta$ $\in$ $\Re$
\begin{equation}\label{eq:solutions/2016/dec/31/3.0.1}
\begin{split}
\lim_{x\to3} f_1(x) = 0\\
\lim_{x\to3} f_2(x) = 0
\end{split}
\end{equation}
Then Using \eqref{eq:solutions/2016/dec/31/3.0.1}
\begin{equation*}
\begin{split}
\lim_{x\to3} (\alpha f_1(x) + \beta f_2(x))\\
= \alpha \left(\lim_{x\to3} f_1(x)\right) + \beta \left(\lim_{x\to3} f_2(x)\right)\\
= \alpha \times 0 + \beta \times 0\\
= 0\\
 \therefore \alpha f_1(x) + \beta f_2(x) \in S_1
\end{split}
\end{equation*}
\item Let $f(x) = 0$\\
then 
\begin{equation*}
\begin{split}
\lim_{x\to3} f(x) = 0\\
\therefore \vec{0} \in S_1
\end{split}
\end{equation*}
\end{enumerate}
Hence, $S_1$ is a vector space.\\
\\
Case2: Test for $S_2$
\begin{enumerate}
\item Let $g_1,g_2$ $\in$ $S_2$ and $\alpha,\beta$ $\in$ $\Re$
\begin{equation}\label{eq:solutions/2016/dec/31/3.0.2}
\begin{split}
\lim_{x\to3} g_1(x) = 1\\
\lim_{x\to3} g_2(x) = 1
\end{split}
\end{equation}
Then Using \eqref{eq:solutions/2016/dec/31/3.0.2}
\begin{equation*}
\begin{split}
\lim_{x\to3} (\alpha g_1(x) + \beta g_2(x))\\
= \alpha \left(\lim_{x\to3} g_1(x)\right) + \beta \left(\lim_{x\to3} g_2(x)\right)\\
= \alpha \times 1 + \beta \times 1\\
= \alpha + \beta\\
 \therefore \alpha g_1(x) + \beta g_2(x) \in S_1 ~~iff~~  \alpha + \beta = 1
\end{split}
\end{equation*}
\item Let $g(x) = 0$\\
then 
\begin{equation*}
\begin{split}
\lim_{x\to3} g(x) = 1\\
\therefore \vec{0} \notin S_1
\end{split}
\end{equation*}
\end{enumerate}
Hence, $S_2$ is not a vector space.\\
\\
Case3: Test for $S_3$
\begin{enumerate}
\item Let $h_1,h_2$ $\in$ $S_3$ and $\alpha,\beta$ $\in$ $\Re$
\begin{equation} \label{eq:solutions/2016/dec/31/3.0.3}
\begin{split}
\lim_{x\to3} h_1(x) ~exists\\
\lim_{x\to3} h_2(x) ~exists
\end{split}
\end{equation}
Then Using \eqref{eq:solutions/2016/dec/31/3.0.3}
\begin{equation*}
\begin{split}
\lim_{x\to3} (\alpha h_1(x) + \beta h_2(x)) ~exists\\
 \therefore \alpha h_1(x) + \beta h_2(x) \in S_3
\end{split}
\end{equation*}
\item Let $h(x) = 0$\\
then 
\begin{equation*}
\begin{split}
\lim_{x\to3^-} h(x) = 0 = \lim_{x\to3^+} h(x)\\
\therefore \vec{0} \in S_1
\end{split}
\end{equation*}
\end{enumerate}
Hence, $S_3$ is a vector space.\\
\\
Therefore, Option (3) is correct.
