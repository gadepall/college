Let the two systems of homogenous equations be 
\begin{align}
   \vec{A}\vec{x}=\vec{0}\\
 \vec{B}\vec{y}=\vec{0}
\end{align}
We can write
\begin{align}
  \vec{C}\vec{A}\vec{x}=\vec{0}\label{eq:solutions/1/1/6/1}\\
 \vec{D}\vec{B}\vec{y}=\vec{0}\label{eq:solutions/1/1/6/2}
\end{align}
where $\vec{C}$ and $\vec{D}$ are product of elementary matrices that reduce $\vec{A}$ and $\vec{B}$ into their reduced row echelon forms $\vec{R_1}$ and $\vec{R_2}$\\
\eqref{eq:solutions/1/1/6/1} and \eqref{eq:solutions/1/1/6/2} imply
\begin{align}
    \vec{R_1}\vec{x}=0\\
     \vec{R_2}\vec{y}=0
\end{align}
Given that they have same solution, we can write
\begin{align}
     \vec{R_1}\vec{x}=0\label{eq:solutions/1/1/6/3}\\
     \vec{R_2}\vec{x}=0\label{eq:solutions/1/1/6/4}\\
     \implies (\vec{R_1}-\vec{R_2})\vec{x}=0\label{eq:solutions/1/1/6/5}
\end{align}
Note that for a solution to exist, $\vec{R_1}$ and $\vec{R_2}$ can be either of matrices
\begin{align}
    \myvec{1&0\\0&0} \text{or } \myvec{1&0\\0&1}
\end{align}
{Case 1}
Let us assume that the solution is unique.
The unique solution is
\begin{align}
    \vec{x}=\vec{0}
\end{align}
Since they have the same solution, both $\vec{R_1},\vec{R_2}$ must have their rank as 2.\\
So,
\begin{align}
    \vec{R_1}=\vec{R_2}=\myvec{1&0\\0&1}
\end{align}

{Case 2}
Let us assume that \eqref{eq:solutions/1/1/6/1},\eqref{eq:solutions/1/1/6/2} have infinitely many solutions\\
So,
\begin{align}
\text{rank}(\vec{A})= \text{rank}(\vec{B}) = 1
\end{align}
equation \eqref{eq:solutions/1/1/6/5} for solutions other than zero solution implies
\begin{align}
    \vec{R_1}=\vec{R_2}=\myvec{1&0\\0&0}
\end{align}
So, in both the cases, we have 
\begin{align}
     \vec{R_1}=\vec{R_2}\\
     \implies \vec{C}\vec{A}=\vec{D}\vec{B}
\end{align}
Since $\vec{C},\vec{D}$ are product of elementary matrices, they are invertible.
\begin{align}
    \implies \vec{A}= \vec{C^{-1}}\vec{D}\vec{B}\label{eq:solutions/1/1/6/6}\\
    \vec{B}=\vec{D^{-1}}\vec{C}\vec{A}\label{eq:solutions/1/1/6/7}\\
    \text{Let } \vec{C^{-1}}\vec{D}=\vec{E}
\end{align}
where $\vec{E}$ is also a product of elementary matrices\\
\eqref{eq:solutions/1/1/6/6} and \eqref{eq:solutions/1/1/6/7} hence become
\begin{align}
\vec{A}=\vec{E}\vec{B}\\
\vec{B}=\vec{E^{-1}}\vec{A}
\end{align}
Hence the two systems of equations are equivalent.

