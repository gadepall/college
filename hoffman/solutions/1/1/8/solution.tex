The characteristic of a field is defined to be the smallest number of times one must use the field's multiplicative identity (1) in a sum to get the additive identity. If this sum never reaches the additive identity (0), then the field is said to have characteristic zero.\\
Let $\mathbb{Q}$ be the rational number field. Hence,
\begin{align}
0 &\in \mathbb{Q} \qquad{\text{[Additive Identity]}}\\
1 &\in \mathbb{Q}
\qquad{\text{[Multiplicative Identity]}}
\end{align}
As addition is defined on $\mathbb{Q}$ hence we have, 
\begin{align}
1 &\not= 0\label{eq:solutions/1/1/8/eq1}\\
1+1 = 2 &\not= 0\label{eq:solutions/1/1/8/eq2}
\intertext{And so on,}
1+1+\dots+1 = n &\not= 0\label{eq:solutions/1/1/8/eq3}
\end{align}
From the definition of characteristic of a field and from \eqref{eq:solutions/1/1/8/eq1}, \eqref{eq:solutions/1/1/8/eq2} and so on up-to \eqref{eq:solutions/1/1/8/eq3}, the rational number field, $\mathbb{Q}$ has characteristic 0.
