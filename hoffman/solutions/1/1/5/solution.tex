
  To prove that ($\mathbb{F}$,+,$\cdot$) is a field we need to satisfy the following,
 \begin{enumerate}
\item{+ and $\cdot$ should be closed}
\begin{itemize}
\item For any a and b in $\mathbb{F}$, a+b $\in$ $\mathbb{F}$ and a $\cdot$ b $\in$ $\mathbb{F}$.For example 0+0=0 and 0$\cdot$0=0.
\end{itemize}
\item{+ and $\cdot$ should be commutative}
\begin{itemize}
\item For any a and b in $\mathbb{F}$, a+b = b+a and a $\cdot$ b = b$\cdot$ a. For example 0+1=1+0 and 0$\cdot$ 1=1$\cdot$ 0.
\end{itemize}
\item{+ and $\cdot$ should be associative}
\begin{itemize}
\item For any a and b in $\mathbb{F}$, a+(b+c) = (a+b)+c and a$\cdot$ (b $\cdot$ c) =  (a $\cdot$ b) $\cdot$ c. For example 0+(1+0)=(0+1)+0 and 0$\cdot$(1$\cdot$0)=(0$\cdot$1)$\cdot$0. 
\end{itemize}
\item{+ and $\cdot$ operations should have an identity element}
\begin{itemize}
\item If we perform a + 0 then for any value of a from $\mathbb{F}$ the result will be a itself. Hence 0 is an identity element of + operation.If we perform a $\cdot$ 1 then for any value of a from $\mathbb{F}$ the result will be a itself. Hence 1 is an identity element of $\cdot$ operation. 
\end{itemize}
\item{$\forall$ a $\in$ $\mathbb{F}$ there exists an additive inverse}
\begin{itemize}
\item For additive inverse to exist, $\forall$ a in $\mathbb{F}$ a+(-a)=0. For example. 1-1=0 and 0-0=0.
\end{itemize}
\item{$\forall$ a $\in$ $\mathbb{F}$ such that a is non zero there exists a multiplicative inverse}
\begin{itemize}
\item 
For multiplicative inverse to exist, $\forall$ a such that a is non zero in $\mathbb{F},  a\cdot a^{-1}$=1.For example $1\cdot 1^{-1}=1$.
\end{itemize}
\item{+ and $\cdot$ should hold distributive property}
\begin{itemize}
\item For any a,b and c in $\mathbb{F}$ the property a$\cdot$(b+c)=a$\cdot$b+a$\cdot$c should always hold true.For example 0$\cdot$(1+1)=0$\cdot$1+0$\cdot$1.
\end{itemize}
\end{enumerate}
Since the above properties are satisfied we can say that ($\mathbb{F}$,+,$\cdot$) is a field.
