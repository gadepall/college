
\begin{equation} \label{eq:solutions/1/1/3/eq:eq2}
\begin{split}
x_1-x_3=0\\
x_2+3x_3=0
\end{split}
\end{equation}
System of linear equations in \eqref{eq:solutions/1/1/3/eq:eq1} can be expressed in matrix form as,
\begin{align}
    &\vec{A}\vec{x}=0\\
    &\myvec{-1&1&4\\1&3&8\\\frac{1}{2}&1&\frac{5}{2}}\vec{x}=0
\end{align}
System of linear equations in \eqref{eq:solutions/1/1/3/eq:eq2} can be expressed in matrix form as,
\begin{align}
    &\vec{B}\vec{x}=0\\
    &\myvec{1&0&-1\\0&1&3}\vec{x}=0
\end{align}
Two system of linear equations are equivalent if one system can be expressed as a linear combination of other system.\\
Matrix $\vec{B}$ can be obtained from matrix $\vec{A}$ as,
\begin{align}
&   \vec{B} = \vec{C}\vec{A}\\
&   \myvec{1&0&-1\\0&1&3} = \vec{C}\myvec{-1&1&4\\1&3&8\\\frac{1}{2}&1&\frac{5}{2}}\\
& \vec{C} = \myvec{-1&1&-2\\\frac{1}{2}&-\frac{1}{2}&2}
\end{align}
Now, writing equations in matrix-vector form,
\begin{align*}
    &x_1-x_3=\myvec{1&0&-1}\vec{x}
\end{align*}
\begin{multline}
\implies \myvec{1&0&-1}\vec{x} = -1\myvec{-1&1&4}\vec{x}\\+1\myvec{1&3&8}\vec{x}-2\myvec{\frac{1}{2}&1&\frac{5}{2}}\vec{x}\label{eq:solutions/1/1/3/eq:eq3}
\end{multline}
\begin{align*}
    &x_2+3x_3=\myvec{0&1&3}\vec{x}
\end{align*}
\begin{multline}
\implies \myvec{0&1&3}\vec{x} = \frac{1}{2}\myvec{-1&1&4}\vec{x}\\-\frac{1}{2}\myvec{1&3&8}\vec{x}+2\myvec{\frac{1}{2}&1&\frac{5}{2}}\vec{x}\label{eq:solutions/1/1/3/eq:eq4}
\end{multline}
Equations \eqref{eq:solutions/1/1/3/eq:eq3} and \eqref{eq:solutions/1/1/3/eq:eq4} is same as multiplying $\vec{C}$ with $\vec{A}$ which is the linear combination of rows of matrix $\vec{A}$.\\
Thus each equation in second system can be expressed as linear combination of the equations in first system.\\
Therefore, the two system of linear equations are equivalent.
