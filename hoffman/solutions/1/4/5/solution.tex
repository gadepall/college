The matrix $\vec{A}$ is in row reduced echolon form with four pivot elements. Therefore the
rank($\vec{A}$) is 4. Hence the rows of matrix $\vec{A}$ constitute of 4 linearly independent
vectors. Thus it can be concluded that matrix $\vec{A}$ is invertible. Using Gauss-Jordan Elimination,
if there exists an elimentary matrix $\vec{E}$ such that $\vec{E[A \ I]=[I \ E]}$ then $\vec{E}$ is the inverse of A i.e $\vec{E}=\vec{A^{-1}}$.
\begin{align}
	\vec{[A \ I]} = \myvec{ 1 & 2 & 3 & 4 &| & 1 & 0 & 0 & 0 \\
				0 & 2 & 3 & 4 &| & 0 & 1 & 0 & 0 \\
				0 & 0 & 3 & 4 &| & 0 & 0 & 1 & 0 \\
				0 & 0 & 0 & 4 &| & 0 & 0 & 0 & 1 }\\
			\xleftrightarrow[]{R_1 \leftarrow R_1 - R_2}
			\myvec{ 1 & 0 & 0 & 0 &| & 1 & -1 & 0 & 0 \\
                                0 & 2 & 3 & 4 &| & 0 & 1 & 0 & 0 \\
                                0 & 0 & 3 & 4 &| & 0 & 0 & 1 & 0 \\
                                0 & 0 & 0 & 4 &| & 0 & 0 & 0 & 1 }\\
			\xleftrightarrow[]{R_2 \leftarrow R_2 - R_3}
                        \myvec{ 1 & 0 & 0 & 0 &| & 1 & -1 & 0 & 0 \\
                                0 & 2 & 0 & 0 &| & 0 & 1 & -1 & 0 \\
                                0 & 0 & 3 & 4 &| & 0 & 0 & 1 & 0 \\
                                0 & 0 & 0 & 4 &| & 0 & 0 & 0 & 1 }
\end{align}
\begin{align}
			\xleftrightarrow[]{R_3 \leftarrow R_3 - R_4}
                        \myvec{ 1 & 0 & 0 & 0 &| & 1 & -1 & 0 & 0 \\
                                0 & 2 & 0 & 0 &| & 0 & 1 & -1 & 0 \\
                                0 & 0 & 3 & 0 &| & 0 & 0 & 1 & -1 \\
                                0 & 0 & 0 & 4 &| & 0 & 0 & 0 & 1 }\\
\xleftrightarrow[R_2 \leftarrow \frac{R_2}{2} \ R_3 \leftarrow \frac{R_3}{3}]
		{R_4 \leftarrow \frac{R_4}{4}}
                \myvec{ 1 & 0 & 0 & 0 &| & 1 & -1 & 0 & 0 \\
			0 & 1 & 0 & 0 &| & 0 & \frac{1}{2} & -\frac{1}{2} & 0 \\
			0 & 0 & 1 & 0 &| & 0 & 0 & \frac{1}{3} & -\frac{1}{3} \\
			0 & 0 & 0 & 1 &| & 0 & 0 & 0 & \frac{1}{4}}\nonumber \\ 
			= \vec{[I \ E]}
\end{align}
Therefore, for the given problem,
\begin{align}
        \vec{A^{-1}} = \myvec{ 1 & -1 & 0 & 0 \\
                        0 & \frac{1}{2} & -\frac{1}{2} & 0 \\
                        0 & 0 & \frac{1}{3} & -\frac{1}{3} \\
                        0 & 0 & 0 & \frac{1}{4}}
\end{align}
{Generalization of above result to a matrix of any arbitrary size:}
Let
\begin{align}
	\vec{A} = \myvec{ a_1 & a_2 & a_3 & \ldots & a_N \\
			    0 & a_2 & a_3 & \ldots & a_N \\
			  \vdots & & \vdots &  & \vdots  \\
			    0 & \ldots & \ldots & \ldots & a_N }
\end{align}
Then
\begin{align}
	\vec{E}_1\vec{A} &= \myvec{1 & -1 & 0 & \ldots & 0 \\
			    0 & 1 & 0 & \ldots & 0 \\
			  \vdots & & \vdots &  & \vdots  \\
			    0 & \ldots & \ldots & \ldots & 1 }
	\myvec{ a_1 & a_2 & a_3 & \ldots & a_N \\
			    0 & a_2 & a_3 & \ldots & a_N \\
			  \vdots & & \vdots &  & \vdots  \\
			    0 & \ldots & \ldots & \ldots & a_N } \\
	&=
	\myvec{ a_1 & 0 & 0 & \ldots & 0 \\
			    0 & a_2 & a_3 & \ldots & a_N \\
			  \vdots & & \vdots &  & \vdots  \\
			    0 & \ldots & \ldots & \ldots & a_N }
\end{align}
\begin{align}
	\vec{E}_2\vec{E}_1\vec{A} &= \myvec{1 & 0 & 0 & \ldots & 0 \\
			    0 & 1 &-1 & \ldots & 0 \\
			  \vdots & & \vdots &  & \vdots  \\
			    0 & \ldots & \ldots & \ldots & 1 }
	\myvec{ a_1 & 0 & 0 & \ldots & 0 \\
			    0 & a_2 & a_3 & \ldots & a_N \\
			  \vdots & & \vdots &  & \vdots  \\
			    0 & \ldots & \ldots & \ldots & a_N } \\
		&=
	\myvec{ a_1 & 0 & 0 & \ldots & 0 \\
			    0 & a_2 & 0 & \ldots & 0 \\
			  \vdots & & \vdots &  & \vdots  \\
			    0 & \ldots & \ldots & \ldots & a_N }
\end{align}
Proceeding in similar manner, we get
\begin{align}
	\vec{E}_N\vec{E}_{N-1}\ldots\vec{E}_2\vec{E}_1\vec{A} = \vec{U} 
	&= 
	\myvec{ a_1 & 0 & 0 & \ldots & 0 \\
		0 & a_2 & 0 & \ldots & 0 \\
		0 & 0   & a_3 & \ldots & 0 \\
		\vdots & & \vdots &  & \vdots  \\
		    0 & \ldots & \ldots & \ldots & a_N } \\
	&= \text{diag} \myvec{a_1 & a_2 & \ldots & a_N}
\end{align}
\begin{align}
	\implies \quad \vec{A} = \vec{L} \vec{U} 
\end{align}
	where \quad 
	$\vec{L} = \vec{E}_1^{-1} \vec{E}_2^{-1} \ldots \vec{E}_N^{-1}$ 
\begin{align}
	\implies \quad \vec{A}^{-1}  = \vec{U}^{-1} \vec{L}^{-1} 
\end{align}
\begin{align}
	\implies \quad \vec{A}^{-1}  
	= 
	\myvec{ \frac{1}{a_1} & 0 & 0 & \ldots & 0 \\
		0 & \frac{1}{a_2} & 0 & \ldots & 0 \\
		0 & 0   & \frac{1}{a_3} & \ldots & 0 \\
		\vdots & & \vdots &  & \vdots  \\
		    0 & \ldots & \ldots & \ldots & \frac{1}{a_N} }
	 \myvec{1 & -1 & 0 & \ldots & 0 \\
			    0 & 1 &-1 & \ldots & 0 \\
			  \vdots & & \vdots &  & \vdots  \\
			    0 & \ldots & \ldots & \ldots & 1 }
\end{align}
Therefore
\begin{align}
	\vec{A^{-1}} = \myvec{\frac{1}{a_1} &-\frac{1}{a_1} & 0 & 0&\ldots &0 \\
			0 & \frac{1}{a_2} & -\frac{1}{a_2} & 0 & \ldots &0 \\
			0 & 0 & \frac{1}{a_3} & -\frac{1}{a_3} & \ldots &0\\
			0 & 0 & 0 & 0 & \ldots & \frac{1}{a_N}} \label{eq:solutions/1/4/5/result}
\end{align}
From (\ref{eq:solutions/1/4/5/result}) for the above problem 
\begin{align}
        \vec{A^{-1}} = \myvec{ 1 & -1 & 0 & 0 \\
                        0 & \frac{1}{2} & -\frac{1}{2} & 0 \\
                        0 & 0 & \frac{1}{3} & -\frac{1}{3} \\
                        0 & 0 & 0 & \frac{1}{4}}
\end{align}

