The matrix $\vec{A}$ can be given by, 
\begin{align}
    \vec{A} = \myvec{\vec{m}&\vec{n}} \\
    \vec{m} = \myvec{m_1\\m_2}, \vec{n} = \myvec{n_1\\n_2}
\end{align}
Now, 
\begin{align}
    \vec{A}^{2} &= \vec{A}\vec{A} = \vec{0} \\
    \implies\vec{A}^{2} &= \myvec{\vec{A}\vec{m} & \vec{A}\vec{n}} = \myvec{\vec{0}&\vec{0}} \label{eq:solutions/1/3/3/A2}
\end{align}
From \eqref{eq:solutions/1/3/3/A2}, we say that the the null space of $\vec{A}$ contains columns of matrix $\vec{A}$. Also atleast one of the columns must be non-zero since given $\vec{A}\ne0$. 
Thus, the null space of $\vec{A}$ contains non zero vectors, $rank(\vec{A})<2$. Hence, $\vec{A}$ is a singular matrix.
This implies that the columns of $\vec{A}$ are linearly dependent.
\begin{align}
    \vec{A}\vec{x}=0 \\
    \myvec{\vec{m}&\vec{n}}\myvec{x_1\\x_2} = 0 \\
    x_1\vec{m}+x_2\vec{n} = 0 \\
    \vec{n}= \frac{-x_1}{x_2}\vec{m} \\
    \implies\vec{n}=k\vec{m} \label{eq:solutions/1/3/3/k}
\end{align}
where $\vec{m}\ne0$ as $\vec{A}\ne0$

Now from \eqref{eq:solutions/1/3/3/A2},
\begin{align}
    \vec{A}\vec{m} &= 0 \\
    m_1\vec{m} + m_2\vec{n} &= 0 \\
    \brak{m_1+km_2}\vec{m} &= 0
\end{align}
Thus we get, $m_1=-km_2$
\begin{align}
    \vec{A} = \myvec{-km_2 & -k^2m_2 \\ m_2 & km_2} ; m_2\ne0 \label{eq:solutions/1/3/3/A_1}
\end{align}
\eqref{eq:solutions/1/3/3/k} can be written as, 
\begin{align}
    \implies\vec{m}=\frac{1}{k}\vec{n} \\
    \implies\vec{m} = c\vec{n}
\end{align}
where $\vec{n}\ne0$ as $\vec{A}\ne0$

From \eqref{eq:solutions/1/3/3/A2}, 
\begin{align}
    \vec{A}\vec{n} &=0 \\
    n_1\vec{m}+n_2\vec{n} &=0 \\
    \brak{cn_1+n_2}\vec{n} &=0
\end{align}
Thus we get, $n_2=-cn_1$
\begin{align}
    \vec{A} = \myvec{cn_1 & n_1 \\ -c^2n_1 & -cn_1} ; n_1\ne0 \label{eq:solutions/1/3/3/A_2}
\end{align}

From \eqref{eq:solutions/1/3/3/A_1}, \eqref{eq:solutions/1/3/3/A_2} two different 2$\times$2 matrices $\vec{A}$ can be given as,
\begin{align}
    \vec{A} &= \myvec{0&0\\2&0} \\
    \vec{A} &= \myvec{0&2 \\0&0}
\end{align}
