
 Since $\vec{R}$ and $\vec{R}^{'}$ are 2 $\times$ 3 row-reduced echelon matrices they can be of following three types:-
\begin{enumerate}
\item Suppose matrix $\vec{R}$ has one non-zero row then $\vec{R}\vec{X}$=0 will have two 
free variables.Since $\vec{R}^{'}\vec{X}$=0 will have the exact same solution as 
$\vec{R}\vec{X}=0, \vec{R}^{'} \vec{X}$=0 
will also have two free variables. Thus 
$\vec{R}^{'}$ have one non zero row.
Now let's consider a matrix $\vec{A}$ with the first 
row as the non-zero row $\vec{R}$ and second row as the second row of 
$\vec{R}^{'}$.
\begin{align}
\vec{R}=\myvec{1&a&b\\0&0&0}\\
\vec{R}^{'}=\myvec{1&c&d\\0&0&0}\\
  \end{align}
  Let $\vec{X}$ satisfy
  \begin{align}
  \vec{R}\vec{X}=0\label{eq:solutions/1/2/10/b}\\
  \myvec{1 & \vec{a}^T}\myvec{x \\ \vec{y}} = 0\\
  x+\vec{a}^T\vec{y}=0\label{eq:solutions/1/2/10/abcdef}
  \end{align}
  where 
  \begin{align}
  \vec{a}=\myvec{a \\ b}
  \end{align}
  \begin{align}
  \vec{R}^{'}\vec{X}=0\label{eq:solutions/1/2/10/c}\\
   \myvec{1 & \vec{b}^T}\myvec{x \\ \vec{y}} = 0\\
   x+\vec{b}^T\vec{y}=0\label{eq:solutions/1/2/10/abcde}
  \end{align}
  where 
  \begin{align}
  \vec{b}=\myvec{c \\ d}
  \end{align}
  Subtracting \eqref{eq:solutions/1/2/10/abcde} from \eqref{eq:solutions/1/2/10/abcdef},
  \begin{align}
  x+\vec{a}^T\vec{y}-x-\vec{b}^T\vec{y}=0\\
  (\vec{a}^T-\vec{b}^T)\vec{y}=0\label{eq:solutions/1/2/10/part1}
  \end{align}
  Since $\vec{y}$ is a $2 \times 1 $ vector,
  \begin{align}
 \implies y_1 \vec{a}-y_2\vec{b} = 0
 \end{align}
 Which can be written as,
 \begin{align}
 \vec{a}=k\vec{b}\label{eq:solutions/1/2/10/ak}
\end{align}
where,k=$\frac{\vec{y}_2}{\vec{y}_1}$ assuming $\vec{y}_1 \not$=0.Now,Substituting \eqref{eq:solutions/1/2/10/ak} in \eqref{eq:solutions/1/2/10/abcdef}
\begin{align}
x+k\vec{b}^T\vec{y}=0\label{eq:solutions/1/2/10/kb}
\end{align}
Comparing \eqref{eq:solutions/1/2/10/kb} with \eqref{eq:solutions/1/2/10/abcde}
\begin{align}
x+\vec{b}^T\vec{y}=0\\
x+k\vec{b}^T\vec{y}=0
\end{align}
Hence k=1 which means $y_1$=$y_2$ and from this we can say that $\vec{a}$=$\vec{b}$.If in the above case we take $y_1$=0 then
\begin{align}
y_1\vec{a}-y_2\vec{b}=0\\
y_2\vec{b}=0\label{eq:solutions/1/2/10/eqnb}
\end{align}
Hence for the \eqref{eq:solutions/1/2/10/eqnb} to be always true $\vec{b}$ should be zero.Now from \eqref{eq:solutions/1/2/10/ak} we will see that  $\vec{a}$ will also be 0.
 Hence, $\vec{R}$=$\vec{R}^{\prime}$
 \item Let $\vec{R}$ and $\vec{R}^{'}$ have all rows as non zero.
\begin{align}\vec{R}=\myvec{1&0&b\\0&1&c} \\\vec{R}^{'}=\myvec{1&0&e\\0&1&f}\end{align}
 Let $\vec{X}$ satisfy
  \begin{align}
  \vec{R}\vec{X}=0\label{eq:solutions/1/2/10/e}\\
  \vec{X}^{T}\vec{R}^{T}=0\label{eq:solutions/1/2/10/fi}
  \end{align}
  Here,
  \begin{align}
  \vec{R}=\myvec{\vec{I}&\vec{a}}\label{eq:solutions/1/2/10/f}\\
  \vec{a}=\myvec{b \\c}\\
  \vec{R}^{T}=\myvec{\vec{I}\\\vec{a}^T}
  \end{align}
  Let,
  \begin{align}
  \vec{X}^T=\myvec{\vec{y}^T&z}\label{eq:solutions/1/2/10/g}
  \end{align}
  where z is a scalar constant. Now,substituting \eqref{eq:solutions/1/2/10/g} and \eqref{eq:solutions/1/2/10/f} in \eqref{eq:solutions/1/2/10/fi}
  \begin{align}
  \myvec{\vec{y}^T&z}\myvec{\vec{I}\\\vec{a}^T}=0\\
  \vec{y}^T+z\vec{a}^T=0\label{eq:solutions/1/2/10/49}
  \end{align}
  Now for,
  \begin{align}
  \vec{R}^{'}\vec{X}=0\label{eq:solutions/1/2/10/ee}\\
  \vec{X}^{T}{\vec{R}^{'}}^{T}=0\label{eq:solutions/1/2/10/fii}
  \end{align}
  Here,
  \begin{align}
  \vec{R}^{'}=\myvec{\vec{I}&\vec{b}}\label{eq:solutions/1/2/10/ff}\\
  \vec{b}=\myvec{e\\f}
  \end{align}
  Let,
  \begin{align}
  \vec{X}^T=\myvec{\vec{y}^T&z}\label{eq:solutions/1/2/10/gg}
  \end{align}
  where z is a scalar constant.Now,substituting \eqref{eq:solutions/1/2/10/gg} and \eqref{eq:solutions/1/2/10/ff} in \eqref{eq:solutions/1/2/10/fii}
  \begin{align}
    \myvec{\vec{y}^T&z}\myvec{\vec{I}\\\vec{b}^T}=0\\
  \vec{y}^T+z\vec{b}^T=0\label{eq:solutions/1/2/10/50}
  \end{align}
  Subtracting \eqref{eq:solutions/1/2/10/50} from \eqref{eq:solutions/1/2/10/49} 
  \begin{align}
    \vec{y}^T+z\vec{a}^T-\vec{y}^T-z\vec{b}^T=0\\
    (\vec{a}^T-\vec{b}^T)z=0\label{eq:solutions/1/2/10/part2}\\
    \vec{a}^T=\vec{b}^T
  \end{align}
  \item Suppose matrix $\vec{R}$ have all the rows as zero then $\vec{R}\vec{X}$=0 will be satisfied for all values of $\vec{X}$. We know that $\vec{R}^{'}\vec{X}$=0 will have the exact same solution as 
 $\vec{R}\vec{X}$=0 then we can say that for all values of $\vec{X}$=0 equation $\vec{R}^{'}\vec{X}$=0 will be satisfied.Hence, $\vec{R}^{'}$=$\vec{R}$=0.

\end{enumerate}
