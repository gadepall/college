%
%	
\begin{align}
\vec{A}=
\myvec{6 & -4 & 0\\
4& -2 & 0\\
-1 &0 &3}
\end{align}
To calculate solution of  $\vec{AX}=2\vec{X}$and all solutions of $\vec{AX}=3\vec{X}$we calculate eigen values of $\vec{A}$:
\begin{align}\label{eq:solutions/1/2/3/eq1}
(\vec{A}-\lambda\vec{I})\vec{X}=0
\end{align}
Substituting values in \eqref{eq:solutions/1/2/3/eq1},
\begin{align}
\myvec{6-\lambda& -4 & 0\\
4 & -2-\lambda & 0\\
-1 &0 &3-\lambda}
\vec{X}=0
\end{align}
Simplifying:
\begin{align}\label{eq:solutions/1/2/3/eqdet}
\myvec{6-\lambda& -4 & 0\\
4 & -2-\lambda & 0\\
-1 &0 &3-\lambda}
\xleftrightarrow[]{R_1 \leftarrow R_1-R_2}\nonumber
\end{align}
\begin{align}
    \myvec{2-\lambda& -2+\lambda & 0\\
4 & -2-\lambda & 0\\
-1 &0 &3-\lambda}
\end{align}
Taking $(3-\lambda)$  and $(2-\lambda)$common from $C_3$ and $R_1$
\begin{align}
(3-\lambda)(2-\lambda)
    \myvec{1&-1&0\\
            4&-2-\lambda &0\\
            -1&0&1}
\end{align}
\begin{align}\label{eq:solutions/1/2/3/eqlambda}
\myvec{1&-1&0\\
            4&-2-\lambda &0\\
            -1&0&1}
            \xleftrightarrow[]{R_2 \leftarrow R_2-4R_1}
    \myvec{1&-1&0\\
            0&-\lambda +2&0\\
            -1&0&1}
\end{align}
Taking $(2-\lambda)$ common from $R_2$:
\begin{align}
(2-\lambda)^2(3-\lambda)
\myvec{
1 &-1 &0\\
0&1&0\\
-1&0&1
}
\end{align}
Eigen values are:
\begin{align}
 \lambda_1=2\\
 \lambda_2=3
 \end{align}
solution to $\vec{AX} =2\vec{X}$ is eigen vector corresponding to $\lambda=2$
\begin{align}
    (\vec{A}-2\vec{I})\vec X=0
\end{align}
Substituting values:
\begin{align}
\myvec{4& -4&0&0\\
4 & -4 & 0&0\\
-1 & 0 &1&0
}
\xleftrightarrow[]{R_1 \leftarrow \frac{R_1}{4}} 
&
\myvec{1&-1&0&0\\
4&-4&0&0\\
-1&0&1&0}
\xleftrightarrow[]{R_2 \leftarrow R_2-4R_1 } \nonumber 
\end{align}
\begin{align}
    \myvec{1&-1&0&0\\
            0&0&0&0\\
            -1&0&1&0}
    \xleftrightarrow[]{R_3 \leftarrow R_3-R_1 }
    \myvec{1&-1&0&0\\
            0&0&0&0\\
            0&-1&1&0}
    \xleftrightarrow[]{R_3 \xleftrightarrow[]{} R_2 } \nonumber
\end{align}
\begin{align}
    \myvec{1&-1&0&0\\
    0&-1&1&0\\
    0&0&0&0}
    \xleftrightarrow[]{R_2 \leftarrow -R_2} 
    \myvec{1&-1&0&0\\
    0&1&-1&0\\
    0&0&0&0}   
    \xleftrightarrow[]{R_1 \leftarrow R_1+R_2} \nonumber
\end{align}
\begin{align} \label{eq:solutions/1/2/3/eq:eq_5}
    \myvec{1&0&-1&0\\
    0&1&-1&0\\
    0&0&0&0} 
\end{align}
So, $x_3$ is a free variable: Let $x_3$ = $c$. 
\begin{align}
 {x_2}-{x_3}=0 \implies{x_2}={x_3}=c \\
 {x_1}-{x_3}=0 \implies{x_1}={x_3}=c 
 \end{align}
 So,the solution to $\vec{AX} =2\vec{X}$is
 \begin{align}\label{eq:solutions/1/2/3/eq:555}
 \vec{X} = 
 c
 \myvec{
 1\\1\\1
 }
\end{align}
 solution of $\vec{A}\vec{X}=3\vec{X}$ is eigen vector corresponding to $\lambda=3$ 
 \begin{align}\label{eq:solutions/1/2/3/eq11}
(\vec{A}-3\vec{I})\vec X=0
\end{align}
substituting we have:
\begin{align}
\myvec{3& -4&0&0\\
1 & -2 & 0&0\\
-1 & 0 &0&0
}
\xleftrightarrow[]{R_1 \leftarrow \frac{R_1}{3} }\nonumber
&
\myvec{1&-\frac{4}{3}&0&0\\
4&-5&0&0\\
-1&0&0&0}\xleftrightarrow[]{R_2 \leftarrow R_2-4R_1} 
\end{align}
\begin{align}
    \myvec{1&-\frac{4}{3}&0&0\\
            0&\frac{1}{3}&0&0\\
            -1&0&0&0}
    \xleftrightarrow[]{R_3 \leftarrow R_3+R_1 }\nonumber
    \myvec{1&-\frac{4}{3}&0&0\\
            0&\frac{1}{3}&0&0\\
            0&-\frac{4}{3}&0&0}
            \xleftrightarrow[]{R_2 \leftarrow\frac{R_2}{3} }\nonumber
\end{align}
\begin{align}
    \myvec{1&\frac{-4}{3}&0&0\\
    0&1&0&0\\
    0&-\frac{4}{3}&0&0}
    \xleftrightarrow[]{R_3 \leftarrow R_3-\frac{4}{3}R_2} \nonumber
    \myvec{1&\frac{4}{3}&0&0\\
           0&1&0&0\\
         0&0&0&0}    \nonumber
          \xleftrightarrow[]{R_3 \leftarrow R_1+\frac{4}{3}R_2}
\end{align}
\begin{align} \label{eq:solutions/1/2/3/eq:eq_15}
    \myvec{1&0&0&0\\
    0&1&0&0\\
    0&0&0&0} 
\end{align}
So $x_3$ is a free variable:
\begin{align}
  x_1=0\\
  x_2=0\\
  x_3=c
\end{align}
 So,the solution to $\vec {AX}$ =3$\vec{X}$ is,
 \begin{align}\label{eq:solutions/1/2/3/eq:222}
 \vec{X} =
 c
 \myvec{
 0\\0\\1
 }
 \end{align}

   

