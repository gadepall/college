
The goal is to effect the transformation $\brak{\vec{A}|\vec{I}} \rightarrow \brak{\vec{I}|\vec{A}^{-1}}$. Augmenting $\vec{A}$ with the $2 \times 2$ identity matrix, we get:
\begin{align}
 \myvec{a & b & 1 & 0 \\ c & d & 0 & 1} 
\end{align}
Now, if $a = 0$, switch the rows. If c is also 0, then the process of reducing $\vec{A}$ to $\vec{I}$ cannot even begin. So, one necessary condition for $\vec{A}$ to be invertible is that the entries a and c are not both 0. 
\begin{enumerate}
    \item Assume that $a \neq 0$, Then :
\begin{align}
     \label{eq:solutions/1/5/8/anotzero}\myvec{a & b & 1 & 0 \\ c & d & 0 & 1}
&\xleftrightarrow{R_1 
= R_1/a} \myvec{1 & \frac{b}{a} & \frac{1}{a} & 0 \\ c & d & 0 & 1}\\
&\xleftrightarrow{R_2 = R_2 - c R_1} \myvec{1 & \frac{b}{a} & \frac{1}{a} & 0 \\ 0 & \frac{ad-bc}{a} & \frac{-c}{a} & 1}\nonumber
\end{align}
Next, assuming that $ad - bc \neq 0$, we get:
\begin{align}
    &\xleftrightarrow{R_1 = R_1 - \frac{b}{ad-bc} R_2} \myvec{1 & 0 & \frac{d}{ad-bc} & \frac{-b}{ad-bc} \\ 0 & \frac{ad-bc}{a} & \frac{-c}{a} & 1} \nonumber\\
    &\xleftrightarrow{R_2 = R_2\frac{a}{ad-bc}} \myvec{1 & 0 & \frac{d}{ad-bc} & \frac{-b}{ad-bc} \\ 0 & 1 & \frac{-c}{ad-bc} & \frac{a}{ad-bc}} \nonumber
\end{align}
Therefore, if $ad - bc \neq 0$, then the matrix is invertible and it's inverse is given by 
\begin{align}
    \myvec{a & b \\ c  & d}^{-1} = \frac{1}{ad-bc}\myvec{d & -b \\ -c  & a}
\end{align}

\item
In \eqref{eq:solutions/1/5/8/anotzero}, we have assumed that $a \neq 0$. Now consider $a = 0$, then, as we have seen before, it is mandatory that $c \neq 0$:
\begin{align}
    \myvec{0 & b & 1 & 0 \\ c & d & 0 & 1} &\xleftrightarrow{R_1\leftrightarrow R_2}  \myvec{c & d & 0 & 1 \\ 0 & b & 1 & 0} \\
    &\xleftrightarrow{R_1 = R_1/c}\myvec{1 & \frac{d}{c} & 0 & \frac{1}{c} \\ 0 & b & 1 & 0}  \nonumber \\
    &\xleftrightarrow{R_1 = R_1 - R_2\times \frac{d}{bc}}\myvec{1 & 0 & -\frac{d}{bc} & \frac{1}{c} \\ 0 & b & 1 & 0}\nonumber\\
    &\xleftrightarrow{R_2 = R_2/b}\myvec{1 & 0 & -\frac{d}{bc} & \frac{1}{c} \\ 0 & 1 & \frac{1}{b} & 0} \nonumber
\end{align}
Therefore, When we consider $a=0$ the matrix is invertible if $bc\neq0$, which is included in the condition $ad-bc\neq0$. 
\item Similarly, consider $c=0$, then, as we have seen before, it is mandatory that $a\neq0$:
\begin{align}
     \myvec{a & b & 1 & 0 \\ 0 & d & 0 & 1} &\xleftrightarrow{R_1 = R_1/a}\myvec{1 & \frac{b}{a} & \frac{1}{a} & 0 \\ 0 & d & 0 & 1}\\
     &\xleftrightarrow{R_1 = R_1 - R_2\times\frac{b}{ad}}\myvec{1 & 0 & \frac{1}{a} & -\frac{b}{ad} \\ 0 & d & 0 & 1}\nonumber\\
     &\xleftrightarrow{R_2 = R_2/d}\myvec{1 & 0 & \frac{1}{a} & -\frac{b}{ad} \\ 0 & 1 & 0 & \frac{1}{d}} \nonumber
\end{align}
\end{enumerate}
Therefore, When we consider $c=0$, the matrix is invertible if $ad\neq0$, which is included in the condition $ad-bc\neq0$.\par
Hence, it is proved from above three cases that the given matrix is invertible iff $ad - bc \neq 0$. 
