
%
%
{Lemma}
Every elementary matrix is invertible and the inverse is again an elementary matrix. If an elementary matrix $E$ is obtained from I by using a certain row or column operation $q$, then $E^{-1}$ is obtained from $I$ by the "inverse" operation $q^{-1}$.

{Solution}
Given $\vec{A}$ is a $m\times n$ matrix. Converting $\vec{A}$ into row reduced echelon form by performing a series of elementary row operations $\vec{P}$. Let $\vec{R^{\prime}}$ be the row reduced echelon matrix. Also, by using the lemma we can tell that $\vec{P}$ is invertible and order $m\times m$.
\begin{align} 
    \vec{R^{\prime}} = \vec{P}\vec{A} \label{eq:solutions/1/5/11/eq:eq_1}
\end{align}
where,
\begin{align}
    \vec{R^{\prime}} = \myvec{\vec{I} & \vec{F} \\ \vec{0} & \vec{0}} \nonumber
    \intertext{$\vec{I}$ is an identity matrix, $\vec{F}$ is Free variables matrix and $\vec{0}$ represents a block of zeroes} 
    \nonumber
\end{align} 
$\vec{R^{\prime}}$ is in row-reduced echelon form. To perform column operations, elementary matrices should be multiplied on the right side in order to convert the $\vec{R^{\prime}}$ into column-reduced echelon form
\begin{align} \label{eq:solutions/1/5/11/eq:eq_2}
    \vec{R} = \vec{R^{\prime}} \vec{Q}
\end{align}
But performing column operations on a matrix is equivalent to performing row operations on the transposed matrix.
\begin{align}
    \vec{R}^T &= {(\vec{R^{\prime}} \vec{Q})}^T \nonumber \\
    \implies \vec{R}^T &= {\vec{Q}}^T {\vec{R^{\prime}}}^T \label{eq:solutions/1/5/11/eq:eq_3}
\end{align}
Hence, by using lemma it can be observed that ${\vec{Q}}^T$ is invertible and of the order $n\times n$. Converting $\vec{R}^T$ to row-reduced echelon is equivalent to converting $\vec{R}$ to column-reduced echelon. 
\begin{align} \label{eq:solutions/1/5/11/eq:eq_4}
    \vec{R} = \vec{P}\vec{A}\vec{Q}
\end{align}
where,
\begin{align}
    \vec{R} = \myvec{\vec{I} & \vec{0} \\ \vec{0} & \vec{0}} \nonumber \\
\end{align}
$\vec{I}$ is an identity matrix and $\vec{0}$ represents a block of zeroes. $\vec{Q}$ is a upper triangular matrix. $\vec{R}$ in \eqref{eq:solutions/1/5/11/eq:eq_4} is in both row and column reduced echelon form. Hence proved.

{Example}
Let,
\begin{align} \label{eq:solutions/1/5/11/eq:eq_41}
    \vec{A} = \myvec{1 & 2 & 3 & 4 \\ 2 & 4 & 5 & 7 \\ 1 & 2 & 3 & 4}
\end{align}
To convert \eqref{eq:solutions/1/5/11/eq:eq_41} into row reduced echelon form, $\vec{A}$ has to be multiplied by $\vec{P}$
\begin{align}
    \vec{P} = \myvec{-5 & 3 & 0 \\ 2 & -1 & 0 \\ -1 & 0 & 1} \label{eq:solutions/1/5/11/eq:eq_42}
\end{align}
\begin{align}
    \vec{R^{\prime}} = \vec{P}\vec{A} = \myvec{1 & 2 & 0 & 1\\ 0 & 0 & 1 & 1 \\ 0 & 0 & 0 & 0} \label{eq:solutions/1/5/11/eq:eq_43}
\end{align}
$\vec{R^{\prime}}$ is in row reduced echelon form. To convert \eqref{eq:solutions/1/5/11/eq:eq_43} into column-reduced echelon form, elementary operations have to be performed on ${\vec{R^{\prime}}}^T$. 
By multiplying all the elementary matrices,
\begin{align} 
    \vec{Q}^T = \myvec{1 & 0 & 0 & 0 \\ 0 & 0 & 1 & 0 \\ -2 & 1 & 0 & 0 \\ -1 & 0 & -1 & 1} \label{eq:solutions/1/5/11/eq:eq_45} \\ 
    \implies \vec{Q} = \myvec{1 & 0 & -2 & -1 \\ 0 & 0 & 1 & 0 \\ 0 & 1 & 0 & -1 \\ 0 & 0 & 0 & 1} \label{eq:solutions/1/5/11/eq:eq_46}
\end{align}
So $\vec{P}\vec{A}\vec{Q}$ is in both row-reduced and column-reduced echelon form.
\begin{align} \label{eq:solutions/1/5/11/eq:eq_47}
    \vec{R} = \vec{P}\vec{A}\vec{Q} = \myvec{1 & 0 & 0 & 0 \\ 0 & 1 & 0 & 0 \\ 0 & 0 & 0 & 0}
\end{align}
The inverses of $\vec{P}$ and $\vec{Q}$ are,
\begin{align} \label{eq:solutions/1/5/11/eq:eq_48}
    \vec{P}^{-1} = \myvec{1 & 3 & 0 \\ 2 & 5 & 0 \\ 1 & 3 & 1}; \quad 
    \vec{Q}^{-1} = \myvec{1 & 2 & 0 & 1 \\ 0 & 0 & 1 & 1 \\ 0 & 1 & 0 & 0 \\ 0 & 0 & 0 & 1}
\end{align}
