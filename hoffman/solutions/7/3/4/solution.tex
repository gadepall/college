%
{\em Theory: }
Table \ref{eq:solutions/7/3/4/table:1} gives the overview of properties of a Jordan block based on characteristic polynomial, minimal polynomial, algebraic multiplicity and geometric multiplicity.
\begin{table*}[!ht]
\begin{center}
\begin{tabular}{|c|c|c|}
\hline
& &\\
Feature & Effect on Jordan block & Example\\
& &\\
\hline

& &\\
characteristic & The multiplicity of $\lambda$ & Let, $f = (x-2)^4$ be characteristic polynomial\\
polynomial or & in the characteristic polynomial & $\vec{J} = \myvec{2 & * & 0 & 0 \\ 0 & 2 & * & 0 \\ 0 & 0 & 2 & * \\ 0 & 0 & 0 & 2}$\\
Algebraic & determines the size of the & where $*$ can be either 1 or 0\\
multiplicity & Jordan block for that eigenvalue.&\\
& $A_M$ = Size of Jordan block for $\lambda$& \\
& &\\
\hline

& & \\
Geometric & The geometric multiplicity & If $A_M = 4; \: G_M = 2; \: \lambda = 2$\\
multiplicity & determines the total number of & $\implies$ There should be 2 Jordan sub blocks\\ 
& Jordan sub blocks for $\lambda$ & for $\lambda = 2$. So, $\vec{J}$ has 2 possibilities\\
& & $\vec{J} = \myvec{2 & 1 & 0 & 0 \\ 0 & 2 & 0 & 0 \\ 0 & 0 & 2 & 1 \\ 0 & 0 & 0 & 2}$;\: or $\vec{J} = \myvec{2 & 1 & 0 & 0 \\ 0 & 2 & 1 & 0 \\ 0 & 0 & 2 & 0 \\ 0 & 0 & 0 & 2}$\\
\hline

& &\\
minimal & The multiplicity of $\lambda$ & Let $p = (x-2)^3$ be minimal polynomial\\
polynomial & in the minimal polynomial& $\implies$ Size of largest sub-block is 3\\
& determines the size of & Hence, one sub-block of size 3 and\\
& the largest sub-block & one sub-block of size 1\\
& (Elementary Jordan block).& $\vec{J} = \myvec{2 & 1 & 0 & 0 \\ 0 & 2 & 1 & 0 \\ 0 & 0 & 2 & 0 \\ 0 & 0 & 0 & 2}$\\
& &\\
\hline
\end{tabular}
\caption{Properties of Jordan blocks and Jordan canonical form}
\label{eq:solutions/7/3/4/table:1}
\end{center}
\end{table*}

From the properties stated in table \ref{eq:solutions/7/3/4/table:1}, the Jordan blocks for eigenvalues of $\vec{A}$ can be written as,
\begin{align}
    \vec{J_1} = \myvec{2 & 1 & 0 \\ 0 & 2 & 0 \\ 0 & 0 & 2}; \quad 
    \vec{J_2} = \myvec{-7 & 0 \\ 0 & -7} \label{eq:solutions/7/3/4/eq:3_1}
\end{align}
Where $\vec{J_1}$ and $\vec{J_1}$ are the Jordan blocks corresponding to $\lambda_1 = 2$ and $\lambda_1 = -7$ respectively. The Jordan form for $\vec{A}$ can be written as,
\begin{align}
    \vec{J} = \myvec{\vec{J_1} & 0 \\ 0 & \vec{J_2}} = \myvec{2 & 1 & 0 & 0 & 0 \\ 0 & 2 & 0 & 0 & 0 \\ 0 & 0 & 2 & 0 & 0 \\ 0 & 0 & 0 & -7 & 0 \\ 0 & 0 & 0 & 0 & -7} 
\end{align}
{\em Inference}
An $n\times n$ matrix with with $\lambda$ as diagonal elements, ones on the super diagonal and zeroes in all other entries is nilpotent with minimal polynomial $(A-\lambda I)^n$
{\em Example}
Let,
\begin{align}
    \vec{A} = \myvec{0 & 1 & 0 \\ 0 & 0 & 1 \\ 0 & 0 & 0} \label{eq:solutions/7/3/4/5_1}
\end{align}
\eqref{eq:solutions/7/3/4/5_1} is nilpotent for minimal polynomial $A^3$
%
