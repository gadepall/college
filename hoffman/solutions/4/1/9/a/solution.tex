The transformation T does integral of a polynomial. Table ref{eq:solutions/4/1/9/a/table:2} provides proof that the transformation T is a linear operator and non-singular. Table ref{eq:solutions/4/1/9/a/table:3} provides proof that T is not invertible, however there exists a left inverse. The parameters used in the proof are listed in the table ref{eq:solutions/4/1/9/a/table:1}.
\begin{table*}[ht!]
\begin{center}
\begin{tabular}{|l|l|}
\hline
\textbf{PARAMETER} & \textbf{DESCRIPTION}\\[0.5ex]
\hline
$\mathbb{F}$ & Field of complex numbers\\[0.5ex]
\hline
$\mathbb{F}^{\infty}$ & Vector space defined on the field $\mathbb{F}$\\[0.5ex]
\hline
$\mathbb{F}\sbrak{x}$ & Subspace of $\mathbb{F}^{\infty}$ spanned by $\cbrak{1,x,x^2,x^3,\hdots}$\\[0.5ex] 
\hline
$T\colon \mathbb{F}\sbrak{x} \rightarrow \mathbb{F}\sbrak{x}$ & Transformation T\\[0.5ex] 
\hline
$\mathit{f}=\sum \limits_{i=0}^n c_i x^i$ & Polynomial $\mathit{f} \in \mathbb{F}\sbrak{x}$\\[0.5ex] 
\hline
$\mathit{f^\prime}=\sum \limits_{i=0}^n c^\prime_i x^i$ & Polynomial $\mathit{f^\prime} \in \mathbb{F}\sbrak{x}$\\[0.5ex] 
\hline
$c_i,c^\prime_i \; \forall i=0,2,\hdots n$ & Scalars in $\mathbb{F}$ and coefficients of polynomials $\mathit{f}$ and $\mathit{f^\prime}$\\[0.5ex] 
\hline
$T\brak{\mathit{f}}=\mathit{g}=\sum \limits_{i=0}^n \frac{c_i}{i+1} x^{i+1}$ & Transformed polynomial $\mathit{g}\in \mathbb{F}\sbrak{x}$
\\[0.5ex] 
\hline
$T\brak{\mathit{f^\prime}}=\mathit{g^\prime}=\sum \limits_{i=0}^n \frac{c_i^\prime}{i+1} x^{i+1}$ & Transformed polynomial $\mathit{g^\prime}\in \mathbb{F}\sbrak{x}$
\\[0.5ex] 
\hline
$\vec{M}_T$ & Transformation matrix for T\\[0.5ex] 
\hline
$N\brak{T}$ & Null Space of T\\[0.5ex] 
\hline
\end{tabular}
\caption{Parameters}
\label{eq:solutions/4/1/9/a/table:1}
\end{center}

\end{table*}

\begin{table*}[ht!]
\begin{center}
\begin{tabular}{|l|l|}
\hline
\textbf{Statement} & \textbf{Derivation} \\[0.5ex]
\hline
$\mathit{f}=\sum \limits_{i=0}^n c_i x^i$ & $\mathit{f}=\myvec{c_0&c_1&c_2&\cdots&c_n}^T_{\brak{n+1}\times1}$
\\ [0.5ex] 
\hline
$T\sbrak{\mathit{f}}=\vec{M}_T\mathit{f}$ & $T\sbrak{\mathit{f}}=
\myvec{0&0&0&\cdots&0\\1&0&0&\cdots&0\\0&\frac{1}{2}&0&\cdots&0\\0&0&\frac{1}{3}&\cdots&0\\\vdots&\vdots&\vdots&\cdots&\vdots\\0&0&0&\cdots&\frac{1}{n+1}}_{\brak{n+2}\times\brak{n+1}}
\myvec{c_0\\c_1\\c_2\\\vdots\\c_n}_{\brak{n+1}\times1}
=\myvec{0\\c_0\\\frac{c_1}{2}\\\frac{c_2}{3}\\\vdots\\\frac{c_n}{n+1}}_{\brak{n+2}\times1}=\mathit{g}\in \mathbb{F}\sbrak{x}$
\\ [0.5ex] 
\hline
T is a linear operator & 
\parbox{10cm}{\begin{align}
T\sbrak{\alpha\mathit{f}+\mathit{f^\prime}}\\
=\vec{M}_T\brak{\alpha\mathit{f}+\mathit{f^\prime}}\\
=\myvec{0&0&0&\cdots&0\\1&0&0&\cdots&0\\0&\frac{1}{2}&0&\cdots&0\\0&0&\frac{1}{3}&\cdots&0\\\vdots&\vdots&\vdots&\cdots&\vdots\\0&0&0&\cdots&\frac{1}{n+1}}
\myvec{\alpha c_0+c_0^\prime\\\alpha c_1+c_1^\prime\\\alpha c_2+c_2^\prime\\\vdots\\\alpha c_n+c_n^\prime}\\
=\myvec{0\\\alpha c_0+c_0^\prime\\\frac{\alpha c_1+c_1^\prime}{2}\\\frac{\alpha c_2+c_2^\prime}{3}\\\vdots\\\frac{\alpha c_n+c_n^\prime}{n+1}}
=\alpha\myvec{0\\c_0\\\frac{c_1}{2}\\\frac{c_2}{3}\\\vdots\\\frac{c_n}{n+1}}+\myvec{0\\c_0^\prime\\\frac{c_1^\prime}{2}\\\frac{c_2^\prime}{3}\\\vdots\\\frac{c_n^\prime}{n+1}}\\
=\alpha T\sbrak{\mathit{f}}+T\sbrak{\mathit{f^\prime}}\\
=\alpha\mathit{g}+\mathit{g^\prime}\\
\therefore T\sbrak{\alpha\mathit{f}+\mathit{f^\prime}}=\alpha T\sbrak{\mathit{f}}+T\sbrak{\mathit{f^\prime}}
\end{align}}
\\ [0.5ex] 
\hline
T is non-singular & \parbox{10cm}{\begin{align}
T\sbrak{\mathit{f}}=0\\
\implies \vec{M}_T\mathit{f}=\vec{0}
\implies \mathit{f}=0 \because \vec{M}_T \ne \vec{0}\\
\implies N\brak{T}=\cbrak{0}
\end{align}}
\\ [0.5ex] 
\hline
\end{tabular}
\caption{Proof for Non-Singular and linear transformation T}
\label{eq:solutions/4/1/9/a/table:2}
\end{center}

\end{table*}

\begin{table*}[ht!]
\begin{center}
\begin{tabular}{|p{4.7cm}|p{10cm}|}
\hline
\textbf{Statement} & \textbf{Derivation} \\[0.5ex]
\hline
T is not invertible & As $\vec{M}_T$ is a non-square matrix with dimensions $\brak{n+2}\times\brak{n+1}$, the transformation T is not invertible
\\ [0.5ex] 
\hline
$\vec{M}_D$ is left inverse of $\vec{M}_T$ 
& \parbox{10cm}{\begin{align}
\vec{M}_D\vec{M}_T=\vec{I}_{n+1}\\
\implies \vec{M}_D=\myvec{0&1&0&0&\cdots&0\\0&0&2&0&\cdots&0\\0&0&0&3&\cdots&0\\\vdots&\vdots&\vdots&\vdots&\cdots&\vdots\\0&0&0&0&\cdots&n+1}
\end{align}}
\\ [0.5ex] 
\hline
\end{tabular}
\caption{Non-Invertibility of transformation T}
\label{eq:solutions/4/1/9/a/table:3}
\end{center}
\end{table*}

