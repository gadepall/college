The given transformation can be written as,
\begin{align}
T(\vec{x})=&\vec{A}\vec{x}\\
=&\myvec{1&0&0\\0&0&1\\0&-2&-1}\vec{x}
\end{align}
Hence,
\begin{align}
\vec{A}=\myvec{1&0&0\\0&0&1\\0&-2&-1}
\end{align}
Now the characteristic equation of $\vec{A}$ is given by,
\begin{align}
\det{(\vec{A} - \lambda \vec{I})} =&0\\
=&\myvec{1-\lambda&0&0\\0&-\lambda&1\\0&-2&-1-\lambda}\\
&\implies(1-\lambda)(\lambda^2 +\lambda+2) =0
\end{align}
Now, simplifying the above equation,
\begin{align}
(1-\lambda)(\lambda^2 +\lambda+2) =0\\
\lambda^2 +\lambda+2-\lambda^3-\lambda^2-2\lambda=0\\
\lambda^3=2-\lambda
\end{align}
 Now using Cayley Hamilton Theorem we get,
\begin{align}
\vec{A}^3=2\vec{I}-\vec{A}
\end{align}
Hence the polynomial $f(\vec{A})$ can be written using the characteristic function of $\vec{A}$ as follows,
\begin{align}
f(\vec{A}) &= -\vec{A}^3 + 2\vec{I}\\
&= 2\vec{I}-\vec{A}+2\vec{I}\\
&= \vec{A}
\end{align}
Hence,
\begin{align}
f(T)(\vec{x})&=\vec{A}\vec{x}\\
&=\myvec{1&0&0\\0&0&1\\0&-2&-1}\vec{x}
\end{align}

