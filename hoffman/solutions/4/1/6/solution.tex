See Tables 
\ref{eq:solutions/4/1/6/table0},
\ref{eq:solutions/4/1/6/table1} and 
\ref{eq:solutions/4/1/6/table2}

\begin{table}[ht!]
\begin{center}
\begin{tabular}{|c|c|}
\hline
& \\
Given & $\vec{S}$ be a set of non-zero\\
& polynomial over a field $F$\\
& \\
& No two elements of $\vec{S}$ have\\
& the same degree\\
& \\
\hline
& \\
To prove & $\vec{S}$ is an independent set\\
& in $\vec{F[x]}$\\
& \\
\hline
& \\
Linear &\\
Independency  & Let $f_1$,$f_2$,...,$f_n$ are the\\
& polynomials and they will be \\
& linearly independent if\\
& $a_1f_1+a_2f_2+...+a_nf_n=\theta$ \\
& for $a_1=a_2=....=a_n=0$ \\
& where $a_1$,$a_2$,...,$a_n$ are \\
& scalars from field $F$\\
\hline
\end{tabular}
\end{center}
\caption{}
\label{eq:solutions/4/1/6/table0}
\end{table}


\begin{table}[ht!]
\begin{center}
\begin{tabular}{|c|c|}
\hline
& \\
Proof & Let the degrees of $f_1$,$f_2$,...,$f_n$ are\\
& $d_1$,$d_2$,...,$d_n$ respectively such that\\
& the degree of $f_i$ = $d_i \neq d_j$\\
& for j=1,2,....,n and $i \neq j$\\
& and $d_1 < d_2 < ....< d_n$\\
& so $d_n$ is the largest degree\\
\hline
& Now, let $a_1f_1+a_2f_2+...+a_nf_n=\theta$\\
& \\
& where $f_1 = \sum_{i=0}^{d_1}k_{1i}x^i$\\
& \\
& $f_2 = \sum_{i=0}^{d_2}k_{2i}x^i$\\
& $\textbf{or,}f_2 = \sum_{i=0}^{d_1}k_{2i}x^i+ \sum_{i=d_1+1}^{d_2}k_{2i}x^i$\\
& \\
& Similarly, $f_{n-1}=\sum_{i=0}^{d_{n-1}}k_{(n-1)i}x^i$\\
& $\implies f_{n-1}= \sum_{i=0}^{d_{n-2}}k_{(n-1)i}x^i+ \sum_{i=d_{n-2}+1}^{d_{n-1}}k_{(n-1)i}x^i$\\
& \\
& and $f_n = \sum_{i=0}^{d_{n-1}}k_{ni}x^i+ \sum_{i=d_{n-1}+1}^{d_{n}}k_{ni}x^i$\\
\hline
& Now,\\
& $a_1f_1+a_2f_2+...+a_nf_n$\\
& $=a_1\sum_{i=0}^{d_1}k_{1i}x^i + a_2(\sum_{i=0}^{d_1}k_{2i}x^i+ \sum_{i=d_1+1}^{d_2}k_{2i}x^i)$\\
& $+ ...+a_n(\sum_{i=0}^{d_1}k_{ni}x^i+ \sum_{i=d_1+1}^{d_2}k_{ni}x^i+..$\\
& $..+\sum_{i=d_{n-1}+1}^{d_{n}}k_{ni}x^i)$\\
& \\
& $=\sum_{i=0}^{d_1}(a_1k_{1i}+a_2k_{2i}+..+a_nk_{ni})x^i$\\
& $+\sum_{i=d_1+1}^{d_2}(a_2k_{2i}+..+a_nk_{ni})x^i$\\
& $+..+\sum_{i=d_{n-1}+1}^{d_{n}}a_nk_{ni}x^i$\\
& \\
& Now, as $a_1f_1+a_2f_2+...+a_nf_n=\theta$ \\
& for $d_{n-1}+1\leq i \leq d_n$, $k_{ni}\neq 0$\\
& so $a_n$ must be 0\\
\hline
& Now, discarding $a_n$ associated term, we get\\
& $\sum_{i=0}^{d_1}(a_1k_{1i}+a_2k_{2i}+..+a_nk_{ni})x^i$\\
& $+\sum_{i=d_1+1}^{d_2}(a_2k_{2i}+..+a_nk_{ni})x^i$\\
& $+..+\sum_{i=d_{n-2}+1}^{d_{n-1}}a_{n-1}k_{(n-1)i}x^i =0$\\
& so, for $d_{n-2}+1\leq i \leq d_{n-1}$, $k_{(n-1)i}\neq 0$\\
& $\implies a_{n-1}=0$ \\
%& In this way, it can be proved that\\
%& $a_1=a_2=....=a_n=0$ for\\
%& $a_1f_1+a_2f_2+...+a_nf_n=\theta$\\
%& $\implies$ $f_1$,$f_2$,....,$f_n$ are\\
%& linearly independent\\
& \\
\hline
\end{tabular}
\end{center}
\caption{}
\label{eq:solutions/4/1/6/table1}
%
\end{table}

\begin{table}[ht!]
\begin{center}
\begin{tabular}{|c|c|}
\hline
& \\
Proof & \\
& Similarly, for $d_1+1 \leq i \leq d_2$\\
& $\sum_{i=0}^{d_1}(a_1k_{1i}+a_2k_{2i})x^i + \sum_{i=d_1+1}^{d_2}a_2k_{2i}x^i$\\
& $\implies a_2=0$ as $k_{2i} \neq 0$\\
& $\implies \sum_{i=0}^{d_1}a_1k_{1i}x^i=0$\\
& $\implies a_1=0$ as $k_{1i} \neq 0$\\
\hline
& In this way, it can be proved that\\
& $a_1=a_2=....=a_{n-1}=a_n=0$ for\\
& $a_1f_1+a_2f_2+...+a_nf_n=\theta$\\
& $\implies$ $f_1$,$f_2$,....,$f_n$ are\\
& linearly independent\\
& \\
\hline
\end{tabular}
\end{center}
\caption{}
\label{eq:solutions/4/1/6/table2}
%
\end{table}


Hence, it is proved that $\vec{S}$ is an independent set in $\vec{F[x]}$.
