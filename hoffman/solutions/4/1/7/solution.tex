
 Let consider we have a set $S$ such that,
 \begin{align}
 S = \left \{    1, ax+b, \brak{ax+b}^2, \brak{ax+b}^3,  \dots  \right \}
 \end{align}
 And let $\left \langle S \right \rangle$ be the subspace, that is spanned by $S$.
 \begin{align}
 \intertext{Since}
  1 \in S\\
 \intertext{and}
  ax+b \in S,\\
  \implies  b.1 + \frac{1}{a}\brak{b +ax} \in \left \langle S \right \rangle \\
  \intertext{and hence, it follows}
   \implies x \in \left \langle S \right \rangle 
 \end{align}
% 
 Now to prove
 \begin{align}
  x^2 \in \left \langle S \right \rangle \\
 \intertext{let consider another element form $S$ which is}
    \brak{ax+b}^2
\end{align}
Subtracting $1.a^2 +2.a.b.x$ from $\brak{ax+b}^2$
 \begin{align}
 \implies \brak{ax+b}^2 - a^2 - 2.a.b.x = a^2.x^2 \label{eq:solutions/4/1/7/2.2}\\
 \implies a^2.x^2 \in \left \langle S \right \rangle\\
 \implies   \frac{1}{a^2} . a^2. x^2 \in S.\\
 \implies x^2 \in  \left \langle S \right \rangle .
 \end{align}
 Now, Thus  Hence using this concept with higher degree we can prove that,
 \begin{align}
   x^n \in  \left \langle S \right \rangle, \forall  n
  \end{align}
Consider,
 \begin{align}
 S' =   \left \{    1, x, x^2, x^3,  \dots  \right \} \label{eq:solutions/4/1/7/2.3}
   \end{align}
 Hence we can say that, \eqref{eq:solutions/4/1/7/2.3} span the space of all polynomials which form with the help of
 \begin{align} 
  \brak{ax + b}^n
  \end{align}
   
Hence we conclude   that $S$ spans the space of all polynomials.
We can summarize our procedure step by step using Table \ref{eq:solutions/4/1/7/t1}.


\onecolumn
\begin{longtable}{|p{2cm}|p{5cm}|p{6cm}|}
%{ \small
%\begin{longtable}{|c|c|c|}
\endfirsthead
\endhead
	   \hline
       \centering
		\textbf{ Sr. No} &\textbf{Description} &  \textbf{ Mathematical  representation}  \\ 
		\hline
			\centering
					1. & Consider  a set $S$ & $S = \left \{    1, ax+b, \dots  \right \}$ \\
					\hline 
					\centering
					2. & Provide a proof that subset $S$ span the subspace $\left \langle S \right \rangle$  & Since
					$1 \in S$
					and
					$ax+b \in S,$ \newline
					$\implies  b.1 + \frac{1}{a}\brak{b +ax} \in \left \langle S \right \rangle$  \newline
					  $\implies x \in \left \langle S \right \rangle$  Given element are $\in S $  \\
					\hline
					\centering
					3. & Repeat  step 2 for the higher degree of polynomial also lie in the subspace and the also lie in the subset $S$. & Since $\brak{ax+b}^2 \in S $  \newline $ \implies \brak{ax+b}^2 - a^2 - 2.a.b.x  = a^2.x^2 \newline 
					\implies a^2.x^2 \in \left \langle S \right \rangle \newline
					\implies   \frac{1}{a^2} . a^2. x^2 \in S.\newline
					\implies x^2 \in  \left \langle S \right \rangle$
					 Given element are $\in S $ \\
					\hline
					\centering
					4. &After providing proof for all element $\in S$ find the basis  . & $ S'=\left \{    1, x, x^2, x^3,  \dots  \right \} $ \\
					\hline
					\centering
				5. & Show the element $\in S'$ are  able to form all element $S$ over $\mathbb{F}$ . & Hence $S$ form basis of $\mathbb{F}$ \\
	      		\hline			      			      		
\caption{Step of solution for given problem}
\label{eq:solutions/4/1/7/t1}
\end{longtable}
%}









