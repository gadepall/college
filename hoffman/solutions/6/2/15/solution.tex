See Table \ref{eq:solutions/6/2/15/table:1}

\onecolumn
\begin{longtable}{|c|c|}
\hline
\multirow{3}{*}{} & \\
$\textbf{Given}$ & $\vec{V}$ is the space of $n \times n$ matrices over $F$\\
& \\
& $\vec{A}$ is a fixed $n \times n$ matrix over $F$\\
& \\
& $\vec{T}$ be the linear operator on $\vec{V}$ such that\\
& $\vec{T}(\vec{B})= \vec{AB}$\\
& \\
\hline
\multirow{3}{*}{} & \\
\textbf{To prove} & $\vec{A}$ and $\vec{T}$ have the same characteristic values\\
& \\
\hline
\multirow{3}{*}{} & \\
\textbf{Theorem} & Let $\lambda$ be a characteristic value of $\vec{T}$\\
& and $\lambda \in F$ and $\vec{v}$ is the corresponding characteristic\\
& vector which is a $n \times n$ matrix, then if $\vec{T}$ is a \\
& linear operator on a finite dimensional space $\vec{V}$,\\
&  it must be $\det{(T-\lambda I)} = 0$ and \\
& for $\vec{(T-\lambda I)v} = \vec{0} $, $\vec{v} \neq \vec{0}$\\
& \\
\hline
\multirow{3}{*}{} & \\
\textbf{Proof} & As per the problem statement, $\vec{Tv} = \vec{\lambda v}$ \\
& \textbf{or,}  $\vec{Av} = \vec{\lambda v}$, as $\vec{Tv} = \vec{Av}$\\
& \textbf{or,}  $\vec{(A-\lambda I)v} = \vec{0}$\\
& From here 2 cases can be arrived. \\
& \\
\hline
Case 1: & $\det{(\vec{(A-\lambda I)})} = 0$, where $\vec{v} \neq \vec{0}$\\
& $\implies$ $\lambda$ is characteristic value of $\vec{A}$.\\
& \\
\hline
Case 2: & $\det{(\vec{(A-\lambda I)})} \neq 0 $\\
& $\implies$ $\vec{(A-\lambda I)}$ is invertible and $\vec{v} = \vec{0}$\\
& so, for $\vec{(T-\lambda I)v} = \vec{0}$ and \\
& $\vec{(T-\lambda I)} \neq \vec{0}$\\
& $\implies$ $\vec{v}$ is not a charcteristic vector of $\vec{T}$\\
& which is a contradiction. So, case 2 is not possible.\\
& \\
\hline
Conclusion & So, from the above 2 cases and from the theorem,\\
&  it can be concluded that $\vec{A}$ and $\vec{T}$ \\
& have the same  characteristic values\\
\hline
\caption{$\textbf{Solution summary}$}
\label{eq:solutions/6/2/15/table:1}
\end{longtable}
%\renewcommand{\theequation}{\theenumi}
%\begin{enumerate}[label=\thesection.\arabic*.,ref=\thesection.\theenumi]
%\numberwithin{equation}{enumi}
%\item Verification of the above problem using python code.\\
%\solution The  following Python code verifies the above solution.
%\begin{lstlisting}
%codes/multiplication_test.py
%\end{lstlisting}
%%%
%\end{enumerate}

\twocolumn


