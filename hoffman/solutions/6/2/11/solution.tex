See Table \ref{eq:solutions/6/2/11/table:1}
\onecolumn
\begin{longtable}{|l|l|}
\hline
\multirow{3}{*}{} & \\
Statement&Solution\\
\hline
&\parbox{6cm}{\begin{align}
\mbox{Let }\vec{N}&=\myvec{a&b\\c&d}\\
\mbox{Since }\vec{N}^2&=0
\end{align}}\\
&If $\myvec{a\\c},\myvec{b\\d}$ are linearly independent then $\vec{N}$ is diagonalizable to $\myvec{0&0\\0&0}$.\\
&\parbox{6cm}{\begin{align}
\mbox{If }\vec{P}\vec{N}\vec{P}^{-1}=0\\
\mbox{then }\vec{N}=\vec{P}^{-1}\vec{0}\vec{P}=0
\end{align}}\\
Proof that&So in this case $\vec{N}$ itself is the zero matrix.\\
$\vec{N}=0$&This contradicts the assumption that $\myvec{a\\c},\myvec{b\\d}$ are linearly independent.\\
&$\therefore$ we can assume that $\myvec{a\\c},\myvec{b\\d}$ are linearly dependent if both are\\
&equal to the zero vector\\
&\parbox{6cm}
{\begin{align}
   \mbox{then } \vec{N} &= 0.
\end{align}}\\
%\hline
%\pagebreak
\hline
&\\
&Therefore we can assume at least one vector is non-zero.\\
Assuming $\myvec{b\\d}$ as&Therefore
$\vec{N}=\myvec{a&0\\c&0}$\\
the zero vector&\\
&\parbox{6cm}{\begin{align}
\mbox{So }\vec{N}^2&=0\\
\implies a^2&=0\\
\therefore a&=0\\
\mbox{Thus }\vec{N}&=\myvec{a&0\\c&0}
\end{align}}\\
&In this case $\vec{N}$ is similar to $\vec{N}=\myvec{0&0\\1&0}$ via the matrix $\vec{P}=\myvec{c&0\\0&1}$\\
&\\
\hline
&\\
Assuming $\myvec{a\\c}$ as&Therefore
$\vec{N}=\myvec{0&b\\0&d}$\\
the zero vector&\\
&\parbox{6cm}{\begin{align}
\mbox{Then }\vec{N}^2&=0\\
\implies d^2&=0\\
\therefore d&=0\\
\mbox{Thus }\vec{N}&=\myvec{0&b\\0&0}
\end{align}}\\
&In this case $\vec{N}$ is similar to $\vec{N}=\myvec{0&0\\b&0}$ via the matrix $\vec{P}=\myvec{0&1\\1&0}$,\\
&which is similar to $\myvec{0&0\\1&0}$ as above.\\
&\\
\hline
&\\
Hence& we can assume neither $\myvec{a\\c}$ or $\myvec{b\\d}$ is the zero vector.\\
&\\
%\hline
%\pagebreak
\hline
&\\
Consequences of &Since they are linearly dependent we can assume,\\
linear&\\
independence&\\
&\parbox{6cm}{\begin{align}
\myvec{b\\d}&=x\myvec{a\\c}\\
\therefore \vec{N}&=\myvec{a&ax\\c&cx}\\
\therefore \vec{N}^2&=0\\
\implies a(a+cx)&=0\\
c(a+cx)&=0\\
ax(a+cx)&=0\\
cx(a+cx)&=0
\end{align}}\\
\hline
&\\
Proof that $\vec{N}$ is&We know that at least one of a or c is not zero.\\
similar over $\mathbb{C}$ to&If a = 0 then c $\neq$ 0, it must be that x = 0.\\
$\myvec{0&0\\1&0}$&So in this case $\vec{N}=\myvec{0&0\\c&0}$ which is similar to $\myvec{0&0\\1&0}$ as before.\\
&\\
&\parbox{6cm}{\begin{align}
\mbox{If } a &\neq0\\
\mbox{then }x &\neq0\\
\mbox{else }a(a+cx)&=0\\
\implies a&=0\\
\mbox{Thus }a+cx&=0\\
\mbox{Hence }\vec{N}&=\myvec{a&ax\\ \frac{-a}{x}&-a}
\end{align}}\\
 &This is similar to $\myvec{a&a\\-a&-a}$ via $\vec{P}=\myvec{\sqrt{x}&0\\0&\frac{1}{\sqrt{x}}}$.\\
 &And $\myvec{a&a\\-a&-a}$ is similar to $\myvec{0&0\\-a&0}$ via $\vec{P}=\myvec{-1&-1\\1&0}$\\
 &And this finally is similar to $\myvec{0&0\\1&0}$ as before.\\
 &\\
\hline
&\\
Conclusion &Thus either $\vec{N}$ = 0 or $\vec{N}$ is similar over $\mathbb{C}$ to $\myvec{0&0\\1&0}$.\\
&\\
\hline
\caption{Solution summary}
\label{eq:solutions/6/2/11/table:1}
\end{longtable}
