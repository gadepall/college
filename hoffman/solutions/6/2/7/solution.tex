See Table     \ref{eq:solutions/6/2/7/table:2}

\onecolumn
\begin{longtable}{|l|l|}
	\hline
	\multirow{3}{*}{Diagonalizable} 
	& \\
	& A linear operator $\vec{T}$ on a finite-dimensional vector space $\vec{V}$ is diagonalizable if and\\
    &only if there exists an basis of $\vec{V}$, consisting of eigen vectors of $\vec{T}$ \\ 
	&\\
	\hline
	\multirow{3}{*}{Theorem}
	& \\
&If $\vec{v}_1,\vec{v}_2,\dots,\vec{v}_k$ are eigenvectors of a linear operator $\vec{T}$ with distinct eigen\\
& values $\lambda_1,\lambda_2,\dots,\lambda_k$,then $\vec{v}_1,\vec{v}_2,\dots,\vec{v}_k$ are linearly independent.\\
&\\
&Let $S_k=\cbrak{\vec{v}_1,\vec{v}_2,\dots,\vec{v}_k}$.Let $P(k): S_k$ is linearly independent.$S_1$ is linearly independent.\\
&So,$P(1)$ holds.Assume $P(k)$ holds for $1 \le k \le n$.Therefore,$S_k$ is linearly independent.\\
&\parbox{10cm}{\begin{align}
\text{Let} \sum_{i=1}^{k+1} a_i\vec{v}_i = 0 \label{eq:solutions/6/2/7/1}\\
\text{Applying $\vec{T}$ on both sides , we get} \nonumber\\
\implies \vec{T}(\sum_{i=1}^{k+1} a_i\vec{v}_i)=0\\
\implies \sum_{i=1}^{k+1}a_i\vec{T}(\vec{v}_i)=0\\
\implies \sum_{i=1}^{k+1}a_i\lambda_i\vec{v}_i=0\\
\implies\sum_{i=1}^{k}a_i\lambda_i\vec{v}_i+a_{k+1}\lambda_{k+1}\vec{v}_{k+1}=0
\label{eq:solutions/6/2/7/2}\\
\text{Multiplying \eqref{eq:solutions/6/2/7/1} by $\lambda_{k+1}$,we get} \nonumber\\
\lambda_{k+1}(\sum_{i=1}^{k+1} a_i\vec{v}_i)=0\\
\implies \sum_{i=1}^{k+1}a_i\lambda_{k+1}\vec{v}_i=0
\end{align}}\\
\hline
&\parbox{10cm}{\begin{align}
\implies\sum_{i=1}^{k}a_i\lambda_{k+1}\vec{v}_i+a_{k+1}\lambda_{k+1}\vec{v}_{k+1}=0\label{eq:solutions/6/2/7/3}\\
\text{Subtracting \eqref{eq:solutions/6/2/7/2} and \eqref{eq:solutions/6/2/7/3},we get} \nonumber \\
 \sum_{i=1}^{k}a_i(\lambda_i-\lambda_{k+1})\vec{v}_i=0\\
 \text{As $\lambda_i$ are distinct $\forall i \le k, a_i = 0$}\\
 \text{Substituting this in \eqref{eq:solutions/6/2/7/1}} \nonumber \\
 \sum_{i=1}^{k+1} a_i\vec{v}_i = 0\\
 \implies a_{k+1}\vec{v}_{k+1} = 0
\end{align}}\\
    & As $\vec{v}_{k+1} \neq 0 \implies a_{k+1}=0$ \\
    &Since $\forall i \le k+1, a_i = 0.S_{k+1}$ is linearly independent\\
    & By principle of mathematic induction, $S_n$ is linearly independent.\\
    & \\
    \hline
    \caption{Definitions and theorem used}
    \label{eq:solutions/6/2/7/table:1}
\end{longtable}
\begin{longtable}{|l|l|}
	\hline
	\multirow{3}{*}{Given} & \\
	& $\vec{T}$ has an n distinct characteristic values and dim$(\vec{V})$ = n\\
    & \\
    \hline
	\multirow{3}{*}{$\vec{T}$ is diagonalizable}
	& \\
	& Let $\lambda_1,\lambda_2,\dots,\lambda_n$ be distinct eigen values of $\vec{T}$ and let $\vec{v}_1,\vec{v}_2,\dots,\vec{v}_n$ be the eigen\\
	& vectors of $\vec{T}$..From above results we can state that \cbrak{\vec{v}_1,\vec{v}_2,\dots,\vec{v}_n} is linearly\\
	&independent.And also given that dim$(\vec{V})$ = n .So,this set forms a basis of $\vec{V}$.\\
	&\cbrak{\vec{v}_1,\vec{v}_2,\dots,\vec{v}_n} is a basis for $\vec{V}$ consisting of eigen vectors of $\vec{T}$.\\
    &So, $\vec{T}$ is diagonalizable.\\
	\hline
	\caption{Solution}
    \label{eq:solutions/6/2/7/table:2}
\end{longtable}
