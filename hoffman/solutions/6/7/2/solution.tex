See Table \ref{eq:solutions/6/7/2/tab:sol}

\onecolumn
\begin{longtable}{|p{5cm}|p{13cm}|}
\hline
\textbf{Statement} &\textbf{Solution}\\
\hline
& \\ 
Given
& \parbox{12cm}{$T$ be the linear operator on $R^2$
the matrix of which in the standard ordered basis is $\myvec{2&1\\0&2}$ \\
$W_1$ be the subspace of $R^2$ spanned by the vector $\epsilon_1=(1,0)$.\\}\\
\hline
& \\ 
To Proove
& \parbox{12cm}{\begin{enumerate}
\item {$W_1$ is invariant under $T$.}
\item{There is no subspace $W_2$ which is invariant under $T$ and is complementary to $W_1$:
$R^2=W_1\oplus W_2$}
\item{Compare with exercise 1 of section 6.5.}
\end{enumerate}}
\\
& \\
\hline
Proof (a) &
\parbox{12cm}{\begin{align}
    \vec{A}=\myvec{2&1\\0&2}\\
    \lvert A -\lambda I \rvert=0\\
      \implies \myvec{2-\lambda & 1\\0& 2- \lambda} \\
    = (2-\lambda)^2=0\\
    \therefore \lambda =2
    \end{align}
    for $\lambda = 2$ , the corresponding vector is
    \begin{align}
    (\vec{A}-\lambda I)X=0\\
    \myvec{0&1\\0&0}X=0\\
    \therefore X=\myvec{1\\0}
    \end{align}
    Hence, $W_1$ be the subspace of $R^2$ spanned by the vector $\epsilon_1=(1,0)$ is invariant under $T$.\\
}
\\
\hline
& \\
Proof (b) &
\parbox{12cm}{ Corresponding to $\lambda=2$, \\ Among, two eigen vectors only one is independent and other one is dependent.\\
 Thus, $P^-1$ does not exist and $A$ can not be diagonalized.\\ Hence, there is no subspace $W_2$ which is invariant under $T$ and is complementary to $W_1$:\\
$R^2=W_1\oplus W_2$\\}\\
\hline
& \\
Observation & 
\parbox{12cm}{In exercise 1 of section 6.5, for $2\times 2$ matrix there is 2 distinct characteristic value, corresponding to which there is a eigen vector. Hence, $P^{-1}$ exists. \\
$\therefore$ the given matrix is diagonalizable.\\}\\
\hline
\caption{Solution}
\label{eq:solutions/6/7/2/tab:sol}
\end{longtable}
\twocolumn
