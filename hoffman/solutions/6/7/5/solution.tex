See Tables     \ref{eq:solutions/6/7/5/table:2}
and     \ref{eq:solutions/6/7/5/table:2}
 
\onecolumn
\begin{longtable}{|c|l|}
    \hline
    \multirow{3}{*}{Diagonalizable Operator} 
	& \\
	& For a linear operator $\vec{T}\colon \vec{V}\longrightarrow \vec{V}$, $\vec{T}$ is a diagonalizable operator if \\
	& $\exists$ some basis for $\Vec{V}$ such that the matrix representing $\vec{T}$ is a diagonal matrix\\
	&i.e.\\
	& $\qquad\qquad\qquad \vec{T}\brak{\vec{X}}=\Vec{A}\Vec{X}$,\\
    &$\implies \vec{A}$ is a diagonalizable matrix\\
	&\\
	\hline
	\multirow{3}{*}{Characteristic Polynomial} 
	& \\
	& For an $n\times n$ matrix $\vec{A}$, characteristic polynomial is defined by,\\
	&\\
	& $\qquad\qquad\qquad p\brak{x}=\mydet{x\Vec{I}-\Vec{A}}$\\
	&\\
	\hline
	\multirow{3}{*}{Minimal Polynomial} 
	&\\
	& Minimal polynomial $m\brak{x}$ is the smallest factor of characteristic polynomial\\
	& $p\brak{x}$ such that,\\
	&\\
	& $\qquad \qquad \qquad m\brak{\vec{A}}=0$\\
	& \\
	& Every root of characteristic polynomial should be the root of minimal\\
	& polynomial\\
	&\\
    \hline
	\multirow{3}{*}{Lagrange Polynomials} 
	& \\
	& For a set of scalars $c_0, c_1,\dots, c_n \in \mathbb{F}$, Lagrange Polynomial is defined as:\\
	&\\
	&$\qquad\qquad\qquad p_j=\displaystyle\prod_{i\neq j}\frac{\brak{x-c_i}}{\brak{c_j-c_i}}$\\
	&\\
	\hline
	\multirow{3}{*}{Theorem} 
	& \\
	& If $\vec{T}$ is a diagonalizable linear operator on a finite dimensional space $\vec{V}$,\\
	&and if $c_1,c_2,\dots,c_k$ are distinct characterictic values of $\vec{T}$, then there exist\\
	&linear operators $\vec{E_1},\vec{E_2},\dots,\vec{E_k}$ such that:\\
	&\\
	& $\quad\brak{1}\quad\vec{T}=c_1\vec{E_1}+\dots+c_k\vec{E_k}$\\
	&$\quad\brak{2}\quad\vec{E_1}+\dots+\vec{E_k}=\vec{I}$\\
	&$\quad\brak{3}\quad\vec{E_i}\vec{E_j}=\vec{0},\quad i\neq j$\\
	&$\quad\brak{4}\quad\vec{E_i}=\vec{E_i}^2,\quad\brak{\vec{E_i}\text{ is a projection}}$\\
	&$\quad\brak{5}\quad\alpha=\vec{E_i}\alpha,\forall\alpha\in\vec{V}$\\
	&\\
	\hline
	\multirow{3}{*}{\shortstack{Relation between Lagrange\\ Polynomials and Projection}} 
	& \\
	& We have:\\
	&$\qquad\qquad \vec{T}=c_1\vec{E_1}+\dots+c_k\vec{E_k}$\\
	&\\
	&If $g$ is any polynomial over field $\mathbb{F}$,\\ 
	&$\qquad\qquad g\brak{\vec{T}}=g\brak{c_1}\vec{E_1}+\dots+g\brak{c_k}\vec{E_k}\qquad\qquad\qquad\dots\brak{1}$\\
	&\\
	&Now,\\
	&$\qquad\qquad p_j=\displaystyle\prod_{i\neq j}\frac{\brak{x-c_i}}{\brak{c_j-c_i}}$\\
	&$\qquad\implies p_j\brak{c_i}=\delta_{ij}$ (Kronecker Delta)$\qquad\qquad\qquad\quad\dots\brak{2}$\\
	&\\
	&From $\brak{1}$ and $\brak{2}$,\\
	&$\qquad\implies p_j\brak{\vec{T}}=\sum_{i=1}^k\delta_{ij}\vec{E}_i=\vec{E_j}$\\
	&\\
	&$\qquad\qquad\implies\boxed{p_j\brak{\vec{T}}=\vec{E_j}}$\\
	&\\
	&$\implies$ Projections $\vec{E_j}$ are polynomials in $\vec{T}$\\
	&\\
	\hline
    \caption{Definitions and results used}
    \label{eq:solutions/6/7/5/table:1}
\end{longtable}
\begin{longtable}{|c|l|}
    \hline
    \multirow{3}{*}{Given} 
	& \\
	&Matrix of $\vec{T}$ in the standard basis of $\mathbb{R}^3:$\\
	& $\qquad\qquad\qquad\vec{A}=\myvec{5&-6&-6\\-1&4&2\\3&-6&-4}$\\
	&\\
	\hline
	\multirow{3}{*}{Characteristic polynomial} 
	& \\
	& $p\brak{x}=\mydet{x\Vec{I}-\Vec{A}}$\\
	&\\
	& $\qquad = \mydet{x&-1&0\\-2&x+2&-2\\-2&3&x-2}$\\
	& $\qquad=x^3-5x^2+8x-4$\\
	&$\qquad=\brak{x-1}\brak{x-2}^2$\\
	&\\
	&$\implies \lambda=1, 2$\\
	&\\
	\hline
	\multirow{3}{*}{Minimal Polynomial} & \\
	& $p\brak{x}=\brak{x-1}\brak{x-2}^b,\quad b\leq2$\\
	&\\
	&$\brak{\vec{A}-\vec{I}}\brak{\vec{A}-2\vec{I}}=\myvec{4&-6&-6\\-1&3&2\\3&-6&-5}\myvec{3&-6&-6\\-1&2&2\\3&-6&-6}=\vec{0}$\\
	&\\
	&Therefore, $\brak{x-1}\brak{x-2}$ is the minimal polynomial.\\
	&\\
	\hline
	\multirow{3}{*}{Lagrange Polynomial} 
	& \\
	& $\qquad\qquad p_j=\displaystyle\prod_{i\neq j} \frac{\brak{x-c_i}}{\brak{c_j-c_i}}$\\
	&\\
	& For characteristic values $c_1=1,\quad c_2=2$,\\
	&\\
	& $\implies p_1=\frac{(x-1)}{2-1}, \qquad p_2=\frac{(x-2)}{1-2}$\\
	&\\
	& $\implies p_1= \brak{x-1}$, and\\
	&$\qquad p_2=\brak{2-x}$\\
	&\\
	\hline
	\multirow{3}{*}{Projection Maps} & \\
	& We know that,\\
	&$\qquad \qquad \qquad \vec{E_j}=p_j\brak{\vec{T
	}}$\\ 
	&\\
	&$\implies \vec{E_1}=\vec{A}-\vec{I}$ and $\vec{E_2}=2\vec{I}-\vec{A}$ \\
	&\\
	&$\implies \vec{E_1}=\myvec{4&-6&-6\\-1&3&2\\3&-6&-5}$, and \\
	&$\qquad \vec{E_2}=\myvec{-3&6&6\\1&-2&-2\\-3&6&6}$\\
	&\\
	\hline
\multirow{3}{*}{Verification} & \\
	& We have, \\
	&$\qquad\qquad\vec{E_1}=\vec{A}-\vec{I}$\\
	&$\qquad\implies \vec{A}-\vec{E_1}=\vec{I}\qquad\qquad\qquad\dots\brak{1}$\\
	&\\
	&$\qquad\qquad\vec{E_2}=2\vec{I}-\vec{A}$\\
	&From $\brak{1},$\\
	&$\qquad\implies\vec{E_2}=2\brak{\vec{A}-\vec{E_1}}-\vec{A}$\\
	&$\qquad\implies \boxed{\vec{A}=2\vec{E_1}+\vec{E_2}}$\\
	&\\
	& Also,\\
	&$\qquad \vec{E_1}=\myvec{4&-6&-6\\-1&3&2\\3&-6&-5},\qquad \vec{E_2}=\myvec{-3&6&6\\1&-2&-2\\-3&6&6}$\\
	&\\
	&$\implies \vec{E_1}+\vec{E_2}=\myvec{4&-6&-6\\-1&3&2\\3&-6&-5}+\myvec{-3&6&6\\1&-2&-2\\-3&6&6}$\\
	&$\qquad\qquad\qquad = \myvec{1&0&0\\0&1&0\\0&0&1}$\\
	&\\
	&$\qquad\implies\boxed{\vec{E_1}+\vec{E_2}=\vec{I}}$\\
	&\\
	&$\vec{E_1}\vec{E_2}=\myvec{4&-6&-6\\-1&3&2\\3&-6&-5}\myvec{-3&6&6\\1&-2&-2\\-3&6&6}$\\
	&$\qquad\quad=\myvec{0&0&0\\0&0&0\\0&0&0}$\\
	&\\
	&$\qquad\implies\boxed{\vec{E_1}\vec{E_2}=\vec{0}}$\\
	&\\
	\hline
	\caption{Using Lagrange Polynomials to represent $\vec{A}$}
    \label{eq:solutions/6/7/5/table:2}
\end{longtable}
\twocolumn
