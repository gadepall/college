See Tables \ref{eq:solutions/6/4/5/table:1} and \ref{eq:solutions/6/4/5/table:2}
%
\begin{table*}[ht!]
\centering
\begin{tabular}{|c|l|}
    \hline
	\multirow{3}{*}{Characteristic Polynomial} 
	& \\
	& For an $n\times n$ matrix $\vec{A}$, characteristic polynomial is defined by,\\
	&\\
	& $\qquad\qquad\qquad p\brak{x}=\mydet{x\Vec{I}-\Vec{A}}$\\
	&\\
	\hline
	\multirow{3}{*}{Cayley-Hamilton Theorem}
    &\\
    & If $p\brak{x}$ is the characteristic polynomial of an $n\times n$ matrix $\vec{A}$, then,\\
    &\\
    &$\qquad \qquad \qquad p\brak{\vec{A}}=\vec{0}$\\
    &\\
    \hline
	\multirow{3}{*}{Minimal Polynomial} 
	&\\
	& Minimal polynomial $m\brak{x}$ is the smallest factor of characteristic polynomial\\
	& $p\brak{x}$ such that,\\
	&\\
	& $\qquad \qquad \qquad m\brak{\vec{A}}=0$\\
	& \\
	& Every root of characteristic polynomial should be the root of minimal\\
	& polynomial\\
	&\\
    \hline
    \multirow{3}{*}{Theorem} 
	&\\
	& Let $\vec{V}$ be a finite-dimensional vector space over the field $\vec{F}$ \\
	& and let $\vec{T}$ be a linear operator on $\vec{V}$\\
	& Then $\vec{T}$ is diagonalizable if and only if the minimal polynomial for $\Vec{T}$\\
	& has the form p = $\brak{x-c_1}$...$\brak{x-c_k}$,where $c_1,c_2,...,c_k$ are distinct elements of $F$. \\
	&\\
    \hline
\end{tabular}
    \caption{Definitions}
\label{eq:solutions/6/4/5/table:1}
\end{table*}
\onecolumn
\begin{longtable}{|c|l|}
    \hline
	\multirow{3}{*}{Proof} 
	& \\
	& $A^2$ = $A$ (Given)\\
	& $\implies$ $A$ satisfies the polynomial $x^2-x$ = $x\brak{x-1}$\\
	&\\
	\hline
	\multirow{3}{*}{Minimal Polynomial} & \\
	& $p\brak{x}=x^a\brak{x-1}^b,\quad a\leq1, b\leq1$\\
	& \\
	& Minimal Polynomial $m_A\brak{x}$, of $A$ divides $x^2-x$, that is $m_A\brak{x}$=$x$\\
	& or $m_A\brak{x}$=$x-1$ or $m_A\brak{x}$=$x(x-1)$\\
	& If $m_A\brak{x}$=$x$, then $A$ = $0$.\\
	& If $m_A\brak{x}$=$x-1$, then $A$ = $I$.\\
	& If $m_A\brak{x}$=$x(x-1)$ \\
	& In all above three cases, the minimal polynomial factors into distinct linears.\\
	& So, it follows $A$ is diagonisable.\\
	&\\
	\hline
	\multirow{3}{*}{Conclusion} & \\
	& In all three cases, the minimal polynomial splits into\\ & distinct linear factors, so it follows that $\vec{A}$ is\\ 
	& diagonalisable according to the Theorem mentioned in definitions section.\\
	&\\
	\hline
	\caption{Illustration of Proof}
    \label{eq:solutions/6/4/5/table:2}
\end{longtable}
