	
	\begin{theorem}\label{eq:solutions/6/3/11/thm1}
		If $\mathbf{T}$ is a linear operator on a finite-dimensional space $\vec{V}$ and c is a characteristic value of $\mathbf{T}$, then the operator $(c\vec{I}-\mathbf{T})$ is singular, i.e.
		\begin{align}
			det\left(c\vec{I}-\mathbf{T}\right) = 0 \nonumber
		\end{align} 
	\end{theorem}
	\begin{theorem}\label{eq:solutions/6/3/11/thm2}
		If $\vec{A}$ and $\vec{B}$ are $n\times n$ matrix, then ($\vec{I}-\vec{AB}$) is invertible if and only if ($\vec{I}-\vec{BA}$) is invertible. 
	\end{theorem}
	
	
	To prove that $\vec{AB}$ and $\vec{BA}$ have the same characteristic values in $\mathbf{F}$, we can use theorem $\ref{eq:solutions/6/3/11/thm1}$ and show
	\begin{align}
%\label{eq:solutions/6/3/11/eq1}
		det\left(c\vec{I}-\mathbf{AB}\right) = det\left(c\vec{I}-\mathbf{BA}\right) = 0
	\end{align}
	
	That is, if $c$ is a characteristic value for $\vec{AB}$ then $c$ is a characteristic value for $\vec{BA}$ as well. Which can be also said as, if $c$ is not the characteristic value of $\vec{BA}$ then it won't be the characteristic value for $\vec{AB}$. \\
	
	Let's say $c$ is not a characteristic value of the matrix $\vec{BA}$, which would imply that 
	
	\begin{align}\label{eq:solutions/6/3/11/eq1}
		det(c\vec{I}-\mathbf{BA}) \neq 0
	\end{align}
	
	There are two cases for this:
	
	\begin{enumerate}
		\item $\textbf{c = 0}$ \\
		In this case $det(- \vec(BA)) \neq 0$. 	
		\begin{align}
			det(- \vec(BA)) &= (-1)^{n} \ det(\vec{B}) \ det(\vec{A}) \nonumber\\ 
			&= (-1)^{n} \ det(\vec{A}) \ det(\vec{B}) \nonumber\\ 
			&= det(- \vec{AB}) \nonumber\\
			&= det(c\vec{I} - \vec{AB}) \nonumber
		\end{align}
		From $\eqref{eq:solutions/6/3/11/eq1}$ we can write this as
		\begin{align}
			det(c\vec{I}-\vec{AB}) \neq 0
		\end{align}
		
		
		\item $\textbf{c $\neq$ 0}$ \\
		In this case 
		\begin{align}
			c\vec{I} - \vec{BA} &= c \ \left(\vec{I} - \frac{1}{c}\vec{BA}\right) \nonumber\\
			\implies det\left(c\left(\vec{I} - \frac{1}{c}\vec{BA}\right)\right) &= c^{n} \ det\left(\vec{I}-\frac{1}{c}\vec{BA}\right) \nonumber
		\end{align}	
		
		From $\eqref{eq:solutions/6/3/11/eq1}$, we get
		
		\begin{align}
			c^{n} \ det\left(\vec{I}-\frac{1}{c}\vec{BA}\right) \neq 0  \nonumber
		\end{align}
		
		Therefore, $\left(\vec{I} - \frac{1}{c}\vec{BA}\right)$ is invertible. From theorem $\ref{eq:solutions/6/3/11/thm2}$ we can say $\left(\vec{I} - \frac{1}{c}\vec{AB}\right)$ is invertible. 
		
		\begin{align}
			\implies det\left(\vec{I} - \frac{1}{c}\vec{AB}\right) &\neq 0 \nonumber\\
			\implies c^{n} \ det\left(\vec{I} - \frac{1}{c}\vec{AB}\right) &\neq 0 \nonumber\\
			\implies det\left(c\vec{I} - \vec{AB}\right) &\neq 0
		\end{align}
		
		
		
	\end{enumerate}
	
	After proving both the cases of $c = 0$ and $c \neq 0$ we say that the characteristic value $c$ must be the same.\\$\therefore$ $\vec{AB}$ and $\vec{BA}$ have the same characteristic values. \\
	
	After proving both the above statements, we can say that the two polynomials of degree $n$ with exactly the same roots, will be equal in nature. So, the characteristic polynomials are equal. \\
	
	Even though the characteristic polynomials are same, they may not necessarily be the same minimal polynomial.
	
	Lets take the below examples
	
	\begin{align}
		\vec{A} = \myvec{0&1\\0&0} \\
		\vec{B} = \myvec{0&0\\0&1}
	\end{align}
	We get the matrices $\vec{AB}$ and $\vec{BA}$ as
	
	\begin{align}
		\vec{AB} = \myvec{0&1\\0&0} \\
		\vec{BA} = \myvec{0&0\\0&0}
	\end{align}
	Then $\vec{AB} = \vec{A}$, whereas $\vec{BA}$ is the zero matrix. Since $\vec{A^2} = 0$ and $\vec{A} \neq 0$, the minimal polynomial of $\vec{AB}$ is $x^2$, whereas the minimal polynomial of $\vec{BA}$ is $x$.
	
	$\therefore$, $\vec{AB}$ and $\vec{BA}$ have the same characteristic polynomial, but the minimal polynomials are not the same.
	
	
