See Tables     \ref{eq:solutions/6/3/7/table:1}
 and     \ref{eq:solutions/6/3/7/table:2}

\onecolumn
\begin{longtable}{|l|l|}
	\hline
	\multirow{3}{*}{Characteristic Polynomial} 
	& \\
	& For an $n\times n$ matrix $\vec{A}$, characteristic polynomial is defined by,\\
	&\\
	& $\qquad\qquad\qquad p\brak{x}=\mydet{x\Vec{I}-\Vec{A}}$\\
	&\\
	\hline
	\multirow{3}{*}{Minimal Polynomial} 
	&\\
	& Minimal polynomial $m\brak{x}$ is the smallest factor of characteristic polynomial\\
	& $p\brak{x}$ such that,\\
	&\\
	& $\qquad \qquad \qquad m\brak{\vec{A}}=0$\\
	& \\
	& Every root of characteristic polynomial should be the root of minimal\\
	& polynomial\\
	&\\
    \hline
    \caption{Definitions and theorem used}
    \label{eq:solutions/6/3/7/table:1}
\end{longtable}
\begin{longtable}{|l|l|}
	\hline
	\multirow{3}{*}{Given} & \\
	& $V$ is the space of polynomials over $\mathbb{R}$ which have degree at most n.\\
    & \\
    \hline
	\multirow{3}{*}{Matrix Representation}
	& \\
	& The basis for the space $V$ is \\
	&\parbox{10cm}
	{\begin{align}
	\mathcal{B}=\cbrak{1,x,x^2,\dots,x^n}
	\end{align}}\\
	& Given that $\vec{D}$ is the differentiation operator.So,\\
	&\parbox{10cm}
	{\begin{align}
	\vec{D}(1)=0\\
	\vec{D}(x)=1\\
	\vdots \nonumber \\
	\vec{D}(x^n)=nx^{n-1}
	\end{align}}\\
	&The vectors of differentiation operator with respect to basis $\mathcal{B}$ \\
	&\parbox{10cm}
	{\begin{align}
	[\vec{D}\brak{1}]_{\mathcal{B}}=\myvec{0\\0\\\vdots\\0}_{(n+1)\times 1},
	[\vec{D}\brak{x}]_{\mathcal{B}}=\myvec{1\\0\\\vdots\\0}_{(n+1)\times 1} \dots
	[\vec{D}\brak{x^n}]_{\mathcal{B}}=\myvec{0\\\vdots\\n\\0}_{(n+1)\times 1}
	\end{align}}\nonumber\\
	&The matrix representation can be written as:\\
	&\parbox{10cm}
	{\begin{align}
	\vec{A}=
	\myvec{0&1&0&\dots&0\\
	0&0&2&\dots&0\\
	\vdots & \vdots & \vdots & \dots & \vdots\\
	0&0&0&\dots&n\\
	0&0&0&\dots&0}
	\end{align}}\\
	&\\
	\hline
	\multirow{3}{*}{Characteristic polynomial} 
	& \\
    &\parbox{10cm}
	{\begin{align}
	p\brak{x}=\mydet{x\Vec{I}-\Vec{A}}=
	\mydet{x&-1&0&\dots&0\\
	0&x&-2&\dots&0\\
	\vdots & \vdots & \vdots & \dots & \vdots\\
	0&0&0&\dots&-n\\
	0&0&0&\dots&x}
	\end{align}}\\
	&It is equal to the product of diagonal entries.\\
	&\parbox{10cm}
	{\begin{align}
	p\brak{x}=x^{n+1}
	\end{align}}\\
	&\\
	\hline
	\multirow{3}{*}{Minimal Polynomial} & \\
	& The minimal polynomial of $\vec{A}$ can be any of $x,x^2,\dots,x^{n+1}$ such that,\\
	&\parbox{10cm}
	{\begin{align}
	m\brak{\vec{A}}=0
	\end{align}}\\
	\hline
	\multirow{3}{*}{Explanation} & \\
	&Let $P(n)$: Minimum polynomial of $\vec{D}$=$x^{n+1}$ i.e $\vec{A}^{n+1}=0$ \\
	& For n=1\\
	&\parbox{10cm}
	{\begin{align}
	\vec{A}=\myvec{0&1\\0&0}\\
	\vec{A^2}=\myvec{0&0\\0&0}
	\end{align}}\\
    &So,$P(1)$ is true.\\
    &Assume $P(k)$ holds for $1 \le k \le n$.\\
	&\parbox{10cm}
	{\begin{align}
	\vec{A}_k=
	\myvec{0&1&0&\dots&0\\
	0&0&2&\dots&0\\
	\vdots & \vdots & \vdots & \dots & \vdots\\
	0&0&0&\dots&k\\
	0&0&0&\dots&0}_{(k+1 \times k+1)}
	\implies \vec{A}_k^{k+1}=\vec{0} \label{eq:solutions/6/3/7/1}
	\end{align}}\\
	& We need to show that $P(k+1)$ is true.\\
	&\parbox{10cm}
	{\begin{align}
	\vec{A}_{k+1}=
	\myvec{0&1&0&\dots&0&0\\
	0&0&2&\dots&0&0\\
	\vdots & \vdots & \vdots & \dots & \vdots& \vdots\\
	0&0&0&\dots&k&0\\
	0&0&0&\dots&0&k+1\\
	0&0&0&\dots&0&0}_{(k+2 \times k+2)}
	\end{align}}\\
	& Expressing in terms of block matrices \\
	&\parbox{10cm}
	{\begin{align}
	\vec{A}_{k+1}=\myvec{\vec{A}_k&\vec{x}\\\vec{0}_{1\times k+1}&0},
    \vec{x}=\myvec{0\\0\\\vdots\\0\\k+1}_{k+1 \times 1}
	\end{align}}\\
	&\\
	\hline
	\multirow{3}{*}{Finding $\vec{A}_{k+1}^{k+2}$} & \\
	&\parbox{10cm}
	{\begin{align}
	\vec{A}_{k+1}^2=\myvec{\vec{A}_k&\vec{x}\\\vec{0}&0}
	\myvec{\vec{A}_k&\vec{x}\\\vec{0}&0}=
	\myvec{\vec{A}_k^2&\vec{A}_k\vec{x}\\0&0}\\
	\vec{A}_{k+1}^3=\myvec{\vec{A}_k^2&\vec{A}_k\vec{x}\\0&0}
	\myvec{\vec{A}_k&\vec{x}\\\vec{0}&0}
	=\myvec{\vec{A}_k^3&\vec{A}_k^2\vec{x}\\0&0}\\
	\vec{A}_{k+1}^{k+2}=\myvec{\vec{A}_k^{k+2}&\vec{A}_k^{k+1}\vec{x}\\0&0}\\
	\text{From \eqref{eq:solutions/6/3/7/1},We know that $\vec{A}_k^{k+1}=\vec{0}$} \nonumber\\
	\implies \vec{A}_{k+1}^{k+2}=\vec{0}
	\end{align}}\\
	&So,$P(k+1)$ is true.\\
	&\\
	&\\
    \hline
	\multirow{3}{*}{Conclusion} & \\
	& From above,by using the principle of induction we can say that\\ &the minimal polynomial is\\
	&\parbox{10cm}
	{\begin{align}
	x^{n+1}
	\end{align}}\\
	& \\
	\hline
	\caption{Finding minimal polynomial}
    \label{eq:solutions/6/3/7/table:2}
\end{longtable}
\twocolumn
