See Tables \ref{eq:solutions/6/3/6/table:1} and \ref{eq:solutions/6/3/6/table:2}

\begin{table*}[ht!]
\centering
\begin{tabular}{|c|l|}
    \hline
	\multirow{3}{*}{Characteristic Polynomial} 
	& \\
	& For an $n\times n$ matrix $\vec{A}$, characteristic polynomial is defined by,\\
	&\\
	& $\qquad\qquad\qquad p\brak{x}=\mydet{x\Vec{I}-\Vec{A}}$\\
	&\\
	\hline
	\multirow{3}{*}{Cayley-Hamilton Theorem}
    &\\
    & If $p\brak{x}$ is the characteristic polynomial of an $n\times n$ matrix $\vec{A}$, then,\\
    &\\
    &$\qquad \qquad \qquad p\brak{\vec{A}}=\vec{0}$\\
    &\\
    \hline
	\multirow{3}{*}{Minimal Polynomial} 
	&\\
	& Minimal polynomial $m\brak{x}$ is the smallest factor of characteristic polynomial\\
	& $p\brak{x}$ such that,\\
	&\\
	& $\qquad \qquad \qquad m\brak{\vec{A}}=0$\\
	& \\
	& Every root of characteristic polynomial should be the root of minimal\\
	& polynomial\\
	&\\
    \hline
\end{tabular}
    \caption{Definitions}
\label{eq:solutions/6/3/6/table:1}
\end{table*}

\onecolumn
\begin{longtable}{|l|l|}
\hline
\multirow{3}{*}{} & \\
Statement&Solution\\
\hline
&\\
Assuming matrix $\vec{A}$ as follows:&Let us Consider 3 $\times$ 3 upper triangular matrix,\\
&\parbox{6cm}{\begin{align*}
    \vec{A}=\myvec{e&a&b\\0&f&c\\0&0&d}
\end{align*}}\\
\hline
Characteristic polynomial of $\vec{A}$&
\parbox{6cm}{\begin{align*}
    \mydet{x\vec{I}-\vec{A}}&=\myvec{x-e&-a&-b\\0&x-f&-c\\0&0&x-d}\\
    &=(x-e)(x-f)(x-d)
\end{align*}}\\
\hline
&\\
Given&The minimum polynomial is\\
&\parbox{6cm}{\begin{align*}
    p(x)&=x^2
\end{align*}}\\
&Therefore p(x) must divide characteristic polynomial.\\
&This will be satisfied only if the values e,f,d are zeros.\\
&\\
\hline
&\\
Characteristic polynomial&\\
when e=0,f=0 and d=0&\parbox{6cm}{\begin{align*}
    \mydet{x\vec{I}-\vec{A}}&=x^3
\end{align*}}\\
&\\
\hline
&\\
Since $p(x)=x^2$&\\
Hence $ p(\vec{A})=\vec{A}^2=\vec{0}_{3\times3}$&Therefore calculating $p(\vec{A})$ as follows:\\
&\parbox{6cm}{\begin{align*}
    \myvec{0&a&b\\0&0&c\\0&0&0} \myvec{0&a&b\\0&0&c\\0&0&0}&=\vec{0}_{3\times3}\\
    \myvec{0&0&ac\\0&0&0\\0&0&0}&=\vec{0}_{3\times3}
\end{align*}}\\
&\\
\hline
\pagebreak
\hline
&\\
&For $\vec{A}^2$ to be a zero matrix, either a=0 or c=0\\
With entries a=0,e=0,f=0,d=0&\\
The matrix $\vec{A}$ will be:&\\
For b=1,c=1&\parbox{6cm}{\begin{align*}
    \vec{A}&=\myvec{0&0&b\\0&0&c\\0&0&0}\\
    \vec{A}&=\myvec{0&0&1\\0&0&1\\0&0&0}
\end{align*}}\\
\hline
&\\
Conclusion&Thus the matrix,\\
&\parbox{6cm}{\begin{align*}
        \vec{A}&=\myvec{0&0&1\\0&0&1\\0&0&0}
\end{align*}}\\
&has the minimal polynomial as $x^2$.\\
&\\
\hline
\caption{Solution summary}
\label{eq:solutions/6/3/6/table:2}
\end{longtable}
\twocolumn
