See Tables \ref{eq:solutions/6/8/1/table:1} 
 \ref{eq:solutions/6/8/1/table:2}
and \ref{eq:solutions/6/8/1/table:3}


\begin{table*}[ht!]
\centering
\begin{tabular}{|c|l|}
    \hline
	\multirow{3}{*}{Characteristic Polynomial} 
	& \\
	& For an $n\times n$ matrix $\vec{A}$, characteristic polynomial is defined by,\\
	&\\
	& $\qquad\qquad\qquad p\brak{x}=\mydet{x\Vec{I}-\Vec{A}}$\\
	&\\
	\hline
	\multirow{3}{*}{Minimal Polynomial} 
	&\\
	& Minimal polynomial $m\brak{x}$ is the smallest factor of characteristic polynomial\\
	& $p\brak{x}$ such that,\\
	&\\
	& $\qquad \qquad \qquad m\brak{\vec{A}}=0$\\
	& \\
	& Every root of characteristic polynomial should be the root of minimal\\
	& polynomial and the minimal polynomial divides the charateristic polynomial.\\
	&\\
	\hline
	\multirow{3}{*}{Basis Theorem} 
	&\\
	& Let $V$ be a subspace of dimension $m$. Then:\\
	& Any $m$ linearly independent vectors in $V$ forms a basis for $V$.\\
	& Any $m$ vectors that span $V$ forms a basis for $V$.\\
	&\\
    \hline
\end{tabular}
    \caption{Definitions}
\label{eq:solutions/6/8/1/table:1}
\end{table*}
\onecolumn
\begin{longtable}{|c|l|}
    \hline
	\multirow{3}{*}{Express Minimal Polynomial} 
	& \\
	& $A$ = \myvec{6&-3&-2\\4&-1&-2\\10&-5&-3} \\
	& Characteristic Polynomial = $\mydet{xI-A}$ = \mydet{x-6&3&2\\-4&x+1&2\\-10&5&x+3}\\
	& By solving above determinant, we find out that\\
	& $x^3-2x^2+x-2$ = $\brak{x-2}\brak{x^2+1}$ \\
	& Since, $T-2I$ $\ne$ 0 and the minimal polynomial divides the characteristic\\
	& polynomial, thus minimal polynomial $p$ for $T$ is $p$ = $m\brak{x}$\\
	& $p$ = $\brak{x-2}\brak{x^2+1}$ \\
	& Put $p_1$ = $\brak{x-2}$ and $p_2$ = $\brak{x^2+1}$\\
	& Thus, $p$ = $p_1p_2$ \\
	&\\
	\hline
	\multirow{3}{*}{Bases $B_1$ and matrix $T_1$} & \\
	& Let $W_1$ = \{$\alpha$ $\in$ $R^3$/$p_1\brak{T}\alpha$ = 0,$\brak{T-2I}\alpha$ = 0\}  \\
	& Therefore, $A-2I$ = $\myvec{4&-3&-2\\4&-3&-2\\10&-5&-5}$ $\xrightarrow{}$ $\myvec{-4&3&2\\-2&1&1\\0&0&0}$ $\xrightarrow{}$ $\myvec{-2&1&1\\0&1&0\\0&0&0}$ \\
	& Rank of $A-2I$ is 2\\
	& Nullity of $A-2I$ = no of columns - Rank = 3 - 2 = 1 \\
	& That means the dimension of $W_1$ is 1\\ 
	& Thus we can let, $\alpha_1$ = \myvec{1\\0\\2} $\in$ $W_1$ (Basis theorem mentioned in Definitions)\\
	& Therefore, $B_1$ = \{$\alpha_1$\} is the basis for $W_1$\\
	& Let $T_1$ be the matrix induced by $T$ on $W_1$\\
	& $T_1\alpha_1$ = $T\alpha_1$ = \myvec{6&-3&-2\\4&-1&-2\\10&-5&-3} \myvec{1\\0\\2} = \myvec{2\\0\\4} = 2 \myvec{1\\0\\2} = 2$\alpha_1$ \\
	& $[T_1]_{B_1}$ = $[2]$ \\
	&\\
	\hline
    \multirow{3}{*}{Bases $B_2$ and matrix $T_2$} & \\
	& Let $W_2$ = \{$\alpha$ $\in$ $R^3$/$p_2\brak{T}\alpha$ = 0,$\brak{T^2+I}\alpha$ = 0\}  \\
	& Therefore, $A^2+I$ = $\myvec{6&-3&-2\\4&-1&-2\\10&-5&-3}\myvec{6&-3&-2\\4&-1&-2\\10&-5&-3}+\myvec{1&0&0\\0&1&0\\0&0&1}$ = $\myvec{5&-5&0\\0&0&0\\10&-10&0}$ \\
	& Rank of $A^2+I$ is 1\\
	& Nullity of $A^2+I$ = no of columns - Rank = 3 - 1 = 2 \\
	& That means the dimension of $W_2$ is 2\\ 
	& Thus we can let, $\alpha_2$ = \myvec{1\\1\\0}, $\alpha_3$ = \myvec{0\\0\\1} $\in$ $W_2$(Basis theorem in Definitions)\\
	& Therefore, $B_2$ = \{$\alpha_2,\alpha_3$\} is the basis for $W_2$\\
	& Let $T_2$ be the matrix induced by $T$ on $W_2$\\
	& $T_2\alpha_2$ = $T\alpha_2$ = \myvec{6&-3&-2\\4&-1&-2\\10&-5&-3} \myvec{1\\1\\0} = \myvec{3\\3\\5} = 3 \myvec{1\\1\\0} + 5 \myvec{0\\0\\1} = 3$\alpha_2$ + 5$\alpha_3$\\
	& $T_2\alpha_3$ = $T\alpha_3$ = \myvec{6&-3&-2\\4&-1&-2\\10&-5&-3} \myvec{0\\0\\1} = \myvec{-2\\-2\\-3} = -2 \myvec{1\\1\\0} + -3 \myvec{0\\0\\1} = -2$\alpha_2$ - 3$\alpha_3$\\
	& $\myvec{\alpha_2 & \alpha_3}[T_2]$ =  $\myvec{\alpha_2 & \alpha_3}$$\myvec{3&-2\\5&-3}$ \\
	& \\
	& $\implies$ $[T_2]_{B_2}$ = $\myvec{3&-2\\5&-3}$\\
	\hline
	\caption{Finding of Basis and corresponding matrix}
    \label{eq:solutions/6/8/1/table:2}
\end{longtable}
\begin{longtable}{|c|l|}
    \hline
	\multirow{3}{*}{Express Minimal Polynomial} 
	& \\
	& $A$ = \myvec{6&-3&-2\\4&-1&-2\\10&-5&-3} \\
	& We get, $p_1$ = $\brak{x-2}$ and $p_2$ = $\brak{x^2+1}$\\
	& Thus, $p$ = $p_1p_2$ \\
	&\\
	\hline
	\multirow{3}{*}{$W_i$} 
	& \\
	& $W_1$ = \{$\alpha$ $\in$ $R^3$/$p_1\brak{T}\alpha$ = 0,$\brak{T-2I}\alpha$ = 0\}\\
	& $W_2$ = \{$\alpha$ $\in$ $R^3$/$p_2\brak{T}\alpha$ = 0,$\brak{T^2+I}\alpha$ = 0\}\\
	& \\
	\hline
	\multirow{3}{*}{$B_1$} 
	& \\
	& $B_1$ = \{$\alpha_1$\} is the basis for $W_1$\\
	& where, $\alpha_1$ = \myvec{1\\0\\2} $\in$ $W_1$\\ \\
	\hline
	\multirow{3}{*}{$B_2$} 
	& \\
	& $B_2$ = \{$\alpha_2,\alpha_3$\} is the basis for $W_2$\\
	& where, $\alpha_2$ = \myvec{1\\1\\0}, $\alpha_3$ = \myvec{0\\0\\1} $\in$ $W_2$\\
	& \\
	\hline
	\multirow{3}{*}{$T_1$} 
	& \\
	& $[T_1]_{B_1}$ = $\myvec{2}$ \\
	\hline
	\multirow{3}{*}{$T_2$} 
	& \\
	& $[T_2]_{B_2}$ = $\myvec{3&-2\\5&-3}$\\
	& \\
	\hline
	\caption{Conclusion of above Results}
    \label{eq:solutions/6/8/1/table:3}
\end{longtable}
\twocolumn
