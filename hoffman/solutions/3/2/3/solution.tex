The transformed vector can be re-written by expanding the columns as follows
\begin{align}
	\myvec{3x_1\\x_1-x_2\\2x_1+x_2+x_3}
	&=\myvec{3\\1\\2}x_1+\myvec{0\\-1\\1}x_2+\myvec{0\\0\\1}x_3\\
	&=\myvec{3&0&0\\
		 1&-1&0\\
		 2&1&1}\myvec{x_1\\x_2\\x_3} \\
	&\implies \vec{T}=\myvec{3&0&0\\
                 1&-1&0\\
                 2&1&1}
\end{align}
Using Gauss-Jordan Elimination to find the inverse of $\vec{T}$, if it exists
\begin{align}
	\myvec{3&0&0&|&1&0&0\\
	       1&-1&0&|&0&1&0\\
	       2&1&1&|&0&0&1}\\
	\xleftrightarrow[]{R_1\leftarrow\frac{R_1}{3}}
	\myvec{1&0&0&|&\frac{1}{3}&0&0\\
               1&-1&0&|&0&1&0\\
               2&1&1&|&0&0&1}\\
	\xleftrightarrow[R_3\leftarrow R_3-2R_1]{R_2\leftarrow R_2-R_1}
	\myvec{1&0&0&|&\frac{1}{3}&0&0\\
	       0&-1&0&|&-\frac{1}{3}&1&0\\
	       0&1&1&|&-\frac{2}{3}&0&1}\\
        \xleftrightarrow[]{R_2\leftarrow \ -R_2}
	\myvec{1&0&0&|&\frac{1}{3}&0&0\\
               0&1&0&|&\frac{1}{3}&-1&0\\
               0&1&1&|&-\frac{2}{3}&0&1}\\
        \xleftrightarrow[]{R_3\leftarrow R_3-R_2}
	\myvec{1&0&0&|&\frac{1}{3}&0&0\\
               0&1&0&|&\frac{1}{3}&-1&0\\
               0&0&1&|&-1&1&1}
\end{align}
Since $rank(\vec{T})=3, \ \vec{T}$ is invertible and the inverse is
\begin{align}
	\vec{T}^{-1}=\myvec{\frac{1}{3}&0&0\\ 
                            \frac{1}{3}&-1&0\\
                            -1&1&1}
\end{align}
Now consider any vector $\myvec{x_1\\x_2\\x_3} \in \vec{R}^3$, then
\begin{align}
	\vec{T}^{-1}\myvec{x_1\\x_2\\x_3}&=\myvec{\frac{1}{3}&0&0\\
                            \frac{1}{3}&-1&0\\
                            -1&1&1}\myvec{x_1\\x_2\\x_3}\\
			    &=\myvec{\frac{x_1}{3}\\\frac{x_1}{3}-x_2\\-x_1+x_2+x_3}
\end{align}
Therefore the transformation $\vec{T}^{-1}$ is defined on $\vec{R}^3 $ as
\begin{align}
\vec{T}^{-1}\myvec{x_1\\x_2\\x_3}=\myvec{\frac{x_1}{3}\\\frac{x_1}{3}-x_2\\-x_1+x_2+x_3}
\end{align}
