\begin{table*}[!ht]
	\begin{center}
\resizebox{2\columnwidth}{!}
{
		\begin{tabular}{|c|c|}
			\hline
			\multirow{3}{*}{singular} &\\
			& A linear transformation $T : \mathbb{R}^{n} \rightarrow \mathbb{R}^{n}$ is said to be singular if $\exists$ some non-zero $\vec{X} \in \mathbb{R}^n$\\
			& s.t $\vec{A}\vec{X} = \vec{0}$ $\text{ i.e } Nullity(A) \neq 0$. \\
			& \\
			& From rank-nullity theorem we can say $rank(A)<n$\\
			& \\
			\hline
			\multirow{3}{*}{non-singular} & \\
			& A linear transformation $T : \mathbb{R}^{n} \rightarrow \mathbb{R}^{m}$ is said to be non-singular if $\vec{A}\vec{X} = \vec{0} \text{ implies } \vec{X} = \vec{0}$\\
			& $\text{ i.e } Nullity(A) = 0$\\
			& \\
			\hline			
			\multirow{3}{*}{onto} & \\
			& A linear transformation $T : \mathbb{R}^{n} \rightarrow \mathbb{R}^{m}$, $m\leq n$ is said to be onto if for every  $\vec{b} \in \mathbb{R}^{m}$,\\
			&  $\vec{A}\vec{X} =\vec{b}$ has atleast one solution $\vec{X} \in \mathbb{R}^{n}$..\\
			& \\
			& i.e $dim(Col(\vec{A})) = m$ or $Rank(\vec{A}) = m$\\
			& \\
			& If $m>n$, then $\vec{A}\vec{X}=\vec{b}$ has no solution because rank-nullity theorem is not satisfied.\\
			& \\
			\hline
		\end{tabular}
}
	\end{center}
\caption{}
\label{table:solutions/3/2/10/1}
\end{table*}
{Proof}
\begin{table*}[!hb]
	\begin{center}
%	\begin{adjustbox}{width=1\textwidth}
\resizebox{2\columnwidth}{!}
{
		\begin{tabular}{|c|c|c|}			
			\hline
			\multicolumn{3}{|c|}{Let $\vec{A}$ be an $m \times n$ matrix with entries in $F$ and}\\
			\multicolumn{3}{|c|}{let $T$ be the linear transformation from $F^{n \times1 }$ into $F^{m \times l}$}\\
			\multicolumn{3}{|c|}{defined by $T(\vec{X}) = \vec{A}\vec{X}$. If,}\\			
			\hline
			& \multicolumn{1}{|c|}{$m < n$} & \multicolumn{1}{c|}{$m > n$}\\
			\hline
			\multirow{4}{*}{singular} & &\\
			& Since $rank(\vec{A})<n$, by definition T is singular & Consider an non-singular $T$ such that $rank(\vec{A}) > n$\\
			&  & \\
			\hline
			\multirow{3}{*}{onto} & &\\
			& Since $m<n$, by definition $T$ can be onto & Since $m>n$, by definition $T$ is not onto.\\
			& &\\
			\hline
			\end{tabular}
}
%			\end{adjustbox}
	\end{center}
\caption{}
\label{table:solutions/3/2/10/2}
\end{table*}
\begin{enumerate}
\item {$m<n$}
\begin{align}
	\text{Let, }T:\mathbb{R}^3 \rightarrow &\mathbb{R}^2\\
	T(\vec{X}) = \vec{A}\vec{X} &= \vec{b}\\
	\text{Let, }\vec{A} &= \myvec{1 & 1 & 0\\0 & 1 & 1}\\
	\text{Consider, }\vec{X} &= \myvec{2\\4\\1}\\
	\implies \vec{A}\vec{X} &= \myvec{1 & 1 & 0\\0 & 1 & 1}\myvec{2\\4\\1}\\
	&= \myvec{6\\5}
\end{align}
Hence T is onto.
\begin{align}
	\text{Consider, } \vec{X} &= \myvec{1\\-1\\1}\\
	\implies \vec{A}\vec{X} &= \myvec{1 & 1 & 0\\0 & 1 & 1}\myvec{1\\-1\\1}\\
	&= \vec{0}
\end{align}
Since $\exists \quad \vec{X} \neq \vec{0}$ such that $\vec{A}\vec{X} = \vec{0}$, T is singular.\\
\\
$\therefore$ T is both onto and singular.
\item {$m>n$}
\begin{align}
	\text{Let, }T:\mathbb{R}^3 \rightarrow &\mathbb{R}^2\\
	T(\vec{X}) = \vec{A}\vec{X} &= \vec{b}\\
	\text{Let, }\vec{A} &= \myvec{1 & 0\\0 & 1\\1 & 0}\\
	\text{Consider, }\vec{X} &= \myvec{-1\\2}\\
	\implies \vec{A}\vec{X} &= \myvec{1 & 0\\0 & 1\\1 & 0}\myvec{-1\\2}\\
	&= \myvec{-1\\2\\-1}\\
\end{align}
$\therefore$ T is not onto, and is also non-singular.
\end{enumerate}
