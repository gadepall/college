See Table \ref{eq:solutions/3/2/11/table:1} and \ref{eq:solutions/3/2/11/table:2}


\onecolumn
\begin{longtable}{|l|l|}
\hline
\multirow{3}{*}{Given} & \\
& $\mathbb{V}$ is a finite-dimensional vector space,\\
& $\vec{T}:\mathbb{V} \rightarrow \mathbb{V}$ and\\
& $ rank (\vec{T}^2) = rank (\vec{T})$\\
&\\
\hline
\multirow{3}{*}{To Prove} & \\
& range and null space of $\vec{T}$ are disjoint\\
&\\
\hline
\multirow{3}{*}{Defining $rank(\vec{T})$} & \\
& Let $\lbrace \beta_1,\ldots,\beta_k,\beta_{k+1},\ldots,\beta_n \rbrace$ is the span of $\mathbb{V}$.\\
& \\
& The linear transformation of $\mathbb{V}$ is\\
& $\lbrace \vec{T}\beta_1,\ldots,\vec{T}\beta_k,\vec{T}\beta_{k+1},\ldots,\vec{T}\beta_n \rbrace$.\\
& \\
& Suppose the $rank (\vec{T}) = k$, then \\
& the basis of $\vec{T}$ is $\lbrace \vec{T}\beta_1,\ldots,\vec{T}\beta_k \rbrace$ \\
& and are linearly independent.\\
& \\
\hline
\multirow{3}{*}{Defining $range(\vec{T}^2)$} & \\
& Now $\vec{T}^2 : \mathbb{V} \rightarrow \mathbb{V}$ be a linear transformation for any $\alpha \in \mathbb{V}$.  \\
& $\therefore \vec{T}^2(\mathbb{V}) = \vec{T}(\vec{T}(\alpha))$ and \\
& $\lbrace \vec{T}^2\beta_1,\ldots,\vec{T}^2\beta_k \rbrace$ span the range of $\vec{T}^2$\\
& \\
& since $ rank (\vec{T}^2) = rank (\vec{T})$\\
& $\implies dim \  range (\vec{T}^2) = dim \  range (\vec{T})$\\
& $\therefore \lbrace \vec{T}^2\beta_1,\ldots,\vec{T}^2\beta_k \rbrace$ must be\\
& basis for $range(\vec{T}^2)$\\
& \\
& \\
& \\
\hline
\multirow{3}{*}{Obtaining $range $} & \\
& Now let $\alpha \in range(\vec{T})$, then\\
$ and \  nullspace \  of \  \vec{T}$
& it can be written as linear combinations of \\
& vectors in $ range(\vec{T})$\\
& $\therefore \alpha = C_1\vec{T}\beta_1+C_2\vec{T}\beta_2+\ldots+C_k\vec{T}\beta_k$\\
& \\
& If $\alpha \in null space (\vec{T})$ also, then\\
& $ \vec{T}(\mathbb{V}) = 0$\\
& $\implies \vec{T}(C_1\vec{T}\beta_1+C_2\vec{T}\beta_2+\ldots+C_k\vec{T}\beta_k) = 0$\\
& $ \implies C_1\vec{T}^2\beta_1+C_2\vec{T}^2\beta_2+\ldots+C_k\vec{T}^2\beta_k = 0$\\
& \\
& since $\lbrace \vec{T}^2\beta_1,\ldots,\vec{T}^2\beta_k \rbrace$ is basis of $\vec{T}^2$\\
& $\implies C_1 = C_2 = \ldots = C_k = 0$\\
& $\implies \mathbb{V} = 0$\\
& $\therefore$ if $\alpha$  is in both $ range(\vec{T}) \  and \  null space(\vec{T})$, \\
& then $\alpha$ must be a zero vector.\\
& \\
& Hence it is proved that \\
& range and null space of $\vec{T}$ are disjoint.\\
&\\
\hline
\multirow{3}{*}{Conclusion} & \\
& The range and null space of $\vec{T}$ are disjoint.\\
&\\
\hline
\caption{Proof}
\label{eq:solutions/3/2/11/table:1}
\end{longtable}
\begin{longtable}{|l|l|}
\hline
\multirow{3}{*}{Example} & \\
& Let $\alpha$ be a basis of $\mathbb{V}$ of $\mathbb{R}^{nxn}$ space and\\
& \\
& consider, $\alpha = \myvec{1&7&-1&-1\\-1&1&2&1\\4&-2&0&-4\\2&3&4&-2}$\\
& \\
& linear transformation of $\alpha$ into $\mathbb{V}, \vec{T}(\alpha) = c\alpha$, where $c$ is a scalar,\\
& then row reduced echelon form of $\vec{T}$ is\\
& $rref(\vec{T}) = \myvec{1&0&0&-1\\0&1&0&0\\0&0&1&0\\0&0&0&0}$\\
& \\
& $\implies rank(\vec{T}) = 3,$\\
& $nullity(\vec{T}) = 1$\\
& \\
& $\implies range(\vec{T}) = \myvec{1&7&-1\\-1&1&2\\4&-2&0\\2&3&4},$\\
& $null space(\vec{T}) = \myvec{1\\0\\0\\1}$\\
& \\
& Now linear transformation $\vec{T}(\vec{T}(\alpha)) = cd\alpha$, where $c \  and \  d$ are scalars.\\
& \\
& Let $ c = d = 1$, then $range(\vec{T}^2) = range (\vec{T})$,\\
& $\implies rank (\vec{T}^2) = rank (\vec{T}) = 3$,\\
& $\implies nullity (\vec{T}^2) = nullity (\vec{T}) = 1$\\
& and $range(\vec{T}) \cap nullspace(\vec{T}) = \lbrace 0\rbrace$\\
& \\
& Hence proved that, the range and null space of $\vec{T}$ are disjoint.\\
& \\
\hline
\caption{Example}
\label{eq:solutions/3/2/11/table:2}
\end{longtable}
\twocolumn
