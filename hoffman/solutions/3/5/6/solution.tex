Let $\vec{b}, \vec{\alpha}$ $\in$ $F^n$ and a is a scalar
\begin{align}
    T(a\vec{\alpha}+\vec{b})=(f_1(a\vec{\alpha}+\vec{b}),\dots,f_m(a\vec{\alpha}+\vec{b})) \nonumber \\
    =(a f_1(\vec{\alpha})+f_1(\vec{b}),\dots,a f_m(\vec{\alpha})+f_m(\vec{b})) \nonumber \\
    =a(f_1(\vec{\alpha}),\dots,f_m(\vec{\alpha}))+(f_1(\vec{b}),\dots,f_m(\vec{b}))\label{eq:solutions/3/5/6/2.1}
\end{align}
The equation \eqref{eq:solutions/3/5/6/2.1} can be written as 
\begin{align}
T(a\vec{\vec{\alpha}}+\vec{b})=aT(\vec{\vec{\alpha}})+T(\vec{b})
\end{align} 
So, T is a linear transformation.\\
Let the matrix $A$ of order $m\times n$ represent any linear transformation $\vec{X} \mapsto A\vec{X}$ from $F^n$ into $F^m$. For i=1,\dots,m ,let
\begin{align}
    f_i(x_1,\dots,x_n)=\sum_{j=1}^nA_{ij}x_j
\end{align}
The transformation into $F^m$, $A\vec{X}$ can be written as 
\begin{align}
   (f_1(\vec{X}),\dots,f_m(\vec{X}))
\end{align}
This is of the form \eqref{eq:solutions/3/5/6/eq:main}
