Let us consider $\vec{\alpha} = \myvec{a\\b\\c}$ such that
\begin{align}
\alpha_1 x_1 + \alpha_2 x_2 + \alpha_2 x_2 = \vec{\alpha}\\
\alpha_1 x_1 + \alpha_2 x_2 + \alpha_2 x_2 = \myvec{a\\b\\c}
\end{align}
The coefficient matrix is :
\begin{align}
A = \myvec{1 & 0 & -1\\0 & 1 & -1\\1 & -2 & 0}
\end{align}
So,
\begin{align}
A\vec{x}= \vec{\alpha}\\
\implies x = A^{-1} \alpha\\
\end{align}
Now to get $A^{-1}$, we will consider Gauss-Jordon theorem. So, we will take $(A|I)$, where $I$ is a $3\times3$ identity matrix.
\begin{multline}
\myvec{1 & 0 & -1 & 1 & 0 & 0\\0 & 1 & -1 & 0 & 1 & 0\\1 & -2 & 0 & 0 & 0 & 1}\xleftrightarrow[]{R_3\leftarrow R_3-R_1}\\
\myvec{1 & 0 & -1 & 1 & 0 & 0\\0 & 1 & -1 & 0 & 1 & 0\\0 & -2 & 1 & -1 & 0 & 1}
\xleftrightarrow[]{R_3\leftarrow R_3+2R_2}\\
\myvec{1 & 0 & -1 & 1 & 0 & 0\\0 & 1 & -1 & 0 & 1 & 0\\0 & 0 & -1 & -1 & 2 & 1}\xleftrightarrow[]{R_3\leftarrow R_3/(-1)}\\
\myvec{1 & 0 & -1 & 1 & 0 & 0\\0 & 1 & -1 & 0 & 1 & 0\\0 & 0 & 1 & 1 & -2 & -1}\xleftrightarrow[R_1\leftarrow R_1+R_3]{R_2\leftarrow R_2+R_3}\\
\myvec{1 & 0 & 0 & 2 & -2 & -1\\0 & 1 & 0 & 1 & -1 & -1\\0 & 0 & 1 & 1 & -2 & -1}
\end{multline}
Now, we can say that
\begin{align}
A^{-1} = \myvec{2 & -2 & -1\\1 & -1 & -1\\1 & -2 & -1}
\end{align}
As
\begin{align}
\vec{x} = A^{-1} \vec{\alpha}\\
\implies \myvec{x_1\\x_2\\x_3}= A^{-1} \myvec{a\\b\\c}\\
\implies \vec{x}= \myvec{2 & -2 & -1\\1 & -1 & -1\\1 & -2 & -1} \myvec{a\\b\\c}
\end{align}

Now, as $f$ is a linear functional on $R^3$,
\begin{align}
\vec{\alpha} = \alpha_1 x_1 + \alpha_2 x_2 + \alpha_2 x_2\\
\implies f(\alpha) = f(\alpha_1 x_1 + \alpha_2 x_2 + \alpha_2 x_2)\\
\implies f(\alpha) = x_1 f(\alpha_1) + x_2f(\alpha_2)+ x_3f(\alpha_3)\\
\implies f(\alpha) = \myvec{x_1 & x_2 & x_3} \myvec{f(\alpha_1)\\f(\alpha_2)\\f(\alpha_3)}\\
\implies f(\alpha) = \vec{x}^T \myvec{f(\alpha_1)\\f(\alpha_2)\\f(\alpha_3)}
\end{align}
As mentioned in the problem statement, $f(\alpha_1)=f(\alpha_2)=0$ and $f(\alpha_3)\neq 0$.

Now,
\begin{align}
f(\alpha) = \vec{x}^T \myvec{f(\alpha_1)\\f(\alpha_2)\\f(\alpha_3)}\\
\implies f(\alpha) =  \vec{x}^T \myvec{0\\0\\f(\alpha_3)}\\
\implies f(\alpha) = \myvec{a\\b\\c}^T \myvec{2 & 1 & 1\\-2 & -1 & -2\\-1 & -1 & -1} \myvec{0\\0\\f(\alpha_3)}\\
\implies f(\alpha) = \myvec{a\\b\\c}^T \myvec{f(\alpha_3)\\-2f(\alpha_3)\\-f(\alpha_3)}\\
\implies f(a,b,c) = f(\alpha_3) \myvec{a\\b\\c}^T \myvec{1\\-2\\-1}
\end{align}
So, the function can be defined as:
\begin{align}
f(a,b,c) = f(\alpha_3) \myvec{a\\b\\c}^T \myvec{1\\-2\\-1}\\
\textbf{or,} f(\vec{\alpha}) = f(\alpha_3) (a-2b-c)
\end{align}
