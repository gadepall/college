See Tables \ref{table:3/3/4/1},
\ref{table:3/3/4/2} and 
\ref{table:3/3/4/3/}.

\begin{table*}[!ht]
\resizebox{2\columnwidth}{!}
{
		\begin{tabular}{|c|c|}
			\hline
			\multirow{3}{*}{Invertible} & \\
			& A linear map $\vec{T}\in \vec{L(V,W)}$ is called invertible if there exists a linear map $\vec{S}\in \vec{L(W,V)}$ \\
			Linear Map & such that $\vec{ST}$ equals the identity map on $\vec{V}$ and $\vec{TS}$ equals the identity map on $\vec{W}$. \\
			& \\
			& A linear map $\vec{S}\in \vec{L(W,V)}$ satisfying $\vec{ST=I_{V}}$ and $\vec{TS=I_{W}}$ is called an inverse of $\vec{T}$.\\
			& \\
			\hline
			\multirow{3}{*}{Isomorphic} & \\
			& Two vector spaces $\vec{V}$ and $\vec{W}$ are called isomorphic if there is an isomorphism from one \\
			Vector Spaces & vector space onto the other one. An isomorphism is an invertible linear map. \\
			& \\
			\hline
			\multirow{3}{*}{Rank Nullity} & \\
			  & Let $\vec{V}$ and $\vec{W}$ be finite dimensional vector spaces. Let $\vec{T}$: $\vec{V}$ $\to$ $\vec{W}$ be a linear transformation \\
			Theorem &  Rank($\vec{T}$) + Nullity($\vec{T}$) = dim $\vec{V}$\\
			& \\
			\hline	
		\end{tabular}
}
\caption{Definition}
\label{table:3/3/4/1}
\end{table*}
\begin{table*}[!ht]
\resizebox{2\columnwidth}{!}
{
	\begin{tabular}{|l|l|}
		\hline
		\multirow{3}{*}{Result 1} & \\
		& The space of all $m \times n$ matrices over the field $\vec{F}$ has dimension $mn$.\\
		\hline
		\multirow{3}{*}{Result 2} & \\
	   & Let $\vec{V}$ and $\vec{W}$ be finite-dimensional vector spaces over the field $\vec{F}$ such that dim $\vec{V}$ = dim $\vec{W}$. If $\vec{T}$ \\
	   & is a linear transformation from $\vec{V}$ into $\vec{W}$, then the following are equivalent:\\
	   & (a). $\vec{T}$ is invertible. \\
	   & (b). $\vec{T}$ is non-singular.\\
	   & (c). $\vec{T}$ is onto, that is, range of $\vec{T}$ is $\vec{W}$.\\
	   \hline	
	\end{tabular}
}
\caption{Results Used}
\label{table:3/3/4/2}
\end{table*}
\begin{table*}[!ht]
\resizebox{2\columnwidth}{!}
{
	\begin{tabular}{|l|l|}
		\hline
		\multirow{3}{*}{Defining} & \\
		& We define set $S$ and set $T$ as\\
		Sets&  $S = \{\brak{a,b}: a,b \in \mathbb{N}, 1 \leq a \leq m, 1 \leq b \leq n \},  \quad T = \{ 1,2,...,mn\}$\\
		& \\
		\hline
		\multirow{3}{*}{Defining} & \\
		& We now define a bijection $\sigma: S \to T$ as\\
		Bijection& \qquad \qquad \brak{a,b} $\to$	\brak{a-1}n + b \\
		& \\
		\hline
		\multirow{3}{*}{Defining} & \\
		& We now define a function $G$ from $F^{m\times n}$ to $F^{mn}$ as follows. Let $\vec{A} \in F^{m\times n}$.\\
		Function $G$ & Then map $\vec{A}$ to the $mn$ tupple that has $\vec{A_{ij}}$ in the $\sigma(i,j)$ position. In other words,\\ 
		& \qquad \qquad \qquad \qquad $\vec{A} \to \brak{\vec{A_{11}},\vec{A_{12}},...,\vec{A_{1n}},...,\vec{A_{m1}},\vec{A_{m2}},...,\vec{A_{mn}}}$\\
		& \\
		\hline
		\multirow{3}{*}{Proving $G$ to} & \\
		& Since, addition in $F^{m\times n}$ and in $F^{mn}$ is performed component-wise, $G\brak{\vec{A+B}} = G(\vec{A}) + G(\vec{B})$\\
	be Linear & and scalar multiplication in $F^{m\times n}$ and in $F^{mn}$ is also defined  as $G\brak{c\vec{A}}= cG\brak{\vec{A}}$. \\
	& \\
		\hline	
		\multirow{3}{*}{Proving $G$ to} & \\
		& $G(\vec{A}) = G(\vec{B})$ \\
		be One-One & $\implies \brak{\vec{A_{11}},\vec{A_{12}},...,\vec{A_{1n}},...,\vec{A_{m1}},\vec{A_{m2}},...,\vec{A_{mn}}}=
		\brak{\vec{B_{11}},\vec{B_{12}},...,\vec{B_{1n}},...,\vec{B_{m1}},\vec{B_{m2}},...,\vec{B_{mn}}}$ \\
		& $ \implies \vec{A_{i,j}} = \vec{B_{ij}} \quad \forall 1 \leq i \leq m ,1 \leq j \leq n $\\
		& $\implies$ $\vec{A}$ = $\vec{B}$\\
		& \\
		\hline
		\multirow{3}{*}{Proving $G$ to} & \\
	    & Since $G$ is one to one, so Null($G$) $=0$. Thus, by Rank-Nullity Theorem dim(Range($G$))$=mn$, \\ 
		be Onto & proving $G$ to be a surjective (onto) map as by Result 1 dimension of $F^{m\times n} = mn$\\
		& \\
		\hline
		\multirow{3}{*}{$F^{m\times n} \cong F^{mn}$} & \\
		& Since $G$ has an inverse and is an isomorphism of $\vec{T}$. Thus, by Result 2\\ 
	    & \qquad \qquad \qquad  $F^{m\times n} \cong F^{mn}$\\
	    & \\
		\hline
	\end{tabular}
}
\caption{Proof}
\label{table:3/3/4/3/}
\end{table*}
$\mathbb{R}^{2\times 2}$ is isomorphic to $\mathbb{R}^{4}$ ie, $\mathbb{R}^{2\times 2}$ $\cong$ $\mathbb{R}^{4}$.
