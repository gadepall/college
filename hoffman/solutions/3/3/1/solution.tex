Let,
\begin{align}
\vec{T} : \vec{V} \rightarrow \mathbb{R}^2\\
\vec{T}(x+iy)=\myvec{x\\y} \label{eq:solutions/3/3/1/eq:1}\\
x,y \in \mathbb{R} \quad i \in \mathbb{C}
\end{align}
Consider two vectors,
\begin{align}
\vec{u},\vec{v} \in \vec{V} 
\end{align}
Using \eqref{eq:solutions/3/3/1/eq:1} if $\vec{T}$ is Linear Transformation. 
\begin{align}
\vec{T}(\vec{u}+c\vec{v})= \vec{T}(\vec{u})+\vec{T}(c\vec{v})=\vec{T}(\vec{u})+c\vec{T}(\vec{v})
\end{align}
Hence this is a Linear transformation. Now, checking if $\vec{T}$ is one-one. let,
\begin{align}
\vec{u}=0 = 0+j0\\
\end{align}
from \eqref{eq:solutions/3/3/1/eq:1},
\begin{align}
\vec{T}(\vec{u})=\myvec{0\\0}\label{eq:solutions/3/3/1/eq:2}
\end{align}
From \eqref{eq:solutions/3/3/1/eq:2} we see $\vec{T}$ is one-one. Now, checking if $\vec{T}$ is onto. let,
\begin{align}
\vec{T}(\vec{u}) =\myvec{a\\b} \label{eq:solutions/3/3/1/eq:3}
\end{align}
where, $a$, $b$ are scalars. Using \eqref{eq:solutions/3/3/1/eq:1}, we see there exists a solution for \eqref{eq:solutions/3/3/1/eq:2} for all $a$, $b$. Hence $\vec{T}$ is onto. Therefore $\vec{T}$ is isomorphic over $\vec{V}$ and $\mathbb{R}^2$
