
\begin{table}[!ht]
\begin{center}
\resizebox{\columnwidth}{!}
{
\begin{tabular}{|c|c|}
\hline
& \\
Given & $\mathcal{T}(T): T \rightarrow UTU^{-1}$\\
& \\
& $U$ is isomorphism of V onto W\\
& that means $U$ is $one-one$\\
& \\
& $\mathcal{T}: L(V,V) \rightarrow L(W,W)$\\
& \\
\hline
&\\
To prove & $\mathcal{T}$ is isomorphism of $L(V,V)$\\
&  onto $L(W,W)$\\
& \\
& It is same as proving\\
& $\mathcal{T}$ is invertible, because \\
& \\
& $isomorphim \implies one-one$\\
& $\implies invertible$ by definition\\
& \\
\hline
& \\
Proof & Consider inverse transformation\\
& $\mathcal{S}: L(W,W) \rightarrow L(V,V)$\\
& $\mathcal{S}: S \rightarrow U^{-1}SU$\\
& \\
& where $U^{-1}SU$ is a composition\\ 
& of 3 linear transformations\\
& $V \xrightarrow{U} W \xrightarrow{S} W \xrightarrow{U^{-1}} V$\\
& \\
& Now consider $\mathcal{S}(UTU^{-1})$,\\
& \\
& $\mathcal{S}(UTU^{-1}) = U^{-1}(UTU^{-1})U = T$\\
& \\
& Similarly consider $\mathcal{T}(U^{-1}SU)$,\\
& \\
& $\mathcal{T}(U^{-1}SU) = U(U^{-1}SU)U^{-1} = S$\\
& \\
& $\implies TS = I \text{ and } ST = I$\\
& \\
& we can say $\mathcal{T}$ is invertible\\
& since we have found an inverse $\mathcal{S}$\\
& \\
& Hence $\mathcal{T}$ is one-one implies \\
& $\mathcal{T}$ isomorphism of $V$ onto $W$\\
&\\
\hline
\end{tabular}
}
\end{center}
\caption{Proof}
\label{table:solutions/3/3/7/2}
\end{table}
{Example}
Let
\begin{equation}
	U = \myvec{1 & 2\\3 & 4}
\end{equation}
here $U$ is an isomorphism from $\mathbb{R}^{2 \times 2}$ to $\mathbb{R}^{2 \times 2}$ since inverse of U exists and
\begin{equation}
	U^{-1} = \myvec{-2 & -\frac{3}{2}\\-1 & -\frac{1}{2}}
\end{equation}
Consider
\begin{equation}
	T = \myvec{-1 & 2\\3 & 1} \in \mathbb{R}^{2 \times 2}
\end{equation}
Now
\begin{align}
	UTU^{-1} &= \myvec{1 & 2\\3 & 4}\myvec{-1 & 2\\3 & 1}\myvec{-2 & -1\\-\frac{3}{2} & -\frac{1}{2}}\\
	&= \myvec{-16 & -7\\-33 & -14} \in \mathbb{R}^{2 \times 2}
\end{align}
Also inverse exists for $T$
\begin{align}
	S = T^{-1} = \myvec{-\frac{1}{2} & \frac{2}{7}\\\frac{3}{7} & \frac{1}{7}}
\end{align}
Since T inverse exists $\mathcal{T}(T) = UTU^{-1}$ is an isomorphism from $\mathbb{R}^{2\times2}$ onto $\mathbb{R}^{2\times2}$.  

\begin{table*}[!ht]
	\begin{center}
\resizebox{2\columnwidth}{!}
{
		\begin{tabular}{|c|c|}
			\hline
			\multirow{3}{*}{linear transformation} & \\
			& Let $V$ and $W$ be vector spaces over field $F$.\\
			& A $\textbf{linear transformation}$ $V$ into $W$ is a function T from V into W such that\\
			& $T(c\alpha + \beta) = c(T\alpha) + T\beta$\\
			& for all $\alpha$ and $\beta$ in $V$ and all scalars in $c$ in $F$.\\
			& \\
			\hline
			\multirow{3}{*}{isomorphism} &\\
			& If $V$ and $W$ are vector spaces over the field $F$, any $one-one$ linear transformation \\
			& $T : V\rightarrow W$ is called $\textbf{isormorphism of}$ $V$ $\textbf{onto}$ $W$\\
			& \\
			\hline			
			\multirow{3}{*}{one-one} & \\
			& A linear transformation $T : \mathbb{R}^{n} \rightarrow \mathbb{R}^{m}$ is said to be $\textbf{one-one}$ if for every  $\vec{b} \in \mathbb{R}^{m}$,\\
			&  $\vec{A}\vec{X} =\vec{b}$ has atmost one solution in $\mathbb{R}^{n}$.\\
			& \\
			& Equivalently, if $T(\vec{u}) = T(\vec{v})$, then $u = v$. \\
			& \\
			& By definition, all $invertible$ transformations are $\textbf{one-one}$\\
			& \\
			\hline
			\multirow{6}{*}{invertible} & \\
			& A linear transformation $T : V \rightarrow W$ is $\textbf{invertible}$ if there exists another\\
			& linear transformation $U: W \rightarrow V$ such that \\
			& $UT$ is the $identity$ transformation on $V$ and $TU$ is the identity transformation on $W$.\\
			& \\
			& $T$ is $\textbf{invertible}$ if and only if $T$ is $one-one$ and $onto$\\
			& \\
			\hline
		\end{tabular}
}
	\end{center}
\caption{Definitions}
\label{table:solutions/3/3/7/1}
\end{table*}
