T : V $\rightarrow$ W is an isomorphism if
 
 (1) T is one one.
 
 (2) T is onto.
\begin{align}
\myvec{1&-2&-1&\sqrt{2}\\0&1+i&1&i\\i&0&1&3}\xrightarrow[]{ref}\myvec{1&0&-i&-3i\\0&2&1-i&i+1\\0&0&0&1}
\end{align}
T is one one over $C^{3}$ if 
\begin{align}
    T\brak{\alpha}=0 \implies \alpha =0
\end{align}
now,
\begin{align}
 \myvec{1&-2&\sqrt{2}\\0&1+i&i\\i&0&3}\alpha = 0
\end{align}
consider the row reduced matrix
\begin{align}
    \myvec{1&-2&\sqrt{2}\\0&1+i&i\\i&0&3}
    \xleftrightarrow[R_3\rightarrow i R_3]{R_3\rightarrow R_3-iR_1}\myvec{1&-2&\sqrt{2}\\0&1+i&i\\0&-2&\sqrt{2}+3i}\\
    \xleftrightarrow[R_3\leftarrow R_3+R_2]{R_2\leftarrow (1-i)R_2}\myvec{1&-2&\sqrt{2}\\0&2&i+1\\0&0&\sqrt{2}+4i+1}
\end{align}
\begin{align}
    \vec{\alpha} = \myvec{0\\0\\0}
\end{align}
Therefore it holds the condition of one one and the rank = no. of pivot columns = 3 (equal to no. of columns).Thus the  vectors are linearly independent hence it is onto . Since T is an isomorphoism onto $C^{3}$. 
  \begin{align}
  T\brak{\alpha_1} = c_1T\brak{\alpha_2}+c_2T\brak{\alpha_3}\label{eq:solutions/3/3/2/a/pr}  
\end{align}
$c_1$ and $c_2$ are scalar.
\begin{align}
 \myvec{1\\0\\i}=c{_1}\myvec{-2\\1+i\\0}+c{_2}\myvec{-1\\1\\1}
\end{align}
\begin{align}
  \myvec{1\\0\\i}=\myvec{-2&-1\\1+i&1\\0&1}\myvec{c{_1}\\c{_2}}
\end{align}
 Now we find $c{_i}$ by row reducing augmented matrix.
\begin{align}
    \myvec{-2&-1&1\\1+i&1&0\\0&1&i}
    \xleftrightarrow[R_2\rightarrow R_3]{R_1\rightarrow -R_1/2}\myvec{1&\frac{1}{2}&-\frac{1}{2}\\0&1&i\\1+i&1&0}\\
    \xleftrightarrow[R_3\leftarrow R_3-(1+i)R_1]{R_1\leftarrow R_1-R_2/2}\myvec{1&0&\frac{-1-i}{2}\\0&1&i\\0&\frac{1-i}{2}&\frac{1+i}{2}}\\
    \xleftrightarrow{R_3\leftarrow R_3-(1-i)/2R_2}\myvec{1&0&\frac{-1-i}{2}\\0&1&i\\0&0&0}
\end{align}
Therefore the coordinate matrix of the vector is 
\begin{align}
    \myvec{c{_1}\\c{_2}}=
     \myvec{\frac{-1-i}{2}\\i}
\end{align}
substituting the $c_{i}$ in \eqref{eq:solutions/3/3/2/a/pr}
\begin{align}
   T\brak{\alpha_1}=-\frac{1+i}{2}T\brak{\alpha_2} +iT\brak{\alpha_3} 
\end{align}
 Hence $\alpha_1$ belongs to the subspace spanned by $\alpha_2$
and $\alpha_3$.

