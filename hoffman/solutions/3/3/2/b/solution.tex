T : V $\rightarrow$ W is an isomorphism if
 
 (1) T is one one.
 
 (2) T is onto.
If $W_1$ and $W_2$ are finite-dimensional subspaces of a vector space V, then $W_1+W_2$ is finite-dimensional and
\begin{align}
    dim (W_1)+ dim(W_2) = dim(W_1 {\displaystyle \cap } W_2) + dim (W_1 + W_2)\label{eq:solutions/3/3/2/c/eqdim}
\end{align}
We have,
\begin{align}
T=\myvec{1&-2&-1&\sqrt{2}\\0&1+i&1&i\\i&0&1&3}&\\
\myvec{1&-2&-1&\sqrt{2}\\0&1+i&1&i\\i&0&1&3}\xrightarrow[]{rref}\myvec{1&0&-i&0\\0&1&\frac{1-i}{2}&0\\0&0&0&1}\label{eq:solutions/3/3/2/c/rref}
\end{align}
T is one one over $C^{3}$ if 
\begin{align}
    T\brak{\alpha}=0 \implies \alpha =0
\end{align}
now,
\begin{align}
 \myvec{1&-2&\sqrt{2}\\0&1+i&i\\i&0&3}\alpha = 0
\end{align}
consider the row reduced matrix
\begin{align}
    \myvec{1&-2&\sqrt{2}\\0&1+i&i\\i&0&3}
    \xleftrightarrow[R_3\rightarrow i R_3]{R_3\rightarrow R_3-iR_1}\myvec{1&-2&\sqrt{2}\\0&1+i&i\\0&-2&\sqrt{2}+3i}\\
    \xleftrightarrow[R_3\leftarrow R_3+R_2]{R_2\leftarrow (1-i)R_2}\myvec{1&-2&\sqrt{2}\\0&2&i+1\\0&0&\sqrt{2}+4i+1}
\end{align}
\begin{align}
    \vec{\alpha} = \myvec{0\\0\\0}
\end{align}
Therefore it holds the condition of one one and the rank = no. of pivot columns = 3 (equal to no. of columns).Thus the  vectors are linearly independent hence it is onto . Hence, T is an isomorphism onto $C^{3}$. \newline
Also, from \eqref{eq:solutions/3/3/2/c/rref}, we observe,
\begin{align}
   T\brak{\alpha_3}=-i T\brak{\alpha_1}+\frac{i-1}{2}T\brak{\alpha_2} 
\end{align}
 Hence $T\brak{\alpha_3}$ belongs to the subspace spanned by $T\brak{\alpha_1}$ and $T\brak{\alpha_2}$.\newline
 Therefore, $\alpha_3$ is in subspace spanned by  $\alpha_1$ and $\alpha_2$. Therefore,
 \begin{align}
     \alpha_3 \in W_1\\
     \implies \alpha_3 \in W_1 {\displaystyle \cap } W_2\label{eq:solutions/3/3/2/c/eq2}
 \end{align}
  Since $T\brak{\alpha_1}$ and $T\brak{\alpha_2}$ are linearly independent, and $T\brak{\alpha_3}$ and $T\brak{\alpha_4}$ are linearly independent, we have,
  \begin{align}
      dim(W_1) = dim(W_2) = 2
  \end{align}
  From \eqref{eq:solutions/3/3/2/c/eqdim},
  \begin{align}
      dim(W_1)+dim(W_2)=dim(W_1 {\displaystyle \cap } W_2)+dim(W_1+W_2)&\\
      dim(W_1+W_2)=3&\\
      \implies dim(W_1 {\displaystyle \cap } W_2)=2+2-3&\\
      \implies dim(W_1 {\displaystyle \cap } W_2)=1&
  \end{align}
  Therefore,
  \begin{align}
      W_1 {\displaystyle \cap } W_2=c\alpha_3
  \end{align}
