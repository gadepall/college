Let $T:\mathbb{V} \xrightarrow{} \mathbb{V}$ be a linear operator, where $\mathbb{V}$ is a finite dimensional vectors space and $U:\mathbb{V} \xrightarrow{} \mathbb{V}$ is also a linear operator such that,
\begin{align}
TU &= I
\end{align}
Where, $I$ is an identity transformation. Now we know that linear transformations are functions. Hence,
\begin{align}
&TU = I \quad{\text{is a function}}\\
\implies& I :\mathbb{V}\xrightarrow{}\mathbb{V}
\end{align}
Such that $T(V) = V$. Defining $TU :\mathbb{V}\xrightarrow{}\mathbb{V}$ to be a linear operator, we have,
\begin{align}
T[U(V_i)] &= V_i \qquad{\text{[$V_i \in \mathbb{V}$]}}
\end{align}
Now we show in the below Table that $T$ is one-one and onto as follows,\\
\begin{table}[h!]
\centering
%\renewcommand\cellalign{lc}
%\makegapedcells
%\resizebox{\columnwidth}{!}
%{
\begin{tabular}{|c|c|} \hline
\textbf{Proof} & \textbf{Conclusion}  \\ \hline
%\makecell
%{
Let $\vec{V_1},\vec{V_2} \in \mathbb{V}$ then, &
         \\If $\vec{V_1} \ne \vec{V_2}$ then, & $T$ is one-one function
         \\$T[U(\vec{V_1})] \ne T[U(\vec{V_2})]$
%} 
&  \\ \hline
%\makecell
%{
$T$ is linear operator on & 
          \\finite dimensional & $T$ is onto function 
          \\ vector space
%} 
& \\\hline
\end{tabular}
%}
\caption{Proof of Invertibility of transformation}
\label{eq:solutions/3/3/9/tab:invert}
\end{table}\\
\\Hence we get from Table \ref{eq:solutions/3/3/9/tab:invert} that, $T$ is invertible.
Hence we get the following,
\begin{align}
TT^{-1} &= I
\end{align}
Where $T^{-1}$ is an inverse function of linear operator $T$. Hence,
\begin{align}
TT^{-1} &= I = TU\\
\implies T^{-1}(TT^{-1}) &= T^{-1}(TU)\\
\implies T^{-1}(I) &= IU\\
\implies T^{-1} &= U\label{eq:solutions/3/3/9/proof}
\end{align}
Hence from \eqref{eq:solutions/3/3/9/proof} it is proven that $T$ is invertible and $T^{-1} = U$
\\
{\em Example: }
Let $D$ be the differential operator $D:\mathbb{V} \xrightarrow{} \mathbb{V}$ where $\mathbb{V}$ is a space of polynomial functions in one variable $x$ over $\mathbb{R}$ as follows,
\begin{align}
D(c_0+c_1x+\dots+c_nx^n) &= c_1+c^{\prime}_2x+\dots+c^{\prime}_nx^{n-1}\label{eq:solutions/3/3/9/See}
\end{align}
We first prove that the vector space $\mathbb{V}$ is infinite dimensional.\\
Suppose to the contrary that $\mathbb{V}$ is finite dimensional vector space and is given by the span of $k$ polynomials in $\mathbb{V}$ as follows,
\begin{align}
span(\mathbb{V}) &= \{p_1,p_2,\dots,p_k\}\label{eq:solutions/3/3/9/basis}
\end{align}
Also let $m$ be the maximum of the degree of these $k$ polynomials in \eqref{eq:solutions/3/3/9/basis}. Now let an element of the vector space $\mathbb{V}$ be,
\begin{align}
cx^{m+1} &\in \mathbb{V}
\end{align}
As maximum degree of the basis of $\mathbb{V}$ is $m$ hence $cx^{m+1}$ cannot be represented by any linear combination of the basis of $\mathbb{V}$. If $\mathbb{F}$ is field corresponding to $\mathbb{V}$ then we have,
\begin{align}
cx^{m+1} &\ne \sum_{i=1}^{k}\alpha_ip_i \quad{\text{[$\alpha_i \in \mathbb{F}$ $\forall i$]}}
\end{align}
Hence, $cx^{m+1}$ is not in the span of ${p_1,p_2,\dots,p_k}$. Hence, $\mathbb{V}$ is infinite dimensional vector space.\\
Next we prove that $D$ is not one-one operator. Let, two different elements from the vector space $\mathbb{V}$ be as follows,
\begin{align}
c_1+x^m &\in \mathbb{V}\\
c_2+x^m &\in \mathbb{V}
\end{align}
From definition \eqref{eq:solutions/3/3/9/See} of operator $D$ we have,
\begin{align}
D(c_1+x^m) &= mx^{m-1}\label{eq:solutions/3/3/9/a1}\\
D(c_2+x^m) &= mx^{m-1}\label{eq:solutions/3/3/9/a2}
\end{align}
From \eqref{eq:solutions/3/3/9/a1} and \eqref{eq:solutions/3/3/9/a2},
\begin{align}
c_1+x^m &\ne c_2+x^m\\
D(c_1+x^m) &= D(c_2+x^m)\label{eq:solutions/3/3/9/Not_Oneone}
\end{align}
Hence from \eqref{eq:solutions/3/3/9/Not_Oneone} we see that $D$ is not One-One operator.\\
And, $U:\mathbb{V} \xrightarrow{} \mathbb{V}$ is another linear operator such that,
\begin{align}
U(c_0+c_1x+\dots+c_nx^n) &= c_0x+c_1\frac{x^2}{2}+\dots+c_n\frac{x^{n+1}}{n+1}
\end{align}
Now, $DU:\mathbb{V} \xrightarrow{} \mathbb{V}$ is a linear operator such that,
\begin{align}
&DU(c_0+c_1x+\dots+c_nx^n) \\
&= D[U(c_0x+c_1\frac{x^2}{2}+\dots+c_n\frac{x^{n+1}}{n+1})]\\
&= D[c_0x+c_1\frac{x^2}{2}+\dots+c_n\frac{x^{n+1}}{n+1}]\\
&= c_0+c_1\frac{2x}{2}+\dots+c_n\frac{(n+1)x^{n}}{n+1}\\
&= c_0+c_1x+\dots+c_nx^n\label{eq:solutions/3/3/9/eq1}
\end{align}
Hence, from \eqref{eq:solutions/3/3/9/eq1},
\begin{align}
DU = I\label{eq:solutions/3/3/9/show1}
\end{align}
Again $UD:\mathbb{V} \xrightarrow{} \mathbb{V}$ is a linear operator such that,
\begin{align}
&UD(c_0+c_1x+\dots+c_nx^n) \\
&= U[D(c_0x+c_1\frac{x^2}{2}+\dots+c_n\frac{x^{n+1}}{n+1})]\\
&= U[c_1+c^{\prime}_2x+\dots+c^{\prime}_nx^{n-1}]\\
&= c_1x+c_2\frac{x^2}{2}+\dots+c_n\frac{x^{n}}{n}\label{eq:solutions/3/3/9/eq2}\
\end{align}
Hence, from \eqref{eq:solutions/3/3/9/eq2},
\begin{align}
UD \ne I\label{eq:solutions/3/3/9/show2}
\end{align}
Hence, from \eqref{eq:solutions/3/3/9/show1} and \eqref{eq:solutions/3/3/9/show2}, $D$ is not invertible.
