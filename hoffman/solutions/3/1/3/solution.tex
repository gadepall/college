Let the vector space of n-dimension be deined as
\begin{align}
        \vec{V}=\cbrak{f:\vec{F}\rightarrow \vec{F} : f(x)=\sum_{k=0}^{n}c_kx^k, \ c_k \in \vec{F} }
\end{align}
The corresponding standard basis for $\vec{V}$ is
\begin{align}
	\cbrak{\myvec{1\\0\\\vdots\\0},\myvec{0\\x\\\vdots\\0},\cdots,\myvec{0\\0\\\vdots\\x^{n-1}}}
\end{align}
\begin{enumerate}[label=\emph{\alph*)}]
\item
Let $f$ and $g \in \vec{V}$ and let $\alpha$ and $\beta \in \vec{F}$ then 
		\begin{align}
			\vec{D}(\alpha f + \beta g)&=\frac{d(\alpha f(x) + \beta g(x))}{dx} \\
			&=\alpha\frac{df(x)}{dx}+\beta\frac{dg(x)}{dx}\\
			&=\alpha(\vec{D}f)+\beta(\vec{D}g)
		\end{align}
Therefore $\vec{D}$ is a linear transformation.\\
The $\vec{D}$ transformation maps the $k^{th}$ basis vector as follows
		\begin{align}
			\vec{D}\myvec{0\\\vdots\\0\\x^k\\\vdots\\0}
			=\myvec{0\\\vdots\\kx^{k-1}\\0\\\vdots\\0}
		\end{align}
Since the transformed vector 
		\begin{align}
			\myvec{0\\\vdots\\kx^{k-1}\\0\\\vdots\\0} \in \vec{V}
		\end{align}
the range of $\vec{D}$ is the vector space $\vec{V}$. Thus the transformation is defined as
$\vec{D}:\vec{V} \rightarrow \vec{V}$.
Therefore the $\vec{D}$ Transformation on the basis vector set is
		\begin{align}
			\vec{D}\myvec{1 & 0 & 0 & \cdots & 0 & 0\\
				      0 & 1 & 0 & \cdots & 0 & 0 \\
				      0 & 0 & 1 & \cdots & 0 & 0 \\
				      \vdots & \vdots & \vdots & \vdots & \vdots & \vdots\\
				      0 & 0 & 0 & \cdots & 1 & 0 \\
				      0 & 0 & 0 & \cdots & 0 & 1}\\
			      =\myvec{0 & 1 & 0 & \cdots & 0 & 0\\
				      0 & 0 & 2 & \cdots & 0 & 0\\
				      0 & 0 & 0 & \cdots & 0 & 0\\
				      \vdots & \vdots & \vdots & \vdots & \vdots & \vdots\\
				      0 & 0 & 0 & \cdots & 0 & n-2 \\
				      0 & 0 & 0 & \cdots & 0 & 0}
		\end{align}
Thus the $\vec{D}$ transformation coefficient matrix is		
\begin{align}
	D=\myvec{0 & 1 & 0 & \cdots & 0 & 0\\
                 0 & 0 & 2 & \cdots & 0 & 0\\
                 0 & 0 & 0 & \cdots & 0 & 0\\
                \vdots & \vdots & \vdots & \vdots & \vdots & \vdots\\
                 0 & 0 & 0 & \cdots & 0 & n-2 \\
                 0 & 0 & 0 & \cdots & 0 & 0}
\end{align}
Since $D$ contains a zero row hence $\mydet{D}=0$. Therefore $\vec{D}$ transformation matrix is 
singular. The nullspace for differentiation transformation is defined as
\begin{align}
        \vec{N}=\cbrak{f \in \vec{V} : \vec{D}f=0} 
\end{align}
		Let the coefficient matrix of $f \in \vec{V}$ be 
		\begin{align}
			\vec{f}=\myvec{c_0\\c_1\\\vdots\\c_{n-1}}
		\end{align}
		then
		\begin{align}
			\vec{D}f&=0 \\
			\implies
			&\myvec{0 & 1 & 0 & \cdots & 0 & 0\\
                 0 & 0 & 2 & \cdots & 0 & 0\\
                 0 & 0 & 0 & \cdots & 0 & 0\\
                \vdots & \vdots & \vdots & \vdots & \vdots & \vdots\\
                 0 & 0 & 0 & \cdots & 0 & n-2 \\
		0 & 0 & 0 & \cdots & 0 & 0}\myvec{c_0\\c_1\\\vdots\\c_{n-1}}=\vec{0} \label{eq:solutions/3/1/3/dx}
		\end{align}
Since $D$ is in row reduced echolon form and $rank(D)=n-1$ the solution of (\ref{eq:solutions/3/1/3/dx}) is
\begin{align}
\vec{f}=\myvec{k\\0\\\vdots\\0}
\end{align}
where $k \in \vec{R}$. Therefore the nullspace for \\$\vec{D}:\vec{V}\rightarrow\vec{V}$ is
\begin{align}
	\vec{N}=\cbrak{\myvec{k\\0\\\vdots\\0}:k \in \vec{R}}
\end{align}
\item
Let $f$ and $g \in \vec{V}$ and let $\alpha$ and $\beta \in \vec{F}$ then
\begin{align}
	\vec{T}(\alpha f + \beta g)&=\int_{0}^{x}(\alpha f(t) + \beta g(t))\,dt\\
	&=\alpha\int_{0}^{x} f(t)\,dt+\beta\int_{0}^{x} g(t)\,dt\\
     &=\alpha(\vec{T}f)+\beta(\vec{T}g)
\end{align}
Therefore $\vec{T}$ is a linear transformation.\\
The $\vec{T}$ transformation maps the $k^{th}$ basis vector as follows
\begin{align}
      \vec{T}\myvec{0\\ \vdots \\ x^k \\0 \\ \vdots \\0}
      =\myvec{0\\ \vdots \\ 0\\ \frac{x^{k+1}}{k+1} \\ \vdots \\0}
\end{align}
Since the transformed vector
\begin{align}
	\myvec{0\\ \vdots \\ 0\\ \frac{x^{k+1}}{k+1} \\ \vdots \\0} \in \vec{V}
\end{align}
the range of $\vec{T}$ is the vector space $\vec{V}$. Thus the transformation is defined as 
$\vec{T} : \vec{V}\rightarrow \vec{V}$.
Therefore the $\vec{T}$ Transformation on the basis vector set is
                \begin{align}
                        \vec{T}\myvec{1 & 0 & 0 & \cdots & 0 & 0\\
                                      0 & 1 & 0 & \cdots & 0 & 0 \\
                                      0 & 0 & 1 & \cdots & 0 & 0 \\
                                      \vdots & \vdots & \vdots & \vdots & \vdots & \vdots\\
                                      0 & 0 & 0 & \cdots & 1 & 0 \\
                                      0 & 0 & 0 & \cdots & 0 & 1} \\
                              =\myvec{0 & 0 & 0 & \cdots & 0 & 0\\
                                      1 & 0 & 0 & \cdots & 0 & 0\\
				      0 & \frac{1}{2} & 0 & \cdots & 0 & 0\\
                                      \vdots & \vdots & \vdots & \vdots & \vdots & \vdots\\
				      0 & 0 & 0 & \cdots & 0 & 0 \\
				      0 & 0 & 0 & \cdots & \frac{1}{n-1} & 0 \\
				      0 & 0 & 0 & \cdots & 0 & \frac{1}{n} }
                \end{align}
Thus the $\vec{T}$ transformation coefficient matrix is
\begin{align}
	                     T=\myvec{0 & 0 & 0 & \cdots & 0 & 0\\
                                      1 & 0 & 0 & \cdots & 0 & 0\\
                                      0 & \frac{1}{2} & 0 & \cdots & 0 & 0\\
                                      \vdots & \vdots & \vdots & \vdots & \vdots & \vdots\\
                                      0 & 0 & 0 & \cdots & 0 & 0 \\
                                      0 & 0 & 0 & \cdots & \frac{1}{n-1} & 0 \\
                                      0 & 0 & 0 & \cdots & 0 & \frac{1}{n} } 
\end{align}
Since $T$ contains a zero row hence $\mydet{T}=0$. Therefore $\vec{T}$ transformation matrix is 
singular.
The nullspace for integration transformation is defined as
\begin{align}
        \vec{N}=\cbrak{f \in \vec{V} : \vec{T}f=0}
\end{align}
                Let the coefficient matrix of $f \in \vec{V}$ be
                \begin{align}
                        \vec{f}=\myvec{c_0\\c_1\\\vdots\\c_{n-1}}
                \end{align}
                then
\begin{align}
	&\vec{T}f=0 \\
	\implies 
	&\myvec{0 & 0 & 0 & \cdots & 0 & 0\\
               1 & 0 & 0 & \cdots & 0 & 0\\
               0 & \frac{1}{2} & 0 & \cdots & 0 & 0\\
               \vdots & \vdots & \vdots & \vdots & \vdots & \vdots\\
               0 & 0 & 0 & \cdots & 0 & 0 \\
               0 & 0 & 0 & \cdots & \frac{1}{n-1} & 0 \\
               0 & 0 & 0 & \cdots & 0 & \frac{1}{n} }
	       \myvec{c_0\\c_1\\\vdots\\c_{n-1}}=\vec{0} \label{eq:solutions/3/1/3/tx}
\end{align}
Since $T$ is in row reduced echolon form and $rank(T)=n$ the solution of (\ref{eq:solutions/3/1/3/tx}) is
\begin{align}
\vec{f}=\myvec{0\\0\\\vdots\\0}
\end{align}
where $k \in \vec{R}$. Therefore the nullspace for \\$\vec{T}:\vec{V}\rightarrow\vec{V}$ is
\begin{align}
        \vec{N}=\cbrak{\myvec{0\\0\\\vdots\\0}:k \in \vec{R}}
\end{align}
\end{enumerate}
