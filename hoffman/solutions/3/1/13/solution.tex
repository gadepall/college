\begin{table}[h!]
\begin{center}
\begin{tabular}{|c|c|}
\hline
& \\
Given & $\vec{T}:\mathbb{V} \rightarrow \mathbb{V}$\\
&\\
\hline
& \\
To prove & a) $ range(\vec{T}) \cap null space(\vec{T}) = \left\lbrace 0 \right\rbrace$\\
& \\
& b) If $\vec{T}(\vec{T}\alpha) = 0$, then $\vec{T}\alpha = 0$.\\
& \\
\hline
\end{tabular}
\end{center}
\end{table}
\begin{table}[h!]
\begin{center}
\begin{tabular}{|c|c|}
\hline
& \\
Proof(a) & Let $\vec{x} \in \mathbb{V}$\\
& and \\
& $\vec{x} \in range(\vec{T}) \cap null space(\vec{T})$\\
& then,\\
& $\vec{x} \in range(\vec{T})$\\
& $\vec{x} \in null space(\vec{T})$\\
& \\
& Consider $\vec{y} \in \mathbb{V}$ whose\\
& linear transformation into $\mathbb{V}$ is $\vec{x}$.\\
& $\implies \vec{T}(\vec{y}) = \vec{x}$\\
& \\
& since $\vec{x} \in $ null space$(\vec{T})$ \\
& and the sub space is linearly independent \\
& $\vec{T}(\vec{x}) = 0$\\
& from above equations \\
& $\vec{T}(\vec{T}(y)) = 0$\\
& \\
& from the definition of linear\\
& transformation of independent vector space\\
& $\vec{T}(y) = 0$\\
& $\implies \vec{x} = 0$\\
& $\implies \left\lbrace 0 \right\rbrace \subseteq range(\vec{T}) \cap null space(\vec{T})$ \\
& $\therefore range(\vec{T}) \cap null space(\vec{T}) = \left\lbrace 0 \right\rbrace$\\
& Hence Proved.\\
& \\
\hline
& \\
Proof(b) & If $\vec{T}(\vec{T}\alpha) = 0$\\
& then, from the definition of linear\\
& transformation, $\vec{T}\alpha$ will \\
& be in the null space of linear \\
& transformation $\vec{T}$ and is linearly \\
& independent \\
& $\therefore \vec{T}\alpha = 0$\\
& \\
\hline
\end{tabular}
\end{center}
\end{table}

\begin{table}[h!]
\begin{center}
\begin{tabular}{|c|c|}
\hline
& \\
Eg: & Let $\alpha \in \mathbb{V}$ and\\
& \\
& $\alpha = \myvec{1&7&-1&-1\\-1&1&2&1\\4&-2&0&-4\\2&3&4&-2}$\\
& \\
& linear transformation of $\alpha$ into $\mathbb{V}$\\
& $\vec{T}(\alpha) = c\alpha$\\
& \\
& then row reduced echelon form of $\vec{T}$ is\\
& $rref(\vec{T}) = \myvec{1&0&0&-1\\0&1&0&0\\0&0&1&0\\0&0&0&0}$\\
& \\
& $\implies rank(\vec{T}) = 3,$\\
& $nullity(\vec{T}) = 1$\\
& \\
& $\implies range(\vec{T}) = \myvec{1&7&-1\\-1&1&2\\4&-2&0\\2&3&4},$\\
& $null space(\vec{T}) = \myvec{1\\0\\0\\1}$\\
& \\
& $\therefore range(\vec{T}) \cap null space(\vec{T}) = \left\lbrace 0 \right\rbrace$\\
& \\
& Hence proved that the intersection \\
& of the range of $\vec{T}$ and \\
& null space of $\vec{T}$ is the zero \\
& subspace of $\mathbb{V}$.\\
& \\
\hline
\end{tabular}
\end{center}
\end{table}

