Two matrices $\vec{A}$ and $\vec{B}$ are said to be similar iff there exists a invertible matrix $\vec{P}$ such that:
\begin{align} \label{eq:solutions/3/4/8/eq:2_1}
    \vec{B} = \vec{P^{-1}}\vec{A}\vec{P}
\end{align}
Let,
\begin{align}
    \vec{A} = \myvec{\cos{\theta} & -\sin{\theta} \\ \sin{\theta} & \cos{\theta}}; \quad
    \vec{B} = \myvec{e^{i\theta} & 0 \\ 0 & e^{-i\theta}} \label{eq:solutions/3/4/8/eq:3_1}
\end{align}
Finding the characteristic polynomial of $\vec{A}$,
\begin{align}
    \mydet{A-\lambda I} &= 
    \mydet{\cos{\theta}-\lambda & -\sin{\theta} \\ 
    \sin{\theta} & \cos{\theta}-\lambda} \nonumber \\
    &= (\cos{\theta-\lambda})^2 + \sin^2{\theta} \nonumber \\
    &= 1+\lambda^2+2\lambda\cos{\theta} \label{eq:solutions/3/4/8/eq:3_2}
\end{align}
The eigenvalues can be calculated by equating the characteristic polynomial to zero. The eigenvalues are,
\begin{align}
    \lambda_1 = \cos{\theta} + i \sin{\theta}; \:
    \lambda_2 = \cos{\theta} - i \sin{\theta} \label{eq:solutions/3/4/8/eq:3_3}
\end{align}
The eigenvectors corresponding to \eqref{eq:solutions/3/4/8/eq:3_3} are,
\begin{align}
    \alpha_1 = \myvec{i \\ 1}; \: \alpha_2 = \myvec{-i \\ 1} \label{eq:solutions/3/4/8/eq:3_4} \\
    \vec{P} = \myvec{\alpha_1 & \alpha_2} = \myvec{i & -i \\ 1 & 1} \label{eq:solutions/3/4/8/eq:3_5} 
\end{align}
Now,
\begin{align}
    \vec{P^{-1}}\vec{A}\vec{P} &= \frac{1}{2i} \myvec{1 & i \\ -1 & i} \myvec{\cos{\theta} & -\sin{\theta} \\ \sin{\theta} & \cos{\theta}} \myvec{i & -i \\ 1 & 1} \nonumber \\
    &= \myvec{\cos{\theta} + i \sin{\theta} & 0 \\ 0 & \cos{\theta} - i \sin{\theta}} \nonumber \\
    &= \myvec{e^{i\theta} & 0 \\ 0 & e^{-i\theta}} = \vec{B} \label{eq:solutions/3/4/8/eq:3_6}
\end{align}
Hence, from \eqref{eq:solutions/3/4/8/eq:3_6}, $\vec{A}$ and $\vec{B}$ are similar matrices. 
