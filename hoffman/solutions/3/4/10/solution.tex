If a linear operator $\vec{S}$ is defined on $\vec{R}^2$ such that $\vec{S}^2=\vec{S}$, then
\begin{align}
	\vec{S}^2-\vec{S}=\vec{0}\\
	\vec{S}(\vec{S}-\vec{I})=\vec{0}\\
	\implies\vec{S}=\vec{0},\vec{S}=\vec{I}
\end{align}
The transformation of a vector $\vec{x} \in \vec{R}^2$ can be represented as
\begin{align}
	\vec{S}\vec{x}=\vec{y}\\
	\implies \vec{S}(\vec{S}\vec{x})=\vec{S}\vec{y}\\
	\implies \vec{S}^2\vec{x}=\vec{S}\vec{y}\\
	\implies \vec{S}\vec{x}=\vec{S}\vec{y}\\
	\implies \vec{x}=\vec{y}
\end{align}
Therefore the transformation of a vector $\vec{x} \in \vec{R}^2$ can be given as
\begin{align}
	\vec{S}\vec{x}=\vec{x} \ \forall \ \vec{x} \in \vec{R}^2 \label{eq:solutions/3/4/10/sx}
\end{align}
Consider the ordered basis set
\begin{align}
        B=\cbrak{\epsilon_1,\epsilon_2} \in \vec{R}^2
\end{align}
and if
\begin{align}
\vec{[S]_B}=\vec{A} \\
        \implies \vec{[S]_B}=\myvec{1&0\\0&0}
\end{align}
Thus we can re-write the column vectors of $\vec{[S]_B}$ using (\ref{eq:solutions/3/4/10/sx}) as
\begin{align}
\vec{S}\myvec{1\\0}=\myvec{1\\0}=1\myvec{1\\0}+0\myvec{0\\1} \\
\vec{S}\myvec{0\\0}=\myvec{0\\0}=0\myvec{1\\0}+0\myvec{0\\1}
\end{align}
Therefore, any vector $\vec{x}$ in column space of $\vec{[S]_B}$ can be uniquely expressed by 
$\cbrak{\myvec{1\\0}}$, hence it forms the basis for column space of $\vec{[S]_B}$. Therefore 
one of the basis vector of $B$ is $\myvec{1\\0}$. The other basis vector can be any vector which 
is linearly independent to $\myvec{1\\0}$. One of the ordered basis set can be 
\begin{align}
	B=\cbrak{\myvec{1\\0},\myvec{1\\1}}
\end{align}
