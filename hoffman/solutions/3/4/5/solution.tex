The basis of the range of linear transformation $T$ is the basis of the column-space of $\vec{A}$ or basis of $C(\vec{A})$. Hence the basis of the range of the linear transformation $T$ is derived by reducing $\vec{A}$ into Reduced-Row Echelon form as follows,
\begin{align}
\myvec{1&2&1\\0&1&1\\-1&3&4}&\xleftrightarrow[R_1=R_1-2R_2]{R_3=R_3+R_1}\myvec{1&0&-1\\0&1&1\\0&5&5}\\
&\xleftrightarrow{R_3=R_3-5R_2}\myvec{1&0&-1\\0&1&1\\0&0&0}\label{eq:solutions/3/4/5/eq1}
\end{align}
From \eqref{eq:solutions/3/4/5/eq1} the basis of the range of linear operator $T$ are as follows,
\begin{align}
\vec{a_1} &= \myvec{1&0&0}\\
\vec{a_2} &= \myvec{0&1&0}
\end{align}
Again, the basis for null-space of linear operator $T$ or $N(\vec{A})$ is a solution of the equation $\vec{Ax} = 0$. From \eqref{eq:solutions/3/4/5/eq1} we have,
\begin{align}
\vec{Ax} &= 0\\
\implies\myvec{1&0&-1\\0&1&1\\0&0&0}\myvec{x_1\\x_2\\x_3} &= 0 \label{eq:solutions/3/4/5/eq2}
\end{align}
Setting the value of the free variable $x_3 = 1$ we get the solution,
\begin{align}
\vec{x} &= \myvec{1\\-1\\1}
\end{align}
Hence, the basis of the null-space of the linear operator $T$ is given by,
\begin{align}
\vec{b} &= \myvec{1&-1&1}
\end{align}
