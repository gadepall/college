Given,
\begin{equation}\label{eq:solutions/3/4/13/eq1}
	T = \displaystyle \sum_{p=1}^m \displaystyle \sum_{q=1}^n A_{pq}E^{p,q}
\end{equation}
where
\begin{align}
	E^{p,q}(\alpha_i) &=
	\begin{cases}
	\beta_p & p=i\\
	0 & \text{otherwise}
	\end{cases}\\
	\label{eq:solutions/3/4/13/eq3}&= \delta_{iq}\beta_p
\end{align}
where $\mathcal{B} = \{\alpha_1, \dots, \alpha_n\}$ is basis of $V$ and $\mathcal{B}' = \{\beta_1, \dots, \beta_n\}$ is basis of $W$.
\\
\\
Consider a vector $\vec{x} = \{x_1, x_2, \dots, x_n\} \in V$,
\begin{align}\label{eq:solutions/3/4/13/eq4}
	\vec{x} &= \displaystyle\sum_{q=1}^{n}x_q \vec{\alpha}_q\\
	\therefore E^{p,q}(\vec{x}) &= \displaystyle\sum_{q=1}^{n}x_qE^{p,q}(\alpha_q)\\
	\label{eq:solutions/3/4/13/eq6}&= x_q \delta_{iq}\beta_p
\end{align}
Consider $T(\vec{x})$, from \eqref{eq:solutions/3/4/13/eq1}
\begin{align}\label{eq:solutions/3/4/13/eq7}
	T(\vec{x}) 
	&= \displaystyle \sum_{p=1}^m \displaystyle \sum_{q=1}^n A_{pq}E^{p,q}(\vec{x})
\end{align}
Substitute \eqref{eq:solutions/3/4/13/eq6} in \eqref{eq:solutions/3/4/13/eq7}
\begin{align}\label{eq:solutions/3/4/13/eq8}
	T(\vec{x}) &= \displaystyle \sum_{p=1}^m \displaystyle \sum_{q=1}^n A_{pq}x_p \delta_{iq}\beta_q
\end{align}
\\
From \eqref{eq:solutions/3/4/13/eq3}, $\delta_{iq}\beta_q$ is the transformation of basis of $V$ to $W$. Hence $T: V \rightarrow W$ is\\
\begin{align}
	T &= \myvec{\sum_{p=1}^nA_{p1}x_p\\\sum_{p=1}^nA_{p2}x_p\\\vdots\\\sum_{p=1}^nA_{pm}x_p}\\
	T &= \myvec{A_{11} & A_{12} & \dots & A_{1n}\\
	A_{21} & A_{22} & \dots & A_{2n}\\
	\vdots & \vdots & \dots & \vdots\\
	A_{m1} & A_{m2} & \dots & A_{mn}}\myvec{x_1\\x_2\\\vdots\\x_n}
\end{align}
\\\\
$\therefore$ We showed that the matrix $A$ with entries $A(p,q) = A_{pq}$ is precisely the matrix of $T$ relative to the pair $\mathcal{B},\mathcal{B}'$.
