The linear functional $f$ on $\mathbb{F}^2$ is defined by,
\begin{align}
f(\vec{x}) &= \vec{a^T}\vec{x}
\quad{\text{$\forall \vec{x} \in \mathbb{F}^2$}}\label{eq:solutions/3/7/1/c/def}
\end{align}
where,
\begin{align}
\vec{a} &= \myvec{a\\b}
\end{align}
We use the following theorem,\\
Let $\mathbb{V}$ and $\mathbb{W}$ be vector spaces, over the field $F$. For each linear transformation $T: \mathbb{V} \xrightarrow{} \mathbb{W}$, there is a unique linear transformation $T^t: \mathbb{W}^* \xrightarrow{} \mathbb{V}^*$ such that,
\begin{align}
(T^tg)(\alpha) &= g(T\alpha)\label{eq:solutions/3/7/1/c/theorem}
\end{align}
$\forall (\vec{x}) \in \mathbb{F}^2$ the given linear operator $T$ defined as,
\begin{align}
T(\vec{x}) &= \vec{A}\vec{x}
\end{align}
Where,
\begin{align}
\vec{A}=\myvec{1&-1\\1&1}\label{eq:solutions/3/7/1/c/A}
\end{align}
Hence,
\begin{align}
\vec{A}\vec{x}=\myvec{1&-1\\1&1}\vec{x} \label{eq:solutions/3/7/1/c/qn}
\end{align}
Consider the following mapping,
\begin{align}
g = T^tf\label{eq:solutions/3/7/1/c/map}
\end{align}
Now,
\begin{align}
g(\vec{x}) &= T^tf(\vec{x})&\label{eq:solutions/3/7/1/c/ABC}
\end{align}
Using \eqref{eq:solutions/3/7/1/c/theorem} in \eqref{eq:solutions/3/7/1/c/ABC},
\begin{align}
&= f(T(\vec{x}))\\
&= \vec{a^T}\vec{A}\vec{x}
\end{align}
Substituting \eqref{eq:solutions/3/7/1/c/A},
\begin{align}
&= \vec{a^T}\myvec{1&-1\\1&1}\vec{x}\\
&= a(x_1-x_2)+b(x_1+x_2) \label{eq:solutions/3/7/1/c/ans}
\end{align}

