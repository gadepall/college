
 The linear operator \textbf{T} can be represented as a matrix $\vec{A}$ as follows
 \begin{align}
\vec{A} = \myvec{0 & -1 \\ 1 & 0}\\
\end{align}
Let suppose,
\begin{align}
\vec{X_1} = \myvec{x_1 \\ x_2}, \vec{X_2} = \myvec{-x_2 \\ x_1}\\
\intertext{And,}
\vec{U} = \myvec{a & b}
\end{align}


\begin{align}
\textbf{T}(x_1, x_2) = \vec{A}\vec{X_1}\\
\implies \vec{A}\vec{X_1} = \myvec{0 & -1 \\ 1 & 0}\vec{X_1} = \vec{X_2}
\intertext{And \eqref{eq:solutions/3/7/1/b/1.2} can be written as}
f(x_1,x_2) = \vec{U}\vec{X_1}
 \end{align}
Now, we have given,
\begin{align}
g = \textbf{T}^{t}f\\ 
\implies g(x_1, x_2) = \textbf{T}^{t} f(x_1, x_2)\label{eq:solutions/3/7/1/b/2.5}
\end{align}
We know that, if \textbf{V} and \textbf{W} be vector spaces over the field $\mathbb{F}$. For each linear transformation \textbf{T} from \textbf{V} into \textbf{W}, there is a unique linear transformation
$\textbf{T}^{t}$ from $\textbf{W*}$ into $\textbf{V*}$ such that,

\begin{align}
(\textbf{T}^{t}g)(\alpha)  = g (\textbf{T}\alpha) \label{eq:solutions/3/7/1/b/2.7}
\end{align}
Where for every $g$ in $\textbf{W*}$ and $\alpha$ in \textbf{V}.\\

Now using \eqref{eq:solutions/3/7/1/b/2.7} in \eqref{eq:solutions/3/7/1/b/2.5} we can write,
\begin{align}
 \textbf{T}^{t} f(x_1, x_2) = f(\textbf{T}(x_1, x_2)) \label{eq:solutions/3/7/1/b/2.8}\\
 \implies f(\textbf{T}(x_1, x_2)) = \vec{U}\vec{A}\vec{X_1}\\
f(\textbf{T}(x_1, x_2)) = \myvec{a & b}\myvec{0 & -1 \\ 1 & 0}\myvec{x_1 \\x_2} = \vec{U}\vec{X_2}\\
\implies f(\textbf{T}(x_1, x_2)) = -ax_2 + bx_1 \label{eq:solutions/3/7/1/b/2.10}
\end{align}
Hence,
\begin{multline}
%\intertext{Combining \eqref{eq:solutions/3/7/1/b/2.5}, \eqref{eq:solutions/3/7/1/b/2.8} and \eqref{eq:solutions/3/7/1/b/2.10} we have}
 % g(x_1, x_2) = -ax_2 + bx_1
 f(\textbf{T}(x_1, x_2)) = -ax_2 + bx_1 =  g(x_1, x_2) \\ = \myvec{a & b}\myvec{-x_2 \\ x_1} = \vec{U}\vec{X_2}
  \end{multline}


