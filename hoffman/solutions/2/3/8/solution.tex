Every 2$\times$2 matrix may be written as
\begin{align}
	\myvec{a & b\\c & d} = a\myvec{1 & 0\\0 & 0} + b\myvec{0 & 1\\0 & 0} + c\myvec{0& 0\\1 & 0} + d\myvec{0 & 0\\0 & 1}
\end{align}
This shows that 
\begin{align}
	\cbrak{\vec{A}_1, \vec{A}_2, \vec{A}_3, \vec{A}_4} = \cbrak{\myvec{1 & 0\\0 & 0}, \myvec{0 & 1\\0 & 0}, \myvec{0& 0\\1 & 0}, \myvec{0 & 0\\0 & 1}}
\end{align}
can be the basis for the space $\vec{V}$ of all $2\times2$ matrices. However $\vec{A}_2$ and $\vec{A}_3$ doesn't satisfy the property of $\vec{A}^2 = \vec{A}$. Consider b = 0 and c = 0, then the matrix 
\begin{align}
	\myvec{a & 0\\ 0& d} 
\end{align}
can't be a basis as it is the linear combination of $\vec{A}_1$ and $\vec{A}_4$. Hence either b or c or both must be non zero. Hence,
\begin{align}
	\vec{A}_2 &= \myvec{1 & 0\\1 & 0}\\
	\vec{A}_3 &=  \myvec{0 & 1\\0 & 1}
\end{align}
Here, $\vec{A}_2^2 = \vec{A}_2$ and $\vec{A}_3^2 = \vec{A}_3$. Therefore the basis can be
\begin{align}
	\cbrak{\vec{A}_1, \vec{A}_2, \vec{A}_3, \vec{A}_4} = \cbrak{\myvec{1 & 0\\0 & 0}, \myvec{1 & 0\\1 & 0}, \myvec{0& 1\\0 & 1}, \myvec{0 & 0\\0 & 1}}
\end{align}
$\cbrak{\vec{A}_1, \vec{A}_2, \vec{A}_3, \vec{A}_4}$ forms the basis, iff they are linearly independent and the linear combination of them span the space $\vec{V}$. To show that they are linearly independent, we show that the equation has a trivial solution.
\begin{align}
	a\myvec{1 & 0\\0 & 0} + b\myvec{1 & 0\\1 & 0} + c\myvec{0& 1\\0 & 1} + d\myvec{0 & 0\\0 & 1} &= \myvec{0 & 0 \\ 0 & 0}\\
	\implies a + b &= 0\\ 
	b &= 0\\
	c &= 0\\ 
	c + d &= 0
\end{align}
The corresponding matrix form is $\vec{Ax} = 0$
\begin{align}
	\myvec{1 & 1& 0 & 0\\ 0 & 0 & 1 & 0\\ 0 & 1 & 0 & 0\\0 & 0 & 1 & 1}\myvec{a\\b\\c\\d} = \myvec{0\\0\\0\\0}
\end{align}
Row reducing the augmented matrix,
\begin{align}
	\myvec{1 & 1& 0 & 0& 0\\ 0 & 0 & 1 & 0& 0\\ 0 & 1 & 0 & 0 &0\\0 & 0 & 1 & 1 & 0}
	\xleftrightarrow[R_4 \leftarrow R_4 - R_3]{R_2 \xleftrightarrow{} R_3}
	\myvec{1 & 1& 0 & 0& 0\\ 0 & 1 & 0 & 0& 0\\ 0 & 0 & 1 & 0 &0\\0 & 0 & 0 & 1 & 0}\\
	\xleftrightarrow{R_1 \leftarrow R_1 - R_2}
	\myvec{1 & 0& 0 & 0& 0\\ 0 & 1 & 0 & 0& 0\\ 0 & 0 & 1 & 0 &0\\0 & 0 & 0 & 1 & 0}
\end{align}
Therefore, a = b = c = d = 0. Hence the matrices are linearly independent. To show that the linear combination of $\cbrak{\vec{A}_1, \vec{A}_2, \vec{A}_3, \vec{A}_4} $ span the space $\vec{V}$, consider an arbitrary matrix,
\begin{align}
	\myvec{w & x\\ y & z}
\end{align} 
Compute a, b, c, d such that
\begin{align}
	\myvec{w & x\\ y & z} &= a\myvec{1 & 0\\0 & 0} + b\myvec{1 & 0\\1 & 0} + c\myvec{0& 1\\0 & 1} + d\myvec{0 & 0\\0 & 1} \label{eq:solutions/2/3/8/eq:eq1}\\
	&= \myvec{a + b & c \\ b & c + d}
\end{align}
Equating the entries, this produces system of linear equations,
\begin{align}
	a + b &= w, y = b, x = c, z = c + d\\
	\implies a &= w - y\\
	 b &= y\\
	  c &= x\\
	  d &= z - x
\end{align}
In particular, there exists atleast one solution regardless of the values of w, x, y, z.
For example, consider the following matrix,
\begin{align}
	\myvec{w & x\\ y & z} = \myvec{3 & 4 \\ -2 & 7}
\end{align}
Here, $a = 5, b = -2, c = 4, d = 3$. Using \eqref{eq:solutions/2/3/8/eq:eq1}, we get
\begin{align}
	5\myvec{1& 0\\0 & 0} - 2\myvec{1 & 0\\1 & 0} + 4\myvec{0& 1\\0 & 1} + 3\myvec{0 & 0\\0 & 1} = \myvec{3 & 4 \\ -2 & 7}
\end{align}
Hence $\cbrak{\myvec{1 & 0\\0 & 0}, \myvec{1 & 0\\1 & 0}, \myvec{0& 1\\0 & 1}, \myvec{0 & 0\\0 & 1}}$ forms the basis for the given space $\vec{V}$.
