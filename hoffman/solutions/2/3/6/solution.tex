	
	Let
	
	\begin{align}
		v_{11} = \myvec{
			1 & 0 \\
			0 & 0 \\
		},
		\quad 
		v_{12} = \myvec{
			0 & 1 \\
			0 & 0 \\
		}
		\quad \\
		v_{21} = \myvec{
			0 & 0 \\
			1 & 0 \\
		},
		\quad 
		v_{22} = \myvec{
			0 & 0 \\
			0 & 1 \\
		}
		\quad
	\end{align}
	
	Suppose $av_{11} + bv_{12} + cv_{21} + dv_{22} = $ $\myvec{
		0 & 0 \\
		0 & 0 \\
	}.$ Then \\
	
	\begin{align}\label{eq:solutions/2/3/6/eq1}
		\myvec{
			a & b \\
			c & d \\
		} = 
		\myvec{
			0 & 0 \\
			0 & 0 \\
		}
		\quad 
	\end{align}\\
	
	The only values of $a, b, c, d$ which makes the equation \eqref{eq:solutions/2/3/6/eq1} satisfied is, when $a=b=c=d=0$. Thus $v_1, v_2, v_3, v_4$ are linearly independent.\\
	
	Now, let $\myvec{
		a & b \\
		c & d \\
	}$ be any $2\times 2$ matrix. Then $\myvec{
		a & b \\
		c & d \\
	}$ $=$ $av_{11} + bv_{12} + cv_{21} + dv_{22}.$ Thus $v_{11}, v_{12}, v_{21}, v_{22}$ span the space of $2\times 2$ matrix.\\
	
	Thus $v_{11}, v_{12}, v_{21}, v_{22}$ are both linearly independent and they span the span of all $2\times 2$ matrices. So, $v_{11}, v_{12}, v_{21}, v_{22}$ constitute a basis for the space of all $2\times 2$ matrices. \\
	
	We know that, the dimension of a vector space $\vec{V}$, denoted by $dim(\vec{V})$, is the number of basis for $\vec{V}$. Therefore, $dim(\vec{V})$ = 4.
	
