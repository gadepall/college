\begin{enumerate}
\item Let the matrices $\vec{A}$ and $\vec{B}$ $\in$ $\vec{V}$, be set of invertible matrix. For them to be a subspace they need to be closed under addition.
Let,
\begin{align}
\vec{A} = \vec{I}\\
\vec{B} = -\vec{I}
\end{align}  
It could be easily proven that both matrices $\vec{A}$ and $\vec{B}$ are invertible as,
\begin{align}
rank(\vec{I}_{nxn}) = rank\brak{\myvec{1&0&\cdots&0\\0&1&\cdots&0\\\vdots&\vdots&\ddots&\vdots\\0&0&\cdots&1}_{nxn}}\\\implies rank(\vec{-I}_{nxn}) = rank(\vec{I}_{nxn}) = n
\label{eq:solutions/2/2/5/i}
\end{align} 
or it is a full rank matrix as there are n pivots.
\begin{align}
\therefore\vec{A}+\vec{B} = \vec{0}. 
\end{align} 
But the zero matrix $\vec{0}$ is non-invertible as,
\begin{align}
rank(\vec{0}_{nxn}) = rank\brak{\myvec{0&0&\cdots&0\\0&0&\cdots&0\\\vdots&\vdots&\ddots&\vdots\\0&0&\cdots&0}_{nxn}}\\\implies rank(\vec{0}_{nxn}) = 0
\end{align} 
\textbf{$\therefore$ the set of invertible matrices are not closed under addition. Hence not a subspace of $\vec{V}$.}

\item Let the matrices $\vec{A_1}$, $\vec{A_2},\cdots, \vec{A_n}$ $\in \vec{V}$, be set of non-invertible matrix. For them to be a subspace they need to be closed under addition. Let,
\begin{align}
\vec{A_1} = \myvec{1&0&\cdots&0\\0&0&\cdots&0\\\vdots&\vdots&\vdots&\vdots\\0&0&\cdots&0}_{nxn}\\
\vec{A_2} = \myvec{0&0&\cdots&0\\0&1&\cdots&0\\0&0&\cdots&0\\\vdots&\vdots&\vdots&\vdots\\0&0&\cdots&0}_{nxn}\\
\vec{A_n}=\myvec{0&0&\cdots&0\\0&0&\cdots&0\\\vdots&\vdots&\vdots&\vdots\\0&0&\cdots&1}_{nxn}\\
\end{align}  
It could be proven that matrices $\vec{A_1}$, $\vec{A_2}, \cdots, \vec{A_n}$ are non-invertible as,
\begin{align}
rank(\vec{A_1}) = rank\brak{\myvec{1&0&\cdots&0\\0&0&\cdots&0\\\vdots&\vdots&\ddots&\vdots\\0&0&\cdots&0}}\\
\implies rank(\vec{A_1}) = 1
\end{align}
or there is only one pivot hence rank is 1. 
\begin{align}
\implies \vec{A_1}+\vec{A_2}+\vec{A_3}+\cdots\vec{A_n} = \vec{I}_{nxn}
\end{align}
Now the identity matrix $\vec{I}$ is invertible as shown in equation \eqref{eq:solutions/2/2/5/i}. \textbf{$\therefore$ the set of non-invertible matrices are not closed under addition. Hence not a subspace of $\vec{V}$.}
\item \textbf{Theorem 1:}. A non-empty subset W of V is a subspace of V if and only if for each pair of vectors $\alpha$, $\beta$ in W and each scalar c $\in$ F, the vector $c\alpha+\beta \in$ W. \\
Let the matrices $\vec{A_1}$ and $\vec{A_2}$ satisfy,
\begin{align}
\vec{A_1B}=\vec{BA_1}\label{eq:solutions/2/2/5/s1}\\
\vec{A_2B}=\vec{BA_2}\label{eq:solutions/2/2/5/s2}
\end{align}
Let, c$\in \vec{F}$ be any constant.   
\begin{align}
\therefore \brak{c\vec{A_1}+\vec{A_2}}\vec{B} = c\vec{A_1B} +\vec{A_2B}\label{eq:solutions/2/2/5/s3}
\end{align}
Substituting from equations \eqref{eq:solutions/2/2/5/s1} and \eqref{eq:solutions/2/2/5/s2} to \eqref{eq:solutions/2/2/5/s3},
\begin{align}
\implies \brak{c\vec{A_1}+\vec{A_2}}\vec{B} = c\vec{BA_1} +\vec{BA_2}\\
\implies \vec{B}c\vec{A_1} +\vec{BA_2}\\
\implies \vec{B}\brak{c\vec{A_1}+\vec{A_2}}
\end{align}
\textbf{Thus, $\brak{c\vec{A_1}+\vec{A_2}}$ satisfy the criteria and from Theorem-1 it can be seen that the set is a subspace of $\vec{V}$.} 
\item Let $\vec{A}$ and $\vec{B} \in \vec{V}$ be set of matrices such that,
\begin{align}
\vec{A^2}=\vec{A}\\
\vec{B^2}=\vec{B}
\end{align}
Now for them to be closed under addition,
\begin{align}
\brak{\vec{A+B}}^2 = \vec{A+B}
\end{align} 
Which is not always same. Example let,
\begin{align}
\vec{A} = \myvec{1&1\\0&0}\\
\vec{B} = \myvec{0&0\\0&1}
\end{align}
Clearly,
\begin{align}
\vec{A}^2 = \myvec{1&1\\0&0}\myvec{1&1\\0&0} = \myvec{1&1\\0&0} = \vec{A}\\
\vec{B}^2 =\myvec{0&0\\0&1}\myvec{0&0\\0&1} = \myvec{0&0\\0&1} = \vec{B}
\end{align}
Now, 
\begin{align}
\vec{A+B} = \myvec{1&1\\0&1}\label{eq:solutions/2/2/5/e1}\\
\implies \brak{\vec{A+B}}^2 = \myvec{1&1\\0&1}\myvec{1&1\\0&1} = \myvec{1&2\\0&1}\label{eq:solutions/2/2/5/e2}
\end{align}
Hence, clearly from equations \eqref{eq:solutions/2/2/5/e1} and \eqref{eq:solutions/2/2/5/e2},
\begin{align}
\brak{\vec{A+B}}^2 \neq \vec{A+B}
\end{align}
\textbf{$\therefore$ the set of all $\vec{A}$ such that $\vec{A}^2=\vec{A}$ is not closed under addition. Hence, not a subspace of $\vec{V}$.}
\end{enumerate}
