 \begin{enumerate}
   \item Prove that $\vec{V_e}$ and $\vec{V_o}$ are subspaces of $\vec{V}$.
 \end{enumerate}

A non-empty subset $\vec{W}$ of $\vec{V}$ is a subspace of $\vec{V}$ if and only if for each pair of vectors $\vec{\alpha}$, $\vec{\beta}$ in $\vec{W}$ and each scalar $c$ in $\vec{F}$ the vector $c\vec{\alpha}+\vec{\beta}$ is again in $\vec{W}$.\\

Let $\vec{u}$, $\vec{v}$ $\in$ $\vec{V_e}$ and $c$ $\in$ $\vec{R}$ and let $\vec{h}$ = $c\vec{u}$ + $\vec{v}$. Then,
\begin{equation} \label{eq:solutions/2/2/8/eq:eq1}
\begin{split}
\vec{h}(-x) &= c \vec{u}(-x) + \vec{v}(-x)\\
 & = c \vec{u}(x) + \vec{v}(x)\\
& = \vec{h}(x)\\
\end{split}
\end{equation}
From \eqref{eq:solutions/2/2/8/eq:eq1}
\begin{align}
\implies \vec{h}(-x) = \vec{h}(x)\\
\implies \vec{h} \in \vec{V_e} \label{eq:solutions/2/2/8/eq:eq2}
\end{align}

Let $\vec{u}$, $\vec{v}$ $\in$ $\vec{V_o}$ and $c$ $\in$ $\vec{R}$ and let $\vec{h}$ = $c\vec{u}$ + $\vec{v}$. Then,
\begin{equation} \label{eq:solutions/2/2/8/eq:eq3}
\begin{split}
\vec{h}(-x) &= c \vec{u}(-x) + \vec{v}(-x)\\
 & = -c \vec{u}(x) - \vec{v}(x)\\
& = -\vec{h}(x)\\
\end{split}
\end{equation}
From \eqref{eq:solutions/2/2/8/eq:eq3}
\begin{align}
\implies \vec{h}(-x) = -\vec{h}(x)\\
\implies \vec{h} \in \vec{V_o} \label{eq:solutions/2/2/8/eq:eq4}
\end{align}
From \eqref{eq:solutions/2/2/8/eq:eq2} and \eqref{eq:solutions/2/2/8/eq:eq4}, $\vec{V_e}$ and $\vec{V_o}$ are subspaces of $\vec{V}$.\\
\begin{enumerate}
    \setcounter{enumi}{1}
   \item Prove that $\vec{V_e} + \vec{V_o} = \vec{V}$.
 \end{enumerate}
Let $\vec{u}$ $\in$ $\vec{V}$
\begin{align}
    \vec{u_e}(x) = \frac{\vec{u}(x)+\vec{u}(-x)}{2} \label{eq:solutions/2/2/8/eq:eq10}\\
    \vec{u_o}(x) = \frac{\vec{u}(x)-\vec{u}(-x)}{2} \label{eq:solutions/2/2/8/eq:eq11}
\end{align}
Equation equation \eqref{eq:solutions/2/2/8/eq:eq10} and \eqref{eq:solutions/2/2/8/eq:eq11}, $\vec{u_e}$ is even and $\vec{u_o}$ is odd. Adding both the equations,
\begin{align}
    \vec{u} = \vec{u_e} + \vec{u_o} \label{eq:solutions/2/2/8/eq:eq12}
\end{align}

\begin{enumerate}
    \setcounter{enumi}{2}
   \item Prove that $\vec{V_e} \cap \vec{V_o} = \{0\}$.
 \end{enumerate}
Let $\vec{u} \in \vec{V_e} \cap \vec{V_o}$
\begin{align}
    \vec{u} \in \vec{V_e} \implies \vec{u}(-x) &= \vec{u}(x) \label{eq:solutions/2/2/8/eq:eq7}\\
    \vec{u} \in \vec{V_o} \implies \vec{u}(-x) &= -\vec{u}(x) \label{eq:solutions/2/2/8/eq:eq8}
\end{align}
Equating \eqref{eq:solutions/2/2/8/eq:eq7} and \eqref{eq:solutions/2/2/8/eq:eq8},
\begin{align}
    \vec{u}(x) = -\vec{u}(x)\\
    \implies 2\vec{u}(x)=0\\
    \implies \vec{u}=0 \label{eq:solutions/2/2/8/eq:eq13}
\end{align}
Equations \eqref{eq:solutions/2/2/8/eq:eq2}, \eqref{eq:solutions/2/2/8/eq:eq4}, \eqref{eq:solutions/2/2/8/eq:eq12}, \eqref{eq:solutions/2/2/8/eq:eq13} proves 1, 2 and 3.
