Expressing the given vectors as the columns of a matrix,
\begin{align}
    \vec{A} = \myvec{1&0&1\\0&1&1\\-1&1&1}
\end{align}
The row reduced echelon form of the matrix on performing elementary row operations can be given as, 
\begin{align}
    \vec{R} = \vec{C}\vec{A}
\end{align}
where $\vec{C}$ is the product of elementary matrices,
\begin{align}
    \vec{C} = \myvec{0&1&-1\\-1&2&-1\\1&-1&1}
\end{align}
Thus we get,  
\begin{align}
    \vec{R} = \myvec{1&0&0\\0&1&0\\0&0&1} \label{eq:solutions/2/1/3/R}
\end{align}
From \eqref{eq:solutions/2/1/3/R}, $rank(\vec{A})=3$. Thus $\vec{A}$ is a full rank matrix. Hence the columns of $\vec{A}$ are linearly independent i.e., the given vectors are linearly independent and forms the basis for $\vec{C}^3$.

Hence any vector $\vec{Y}\in\vec{C}^3$ can be written as the linear combinations of $\myvec{1\\0\\-1}$, $\myvec{0\\1\\1}$ and $\myvec{1\\1\\1}$.


