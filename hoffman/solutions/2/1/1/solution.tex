Let $\vec{F}^n$ be a set of all ordered n-tuples over $\vec{F}$ i.e
\begin{align}
	\vec{F}^n= \cbrak{\myvec{a_1 \\ a_2 \\ \vdots \\ a_n} : 
	a_1, a_2, \ldots, a_n \in \vec{F} }
\end{align}
For $\vec{F}^n$ to be a vector space over $\vec{F}$ it must satisfy the 
closure property of vector addition and scalar multiplication. \\
\\
{\bf Vector Addition in $\vec{F}^n$ :} \\
Let $\vec{\alpha} = \myvec{a_i}$ and
$\vec{\beta} = \myvec{b_i} \ \forall \  i=1,2,\cdots,n \in \vec{F}^n$ 
then 
\begin{align}
	\vec{\alpha} + \vec{\beta} 
	&=\myvec{a_i}+\myvec{b_i} \\
	&=\myvec{a_i+b_i} 
\end{align}
Since 
\begin{align}
	a_i+b_i \in \vec{F} \ \forall \  i=1,2,\cdots,n \\
	\implies \vec{\alpha}+\vec{\beta} \in \vec{F}^n
\end{align}

{\bf Scalar multiplication in $\vec{F}^n$ over $\vec{F}$ :} \\
Let $\vec{\alpha} = \myvec{a_i} \ \forall \ i=1,2,\cdots,n \  
	\in \vec{F}^n$ and  $ a \in \vec{F}$ then
\begin{align}
	a\vec{\alpha}=\myvec{aa_i}
\end{align}
Since
\begin{align}
	aa_i \in \vec{F} \ \forall \ i=1,2\cdots,n \\
	\implies a\vec{\alpha} \in \vec{F}^n
\end{align}
\\
{\bf Associativity of addition in $\vec{F}^n$ :} \\
Let $\vec{\alpha}=\myvec{a_i}, \ 
\vec{\beta}=\myvec{b_i}, 
\ \vec{\gamma}=\myvec{g_i} 
\ \forall \ i=1,2,\cdots,n
\in \vec{F}^n$ then
\begin{align}
	\vec{\alpha}+(\vec{\beta}+\vec{\gamma}) &= 
	\myvec{a_i} + 
      \myvec{b_i+g_i} \\
	&= \myvec{a_i+b_i+g_i} \\
	&= \myvec{a_i+b_i} + \myvec{g_i} \\
	&=(\alpha+\beta)+\gamma
\end{align}

{\bf Existence of additive identity in $\vec{F}^n$ :} \\ 
We have $\vec{0}=\myvec{0\\ 0\\ \vdots\\ 0} \in \vec{F}^n$
and $\alpha=\myvec{a_i} \ \forall \ i=1,2,\cdots,n  \in \vec{F}^n$ then
\begin{align}
	\myvec{a_i}+\myvec{0}
	&=\myvec{a_i+0} \\
	&=\myvec{a_i} 
\end{align}
Therefore $\vec{0}$ is the additive identity in 
$\vec{F}^n$.\\ \\
{\bf Existence of additive inverse of each element of $\vec{F}^n$ :} \\
If $\alpha=\myvec{a_i} \ \forall \ i =1,2,\cdots,n \in \vec{F}^n$ then 
$\myvec{-a_i} \in \vec{F}^n$. 
Also we have
\begin{align}
	\myvec{-a_i}
	+\myvec{a_i}
	&=\vec{0}
\end{align}
Therefore $\vec{-\alpha}=\myvec{-a_i}$ is the 
additive inverse of $\vec{\alpha}$.
Thus $\vec{F}^n$ is an abelian group with respect to addition. \\

Futher we observe that
\begin{enumerate}
\item If $a$ $\in \vec{F}$ and 
	$\vec{\alpha} = \myvec{a_i}, \
		\vec{\beta}=\myvec{b_i} \ \forall \ i=1,2,\cdots,n \in \vec{F}^n$ 
then
\begin{align}
	a(\vec{\alpha}+\vec{\beta}) &=
	a\myvec{a_i+b_i} 
\end{align}
\begin{align}
	&=\myvec{a[a_i+b_i]} \\
	&=\myvec{aa_i+ab_i} \\
	&\myvec{aa_i}
	+ \myvec{ab_i} \\
	&=a\myvec{a_i}
	+ a\myvec{b_i} \\
	&=a\alpha+a\beta
\end{align}
\item If $a$,$b$ $\in \vec{F}$ and 
	$\alpha=\myvec{a_i} \ \forall \ i=1,2,\cdots,n \in \vec{F}^n$ then
\begin{align}
	(a+b)\alpha
	&=\myvec{[a+b]a_i} 
\end{align}
\begin{align}
	&=\myvec{aa_i+ba_i} \\ 
	&=\myvec{aa_i}+\myvec{ba_i} \\
	&=a\myvec{a_i}+b\myvec{a_i} \\
	&=a\alpha+b\alpha
\end{align}
\item If $a$,$b$ $\in \vec{F}$ and 
	$\alpha=\myvec{a_i} \ \forall \ i=1,2,\cdots,n \in \vec{F}^n$ 
then
\begin{align}
     (ab)\alpha&=\myvec{[ab]a_i} \\
	  &=\myvec{a[ba_i]} \\
	  &=a\myvec{ba_i} \\
	  &=a(b\alpha)
\end{align}
\item If 1 is the unity element of $\vec{F}$ and 
	$\alpha=\myvec{a_i} \ \forall \ i=1,2,\cdots,n \in \vec{F}^n$ then
\begin{align}
	1\alpha &=\myvec{1a_i} \\
		&=\myvec{a_i} \\
		&= \alpha
\end{align}
\end{enumerate}
Hence $\vec{F}^n$ is a vector space over $\vec{F}$.
