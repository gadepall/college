\begin{theorem} \label{eq:solutions/2/4/1/1}
Let $\vec{V}$ be an $n$-dimensional vector space over the field $\vec{F}$, and let $\beta$ and $\beta'$ be two ordered basis of $\vec{V}$. Then, there is a unique, necessarily invertible, $n\times n$ matrix $\vec{P}$ with entries in $\vec{F}$ such that 
\begin{enumerate}
	\item $\begin{bmatrix}
	\vec{\alpha}
	\end{bmatrix}$$_\beta$ = $\vec{P}$  $\begin{bmatrix}
	\vec{\alpha} 
\end{bmatrix}$$_{\beta^{'}}$
    \item $\begin{bmatrix}
    	\vec{\alpha}
    \end{bmatrix}$$_{\beta^{'}}$= $\vec{P^{-1}}$$\begin{bmatrix}
    \vec{\alpha}	
\end{bmatrix}$$_\beta$
\end{enumerate}
for every vector $\vec{\alpha}$ in $\vec{V}$. The columns of $\vec{P}$ are given by
\begin{align}
\vec{P_j} =  \begin{bmatrix}
	\vec{\alpha_j}
\end{bmatrix}_{\beta}	\qquad j = 1,2,...,n
\end{align}
\end{theorem}
Firt, we need to show that the set of vectors $\alpha_1$, $\alpha_2$, $\alpha_3$ and $\alpha_4$ are basis for $\Re^4$. For, this we first show that $\alpha_1$, $\alpha_2$, $\alpha_3$ and $\alpha_4$ are linearly independent in $\Re^4$ and also they span $\Re^4$. Consider,
\begin{align}
& \vec{A} = \myvec{\vec{\alpha_1}^T & \vec{\alpha_2}^T & \vec{\alpha_3}^T & \vec{\alpha_4}^T } \\
& \vec{A} = \myvec{1 & 0 & 1 & 0 \\ 1 & 0 & 0 & 0 \\ 0 & 1 & 0 & 0 \\ 0 & 1 & 4 & 2}
\end{align}
Now,
\begin{align}
& \myvec{1 & 0 & 1 & 0 \\ 1 & 0 & 0 & 0 \\ 0 & 1 & 0 & 0 \\ 0 & 1 & 4 & 2} &  \xleftrightarrow[]{r_2=r_2-r_1} &
\myvec{1 & 0 & 1 & 0 \\ 0 & 0 & -1 & 0 \\ 0 & 1 & 0 & 0 \\ 0 & 1 & 4 & 2} \\
& \myvec{1 & 0 & 1 & 0 \\ 0 & 0 & -1 & 0 \\ 0 & 1 & 0 & 0 \\ 0 & 1 & 4 &2}      &\xleftrightarrow[]{r_2 \leftrightarrow r_3} &\myvec{1 & 0 & 1 & 0 \\ 0 & 1 & 0 & 0 \\ 0 & 0 & -1 & 0 \\0&1&4& 2} \\
& \myvec{1 & 0 & 1 & 0 \\ 0 & 1 & 0 & 0 \\ 0 & 0 & -1 & 0 \\0 & 1 & 4 & 2}&\xleftrightarrow[]{r_4=r_4-r_2}&
\myvec{1 & 0 & 1 & 0 \\ 0 & 1 & 0 & 0 \\ 0 & 0 & -1 & 0 \\0 & 0 & 4 & 2} \\
& \myvec{1 & 0 & 1 & 0 \\ 0 & 1 & 0 & 0 \\ 0 & 0 & -1 & 0 \\0 & 0 & 4 & 2}&\xleftrightarrow[]{r_3=-r_3}&
\myvec{1 & 0 & 1 & 0 \\ 0 & 1 & 0 & 0 \\ 0 & 0 & 1 & 0 \\0 & 0 & 4 & 2} \\
& \myvec{1 & 0 & 1 & 0 \\ 0 & 1 & 0 & 0 \\ 0 & 0 & 1 & 0 \\0 & 0 & 4 & 2} &\xleftrightarrow[]{r_4=r_4-4r_3}& \myvec{1 & 0 & 1 & 0 \\ 0 & 1 & 0 & 0 \\ 0 & 0 & 1 & 0 \\0& 0& 0&2} \\ 
&\myvec{1 & 0 & 1 & 0 \\ 0 & 1 & 0 & 0 \\ 0 & 0 & 1 & 0 \\0& 0& 0&2}&\xleftrightarrow[]{r_1=r_1-r_3}&
\myvec{1 & 0 & 0 & 0 \\ 0 & 1 & 0 & 0 \\ 0 & 0 & 1 & 0 \\0& 0& 0&2}\\
& \myvec{1 & 0 & 0 & 0 \\ 0 & 1 & 0 & 0 \\ 0 & 0 & 1 & 0 \\0& 0& 0&2} &\xleftrightarrow[]{r_4=\frac{r_4}{2}}& \myvec{1 & 0 & 0 & 0 \\ 0 & 1 & 0 & 0 \\ 0 & 0 & 1 & 0 \\0& 0& 0&1}\label{eq:solutions/2/4/1/2}
\end{align}
\eqref{eq:solutions/2/4/1/2} is the row reduced echelon form of $\vec{A}$ and since it is identity matrix of order 4, we say that vectors $\vec{\alpha_1}$, $\vec{\alpha_2}$, $\vec{\alpha_3}$ and $\vec{\alpha_4}$ are linearly independent and their column space is $\Re^4$ which means vectors $\vec{\alpha_1}$, $\vec{\alpha_2}$, $\vec{\alpha_3}$ and $\vec{\alpha_4}$ span $\Re^4$.
Hence, vectors $\vec{\alpha_1}$, $\vec{\alpha_2}$, $\vec{\alpha_3}$ and $\vec{\alpha_4}$ form a basis for $\Re^4$. 

Now, we use theorem \eqref{eq:solutions/2/4/1/1}, and if we calculate the inverse of 
\begin{align}
	\vec{A} = \myvec{1 & 0 & 1 & 0 \\ 1 & 0 & 0 & 0 \\ 0 & 1 & 0 & 0 \\ 0 & 1 & 4 & 2}
\end{align}
then the columns of $\vec{A^{-1}}$ will give the coeffients to write the standard basis vectors in terms of $\alpha_i's$. We try to find the inverse of $\vec{A}$ by row-reducing the augumented matrix. 
\begin{align}
	\vec{A} = \myvec{1 & 0 & 1 & 0& 1 &0 &0 & 0 \\ 1 & 0 & 0 & 0& 0&1 &0&0 \\ 0 & 1 & 0 & 0& 0 &0 &1 &0 \\ 0 & 1 & 4 & 2& 0 & 0& 0 &1}
\end{align}
Now, we solve for $\vec{A^{-1}}$ as follows 
\begin{multline}
 \myvec{1 & 0 & 1 & 0& 1 &0 &0 & 0 \\ 1 & 0 & 0 & 0& 0&1 &0&0 \\ 0 & 1 & 0 & 0& 0 &0 &1 &0 \\ 0 & 1 & 4 & 2& 0 & 0& 0 &1} \xleftrightarrow[]{r_2=r_2-r_1} \\ \myvec{1 & 0 & 1 & 0& 1 &0 &0 & 0 \\ 0 & 0 & -1 & 0& -1&1 &0&0 \\ 0 & 1 & 0 & 0& 0 &0 &1 &0 \\ 0 & 1 & 4 & 2& 0 & 0& 0 &1}
\end{multline}
\begin{multline}
\myvec{1 & 0 & 1 & 0& 1 &0 &0 & 0 \\ 0 & 0 & -1 & 0& -1&1 &0&0 \\ 0 & 1 & 0 & 0& 0 &0 &1 &0 \\ 0 & 1 & 4 & 2& 0 & 0& 0 &1}\xleftrightarrow[]{r_2 \leftrightarrow r_3} \\ \myvec{1 & 0 & 1 & 0& 1 &0 &0 & 0 \\0 & 1 & 0 & 0& 0 &0 &1 &0 \\ 0 & 0 & -1 & 0& -1&1 &0&0  \\ 0 & 1 & 4 & 2& 0 & 0& 0 &1}
\end{multline}
\begin{multline}
\myvec{1 & 0 & 1 & 0& 1 &0 &0 & 0 \\0 & 1 & 0 & 0& 0 &0 &1 &0 \\ 0 & 0 & -1 & 0& -1&1 &0&0  \\ 0 & 1 & 4 & 2& 0 & 0& 0 &1} \xleftrightarrow[]{r_4 = r_4-r_2} \\	\myvec{1 & 0 & 1 & 0& 1 &0 &0 & 0 \\0 & 1 & 0 & 0& 0 &0 &1 &0 \\ 0 & 0 & -1 & 0& -1&1 &0&0  \\ 0 & 0 & 4 & 2& 0 & 0& -1 &1}
\end{multline}
\begin{multline}
\myvec{1 & 0 & 1 & 0& 1 &0 &0 & 0 \\0 & 1 & 0 & 0& 0 &0 &1 &0 \\ 0 & 0 & -1 & 0& -1&1 &0&0  \\ 0 & 0 & 4 & 2& 0 & 0& -1 &1} \xleftrightarrow[]{r_3=-r_3} \\ \myvec{1 & 0 & 1 & 0& 1 &0 &0 & 0 \\0 & 1 & 0 & 0& 0 &0 &1 &0 \\ 0 & 0 & 1 & 0& 1&-1 &0&0  \\ 0 & 0 & 4 & 2& 0 & 0& -1 &1}
\end{multline}
\begin{multline}
\myvec{1 & 0 & 1 & 0& 1 &0 &0 & 0 \\0 & 1 & 0 & 0& 0 &0 &1 &0 \\ 0 & 0 & 1 & 0& 1&-1 &0&0  \\ 0 & 0 & 4 & 2& 0 & 0& -1 &1} \xleftrightarrow[]{r_4=r_4-4r_3} \\ \myvec{1 & 0 & 1 & 0& 1 &0 &0 & 0 \\0 & 1 & 0 & 0& 0 &0 &1 &0 \\ 0 & 0 & 1 & 0& 1&-1 &0&0  \\ 0 & 0 & 0 & 2& -4 & 4& -1 &1}
\end{multline}
\begin{multline}
\myvec{1 & 0 & 1 & 0& 1 &0 &0 & 0 \\0 & 1 & 0 & 0& 0 &0 &1 &0 \\ 0 & 0 & 1 & 0& 1&-1 &0&0  \\ 0 & 0 & 0 & 2& -4 & 4& -1 &1} \xleftrightarrow[]{r_1 = r_1-r_3} \\ \myvec{1 & 0 & 0 & 0& 0 &1 &0 & 0 \\0 & 1 & 0 & 0& 0 &0 &1 &0 \\ 0 & 0 & 1 & 0& 1&-1 &0&0  \\ 0 & 0 & 0 & 2& -4 & 4& -1 &1}
\end{multline}
\begin{multline}
\myvec{1 & 0 & 0 & 0& 0 &1 &0 & 0 \\0 & 1 & 0 & 0& 0 &0 &1 &0 \\ 0 & 0 & 1 & 0& 1&-1 &0&0  \\ 0 & 0 & 0 & 2& -4 & 4& -1 &1} \xleftrightarrow[]{r_4 = \frac{r_4}{2}} \\
\myvec{1 & 0 & 0 & 0& 0 &1 &0 & 0 \\0 & 1 & 0 & 0& 0 &0 &1 &0 \\ 0 & 0 & 1 & 0& 1&-1 &0&0  \\ 0 & 0 & 0 & 1& -2 & 2& \frac{-1}{2} & \frac{1}{2}}	\label{eq:solutions/2/4/1/3}
\end{multline}
Thus, by \eqref{eq:solutions/2/4/1/3}, we have
\begin{align}
	\vec{A^{-1}}= \myvec{0 & 1 & 0 & 0 \\ 0 & 0 & 1 & 0 \\ 1 & -1 & 0 & 0 \\ -2 & 2 & \frac{-1}{2} & \frac{1}{2}}
\end{align}
Now, let $\vec{e_1} = \myvec{1 & 0 & 0 & 0}$, $\vec{e_2} = \myvec{0 & 1 & 0 & 0}$, $\vec{e_3} = \myvec{0 & 0 & 1 & 0}$ and $\vec{e_4} = \myvec{0 & 0 & 0 & 1}$ be the standard basis for $\Re^4$. Hence,
\begin{align}
& \vec{e_1} = \alpha_3 - 2 \alpha_4 \\ 
& \vec{e_2} = \alpha_1 - \alpha_3 + 2\alpha_4 \\
& \vec{e_3} = \alpha_2 - \frac{1}{2} \alpha_4 \\
& \vec{e_4} = \frac{1}{2}\alpha_4
\end{align}
