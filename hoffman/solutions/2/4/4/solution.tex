\begin{enumerate}[label=\emph{\alph*)}]
\item 
It is given that $\alpha_1$ and $\alpha_2$ span $\vec{W}$. For $\alpha_1$ and
$\alpha_2$ to be the basis for $\vec{W}$ they must be linearly independent.
Let
\begin{align}
	S_1=\cbrak{\alpha_1,\alpha_2}=\cbrak{\myvec{1\\0\\i},\myvec{1+i\\1\\-1}}
\end{align}
Using row reduction on matrix $\vec{A}=\myvec{\alpha_1 & \alpha_2}$
\begin{align}
	\myvec{1 & 1+i \\ 0 & 1 \\ i & -1}
	\xleftrightarrow[]{R_3 \leftarrow R_3-iR_1}
	\myvec{1 & 1+i \\ 0 & 1 \\ 0 & -i}\\   
	\xleftrightarrow[]{R_3 \leftarrow R_3+iR_2}
	\myvec{1 & 1+i \\ 0 & 1 \\ 0 & 0}\\
	\xleftrightarrow[]{R_1 \leftarrow R_1 - (i+1)R_2}
	 \myvec{1 & 0 \\ 0 & 1 \\ 0 & 0} \label{eq:solutions/2/4/4/rrefA}
\end{align}
Since $\vec{A}$ is a full-rank matrix the column vectors are linearly 
independent. Therefore $S_1= \cbrak{ \alpha_1, \alpha_2 } $ is a basis set for 
$\vec{W}$.
\item
\begin{align}
	\beta_1=\myvec{1\\1\\0} \\
	\beta_2=\myvec{1\\i\\1+i} 
\end{align}
		Since column vectors of $\vec{A}=\myvec{\alpha_1 & \alpha_2}$ are basis for $\vec{W}$ and if $\beta_1$ and $\beta_2 \ \in \vec{W}$ there exist a unique solution $\vec{x}$ such that
\begin{align}
	\myvec{\alpha_1 & \alpha_2}\vec{x}=\myvec{\beta_1 & \beta_2} 
\end{align}
Using row reduction on augmented matrix
		\begin{align}
			\myvec{1 & 1+i & | & 1 & 1\\
			       0 & 1   & | & 1 & i\\
			       i & -1  & | & 0 & 1+i}\\
			\xleftrightarrow[]{R3 \leftarrow R_3-iR-1}
			\myvec{1 & 1+i & | & 1 & 1\\
			       0 & 1   & | & 1 & i\\
			       0 & -i  & | & -i & 1}\\
			\xleftrightarrow[]{R_3 \leftarrow R_3+iR_2}
			\myvec{1 & 1+i & | & 1 & 1\\
			       0 & 1   & | & 1 & i\\
			       0 & 0  &  | & 0 & 0}\\
			\xleftrightarrow[]{R_1 \leftarrow R_1-(i+1)R_2}
			\myvec{1 & 0 &  | & -i & 2-i\\
			       0 & 1   &| & 1  & i\\
			       0 & 0  & | & 0  & 0}\\
			       \implies 
			       \vec{x}=\myvec{-i & 2-i\\1 & i}
		\end{align}
Therefore $\beta_1$ and $\beta_2 \in \vec{W}$.\\
Consider 
\begin{align}
	S_2=\cbrak{\beta_1,\beta_2}=\cbrak{\myvec{1\\1\\0},\myvec{1\\i\\1+i}}
\end{align}
and also let 
\begin{align}
	\vec{B}=\myvec{1 & 1 \\
		       1 & i \\
		       0 & 1+i}
\end{align}
Using row reduction on matrix $\vec{B}$
\begin{align}
	\myvec{1 & 1 \\
               1 & i \\
               0 & 1+i}
        \xleftrightarrow[]{R_2 \leftarrow R_2-R_1}
	\myvec{1 & 1 \\
               0 & i-1 \\
	       0 & 1+i}\\
	\xleftrightarrow[]{R_2\leftarrow \frac{R_2}{i-1}}
	\myvec{1 & 1 \\
               0 & 1 \\
               0 & 1+i}\\
	\xleftrightarrow[R_3 \leftarrow R_3-(i-1)R_2]
	{R_1 \leftarrow R_1 - R_2}
	 \myvec{1 & 0 \\ 0 & 1 \\ 0 & 0} \label{eq:solutions/2/4/4/Binv}
\end{align}
Since $\vec{B}$ is a full rank matrix the column vectors are linearly independent.\\
Let $\alpha$ be any vector in the subspace $\vec{W}$, then it can be expressed as 
span $\cbrak{\alpha_1,\alpha_2}$ i.e
\begin{align}
	\alpha = \myvec{\alpha_1 & \alpha_2}\vec{x_1} =\vec{A}\vec{x_1} \label{eq:solutions/2/4/4/Ax}
\end{align}
$S_2=\cbrak{\beta_1,\beta_2}$ spans $\vec{W}$ if any vector $\alpha \in \ \vec{W}$ can be 
expressed as
\begin{align}
	\alpha=\myvec{\beta_1,\beta_2}\vec{x_2}=\vec{B}\vec{x_2} \label{eq:solutions/2/4/4/Bx}
\end{align}
From (\ref{eq:solutions/2/4/4/Ax}) and (\ref{eq:solutions/2/4/4/Bx}) we conclude
\begin{align}
	\vec{B}\vec{x_2}=\vec{A}\vec{x_1}\\
	\implies \vec{x_2}=\vec{B}^{-1}\vec{A}\vec{x_1} \label{eq:solutions/2/4/4/x2}
\end{align}
Therefore from (\ref{eq:solutions/2/4/4/x2}) $\vec{x_2}$ exists if $\vec{B}$ is invertible. From
(\ref{eq:solutions/2/4/4/Binv}) we conclude $\vec{x_2}$ exists and hence any vector $\alpha \in \ \vec{W}$
can be expressed as span$\cbrak{\beta_1,\beta_2}$. Therefore $\cbrak{\beta_1,\beta_2}$
is basis for $\vec{W}$.
\item
	Since $\alpha_1,\alpha_2 \in \ \vec{W}$ and $\cbrak{\beta_1,\beta_2}$ are ordered
	basis for $\vec{W}$ there must exist unique value of $\vec{x}$ such that
	\begin{align}
		\myvec{\beta_1 & \beta_2}\vec{x}=\myvec{\alpha_1 & \alpha_2} \label{eq:solutions/2/4/4/a1}
	\end{align}
Using row reduction on (\ref{eq:solutions/2/4/4/a1}) we get,
\begin{align}
	\myvec{1 & 1 & | & 1 & 1+i\\
	       1 & i & | & 0 & 1\\
	       0 & 1+i &| & i & -1}\\
	\xleftrightarrow[]{R_2 \leftarrow R_2-R_1}
	\myvec{1 & 1 & | & 1 & 1+i\\
	       0 & i-1 & | & -1 & -i\\
	       0 & 1+i &| & i & -1}\\
        \xleftrightarrow[]{R_2 \leftarrow \frac{R_2}{i-1}}
	\myvec{1 & 1 & | & 1 & 1+i\\
	       0 & 1 & | & \frac{1+i}{2} & \frac{-1+i}{2}\\
	       0 & 1+i &| & i & -1}\\
	\xleftrightarrow[]{R_3 \leftarrow R_3 - (i+1)R_2}
	\myvec{1 & 1 & | & 1 & 1+i\\
	       0 & 1 & | & \frac{1+i}{2} & \frac{-1+i}{2} \\
	       0 & 0 &| & 0 & 0}\\
	\xleftrightarrow[]{R_1 \leftarrow R_2- R_1}
	\myvec{1 & 0 & | & \frac{1-i}{2} & \frac{3+i}{2} \\
	       0 & 1 & | & \frac{1+i}{2} & \frac{-1+i}{2}\\
	       0 & 0 &| & 0 & 0}\\
	\implies 
	\vec{x}=\frac{1}{2}\myvec{1-i & 3+i\\ 1+i & -1+i} \label{eq:solutions/2/4/4/x}
\end{align}
Thus the column vectors of (\ref{eq:solutions/2/4/4/x}) are corresponding coordinates of $\alpha_1$ and $\alpha_2$ 
in ordered basis $\cbrak{\beta_1,\beta_2}$.
\end{enumerate}
