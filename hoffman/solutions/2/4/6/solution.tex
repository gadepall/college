	\begin{enumerate}[label=(\alph*)]
		\item To check for independence, lets represent the function in a polynomial format as \\
		
		\begin{align}
			\alpha f_1 + \beta f_2 + \gamma f_3 = 0\\
			\alpha + \beta e^{ix} + \gamma e^{-ix} = 0 
		\end{align}
		
		Multiply the whole equation with $e^{ix}$ 
		
		\begin{align}
			\beta (e^{ix})^{2} + \alpha e^{ix} + \gamma = 0
		\end{align}
		
		Let $y = e^{ix}$
		\begin{align}
			\beta y^{2} + \alpha y + \gamma = 0 \label{eq:solutions/2/4/6/eq1}
		\end{align}
		
		The above quadratic polynomial in $y$ can be zero for atmost two values of $y$. But, 
		\begin{align}
			y = e^{ix} \quad x \in \mathbf{R}
		\end{align}
		So $\eqref{eq:solutions/2/4/6/eq1}$ cannot be zero for all $y = e^{ix}$. Which implies there is a contradiction. \\
		
		Then, the only case when $\eqref{eq:solutions/2/4/6/eq1}$ gets satisfied is
		
		\begin{align}
			\alpha = \beta = \gamma = 0
		\end{align}
		Therefore, $f_1, f_2, f_3$ are linearly independent. \\
		
		\item We need to find the coordinates of vectors $g_i$ where $i = 1, 2, 3$ in ordered basis
		\begin{align}
			B = \myvec{f_1 & f_2 & f_3} \label{eq:solutions/2/4/6/eq5}
		\end{align}
		
		It is given that $g_1 = 1$, which can be written as 
		\begin{align}
			g_1 = f_1 \label{eq:solutions/2/4/6/eq4}   \\ 
			\implies g_1 = \myvec{f_1 & f_2 & f_3}\myvec{1\\0\\0}
		\end{align}
		
		We can use the following identities:-
		
		\begin{align}
			\cos (x) = \frac{1}{2}e^{ix} + \frac{1}{2}e^{-ix} \label{eq:solutions/2/4/6/eq2}\\
			\sin (x) = \frac{1}{2i}e^{ix} - \frac{1}{2i}e^{-ix} \label{eq:solutions/2/4/6/eq3}
		\end{align}
		
		
		Comparing equations \eqref{eq:solutions/2/4/6/eq2} and \eqref{eq:solutions/2/4/6/eq3} with $f_2, f_3$, we can write $g_2$ and $g_3$ as 
		
		\begin{align}
			g_2 = \frac{1}{2}f_2 + \frac{1}{2}f_3 \\
			\implies g_2 = \myvec{f_1 & f_2 & f_3} \myvec{0\\\frac{1}{2}\\\frac{1}{2}}
		\end{align}
		
		\begin{align}
			g_3 = \frac{1}{2i}f_2 - \frac{1}{2i}f_3 \\
			\implies g_3 = \myvec{f_1 & f_2 & f_3} \myvec{0\\\frac{1}{2i}\\\frac{-1}{2i}}
		\end{align}
		
		Therefore, the required matrix $P$ is
		\begin{align}
			P = \myvec{1&0&0\\0&\frac{1}{2}&\frac{1}{2i}\\0&\frac{1}{2}&\frac{-1}{2i}}
		\end{align}
		
	\end{enumerate}
	
