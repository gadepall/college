\documentclass[journal,12pt,twocolumn]{IEEEtran}
%
\usepackage{setspace}
\usepackage{gensymb}
%\doublespacing
\singlespacing

%\usepackage{graphicx}
%\usepackage{amssymb}
%\usepackage{relsize}
%\usepackage[cmex10]{amsmath}
\usepackage{amsmath}
\usepackage{mathtools}
\usepackage{siunitx}
%\usepackage{amsthm}
%\interdisplaylinepenalty=2500
%\savesymbol{iint}
%\usepackage{txfonts}
%\restoresymbol{TXF}{iint}
%\usepackage{wasysym}
\usepackage{amsthm}
\usepackage{iithtlc}
\usepackage{mathrsfs}
\usepackage{txfonts}
\usepackage{stfloats}
\usepackage{steinmetz}
%\usepackage{bm}
\usepackage{cite}
\usepackage{cases}
\usepackage{subfig}
%\usepackage{xtab}
\usepackage{longtable}
\usepackage{multirow}
%\usepackage{algorithm}
%\usepackage{algpseudocode}
\usepackage{enumitem}
\usepackage{tikz}
\usepackage{circuitikz}
\usepackage{verbatim}
\usepackage{tfrupee}
\usepackage[breaklinks=true]{hyperref}
%\usepackage{stmaryrd}
\usepackage{tkz-euclide} % loads  TikZ and tkz-base
%\usetkzobj{all}
\usetikzlibrary{calc,math}
\usetikzlibrary{fadings}
\usepackage{listings}
    \usepackage{color}                                            %%
    \usepackage{array}                                            %%
    \usepackage{longtable}                                        %%
    \usepackage{calc}                                             %%
    \usepackage{multirow}                                         %%
    \usepackage{hhline}                                           %%
    \usepackage{ifthen}                                           %%
  %optionally (for landscape tables embedded in another document): %%
    \usepackage{lscape}     
\usepackage{multicol}
\usepackage{chngcntr}
%\usepackage{enumerate}

%\usepackage{wasysym}
%\newcounter{MYtempeqncnt}
\DeclareMathOperator*{\Res}{Res}
\DeclareMathOperator*{\rank}{rank}
\DeclareMathOperator*{\nullity}{nullity}
%\renewcommand{\baselinestretch}{2}
\renewcommand\thesection{\arabic{section}}
\renewcommand\thesubsection{\thesection.\arabic{subsection}}
\renewcommand\thesubsubsection{\thesubsection.\arabic{subsubsection}}

\renewcommand\thesectiondis{\arabic{section}}
\renewcommand\thesubsectiondis{\thesectiondis.\arabic{subsection}}
\renewcommand\thesubsubsectiondis{\thesubsectiondis.\arabic{subsubsection}}

% correct bad hyphenation here
\hyphenation{op-tical net-works semi-conduc-tor}
\def\inputGnumericTable{}                                 %%

\lstset{
%language=C,
frame=single, 
breaklines=true,
columns=fullflexible
}
%\lstset{
%language=tex,
%frame=single, 
%breaklines=true
%}

\begin{document}
%


\newtheorem{theorem}{Theorem}[section]
\newtheorem{problem}{Problem}
\newtheorem{proposition}{Proposition}[section]
\newtheorem{lemma}{Lemma}[section]
\newtheorem{corollary}[theorem]{Corollary}
\newtheorem{example}{Example}[section]
\newtheorem{definition}[problem]{Definition}
%\newtheorem{thm}{Theorem}[section] 
%\newtheorem{defn}[thm]{Definition}
%\newtheorem{algorithm}{Algorithm}[section]
%\newtheorem{cor}{Corollary}
\newcommand{\BEQA}{\begin{eqnarray}}
\newcommand{\EEQA}{\end{eqnarray}}
\newcommand{\define}{\stackrel{\triangle}{=}}

\bibliographystyle{IEEEtran}
%\bibliographystyle{ieeetr}


\providecommand{\mbf}{\mathbf}
\providecommand{\pr}[1]{\ensuremath{\Pr\left(#1\right)}}
\providecommand{\qfunc}[1]{\ensuremath{Q\left(#1\right)}}
\providecommand{\sbrak}[1]{\ensuremath{{}\left[#1\right]}}
\providecommand{\lsbrak}[1]{\ensuremath{{}\left[#1\right.}}
\providecommand{\rsbrak}[1]{\ensuremath{{}\left.#1\right]}}
\providecommand{\brak}[1]{\ensuremath{\left(#1\right)}}
\providecommand{\lbrak}[1]{\ensuremath{\left(#1\right.}}
\providecommand{\rbrak}[1]{\ensuremath{\left.#1\right)}}
\providecommand{\cbrak}[1]{\ensuremath{\left\{#1\right\}}}
\providecommand{\lcbrak}[1]{\ensuremath{\left\{#1\right.}}
\providecommand{\rcbrak}[1]{\ensuremath{\left.#1\right\}}}
\theoremstyle{remark}
\newtheorem{rem}{Remark}
\newcommand{\sgn}{\mathop{\mathrm{sgn}}}
\providecommand{\abs}[1]{\left\vert#1\right\vert}
\providecommand{\res}[1]{\Res\displaylimits_{#1}} 
\providecommand{\norm}[1]{\left\lVert#1\right\rVert}
%\providecommand{\norm}[1]{\lVert#1\rVert}
\providecommand{\mtx}[1]{\mathbf{#1}}
\providecommand{\mean}[1]{E\left[ #1 \right]}
\providecommand{\fourier}{\overset{\mathcal{F}}{ \rightleftharpoons}}
%\providecommand{\hilbert}{\overset{\mathcal{H}}{ \rightleftharpoons}}
\providecommand{\system}{\overset{\mathcal{H}}{ \longleftrightarrow}}
	%\newcommand{\solution}[2]{\textbf{Solution:}{#1}}
\newcommand{\solution}{\noindent \textbf{Solution: }}
\newcommand{\cosec}{\,\text{cosec}\,}
\providecommand{\dec}[2]{\ensuremath{\overset{#1}{\underset{#2}{\gtrless}}}}
\newcommand{\myvec}[1]{\ensuremath{\begin{pmatrix}#1\end{pmatrix}}}
\newcommand{\mydet}[1]{\ensuremath{\begin{vmatrix}#1\end{vmatrix}}}
%\numberwithin{equation}{section}
\numberwithin{equation}{subsection}
%\numberwithin{problem}{section}
%\numberwithin{definition}{section}
\makeatletter
\@addtoreset{figure}{problem}
\makeatother

\let\StandardTheFigure\thefigure
\let\vec\mathbf
%\renewcommand{\thefigure}{\theproblem.\arabic{figure}}
\renewcommand{\thefigure}{\theproblem}
%\setlist[enumerate,1]{before=\renewcommand\theequation{\theenumi.\arabic{equation}}
%\counterwithin{equation}{enumi}


%\renewcommand{\theequation}{\arabic{subsection}.\arabic{equation}}

\def\putbox#1#2#3{\makebox[0in][l]{\makebox[#1][l]{}\raisebox{\baselineskip}[0in][0in]{\raisebox{#2}[0in][0in]{#3}}}}
     \def\rightbox#1{\makebox[0in][r]{#1}}
     \def\centbox#1{\makebox[0in]{#1}}
     \def\topbox#1{\raisebox{-\baselineskip}[0in][0in]{#1}}
     \def\midbox#1{\raisebox{-0.5\baselineskip}[0in][0in]{#1}}

\vspace{3cm}

\title{
	\logo{
Solutions: Linear Algebra by Hoffman and Kunze
	}
}
\author{ G V V Sharma$^{*}$% <-this % stops a space
	\thanks{*The author is with the Department
		of Electrical Engineering, Indian Institute of Technology, Hyderabad
		502285 India e-mail:  gadepall@iith.ac.in. All content in this manual is released under GNU GPL.  Free and open source.}
	
}	
%\title{
%	\logo{Matrix Analysis through Octave}{\begin{center}\includegraphics[scale=.24]{tlc}\end{center}}{}{HAMDSP}
%}


% paper title
% can use linebreaks \\ within to get better formatting as desired
%\title{Matrix Analysis through Octave}
%
%
% author names and IEEE memberships
% note positions of commas and nonbreaking spaces ( ~ ) LaTeX will not break
% a structure at a ~ so this keeps an author's name from being broken across
% two lines.
% use \thanks{} to gain access to the first footnote area
% a separate \thanks must be used for each paragraph as LaTeX2e's \thanks
% was not built to handle multiple paragraphs
%

%\author{<-this % stops a space
%\thanks{}}
%}
% note the % following the last \IEEEmembership and also \thanks - 
% these prevent an unwanted space from occurring between the last author name
% and the end of the author line. i.e., if you had this:
% 
% \author{....lastname \thanks{...} \thanks{...} }
%                     ^------------^------------^----Do not want these spaces!
%
% a space would be appended to the last name and could cause every name on that
% line to be shifted left slightly. This is one of those "LaTeX things". For
% instance, "\textbf{A} \textbf{B}" will typeset as "A B" not "AB". To get
% "AB" then you have to do: "\textbf{A}\textbf{B}"
% \thanks is no different in this regard, so shield the last } of each \thanks
% that ends a line with a % and do not let a space in before the next \thanks.
% Spaces after \IEEEmembership other than the last one are OK (and needed) as
% you are supposed to have spaces between the names. For what it is worth,
% this is a minor point as most people would not even notice if the said evil
% space somehow managed to creep in.



% The paper headers
%\markboth{Journal of \LaTeX\ Class Files,~Vol.~6, No.~1, January~2007}%
%{Shell \MakeLowercase{\textit{et al.}}: Bare Demo of IEEEtran.cls for Journals}
% The only time the second header will appear is for the odd numbered pages
% after the title page when using the twoside option.
% 
% *** Note that you probably will NOT want to include the author's ***
% *** name in the headers of peer review papers.                   ***
% You can use \ifCLASSOPTIONpeerreview for conditional compilation here if
% you desire.




% If you want to put a publisher's ID mark on the page you can do it like
% this:
%\IEEEpubid{0000--0000/00\$00.00~\copyright~2007 IEEE}
% Remember, if you use this you must call \IEEEpubidadjcol in the second
% column for its text to clear the IEEEpubid mark.



% make the title area
\maketitle

\newpage

\tableofcontents

\bigskip

\renewcommand{\thefigure}{\theenumi}
\renewcommand{\thetable}{\theenumi}
%\renewcommand{\theequation}{\theenumi}

%\begin{abstract}
%%\boldmath
%In this letter, an algorithm for evaluating the exact analytical bit error rate  (BER)  for the piecewise linear (PL) combiner for  multiple relays is presented. Previous results were available only for upto three relays. The algorithm is unique in the sense that  the actual mathematical expressions, that are prohibitively large, need not be explicitly obtained. The diversity gain due to multiple relays is shown through plots of the analytical BER, well supported by simulations. 
%
%\end{abstract}
% IEEEtran.cls defaults to using nonbold math in the Abstract.
% This preserves the distinction between vectors and scalars. However,
% if the journal you are submitting to favors bold math in the abstract,
% then you can use LaTeX's standard command \boldmath at the very start
% of the abstract to achieve this. Many IEEE journals frown on math
% in the abstract anyway.

% Note that keywords are not normally used for peerreview papers.
%\begin{IEEEkeywords}
%Cooperative diversity, decode and forward, piecewise linear
%\end{IEEEkeywords}



% For peer review papers, you can put extra information on the cover
% page as needed:
% \ifCLASSOPTIONpeerreview
% \begin{center} \bfseries EDICS Category: 3-BBND \end{center}
% \fi
%
% For peerreview papers, this IEEEtran command inserts a page break and
% creates the second title. It will be ignored for other modes.
%\IEEEpeerreviewmaketitle

\begin{abstract}
This book provides solutions to the Linear Algebra book by Hoffman and Kunze.
\end{abstract}



%\section{Exercises}
%
\section{Linear Equations}
\subsection{Fields and Linear Equations}
\renewcommand{\theequation}{\theenumi}
\renewcommand{\thefigure}{\theenumi}
\begin{enumerate}[label=\thesubsection.\arabic*.,ref=\thesubsection.\theenumi]
\numberwithin{equation}{enumi}
\numberwithin{figure}{enumi}
%
\item  Verify that the set of complex numbers numbers described in the form of c where x and y are rational is a sub-field of C.
\\
\solution
See Table \ref{eq:solutions/6/2/11/table:1}
\onecolumn
\begin{longtable}{|l|l|}
\hline
\multirow{3}{*}{} & \\
Statement&Solution\\
\hline
&\parbox{6cm}{\begin{align}
\mbox{Let }\vec{N}&=\myvec{a&b\\c&d}\\
\mbox{Since }\vec{N}^2&=0
\end{align}}\\
&If $\myvec{a\\c},\myvec{b\\d}$ are linearly independent then $\vec{N}$ is diagonalizable to $\myvec{0&0\\0&0}$.\\
&\parbox{6cm}{\begin{align}
\mbox{If }\vec{P}\vec{N}\vec{P}^{-1}=0\\
\mbox{then }\vec{N}=\vec{P}^{-1}\vec{0}\vec{P}=0
\end{align}}\\
Proof that&So in this case $\vec{N}$ itself is the zero matrix.\\
$\vec{N}=0$&This contradicts the assumption that $\myvec{a\\c},\myvec{b\\d}$ are linearly independent.\\
&$\therefore$ we can assume that $\myvec{a\\c},\myvec{b\\d}$ are linearly dependent if both are\\
&equal to the zero vector\\
&\parbox{6cm}
{\begin{align}
   \mbox{then } \vec{N} &= 0.
\end{align}}\\
%\hline
%\pagebreak
\hline
&\\
&Therefore we can assume at least one vector is non-zero.\\
Assuming $\myvec{b\\d}$ as&Therefore
$\vec{N}=\myvec{a&0\\c&0}$\\
the zero vector&\\
&\parbox{6cm}{\begin{align}
\mbox{So }\vec{N}^2&=0\\
\implies a^2&=0\\
\therefore a&=0\\
\mbox{Thus }\vec{N}&=\myvec{a&0\\c&0}
\end{align}}\\
&In this case $\vec{N}$ is similar to $\vec{N}=\myvec{0&0\\1&0}$ via the matrix $\vec{P}=\myvec{c&0\\0&1}$\\
&\\
\hline
&\\
Assuming $\myvec{a\\c}$ as&Therefore
$\vec{N}=\myvec{0&b\\0&d}$\\
the zero vector&\\
&\parbox{6cm}{\begin{align}
\mbox{Then }\vec{N}^2&=0\\
\implies d^2&=0\\
\therefore d&=0\\
\mbox{Thus }\vec{N}&=\myvec{0&b\\0&0}
\end{align}}\\
&In this case $\vec{N}$ is similar to $\vec{N}=\myvec{0&0\\b&0}$ via the matrix $\vec{P}=\myvec{0&1\\1&0}$,\\
&which is similar to $\myvec{0&0\\1&0}$ as above.\\
&\\
\hline
&\\
Hence& we can assume neither $\myvec{a\\c}$ or $\myvec{b\\d}$ is the zero vector.\\
&\\
%\hline
%\pagebreak
\hline
&\\
Consequences of &Since they are linearly dependent we can assume,\\
linear&\\
independence&\\
&\parbox{6cm}{\begin{align}
\myvec{b\\d}&=x\myvec{a\\c}\\
\therefore \vec{N}&=\myvec{a&ax\\c&cx}\\
\therefore \vec{N}^2&=0\\
\implies a(a+cx)&=0\\
c(a+cx)&=0\\
ax(a+cx)&=0\\
cx(a+cx)&=0
\end{align}}\\
\hline
&\\
Proof that $\vec{N}$ is&We know that at least one of a or c is not zero.\\
similar over $\mathbb{C}$ to&If a = 0 then c $\neq$ 0, it must be that x = 0.\\
$\myvec{0&0\\1&0}$&So in this case $\vec{N}=\myvec{0&0\\c&0}$ which is similar to $\myvec{0&0\\1&0}$ as before.\\
&\\
&\parbox{6cm}{\begin{align}
\mbox{If } a &\neq0\\
\mbox{then }x &\neq0\\
\mbox{else }a(a+cx)&=0\\
\implies a&=0\\
\mbox{Thus }a+cx&=0\\
\mbox{Hence }\vec{N}&=\myvec{a&ax\\ \frac{-a}{x}&-a}
\end{align}}\\
 &This is similar to $\myvec{a&a\\-a&-a}$ via $\vec{P}=\myvec{\sqrt{x}&0\\0&\frac{1}{\sqrt{x}}}$.\\
 &And $\myvec{a&a\\-a&-a}$ is similar to $\myvec{0&0\\-a&0}$ via $\vec{P}=\myvec{-1&-1\\1&0}$\\
 &And this finally is similar to $\myvec{0&0\\1&0}$ as before.\\
 &\\
\hline
&\\
Conclusion &Thus either $\vec{N}$ = 0 or $\vec{N}$ is similar over $\mathbb{C}$ to $\myvec{0&0\\1&0}$.\\
&\\
\hline
\caption{Solution summary}
\label{eq:solutions/6/2/11/table:1}
\end{longtable}

\item Let $\mathbb{F}$ be the field of complex numbers. Are the following two systems of linear equations equivalent? If so, express each equation in each system as a linear combination of the equations in the other system.
\begin{align*}
    x_1 - x_2 &=0\\
    2x_1 + x_2 &=0
\end{align*}
and 
\begin{align*}
    3x_1 + x_2 &=0 \\
    x_1 + x_2 &=0
\end{align*}
%
\solution
See Table \ref{eq:solutions/6/2/11/table:1}
\onecolumn
\begin{longtable}{|l|l|}
\hline
\multirow{3}{*}{} & \\
Statement&Solution\\
\hline
&\parbox{6cm}{\begin{align}
\mbox{Let }\vec{N}&=\myvec{a&b\\c&d}\\
\mbox{Since }\vec{N}^2&=0
\end{align}}\\
&If $\myvec{a\\c},\myvec{b\\d}$ are linearly independent then $\vec{N}$ is diagonalizable to $\myvec{0&0\\0&0}$.\\
&\parbox{6cm}{\begin{align}
\mbox{If }\vec{P}\vec{N}\vec{P}^{-1}=0\\
\mbox{then }\vec{N}=\vec{P}^{-1}\vec{0}\vec{P}=0
\end{align}}\\
Proof that&So in this case $\vec{N}$ itself is the zero matrix.\\
$\vec{N}=0$&This contradicts the assumption that $\myvec{a\\c},\myvec{b\\d}$ are linearly independent.\\
&$\therefore$ we can assume that $\myvec{a\\c},\myvec{b\\d}$ are linearly dependent if both are\\
&equal to the zero vector\\
&\parbox{6cm}
{\begin{align}
   \mbox{then } \vec{N} &= 0.
\end{align}}\\
%\hline
%\pagebreak
\hline
&\\
&Therefore we can assume at least one vector is non-zero.\\
Assuming $\myvec{b\\d}$ as&Therefore
$\vec{N}=\myvec{a&0\\c&0}$\\
the zero vector&\\
&\parbox{6cm}{\begin{align}
\mbox{So }\vec{N}^2&=0\\
\implies a^2&=0\\
\therefore a&=0\\
\mbox{Thus }\vec{N}&=\myvec{a&0\\c&0}
\end{align}}\\
&In this case $\vec{N}$ is similar to $\vec{N}=\myvec{0&0\\1&0}$ via the matrix $\vec{P}=\myvec{c&0\\0&1}$\\
&\\
\hline
&\\
Assuming $\myvec{a\\c}$ as&Therefore
$\vec{N}=\myvec{0&b\\0&d}$\\
the zero vector&\\
&\parbox{6cm}{\begin{align}
\mbox{Then }\vec{N}^2&=0\\
\implies d^2&=0\\
\therefore d&=0\\
\mbox{Thus }\vec{N}&=\myvec{0&b\\0&0}
\end{align}}\\
&In this case $\vec{N}$ is similar to $\vec{N}=\myvec{0&0\\b&0}$ via the matrix $\vec{P}=\myvec{0&1\\1&0}$,\\
&which is similar to $\myvec{0&0\\1&0}$ as above.\\
&\\
\hline
&\\
Hence& we can assume neither $\myvec{a\\c}$ or $\myvec{b\\d}$ is the zero vector.\\
&\\
%\hline
%\pagebreak
\hline
&\\
Consequences of &Since they are linearly dependent we can assume,\\
linear&\\
independence&\\
&\parbox{6cm}{\begin{align}
\myvec{b\\d}&=x\myvec{a\\c}\\
\therefore \vec{N}&=\myvec{a&ax\\c&cx}\\
\therefore \vec{N}^2&=0\\
\implies a(a+cx)&=0\\
c(a+cx)&=0\\
ax(a+cx)&=0\\
cx(a+cx)&=0
\end{align}}\\
\hline
&\\
Proof that $\vec{N}$ is&We know that at least one of a or c is not zero.\\
similar over $\mathbb{C}$ to&If a = 0 then c $\neq$ 0, it must be that x = 0.\\
$\myvec{0&0\\1&0}$&So in this case $\vec{N}=\myvec{0&0\\c&0}$ which is similar to $\myvec{0&0\\1&0}$ as before.\\
&\\
&\parbox{6cm}{\begin{align}
\mbox{If } a &\neq0\\
\mbox{then }x &\neq0\\
\mbox{else }a(a+cx)&=0\\
\implies a&=0\\
\mbox{Thus }a+cx&=0\\
\mbox{Hence }\vec{N}&=\myvec{a&ax\\ \frac{-a}{x}&-a}
\end{align}}\\
 &This is similar to $\myvec{a&a\\-a&-a}$ via $\vec{P}=\myvec{\sqrt{x}&0\\0&\frac{1}{\sqrt{x}}}$.\\
 &And $\myvec{a&a\\-a&-a}$ is similar to $\myvec{0&0\\-a&0}$ via $\vec{P}=\myvec{-1&-1\\1&0}$\\
 &And this finally is similar to $\myvec{0&0\\1&0}$ as before.\\
 &\\
\hline
&\\
Conclusion &Thus either $\vec{N}$ = 0 or $\vec{N}$ is similar over $\mathbb{C}$ to $\myvec{0&0\\1&0}$.\\
&\\
\hline
\caption{Solution summary}
\label{eq:solutions/6/2/11/table:1}
\end{longtable}


\item Are the following two systems of linear equations equivalent?
\begin{equation} \label{eq:solutions/1/1/3/eq:eq1}
\begin{split}
    -x_1+x_2+4x_3=0\\
    x_1+3x_2+8x_3=0\\
    \frac{1}{2}x_1+x_2+\frac{5}{2}x_3=0
\end{split}
\end{equation}
\solution
See Table \ref{eq:solutions/6/2/11/table:1}
\onecolumn
\begin{longtable}{|l|l|}
\hline
\multirow{3}{*}{} & \\
Statement&Solution\\
\hline
&\parbox{6cm}{\begin{align}
\mbox{Let }\vec{N}&=\myvec{a&b\\c&d}\\
\mbox{Since }\vec{N}^2&=0
\end{align}}\\
&If $\myvec{a\\c},\myvec{b\\d}$ are linearly independent then $\vec{N}$ is diagonalizable to $\myvec{0&0\\0&0}$.\\
&\parbox{6cm}{\begin{align}
\mbox{If }\vec{P}\vec{N}\vec{P}^{-1}=0\\
\mbox{then }\vec{N}=\vec{P}^{-1}\vec{0}\vec{P}=0
\end{align}}\\
Proof that&So in this case $\vec{N}$ itself is the zero matrix.\\
$\vec{N}=0$&This contradicts the assumption that $\myvec{a\\c},\myvec{b\\d}$ are linearly independent.\\
&$\therefore$ we can assume that $\myvec{a\\c},\myvec{b\\d}$ are linearly dependent if both are\\
&equal to the zero vector\\
&\parbox{6cm}
{\begin{align}
   \mbox{then } \vec{N} &= 0.
\end{align}}\\
%\hline
%\pagebreak
\hline
&\\
&Therefore we can assume at least one vector is non-zero.\\
Assuming $\myvec{b\\d}$ as&Therefore
$\vec{N}=\myvec{a&0\\c&0}$\\
the zero vector&\\
&\parbox{6cm}{\begin{align}
\mbox{So }\vec{N}^2&=0\\
\implies a^2&=0\\
\therefore a&=0\\
\mbox{Thus }\vec{N}&=\myvec{a&0\\c&0}
\end{align}}\\
&In this case $\vec{N}$ is similar to $\vec{N}=\myvec{0&0\\1&0}$ via the matrix $\vec{P}=\myvec{c&0\\0&1}$\\
&\\
\hline
&\\
Assuming $\myvec{a\\c}$ as&Therefore
$\vec{N}=\myvec{0&b\\0&d}$\\
the zero vector&\\
&\parbox{6cm}{\begin{align}
\mbox{Then }\vec{N}^2&=0\\
\implies d^2&=0\\
\therefore d&=0\\
\mbox{Thus }\vec{N}&=\myvec{0&b\\0&0}
\end{align}}\\
&In this case $\vec{N}$ is similar to $\vec{N}=\myvec{0&0\\b&0}$ via the matrix $\vec{P}=\myvec{0&1\\1&0}$,\\
&which is similar to $\myvec{0&0\\1&0}$ as above.\\
&\\
\hline
&\\
Hence& we can assume neither $\myvec{a\\c}$ or $\myvec{b\\d}$ is the zero vector.\\
&\\
%\hline
%\pagebreak
\hline
&\\
Consequences of &Since they are linearly dependent we can assume,\\
linear&\\
independence&\\
&\parbox{6cm}{\begin{align}
\myvec{b\\d}&=x\myvec{a\\c}\\
\therefore \vec{N}&=\myvec{a&ax\\c&cx}\\
\therefore \vec{N}^2&=0\\
\implies a(a+cx)&=0\\
c(a+cx)&=0\\
ax(a+cx)&=0\\
cx(a+cx)&=0
\end{align}}\\
\hline
&\\
Proof that $\vec{N}$ is&We know that at least one of a or c is not zero.\\
similar over $\mathbb{C}$ to&If a = 0 then c $\neq$ 0, it must be that x = 0.\\
$\myvec{0&0\\1&0}$&So in this case $\vec{N}=\myvec{0&0\\c&0}$ which is similar to $\myvec{0&0\\1&0}$ as before.\\
&\\
&\parbox{6cm}{\begin{align}
\mbox{If } a &\neq0\\
\mbox{then }x &\neq0\\
\mbox{else }a(a+cx)&=0\\
\implies a&=0\\
\mbox{Thus }a+cx&=0\\
\mbox{Hence }\vec{N}&=\myvec{a&ax\\ \frac{-a}{x}&-a}
\end{align}}\\
 &This is similar to $\myvec{a&a\\-a&-a}$ via $\vec{P}=\myvec{\sqrt{x}&0\\0&\frac{1}{\sqrt{x}}}$.\\
 &And $\myvec{a&a\\-a&-a}$ is similar to $\myvec{0&0\\-a&0}$ via $\vec{P}=\myvec{-1&-1\\1&0}$\\
 &And this finally is similar to $\myvec{0&0\\1&0}$ as before.\\
 &\\
\hline
&\\
Conclusion &Thus either $\vec{N}$ = 0 or $\vec{N}$ is similar over $\mathbb{C}$ to $\myvec{0&0\\1&0}$.\\
&\\
\hline
\caption{Solution summary}
\label{eq:solutions/6/2/11/table:1}
\end{longtable}


\item Let $\mathbb{F}$ be a set which contains exactly two elements,0 and 1.Define an addition and multiplication by tables.
\begin{table}[h!]
  \begin{center}
    \label{eq:solutions/1/1/5/tab:table1}
    \begin{tabular}{l|c|r}
      + & 0 & 1 \\
      \hline
      0 & 0 & 1\\
      1 & 1 & 0
    \end{tabular}
  \end{center}
\end{table}
\begin{table}[h!]
  \begin{center}
    \label{eq:solutions/1/1/5/tab:table2}
    \begin{tabular}{l|c|r}
      $\cdot$ & 0 & 1 \\
      \hline
      0 & 0 & 0\\
      1 & 0 & 1
    \end{tabular}
  \end{center}
\end{table}
  Verify that the set $\mathbb{F}$, together with these two operations, is a field.
%
\\
\solution
See Table \ref{eq:solutions/6/2/11/table:1}
\onecolumn
\begin{longtable}{|l|l|}
\hline
\multirow{3}{*}{} & \\
Statement&Solution\\
\hline
&\parbox{6cm}{\begin{align}
\mbox{Let }\vec{N}&=\myvec{a&b\\c&d}\\
\mbox{Since }\vec{N}^2&=0
\end{align}}\\
&If $\myvec{a\\c},\myvec{b\\d}$ are linearly independent then $\vec{N}$ is diagonalizable to $\myvec{0&0\\0&0}$.\\
&\parbox{6cm}{\begin{align}
\mbox{If }\vec{P}\vec{N}\vec{P}^{-1}=0\\
\mbox{then }\vec{N}=\vec{P}^{-1}\vec{0}\vec{P}=0
\end{align}}\\
Proof that&So in this case $\vec{N}$ itself is the zero matrix.\\
$\vec{N}=0$&This contradicts the assumption that $\myvec{a\\c},\myvec{b\\d}$ are linearly independent.\\
&$\therefore$ we can assume that $\myvec{a\\c},\myvec{b\\d}$ are linearly dependent if both are\\
&equal to the zero vector\\
&\parbox{6cm}
{\begin{align}
   \mbox{then } \vec{N} &= 0.
\end{align}}\\
%\hline
%\pagebreak
\hline
&\\
&Therefore we can assume at least one vector is non-zero.\\
Assuming $\myvec{b\\d}$ as&Therefore
$\vec{N}=\myvec{a&0\\c&0}$\\
the zero vector&\\
&\parbox{6cm}{\begin{align}
\mbox{So }\vec{N}^2&=0\\
\implies a^2&=0\\
\therefore a&=0\\
\mbox{Thus }\vec{N}&=\myvec{a&0\\c&0}
\end{align}}\\
&In this case $\vec{N}$ is similar to $\vec{N}=\myvec{0&0\\1&0}$ via the matrix $\vec{P}=\myvec{c&0\\0&1}$\\
&\\
\hline
&\\
Assuming $\myvec{a\\c}$ as&Therefore
$\vec{N}=\myvec{0&b\\0&d}$\\
the zero vector&\\
&\parbox{6cm}{\begin{align}
\mbox{Then }\vec{N}^2&=0\\
\implies d^2&=0\\
\therefore d&=0\\
\mbox{Thus }\vec{N}&=\myvec{0&b\\0&0}
\end{align}}\\
&In this case $\vec{N}$ is similar to $\vec{N}=\myvec{0&0\\b&0}$ via the matrix $\vec{P}=\myvec{0&1\\1&0}$,\\
&which is similar to $\myvec{0&0\\1&0}$ as above.\\
&\\
\hline
&\\
Hence& we can assume neither $\myvec{a\\c}$ or $\myvec{b\\d}$ is the zero vector.\\
&\\
%\hline
%\pagebreak
\hline
&\\
Consequences of &Since they are linearly dependent we can assume,\\
linear&\\
independence&\\
&\parbox{6cm}{\begin{align}
\myvec{b\\d}&=x\myvec{a\\c}\\
\therefore \vec{N}&=\myvec{a&ax\\c&cx}\\
\therefore \vec{N}^2&=0\\
\implies a(a+cx)&=0\\
c(a+cx)&=0\\
ax(a+cx)&=0\\
cx(a+cx)&=0
\end{align}}\\
\hline
&\\
Proof that $\vec{N}$ is&We know that at least one of a or c is not zero.\\
similar over $\mathbb{C}$ to&If a = 0 then c $\neq$ 0, it must be that x = 0.\\
$\myvec{0&0\\1&0}$&So in this case $\vec{N}=\myvec{0&0\\c&0}$ which is similar to $\myvec{0&0\\1&0}$ as before.\\
&\\
&\parbox{6cm}{\begin{align}
\mbox{If } a &\neq0\\
\mbox{then }x &\neq0\\
\mbox{else }a(a+cx)&=0\\
\implies a&=0\\
\mbox{Thus }a+cx&=0\\
\mbox{Hence }\vec{N}&=\myvec{a&ax\\ \frac{-a}{x}&-a}
\end{align}}\\
 &This is similar to $\myvec{a&a\\-a&-a}$ via $\vec{P}=\myvec{\sqrt{x}&0\\0&\frac{1}{\sqrt{x}}}$.\\
 &And $\myvec{a&a\\-a&-a}$ is similar to $\myvec{0&0\\-a&0}$ via $\vec{P}=\myvec{-1&-1\\1&0}$\\
 &And this finally is similar to $\myvec{0&0\\1&0}$ as before.\\
 &\\
\hline
&\\
Conclusion &Thus either $\vec{N}$ = 0 or $\vec{N}$ is similar over $\mathbb{C}$ to $\myvec{0&0\\1&0}$.\\
&\\
\hline
\caption{Solution summary}
\label{eq:solutions/6/2/11/table:1}
\end{longtable}

\item Prove that if two homogenous systems of linear equations in two unknowns have the same solutions, then they are equivalent.
\\
\solution
See Table \ref{eq:solutions/6/2/11/table:1}
\onecolumn
\begin{longtable}{|l|l|}
\hline
\multirow{3}{*}{} & \\
Statement&Solution\\
\hline
&\parbox{6cm}{\begin{align}
\mbox{Let }\vec{N}&=\myvec{a&b\\c&d}\\
\mbox{Since }\vec{N}^2&=0
\end{align}}\\
&If $\myvec{a\\c},\myvec{b\\d}$ are linearly independent then $\vec{N}$ is diagonalizable to $\myvec{0&0\\0&0}$.\\
&\parbox{6cm}{\begin{align}
\mbox{If }\vec{P}\vec{N}\vec{P}^{-1}=0\\
\mbox{then }\vec{N}=\vec{P}^{-1}\vec{0}\vec{P}=0
\end{align}}\\
Proof that&So in this case $\vec{N}$ itself is the zero matrix.\\
$\vec{N}=0$&This contradicts the assumption that $\myvec{a\\c},\myvec{b\\d}$ are linearly independent.\\
&$\therefore$ we can assume that $\myvec{a\\c},\myvec{b\\d}$ are linearly dependent if both are\\
&equal to the zero vector\\
&\parbox{6cm}
{\begin{align}
   \mbox{then } \vec{N} &= 0.
\end{align}}\\
%\hline
%\pagebreak
\hline
&\\
&Therefore we can assume at least one vector is non-zero.\\
Assuming $\myvec{b\\d}$ as&Therefore
$\vec{N}=\myvec{a&0\\c&0}$\\
the zero vector&\\
&\parbox{6cm}{\begin{align}
\mbox{So }\vec{N}^2&=0\\
\implies a^2&=0\\
\therefore a&=0\\
\mbox{Thus }\vec{N}&=\myvec{a&0\\c&0}
\end{align}}\\
&In this case $\vec{N}$ is similar to $\vec{N}=\myvec{0&0\\1&0}$ via the matrix $\vec{P}=\myvec{c&0\\0&1}$\\
&\\
\hline
&\\
Assuming $\myvec{a\\c}$ as&Therefore
$\vec{N}=\myvec{0&b\\0&d}$\\
the zero vector&\\
&\parbox{6cm}{\begin{align}
\mbox{Then }\vec{N}^2&=0\\
\implies d^2&=0\\
\therefore d&=0\\
\mbox{Thus }\vec{N}&=\myvec{0&b\\0&0}
\end{align}}\\
&In this case $\vec{N}$ is similar to $\vec{N}=\myvec{0&0\\b&0}$ via the matrix $\vec{P}=\myvec{0&1\\1&0}$,\\
&which is similar to $\myvec{0&0\\1&0}$ as above.\\
&\\
\hline
&\\
Hence& we can assume neither $\myvec{a\\c}$ or $\myvec{b\\d}$ is the zero vector.\\
&\\
%\hline
%\pagebreak
\hline
&\\
Consequences of &Since they are linearly dependent we can assume,\\
linear&\\
independence&\\
&\parbox{6cm}{\begin{align}
\myvec{b\\d}&=x\myvec{a\\c}\\
\therefore \vec{N}&=\myvec{a&ax\\c&cx}\\
\therefore \vec{N}^2&=0\\
\implies a(a+cx)&=0\\
c(a+cx)&=0\\
ax(a+cx)&=0\\
cx(a+cx)&=0
\end{align}}\\
\hline
&\\
Proof that $\vec{N}$ is&We know that at least one of a or c is not zero.\\
similar over $\mathbb{C}$ to&If a = 0 then c $\neq$ 0, it must be that x = 0.\\
$\myvec{0&0\\1&0}$&So in this case $\vec{N}=\myvec{0&0\\c&0}$ which is similar to $\myvec{0&0\\1&0}$ as before.\\
&\\
&\parbox{6cm}{\begin{align}
\mbox{If } a &\neq0\\
\mbox{then }x &\neq0\\
\mbox{else }a(a+cx)&=0\\
\implies a&=0\\
\mbox{Thus }a+cx&=0\\
\mbox{Hence }\vec{N}&=\myvec{a&ax\\ \frac{-a}{x}&-a}
\end{align}}\\
 &This is similar to $\myvec{a&a\\-a&-a}$ via $\vec{P}=\myvec{\sqrt{x}&0\\0&\frac{1}{\sqrt{x}}}$.\\
 &And $\myvec{a&a\\-a&-a}$ is similar to $\myvec{0&0\\-a&0}$ via $\vec{P}=\myvec{-1&-1\\1&0}$\\
 &And this finally is similar to $\myvec{0&0\\1&0}$ as before.\\
 &\\
\hline
&\\
Conclusion &Thus either $\vec{N}$ = 0 or $\vec{N}$ is similar over $\mathbb{C}$ to $\myvec{0&0\\1&0}$.\\
&\\
\hline
\caption{Solution summary}
\label{eq:solutions/6/2/11/table:1}
\end{longtable}

\item Prove that each subfield of the field of complex number contains every rational number
\\
\solution
See Table \ref{eq:solutions/6/2/11/table:1}
\onecolumn
\begin{longtable}{|l|l|}
\hline
\multirow{3}{*}{} & \\
Statement&Solution\\
\hline
&\parbox{6cm}{\begin{align}
\mbox{Let }\vec{N}&=\myvec{a&b\\c&d}\\
\mbox{Since }\vec{N}^2&=0
\end{align}}\\
&If $\myvec{a\\c},\myvec{b\\d}$ are linearly independent then $\vec{N}$ is diagonalizable to $\myvec{0&0\\0&0}$.\\
&\parbox{6cm}{\begin{align}
\mbox{If }\vec{P}\vec{N}\vec{P}^{-1}=0\\
\mbox{then }\vec{N}=\vec{P}^{-1}\vec{0}\vec{P}=0
\end{align}}\\
Proof that&So in this case $\vec{N}$ itself is the zero matrix.\\
$\vec{N}=0$&This contradicts the assumption that $\myvec{a\\c},\myvec{b\\d}$ are linearly independent.\\
&$\therefore$ we can assume that $\myvec{a\\c},\myvec{b\\d}$ are linearly dependent if both are\\
&equal to the zero vector\\
&\parbox{6cm}
{\begin{align}
   \mbox{then } \vec{N} &= 0.
\end{align}}\\
%\hline
%\pagebreak
\hline
&\\
&Therefore we can assume at least one vector is non-zero.\\
Assuming $\myvec{b\\d}$ as&Therefore
$\vec{N}=\myvec{a&0\\c&0}$\\
the zero vector&\\
&\parbox{6cm}{\begin{align}
\mbox{So }\vec{N}^2&=0\\
\implies a^2&=0\\
\therefore a&=0\\
\mbox{Thus }\vec{N}&=\myvec{a&0\\c&0}
\end{align}}\\
&In this case $\vec{N}$ is similar to $\vec{N}=\myvec{0&0\\1&0}$ via the matrix $\vec{P}=\myvec{c&0\\0&1}$\\
&\\
\hline
&\\
Assuming $\myvec{a\\c}$ as&Therefore
$\vec{N}=\myvec{0&b\\0&d}$\\
the zero vector&\\
&\parbox{6cm}{\begin{align}
\mbox{Then }\vec{N}^2&=0\\
\implies d^2&=0\\
\therefore d&=0\\
\mbox{Thus }\vec{N}&=\myvec{0&b\\0&0}
\end{align}}\\
&In this case $\vec{N}$ is similar to $\vec{N}=\myvec{0&0\\b&0}$ via the matrix $\vec{P}=\myvec{0&1\\1&0}$,\\
&which is similar to $\myvec{0&0\\1&0}$ as above.\\
&\\
\hline
&\\
Hence& we can assume neither $\myvec{a\\c}$ or $\myvec{b\\d}$ is the zero vector.\\
&\\
%\hline
%\pagebreak
\hline
&\\
Consequences of &Since they are linearly dependent we can assume,\\
linear&\\
independence&\\
&\parbox{6cm}{\begin{align}
\myvec{b\\d}&=x\myvec{a\\c}\\
\therefore \vec{N}&=\myvec{a&ax\\c&cx}\\
\therefore \vec{N}^2&=0\\
\implies a(a+cx)&=0\\
c(a+cx)&=0\\
ax(a+cx)&=0\\
cx(a+cx)&=0
\end{align}}\\
\hline
&\\
Proof that $\vec{N}$ is&We know that at least one of a or c is not zero.\\
similar over $\mathbb{C}$ to&If a = 0 then c $\neq$ 0, it must be that x = 0.\\
$\myvec{0&0\\1&0}$&So in this case $\vec{N}=\myvec{0&0\\c&0}$ which is similar to $\myvec{0&0\\1&0}$ as before.\\
&\\
&\parbox{6cm}{\begin{align}
\mbox{If } a &\neq0\\
\mbox{then }x &\neq0\\
\mbox{else }a(a+cx)&=0\\
\implies a&=0\\
\mbox{Thus }a+cx&=0\\
\mbox{Hence }\vec{N}&=\myvec{a&ax\\ \frac{-a}{x}&-a}
\end{align}}\\
 &This is similar to $\myvec{a&a\\-a&-a}$ via $\vec{P}=\myvec{\sqrt{x}&0\\0&\frac{1}{\sqrt{x}}}$.\\
 &And $\myvec{a&a\\-a&-a}$ is similar to $\myvec{0&0\\-a&0}$ via $\vec{P}=\myvec{-1&-1\\1&0}$\\
 &And this finally is similar to $\myvec{0&0\\1&0}$ as before.\\
 &\\
\hline
&\\
Conclusion &Thus either $\vec{N}$ = 0 or $\vec{N}$ is similar over $\mathbb{C}$ to $\myvec{0&0\\1&0}$.\\
&\\
\hline
\caption{Solution summary}
\label{eq:solutions/6/2/11/table:1}
\end{longtable}

%
\item Prove that, each field of the characteristic zero contains a copy of the rational number field.
\\
\solution
See Table \ref{eq:solutions/6/2/11/table:1}
\onecolumn
\begin{longtable}{|l|l|}
\hline
\multirow{3}{*}{} & \\
Statement&Solution\\
\hline
&\parbox{6cm}{\begin{align}
\mbox{Let }\vec{N}&=\myvec{a&b\\c&d}\\
\mbox{Since }\vec{N}^2&=0
\end{align}}\\
&If $\myvec{a\\c},\myvec{b\\d}$ are linearly independent then $\vec{N}$ is diagonalizable to $\myvec{0&0\\0&0}$.\\
&\parbox{6cm}{\begin{align}
\mbox{If }\vec{P}\vec{N}\vec{P}^{-1}=0\\
\mbox{then }\vec{N}=\vec{P}^{-1}\vec{0}\vec{P}=0
\end{align}}\\
Proof that&So in this case $\vec{N}$ itself is the zero matrix.\\
$\vec{N}=0$&This contradicts the assumption that $\myvec{a\\c},\myvec{b\\d}$ are linearly independent.\\
&$\therefore$ we can assume that $\myvec{a\\c},\myvec{b\\d}$ are linearly dependent if both are\\
&equal to the zero vector\\
&\parbox{6cm}
{\begin{align}
   \mbox{then } \vec{N} &= 0.
\end{align}}\\
%\hline
%\pagebreak
\hline
&\\
&Therefore we can assume at least one vector is non-zero.\\
Assuming $\myvec{b\\d}$ as&Therefore
$\vec{N}=\myvec{a&0\\c&0}$\\
the zero vector&\\
&\parbox{6cm}{\begin{align}
\mbox{So }\vec{N}^2&=0\\
\implies a^2&=0\\
\therefore a&=0\\
\mbox{Thus }\vec{N}&=\myvec{a&0\\c&0}
\end{align}}\\
&In this case $\vec{N}$ is similar to $\vec{N}=\myvec{0&0\\1&0}$ via the matrix $\vec{P}=\myvec{c&0\\0&1}$\\
&\\
\hline
&\\
Assuming $\myvec{a\\c}$ as&Therefore
$\vec{N}=\myvec{0&b\\0&d}$\\
the zero vector&\\
&\parbox{6cm}{\begin{align}
\mbox{Then }\vec{N}^2&=0\\
\implies d^2&=0\\
\therefore d&=0\\
\mbox{Thus }\vec{N}&=\myvec{0&b\\0&0}
\end{align}}\\
&In this case $\vec{N}$ is similar to $\vec{N}=\myvec{0&0\\b&0}$ via the matrix $\vec{P}=\myvec{0&1\\1&0}$,\\
&which is similar to $\myvec{0&0\\1&0}$ as above.\\
&\\
\hline
&\\
Hence& we can assume neither $\myvec{a\\c}$ or $\myvec{b\\d}$ is the zero vector.\\
&\\
%\hline
%\pagebreak
\hline
&\\
Consequences of &Since they are linearly dependent we can assume,\\
linear&\\
independence&\\
&\parbox{6cm}{\begin{align}
\myvec{b\\d}&=x\myvec{a\\c}\\
\therefore \vec{N}&=\myvec{a&ax\\c&cx}\\
\therefore \vec{N}^2&=0\\
\implies a(a+cx)&=0\\
c(a+cx)&=0\\
ax(a+cx)&=0\\
cx(a+cx)&=0
\end{align}}\\
\hline
&\\
Proof that $\vec{N}$ is&We know that at least one of a or c is not zero.\\
similar over $\mathbb{C}$ to&If a = 0 then c $\neq$ 0, it must be that x = 0.\\
$\myvec{0&0\\1&0}$&So in this case $\vec{N}=\myvec{0&0\\c&0}$ which is similar to $\myvec{0&0\\1&0}$ as before.\\
&\\
&\parbox{6cm}{\begin{align}
\mbox{If } a &\neq0\\
\mbox{then }x &\neq0\\
\mbox{else }a(a+cx)&=0\\
\implies a&=0\\
\mbox{Thus }a+cx&=0\\
\mbox{Hence }\vec{N}&=\myvec{a&ax\\ \frac{-a}{x}&-a}
\end{align}}\\
 &This is similar to $\myvec{a&a\\-a&-a}$ via $\vec{P}=\myvec{\sqrt{x}&0\\0&\frac{1}{\sqrt{x}}}$.\\
 &And $\myvec{a&a\\-a&-a}$ is similar to $\myvec{0&0\\-a&0}$ via $\vec{P}=\myvec{-1&-1\\1&0}$\\
 &And this finally is similar to $\myvec{0&0\\1&0}$ as before.\\
 &\\
\hline
&\\
Conclusion &Thus either $\vec{N}$ = 0 or $\vec{N}$ is similar over $\mathbb{C}$ to $\myvec{0&0\\1&0}$.\\
&\\
\hline
\caption{Solution summary}
\label{eq:solutions/6/2/11/table:1}
\end{longtable}

\end{enumerate}



%
\subsection{Matrices and Elementary Row Operations}
\renewcommand{\theequation}{\theenumi}
\renewcommand{\thefigure}{\theenumi}
\begin{enumerate}[label=\thesubsection.\arabic*.,ref=\thesubsection.\theenumi]
\numberwithin{equation}{enumi}
\numberwithin{figure}{enumi}
%
\item Let
\begin{align}
    \vec{A}=\myvec{a & b\\c & d}
\end{align}
be a $2\times2$ matrix with complex entries. Suppose A is row-reduced and also that $a+b+c+d=0$. Prove that there are exactly three such matrices. 
\\
\solution
See Table \ref{eq:solutions/6/2/11/table:1}
\onecolumn
\begin{longtable}{|l|l|}
\hline
\multirow{3}{*}{} & \\
Statement&Solution\\
\hline
&\parbox{6cm}{\begin{align}
\mbox{Let }\vec{N}&=\myvec{a&b\\c&d}\\
\mbox{Since }\vec{N}^2&=0
\end{align}}\\
&If $\myvec{a\\c},\myvec{b\\d}$ are linearly independent then $\vec{N}$ is diagonalizable to $\myvec{0&0\\0&0}$.\\
&\parbox{6cm}{\begin{align}
\mbox{If }\vec{P}\vec{N}\vec{P}^{-1}=0\\
\mbox{then }\vec{N}=\vec{P}^{-1}\vec{0}\vec{P}=0
\end{align}}\\
Proof that&So in this case $\vec{N}$ itself is the zero matrix.\\
$\vec{N}=0$&This contradicts the assumption that $\myvec{a\\c},\myvec{b\\d}$ are linearly independent.\\
&$\therefore$ we can assume that $\myvec{a\\c},\myvec{b\\d}$ are linearly dependent if both are\\
&equal to the zero vector\\
&\parbox{6cm}
{\begin{align}
   \mbox{then } \vec{N} &= 0.
\end{align}}\\
%\hline
%\pagebreak
\hline
&\\
&Therefore we can assume at least one vector is non-zero.\\
Assuming $\myvec{b\\d}$ as&Therefore
$\vec{N}=\myvec{a&0\\c&0}$\\
the zero vector&\\
&\parbox{6cm}{\begin{align}
\mbox{So }\vec{N}^2&=0\\
\implies a^2&=0\\
\therefore a&=0\\
\mbox{Thus }\vec{N}&=\myvec{a&0\\c&0}
\end{align}}\\
&In this case $\vec{N}$ is similar to $\vec{N}=\myvec{0&0\\1&0}$ via the matrix $\vec{P}=\myvec{c&0\\0&1}$\\
&\\
\hline
&\\
Assuming $\myvec{a\\c}$ as&Therefore
$\vec{N}=\myvec{0&b\\0&d}$\\
the zero vector&\\
&\parbox{6cm}{\begin{align}
\mbox{Then }\vec{N}^2&=0\\
\implies d^2&=0\\
\therefore d&=0\\
\mbox{Thus }\vec{N}&=\myvec{0&b\\0&0}
\end{align}}\\
&In this case $\vec{N}$ is similar to $\vec{N}=\myvec{0&0\\b&0}$ via the matrix $\vec{P}=\myvec{0&1\\1&0}$,\\
&which is similar to $\myvec{0&0\\1&0}$ as above.\\
&\\
\hline
&\\
Hence& we can assume neither $\myvec{a\\c}$ or $\myvec{b\\d}$ is the zero vector.\\
&\\
%\hline
%\pagebreak
\hline
&\\
Consequences of &Since they are linearly dependent we can assume,\\
linear&\\
independence&\\
&\parbox{6cm}{\begin{align}
\myvec{b\\d}&=x\myvec{a\\c}\\
\therefore \vec{N}&=\myvec{a&ax\\c&cx}\\
\therefore \vec{N}^2&=0\\
\implies a(a+cx)&=0\\
c(a+cx)&=0\\
ax(a+cx)&=0\\
cx(a+cx)&=0
\end{align}}\\
\hline
&\\
Proof that $\vec{N}$ is&We know that at least one of a or c is not zero.\\
similar over $\mathbb{C}$ to&If a = 0 then c $\neq$ 0, it must be that x = 0.\\
$\myvec{0&0\\1&0}$&So in this case $\vec{N}=\myvec{0&0\\c&0}$ which is similar to $\myvec{0&0\\1&0}$ as before.\\
&\\
&\parbox{6cm}{\begin{align}
\mbox{If } a &\neq0\\
\mbox{then }x &\neq0\\
\mbox{else }a(a+cx)&=0\\
\implies a&=0\\
\mbox{Thus }a+cx&=0\\
\mbox{Hence }\vec{N}&=\myvec{a&ax\\ \frac{-a}{x}&-a}
\end{align}}\\
 &This is similar to $\myvec{a&a\\-a&-a}$ via $\vec{P}=\myvec{\sqrt{x}&0\\0&\frac{1}{\sqrt{x}}}$.\\
 &And $\myvec{a&a\\-a&-a}$ is similar to $\myvec{0&0\\-a&0}$ via $\vec{P}=\myvec{-1&-1\\1&0}$\\
 &And this finally is similar to $\myvec{0&0\\1&0}$ as before.\\
 &\\
\hline
&\\
Conclusion &Thus either $\vec{N}$ = 0 or $\vec{N}$ is similar over $\mathbb{C}$ to $\myvec{0&0\\1&0}$.\\
&\\
\hline
\caption{Solution summary}
\label{eq:solutions/6/2/11/table:1}
\end{longtable}

\item Prove that the interchange of two rows of a matrix can be accomplished by a finite sequence of elementary row operations of the other two types.
\\
\solution
See Table \ref{eq:solutions/6/2/11/table:1}
\onecolumn
\begin{longtable}{|l|l|}
\hline
\multirow{3}{*}{} & \\
Statement&Solution\\
\hline
&\parbox{6cm}{\begin{align}
\mbox{Let }\vec{N}&=\myvec{a&b\\c&d}\\
\mbox{Since }\vec{N}^2&=0
\end{align}}\\
&If $\myvec{a\\c},\myvec{b\\d}$ are linearly independent then $\vec{N}$ is diagonalizable to $\myvec{0&0\\0&0}$.\\
&\parbox{6cm}{\begin{align}
\mbox{If }\vec{P}\vec{N}\vec{P}^{-1}=0\\
\mbox{then }\vec{N}=\vec{P}^{-1}\vec{0}\vec{P}=0
\end{align}}\\
Proof that&So in this case $\vec{N}$ itself is the zero matrix.\\
$\vec{N}=0$&This contradicts the assumption that $\myvec{a\\c},\myvec{b\\d}$ are linearly independent.\\
&$\therefore$ we can assume that $\myvec{a\\c},\myvec{b\\d}$ are linearly dependent if both are\\
&equal to the zero vector\\
&\parbox{6cm}
{\begin{align}
   \mbox{then } \vec{N} &= 0.
\end{align}}\\
%\hline
%\pagebreak
\hline
&\\
&Therefore we can assume at least one vector is non-zero.\\
Assuming $\myvec{b\\d}$ as&Therefore
$\vec{N}=\myvec{a&0\\c&0}$\\
the zero vector&\\
&\parbox{6cm}{\begin{align}
\mbox{So }\vec{N}^2&=0\\
\implies a^2&=0\\
\therefore a&=0\\
\mbox{Thus }\vec{N}&=\myvec{a&0\\c&0}
\end{align}}\\
&In this case $\vec{N}$ is similar to $\vec{N}=\myvec{0&0\\1&0}$ via the matrix $\vec{P}=\myvec{c&0\\0&1}$\\
&\\
\hline
&\\
Assuming $\myvec{a\\c}$ as&Therefore
$\vec{N}=\myvec{0&b\\0&d}$\\
the zero vector&\\
&\parbox{6cm}{\begin{align}
\mbox{Then }\vec{N}^2&=0\\
\implies d^2&=0\\
\therefore d&=0\\
\mbox{Thus }\vec{N}&=\myvec{0&b\\0&0}
\end{align}}\\
&In this case $\vec{N}$ is similar to $\vec{N}=\myvec{0&0\\b&0}$ via the matrix $\vec{P}=\myvec{0&1\\1&0}$,\\
&which is similar to $\myvec{0&0\\1&0}$ as above.\\
&\\
\hline
&\\
Hence& we can assume neither $\myvec{a\\c}$ or $\myvec{b\\d}$ is the zero vector.\\
&\\
%\hline
%\pagebreak
\hline
&\\
Consequences of &Since they are linearly dependent we can assume,\\
linear&\\
independence&\\
&\parbox{6cm}{\begin{align}
\myvec{b\\d}&=x\myvec{a\\c}\\
\therefore \vec{N}&=\myvec{a&ax\\c&cx}\\
\therefore \vec{N}^2&=0\\
\implies a(a+cx)&=0\\
c(a+cx)&=0\\
ax(a+cx)&=0\\
cx(a+cx)&=0
\end{align}}\\
\hline
&\\
Proof that $\vec{N}$ is&We know that at least one of a or c is not zero.\\
similar over $\mathbb{C}$ to&If a = 0 then c $\neq$ 0, it must be that x = 0.\\
$\myvec{0&0\\1&0}$&So in this case $\vec{N}=\myvec{0&0\\c&0}$ which is similar to $\myvec{0&0\\1&0}$ as before.\\
&\\
&\parbox{6cm}{\begin{align}
\mbox{If } a &\neq0\\
\mbox{then }x &\neq0\\
\mbox{else }a(a+cx)&=0\\
\implies a&=0\\
\mbox{Thus }a+cx&=0\\
\mbox{Hence }\vec{N}&=\myvec{a&ax\\ \frac{-a}{x}&-a}
\end{align}}\\
 &This is similar to $\myvec{a&a\\-a&-a}$ via $\vec{P}=\myvec{\sqrt{x}&0\\0&\frac{1}{\sqrt{x}}}$.\\
 &And $\myvec{a&a\\-a&-a}$ is similar to $\myvec{0&0\\-a&0}$ via $\vec{P}=\myvec{-1&-1\\1&0}$\\
 &And this finally is similar to $\myvec{0&0\\1&0}$ as before.\\
 &\\
\hline
&\\
Conclusion &Thus either $\vec{N}$ = 0 or $\vec{N}$ is similar over $\mathbb{C}$ to $\myvec{0&0\\1&0}$.\\
&\\
\hline
\caption{Solution summary}
\label{eq:solutions/6/2/11/table:1}
\end{longtable}

%
\end{enumerate}



%
\subsection{Row Reduced Echelon Matrices}
\renewcommand{\theequation}{\theenumi}
\renewcommand{\thefigure}{\theenumi}
\begin{enumerate}[label=\thesubsection.\arabic*.,ref=\thesubsection.\theenumi]
\numberwithin{equation}{enumi}
\numberwithin{figure}{enumi}
%
\item Find all solutions to the following system of equations by row-reducing the co-efficient matrix:
\begin{align}
\frac{1}{3}x_1 +2x_2 - 6x_3 =0\\
-4x_1\quad \quad+ 5x_3=0\\
-3x_1+6x_2-13x_3=0\\
-\frac{7}{3}x_1 +2x_2 - \frac{8}{3}x_3 =0
\end{align}
\\
\solution
See Table \ref{eq:solutions/6/2/11/table:1}
\onecolumn
\begin{longtable}{|l|l|}
\hline
\multirow{3}{*}{} & \\
Statement&Solution\\
\hline
&\parbox{6cm}{\begin{align}
\mbox{Let }\vec{N}&=\myvec{a&b\\c&d}\\
\mbox{Since }\vec{N}^2&=0
\end{align}}\\
&If $\myvec{a\\c},\myvec{b\\d}$ are linearly independent then $\vec{N}$ is diagonalizable to $\myvec{0&0\\0&0}$.\\
&\parbox{6cm}{\begin{align}
\mbox{If }\vec{P}\vec{N}\vec{P}^{-1}=0\\
\mbox{then }\vec{N}=\vec{P}^{-1}\vec{0}\vec{P}=0
\end{align}}\\
Proof that&So in this case $\vec{N}$ itself is the zero matrix.\\
$\vec{N}=0$&This contradicts the assumption that $\myvec{a\\c},\myvec{b\\d}$ are linearly independent.\\
&$\therefore$ we can assume that $\myvec{a\\c},\myvec{b\\d}$ are linearly dependent if both are\\
&equal to the zero vector\\
&\parbox{6cm}
{\begin{align}
   \mbox{then } \vec{N} &= 0.
\end{align}}\\
%\hline
%\pagebreak
\hline
&\\
&Therefore we can assume at least one vector is non-zero.\\
Assuming $\myvec{b\\d}$ as&Therefore
$\vec{N}=\myvec{a&0\\c&0}$\\
the zero vector&\\
&\parbox{6cm}{\begin{align}
\mbox{So }\vec{N}^2&=0\\
\implies a^2&=0\\
\therefore a&=0\\
\mbox{Thus }\vec{N}&=\myvec{a&0\\c&0}
\end{align}}\\
&In this case $\vec{N}$ is similar to $\vec{N}=\myvec{0&0\\1&0}$ via the matrix $\vec{P}=\myvec{c&0\\0&1}$\\
&\\
\hline
&\\
Assuming $\myvec{a\\c}$ as&Therefore
$\vec{N}=\myvec{0&b\\0&d}$\\
the zero vector&\\
&\parbox{6cm}{\begin{align}
\mbox{Then }\vec{N}^2&=0\\
\implies d^2&=0\\
\therefore d&=0\\
\mbox{Thus }\vec{N}&=\myvec{0&b\\0&0}
\end{align}}\\
&In this case $\vec{N}$ is similar to $\vec{N}=\myvec{0&0\\b&0}$ via the matrix $\vec{P}=\myvec{0&1\\1&0}$,\\
&which is similar to $\myvec{0&0\\1&0}$ as above.\\
&\\
\hline
&\\
Hence& we can assume neither $\myvec{a\\c}$ or $\myvec{b\\d}$ is the zero vector.\\
&\\
%\hline
%\pagebreak
\hline
&\\
Consequences of &Since they are linearly dependent we can assume,\\
linear&\\
independence&\\
&\parbox{6cm}{\begin{align}
\myvec{b\\d}&=x\myvec{a\\c}\\
\therefore \vec{N}&=\myvec{a&ax\\c&cx}\\
\therefore \vec{N}^2&=0\\
\implies a(a+cx)&=0\\
c(a+cx)&=0\\
ax(a+cx)&=0\\
cx(a+cx)&=0
\end{align}}\\
\hline
&\\
Proof that $\vec{N}$ is&We know that at least one of a or c is not zero.\\
similar over $\mathbb{C}$ to&If a = 0 then c $\neq$ 0, it must be that x = 0.\\
$\myvec{0&0\\1&0}$&So in this case $\vec{N}=\myvec{0&0\\c&0}$ which is similar to $\myvec{0&0\\1&0}$ as before.\\
&\\
&\parbox{6cm}{\begin{align}
\mbox{If } a &\neq0\\
\mbox{then }x &\neq0\\
\mbox{else }a(a+cx)&=0\\
\implies a&=0\\
\mbox{Thus }a+cx&=0\\
\mbox{Hence }\vec{N}&=\myvec{a&ax\\ \frac{-a}{x}&-a}
\end{align}}\\
 &This is similar to $\myvec{a&a\\-a&-a}$ via $\vec{P}=\myvec{\sqrt{x}&0\\0&\frac{1}{\sqrt{x}}}$.\\
 &And $\myvec{a&a\\-a&-a}$ is similar to $\myvec{0&0\\-a&0}$ via $\vec{P}=\myvec{-1&-1\\1&0}$\\
 &And this finally is similar to $\myvec{0&0\\1&0}$ as before.\\
 &\\
\hline
&\\
Conclusion &Thus either $\vec{N}$ = 0 or $\vec{N}$ is similar over $\mathbb{C}$ to $\myvec{0&0\\1&0}$.\\
&\\
\hline
\caption{Solution summary}
\label{eq:solutions/6/2/11/table:1}
\end{longtable}

%
\item Let
\begin{align}
    \vec{A}=\myvec{3 & -6 & 2 & -1\\-2 & 4 & 1 & 3\\0 & 0 & 1& 1\\1 & -2 & 1 & 0} 
\end{align}
For which $(y_1,y_2,y_3,y_4)$ does the system of equations $\vec{A}\vec{X}=\vec{Y}$ have a solution ? 
%
\solution
See Table \ref{eq:solutions/6/2/11/table:1}
\onecolumn
\begin{longtable}{|l|l|}
\hline
\multirow{3}{*}{} & \\
Statement&Solution\\
\hline
&\parbox{6cm}{\begin{align}
\mbox{Let }\vec{N}&=\myvec{a&b\\c&d}\\
\mbox{Since }\vec{N}^2&=0
\end{align}}\\
&If $\myvec{a\\c},\myvec{b\\d}$ are linearly independent then $\vec{N}$ is diagonalizable to $\myvec{0&0\\0&0}$.\\
&\parbox{6cm}{\begin{align}
\mbox{If }\vec{P}\vec{N}\vec{P}^{-1}=0\\
\mbox{then }\vec{N}=\vec{P}^{-1}\vec{0}\vec{P}=0
\end{align}}\\
Proof that&So in this case $\vec{N}$ itself is the zero matrix.\\
$\vec{N}=0$&This contradicts the assumption that $\myvec{a\\c},\myvec{b\\d}$ are linearly independent.\\
&$\therefore$ we can assume that $\myvec{a\\c},\myvec{b\\d}$ are linearly dependent if both are\\
&equal to the zero vector\\
&\parbox{6cm}
{\begin{align}
   \mbox{then } \vec{N} &= 0.
\end{align}}\\
%\hline
%\pagebreak
\hline
&\\
&Therefore we can assume at least one vector is non-zero.\\
Assuming $\myvec{b\\d}$ as&Therefore
$\vec{N}=\myvec{a&0\\c&0}$\\
the zero vector&\\
&\parbox{6cm}{\begin{align}
\mbox{So }\vec{N}^2&=0\\
\implies a^2&=0\\
\therefore a&=0\\
\mbox{Thus }\vec{N}&=\myvec{a&0\\c&0}
\end{align}}\\
&In this case $\vec{N}$ is similar to $\vec{N}=\myvec{0&0\\1&0}$ via the matrix $\vec{P}=\myvec{c&0\\0&1}$\\
&\\
\hline
&\\
Assuming $\myvec{a\\c}$ as&Therefore
$\vec{N}=\myvec{0&b\\0&d}$\\
the zero vector&\\
&\parbox{6cm}{\begin{align}
\mbox{Then }\vec{N}^2&=0\\
\implies d^2&=0\\
\therefore d&=0\\
\mbox{Thus }\vec{N}&=\myvec{0&b\\0&0}
\end{align}}\\
&In this case $\vec{N}$ is similar to $\vec{N}=\myvec{0&0\\b&0}$ via the matrix $\vec{P}=\myvec{0&1\\1&0}$,\\
&which is similar to $\myvec{0&0\\1&0}$ as above.\\
&\\
\hline
&\\
Hence& we can assume neither $\myvec{a\\c}$ or $\myvec{b\\d}$ is the zero vector.\\
&\\
%\hline
%\pagebreak
\hline
&\\
Consequences of &Since they are linearly dependent we can assume,\\
linear&\\
independence&\\
&\parbox{6cm}{\begin{align}
\myvec{b\\d}&=x\myvec{a\\c}\\
\therefore \vec{N}&=\myvec{a&ax\\c&cx}\\
\therefore \vec{N}^2&=0\\
\implies a(a+cx)&=0\\
c(a+cx)&=0\\
ax(a+cx)&=0\\
cx(a+cx)&=0
\end{align}}\\
\hline
&\\
Proof that $\vec{N}$ is&We know that at least one of a or c is not zero.\\
similar over $\mathbb{C}$ to&If a = 0 then c $\neq$ 0, it must be that x = 0.\\
$\myvec{0&0\\1&0}$&So in this case $\vec{N}=\myvec{0&0\\c&0}$ which is similar to $\myvec{0&0\\1&0}$ as before.\\
&\\
&\parbox{6cm}{\begin{align}
\mbox{If } a &\neq0\\
\mbox{then }x &\neq0\\
\mbox{else }a(a+cx)&=0\\
\implies a&=0\\
\mbox{Thus }a+cx&=0\\
\mbox{Hence }\vec{N}&=\myvec{a&ax\\ \frac{-a}{x}&-a}
\end{align}}\\
 &This is similar to $\myvec{a&a\\-a&-a}$ via $\vec{P}=\myvec{\sqrt{x}&0\\0&\frac{1}{\sqrt{x}}}$.\\
 &And $\myvec{a&a\\-a&-a}$ is similar to $\myvec{0&0\\-a&0}$ via $\vec{P}=\myvec{-1&-1\\1&0}$\\
 &And this finally is similar to $\myvec{0&0\\1&0}$ as before.\\
 &\\
\hline
&\\
Conclusion &Thus either $\vec{N}$ = 0 or $\vec{N}$ is similar over $\mathbb{C}$ to $\myvec{0&0\\1&0}$.\\
&\\
\hline
\caption{Solution summary}
\label{eq:solutions/6/2/11/table:1}
\end{longtable}

%
%
\item Suppose $\vec{R}$ and $\vec{R}^{'}$ are 2 $\times$ 3 row-reduced echelon matrices and that the system $\vec{R}$$\vec{X}$=0 and $\vec{R}^{'}\vec{X}$=0 have exactly the same solutions. Prove 
that $\vec{R}=\vec{R}^{'}$.

\solution
See Table \ref{eq:solutions/6/2/11/table:1}
\onecolumn
\begin{longtable}{|l|l|}
\hline
\multirow{3}{*}{} & \\
Statement&Solution\\
\hline
&\parbox{6cm}{\begin{align}
\mbox{Let }\vec{N}&=\myvec{a&b\\c&d}\\
\mbox{Since }\vec{N}^2&=0
\end{align}}\\
&If $\myvec{a\\c},\myvec{b\\d}$ are linearly independent then $\vec{N}$ is diagonalizable to $\myvec{0&0\\0&0}$.\\
&\parbox{6cm}{\begin{align}
\mbox{If }\vec{P}\vec{N}\vec{P}^{-1}=0\\
\mbox{then }\vec{N}=\vec{P}^{-1}\vec{0}\vec{P}=0
\end{align}}\\
Proof that&So in this case $\vec{N}$ itself is the zero matrix.\\
$\vec{N}=0$&This contradicts the assumption that $\myvec{a\\c},\myvec{b\\d}$ are linearly independent.\\
&$\therefore$ we can assume that $\myvec{a\\c},\myvec{b\\d}$ are linearly dependent if both are\\
&equal to the zero vector\\
&\parbox{6cm}
{\begin{align}
   \mbox{then } \vec{N} &= 0.
\end{align}}\\
%\hline
%\pagebreak
\hline
&\\
&Therefore we can assume at least one vector is non-zero.\\
Assuming $\myvec{b\\d}$ as&Therefore
$\vec{N}=\myvec{a&0\\c&0}$\\
the zero vector&\\
&\parbox{6cm}{\begin{align}
\mbox{So }\vec{N}^2&=0\\
\implies a^2&=0\\
\therefore a&=0\\
\mbox{Thus }\vec{N}&=\myvec{a&0\\c&0}
\end{align}}\\
&In this case $\vec{N}$ is similar to $\vec{N}=\myvec{0&0\\1&0}$ via the matrix $\vec{P}=\myvec{c&0\\0&1}$\\
&\\
\hline
&\\
Assuming $\myvec{a\\c}$ as&Therefore
$\vec{N}=\myvec{0&b\\0&d}$\\
the zero vector&\\
&\parbox{6cm}{\begin{align}
\mbox{Then }\vec{N}^2&=0\\
\implies d^2&=0\\
\therefore d&=0\\
\mbox{Thus }\vec{N}&=\myvec{0&b\\0&0}
\end{align}}\\
&In this case $\vec{N}$ is similar to $\vec{N}=\myvec{0&0\\b&0}$ via the matrix $\vec{P}=\myvec{0&1\\1&0}$,\\
&which is similar to $\myvec{0&0\\1&0}$ as above.\\
&\\
\hline
&\\
Hence& we can assume neither $\myvec{a\\c}$ or $\myvec{b\\d}$ is the zero vector.\\
&\\
%\hline
%\pagebreak
\hline
&\\
Consequences of &Since they are linearly dependent we can assume,\\
linear&\\
independence&\\
&\parbox{6cm}{\begin{align}
\myvec{b\\d}&=x\myvec{a\\c}\\
\therefore \vec{N}&=\myvec{a&ax\\c&cx}\\
\therefore \vec{N}^2&=0\\
\implies a(a+cx)&=0\\
c(a+cx)&=0\\
ax(a+cx)&=0\\
cx(a+cx)&=0
\end{align}}\\
\hline
&\\
Proof that $\vec{N}$ is&We know that at least one of a or c is not zero.\\
similar over $\mathbb{C}$ to&If a = 0 then c $\neq$ 0, it must be that x = 0.\\
$\myvec{0&0\\1&0}$&So in this case $\vec{N}=\myvec{0&0\\c&0}$ which is similar to $\myvec{0&0\\1&0}$ as before.\\
&\\
&\parbox{6cm}{\begin{align}
\mbox{If } a &\neq0\\
\mbox{then }x &\neq0\\
\mbox{else }a(a+cx)&=0\\
\implies a&=0\\
\mbox{Thus }a+cx&=0\\
\mbox{Hence }\vec{N}&=\myvec{a&ax\\ \frac{-a}{x}&-a}
\end{align}}\\
 &This is similar to $\myvec{a&a\\-a&-a}$ via $\vec{P}=\myvec{\sqrt{x}&0\\0&\frac{1}{\sqrt{x}}}$.\\
 &And $\myvec{a&a\\-a&-a}$ is similar to $\myvec{0&0\\-a&0}$ via $\vec{P}=\myvec{-1&-1\\1&0}$\\
 &And this finally is similar to $\myvec{0&0\\1&0}$ as before.\\
 &\\
\hline
&\\
Conclusion &Thus either $\vec{N}$ = 0 or $\vec{N}$ is similar over $\mathbb{C}$ to $\myvec{0&0\\1&0}$.\\
&\\
\hline
\caption{Solution summary}
\label{eq:solutions/6/2/11/table:1}
\end{longtable}

\end{enumerate}



\subsection{Matrix Multiplication}
\renewcommand{\theequation}{\theenumi}
\renewcommand{\thefigure}{\theenumi}
\begin{enumerate}[label=\thesubsection.\arabic*.,ref=\thesubsection.\theenumi]
\numberwithin{equation}{enumi}
\numberwithin{figure}{enumi}
%
\item Let
\begin{align}
   A=\myvec{1&-1&1\\2&0&1\\3&0&1}, B=\myvec{2&-2\\1&3\\4&4}
\end{align}
Verify directly that $ A(AB)=A^2B $
%
\solution
See Table \ref{eq:solutions/6/2/11/table:1}
\onecolumn
\begin{longtable}{|l|l|}
\hline
\multirow{3}{*}{} & \\
Statement&Solution\\
\hline
&\parbox{6cm}{\begin{align}
\mbox{Let }\vec{N}&=\myvec{a&b\\c&d}\\
\mbox{Since }\vec{N}^2&=0
\end{align}}\\
&If $\myvec{a\\c},\myvec{b\\d}$ are linearly independent then $\vec{N}$ is diagonalizable to $\myvec{0&0\\0&0}$.\\
&\parbox{6cm}{\begin{align}
\mbox{If }\vec{P}\vec{N}\vec{P}^{-1}=0\\
\mbox{then }\vec{N}=\vec{P}^{-1}\vec{0}\vec{P}=0
\end{align}}\\
Proof that&So in this case $\vec{N}$ itself is the zero matrix.\\
$\vec{N}=0$&This contradicts the assumption that $\myvec{a\\c},\myvec{b\\d}$ are linearly independent.\\
&$\therefore$ we can assume that $\myvec{a\\c},\myvec{b\\d}$ are linearly dependent if both are\\
&equal to the zero vector\\
&\parbox{6cm}
{\begin{align}
   \mbox{then } \vec{N} &= 0.
\end{align}}\\
%\hline
%\pagebreak
\hline
&\\
&Therefore we can assume at least one vector is non-zero.\\
Assuming $\myvec{b\\d}$ as&Therefore
$\vec{N}=\myvec{a&0\\c&0}$\\
the zero vector&\\
&\parbox{6cm}{\begin{align}
\mbox{So }\vec{N}^2&=0\\
\implies a^2&=0\\
\therefore a&=0\\
\mbox{Thus }\vec{N}&=\myvec{a&0\\c&0}
\end{align}}\\
&In this case $\vec{N}$ is similar to $\vec{N}=\myvec{0&0\\1&0}$ via the matrix $\vec{P}=\myvec{c&0\\0&1}$\\
&\\
\hline
&\\
Assuming $\myvec{a\\c}$ as&Therefore
$\vec{N}=\myvec{0&b\\0&d}$\\
the zero vector&\\
&\parbox{6cm}{\begin{align}
\mbox{Then }\vec{N}^2&=0\\
\implies d^2&=0\\
\therefore d&=0\\
\mbox{Thus }\vec{N}&=\myvec{0&b\\0&0}
\end{align}}\\
&In this case $\vec{N}$ is similar to $\vec{N}=\myvec{0&0\\b&0}$ via the matrix $\vec{P}=\myvec{0&1\\1&0}$,\\
&which is similar to $\myvec{0&0\\1&0}$ as above.\\
&\\
\hline
&\\
Hence& we can assume neither $\myvec{a\\c}$ or $\myvec{b\\d}$ is the zero vector.\\
&\\
%\hline
%\pagebreak
\hline
&\\
Consequences of &Since they are linearly dependent we can assume,\\
linear&\\
independence&\\
&\parbox{6cm}{\begin{align}
\myvec{b\\d}&=x\myvec{a\\c}\\
\therefore \vec{N}&=\myvec{a&ax\\c&cx}\\
\therefore \vec{N}^2&=0\\
\implies a(a+cx)&=0\\
c(a+cx)&=0\\
ax(a+cx)&=0\\
cx(a+cx)&=0
\end{align}}\\
\hline
&\\
Proof that $\vec{N}$ is&We know that at least one of a or c is not zero.\\
similar over $\mathbb{C}$ to&If a = 0 then c $\neq$ 0, it must be that x = 0.\\
$\myvec{0&0\\1&0}$&So in this case $\vec{N}=\myvec{0&0\\c&0}$ which is similar to $\myvec{0&0\\1&0}$ as before.\\
&\\
&\parbox{6cm}{\begin{align}
\mbox{If } a &\neq0\\
\mbox{then }x &\neq0\\
\mbox{else }a(a+cx)&=0\\
\implies a&=0\\
\mbox{Thus }a+cx&=0\\
\mbox{Hence }\vec{N}&=\myvec{a&ax\\ \frac{-a}{x}&-a}
\end{align}}\\
 &This is similar to $\myvec{a&a\\-a&-a}$ via $\vec{P}=\myvec{\sqrt{x}&0\\0&\frac{1}{\sqrt{x}}}$.\\
 &And $\myvec{a&a\\-a&-a}$ is similar to $\myvec{0&0\\-a&0}$ via $\vec{P}=\myvec{-1&-1\\1&0}$\\
 &And this finally is similar to $\myvec{0&0\\1&0}$ as before.\\
 &\\
\hline
&\\
Conclusion &Thus either $\vec{N}$ = 0 or $\vec{N}$ is similar over $\mathbb{C}$ to $\myvec{0&0\\1&0}$.\\
&\\
\hline
\caption{Solution summary}
\label{eq:solutions/6/2/11/table:1}
\end{longtable}

%

%
\item Find two different 2$\times$2 matrices $\vec{A}$ such that 

$\vec{A}^{2}=0$ but $\vec{A}\ne0$
%
\\
\solution
See Table \ref{eq:solutions/6/2/11/table:1}
\onecolumn
\begin{longtable}{|l|l|}
\hline
\multirow{3}{*}{} & \\
Statement&Solution\\
\hline
&\parbox{6cm}{\begin{align}
\mbox{Let }\vec{N}&=\myvec{a&b\\c&d}\\
\mbox{Since }\vec{N}^2&=0
\end{align}}\\
&If $\myvec{a\\c},\myvec{b\\d}$ are linearly independent then $\vec{N}$ is diagonalizable to $\myvec{0&0\\0&0}$.\\
&\parbox{6cm}{\begin{align}
\mbox{If }\vec{P}\vec{N}\vec{P}^{-1}=0\\
\mbox{then }\vec{N}=\vec{P}^{-1}\vec{0}\vec{P}=0
\end{align}}\\
Proof that&So in this case $\vec{N}$ itself is the zero matrix.\\
$\vec{N}=0$&This contradicts the assumption that $\myvec{a\\c},\myvec{b\\d}$ are linearly independent.\\
&$\therefore$ we can assume that $\myvec{a\\c},\myvec{b\\d}$ are linearly dependent if both are\\
&equal to the zero vector\\
&\parbox{6cm}
{\begin{align}
   \mbox{then } \vec{N} &= 0.
\end{align}}\\
%\hline
%\pagebreak
\hline
&\\
&Therefore we can assume at least one vector is non-zero.\\
Assuming $\myvec{b\\d}$ as&Therefore
$\vec{N}=\myvec{a&0\\c&0}$\\
the zero vector&\\
&\parbox{6cm}{\begin{align}
\mbox{So }\vec{N}^2&=0\\
\implies a^2&=0\\
\therefore a&=0\\
\mbox{Thus }\vec{N}&=\myvec{a&0\\c&0}
\end{align}}\\
&In this case $\vec{N}$ is similar to $\vec{N}=\myvec{0&0\\1&0}$ via the matrix $\vec{P}=\myvec{c&0\\0&1}$\\
&\\
\hline
&\\
Assuming $\myvec{a\\c}$ as&Therefore
$\vec{N}=\myvec{0&b\\0&d}$\\
the zero vector&\\
&\parbox{6cm}{\begin{align}
\mbox{Then }\vec{N}^2&=0\\
\implies d^2&=0\\
\therefore d&=0\\
\mbox{Thus }\vec{N}&=\myvec{0&b\\0&0}
\end{align}}\\
&In this case $\vec{N}$ is similar to $\vec{N}=\myvec{0&0\\b&0}$ via the matrix $\vec{P}=\myvec{0&1\\1&0}$,\\
&which is similar to $\myvec{0&0\\1&0}$ as above.\\
&\\
\hline
&\\
Hence& we can assume neither $\myvec{a\\c}$ or $\myvec{b\\d}$ is the zero vector.\\
&\\
%\hline
%\pagebreak
\hline
&\\
Consequences of &Since they are linearly dependent we can assume,\\
linear&\\
independence&\\
&\parbox{6cm}{\begin{align}
\myvec{b\\d}&=x\myvec{a\\c}\\
\therefore \vec{N}&=\myvec{a&ax\\c&cx}\\
\therefore \vec{N}^2&=0\\
\implies a(a+cx)&=0\\
c(a+cx)&=0\\
ax(a+cx)&=0\\
cx(a+cx)&=0
\end{align}}\\
\hline
&\\
Proof that $\vec{N}$ is&We know that at least one of a or c is not zero.\\
similar over $\mathbb{C}$ to&If a = 0 then c $\neq$ 0, it must be that x = 0.\\
$\myvec{0&0\\1&0}$&So in this case $\vec{N}=\myvec{0&0\\c&0}$ which is similar to $\myvec{0&0\\1&0}$ as before.\\
&\\
&\parbox{6cm}{\begin{align}
\mbox{If } a &\neq0\\
\mbox{then }x &\neq0\\
\mbox{else }a(a+cx)&=0\\
\implies a&=0\\
\mbox{Thus }a+cx&=0\\
\mbox{Hence }\vec{N}&=\myvec{a&ax\\ \frac{-a}{x}&-a}
\end{align}}\\
 &This is similar to $\myvec{a&a\\-a&-a}$ via $\vec{P}=\myvec{\sqrt{x}&0\\0&\frac{1}{\sqrt{x}}}$.\\
 &And $\myvec{a&a\\-a&-a}$ is similar to $\myvec{0&0\\-a&0}$ via $\vec{P}=\myvec{-1&-1\\1&0}$\\
 &And this finally is similar to $\myvec{0&0\\1&0}$ as before.\\
 &\\
\hline
&\\
Conclusion &Thus either $\vec{N}$ = 0 or $\vec{N}$ is similar over $\mathbb{C}$ to $\myvec{0&0\\1&0}$.\\
&\\
\hline
\caption{Solution summary}
\label{eq:solutions/6/2/11/table:1}
\end{longtable}


%
\item Let $\vec{A}$ be an $m \times n$ matrix and $\vec{B}$ be an $n \times k$ matrix.Show that the columns of $\vec{C}=\vec{A}\vec{B}$ are linear combinations of columns of $\vec{A}$.If $\vec{\alpha_1},\vec{\alpha_2},\hdots,\vec{\alpha_n}$ are the columns of $\vec{A}$ and $\vec{\gamma_1},\vec{\gamma_2},\hdots,\vec{\gamma_k}$ are the columns of $\vec{C}$ then,
\begin{align}
    \vec{\gamma_j}= \sum_{r=1}^{n}B_{rj}\vec{\alpha_r} 
\end{align}
%
\solution
See Table \ref{eq:solutions/6/2/11/table:1}
\onecolumn
\begin{longtable}{|l|l|}
\hline
\multirow{3}{*}{} & \\
Statement&Solution\\
\hline
&\parbox{6cm}{\begin{align}
\mbox{Let }\vec{N}&=\myvec{a&b\\c&d}\\
\mbox{Since }\vec{N}^2&=0
\end{align}}\\
&If $\myvec{a\\c},\myvec{b\\d}$ are linearly independent then $\vec{N}$ is diagonalizable to $\myvec{0&0\\0&0}$.\\
&\parbox{6cm}{\begin{align}
\mbox{If }\vec{P}\vec{N}\vec{P}^{-1}=0\\
\mbox{then }\vec{N}=\vec{P}^{-1}\vec{0}\vec{P}=0
\end{align}}\\
Proof that&So in this case $\vec{N}$ itself is the zero matrix.\\
$\vec{N}=0$&This contradicts the assumption that $\myvec{a\\c},\myvec{b\\d}$ are linearly independent.\\
&$\therefore$ we can assume that $\myvec{a\\c},\myvec{b\\d}$ are linearly dependent if both are\\
&equal to the zero vector\\
&\parbox{6cm}
{\begin{align}
   \mbox{then } \vec{N} &= 0.
\end{align}}\\
%\hline
%\pagebreak
\hline
&\\
&Therefore we can assume at least one vector is non-zero.\\
Assuming $\myvec{b\\d}$ as&Therefore
$\vec{N}=\myvec{a&0\\c&0}$\\
the zero vector&\\
&\parbox{6cm}{\begin{align}
\mbox{So }\vec{N}^2&=0\\
\implies a^2&=0\\
\therefore a&=0\\
\mbox{Thus }\vec{N}&=\myvec{a&0\\c&0}
\end{align}}\\
&In this case $\vec{N}$ is similar to $\vec{N}=\myvec{0&0\\1&0}$ via the matrix $\vec{P}=\myvec{c&0\\0&1}$\\
&\\
\hline
&\\
Assuming $\myvec{a\\c}$ as&Therefore
$\vec{N}=\myvec{0&b\\0&d}$\\
the zero vector&\\
&\parbox{6cm}{\begin{align}
\mbox{Then }\vec{N}^2&=0\\
\implies d^2&=0\\
\therefore d&=0\\
\mbox{Thus }\vec{N}&=\myvec{0&b\\0&0}
\end{align}}\\
&In this case $\vec{N}$ is similar to $\vec{N}=\myvec{0&0\\b&0}$ via the matrix $\vec{P}=\myvec{0&1\\1&0}$,\\
&which is similar to $\myvec{0&0\\1&0}$ as above.\\
&\\
\hline
&\\
Hence& we can assume neither $\myvec{a\\c}$ or $\myvec{b\\d}$ is the zero vector.\\
&\\
%\hline
%\pagebreak
\hline
&\\
Consequences of &Since they are linearly dependent we can assume,\\
linear&\\
independence&\\
&\parbox{6cm}{\begin{align}
\myvec{b\\d}&=x\myvec{a\\c}\\
\therefore \vec{N}&=\myvec{a&ax\\c&cx}\\
\therefore \vec{N}^2&=0\\
\implies a(a+cx)&=0\\
c(a+cx)&=0\\
ax(a+cx)&=0\\
cx(a+cx)&=0
\end{align}}\\
\hline
&\\
Proof that $\vec{N}$ is&We know that at least one of a or c is not zero.\\
similar over $\mathbb{C}$ to&If a = 0 then c $\neq$ 0, it must be that x = 0.\\
$\myvec{0&0\\1&0}$&So in this case $\vec{N}=\myvec{0&0\\c&0}$ which is similar to $\myvec{0&0\\1&0}$ as before.\\
&\\
&\parbox{6cm}{\begin{align}
\mbox{If } a &\neq0\\
\mbox{then }x &\neq0\\
\mbox{else }a(a+cx)&=0\\
\implies a&=0\\
\mbox{Thus }a+cx&=0\\
\mbox{Hence }\vec{N}&=\myvec{a&ax\\ \frac{-a}{x}&-a}
\end{align}}\\
 &This is similar to $\myvec{a&a\\-a&-a}$ via $\vec{P}=\myvec{\sqrt{x}&0\\0&\frac{1}{\sqrt{x}}}$.\\
 &And $\myvec{a&a\\-a&-a}$ is similar to $\myvec{0&0\\-a&0}$ via $\vec{P}=\myvec{-1&-1\\1&0}$\\
 &And this finally is similar to $\myvec{0&0\\1&0}$ as before.\\
 &\\
\hline
&\\
Conclusion &Thus either $\vec{N}$ = 0 or $\vec{N}$ is similar over $\mathbb{C}$ to $\myvec{0&0\\1&0}$.\\
&\\
\hline
\caption{Solution summary}
\label{eq:solutions/6/2/11/table:1}
\end{longtable}

%
\item Let $\vec{A}$ and $\vec{B}$ be $n \times n$ matrices such that $\vec{AB}=\vec{I}$. Prove that $\vec{BA}=\vec{I}$.
%
\solution
See Table \ref{eq:solutions/6/2/11/table:1}
\onecolumn
\begin{longtable}{|l|l|}
\hline
\multirow{3}{*}{} & \\
Statement&Solution\\
\hline
&\parbox{6cm}{\begin{align}
\mbox{Let }\vec{N}&=\myvec{a&b\\c&d}\\
\mbox{Since }\vec{N}^2&=0
\end{align}}\\
&If $\myvec{a\\c},\myvec{b\\d}$ are linearly independent then $\vec{N}$ is diagonalizable to $\myvec{0&0\\0&0}$.\\
&\parbox{6cm}{\begin{align}
\mbox{If }\vec{P}\vec{N}\vec{P}^{-1}=0\\
\mbox{then }\vec{N}=\vec{P}^{-1}\vec{0}\vec{P}=0
\end{align}}\\
Proof that&So in this case $\vec{N}$ itself is the zero matrix.\\
$\vec{N}=0$&This contradicts the assumption that $\myvec{a\\c},\myvec{b\\d}$ are linearly independent.\\
&$\therefore$ we can assume that $\myvec{a\\c},\myvec{b\\d}$ are linearly dependent if both are\\
&equal to the zero vector\\
&\parbox{6cm}
{\begin{align}
   \mbox{then } \vec{N} &= 0.
\end{align}}\\
%\hline
%\pagebreak
\hline
&\\
&Therefore we can assume at least one vector is non-zero.\\
Assuming $\myvec{b\\d}$ as&Therefore
$\vec{N}=\myvec{a&0\\c&0}$\\
the zero vector&\\
&\parbox{6cm}{\begin{align}
\mbox{So }\vec{N}^2&=0\\
\implies a^2&=0\\
\therefore a&=0\\
\mbox{Thus }\vec{N}&=\myvec{a&0\\c&0}
\end{align}}\\
&In this case $\vec{N}$ is similar to $\vec{N}=\myvec{0&0\\1&0}$ via the matrix $\vec{P}=\myvec{c&0\\0&1}$\\
&\\
\hline
&\\
Assuming $\myvec{a\\c}$ as&Therefore
$\vec{N}=\myvec{0&b\\0&d}$\\
the zero vector&\\
&\parbox{6cm}{\begin{align}
\mbox{Then }\vec{N}^2&=0\\
\implies d^2&=0\\
\therefore d&=0\\
\mbox{Thus }\vec{N}&=\myvec{0&b\\0&0}
\end{align}}\\
&In this case $\vec{N}$ is similar to $\vec{N}=\myvec{0&0\\b&0}$ via the matrix $\vec{P}=\myvec{0&1\\1&0}$,\\
&which is similar to $\myvec{0&0\\1&0}$ as above.\\
&\\
\hline
&\\
Hence& we can assume neither $\myvec{a\\c}$ or $\myvec{b\\d}$ is the zero vector.\\
&\\
%\hline
%\pagebreak
\hline
&\\
Consequences of &Since they are linearly dependent we can assume,\\
linear&\\
independence&\\
&\parbox{6cm}{\begin{align}
\myvec{b\\d}&=x\myvec{a\\c}\\
\therefore \vec{N}&=\myvec{a&ax\\c&cx}\\
\therefore \vec{N}^2&=0\\
\implies a(a+cx)&=0\\
c(a+cx)&=0\\
ax(a+cx)&=0\\
cx(a+cx)&=0
\end{align}}\\
\hline
&\\
Proof that $\vec{N}$ is&We know that at least one of a or c is not zero.\\
similar over $\mathbb{C}$ to&If a = 0 then c $\neq$ 0, it must be that x = 0.\\
$\myvec{0&0\\1&0}$&So in this case $\vec{N}=\myvec{0&0\\c&0}$ which is similar to $\myvec{0&0\\1&0}$ as before.\\
&\\
&\parbox{6cm}{\begin{align}
\mbox{If } a &\neq0\\
\mbox{then }x &\neq0\\
\mbox{else }a(a+cx)&=0\\
\implies a&=0\\
\mbox{Thus }a+cx&=0\\
\mbox{Hence }\vec{N}&=\myvec{a&ax\\ \frac{-a}{x}&-a}
\end{align}}\\
 &This is similar to $\myvec{a&a\\-a&-a}$ via $\vec{P}=\myvec{\sqrt{x}&0\\0&\frac{1}{\sqrt{x}}}$.\\
 &And $\myvec{a&a\\-a&-a}$ is similar to $\myvec{0&0\\-a&0}$ via $\vec{P}=\myvec{-1&-1\\1&0}$\\
 &And this finally is similar to $\myvec{0&0\\1&0}$ as before.\\
 &\\
\hline
&\\
Conclusion &Thus either $\vec{N}$ = 0 or $\vec{N}$ is similar over $\mathbb{C}$ to $\myvec{0&0\\1&0}$.\\
&\\
\hline
\caption{Solution summary}
\label{eq:solutions/6/2/11/table:1}
\end{longtable}

%
\item Let,
\begin{align}
\vec{C}=\myvec{C_{11}&C_{12}\\C_{21}&C_{22}}
\end{align}
be a 2$\times$2 matrix. We inquire when it is possible to find 2$\times$2 matrices $\vec{A}$ and $\vec{B}$ such that $\vec{C}$=$\vec{AB-BA}$. Prove that such matrices can be found if and only if $C_{11}+C_{22}$=0.
%
\solution
See Table \ref{eq:solutions/6/2/11/table:1}
\onecolumn
\begin{longtable}{|l|l|}
\hline
\multirow{3}{*}{} & \\
Statement&Solution\\
\hline
&\parbox{6cm}{\begin{align}
\mbox{Let }\vec{N}&=\myvec{a&b\\c&d}\\
\mbox{Since }\vec{N}^2&=0
\end{align}}\\
&If $\myvec{a\\c},\myvec{b\\d}$ are linearly independent then $\vec{N}$ is diagonalizable to $\myvec{0&0\\0&0}$.\\
&\parbox{6cm}{\begin{align}
\mbox{If }\vec{P}\vec{N}\vec{P}^{-1}=0\\
\mbox{then }\vec{N}=\vec{P}^{-1}\vec{0}\vec{P}=0
\end{align}}\\
Proof that&So in this case $\vec{N}$ itself is the zero matrix.\\
$\vec{N}=0$&This contradicts the assumption that $\myvec{a\\c},\myvec{b\\d}$ are linearly independent.\\
&$\therefore$ we can assume that $\myvec{a\\c},\myvec{b\\d}$ are linearly dependent if both are\\
&equal to the zero vector\\
&\parbox{6cm}
{\begin{align}
   \mbox{then } \vec{N} &= 0.
\end{align}}\\
%\hline
%\pagebreak
\hline
&\\
&Therefore we can assume at least one vector is non-zero.\\
Assuming $\myvec{b\\d}$ as&Therefore
$\vec{N}=\myvec{a&0\\c&0}$\\
the zero vector&\\
&\parbox{6cm}{\begin{align}
\mbox{So }\vec{N}^2&=0\\
\implies a^2&=0\\
\therefore a&=0\\
\mbox{Thus }\vec{N}&=\myvec{a&0\\c&0}
\end{align}}\\
&In this case $\vec{N}$ is similar to $\vec{N}=\myvec{0&0\\1&0}$ via the matrix $\vec{P}=\myvec{c&0\\0&1}$\\
&\\
\hline
&\\
Assuming $\myvec{a\\c}$ as&Therefore
$\vec{N}=\myvec{0&b\\0&d}$\\
the zero vector&\\
&\parbox{6cm}{\begin{align}
\mbox{Then }\vec{N}^2&=0\\
\implies d^2&=0\\
\therefore d&=0\\
\mbox{Thus }\vec{N}&=\myvec{0&b\\0&0}
\end{align}}\\
&In this case $\vec{N}$ is similar to $\vec{N}=\myvec{0&0\\b&0}$ via the matrix $\vec{P}=\myvec{0&1\\1&0}$,\\
&which is similar to $\myvec{0&0\\1&0}$ as above.\\
&\\
\hline
&\\
Hence& we can assume neither $\myvec{a\\c}$ or $\myvec{b\\d}$ is the zero vector.\\
&\\
%\hline
%\pagebreak
\hline
&\\
Consequences of &Since they are linearly dependent we can assume,\\
linear&\\
independence&\\
&\parbox{6cm}{\begin{align}
\myvec{b\\d}&=x\myvec{a\\c}\\
\therefore \vec{N}&=\myvec{a&ax\\c&cx}\\
\therefore \vec{N}^2&=0\\
\implies a(a+cx)&=0\\
c(a+cx)&=0\\
ax(a+cx)&=0\\
cx(a+cx)&=0
\end{align}}\\
\hline
&\\
Proof that $\vec{N}$ is&We know that at least one of a or c is not zero.\\
similar over $\mathbb{C}$ to&If a = 0 then c $\neq$ 0, it must be that x = 0.\\
$\myvec{0&0\\1&0}$&So in this case $\vec{N}=\myvec{0&0\\c&0}$ which is similar to $\myvec{0&0\\1&0}$ as before.\\
&\\
&\parbox{6cm}{\begin{align}
\mbox{If } a &\neq0\\
\mbox{then }x &\neq0\\
\mbox{else }a(a+cx)&=0\\
\implies a&=0\\
\mbox{Thus }a+cx&=0\\
\mbox{Hence }\vec{N}&=\myvec{a&ax\\ \frac{-a}{x}&-a}
\end{align}}\\
 &This is similar to $\myvec{a&a\\-a&-a}$ via $\vec{P}=\myvec{\sqrt{x}&0\\0&\frac{1}{\sqrt{x}}}$.\\
 &And $\myvec{a&a\\-a&-a}$ is similar to $\myvec{0&0\\-a&0}$ via $\vec{P}=\myvec{-1&-1\\1&0}$\\
 &And this finally is similar to $\myvec{0&0\\1&0}$ as before.\\
 &\\
\hline
&\\
Conclusion &Thus either $\vec{N}$ = 0 or $\vec{N}$ is similar over $\mathbb{C}$ to $\myvec{0&0\\1&0}$.\\
&\\
\hline
\caption{Solution summary}
\label{eq:solutions/6/2/11/table:1}
\end{longtable}


%\item Discover whether
%\begin{align}
%	\vec{A}=\myvec{ 1 & 2 & 3 & 4 \\
%			0 & 2 & 3 & 4 \\
%			0 & 0 & 3 & 4 \\
%			0 & 0 & 0 & 4}
%\end{align}
%is invertible, and find $\vec{A^{-1}}$ if it exists.
%\\
%%
%\solution
%See Table \ref{eq:solutions/6/2/11/table:1}
\onecolumn
\begin{longtable}{|l|l|}
\hline
\multirow{3}{*}{} & \\
Statement&Solution\\
\hline
&\parbox{6cm}{\begin{align}
\mbox{Let }\vec{N}&=\myvec{a&b\\c&d}\\
\mbox{Since }\vec{N}^2&=0
\end{align}}\\
&If $\myvec{a\\c},\myvec{b\\d}$ are linearly independent then $\vec{N}$ is diagonalizable to $\myvec{0&0\\0&0}$.\\
&\parbox{6cm}{\begin{align}
\mbox{If }\vec{P}\vec{N}\vec{P}^{-1}=0\\
\mbox{then }\vec{N}=\vec{P}^{-1}\vec{0}\vec{P}=0
\end{align}}\\
Proof that&So in this case $\vec{N}$ itself is the zero matrix.\\
$\vec{N}=0$&This contradicts the assumption that $\myvec{a\\c},\myvec{b\\d}$ are linearly independent.\\
&$\therefore$ we can assume that $\myvec{a\\c},\myvec{b\\d}$ are linearly dependent if both are\\
&equal to the zero vector\\
&\parbox{6cm}
{\begin{align}
   \mbox{then } \vec{N} &= 0.
\end{align}}\\
%\hline
%\pagebreak
\hline
&\\
&Therefore we can assume at least one vector is non-zero.\\
Assuming $\myvec{b\\d}$ as&Therefore
$\vec{N}=\myvec{a&0\\c&0}$\\
the zero vector&\\
&\parbox{6cm}{\begin{align}
\mbox{So }\vec{N}^2&=0\\
\implies a^2&=0\\
\therefore a&=0\\
\mbox{Thus }\vec{N}&=\myvec{a&0\\c&0}
\end{align}}\\
&In this case $\vec{N}$ is similar to $\vec{N}=\myvec{0&0\\1&0}$ via the matrix $\vec{P}=\myvec{c&0\\0&1}$\\
&\\
\hline
&\\
Assuming $\myvec{a\\c}$ as&Therefore
$\vec{N}=\myvec{0&b\\0&d}$\\
the zero vector&\\
&\parbox{6cm}{\begin{align}
\mbox{Then }\vec{N}^2&=0\\
\implies d^2&=0\\
\therefore d&=0\\
\mbox{Thus }\vec{N}&=\myvec{0&b\\0&0}
\end{align}}\\
&In this case $\vec{N}$ is similar to $\vec{N}=\myvec{0&0\\b&0}$ via the matrix $\vec{P}=\myvec{0&1\\1&0}$,\\
&which is similar to $\myvec{0&0\\1&0}$ as above.\\
&\\
\hline
&\\
Hence& we can assume neither $\myvec{a\\c}$ or $\myvec{b\\d}$ is the zero vector.\\
&\\
%\hline
%\pagebreak
\hline
&\\
Consequences of &Since they are linearly dependent we can assume,\\
linear&\\
independence&\\
&\parbox{6cm}{\begin{align}
\myvec{b\\d}&=x\myvec{a\\c}\\
\therefore \vec{N}&=\myvec{a&ax\\c&cx}\\
\therefore \vec{N}^2&=0\\
\implies a(a+cx)&=0\\
c(a+cx)&=0\\
ax(a+cx)&=0\\
cx(a+cx)&=0
\end{align}}\\
\hline
&\\
Proof that $\vec{N}$ is&We know that at least one of a or c is not zero.\\
similar over $\mathbb{C}$ to&If a = 0 then c $\neq$ 0, it must be that x = 0.\\
$\myvec{0&0\\1&0}$&So in this case $\vec{N}=\myvec{0&0\\c&0}$ which is similar to $\myvec{0&0\\1&0}$ as before.\\
&\\
&\parbox{6cm}{\begin{align}
\mbox{If } a &\neq0\\
\mbox{then }x &\neq0\\
\mbox{else }a(a+cx)&=0\\
\implies a&=0\\
\mbox{Thus }a+cx&=0\\
\mbox{Hence }\vec{N}&=\myvec{a&ax\\ \frac{-a}{x}&-a}
\end{align}}\\
 &This is similar to $\myvec{a&a\\-a&-a}$ via $\vec{P}=\myvec{\sqrt{x}&0\\0&\frac{1}{\sqrt{x}}}$.\\
 &And $\myvec{a&a\\-a&-a}$ is similar to $\myvec{0&0\\-a&0}$ via $\vec{P}=\myvec{-1&-1\\1&0}$\\
 &And this finally is similar to $\myvec{0&0\\1&0}$ as before.\\
 &\\
\hline
&\\
Conclusion &Thus either $\vec{N}$ = 0 or $\vec{N}$ is similar over $\mathbb{C}$ to $\myvec{0&0\\1&0}$.\\
&\\
\hline
\caption{Solution summary}
\label{eq:solutions/6/2/11/table:1}
\end{longtable}

%
%\item Suppose $\vec{A}$ is a 2$\times$1  matrix and $\vec{B}$ is 1$\times$2 matrix.Prove that $\vec{C}$=$\vec{A}$$\vec{B}$ is non invertible.
%
%\solution
%See Table \ref{eq:solutions/6/2/11/table:1}
\onecolumn
\begin{longtable}{|l|l|}
\hline
\multirow{3}{*}{} & \\
Statement&Solution\\
\hline
&\parbox{6cm}{\begin{align}
\mbox{Let }\vec{N}&=\myvec{a&b\\c&d}\\
\mbox{Since }\vec{N}^2&=0
\end{align}}\\
&If $\myvec{a\\c},\myvec{b\\d}$ are linearly independent then $\vec{N}$ is diagonalizable to $\myvec{0&0\\0&0}$.\\
&\parbox{6cm}{\begin{align}
\mbox{If }\vec{P}\vec{N}\vec{P}^{-1}=0\\
\mbox{then }\vec{N}=\vec{P}^{-1}\vec{0}\vec{P}=0
\end{align}}\\
Proof that&So in this case $\vec{N}$ itself is the zero matrix.\\
$\vec{N}=0$&This contradicts the assumption that $\myvec{a\\c},\myvec{b\\d}$ are linearly independent.\\
&$\therefore$ we can assume that $\myvec{a\\c},\myvec{b\\d}$ are linearly dependent if both are\\
&equal to the zero vector\\
&\parbox{6cm}
{\begin{align}
   \mbox{then } \vec{N} &= 0.
\end{align}}\\
%\hline
%\pagebreak
\hline
&\\
&Therefore we can assume at least one vector is non-zero.\\
Assuming $\myvec{b\\d}$ as&Therefore
$\vec{N}=\myvec{a&0\\c&0}$\\
the zero vector&\\
&\parbox{6cm}{\begin{align}
\mbox{So }\vec{N}^2&=0\\
\implies a^2&=0\\
\therefore a&=0\\
\mbox{Thus }\vec{N}&=\myvec{a&0\\c&0}
\end{align}}\\
&In this case $\vec{N}$ is similar to $\vec{N}=\myvec{0&0\\1&0}$ via the matrix $\vec{P}=\myvec{c&0\\0&1}$\\
&\\
\hline
&\\
Assuming $\myvec{a\\c}$ as&Therefore
$\vec{N}=\myvec{0&b\\0&d}$\\
the zero vector&\\
&\parbox{6cm}{\begin{align}
\mbox{Then }\vec{N}^2&=0\\
\implies d^2&=0\\
\therefore d&=0\\
\mbox{Thus }\vec{N}&=\myvec{0&b\\0&0}
\end{align}}\\
&In this case $\vec{N}$ is similar to $\vec{N}=\myvec{0&0\\b&0}$ via the matrix $\vec{P}=\myvec{0&1\\1&0}$,\\
&which is similar to $\myvec{0&0\\1&0}$ as above.\\
&\\
\hline
&\\
Hence& we can assume neither $\myvec{a\\c}$ or $\myvec{b\\d}$ is the zero vector.\\
&\\
%\hline
%\pagebreak
\hline
&\\
Consequences of &Since they are linearly dependent we can assume,\\
linear&\\
independence&\\
&\parbox{6cm}{\begin{align}
\myvec{b\\d}&=x\myvec{a\\c}\\
\therefore \vec{N}&=\myvec{a&ax\\c&cx}\\
\therefore \vec{N}^2&=0\\
\implies a(a+cx)&=0\\
c(a+cx)&=0\\
ax(a+cx)&=0\\
cx(a+cx)&=0
\end{align}}\\
\hline
&\\
Proof that $\vec{N}$ is&We know that at least one of a or c is not zero.\\
similar over $\mathbb{C}$ to&If a = 0 then c $\neq$ 0, it must be that x = 0.\\
$\myvec{0&0\\1&0}$&So in this case $\vec{N}=\myvec{0&0\\c&0}$ which is similar to $\myvec{0&0\\1&0}$ as before.\\
&\\
&\parbox{6cm}{\begin{align}
\mbox{If } a &\neq0\\
\mbox{then }x &\neq0\\
\mbox{else }a(a+cx)&=0\\
\implies a&=0\\
\mbox{Thus }a+cx&=0\\
\mbox{Hence }\vec{N}&=\myvec{a&ax\\ \frac{-a}{x}&-a}
\end{align}}\\
 &This is similar to $\myvec{a&a\\-a&-a}$ via $\vec{P}=\myvec{\sqrt{x}&0\\0&\frac{1}{\sqrt{x}}}$.\\
 &And $\myvec{a&a\\-a&-a}$ is similar to $\myvec{0&0\\-a&0}$ via $\vec{P}=\myvec{-1&-1\\1&0}$\\
 &And this finally is similar to $\myvec{0&0\\1&0}$ as before.\\
 &\\
\hline
&\\
Conclusion &Thus either $\vec{N}$ = 0 or $\vec{N}$ is similar over $\mathbb{C}$ to $\myvec{0&0\\1&0}$.\\
&\\
\hline
\caption{Solution summary}
\label{eq:solutions/6/2/11/table:1}
\end{longtable}

%\item Let $\vec{A}$ be an $n \times n$ (square) matrix, Prove the following two statements:
% \begin{enumerate}
%   \item If $\vec{A}$ is invertible and $\vec{A}\vec{B}=0$ for some $n \times n$ matrix $\vec{B}$, then $\vec{B}=0$.
%   \item If $\vec{A}$ is not invertible, then there exists an $n \times n$ matrix $\vec{B}$ such that $\vec{A}\vec{B}=0$ but $\vec{B} \not= 0$.
%   \item If $\vec{A}$ is invertible and $\vec{A}\vec{B}=0$ for some $n \times n$ matrix $\vec{B}$, then $\vec{B}=0$.
% \end{enumerate}
%
%\solution
%See Table \ref{eq:solutions/6/2/11/table:1}
\onecolumn
\begin{longtable}{|l|l|}
\hline
\multirow{3}{*}{} & \\
Statement&Solution\\
\hline
&\parbox{6cm}{\begin{align}
\mbox{Let }\vec{N}&=\myvec{a&b\\c&d}\\
\mbox{Since }\vec{N}^2&=0
\end{align}}\\
&If $\myvec{a\\c},\myvec{b\\d}$ are linearly independent then $\vec{N}$ is diagonalizable to $\myvec{0&0\\0&0}$.\\
&\parbox{6cm}{\begin{align}
\mbox{If }\vec{P}\vec{N}\vec{P}^{-1}=0\\
\mbox{then }\vec{N}=\vec{P}^{-1}\vec{0}\vec{P}=0
\end{align}}\\
Proof that&So in this case $\vec{N}$ itself is the zero matrix.\\
$\vec{N}=0$&This contradicts the assumption that $\myvec{a\\c},\myvec{b\\d}$ are linearly independent.\\
&$\therefore$ we can assume that $\myvec{a\\c},\myvec{b\\d}$ are linearly dependent if both are\\
&equal to the zero vector\\
&\parbox{6cm}
{\begin{align}
   \mbox{then } \vec{N} &= 0.
\end{align}}\\
%\hline
%\pagebreak
\hline
&\\
&Therefore we can assume at least one vector is non-zero.\\
Assuming $\myvec{b\\d}$ as&Therefore
$\vec{N}=\myvec{a&0\\c&0}$\\
the zero vector&\\
&\parbox{6cm}{\begin{align}
\mbox{So }\vec{N}^2&=0\\
\implies a^2&=0\\
\therefore a&=0\\
\mbox{Thus }\vec{N}&=\myvec{a&0\\c&0}
\end{align}}\\
&In this case $\vec{N}$ is similar to $\vec{N}=\myvec{0&0\\1&0}$ via the matrix $\vec{P}=\myvec{c&0\\0&1}$\\
&\\
\hline
&\\
Assuming $\myvec{a\\c}$ as&Therefore
$\vec{N}=\myvec{0&b\\0&d}$\\
the zero vector&\\
&\parbox{6cm}{\begin{align}
\mbox{Then }\vec{N}^2&=0\\
\implies d^2&=0\\
\therefore d&=0\\
\mbox{Thus }\vec{N}&=\myvec{0&b\\0&0}
\end{align}}\\
&In this case $\vec{N}$ is similar to $\vec{N}=\myvec{0&0\\b&0}$ via the matrix $\vec{P}=\myvec{0&1\\1&0}$,\\
&which is similar to $\myvec{0&0\\1&0}$ as above.\\
&\\
\hline
&\\
Hence& we can assume neither $\myvec{a\\c}$ or $\myvec{b\\d}$ is the zero vector.\\
&\\
%\hline
%\pagebreak
\hline
&\\
Consequences of &Since they are linearly dependent we can assume,\\
linear&\\
independence&\\
&\parbox{6cm}{\begin{align}
\myvec{b\\d}&=x\myvec{a\\c}\\
\therefore \vec{N}&=\myvec{a&ax\\c&cx}\\
\therefore \vec{N}^2&=0\\
\implies a(a+cx)&=0\\
c(a+cx)&=0\\
ax(a+cx)&=0\\
cx(a+cx)&=0
\end{align}}\\
\hline
&\\
Proof that $\vec{N}$ is&We know that at least one of a or c is not zero.\\
similar over $\mathbb{C}$ to&If a = 0 then c $\neq$ 0, it must be that x = 0.\\
$\myvec{0&0\\1&0}$&So in this case $\vec{N}=\myvec{0&0\\c&0}$ which is similar to $\myvec{0&0\\1&0}$ as before.\\
&\\
&\parbox{6cm}{\begin{align}
\mbox{If } a &\neq0\\
\mbox{then }x &\neq0\\
\mbox{else }a(a+cx)&=0\\
\implies a&=0\\
\mbox{Thus }a+cx&=0\\
\mbox{Hence }\vec{N}&=\myvec{a&ax\\ \frac{-a}{x}&-a}
\end{align}}\\
 &This is similar to $\myvec{a&a\\-a&-a}$ via $\vec{P}=\myvec{\sqrt{x}&0\\0&\frac{1}{\sqrt{x}}}$.\\
 &And $\myvec{a&a\\-a&-a}$ is similar to $\myvec{0&0\\-a&0}$ via $\vec{P}=\myvec{-1&-1\\1&0}$\\
 &And this finally is similar to $\myvec{0&0\\1&0}$ as before.\\
 &\\
\hline
&\\
Conclusion &Thus either $\vec{N}$ = 0 or $\vec{N}$ is similar over $\mathbb{C}$ to $\myvec{0&0\\1&0}$.\\
&\\
\hline
\caption{Solution summary}
\label{eq:solutions/6/2/11/table:1}
\end{longtable}

\end{enumerate}



\subsection{Invertible Matrices}
\renewcommand{\theequation}{\theenumi}
\renewcommand{\thefigure}{\theenumi}
\begin{enumerate}[label=\thesubsection.\arabic*.,ref=\thesubsection.\theenumi]
\numberwithin{equation}{enumi}
\numberwithin{figure}{enumi}
%
\item Let
\begin{align}
\vec{A}=\myvec{1&2&1&0\\-1&0&3&5\\1&-2&1&1}  
\end{align}
Find a row-reduced echelon matrix $\vec{R}$ which is row-equivalent to $\vec{A}$ and an invertible 3x3 matrix $\vec{P}$ such that $\vec{R}$ = $\vec{P}$ $\vec{A}$.
%
%
\solution
See Table \ref{eq:solutions/6/2/11/table:1}
\onecolumn
\begin{longtable}{|l|l|}
\hline
\multirow{3}{*}{} & \\
Statement&Solution\\
\hline
&\parbox{6cm}{\begin{align}
\mbox{Let }\vec{N}&=\myvec{a&b\\c&d}\\
\mbox{Since }\vec{N}^2&=0
\end{align}}\\
&If $\myvec{a\\c},\myvec{b\\d}$ are linearly independent then $\vec{N}$ is diagonalizable to $\myvec{0&0\\0&0}$.\\
&\parbox{6cm}{\begin{align}
\mbox{If }\vec{P}\vec{N}\vec{P}^{-1}=0\\
\mbox{then }\vec{N}=\vec{P}^{-1}\vec{0}\vec{P}=0
\end{align}}\\
Proof that&So in this case $\vec{N}$ itself is the zero matrix.\\
$\vec{N}=0$&This contradicts the assumption that $\myvec{a\\c},\myvec{b\\d}$ are linearly independent.\\
&$\therefore$ we can assume that $\myvec{a\\c},\myvec{b\\d}$ are linearly dependent if both are\\
&equal to the zero vector\\
&\parbox{6cm}
{\begin{align}
   \mbox{then } \vec{N} &= 0.
\end{align}}\\
%\hline
%\pagebreak
\hline
&\\
&Therefore we can assume at least one vector is non-zero.\\
Assuming $\myvec{b\\d}$ as&Therefore
$\vec{N}=\myvec{a&0\\c&0}$\\
the zero vector&\\
&\parbox{6cm}{\begin{align}
\mbox{So }\vec{N}^2&=0\\
\implies a^2&=0\\
\therefore a&=0\\
\mbox{Thus }\vec{N}&=\myvec{a&0\\c&0}
\end{align}}\\
&In this case $\vec{N}$ is similar to $\vec{N}=\myvec{0&0\\1&0}$ via the matrix $\vec{P}=\myvec{c&0\\0&1}$\\
&\\
\hline
&\\
Assuming $\myvec{a\\c}$ as&Therefore
$\vec{N}=\myvec{0&b\\0&d}$\\
the zero vector&\\
&\parbox{6cm}{\begin{align}
\mbox{Then }\vec{N}^2&=0\\
\implies d^2&=0\\
\therefore d&=0\\
\mbox{Thus }\vec{N}&=\myvec{0&b\\0&0}
\end{align}}\\
&In this case $\vec{N}$ is similar to $\vec{N}=\myvec{0&0\\b&0}$ via the matrix $\vec{P}=\myvec{0&1\\1&0}$,\\
&which is similar to $\myvec{0&0\\1&0}$ as above.\\
&\\
\hline
&\\
Hence& we can assume neither $\myvec{a\\c}$ or $\myvec{b\\d}$ is the zero vector.\\
&\\
%\hline
%\pagebreak
\hline
&\\
Consequences of &Since they are linearly dependent we can assume,\\
linear&\\
independence&\\
&\parbox{6cm}{\begin{align}
\myvec{b\\d}&=x\myvec{a\\c}\\
\therefore \vec{N}&=\myvec{a&ax\\c&cx}\\
\therefore \vec{N}^2&=0\\
\implies a(a+cx)&=0\\
c(a+cx)&=0\\
ax(a+cx)&=0\\
cx(a+cx)&=0
\end{align}}\\
\hline
&\\
Proof that $\vec{N}$ is&We know that at least one of a or c is not zero.\\
similar over $\mathbb{C}$ to&If a = 0 then c $\neq$ 0, it must be that x = 0.\\
$\myvec{0&0\\1&0}$&So in this case $\vec{N}=\myvec{0&0\\c&0}$ which is similar to $\myvec{0&0\\1&0}$ as before.\\
&\\
&\parbox{6cm}{\begin{align}
\mbox{If } a &\neq0\\
\mbox{then }x &\neq0\\
\mbox{else }a(a+cx)&=0\\
\implies a&=0\\
\mbox{Thus }a+cx&=0\\
\mbox{Hence }\vec{N}&=\myvec{a&ax\\ \frac{-a}{x}&-a}
\end{align}}\\
 &This is similar to $\myvec{a&a\\-a&-a}$ via $\vec{P}=\myvec{\sqrt{x}&0\\0&\frac{1}{\sqrt{x}}}$.\\
 &And $\myvec{a&a\\-a&-a}$ is similar to $\myvec{0&0\\-a&0}$ via $\vec{P}=\myvec{-1&-1\\1&0}$\\
 &And this finally is similar to $\myvec{0&0\\1&0}$ as before.\\
 &\\
\hline
&\\
Conclusion &Thus either $\vec{N}$ = 0 or $\vec{N}$ is similar over $\mathbb{C}$ to $\myvec{0&0\\1&0}$.\\
&\\
\hline
\caption{Solution summary}
\label{eq:solutions/6/2/11/table:1}
\end{longtable}

\item Let $\vec{A}= \myvec{2 & 0 & i \\ 1 & -3 & -i \\ i & 1 &  1}$, find a row-reduced echelon matrix $\vec{R}$ which is row-equivalent to $\vec{A}$ and an invertible 3x3 matrix $\vec{P}$ such that $\vec{R}$ = $\vec{P}$ $\vec{A}$.
%
\solution
See Table \ref{eq:solutions/6/2/11/table:1}
\onecolumn
\begin{longtable}{|l|l|}
\hline
\multirow{3}{*}{} & \\
Statement&Solution\\
\hline
&\parbox{6cm}{\begin{align}
\mbox{Let }\vec{N}&=\myvec{a&b\\c&d}\\
\mbox{Since }\vec{N}^2&=0
\end{align}}\\
&If $\myvec{a\\c},\myvec{b\\d}$ are linearly independent then $\vec{N}$ is diagonalizable to $\myvec{0&0\\0&0}$.\\
&\parbox{6cm}{\begin{align}
\mbox{If }\vec{P}\vec{N}\vec{P}^{-1}=0\\
\mbox{then }\vec{N}=\vec{P}^{-1}\vec{0}\vec{P}=0
\end{align}}\\
Proof that&So in this case $\vec{N}$ itself is the zero matrix.\\
$\vec{N}=0$&This contradicts the assumption that $\myvec{a\\c},\myvec{b\\d}$ are linearly independent.\\
&$\therefore$ we can assume that $\myvec{a\\c},\myvec{b\\d}$ are linearly dependent if both are\\
&equal to the zero vector\\
&\parbox{6cm}
{\begin{align}
   \mbox{then } \vec{N} &= 0.
\end{align}}\\
%\hline
%\pagebreak
\hline
&\\
&Therefore we can assume at least one vector is non-zero.\\
Assuming $\myvec{b\\d}$ as&Therefore
$\vec{N}=\myvec{a&0\\c&0}$\\
the zero vector&\\
&\parbox{6cm}{\begin{align}
\mbox{So }\vec{N}^2&=0\\
\implies a^2&=0\\
\therefore a&=0\\
\mbox{Thus }\vec{N}&=\myvec{a&0\\c&0}
\end{align}}\\
&In this case $\vec{N}$ is similar to $\vec{N}=\myvec{0&0\\1&0}$ via the matrix $\vec{P}=\myvec{c&0\\0&1}$\\
&\\
\hline
&\\
Assuming $\myvec{a\\c}$ as&Therefore
$\vec{N}=\myvec{0&b\\0&d}$\\
the zero vector&\\
&\parbox{6cm}{\begin{align}
\mbox{Then }\vec{N}^2&=0\\
\implies d^2&=0\\
\therefore d&=0\\
\mbox{Thus }\vec{N}&=\myvec{0&b\\0&0}
\end{align}}\\
&In this case $\vec{N}$ is similar to $\vec{N}=\myvec{0&0\\b&0}$ via the matrix $\vec{P}=\myvec{0&1\\1&0}$,\\
&which is similar to $\myvec{0&0\\1&0}$ as above.\\
&\\
\hline
&\\
Hence& we can assume neither $\myvec{a\\c}$ or $\myvec{b\\d}$ is the zero vector.\\
&\\
%\hline
%\pagebreak
\hline
&\\
Consequences of &Since they are linearly dependent we can assume,\\
linear&\\
independence&\\
&\parbox{6cm}{\begin{align}
\myvec{b\\d}&=x\myvec{a\\c}\\
\therefore \vec{N}&=\myvec{a&ax\\c&cx}\\
\therefore \vec{N}^2&=0\\
\implies a(a+cx)&=0\\
c(a+cx)&=0\\
ax(a+cx)&=0\\
cx(a+cx)&=0
\end{align}}\\
\hline
&\\
Proof that $\vec{N}$ is&We know that at least one of a or c is not zero.\\
similar over $\mathbb{C}$ to&If a = 0 then c $\neq$ 0, it must be that x = 0.\\
$\myvec{0&0\\1&0}$&So in this case $\vec{N}=\myvec{0&0\\c&0}$ which is similar to $\myvec{0&0\\1&0}$ as before.\\
&\\
&\parbox{6cm}{\begin{align}
\mbox{If } a &\neq0\\
\mbox{then }x &\neq0\\
\mbox{else }a(a+cx)&=0\\
\implies a&=0\\
\mbox{Thus }a+cx&=0\\
\mbox{Hence }\vec{N}&=\myvec{a&ax\\ \frac{-a}{x}&-a}
\end{align}}\\
 &This is similar to $\myvec{a&a\\-a&-a}$ via $\vec{P}=\myvec{\sqrt{x}&0\\0&\frac{1}{\sqrt{x}}}$.\\
 &And $\myvec{a&a\\-a&-a}$ is similar to $\myvec{0&0\\-a&0}$ via $\vec{P}=\myvec{-1&-1\\1&0}$\\
 &And this finally is similar to $\myvec{0&0\\1&0}$ as before.\\
 &\\
\hline
&\\
Conclusion &Thus either $\vec{N}$ = 0 or $\vec{N}$ is similar over $\mathbb{C}$ to $\myvec{0&0\\1&0}$.\\
&\\
\hline
\caption{Solution summary}
\label{eq:solutions/6/2/11/table:1}
\end{longtable}

%
\item For each of the two matrices use elementary row operations to discover whether it is invertible, and to find the inverse in case it is invertible.\\

$\vec{A}=\myvec{2&5&-1\\4&-1&2\\6&4&1},
\vec{B}=\myvec{1&-1&2\\3&2&4\\0&1&-2}$
%
\\
%
\solution
See Table \ref{eq:solutions/6/2/11/table:1}
\onecolumn
\begin{longtable}{|l|l|}
\hline
\multirow{3}{*}{} & \\
Statement&Solution\\
\hline
&\parbox{6cm}{\begin{align}
\mbox{Let }\vec{N}&=\myvec{a&b\\c&d}\\
\mbox{Since }\vec{N}^2&=0
\end{align}}\\
&If $\myvec{a\\c},\myvec{b\\d}$ are linearly independent then $\vec{N}$ is diagonalizable to $\myvec{0&0\\0&0}$.\\
&\parbox{6cm}{\begin{align}
\mbox{If }\vec{P}\vec{N}\vec{P}^{-1}=0\\
\mbox{then }\vec{N}=\vec{P}^{-1}\vec{0}\vec{P}=0
\end{align}}\\
Proof that&So in this case $\vec{N}$ itself is the zero matrix.\\
$\vec{N}=0$&This contradicts the assumption that $\myvec{a\\c},\myvec{b\\d}$ are linearly independent.\\
&$\therefore$ we can assume that $\myvec{a\\c},\myvec{b\\d}$ are linearly dependent if both are\\
&equal to the zero vector\\
&\parbox{6cm}
{\begin{align}
   \mbox{then } \vec{N} &= 0.
\end{align}}\\
%\hline
%\pagebreak
\hline
&\\
&Therefore we can assume at least one vector is non-zero.\\
Assuming $\myvec{b\\d}$ as&Therefore
$\vec{N}=\myvec{a&0\\c&0}$\\
the zero vector&\\
&\parbox{6cm}{\begin{align}
\mbox{So }\vec{N}^2&=0\\
\implies a^2&=0\\
\therefore a&=0\\
\mbox{Thus }\vec{N}&=\myvec{a&0\\c&0}
\end{align}}\\
&In this case $\vec{N}$ is similar to $\vec{N}=\myvec{0&0\\1&0}$ via the matrix $\vec{P}=\myvec{c&0\\0&1}$\\
&\\
\hline
&\\
Assuming $\myvec{a\\c}$ as&Therefore
$\vec{N}=\myvec{0&b\\0&d}$\\
the zero vector&\\
&\parbox{6cm}{\begin{align}
\mbox{Then }\vec{N}^2&=0\\
\implies d^2&=0\\
\therefore d&=0\\
\mbox{Thus }\vec{N}&=\myvec{0&b\\0&0}
\end{align}}\\
&In this case $\vec{N}$ is similar to $\vec{N}=\myvec{0&0\\b&0}$ via the matrix $\vec{P}=\myvec{0&1\\1&0}$,\\
&which is similar to $\myvec{0&0\\1&0}$ as above.\\
&\\
\hline
&\\
Hence& we can assume neither $\myvec{a\\c}$ or $\myvec{b\\d}$ is the zero vector.\\
&\\
%\hline
%\pagebreak
\hline
&\\
Consequences of &Since they are linearly dependent we can assume,\\
linear&\\
independence&\\
&\parbox{6cm}{\begin{align}
\myvec{b\\d}&=x\myvec{a\\c}\\
\therefore \vec{N}&=\myvec{a&ax\\c&cx}\\
\therefore \vec{N}^2&=0\\
\implies a(a+cx)&=0\\
c(a+cx)&=0\\
ax(a+cx)&=0\\
cx(a+cx)&=0
\end{align}}\\
\hline
&\\
Proof that $\vec{N}$ is&We know that at least one of a or c is not zero.\\
similar over $\mathbb{C}$ to&If a = 0 then c $\neq$ 0, it must be that x = 0.\\
$\myvec{0&0\\1&0}$&So in this case $\vec{N}=\myvec{0&0\\c&0}$ which is similar to $\myvec{0&0\\1&0}$ as before.\\
&\\
&\parbox{6cm}{\begin{align}
\mbox{If } a &\neq0\\
\mbox{then }x &\neq0\\
\mbox{else }a(a+cx)&=0\\
\implies a&=0\\
\mbox{Thus }a+cx&=0\\
\mbox{Hence }\vec{N}&=\myvec{a&ax\\ \frac{-a}{x}&-a}
\end{align}}\\
 &This is similar to $\myvec{a&a\\-a&-a}$ via $\vec{P}=\myvec{\sqrt{x}&0\\0&\frac{1}{\sqrt{x}}}$.\\
 &And $\myvec{a&a\\-a&-a}$ is similar to $\myvec{0&0\\-a&0}$ via $\vec{P}=\myvec{-1&-1\\1&0}$\\
 &And this finally is similar to $\myvec{0&0\\1&0}$ as before.\\
 &\\
\hline
&\\
Conclusion &Thus either $\vec{N}$ = 0 or $\vec{N}$ is similar over $\mathbb{C}$ to $\myvec{0&0\\1&0}$.\\
&\\
\hline
\caption{Solution summary}
\label{eq:solutions/6/2/11/table:1}
\end{longtable}

\item Let 
\begin{align}
	\vec{A} = \myvec{5 & 0 & 0\\1 & 5 & 0\\0 & 1 & 5}
\end{align}
For which $\vec{X}$ does there exist a scalar c such that $\vec{AX} = c \vec{X}$
%
\\
%
\solution
See Table \ref{eq:solutions/6/2/11/table:1}
\onecolumn
\begin{longtable}{|l|l|}
\hline
\multirow{3}{*}{} & \\
Statement&Solution\\
\hline
&\parbox{6cm}{\begin{align}
\mbox{Let }\vec{N}&=\myvec{a&b\\c&d}\\
\mbox{Since }\vec{N}^2&=0
\end{align}}\\
&If $\myvec{a\\c},\myvec{b\\d}$ are linearly independent then $\vec{N}$ is diagonalizable to $\myvec{0&0\\0&0}$.\\
&\parbox{6cm}{\begin{align}
\mbox{If }\vec{P}\vec{N}\vec{P}^{-1}=0\\
\mbox{then }\vec{N}=\vec{P}^{-1}\vec{0}\vec{P}=0
\end{align}}\\
Proof that&So in this case $\vec{N}$ itself is the zero matrix.\\
$\vec{N}=0$&This contradicts the assumption that $\myvec{a\\c},\myvec{b\\d}$ are linearly independent.\\
&$\therefore$ we can assume that $\myvec{a\\c},\myvec{b\\d}$ are linearly dependent if both are\\
&equal to the zero vector\\
&\parbox{6cm}
{\begin{align}
   \mbox{then } \vec{N} &= 0.
\end{align}}\\
%\hline
%\pagebreak
\hline
&\\
&Therefore we can assume at least one vector is non-zero.\\
Assuming $\myvec{b\\d}$ as&Therefore
$\vec{N}=\myvec{a&0\\c&0}$\\
the zero vector&\\
&\parbox{6cm}{\begin{align}
\mbox{So }\vec{N}^2&=0\\
\implies a^2&=0\\
\therefore a&=0\\
\mbox{Thus }\vec{N}&=\myvec{a&0\\c&0}
\end{align}}\\
&In this case $\vec{N}$ is similar to $\vec{N}=\myvec{0&0\\1&0}$ via the matrix $\vec{P}=\myvec{c&0\\0&1}$\\
&\\
\hline
&\\
Assuming $\myvec{a\\c}$ as&Therefore
$\vec{N}=\myvec{0&b\\0&d}$\\
the zero vector&\\
&\parbox{6cm}{\begin{align}
\mbox{Then }\vec{N}^2&=0\\
\implies d^2&=0\\
\therefore d&=0\\
\mbox{Thus }\vec{N}&=\myvec{0&b\\0&0}
\end{align}}\\
&In this case $\vec{N}$ is similar to $\vec{N}=\myvec{0&0\\b&0}$ via the matrix $\vec{P}=\myvec{0&1\\1&0}$,\\
&which is similar to $\myvec{0&0\\1&0}$ as above.\\
&\\
\hline
&\\
Hence& we can assume neither $\myvec{a\\c}$ or $\myvec{b\\d}$ is the zero vector.\\
&\\
%\hline
%\pagebreak
\hline
&\\
Consequences of &Since they are linearly dependent we can assume,\\
linear&\\
independence&\\
&\parbox{6cm}{\begin{align}
\myvec{b\\d}&=x\myvec{a\\c}\\
\therefore \vec{N}&=\myvec{a&ax\\c&cx}\\
\therefore \vec{N}^2&=0\\
\implies a(a+cx)&=0\\
c(a+cx)&=0\\
ax(a+cx)&=0\\
cx(a+cx)&=0
\end{align}}\\
\hline
&\\
Proof that $\vec{N}$ is&We know that at least one of a or c is not zero.\\
similar over $\mathbb{C}$ to&If a = 0 then c $\neq$ 0, it must be that x = 0.\\
$\myvec{0&0\\1&0}$&So in this case $\vec{N}=\myvec{0&0\\c&0}$ which is similar to $\myvec{0&0\\1&0}$ as before.\\
&\\
&\parbox{6cm}{\begin{align}
\mbox{If } a &\neq0\\
\mbox{then }x &\neq0\\
\mbox{else }a(a+cx)&=0\\
\implies a&=0\\
\mbox{Thus }a+cx&=0\\
\mbox{Hence }\vec{N}&=\myvec{a&ax\\ \frac{-a}{x}&-a}
\end{align}}\\
 &This is similar to $\myvec{a&a\\-a&-a}$ via $\vec{P}=\myvec{\sqrt{x}&0\\0&\frac{1}{\sqrt{x}}}$.\\
 &And $\myvec{a&a\\-a&-a}$ is similar to $\myvec{0&0\\-a&0}$ via $\vec{P}=\myvec{-1&-1\\1&0}$\\
 &And this finally is similar to $\myvec{0&0\\1&0}$ as before.\\
 &\\
\hline
&\\
Conclusion &Thus either $\vec{N}$ = 0 or $\vec{N}$ is similar over $\mathbb{C}$ to $\myvec{0&0\\1&0}$.\\
&\\
\hline
\caption{Solution summary}
\label{eq:solutions/6/2/11/table:1}
\end{longtable}

\item Discover whether
\begin{align}
	\vec{A}=\myvec{ 1 & 2 & 3 & 4 \\
			0 & 2 & 3 & 4 \\
			0 & 0 & 3 & 4 \\
			0 & 0 & 0 & 4}
\end{align}
is invertible, and find $\vec{A^{-1}}$ if it exists.
\\
%
\solution
See Table \ref{eq:solutions/6/2/11/table:1}
\onecolumn
\begin{longtable}{|l|l|}
\hline
\multirow{3}{*}{} & \\
Statement&Solution\\
\hline
&\parbox{6cm}{\begin{align}
\mbox{Let }\vec{N}&=\myvec{a&b\\c&d}\\
\mbox{Since }\vec{N}^2&=0
\end{align}}\\
&If $\myvec{a\\c},\myvec{b\\d}$ are linearly independent then $\vec{N}$ is diagonalizable to $\myvec{0&0\\0&0}$.\\
&\parbox{6cm}{\begin{align}
\mbox{If }\vec{P}\vec{N}\vec{P}^{-1}=0\\
\mbox{then }\vec{N}=\vec{P}^{-1}\vec{0}\vec{P}=0
\end{align}}\\
Proof that&So in this case $\vec{N}$ itself is the zero matrix.\\
$\vec{N}=0$&This contradicts the assumption that $\myvec{a\\c},\myvec{b\\d}$ are linearly independent.\\
&$\therefore$ we can assume that $\myvec{a\\c},\myvec{b\\d}$ are linearly dependent if both are\\
&equal to the zero vector\\
&\parbox{6cm}
{\begin{align}
   \mbox{then } \vec{N} &= 0.
\end{align}}\\
%\hline
%\pagebreak
\hline
&\\
&Therefore we can assume at least one vector is non-zero.\\
Assuming $\myvec{b\\d}$ as&Therefore
$\vec{N}=\myvec{a&0\\c&0}$\\
the zero vector&\\
&\parbox{6cm}{\begin{align}
\mbox{So }\vec{N}^2&=0\\
\implies a^2&=0\\
\therefore a&=0\\
\mbox{Thus }\vec{N}&=\myvec{a&0\\c&0}
\end{align}}\\
&In this case $\vec{N}$ is similar to $\vec{N}=\myvec{0&0\\1&0}$ via the matrix $\vec{P}=\myvec{c&0\\0&1}$\\
&\\
\hline
&\\
Assuming $\myvec{a\\c}$ as&Therefore
$\vec{N}=\myvec{0&b\\0&d}$\\
the zero vector&\\
&\parbox{6cm}{\begin{align}
\mbox{Then }\vec{N}^2&=0\\
\implies d^2&=0\\
\therefore d&=0\\
\mbox{Thus }\vec{N}&=\myvec{0&b\\0&0}
\end{align}}\\
&In this case $\vec{N}$ is similar to $\vec{N}=\myvec{0&0\\b&0}$ via the matrix $\vec{P}=\myvec{0&1\\1&0}$,\\
&which is similar to $\myvec{0&0\\1&0}$ as above.\\
&\\
\hline
&\\
Hence& we can assume neither $\myvec{a\\c}$ or $\myvec{b\\d}$ is the zero vector.\\
&\\
%\hline
%\pagebreak
\hline
&\\
Consequences of &Since they are linearly dependent we can assume,\\
linear&\\
independence&\\
&\parbox{6cm}{\begin{align}
\myvec{b\\d}&=x\myvec{a\\c}\\
\therefore \vec{N}&=\myvec{a&ax\\c&cx}\\
\therefore \vec{N}^2&=0\\
\implies a(a+cx)&=0\\
c(a+cx)&=0\\
ax(a+cx)&=0\\
cx(a+cx)&=0
\end{align}}\\
\hline
&\\
Proof that $\vec{N}$ is&We know that at least one of a or c is not zero.\\
similar over $\mathbb{C}$ to&If a = 0 then c $\neq$ 0, it must be that x = 0.\\
$\myvec{0&0\\1&0}$&So in this case $\vec{N}=\myvec{0&0\\c&0}$ which is similar to $\myvec{0&0\\1&0}$ as before.\\
&\\
&\parbox{6cm}{\begin{align}
\mbox{If } a &\neq0\\
\mbox{then }x &\neq0\\
\mbox{else }a(a+cx)&=0\\
\implies a&=0\\
\mbox{Thus }a+cx&=0\\
\mbox{Hence }\vec{N}&=\myvec{a&ax\\ \frac{-a}{x}&-a}
\end{align}}\\
 &This is similar to $\myvec{a&a\\-a&-a}$ via $\vec{P}=\myvec{\sqrt{x}&0\\0&\frac{1}{\sqrt{x}}}$.\\
 &And $\myvec{a&a\\-a&-a}$ is similar to $\myvec{0&0\\-a&0}$ via $\vec{P}=\myvec{-1&-1\\1&0}$\\
 &And this finally is similar to $\myvec{0&0\\1&0}$ as before.\\
 &\\
\hline
&\\
Conclusion &Thus either $\vec{N}$ = 0 or $\vec{N}$ is similar over $\mathbb{C}$ to $\myvec{0&0\\1&0}$.\\
&\\
\hline
\caption{Solution summary}
\label{eq:solutions/6/2/11/table:1}
\end{longtable}

%
\item Suppose $\vec{A}$ is a 2$\times$1  matrix and $\vec{B}$ is 1$\times$2 matrix.Prove that $\vec{C}$=$\vec{A}$$\vec{B}$ is non invertible.

\solution
See Table \ref{eq:solutions/6/2/11/table:1}
\onecolumn
\begin{longtable}{|l|l|}
\hline
\multirow{3}{*}{} & \\
Statement&Solution\\
\hline
&\parbox{6cm}{\begin{align}
\mbox{Let }\vec{N}&=\myvec{a&b\\c&d}\\
\mbox{Since }\vec{N}^2&=0
\end{align}}\\
&If $\myvec{a\\c},\myvec{b\\d}$ are linearly independent then $\vec{N}$ is diagonalizable to $\myvec{0&0\\0&0}$.\\
&\parbox{6cm}{\begin{align}
\mbox{If }\vec{P}\vec{N}\vec{P}^{-1}=0\\
\mbox{then }\vec{N}=\vec{P}^{-1}\vec{0}\vec{P}=0
\end{align}}\\
Proof that&So in this case $\vec{N}$ itself is the zero matrix.\\
$\vec{N}=0$&This contradicts the assumption that $\myvec{a\\c},\myvec{b\\d}$ are linearly independent.\\
&$\therefore$ we can assume that $\myvec{a\\c},\myvec{b\\d}$ are linearly dependent if both are\\
&equal to the zero vector\\
&\parbox{6cm}
{\begin{align}
   \mbox{then } \vec{N} &= 0.
\end{align}}\\
%\hline
%\pagebreak
\hline
&\\
&Therefore we can assume at least one vector is non-zero.\\
Assuming $\myvec{b\\d}$ as&Therefore
$\vec{N}=\myvec{a&0\\c&0}$\\
the zero vector&\\
&\parbox{6cm}{\begin{align}
\mbox{So }\vec{N}^2&=0\\
\implies a^2&=0\\
\therefore a&=0\\
\mbox{Thus }\vec{N}&=\myvec{a&0\\c&0}
\end{align}}\\
&In this case $\vec{N}$ is similar to $\vec{N}=\myvec{0&0\\1&0}$ via the matrix $\vec{P}=\myvec{c&0\\0&1}$\\
&\\
\hline
&\\
Assuming $\myvec{a\\c}$ as&Therefore
$\vec{N}=\myvec{0&b\\0&d}$\\
the zero vector&\\
&\parbox{6cm}{\begin{align}
\mbox{Then }\vec{N}^2&=0\\
\implies d^2&=0\\
\therefore d&=0\\
\mbox{Thus }\vec{N}&=\myvec{0&b\\0&0}
\end{align}}\\
&In this case $\vec{N}$ is similar to $\vec{N}=\myvec{0&0\\b&0}$ via the matrix $\vec{P}=\myvec{0&1\\1&0}$,\\
&which is similar to $\myvec{0&0\\1&0}$ as above.\\
&\\
\hline
&\\
Hence& we can assume neither $\myvec{a\\c}$ or $\myvec{b\\d}$ is the zero vector.\\
&\\
%\hline
%\pagebreak
\hline
&\\
Consequences of &Since they are linearly dependent we can assume,\\
linear&\\
independence&\\
&\parbox{6cm}{\begin{align}
\myvec{b\\d}&=x\myvec{a\\c}\\
\therefore \vec{N}&=\myvec{a&ax\\c&cx}\\
\therefore \vec{N}^2&=0\\
\implies a(a+cx)&=0\\
c(a+cx)&=0\\
ax(a+cx)&=0\\
cx(a+cx)&=0
\end{align}}\\
\hline
&\\
Proof that $\vec{N}$ is&We know that at least one of a or c is not zero.\\
similar over $\mathbb{C}$ to&If a = 0 then c $\neq$ 0, it must be that x = 0.\\
$\myvec{0&0\\1&0}$&So in this case $\vec{N}=\myvec{0&0\\c&0}$ which is similar to $\myvec{0&0\\1&0}$ as before.\\
&\\
&\parbox{6cm}{\begin{align}
\mbox{If } a &\neq0\\
\mbox{then }x &\neq0\\
\mbox{else }a(a+cx)&=0\\
\implies a&=0\\
\mbox{Thus }a+cx&=0\\
\mbox{Hence }\vec{N}&=\myvec{a&ax\\ \frac{-a}{x}&-a}
\end{align}}\\
 &This is similar to $\myvec{a&a\\-a&-a}$ via $\vec{P}=\myvec{\sqrt{x}&0\\0&\frac{1}{\sqrt{x}}}$.\\
 &And $\myvec{a&a\\-a&-a}$ is similar to $\myvec{0&0\\-a&0}$ via $\vec{P}=\myvec{-1&-1\\1&0}$\\
 &And this finally is similar to $\myvec{0&0\\1&0}$ as before.\\
 &\\
\hline
&\\
Conclusion &Thus either $\vec{N}$ = 0 or $\vec{N}$ is similar over $\mathbb{C}$ to $\myvec{0&0\\1&0}$.\\
&\\
\hline
\caption{Solution summary}
\label{eq:solutions/6/2/11/table:1}
\end{longtable}

\item Let $\vec{A}$ be an $n \times n$ (square) matrix, Prove the following two statements:
 \begin{enumerate}
   \item If $\vec{A}$ is invertible and $\vec{A}\vec{B}=0$ for some $n \times n$ matrix $\vec{B}$, then $\vec{B}=0$.
   \item If $\vec{A}$ is not invertible, then there exists an $n \times n$ matrix $\vec{B}$ such that $\vec{A}\vec{B}=0$ but $\vec{B} \not= 0$.
%   \item If $\vec{A}$ is invertible and $\vec{A}\vec{B}=0$ for some $n \times n$ matrix $\vec{B}$, then $\vec{B}=0$.
 \end{enumerate}
%
\solution
See Table \ref{eq:solutions/6/2/11/table:1}
\onecolumn
\begin{longtable}{|l|l|}
\hline
\multirow{3}{*}{} & \\
Statement&Solution\\
\hline
&\parbox{6cm}{\begin{align}
\mbox{Let }\vec{N}&=\myvec{a&b\\c&d}\\
\mbox{Since }\vec{N}^2&=0
\end{align}}\\
&If $\myvec{a\\c},\myvec{b\\d}$ are linearly independent then $\vec{N}$ is diagonalizable to $\myvec{0&0\\0&0}$.\\
&\parbox{6cm}{\begin{align}
\mbox{If }\vec{P}\vec{N}\vec{P}^{-1}=0\\
\mbox{then }\vec{N}=\vec{P}^{-1}\vec{0}\vec{P}=0
\end{align}}\\
Proof that&So in this case $\vec{N}$ itself is the zero matrix.\\
$\vec{N}=0$&This contradicts the assumption that $\myvec{a\\c},\myvec{b\\d}$ are linearly independent.\\
&$\therefore$ we can assume that $\myvec{a\\c},\myvec{b\\d}$ are linearly dependent if both are\\
&equal to the zero vector\\
&\parbox{6cm}
{\begin{align}
   \mbox{then } \vec{N} &= 0.
\end{align}}\\
%\hline
%\pagebreak
\hline
&\\
&Therefore we can assume at least one vector is non-zero.\\
Assuming $\myvec{b\\d}$ as&Therefore
$\vec{N}=\myvec{a&0\\c&0}$\\
the zero vector&\\
&\parbox{6cm}{\begin{align}
\mbox{So }\vec{N}^2&=0\\
\implies a^2&=0\\
\therefore a&=0\\
\mbox{Thus }\vec{N}&=\myvec{a&0\\c&0}
\end{align}}\\
&In this case $\vec{N}$ is similar to $\vec{N}=\myvec{0&0\\1&0}$ via the matrix $\vec{P}=\myvec{c&0\\0&1}$\\
&\\
\hline
&\\
Assuming $\myvec{a\\c}$ as&Therefore
$\vec{N}=\myvec{0&b\\0&d}$\\
the zero vector&\\
&\parbox{6cm}{\begin{align}
\mbox{Then }\vec{N}^2&=0\\
\implies d^2&=0\\
\therefore d&=0\\
\mbox{Thus }\vec{N}&=\myvec{0&b\\0&0}
\end{align}}\\
&In this case $\vec{N}$ is similar to $\vec{N}=\myvec{0&0\\b&0}$ via the matrix $\vec{P}=\myvec{0&1\\1&0}$,\\
&which is similar to $\myvec{0&0\\1&0}$ as above.\\
&\\
\hline
&\\
Hence& we can assume neither $\myvec{a\\c}$ or $\myvec{b\\d}$ is the zero vector.\\
&\\
%\hline
%\pagebreak
\hline
&\\
Consequences of &Since they are linearly dependent we can assume,\\
linear&\\
independence&\\
&\parbox{6cm}{\begin{align}
\myvec{b\\d}&=x\myvec{a\\c}\\
\therefore \vec{N}&=\myvec{a&ax\\c&cx}\\
\therefore \vec{N}^2&=0\\
\implies a(a+cx)&=0\\
c(a+cx)&=0\\
ax(a+cx)&=0\\
cx(a+cx)&=0
\end{align}}\\
\hline
&\\
Proof that $\vec{N}$ is&We know that at least one of a or c is not zero.\\
similar over $\mathbb{C}$ to&If a = 0 then c $\neq$ 0, it must be that x = 0.\\
$\myvec{0&0\\1&0}$&So in this case $\vec{N}=\myvec{0&0\\c&0}$ which is similar to $\myvec{0&0\\1&0}$ as before.\\
&\\
&\parbox{6cm}{\begin{align}
\mbox{If } a &\neq0\\
\mbox{then }x &\neq0\\
\mbox{else }a(a+cx)&=0\\
\implies a&=0\\
\mbox{Thus }a+cx&=0\\
\mbox{Hence }\vec{N}&=\myvec{a&ax\\ \frac{-a}{x}&-a}
\end{align}}\\
 &This is similar to $\myvec{a&a\\-a&-a}$ via $\vec{P}=\myvec{\sqrt{x}&0\\0&\frac{1}{\sqrt{x}}}$.\\
 &And $\myvec{a&a\\-a&-a}$ is similar to $\myvec{0&0\\-a&0}$ via $\vec{P}=\myvec{-1&-1\\1&0}$\\
 &And this finally is similar to $\myvec{0&0\\1&0}$ as before.\\
 &\\
\hline
&\\
Conclusion &Thus either $\vec{N}$ = 0 or $\vec{N}$ is similar over $\mathbb{C}$ to $\myvec{0&0\\1&0}$.\\
&\\
\hline
\caption{Solution summary}
\label{eq:solutions/6/2/11/table:1}
\end{longtable}

%
\item Let
\begin{align}
    \vec{A} = \myvec{a & b \\c & d}  \label{eq:solutions/1/5/8/given_matrix}
\end{align}
Prove, using elementary row operations that A is invertible if and only if $\brak{ad - bc} \neq 0$
\\
%
\solution
See Table \ref{eq:solutions/6/2/11/table:1}
\onecolumn
\begin{longtable}{|l|l|}
\hline
\multirow{3}{*}{} & \\
Statement&Solution\\
\hline
&\parbox{6cm}{\begin{align}
\mbox{Let }\vec{N}&=\myvec{a&b\\c&d}\\
\mbox{Since }\vec{N}^2&=0
\end{align}}\\
&If $\myvec{a\\c},\myvec{b\\d}$ are linearly independent then $\vec{N}$ is diagonalizable to $\myvec{0&0\\0&0}$.\\
&\parbox{6cm}{\begin{align}
\mbox{If }\vec{P}\vec{N}\vec{P}^{-1}=0\\
\mbox{then }\vec{N}=\vec{P}^{-1}\vec{0}\vec{P}=0
\end{align}}\\
Proof that&So in this case $\vec{N}$ itself is the zero matrix.\\
$\vec{N}=0$&This contradicts the assumption that $\myvec{a\\c},\myvec{b\\d}$ are linearly independent.\\
&$\therefore$ we can assume that $\myvec{a\\c},\myvec{b\\d}$ are linearly dependent if both are\\
&equal to the zero vector\\
&\parbox{6cm}
{\begin{align}
   \mbox{then } \vec{N} &= 0.
\end{align}}\\
%\hline
%\pagebreak
\hline
&\\
&Therefore we can assume at least one vector is non-zero.\\
Assuming $\myvec{b\\d}$ as&Therefore
$\vec{N}=\myvec{a&0\\c&0}$\\
the zero vector&\\
&\parbox{6cm}{\begin{align}
\mbox{So }\vec{N}^2&=0\\
\implies a^2&=0\\
\therefore a&=0\\
\mbox{Thus }\vec{N}&=\myvec{a&0\\c&0}
\end{align}}\\
&In this case $\vec{N}$ is similar to $\vec{N}=\myvec{0&0\\1&0}$ via the matrix $\vec{P}=\myvec{c&0\\0&1}$\\
&\\
\hline
&\\
Assuming $\myvec{a\\c}$ as&Therefore
$\vec{N}=\myvec{0&b\\0&d}$\\
the zero vector&\\
&\parbox{6cm}{\begin{align}
\mbox{Then }\vec{N}^2&=0\\
\implies d^2&=0\\
\therefore d&=0\\
\mbox{Thus }\vec{N}&=\myvec{0&b\\0&0}
\end{align}}\\
&In this case $\vec{N}$ is similar to $\vec{N}=\myvec{0&0\\b&0}$ via the matrix $\vec{P}=\myvec{0&1\\1&0}$,\\
&which is similar to $\myvec{0&0\\1&0}$ as above.\\
&\\
\hline
&\\
Hence& we can assume neither $\myvec{a\\c}$ or $\myvec{b\\d}$ is the zero vector.\\
&\\
%\hline
%\pagebreak
\hline
&\\
Consequences of &Since they are linearly dependent we can assume,\\
linear&\\
independence&\\
&\parbox{6cm}{\begin{align}
\myvec{b\\d}&=x\myvec{a\\c}\\
\therefore \vec{N}&=\myvec{a&ax\\c&cx}\\
\therefore \vec{N}^2&=0\\
\implies a(a+cx)&=0\\
c(a+cx)&=0\\
ax(a+cx)&=0\\
cx(a+cx)&=0
\end{align}}\\
\hline
&\\
Proof that $\vec{N}$ is&We know that at least one of a or c is not zero.\\
similar over $\mathbb{C}$ to&If a = 0 then c $\neq$ 0, it must be that x = 0.\\
$\myvec{0&0\\1&0}$&So in this case $\vec{N}=\myvec{0&0\\c&0}$ which is similar to $\myvec{0&0\\1&0}$ as before.\\
&\\
&\parbox{6cm}{\begin{align}
\mbox{If } a &\neq0\\
\mbox{then }x &\neq0\\
\mbox{else }a(a+cx)&=0\\
\implies a&=0\\
\mbox{Thus }a+cx&=0\\
\mbox{Hence }\vec{N}&=\myvec{a&ax\\ \frac{-a}{x}&-a}
\end{align}}\\
 &This is similar to $\myvec{a&a\\-a&-a}$ via $\vec{P}=\myvec{\sqrt{x}&0\\0&\frac{1}{\sqrt{x}}}$.\\
 &And $\myvec{a&a\\-a&-a}$ is similar to $\myvec{0&0\\-a&0}$ via $\vec{P}=\myvec{-1&-1\\1&0}$\\
 &And this finally is similar to $\myvec{0&0\\1&0}$ as before.\\
 &\\
\hline
&\\
Conclusion &Thus either $\vec{N}$ = 0 or $\vec{N}$ is similar over $\mathbb{C}$ to $\myvec{0&0\\1&0}$.\\
&\\
\hline
\caption{Solution summary}
\label{eq:solutions/6/2/11/table:1}
\end{longtable}

\item An $n\times n$ matrix $\vec{A}$ is called upper-triangular if $\vec{A_{ij}}=0$ for $i>j$, that is, if every entry below the main diagonal is 0. Prove that an upper-triangular (square) matrix is invertible if and only if every entry on its main diagonal is different from 0.
\solution
See Table \ref{eq:solutions/6/2/11/table:1}
\onecolumn
\begin{longtable}{|l|l|}
\hline
\multirow{3}{*}{} & \\
Statement&Solution\\
\hline
&\parbox{6cm}{\begin{align}
\mbox{Let }\vec{N}&=\myvec{a&b\\c&d}\\
\mbox{Since }\vec{N}^2&=0
\end{align}}\\
&If $\myvec{a\\c},\myvec{b\\d}$ are linearly independent then $\vec{N}$ is diagonalizable to $\myvec{0&0\\0&0}$.\\
&\parbox{6cm}{\begin{align}
\mbox{If }\vec{P}\vec{N}\vec{P}^{-1}=0\\
\mbox{then }\vec{N}=\vec{P}^{-1}\vec{0}\vec{P}=0
\end{align}}\\
Proof that&So in this case $\vec{N}$ itself is the zero matrix.\\
$\vec{N}=0$&This contradicts the assumption that $\myvec{a\\c},\myvec{b\\d}$ are linearly independent.\\
&$\therefore$ we can assume that $\myvec{a\\c},\myvec{b\\d}$ are linearly dependent if both are\\
&equal to the zero vector\\
&\parbox{6cm}
{\begin{align}
   \mbox{then } \vec{N} &= 0.
\end{align}}\\
%\hline
%\pagebreak
\hline
&\\
&Therefore we can assume at least one vector is non-zero.\\
Assuming $\myvec{b\\d}$ as&Therefore
$\vec{N}=\myvec{a&0\\c&0}$\\
the zero vector&\\
&\parbox{6cm}{\begin{align}
\mbox{So }\vec{N}^2&=0\\
\implies a^2&=0\\
\therefore a&=0\\
\mbox{Thus }\vec{N}&=\myvec{a&0\\c&0}
\end{align}}\\
&In this case $\vec{N}$ is similar to $\vec{N}=\myvec{0&0\\1&0}$ via the matrix $\vec{P}=\myvec{c&0\\0&1}$\\
&\\
\hline
&\\
Assuming $\myvec{a\\c}$ as&Therefore
$\vec{N}=\myvec{0&b\\0&d}$\\
the zero vector&\\
&\parbox{6cm}{\begin{align}
\mbox{Then }\vec{N}^2&=0\\
\implies d^2&=0\\
\therefore d&=0\\
\mbox{Thus }\vec{N}&=\myvec{0&b\\0&0}
\end{align}}\\
&In this case $\vec{N}$ is similar to $\vec{N}=\myvec{0&0\\b&0}$ via the matrix $\vec{P}=\myvec{0&1\\1&0}$,\\
&which is similar to $\myvec{0&0\\1&0}$ as above.\\
&\\
\hline
&\\
Hence& we can assume neither $\myvec{a\\c}$ or $\myvec{b\\d}$ is the zero vector.\\
&\\
%\hline
%\pagebreak
\hline
&\\
Consequences of &Since they are linearly dependent we can assume,\\
linear&\\
independence&\\
&\parbox{6cm}{\begin{align}
\myvec{b\\d}&=x\myvec{a\\c}\\
\therefore \vec{N}&=\myvec{a&ax\\c&cx}\\
\therefore \vec{N}^2&=0\\
\implies a(a+cx)&=0\\
c(a+cx)&=0\\
ax(a+cx)&=0\\
cx(a+cx)&=0
\end{align}}\\
\hline
&\\
Proof that $\vec{N}$ is&We know that at least one of a or c is not zero.\\
similar over $\mathbb{C}$ to&If a = 0 then c $\neq$ 0, it must be that x = 0.\\
$\myvec{0&0\\1&0}$&So in this case $\vec{N}=\myvec{0&0\\c&0}$ which is similar to $\myvec{0&0\\1&0}$ as before.\\
&\\
&\parbox{6cm}{\begin{align}
\mbox{If } a &\neq0\\
\mbox{then }x &\neq0\\
\mbox{else }a(a+cx)&=0\\
\implies a&=0\\
\mbox{Thus }a+cx&=0\\
\mbox{Hence }\vec{N}&=\myvec{a&ax\\ \frac{-a}{x}&-a}
\end{align}}\\
 &This is similar to $\myvec{a&a\\-a&-a}$ via $\vec{P}=\myvec{\sqrt{x}&0\\0&\frac{1}{\sqrt{x}}}$.\\
 &And $\myvec{a&a\\-a&-a}$ is similar to $\myvec{0&0\\-a&0}$ via $\vec{P}=\myvec{-1&-1\\1&0}$\\
 &And this finally is similar to $\myvec{0&0\\1&0}$ as before.\\
 &\\
\hline
&\\
Conclusion &Thus either $\vec{N}$ = 0 or $\vec{N}$ is similar over $\mathbb{C}$ to $\myvec{0&0\\1&0}$.\\
&\\
\hline
\caption{Solution summary}
\label{eq:solutions/6/2/11/table:1}
\end{longtable}

\item Let $A$ be a $m\times n$ matrix. Show that by a finite number of elementary row and/or column operations one can pass from $A$ to a matrix $R$ which is both row-reduced echelon and column-reduced echelon , i.e., $R_{ij}=0$ if $i\neq j$, $R_{ii}=1$, $1\leq i\leq r$, $R_{ii}=0$, if $i>r$. Show that $R = PAQ$, where $P$ is an invertible $m\times m$ matrix and $Q$ is an invertible $n\times n$ matrix.
\\
\solution
See Table \ref{eq:solutions/6/2/11/table:1}
\onecolumn
\begin{longtable}{|l|l|}
\hline
\multirow{3}{*}{} & \\
Statement&Solution\\
\hline
&\parbox{6cm}{\begin{align}
\mbox{Let }\vec{N}&=\myvec{a&b\\c&d}\\
\mbox{Since }\vec{N}^2&=0
\end{align}}\\
&If $\myvec{a\\c},\myvec{b\\d}$ are linearly independent then $\vec{N}$ is diagonalizable to $\myvec{0&0\\0&0}$.\\
&\parbox{6cm}{\begin{align}
\mbox{If }\vec{P}\vec{N}\vec{P}^{-1}=0\\
\mbox{then }\vec{N}=\vec{P}^{-1}\vec{0}\vec{P}=0
\end{align}}\\
Proof that&So in this case $\vec{N}$ itself is the zero matrix.\\
$\vec{N}=0$&This contradicts the assumption that $\myvec{a\\c},\myvec{b\\d}$ are linearly independent.\\
&$\therefore$ we can assume that $\myvec{a\\c},\myvec{b\\d}$ are linearly dependent if both are\\
&equal to the zero vector\\
&\parbox{6cm}
{\begin{align}
   \mbox{then } \vec{N} &= 0.
\end{align}}\\
%\hline
%\pagebreak
\hline
&\\
&Therefore we can assume at least one vector is non-zero.\\
Assuming $\myvec{b\\d}$ as&Therefore
$\vec{N}=\myvec{a&0\\c&0}$\\
the zero vector&\\
&\parbox{6cm}{\begin{align}
\mbox{So }\vec{N}^2&=0\\
\implies a^2&=0\\
\therefore a&=0\\
\mbox{Thus }\vec{N}&=\myvec{a&0\\c&0}
\end{align}}\\
&In this case $\vec{N}$ is similar to $\vec{N}=\myvec{0&0\\1&0}$ via the matrix $\vec{P}=\myvec{c&0\\0&1}$\\
&\\
\hline
&\\
Assuming $\myvec{a\\c}$ as&Therefore
$\vec{N}=\myvec{0&b\\0&d}$\\
the zero vector&\\
&\parbox{6cm}{\begin{align}
\mbox{Then }\vec{N}^2&=0\\
\implies d^2&=0\\
\therefore d&=0\\
\mbox{Thus }\vec{N}&=\myvec{0&b\\0&0}
\end{align}}\\
&In this case $\vec{N}$ is similar to $\vec{N}=\myvec{0&0\\b&0}$ via the matrix $\vec{P}=\myvec{0&1\\1&0}$,\\
&which is similar to $\myvec{0&0\\1&0}$ as above.\\
&\\
\hline
&\\
Hence& we can assume neither $\myvec{a\\c}$ or $\myvec{b\\d}$ is the zero vector.\\
&\\
%\hline
%\pagebreak
\hline
&\\
Consequences of &Since they are linearly dependent we can assume,\\
linear&\\
independence&\\
&\parbox{6cm}{\begin{align}
\myvec{b\\d}&=x\myvec{a\\c}\\
\therefore \vec{N}&=\myvec{a&ax\\c&cx}\\
\therefore \vec{N}^2&=0\\
\implies a(a+cx)&=0\\
c(a+cx)&=0\\
ax(a+cx)&=0\\
cx(a+cx)&=0
\end{align}}\\
\hline
&\\
Proof that $\vec{N}$ is&We know that at least one of a or c is not zero.\\
similar over $\mathbb{C}$ to&If a = 0 then c $\neq$ 0, it must be that x = 0.\\
$\myvec{0&0\\1&0}$&So in this case $\vec{N}=\myvec{0&0\\c&0}$ which is similar to $\myvec{0&0\\1&0}$ as before.\\
&\\
&\parbox{6cm}{\begin{align}
\mbox{If } a &\neq0\\
\mbox{then }x &\neq0\\
\mbox{else }a(a+cx)&=0\\
\implies a&=0\\
\mbox{Thus }a+cx&=0\\
\mbox{Hence }\vec{N}&=\myvec{a&ax\\ \frac{-a}{x}&-a}
\end{align}}\\
 &This is similar to $\myvec{a&a\\-a&-a}$ via $\vec{P}=\myvec{\sqrt{x}&0\\0&\frac{1}{\sqrt{x}}}$.\\
 &And $\myvec{a&a\\-a&-a}$ is similar to $\myvec{0&0\\-a&0}$ via $\vec{P}=\myvec{-1&-1\\1&0}$\\
 &And this finally is similar to $\myvec{0&0\\1&0}$ as before.\\
 &\\
\hline
&\\
Conclusion &Thus either $\vec{N}$ = 0 or $\vec{N}$ is similar over $\mathbb{C}$ to $\myvec{0&0\\1&0}$.\\
&\\
\hline
\caption{Solution summary}
\label{eq:solutions/6/2/11/table:1}
\end{longtable}

\end{enumerate}



\section{Vector Spaces}
\subsection{Vector Spaces}
\renewcommand{\theequation}{\theenumi}
\renewcommand{\thefigure}{\theenumi}
\begin{enumerate}[label=\thesubsection.\arabic*.,ref=\thesubsection.\theenumi]
\numberwithin{equation}{enumi}
\numberwithin{figure}{enumi}
%
\item Let V be the set of all pairs (x,y) of real numbers and let F be the field of real numbers. Define 
\begin{align}
(x,y)+(x_1,y_1)=(x+x_1,y+y_1)\\
c(x,y)=(cx,y)
\end{align}
Is V with these operations , a vector space over the field of real numbers ?
%
\\
\solution
See Table \ref{eq:solutions/6/2/11/table:1}
\onecolumn
\begin{longtable}{|l|l|}
\hline
\multirow{3}{*}{} & \\
Statement&Solution\\
\hline
&\parbox{6cm}{\begin{align}
\mbox{Let }\vec{N}&=\myvec{a&b\\c&d}\\
\mbox{Since }\vec{N}^2&=0
\end{align}}\\
&If $\myvec{a\\c},\myvec{b\\d}$ are linearly independent then $\vec{N}$ is diagonalizable to $\myvec{0&0\\0&0}$.\\
&\parbox{6cm}{\begin{align}
\mbox{If }\vec{P}\vec{N}\vec{P}^{-1}=0\\
\mbox{then }\vec{N}=\vec{P}^{-1}\vec{0}\vec{P}=0
\end{align}}\\
Proof that&So in this case $\vec{N}$ itself is the zero matrix.\\
$\vec{N}=0$&This contradicts the assumption that $\myvec{a\\c},\myvec{b\\d}$ are linearly independent.\\
&$\therefore$ we can assume that $\myvec{a\\c},\myvec{b\\d}$ are linearly dependent if both are\\
&equal to the zero vector\\
&\parbox{6cm}
{\begin{align}
   \mbox{then } \vec{N} &= 0.
\end{align}}\\
%\hline
%\pagebreak
\hline
&\\
&Therefore we can assume at least one vector is non-zero.\\
Assuming $\myvec{b\\d}$ as&Therefore
$\vec{N}=\myvec{a&0\\c&0}$\\
the zero vector&\\
&\parbox{6cm}{\begin{align}
\mbox{So }\vec{N}^2&=0\\
\implies a^2&=0\\
\therefore a&=0\\
\mbox{Thus }\vec{N}&=\myvec{a&0\\c&0}
\end{align}}\\
&In this case $\vec{N}$ is similar to $\vec{N}=\myvec{0&0\\1&0}$ via the matrix $\vec{P}=\myvec{c&0\\0&1}$\\
&\\
\hline
&\\
Assuming $\myvec{a\\c}$ as&Therefore
$\vec{N}=\myvec{0&b\\0&d}$\\
the zero vector&\\
&\parbox{6cm}{\begin{align}
\mbox{Then }\vec{N}^2&=0\\
\implies d^2&=0\\
\therefore d&=0\\
\mbox{Thus }\vec{N}&=\myvec{0&b\\0&0}
\end{align}}\\
&In this case $\vec{N}$ is similar to $\vec{N}=\myvec{0&0\\b&0}$ via the matrix $\vec{P}=\myvec{0&1\\1&0}$,\\
&which is similar to $\myvec{0&0\\1&0}$ as above.\\
&\\
\hline
&\\
Hence& we can assume neither $\myvec{a\\c}$ or $\myvec{b\\d}$ is the zero vector.\\
&\\
%\hline
%\pagebreak
\hline
&\\
Consequences of &Since they are linearly dependent we can assume,\\
linear&\\
independence&\\
&\parbox{6cm}{\begin{align}
\myvec{b\\d}&=x\myvec{a\\c}\\
\therefore \vec{N}&=\myvec{a&ax\\c&cx}\\
\therefore \vec{N}^2&=0\\
\implies a(a+cx)&=0\\
c(a+cx)&=0\\
ax(a+cx)&=0\\
cx(a+cx)&=0
\end{align}}\\
\hline
&\\
Proof that $\vec{N}$ is&We know that at least one of a or c is not zero.\\
similar over $\mathbb{C}$ to&If a = 0 then c $\neq$ 0, it must be that x = 0.\\
$\myvec{0&0\\1&0}$&So in this case $\vec{N}=\myvec{0&0\\c&0}$ which is similar to $\myvec{0&0\\1&0}$ as before.\\
&\\
&\parbox{6cm}{\begin{align}
\mbox{If } a &\neq0\\
\mbox{then }x &\neq0\\
\mbox{else }a(a+cx)&=0\\
\implies a&=0\\
\mbox{Thus }a+cx&=0\\
\mbox{Hence }\vec{N}&=\myvec{a&ax\\ \frac{-a}{x}&-a}
\end{align}}\\
 &This is similar to $\myvec{a&a\\-a&-a}$ via $\vec{P}=\myvec{\sqrt{x}&0\\0&\frac{1}{\sqrt{x}}}$.\\
 &And $\myvec{a&a\\-a&-a}$ is similar to $\myvec{0&0\\-a&0}$ via $\vec{P}=\myvec{-1&-1\\1&0}$\\
 &And this finally is similar to $\myvec{0&0\\1&0}$ as before.\\
 &\\
\hline
&\\
Conclusion &Thus either $\vec{N}$ = 0 or $\vec{N}$ is similar over $\mathbb{C}$ to $\myvec{0&0\\1&0}$.\\
&\\
\hline
\caption{Solution summary}
\label{eq:solutions/6/2/11/table:1}
\end{longtable}

%
\item Let $\vec{V}$ be the  set of pairs $(x,y)$ of real numbers and let $\vec{F}$ be the field of real numbers. Define
\begin{align}
    (x,y)+(x_1,y_1) &= (x+x_1,0) \label{eq:solutions/2/1/7/eq:eq1}\\
    c(x,y) &= (cx,0)
\end{align}
Is $\vec{V}$, with these operations, a vector space?
\\
\solution
See Table \ref{eq:solutions/6/2/11/table:1}
\onecolumn
\begin{longtable}{|l|l|}
\hline
\multirow{3}{*}{} & \\
Statement&Solution\\
\hline
&\parbox{6cm}{\begin{align}
\mbox{Let }\vec{N}&=\myvec{a&b\\c&d}\\
\mbox{Since }\vec{N}^2&=0
\end{align}}\\
&If $\myvec{a\\c},\myvec{b\\d}$ are linearly independent then $\vec{N}$ is diagonalizable to $\myvec{0&0\\0&0}$.\\
&\parbox{6cm}{\begin{align}
\mbox{If }\vec{P}\vec{N}\vec{P}^{-1}=0\\
\mbox{then }\vec{N}=\vec{P}^{-1}\vec{0}\vec{P}=0
\end{align}}\\
Proof that&So in this case $\vec{N}$ itself is the zero matrix.\\
$\vec{N}=0$&This contradicts the assumption that $\myvec{a\\c},\myvec{b\\d}$ are linearly independent.\\
&$\therefore$ we can assume that $\myvec{a\\c},\myvec{b\\d}$ are linearly dependent if both are\\
&equal to the zero vector\\
&\parbox{6cm}
{\begin{align}
   \mbox{then } \vec{N} &= 0.
\end{align}}\\
%\hline
%\pagebreak
\hline
&\\
&Therefore we can assume at least one vector is non-zero.\\
Assuming $\myvec{b\\d}$ as&Therefore
$\vec{N}=\myvec{a&0\\c&0}$\\
the zero vector&\\
&\parbox{6cm}{\begin{align}
\mbox{So }\vec{N}^2&=0\\
\implies a^2&=0\\
\therefore a&=0\\
\mbox{Thus }\vec{N}&=\myvec{a&0\\c&0}
\end{align}}\\
&In this case $\vec{N}$ is similar to $\vec{N}=\myvec{0&0\\1&0}$ via the matrix $\vec{P}=\myvec{c&0\\0&1}$\\
&\\
\hline
&\\
Assuming $\myvec{a\\c}$ as&Therefore
$\vec{N}=\myvec{0&b\\0&d}$\\
the zero vector&\\
&\parbox{6cm}{\begin{align}
\mbox{Then }\vec{N}^2&=0\\
\implies d^2&=0\\
\therefore d&=0\\
\mbox{Thus }\vec{N}&=\myvec{0&b\\0&0}
\end{align}}\\
&In this case $\vec{N}$ is similar to $\vec{N}=\myvec{0&0\\b&0}$ via the matrix $\vec{P}=\myvec{0&1\\1&0}$,\\
&which is similar to $\myvec{0&0\\1&0}$ as above.\\
&\\
\hline
&\\
Hence& we can assume neither $\myvec{a\\c}$ or $\myvec{b\\d}$ is the zero vector.\\
&\\
%\hline
%\pagebreak
\hline
&\\
Consequences of &Since they are linearly dependent we can assume,\\
linear&\\
independence&\\
&\parbox{6cm}{\begin{align}
\myvec{b\\d}&=x\myvec{a\\c}\\
\therefore \vec{N}&=\myvec{a&ax\\c&cx}\\
\therefore \vec{N}^2&=0\\
\implies a(a+cx)&=0\\
c(a+cx)&=0\\
ax(a+cx)&=0\\
cx(a+cx)&=0
\end{align}}\\
\hline
&\\
Proof that $\vec{N}$ is&We know that at least one of a or c is not zero.\\
similar over $\mathbb{C}$ to&If a = 0 then c $\neq$ 0, it must be that x = 0.\\
$\myvec{0&0\\1&0}$&So in this case $\vec{N}=\myvec{0&0\\c&0}$ which is similar to $\myvec{0&0\\1&0}$ as before.\\
&\\
&\parbox{6cm}{\begin{align}
\mbox{If } a &\neq0\\
\mbox{then }x &\neq0\\
\mbox{else }a(a+cx)&=0\\
\implies a&=0\\
\mbox{Thus }a+cx&=0\\
\mbox{Hence }\vec{N}&=\myvec{a&ax\\ \frac{-a}{x}&-a}
\end{align}}\\
 &This is similar to $\myvec{a&a\\-a&-a}$ via $\vec{P}=\myvec{\sqrt{x}&0\\0&\frac{1}{\sqrt{x}}}$.\\
 &And $\myvec{a&a\\-a&-a}$ is similar to $\myvec{0&0\\-a&0}$ via $\vec{P}=\myvec{-1&-1\\1&0}$\\
 &And this finally is similar to $\myvec{0&0\\1&0}$ as before.\\
 &\\
\hline
&\\
Conclusion &Thus either $\vec{N}$ = 0 or $\vec{N}$ is similar over $\mathbb{C}$ to $\myvec{0&0\\1&0}$.\\
&\\
\hline
\caption{Solution summary}
\label{eq:solutions/6/2/11/table:1}
\end{longtable}

\end{enumerate}



\subsection{Subspaces}
\renewcommand{\theequation}{\theenumi}
\renewcommand{\thefigure}{\theenumi}
\begin{enumerate}[label=\thesubsection.\arabic*.,ref=\thesubsection.\theenumi]
\numberwithin{equation}{enumi}
\numberwithin{figure}{enumi}
%
\item Which of the following set of vectors
\begin{align}\alpha=(a_1,a_2...........a_n)\nonumber\end{align} in $\vec{R^n}$ are subspace of $\vec{R^n}$ $(n\geq3)$?\\
\begin{enumerate}
\item All $\alpha$ such that $a_1\geq0$\\
\item All $\alpha$ such that $a_1+3a_2=a_3$\\
\item All $\alpha$ such that$a_2=a_1^{2}$\\
\item All $\alpha$ such that $a_1a_2=0$\\
\item All $\alpha$ such that $a_2$ is rational
\end{enumerate}
%
\solution
See Table \ref{eq:solutions/6/2/11/table:1}
\onecolumn
\begin{longtable}{|l|l|}
\hline
\multirow{3}{*}{} & \\
Statement&Solution\\
\hline
&\parbox{6cm}{\begin{align}
\mbox{Let }\vec{N}&=\myvec{a&b\\c&d}\\
\mbox{Since }\vec{N}^2&=0
\end{align}}\\
&If $\myvec{a\\c},\myvec{b\\d}$ are linearly independent then $\vec{N}$ is diagonalizable to $\myvec{0&0\\0&0}$.\\
&\parbox{6cm}{\begin{align}
\mbox{If }\vec{P}\vec{N}\vec{P}^{-1}=0\\
\mbox{then }\vec{N}=\vec{P}^{-1}\vec{0}\vec{P}=0
\end{align}}\\
Proof that&So in this case $\vec{N}$ itself is the zero matrix.\\
$\vec{N}=0$&This contradicts the assumption that $\myvec{a\\c},\myvec{b\\d}$ are linearly independent.\\
&$\therefore$ we can assume that $\myvec{a\\c},\myvec{b\\d}$ are linearly dependent if both are\\
&equal to the zero vector\\
&\parbox{6cm}
{\begin{align}
   \mbox{then } \vec{N} &= 0.
\end{align}}\\
%\hline
%\pagebreak
\hline
&\\
&Therefore we can assume at least one vector is non-zero.\\
Assuming $\myvec{b\\d}$ as&Therefore
$\vec{N}=\myvec{a&0\\c&0}$\\
the zero vector&\\
&\parbox{6cm}{\begin{align}
\mbox{So }\vec{N}^2&=0\\
\implies a^2&=0\\
\therefore a&=0\\
\mbox{Thus }\vec{N}&=\myvec{a&0\\c&0}
\end{align}}\\
&In this case $\vec{N}$ is similar to $\vec{N}=\myvec{0&0\\1&0}$ via the matrix $\vec{P}=\myvec{c&0\\0&1}$\\
&\\
\hline
&\\
Assuming $\myvec{a\\c}$ as&Therefore
$\vec{N}=\myvec{0&b\\0&d}$\\
the zero vector&\\
&\parbox{6cm}{\begin{align}
\mbox{Then }\vec{N}^2&=0\\
\implies d^2&=0\\
\therefore d&=0\\
\mbox{Thus }\vec{N}&=\myvec{0&b\\0&0}
\end{align}}\\
&In this case $\vec{N}$ is similar to $\vec{N}=\myvec{0&0\\b&0}$ via the matrix $\vec{P}=\myvec{0&1\\1&0}$,\\
&which is similar to $\myvec{0&0\\1&0}$ as above.\\
&\\
\hline
&\\
Hence& we can assume neither $\myvec{a\\c}$ or $\myvec{b\\d}$ is the zero vector.\\
&\\
%\hline
%\pagebreak
\hline
&\\
Consequences of &Since they are linearly dependent we can assume,\\
linear&\\
independence&\\
&\parbox{6cm}{\begin{align}
\myvec{b\\d}&=x\myvec{a\\c}\\
\therefore \vec{N}&=\myvec{a&ax\\c&cx}\\
\therefore \vec{N}^2&=0\\
\implies a(a+cx)&=0\\
c(a+cx)&=0\\
ax(a+cx)&=0\\
cx(a+cx)&=0
\end{align}}\\
\hline
&\\
Proof that $\vec{N}$ is&We know that at least one of a or c is not zero.\\
similar over $\mathbb{C}$ to&If a = 0 then c $\neq$ 0, it must be that x = 0.\\
$\myvec{0&0\\1&0}$&So in this case $\vec{N}=\myvec{0&0\\c&0}$ which is similar to $\myvec{0&0\\1&0}$ as before.\\
&\\
&\parbox{6cm}{\begin{align}
\mbox{If } a &\neq0\\
\mbox{then }x &\neq0\\
\mbox{else }a(a+cx)&=0\\
\implies a&=0\\
\mbox{Thus }a+cx&=0\\
\mbox{Hence }\vec{N}&=\myvec{a&ax\\ \frac{-a}{x}&-a}
\end{align}}\\
 &This is similar to $\myvec{a&a\\-a&-a}$ via $\vec{P}=\myvec{\sqrt{x}&0\\0&\frac{1}{\sqrt{x}}}$.\\
 &And $\myvec{a&a\\-a&-a}$ is similar to $\myvec{0&0\\-a&0}$ via $\vec{P}=\myvec{-1&-1\\1&0}$\\
 &And this finally is similar to $\myvec{0&0\\1&0}$ as before.\\
 &\\
\hline
&\\
Conclusion &Thus either $\vec{N}$ = 0 or $\vec{N}$ is similar over $\mathbb{C}$ to $\myvec{0&0\\1&0}$.\\
&\\
\hline
\caption{Solution summary}
\label{eq:solutions/6/2/11/table:1}
\end{longtable}

\item Let $\vec{V}$ be the (real) vector space of all functions $f$
from $\vec{R}$ into $\vec{R}$. 
%
\begin{enumerate}
\item Is $f(0) = f(1)$ a
subspace of $\vec{V}$ ?
\\
\solution
See Table \ref{eq:solutions/6/2/11/table:1}
\onecolumn
\begin{longtable}{|l|l|}
\hline
\multirow{3}{*}{} & \\
Statement&Solution\\
\hline
&\parbox{6cm}{\begin{align}
\mbox{Let }\vec{N}&=\myvec{a&b\\c&d}\\
\mbox{Since }\vec{N}^2&=0
\end{align}}\\
&If $\myvec{a\\c},\myvec{b\\d}$ are linearly independent then $\vec{N}$ is diagonalizable to $\myvec{0&0\\0&0}$.\\
&\parbox{6cm}{\begin{align}
\mbox{If }\vec{P}\vec{N}\vec{P}^{-1}=0\\
\mbox{then }\vec{N}=\vec{P}^{-1}\vec{0}\vec{P}=0
\end{align}}\\
Proof that&So in this case $\vec{N}$ itself is the zero matrix.\\
$\vec{N}=0$&This contradicts the assumption that $\myvec{a\\c},\myvec{b\\d}$ are linearly independent.\\
&$\therefore$ we can assume that $\myvec{a\\c},\myvec{b\\d}$ are linearly dependent if both are\\
&equal to the zero vector\\
&\parbox{6cm}
{\begin{align}
   \mbox{then } \vec{N} &= 0.
\end{align}}\\
%\hline
%\pagebreak
\hline
&\\
&Therefore we can assume at least one vector is non-zero.\\
Assuming $\myvec{b\\d}$ as&Therefore
$\vec{N}=\myvec{a&0\\c&0}$\\
the zero vector&\\
&\parbox{6cm}{\begin{align}
\mbox{So }\vec{N}^2&=0\\
\implies a^2&=0\\
\therefore a&=0\\
\mbox{Thus }\vec{N}&=\myvec{a&0\\c&0}
\end{align}}\\
&In this case $\vec{N}$ is similar to $\vec{N}=\myvec{0&0\\1&0}$ via the matrix $\vec{P}=\myvec{c&0\\0&1}$\\
&\\
\hline
&\\
Assuming $\myvec{a\\c}$ as&Therefore
$\vec{N}=\myvec{0&b\\0&d}$\\
the zero vector&\\
&\parbox{6cm}{\begin{align}
\mbox{Then }\vec{N}^2&=0\\
\implies d^2&=0\\
\therefore d&=0\\
\mbox{Thus }\vec{N}&=\myvec{0&b\\0&0}
\end{align}}\\
&In this case $\vec{N}$ is similar to $\vec{N}=\myvec{0&0\\b&0}$ via the matrix $\vec{P}=\myvec{0&1\\1&0}$,\\
&which is similar to $\myvec{0&0\\1&0}$ as above.\\
&\\
\hline
&\\
Hence& we can assume neither $\myvec{a\\c}$ or $\myvec{b\\d}$ is the zero vector.\\
&\\
%\hline
%\pagebreak
\hline
&\\
Consequences of &Since they are linearly dependent we can assume,\\
linear&\\
independence&\\
&\parbox{6cm}{\begin{align}
\myvec{b\\d}&=x\myvec{a\\c}\\
\therefore \vec{N}&=\myvec{a&ax\\c&cx}\\
\therefore \vec{N}^2&=0\\
\implies a(a+cx)&=0\\
c(a+cx)&=0\\
ax(a+cx)&=0\\
cx(a+cx)&=0
\end{align}}\\
\hline
&\\
Proof that $\vec{N}$ is&We know that at least one of a or c is not zero.\\
similar over $\mathbb{C}$ to&If a = 0 then c $\neq$ 0, it must be that x = 0.\\
$\myvec{0&0\\1&0}$&So in this case $\vec{N}=\myvec{0&0\\c&0}$ which is similar to $\myvec{0&0\\1&0}$ as before.\\
&\\
&\parbox{6cm}{\begin{align}
\mbox{If } a &\neq0\\
\mbox{then }x &\neq0\\
\mbox{else }a(a+cx)&=0\\
\implies a&=0\\
\mbox{Thus }a+cx&=0\\
\mbox{Hence }\vec{N}&=\myvec{a&ax\\ \frac{-a}{x}&-a}
\end{align}}\\
 &This is similar to $\myvec{a&a\\-a&-a}$ via $\vec{P}=\myvec{\sqrt{x}&0\\0&\frac{1}{\sqrt{x}}}$.\\
 &And $\myvec{a&a\\-a&-a}$ is similar to $\myvec{0&0\\-a&0}$ via $\vec{P}=\myvec{-1&-1\\1&0}$\\
 &And this finally is similar to $\myvec{0&0\\1&0}$ as before.\\
 &\\
\hline
&\\
Conclusion &Thus either $\vec{N}$ = 0 or $\vec{N}$ is similar over $\mathbb{C}$ to $\myvec{0&0\\1&0}$.\\
&\\
\hline
\caption{Solution summary}
\label{eq:solutions/6/2/11/table:1}
\end{longtable}

\item Let $\vec{V}$ be the (real) vector space of all functions $f$
from $\vec{R}$ into $\vec{R}$. Is $f(3)=1+f(-5)$ a
subspace of $\vec{V}$ 
%
\\
\solution
See Table \ref{eq:solutions/6/2/11/table:1}
\onecolumn
\begin{longtable}{|l|l|}
\hline
\multirow{3}{*}{} & \\
Statement&Solution\\
\hline
&\parbox{6cm}{\begin{align}
\mbox{Let }\vec{N}&=\myvec{a&b\\c&d}\\
\mbox{Since }\vec{N}^2&=0
\end{align}}\\
&If $\myvec{a\\c},\myvec{b\\d}$ are linearly independent then $\vec{N}$ is diagonalizable to $\myvec{0&0\\0&0}$.\\
&\parbox{6cm}{\begin{align}
\mbox{If }\vec{P}\vec{N}\vec{P}^{-1}=0\\
\mbox{then }\vec{N}=\vec{P}^{-1}\vec{0}\vec{P}=0
\end{align}}\\
Proof that&So in this case $\vec{N}$ itself is the zero matrix.\\
$\vec{N}=0$&This contradicts the assumption that $\myvec{a\\c},\myvec{b\\d}$ are linearly independent.\\
&$\therefore$ we can assume that $\myvec{a\\c},\myvec{b\\d}$ are linearly dependent if both are\\
&equal to the zero vector\\
&\parbox{6cm}
{\begin{align}
   \mbox{then } \vec{N} &= 0.
\end{align}}\\
%\hline
%\pagebreak
\hline
&\\
&Therefore we can assume at least one vector is non-zero.\\
Assuming $\myvec{b\\d}$ as&Therefore
$\vec{N}=\myvec{a&0\\c&0}$\\
the zero vector&\\
&\parbox{6cm}{\begin{align}
\mbox{So }\vec{N}^2&=0\\
\implies a^2&=0\\
\therefore a&=0\\
\mbox{Thus }\vec{N}&=\myvec{a&0\\c&0}
\end{align}}\\
&In this case $\vec{N}$ is similar to $\vec{N}=\myvec{0&0\\1&0}$ via the matrix $\vec{P}=\myvec{c&0\\0&1}$\\
&\\
\hline
&\\
Assuming $\myvec{a\\c}$ as&Therefore
$\vec{N}=\myvec{0&b\\0&d}$\\
the zero vector&\\
&\parbox{6cm}{\begin{align}
\mbox{Then }\vec{N}^2&=0\\
\implies d^2&=0\\
\therefore d&=0\\
\mbox{Thus }\vec{N}&=\myvec{0&b\\0&0}
\end{align}}\\
&In this case $\vec{N}$ is similar to $\vec{N}=\myvec{0&0\\b&0}$ via the matrix $\vec{P}=\myvec{0&1\\1&0}$,\\
&which is similar to $\myvec{0&0\\1&0}$ as above.\\
&\\
\hline
&\\
Hence& we can assume neither $\myvec{a\\c}$ or $\myvec{b\\d}$ is the zero vector.\\
&\\
%\hline
%\pagebreak
\hline
&\\
Consequences of &Since they are linearly dependent we can assume,\\
linear&\\
independence&\\
&\parbox{6cm}{\begin{align}
\myvec{b\\d}&=x\myvec{a\\c}\\
\therefore \vec{N}&=\myvec{a&ax\\c&cx}\\
\therefore \vec{N}^2&=0\\
\implies a(a+cx)&=0\\
c(a+cx)&=0\\
ax(a+cx)&=0\\
cx(a+cx)&=0
\end{align}}\\
\hline
&\\
Proof that $\vec{N}$ is&We know that at least one of a or c is not zero.\\
similar over $\mathbb{C}$ to&If a = 0 then c $\neq$ 0, it must be that x = 0.\\
$\myvec{0&0\\1&0}$&So in this case $\vec{N}=\myvec{0&0\\c&0}$ which is similar to $\myvec{0&0\\1&0}$ as before.\\
&\\
&\parbox{6cm}{\begin{align}
\mbox{If } a &\neq0\\
\mbox{then }x &\neq0\\
\mbox{else }a(a+cx)&=0\\
\implies a&=0\\
\mbox{Thus }a+cx&=0\\
\mbox{Hence }\vec{N}&=\myvec{a&ax\\ \frac{-a}{x}&-a}
\end{align}}\\
 &This is similar to $\myvec{a&a\\-a&-a}$ via $\vec{P}=\myvec{\sqrt{x}&0\\0&\frac{1}{\sqrt{x}}}$.\\
 &And $\myvec{a&a\\-a&-a}$ is similar to $\myvec{0&0\\-a&0}$ via $\vec{P}=\myvec{-1&-1\\1&0}$\\
 &And this finally is similar to $\myvec{0&0\\1&0}$ as before.\\
 &\\
\hline
&\\
Conclusion &Thus either $\vec{N}$ = 0 or $\vec{N}$ is similar over $\mathbb{C}$ to $\myvec{0&0\\1&0}$.\\
&\\
\hline
\caption{Solution summary}
\label{eq:solutions/6/2/11/table:1}
\end{longtable}

%
\item Let $\vec{V}$ be the (real) vector space of all functions $f$
from $\vec{R}$ into $\vec{R}$. Is $f(-1) = 0 $ a
subspace of $\vec{V}$ 
%
\\
\solution
See Table \ref{eq:solutions/6/2/11/table:1}
\onecolumn
\begin{longtable}{|l|l|}
\hline
\multirow{3}{*}{} & \\
Statement&Solution\\
\hline
&\parbox{6cm}{\begin{align}
\mbox{Let }\vec{N}&=\myvec{a&b\\c&d}\\
\mbox{Since }\vec{N}^2&=0
\end{align}}\\
&If $\myvec{a\\c},\myvec{b\\d}$ are linearly independent then $\vec{N}$ is diagonalizable to $\myvec{0&0\\0&0}$.\\
&\parbox{6cm}{\begin{align}
\mbox{If }\vec{P}\vec{N}\vec{P}^{-1}=0\\
\mbox{then }\vec{N}=\vec{P}^{-1}\vec{0}\vec{P}=0
\end{align}}\\
Proof that&So in this case $\vec{N}$ itself is the zero matrix.\\
$\vec{N}=0$&This contradicts the assumption that $\myvec{a\\c},\myvec{b\\d}$ are linearly independent.\\
&$\therefore$ we can assume that $\myvec{a\\c},\myvec{b\\d}$ are linearly dependent if both are\\
&equal to the zero vector\\
&\parbox{6cm}
{\begin{align}
   \mbox{then } \vec{N} &= 0.
\end{align}}\\
%\hline
%\pagebreak
\hline
&\\
&Therefore we can assume at least one vector is non-zero.\\
Assuming $\myvec{b\\d}$ as&Therefore
$\vec{N}=\myvec{a&0\\c&0}$\\
the zero vector&\\
&\parbox{6cm}{\begin{align}
\mbox{So }\vec{N}^2&=0\\
\implies a^2&=0\\
\therefore a&=0\\
\mbox{Thus }\vec{N}&=\myvec{a&0\\c&0}
\end{align}}\\
&In this case $\vec{N}$ is similar to $\vec{N}=\myvec{0&0\\1&0}$ via the matrix $\vec{P}=\myvec{c&0\\0&1}$\\
&\\
\hline
&\\
Assuming $\myvec{a\\c}$ as&Therefore
$\vec{N}=\myvec{0&b\\0&d}$\\
the zero vector&\\
&\parbox{6cm}{\begin{align}
\mbox{Then }\vec{N}^2&=0\\
\implies d^2&=0\\
\therefore d&=0\\
\mbox{Thus }\vec{N}&=\myvec{0&b\\0&0}
\end{align}}\\
&In this case $\vec{N}$ is similar to $\vec{N}=\myvec{0&0\\b&0}$ via the matrix $\vec{P}=\myvec{0&1\\1&0}$,\\
&which is similar to $\myvec{0&0\\1&0}$ as above.\\
&\\
\hline
&\\
Hence& we can assume neither $\myvec{a\\c}$ or $\myvec{b\\d}$ is the zero vector.\\
&\\
%\hline
%\pagebreak
\hline
&\\
Consequences of &Since they are linearly dependent we can assume,\\
linear&\\
independence&\\
&\parbox{6cm}{\begin{align}
\myvec{b\\d}&=x\myvec{a\\c}\\
\therefore \vec{N}&=\myvec{a&ax\\c&cx}\\
\therefore \vec{N}^2&=0\\
\implies a(a+cx)&=0\\
c(a+cx)&=0\\
ax(a+cx)&=0\\
cx(a+cx)&=0
\end{align}}\\
\hline
&\\
Proof that $\vec{N}$ is&We know that at least one of a or c is not zero.\\
similar over $\mathbb{C}$ to&If a = 0 then c $\neq$ 0, it must be that x = 0.\\
$\myvec{0&0\\1&0}$&So in this case $\vec{N}=\myvec{0&0\\c&0}$ which is similar to $\myvec{0&0\\1&0}$ as before.\\
&\\
&\parbox{6cm}{\begin{align}
\mbox{If } a &\neq0\\
\mbox{then }x &\neq0\\
\mbox{else }a(a+cx)&=0\\
\implies a&=0\\
\mbox{Thus }a+cx&=0\\
\mbox{Hence }\vec{N}&=\myvec{a&ax\\ \frac{-a}{x}&-a}
\end{align}}\\
 &This is similar to $\myvec{a&a\\-a&-a}$ via $\vec{P}=\myvec{\sqrt{x}&0\\0&\frac{1}{\sqrt{x}}}$.\\
 &And $\myvec{a&a\\-a&-a}$ is similar to $\myvec{0&0\\-a&0}$ via $\vec{P}=\myvec{-1&-1\\1&0}$\\
 &And this finally is similar to $\myvec{0&0\\1&0}$ as before.\\
 &\\
\hline
&\\
Conclusion &Thus either $\vec{N}$ = 0 or $\vec{N}$ is similar over $\mathbb{C}$ to $\myvec{0&0\\1&0}$.\\
&\\
\hline
\caption{Solution summary}
\label{eq:solutions/6/2/11/table:1}
\end{longtable}

\end{enumerate}
\item Is the vector $\myvec{3\\-1\\0\\-1}$ in the subspace of $\vec{R}^{4}$ spanned by the vectors $\myvec{2\\-1\\3\\2}$, $\myvec{-1\\1\\1\\-3}$ and $\myvec{1\\1\\9\\-5}$ ?
%
\\
\solution
See Table \ref{eq:solutions/6/2/11/table:1}
\onecolumn
\begin{longtable}{|l|l|}
\hline
\multirow{3}{*}{} & \\
Statement&Solution\\
\hline
&\parbox{6cm}{\begin{align}
\mbox{Let }\vec{N}&=\myvec{a&b\\c&d}\\
\mbox{Since }\vec{N}^2&=0
\end{align}}\\
&If $\myvec{a\\c},\myvec{b\\d}$ are linearly independent then $\vec{N}$ is diagonalizable to $\myvec{0&0\\0&0}$.\\
&\parbox{6cm}{\begin{align}
\mbox{If }\vec{P}\vec{N}\vec{P}^{-1}=0\\
\mbox{then }\vec{N}=\vec{P}^{-1}\vec{0}\vec{P}=0
\end{align}}\\
Proof that&So in this case $\vec{N}$ itself is the zero matrix.\\
$\vec{N}=0$&This contradicts the assumption that $\myvec{a\\c},\myvec{b\\d}$ are linearly independent.\\
&$\therefore$ we can assume that $\myvec{a\\c},\myvec{b\\d}$ are linearly dependent if both are\\
&equal to the zero vector\\
&\parbox{6cm}
{\begin{align}
   \mbox{then } \vec{N} &= 0.
\end{align}}\\
%\hline
%\pagebreak
\hline
&\\
&Therefore we can assume at least one vector is non-zero.\\
Assuming $\myvec{b\\d}$ as&Therefore
$\vec{N}=\myvec{a&0\\c&0}$\\
the zero vector&\\
&\parbox{6cm}{\begin{align}
\mbox{So }\vec{N}^2&=0\\
\implies a^2&=0\\
\therefore a&=0\\
\mbox{Thus }\vec{N}&=\myvec{a&0\\c&0}
\end{align}}\\
&In this case $\vec{N}$ is similar to $\vec{N}=\myvec{0&0\\1&0}$ via the matrix $\vec{P}=\myvec{c&0\\0&1}$\\
&\\
\hline
&\\
Assuming $\myvec{a\\c}$ as&Therefore
$\vec{N}=\myvec{0&b\\0&d}$\\
the zero vector&\\
&\parbox{6cm}{\begin{align}
\mbox{Then }\vec{N}^2&=0\\
\implies d^2&=0\\
\therefore d&=0\\
\mbox{Thus }\vec{N}&=\myvec{0&b\\0&0}
\end{align}}\\
&In this case $\vec{N}$ is similar to $\vec{N}=\myvec{0&0\\b&0}$ via the matrix $\vec{P}=\myvec{0&1\\1&0}$,\\
&which is similar to $\myvec{0&0\\1&0}$ as above.\\
&\\
\hline
&\\
Hence& we can assume neither $\myvec{a\\c}$ or $\myvec{b\\d}$ is the zero vector.\\
&\\
%\hline
%\pagebreak
\hline
&\\
Consequences of &Since they are linearly dependent we can assume,\\
linear&\\
independence&\\
&\parbox{6cm}{\begin{align}
\myvec{b\\d}&=x\myvec{a\\c}\\
\therefore \vec{N}&=\myvec{a&ax\\c&cx}\\
\therefore \vec{N}^2&=0\\
\implies a(a+cx)&=0\\
c(a+cx)&=0\\
ax(a+cx)&=0\\
cx(a+cx)&=0
\end{align}}\\
\hline
&\\
Proof that $\vec{N}$ is&We know that at least one of a or c is not zero.\\
similar over $\mathbb{C}$ to&If a = 0 then c $\neq$ 0, it must be that x = 0.\\
$\myvec{0&0\\1&0}$&So in this case $\vec{N}=\myvec{0&0\\c&0}$ which is similar to $\myvec{0&0\\1&0}$ as before.\\
&\\
&\parbox{6cm}{\begin{align}
\mbox{If } a &\neq0\\
\mbox{then }x &\neq0\\
\mbox{else }a(a+cx)&=0\\
\implies a&=0\\
\mbox{Thus }a+cx&=0\\
\mbox{Hence }\vec{N}&=\myvec{a&ax\\ \frac{-a}{x}&-a}
\end{align}}\\
 &This is similar to $\myvec{a&a\\-a&-a}$ via $\vec{P}=\myvec{\sqrt{x}&0\\0&\frac{1}{\sqrt{x}}}$.\\
 &And $\myvec{a&a\\-a&-a}$ is similar to $\myvec{0&0\\-a&0}$ via $\vec{P}=\myvec{-1&-1\\1&0}$\\
 &And this finally is similar to $\myvec{0&0\\1&0}$ as before.\\
 &\\
\hline
&\\
Conclusion &Thus either $\vec{N}$ = 0 or $\vec{N}$ is similar over $\mathbb{C}$ to $\myvec{0&0\\1&0}$.\\
&\\
\hline
\caption{Solution summary}
\label{eq:solutions/6/2/11/table:1}
\end{longtable}

%
\item  Let $\vec{W}$ be the set of all $(x_1,x_2,x_3,x_4,x_5)$ in $\mathbb{R}^5$ which satisfy
 \begin{align}
 2x_1-x_2+\frac{4}{3}x_3-x_4=0\\x_1+\frac{2}{3}x_3-x_5=0\\9x_1-3x_2+6x_3-3x_4-3x_5=0
 \end{align}
 Find a finite set of vectors which spans $\vec{W}$.

\solution
See Table \ref{eq:solutions/6/2/11/table:1}
\onecolumn
\begin{longtable}{|l|l|}
\hline
\multirow{3}{*}{} & \\
Statement&Solution\\
\hline
&\parbox{6cm}{\begin{align}
\mbox{Let }\vec{N}&=\myvec{a&b\\c&d}\\
\mbox{Since }\vec{N}^2&=0
\end{align}}\\
&If $\myvec{a\\c},\myvec{b\\d}$ are linearly independent then $\vec{N}$ is diagonalizable to $\myvec{0&0\\0&0}$.\\
&\parbox{6cm}{\begin{align}
\mbox{If }\vec{P}\vec{N}\vec{P}^{-1}=0\\
\mbox{then }\vec{N}=\vec{P}^{-1}\vec{0}\vec{P}=0
\end{align}}\\
Proof that&So in this case $\vec{N}$ itself is the zero matrix.\\
$\vec{N}=0$&This contradicts the assumption that $\myvec{a\\c},\myvec{b\\d}$ are linearly independent.\\
&$\therefore$ we can assume that $\myvec{a\\c},\myvec{b\\d}$ are linearly dependent if both are\\
&equal to the zero vector\\
&\parbox{6cm}
{\begin{align}
   \mbox{then } \vec{N} &= 0.
\end{align}}\\
%\hline
%\pagebreak
\hline
&\\
&Therefore we can assume at least one vector is non-zero.\\
Assuming $\myvec{b\\d}$ as&Therefore
$\vec{N}=\myvec{a&0\\c&0}$\\
the zero vector&\\
&\parbox{6cm}{\begin{align}
\mbox{So }\vec{N}^2&=0\\
\implies a^2&=0\\
\therefore a&=0\\
\mbox{Thus }\vec{N}&=\myvec{a&0\\c&0}
\end{align}}\\
&In this case $\vec{N}$ is similar to $\vec{N}=\myvec{0&0\\1&0}$ via the matrix $\vec{P}=\myvec{c&0\\0&1}$\\
&\\
\hline
&\\
Assuming $\myvec{a\\c}$ as&Therefore
$\vec{N}=\myvec{0&b\\0&d}$\\
the zero vector&\\
&\parbox{6cm}{\begin{align}
\mbox{Then }\vec{N}^2&=0\\
\implies d^2&=0\\
\therefore d&=0\\
\mbox{Thus }\vec{N}&=\myvec{0&b\\0&0}
\end{align}}\\
&In this case $\vec{N}$ is similar to $\vec{N}=\myvec{0&0\\b&0}$ via the matrix $\vec{P}=\myvec{0&1\\1&0}$,\\
&which is similar to $\myvec{0&0\\1&0}$ as above.\\
&\\
\hline
&\\
Hence& we can assume neither $\myvec{a\\c}$ or $\myvec{b\\d}$ is the zero vector.\\
&\\
%\hline
%\pagebreak
\hline
&\\
Consequences of &Since they are linearly dependent we can assume,\\
linear&\\
independence&\\
&\parbox{6cm}{\begin{align}
\myvec{b\\d}&=x\myvec{a\\c}\\
\therefore \vec{N}&=\myvec{a&ax\\c&cx}\\
\therefore \vec{N}^2&=0\\
\implies a(a+cx)&=0\\
c(a+cx)&=0\\
ax(a+cx)&=0\\
cx(a+cx)&=0
\end{align}}\\
\hline
&\\
Proof that $\vec{N}$ is&We know that at least one of a or c is not zero.\\
similar over $\mathbb{C}$ to&If a = 0 then c $\neq$ 0, it must be that x = 0.\\
$\myvec{0&0\\1&0}$&So in this case $\vec{N}=\myvec{0&0\\c&0}$ which is similar to $\myvec{0&0\\1&0}$ as before.\\
&\\
&\parbox{6cm}{\begin{align}
\mbox{If } a &\neq0\\
\mbox{then }x &\neq0\\
\mbox{else }a(a+cx)&=0\\
\implies a&=0\\
\mbox{Thus }a+cx&=0\\
\mbox{Hence }\vec{N}&=\myvec{a&ax\\ \frac{-a}{x}&-a}
\end{align}}\\
 &This is similar to $\myvec{a&a\\-a&-a}$ via $\vec{P}=\myvec{\sqrt{x}&0\\0&\frac{1}{\sqrt{x}}}$.\\
 &And $\myvec{a&a\\-a&-a}$ is similar to $\myvec{0&0\\-a&0}$ via $\vec{P}=\myvec{-1&-1\\1&0}$\\
 &And this finally is similar to $\myvec{0&0\\1&0}$ as before.\\
 &\\
\hline
&\\
Conclusion &Thus either $\vec{N}$ = 0 or $\vec{N}$ is similar over $\mathbb{C}$ to $\myvec{0&0\\1&0}$.\\
&\\
\hline
\caption{Solution summary}
\label{eq:solutions/6/2/11/table:1}
\end{longtable}

\item Let $\vec{F}$ be a field and let n be a positive integer (n$\geq$2). Let $\vec{V}$ be the vector space of all n$\times$n matrices over $\vec{F}$. Which of the following set of matrices $\vec{A}$ in $\vec{V}$ are subspaces of   $\vec{V}$?
\begin{enumerate}
\item all invertible $\vec{A}$;
\item all non-invertible $\vec{A}$;
\item all $\vec{A}$ such that $\vec{AB}=\vec{BA}$, where $\vec{B}$ is some fixed matrix in $\vec{V}$;
\item all $\vec{A}$ such that $\vec{A}^2 = \vec{A}$.
\end{enumerate}
%
\solution
See Table \ref{eq:solutions/6/2/11/table:1}
\onecolumn
\begin{longtable}{|l|l|}
\hline
\multirow{3}{*}{} & \\
Statement&Solution\\
\hline
&\parbox{6cm}{\begin{align}
\mbox{Let }\vec{N}&=\myvec{a&b\\c&d}\\
\mbox{Since }\vec{N}^2&=0
\end{align}}\\
&If $\myvec{a\\c},\myvec{b\\d}$ are linearly independent then $\vec{N}$ is diagonalizable to $\myvec{0&0\\0&0}$.\\
&\parbox{6cm}{\begin{align}
\mbox{If }\vec{P}\vec{N}\vec{P}^{-1}=0\\
\mbox{then }\vec{N}=\vec{P}^{-1}\vec{0}\vec{P}=0
\end{align}}\\
Proof that&So in this case $\vec{N}$ itself is the zero matrix.\\
$\vec{N}=0$&This contradicts the assumption that $\myvec{a\\c},\myvec{b\\d}$ are linearly independent.\\
&$\therefore$ we can assume that $\myvec{a\\c},\myvec{b\\d}$ are linearly dependent if both are\\
&equal to the zero vector\\
&\parbox{6cm}
{\begin{align}
   \mbox{then } \vec{N} &= 0.
\end{align}}\\
%\hline
%\pagebreak
\hline
&\\
&Therefore we can assume at least one vector is non-zero.\\
Assuming $\myvec{b\\d}$ as&Therefore
$\vec{N}=\myvec{a&0\\c&0}$\\
the zero vector&\\
&\parbox{6cm}{\begin{align}
\mbox{So }\vec{N}^2&=0\\
\implies a^2&=0\\
\therefore a&=0\\
\mbox{Thus }\vec{N}&=\myvec{a&0\\c&0}
\end{align}}\\
&In this case $\vec{N}$ is similar to $\vec{N}=\myvec{0&0\\1&0}$ via the matrix $\vec{P}=\myvec{c&0\\0&1}$\\
&\\
\hline
&\\
Assuming $\myvec{a\\c}$ as&Therefore
$\vec{N}=\myvec{0&b\\0&d}$\\
the zero vector&\\
&\parbox{6cm}{\begin{align}
\mbox{Then }\vec{N}^2&=0\\
\implies d^2&=0\\
\therefore d&=0\\
\mbox{Thus }\vec{N}&=\myvec{0&b\\0&0}
\end{align}}\\
&In this case $\vec{N}$ is similar to $\vec{N}=\myvec{0&0\\b&0}$ via the matrix $\vec{P}=\myvec{0&1\\1&0}$,\\
&which is similar to $\myvec{0&0\\1&0}$ as above.\\
&\\
\hline
&\\
Hence& we can assume neither $\myvec{a\\c}$ or $\myvec{b\\d}$ is the zero vector.\\
&\\
%\hline
%\pagebreak
\hline
&\\
Consequences of &Since they are linearly dependent we can assume,\\
linear&\\
independence&\\
&\parbox{6cm}{\begin{align}
\myvec{b\\d}&=x\myvec{a\\c}\\
\therefore \vec{N}&=\myvec{a&ax\\c&cx}\\
\therefore \vec{N}^2&=0\\
\implies a(a+cx)&=0\\
c(a+cx)&=0\\
ax(a+cx)&=0\\
cx(a+cx)&=0
\end{align}}\\
\hline
&\\
Proof that $\vec{N}$ is&We know that at least one of a or c is not zero.\\
similar over $\mathbb{C}$ to&If a = 0 then c $\neq$ 0, it must be that x = 0.\\
$\myvec{0&0\\1&0}$&So in this case $\vec{N}=\myvec{0&0\\c&0}$ which is similar to $\myvec{0&0\\1&0}$ as before.\\
&\\
&\parbox{6cm}{\begin{align}
\mbox{If } a &\neq0\\
\mbox{then }x &\neq0\\
\mbox{else }a(a+cx)&=0\\
\implies a&=0\\
\mbox{Thus }a+cx&=0\\
\mbox{Hence }\vec{N}&=\myvec{a&ax\\ \frac{-a}{x}&-a}
\end{align}}\\
 &This is similar to $\myvec{a&a\\-a&-a}$ via $\vec{P}=\myvec{\sqrt{x}&0\\0&\frac{1}{\sqrt{x}}}$.\\
 &And $\myvec{a&a\\-a&-a}$ is similar to $\myvec{0&0\\-a&0}$ via $\vec{P}=\myvec{-1&-1\\1&0}$\\
 &And this finally is similar to $\myvec{0&0\\1&0}$ as before.\\
 &\\
\hline
&\\
Conclusion &Thus either $\vec{N}$ = 0 or $\vec{N}$ is similar over $\mathbb{C}$ to $\myvec{0&0\\1&0}$.\\
&\\
\hline
\caption{Solution summary}
\label{eq:solutions/6/2/11/table:1}
\end{longtable}

%
\item 
\begin{enumerate}[label=\alph*.]
    \item Prove that only subspace of $\mathbb{R}^1$ are $\mathbb{R}^1$ and the zero subspace 
    \item Prove that a subspace of $\mathbb{R}^2$ is $\mathbb{R}^2$, or the zero subspace, or consists of all scalar multiples of some fixed vector in $\mathbb{R}^2$. (The last type of subspace is, intuitively,
a straight line through the origin.)
	\item Can you describe the subspaces of $\mathbb{R}^3$ ?
\end{enumerate}
%
\solution
See Table \ref{eq:solutions/6/2/11/table:1}
\onecolumn
\begin{longtable}{|l|l|}
\hline
\multirow{3}{*}{} & \\
Statement&Solution\\
\hline
&\parbox{6cm}{\begin{align}
\mbox{Let }\vec{N}&=\myvec{a&b\\c&d}\\
\mbox{Since }\vec{N}^2&=0
\end{align}}\\
&If $\myvec{a\\c},\myvec{b\\d}$ are linearly independent then $\vec{N}$ is diagonalizable to $\myvec{0&0\\0&0}$.\\
&\parbox{6cm}{\begin{align}
\mbox{If }\vec{P}\vec{N}\vec{P}^{-1}=0\\
\mbox{then }\vec{N}=\vec{P}^{-1}\vec{0}\vec{P}=0
\end{align}}\\
Proof that&So in this case $\vec{N}$ itself is the zero matrix.\\
$\vec{N}=0$&This contradicts the assumption that $\myvec{a\\c},\myvec{b\\d}$ are linearly independent.\\
&$\therefore$ we can assume that $\myvec{a\\c},\myvec{b\\d}$ are linearly dependent if both are\\
&equal to the zero vector\\
&\parbox{6cm}
{\begin{align}
   \mbox{then } \vec{N} &= 0.
\end{align}}\\
%\hline
%\pagebreak
\hline
&\\
&Therefore we can assume at least one vector is non-zero.\\
Assuming $\myvec{b\\d}$ as&Therefore
$\vec{N}=\myvec{a&0\\c&0}$\\
the zero vector&\\
&\parbox{6cm}{\begin{align}
\mbox{So }\vec{N}^2&=0\\
\implies a^2&=0\\
\therefore a&=0\\
\mbox{Thus }\vec{N}&=\myvec{a&0\\c&0}
\end{align}}\\
&In this case $\vec{N}$ is similar to $\vec{N}=\myvec{0&0\\1&0}$ via the matrix $\vec{P}=\myvec{c&0\\0&1}$\\
&\\
\hline
&\\
Assuming $\myvec{a\\c}$ as&Therefore
$\vec{N}=\myvec{0&b\\0&d}$\\
the zero vector&\\
&\parbox{6cm}{\begin{align}
\mbox{Then }\vec{N}^2&=0\\
\implies d^2&=0\\
\therefore d&=0\\
\mbox{Thus }\vec{N}&=\myvec{0&b\\0&0}
\end{align}}\\
&In this case $\vec{N}$ is similar to $\vec{N}=\myvec{0&0\\b&0}$ via the matrix $\vec{P}=\myvec{0&1\\1&0}$,\\
&which is similar to $\myvec{0&0\\1&0}$ as above.\\
&\\
\hline
&\\
Hence& we can assume neither $\myvec{a\\c}$ or $\myvec{b\\d}$ is the zero vector.\\
&\\
%\hline
%\pagebreak
\hline
&\\
Consequences of &Since they are linearly dependent we can assume,\\
linear&\\
independence&\\
&\parbox{6cm}{\begin{align}
\myvec{b\\d}&=x\myvec{a\\c}\\
\therefore \vec{N}&=\myvec{a&ax\\c&cx}\\
\therefore \vec{N}^2&=0\\
\implies a(a+cx)&=0\\
c(a+cx)&=0\\
ax(a+cx)&=0\\
cx(a+cx)&=0
\end{align}}\\
\hline
&\\
Proof that $\vec{N}$ is&We know that at least one of a or c is not zero.\\
similar over $\mathbb{C}$ to&If a = 0 then c $\neq$ 0, it must be that x = 0.\\
$\myvec{0&0\\1&0}$&So in this case $\vec{N}=\myvec{0&0\\c&0}$ which is similar to $\myvec{0&0\\1&0}$ as before.\\
&\\
&\parbox{6cm}{\begin{align}
\mbox{If } a &\neq0\\
\mbox{then }x &\neq0\\
\mbox{else }a(a+cx)&=0\\
\implies a&=0\\
\mbox{Thus }a+cx&=0\\
\mbox{Hence }\vec{N}&=\myvec{a&ax\\ \frac{-a}{x}&-a}
\end{align}}\\
 &This is similar to $\myvec{a&a\\-a&-a}$ via $\vec{P}=\myvec{\sqrt{x}&0\\0&\frac{1}{\sqrt{x}}}$.\\
 &And $\myvec{a&a\\-a&-a}$ is similar to $\myvec{0&0\\-a&0}$ via $\vec{P}=\myvec{-1&-1\\1&0}$\\
 &And this finally is similar to $\myvec{0&0\\1&0}$ as before.\\
 &\\
\hline
&\\
Conclusion &Thus either $\vec{N}$ = 0 or $\vec{N}$ is similar over $\mathbb{C}$ to $\myvec{0&0\\1&0}$.\\
&\\
\hline
\caption{Solution summary}
\label{eq:solutions/6/2/11/table:1}
\end{longtable}

%
\item Let $\vec{W}_1$ and $\vec{W}_2$ be subspaces of a vector space $\vec{V}$ such that the set-theoretic union of $\vec{W}_1$ and $\vec{W}_2$ is also a subspace. Prove that one of the spaces $\vec{W}_i$ is contained in the other.
%
\solution
See Table \ref{eq:solutions/6/2/11/table:1}
\onecolumn
\begin{longtable}{|l|l|}
\hline
\multirow{3}{*}{} & \\
Statement&Solution\\
\hline
&\parbox{6cm}{\begin{align}
\mbox{Let }\vec{N}&=\myvec{a&b\\c&d}\\
\mbox{Since }\vec{N}^2&=0
\end{align}}\\
&If $\myvec{a\\c},\myvec{b\\d}$ are linearly independent then $\vec{N}$ is diagonalizable to $\myvec{0&0\\0&0}$.\\
&\parbox{6cm}{\begin{align}
\mbox{If }\vec{P}\vec{N}\vec{P}^{-1}=0\\
\mbox{then }\vec{N}=\vec{P}^{-1}\vec{0}\vec{P}=0
\end{align}}\\
Proof that&So in this case $\vec{N}$ itself is the zero matrix.\\
$\vec{N}=0$&This contradicts the assumption that $\myvec{a\\c},\myvec{b\\d}$ are linearly independent.\\
&$\therefore$ we can assume that $\myvec{a\\c},\myvec{b\\d}$ are linearly dependent if both are\\
&equal to the zero vector\\
&\parbox{6cm}
{\begin{align}
   \mbox{then } \vec{N} &= 0.
\end{align}}\\
%\hline
%\pagebreak
\hline
&\\
&Therefore we can assume at least one vector is non-zero.\\
Assuming $\myvec{b\\d}$ as&Therefore
$\vec{N}=\myvec{a&0\\c&0}$\\
the zero vector&\\
&\parbox{6cm}{\begin{align}
\mbox{So }\vec{N}^2&=0\\
\implies a^2&=0\\
\therefore a&=0\\
\mbox{Thus }\vec{N}&=\myvec{a&0\\c&0}
\end{align}}\\
&In this case $\vec{N}$ is similar to $\vec{N}=\myvec{0&0\\1&0}$ via the matrix $\vec{P}=\myvec{c&0\\0&1}$\\
&\\
\hline
&\\
Assuming $\myvec{a\\c}$ as&Therefore
$\vec{N}=\myvec{0&b\\0&d}$\\
the zero vector&\\
&\parbox{6cm}{\begin{align}
\mbox{Then }\vec{N}^2&=0\\
\implies d^2&=0\\
\therefore d&=0\\
\mbox{Thus }\vec{N}&=\myvec{0&b\\0&0}
\end{align}}\\
&In this case $\vec{N}$ is similar to $\vec{N}=\myvec{0&0\\b&0}$ via the matrix $\vec{P}=\myvec{0&1\\1&0}$,\\
&which is similar to $\myvec{0&0\\1&0}$ as above.\\
&\\
\hline
&\\
Hence& we can assume neither $\myvec{a\\c}$ or $\myvec{b\\d}$ is the zero vector.\\
&\\
%\hline
%\pagebreak
\hline
&\\
Consequences of &Since they are linearly dependent we can assume,\\
linear&\\
independence&\\
&\parbox{6cm}{\begin{align}
\myvec{b\\d}&=x\myvec{a\\c}\\
\therefore \vec{N}&=\myvec{a&ax\\c&cx}\\
\therefore \vec{N}^2&=0\\
\implies a(a+cx)&=0\\
c(a+cx)&=0\\
ax(a+cx)&=0\\
cx(a+cx)&=0
\end{align}}\\
\hline
&\\
Proof that $\vec{N}$ is&We know that at least one of a or c is not zero.\\
similar over $\mathbb{C}$ to&If a = 0 then c $\neq$ 0, it must be that x = 0.\\
$\myvec{0&0\\1&0}$&So in this case $\vec{N}=\myvec{0&0\\c&0}$ which is similar to $\myvec{0&0\\1&0}$ as before.\\
&\\
&\parbox{6cm}{\begin{align}
\mbox{If } a &\neq0\\
\mbox{then }x &\neq0\\
\mbox{else }a(a+cx)&=0\\
\implies a&=0\\
\mbox{Thus }a+cx&=0\\
\mbox{Hence }\vec{N}&=\myvec{a&ax\\ \frac{-a}{x}&-a}
\end{align}}\\
 &This is similar to $\myvec{a&a\\-a&-a}$ via $\vec{P}=\myvec{\sqrt{x}&0\\0&\frac{1}{\sqrt{x}}}$.\\
 &And $\myvec{a&a\\-a&-a}$ is similar to $\myvec{0&0\\-a&0}$ via $\vec{P}=\myvec{-1&-1\\1&0}$\\
 &And this finally is similar to $\myvec{0&0\\1&0}$ as before.\\
 &\\
\hline
&\\
Conclusion &Thus either $\vec{N}$ = 0 or $\vec{N}$ is similar over $\mathbb{C}$ to $\myvec{0&0\\1&0}$.\\
&\\
\hline
\caption{Solution summary}
\label{eq:solutions/6/2/11/table:1}
\end{longtable}

\item Let $\vec{V}$ be the  vector space of all functions from $\vec{R}$ into $\vec{R}$; let $\vec{V_e}$ be the subset of even functions, $f(-x) = f(x)$; let $\vec{V_o}$ be the subset of odd functions, $f(-x) = -f(x)$.
\begin{enumerate}
   \item Prove that $\vec{V_e}$ and $\vec{V_o}$ are subspaces of $\vec{V}$
   \item Prove that $\vec{V_e} + \vec{V_o} = \vec{V}$
    \item Prove that $\vec{V_e} \cap \vec{V_o} = \{0\}$
 \end{enumerate}
%
\solution
See Table \ref{eq:solutions/6/2/11/table:1}
\onecolumn
\begin{longtable}{|l|l|}
\hline
\multirow{3}{*}{} & \\
Statement&Solution\\
\hline
&\parbox{6cm}{\begin{align}
\mbox{Let }\vec{N}&=\myvec{a&b\\c&d}\\
\mbox{Since }\vec{N}^2&=0
\end{align}}\\
&If $\myvec{a\\c},\myvec{b\\d}$ are linearly independent then $\vec{N}$ is diagonalizable to $\myvec{0&0\\0&0}$.\\
&\parbox{6cm}{\begin{align}
\mbox{If }\vec{P}\vec{N}\vec{P}^{-1}=0\\
\mbox{then }\vec{N}=\vec{P}^{-1}\vec{0}\vec{P}=0
\end{align}}\\
Proof that&So in this case $\vec{N}$ itself is the zero matrix.\\
$\vec{N}=0$&This contradicts the assumption that $\myvec{a\\c},\myvec{b\\d}$ are linearly independent.\\
&$\therefore$ we can assume that $\myvec{a\\c},\myvec{b\\d}$ are linearly dependent if both are\\
&equal to the zero vector\\
&\parbox{6cm}
{\begin{align}
   \mbox{then } \vec{N} &= 0.
\end{align}}\\
%\hline
%\pagebreak
\hline
&\\
&Therefore we can assume at least one vector is non-zero.\\
Assuming $\myvec{b\\d}$ as&Therefore
$\vec{N}=\myvec{a&0\\c&0}$\\
the zero vector&\\
&\parbox{6cm}{\begin{align}
\mbox{So }\vec{N}^2&=0\\
\implies a^2&=0\\
\therefore a&=0\\
\mbox{Thus }\vec{N}&=\myvec{a&0\\c&0}
\end{align}}\\
&In this case $\vec{N}$ is similar to $\vec{N}=\myvec{0&0\\1&0}$ via the matrix $\vec{P}=\myvec{c&0\\0&1}$\\
&\\
\hline
&\\
Assuming $\myvec{a\\c}$ as&Therefore
$\vec{N}=\myvec{0&b\\0&d}$\\
the zero vector&\\
&\parbox{6cm}{\begin{align}
\mbox{Then }\vec{N}^2&=0\\
\implies d^2&=0\\
\therefore d&=0\\
\mbox{Thus }\vec{N}&=\myvec{0&b\\0&0}
\end{align}}\\
&In this case $\vec{N}$ is similar to $\vec{N}=\myvec{0&0\\b&0}$ via the matrix $\vec{P}=\myvec{0&1\\1&0}$,\\
&which is similar to $\myvec{0&0\\1&0}$ as above.\\
&\\
\hline
&\\
Hence& we can assume neither $\myvec{a\\c}$ or $\myvec{b\\d}$ is the zero vector.\\
&\\
%\hline
%\pagebreak
\hline
&\\
Consequences of &Since they are linearly dependent we can assume,\\
linear&\\
independence&\\
&\parbox{6cm}{\begin{align}
\myvec{b\\d}&=x\myvec{a\\c}\\
\therefore \vec{N}&=\myvec{a&ax\\c&cx}\\
\therefore \vec{N}^2&=0\\
\implies a(a+cx)&=0\\
c(a+cx)&=0\\
ax(a+cx)&=0\\
cx(a+cx)&=0
\end{align}}\\
\hline
&\\
Proof that $\vec{N}$ is&We know that at least one of a or c is not zero.\\
similar over $\mathbb{C}$ to&If a = 0 then c $\neq$ 0, it must be that x = 0.\\
$\myvec{0&0\\1&0}$&So in this case $\vec{N}=\myvec{0&0\\c&0}$ which is similar to $\myvec{0&0\\1&0}$ as before.\\
&\\
&\parbox{6cm}{\begin{align}
\mbox{If } a &\neq0\\
\mbox{then }x &\neq0\\
\mbox{else }a(a+cx)&=0\\
\implies a&=0\\
\mbox{Thus }a+cx&=0\\
\mbox{Hence }\vec{N}&=\myvec{a&ax\\ \frac{-a}{x}&-a}
\end{align}}\\
 &This is similar to $\myvec{a&a\\-a&-a}$ via $\vec{P}=\myvec{\sqrt{x}&0\\0&\frac{1}{\sqrt{x}}}$.\\
 &And $\myvec{a&a\\-a&-a}$ is similar to $\myvec{0&0\\-a&0}$ via $\vec{P}=\myvec{-1&-1\\1&0}$\\
 &And this finally is similar to $\myvec{0&0\\1&0}$ as before.\\
 &\\
\hline
&\\
Conclusion &Thus either $\vec{N}$ = 0 or $\vec{N}$ is similar over $\mathbb{C}$ to $\myvec{0&0\\1&0}$.\\
&\\
\hline
\caption{Solution summary}
\label{eq:solutions/6/2/11/table:1}
\end{longtable}

\item Let $\vec{W_1}$ and $\vec{W_2}$ be subspaces of a vector space $\vec{V}$ such that 
\begin{align}
   \vec{W_1}+\vec{W_2}=\vec{V} \\
   \text{and  }  \vec{W_1}\cap \vec{W_2}=\vec{0}\label{eq:solutions/2/2/9/1}
\end{align}
Prove that for each vector $\vec{\alpha}$ in $\vec{V}$ there are unique vectors $\vec{\alpha_1}$ in $\vec{W_1}$ and $\vec{\alpha_2}$ in $\vec{W_2}$ such that
\begin{align}
    \vec{\alpha}=\vec{\alpha_1}+\vec{\alpha_2}\label{eq:solutions/2/2/9/2}
\end{align}
%
\\
\solution
See Table \ref{eq:solutions/6/2/11/table:1}
\onecolumn
\begin{longtable}{|l|l|}
\hline
\multirow{3}{*}{} & \\
Statement&Solution\\
\hline
&\parbox{6cm}{\begin{align}
\mbox{Let }\vec{N}&=\myvec{a&b\\c&d}\\
\mbox{Since }\vec{N}^2&=0
\end{align}}\\
&If $\myvec{a\\c},\myvec{b\\d}$ are linearly independent then $\vec{N}$ is diagonalizable to $\myvec{0&0\\0&0}$.\\
&\parbox{6cm}{\begin{align}
\mbox{If }\vec{P}\vec{N}\vec{P}^{-1}=0\\
\mbox{then }\vec{N}=\vec{P}^{-1}\vec{0}\vec{P}=0
\end{align}}\\
Proof that&So in this case $\vec{N}$ itself is the zero matrix.\\
$\vec{N}=0$&This contradicts the assumption that $\myvec{a\\c},\myvec{b\\d}$ are linearly independent.\\
&$\therefore$ we can assume that $\myvec{a\\c},\myvec{b\\d}$ are linearly dependent if both are\\
&equal to the zero vector\\
&\parbox{6cm}
{\begin{align}
   \mbox{then } \vec{N} &= 0.
\end{align}}\\
%\hline
%\pagebreak
\hline
&\\
&Therefore we can assume at least one vector is non-zero.\\
Assuming $\myvec{b\\d}$ as&Therefore
$\vec{N}=\myvec{a&0\\c&0}$\\
the zero vector&\\
&\parbox{6cm}{\begin{align}
\mbox{So }\vec{N}^2&=0\\
\implies a^2&=0\\
\therefore a&=0\\
\mbox{Thus }\vec{N}&=\myvec{a&0\\c&0}
\end{align}}\\
&In this case $\vec{N}$ is similar to $\vec{N}=\myvec{0&0\\1&0}$ via the matrix $\vec{P}=\myvec{c&0\\0&1}$\\
&\\
\hline
&\\
Assuming $\myvec{a\\c}$ as&Therefore
$\vec{N}=\myvec{0&b\\0&d}$\\
the zero vector&\\
&\parbox{6cm}{\begin{align}
\mbox{Then }\vec{N}^2&=0\\
\implies d^2&=0\\
\therefore d&=0\\
\mbox{Thus }\vec{N}&=\myvec{0&b\\0&0}
\end{align}}\\
&In this case $\vec{N}$ is similar to $\vec{N}=\myvec{0&0\\b&0}$ via the matrix $\vec{P}=\myvec{0&1\\1&0}$,\\
&which is similar to $\myvec{0&0\\1&0}$ as above.\\
&\\
\hline
&\\
Hence& we can assume neither $\myvec{a\\c}$ or $\myvec{b\\d}$ is the zero vector.\\
&\\
%\hline
%\pagebreak
\hline
&\\
Consequences of &Since they are linearly dependent we can assume,\\
linear&\\
independence&\\
&\parbox{6cm}{\begin{align}
\myvec{b\\d}&=x\myvec{a\\c}\\
\therefore \vec{N}&=\myvec{a&ax\\c&cx}\\
\therefore \vec{N}^2&=0\\
\implies a(a+cx)&=0\\
c(a+cx)&=0\\
ax(a+cx)&=0\\
cx(a+cx)&=0
\end{align}}\\
\hline
&\\
Proof that $\vec{N}$ is&We know that at least one of a or c is not zero.\\
similar over $\mathbb{C}$ to&If a = 0 then c $\neq$ 0, it must be that x = 0.\\
$\myvec{0&0\\1&0}$&So in this case $\vec{N}=\myvec{0&0\\c&0}$ which is similar to $\myvec{0&0\\1&0}$ as before.\\
&\\
&\parbox{6cm}{\begin{align}
\mbox{If } a &\neq0\\
\mbox{then }x &\neq0\\
\mbox{else }a(a+cx)&=0\\
\implies a&=0\\
\mbox{Thus }a+cx&=0\\
\mbox{Hence }\vec{N}&=\myvec{a&ax\\ \frac{-a}{x}&-a}
\end{align}}\\
 &This is similar to $\myvec{a&a\\-a&-a}$ via $\vec{P}=\myvec{\sqrt{x}&0\\0&\frac{1}{\sqrt{x}}}$.\\
 &And $\myvec{a&a\\-a&-a}$ is similar to $\myvec{0&0\\-a&0}$ via $\vec{P}=\myvec{-1&-1\\1&0}$\\
 &And this finally is similar to $\myvec{0&0\\1&0}$ as before.\\
 &\\
\hline
&\\
Conclusion &Thus either $\vec{N}$ = 0 or $\vec{N}$ is similar over $\mathbb{C}$ to $\myvec{0&0\\1&0}$.\\
&\\
\hline
\caption{Solution summary}
\label{eq:solutions/6/2/11/table:1}
\end{longtable}

%
\end{enumerate}



\subsection{Bases and Dimension}
\renewcommand{\theequation}{\theenumi}
\renewcommand{\thefigure}{\theenumi}
\begin{enumerate}[label=\thesubsection.\arabic*.,ref=\thesubsection.\theenumi]
\numberwithin{equation}{enumi}
\numberwithin{figure}{enumi}
\item Are the vectors 
\begin{align}
\alpha{_1}=\myvec{1,1,2,4} , \alpha{_2}=\myvec{2,-1,-5,2} \\
\alpha{_3}=\myvec{1,-1,-4,0} , \alpha{_4}=\myvec{2,1,1,6}
\end{align}
linearly independent in $R^{4}$
%
\\
\solution
See Table \ref{eq:solutions/6/2/11/table:1}
\onecolumn
\begin{longtable}{|l|l|}
\hline
\multirow{3}{*}{} & \\
Statement&Solution\\
\hline
&\parbox{6cm}{\begin{align}
\mbox{Let }\vec{N}&=\myvec{a&b\\c&d}\\
\mbox{Since }\vec{N}^2&=0
\end{align}}\\
&If $\myvec{a\\c},\myvec{b\\d}$ are linearly independent then $\vec{N}$ is diagonalizable to $\myvec{0&0\\0&0}$.\\
&\parbox{6cm}{\begin{align}
\mbox{If }\vec{P}\vec{N}\vec{P}^{-1}=0\\
\mbox{then }\vec{N}=\vec{P}^{-1}\vec{0}\vec{P}=0
\end{align}}\\
Proof that&So in this case $\vec{N}$ itself is the zero matrix.\\
$\vec{N}=0$&This contradicts the assumption that $\myvec{a\\c},\myvec{b\\d}$ are linearly independent.\\
&$\therefore$ we can assume that $\myvec{a\\c},\myvec{b\\d}$ are linearly dependent if both are\\
&equal to the zero vector\\
&\parbox{6cm}
{\begin{align}
   \mbox{then } \vec{N} &= 0.
\end{align}}\\
%\hline
%\pagebreak
\hline
&\\
&Therefore we can assume at least one vector is non-zero.\\
Assuming $\myvec{b\\d}$ as&Therefore
$\vec{N}=\myvec{a&0\\c&0}$\\
the zero vector&\\
&\parbox{6cm}{\begin{align}
\mbox{So }\vec{N}^2&=0\\
\implies a^2&=0\\
\therefore a&=0\\
\mbox{Thus }\vec{N}&=\myvec{a&0\\c&0}
\end{align}}\\
&In this case $\vec{N}$ is similar to $\vec{N}=\myvec{0&0\\1&0}$ via the matrix $\vec{P}=\myvec{c&0\\0&1}$\\
&\\
\hline
&\\
Assuming $\myvec{a\\c}$ as&Therefore
$\vec{N}=\myvec{0&b\\0&d}$\\
the zero vector&\\
&\parbox{6cm}{\begin{align}
\mbox{Then }\vec{N}^2&=0\\
\implies d^2&=0\\
\therefore d&=0\\
\mbox{Thus }\vec{N}&=\myvec{0&b\\0&0}
\end{align}}\\
&In this case $\vec{N}$ is similar to $\vec{N}=\myvec{0&0\\b&0}$ via the matrix $\vec{P}=\myvec{0&1\\1&0}$,\\
&which is similar to $\myvec{0&0\\1&0}$ as above.\\
&\\
\hline
&\\
Hence& we can assume neither $\myvec{a\\c}$ or $\myvec{b\\d}$ is the zero vector.\\
&\\
%\hline
%\pagebreak
\hline
&\\
Consequences of &Since they are linearly dependent we can assume,\\
linear&\\
independence&\\
&\parbox{6cm}{\begin{align}
\myvec{b\\d}&=x\myvec{a\\c}\\
\therefore \vec{N}&=\myvec{a&ax\\c&cx}\\
\therefore \vec{N}^2&=0\\
\implies a(a+cx)&=0\\
c(a+cx)&=0\\
ax(a+cx)&=0\\
cx(a+cx)&=0
\end{align}}\\
\hline
&\\
Proof that $\vec{N}$ is&We know that at least one of a or c is not zero.\\
similar over $\mathbb{C}$ to&If a = 0 then c $\neq$ 0, it must be that x = 0.\\
$\myvec{0&0\\1&0}$&So in this case $\vec{N}=\myvec{0&0\\c&0}$ which is similar to $\myvec{0&0\\1&0}$ as before.\\
&\\
&\parbox{6cm}{\begin{align}
\mbox{If } a &\neq0\\
\mbox{then }x &\neq0\\
\mbox{else }a(a+cx)&=0\\
\implies a&=0\\
\mbox{Thus }a+cx&=0\\
\mbox{Hence }\vec{N}&=\myvec{a&ax\\ \frac{-a}{x}&-a}
\end{align}}\\
 &This is similar to $\myvec{a&a\\-a&-a}$ via $\vec{P}=\myvec{\sqrt{x}&0\\0&\frac{1}{\sqrt{x}}}$.\\
 &And $\myvec{a&a\\-a&-a}$ is similar to $\myvec{0&0\\-a&0}$ via $\vec{P}=\myvec{-1&-1\\1&0}$\\
 &And this finally is similar to $\myvec{0&0\\1&0}$ as before.\\
 &\\
\hline
&\\
Conclusion &Thus either $\vec{N}$ = 0 or $\vec{N}$ is similar over $\mathbb{C}$ to $\myvec{0&0\\1&0}$.\\
&\\
\hline
\caption{Solution summary}
\label{eq:solutions/6/2/11/table:1}
\end{longtable}

\item    Let $\vec{V}$ be a vector space over a subfield $\vec{F}$ of complex numbers. Suppose $\vec{\alpha}$, $\vec{\beta}$ and $\vec{\gamma}$ are linearly independent vectors in $\vec{V}$. Prove that ($\vec{\alpha}$+$\beta$),($\vec{\beta}$+$\vec{\gamma}$) and ($\vec{\gamma}$+$\vec{\alpha}$) are linearly independent.
%
\\
\solution
See Table \ref{eq:solutions/6/2/11/table:1}
\onecolumn
\begin{longtable}{|l|l|}
\hline
\multirow{3}{*}{} & \\
Statement&Solution\\
\hline
&\parbox{6cm}{\begin{align}
\mbox{Let }\vec{N}&=\myvec{a&b\\c&d}\\
\mbox{Since }\vec{N}^2&=0
\end{align}}\\
&If $\myvec{a\\c},\myvec{b\\d}$ are linearly independent then $\vec{N}$ is diagonalizable to $\myvec{0&0\\0&0}$.\\
&\parbox{6cm}{\begin{align}
\mbox{If }\vec{P}\vec{N}\vec{P}^{-1}=0\\
\mbox{then }\vec{N}=\vec{P}^{-1}\vec{0}\vec{P}=0
\end{align}}\\
Proof that&So in this case $\vec{N}$ itself is the zero matrix.\\
$\vec{N}=0$&This contradicts the assumption that $\myvec{a\\c},\myvec{b\\d}$ are linearly independent.\\
&$\therefore$ we can assume that $\myvec{a\\c},\myvec{b\\d}$ are linearly dependent if both are\\
&equal to the zero vector\\
&\parbox{6cm}
{\begin{align}
   \mbox{then } \vec{N} &= 0.
\end{align}}\\
%\hline
%\pagebreak
\hline
&\\
&Therefore we can assume at least one vector is non-zero.\\
Assuming $\myvec{b\\d}$ as&Therefore
$\vec{N}=\myvec{a&0\\c&0}$\\
the zero vector&\\
&\parbox{6cm}{\begin{align}
\mbox{So }\vec{N}^2&=0\\
\implies a^2&=0\\
\therefore a&=0\\
\mbox{Thus }\vec{N}&=\myvec{a&0\\c&0}
\end{align}}\\
&In this case $\vec{N}$ is similar to $\vec{N}=\myvec{0&0\\1&0}$ via the matrix $\vec{P}=\myvec{c&0\\0&1}$\\
&\\
\hline
&\\
Assuming $\myvec{a\\c}$ as&Therefore
$\vec{N}=\myvec{0&b\\0&d}$\\
the zero vector&\\
&\parbox{6cm}{\begin{align}
\mbox{Then }\vec{N}^2&=0\\
\implies d^2&=0\\
\therefore d&=0\\
\mbox{Thus }\vec{N}&=\myvec{0&b\\0&0}
\end{align}}\\
&In this case $\vec{N}$ is similar to $\vec{N}=\myvec{0&0\\b&0}$ via the matrix $\vec{P}=\myvec{0&1\\1&0}$,\\
&which is similar to $\myvec{0&0\\1&0}$ as above.\\
&\\
\hline
&\\
Hence& we can assume neither $\myvec{a\\c}$ or $\myvec{b\\d}$ is the zero vector.\\
&\\
%\hline
%\pagebreak
\hline
&\\
Consequences of &Since they are linearly dependent we can assume,\\
linear&\\
independence&\\
&\parbox{6cm}{\begin{align}
\myvec{b\\d}&=x\myvec{a\\c}\\
\therefore \vec{N}&=\myvec{a&ax\\c&cx}\\
\therefore \vec{N}^2&=0\\
\implies a(a+cx)&=0\\
c(a+cx)&=0\\
ax(a+cx)&=0\\
cx(a+cx)&=0
\end{align}}\\
\hline
&\\
Proof that $\vec{N}$ is&We know that at least one of a or c is not zero.\\
similar over $\mathbb{C}$ to&If a = 0 then c $\neq$ 0, it must be that x = 0.\\
$\myvec{0&0\\1&0}$&So in this case $\vec{N}=\myvec{0&0\\c&0}$ which is similar to $\myvec{0&0\\1&0}$ as before.\\
&\\
&\parbox{6cm}{\begin{align}
\mbox{If } a &\neq0\\
\mbox{then }x &\neq0\\
\mbox{else }a(a+cx)&=0\\
\implies a&=0\\
\mbox{Thus }a+cx&=0\\
\mbox{Hence }\vec{N}&=\myvec{a&ax\\ \frac{-a}{x}&-a}
\end{align}}\\
 &This is similar to $\myvec{a&a\\-a&-a}$ via $\vec{P}=\myvec{\sqrt{x}&0\\0&\frac{1}{\sqrt{x}}}$.\\
 &And $\myvec{a&a\\-a&-a}$ is similar to $\myvec{0&0\\-a&0}$ via $\vec{P}=\myvec{-1&-1\\1&0}$\\
 &And this finally is similar to $\myvec{0&0\\1&0}$ as before.\\
 &\\
\hline
&\\
Conclusion &Thus either $\vec{N}$ = 0 or $\vec{N}$ is similar over $\mathbb{C}$ to $\myvec{0&0\\1&0}$.\\
&\\
\hline
\caption{Solution summary}
\label{eq:solutions/6/2/11/table:1}
\end{longtable}

%
\item Prove that the space of all $\vec{m} x \vec{n}$ matrices over the field $\vec{F}$ has dimension mn, by exhibiting a basis for this space. 
%
\\
\solution
See Table \ref{eq:solutions/6/2/11/table:1}
\onecolumn
\begin{longtable}{|l|l|}
\hline
\multirow{3}{*}{} & \\
Statement&Solution\\
\hline
&\parbox{6cm}{\begin{align}
\mbox{Let }\vec{N}&=\myvec{a&b\\c&d}\\
\mbox{Since }\vec{N}^2&=0
\end{align}}\\
&If $\myvec{a\\c},\myvec{b\\d}$ are linearly independent then $\vec{N}$ is diagonalizable to $\myvec{0&0\\0&0}$.\\
&\parbox{6cm}{\begin{align}
\mbox{If }\vec{P}\vec{N}\vec{P}^{-1}=0\\
\mbox{then }\vec{N}=\vec{P}^{-1}\vec{0}\vec{P}=0
\end{align}}\\
Proof that&So in this case $\vec{N}$ itself is the zero matrix.\\
$\vec{N}=0$&This contradicts the assumption that $\myvec{a\\c},\myvec{b\\d}$ are linearly independent.\\
&$\therefore$ we can assume that $\myvec{a\\c},\myvec{b\\d}$ are linearly dependent if both are\\
&equal to the zero vector\\
&\parbox{6cm}
{\begin{align}
   \mbox{then } \vec{N} &= 0.
\end{align}}\\
%\hline
%\pagebreak
\hline
&\\
&Therefore we can assume at least one vector is non-zero.\\
Assuming $\myvec{b\\d}$ as&Therefore
$\vec{N}=\myvec{a&0\\c&0}$\\
the zero vector&\\
&\parbox{6cm}{\begin{align}
\mbox{So }\vec{N}^2&=0\\
\implies a^2&=0\\
\therefore a&=0\\
\mbox{Thus }\vec{N}&=\myvec{a&0\\c&0}
\end{align}}\\
&In this case $\vec{N}$ is similar to $\vec{N}=\myvec{0&0\\1&0}$ via the matrix $\vec{P}=\myvec{c&0\\0&1}$\\
&\\
\hline
&\\
Assuming $\myvec{a\\c}$ as&Therefore
$\vec{N}=\myvec{0&b\\0&d}$\\
the zero vector&\\
&\parbox{6cm}{\begin{align}
\mbox{Then }\vec{N}^2&=0\\
\implies d^2&=0\\
\therefore d&=0\\
\mbox{Thus }\vec{N}&=\myvec{0&b\\0&0}
\end{align}}\\
&In this case $\vec{N}$ is similar to $\vec{N}=\myvec{0&0\\b&0}$ via the matrix $\vec{P}=\myvec{0&1\\1&0}$,\\
&which is similar to $\myvec{0&0\\1&0}$ as above.\\
&\\
\hline
&\\
Hence& we can assume neither $\myvec{a\\c}$ or $\myvec{b\\d}$ is the zero vector.\\
&\\
%\hline
%\pagebreak
\hline
&\\
Consequences of &Since they are linearly dependent we can assume,\\
linear&\\
independence&\\
&\parbox{6cm}{\begin{align}
\myvec{b\\d}&=x\myvec{a\\c}\\
\therefore \vec{N}&=\myvec{a&ax\\c&cx}\\
\therefore \vec{N}^2&=0\\
\implies a(a+cx)&=0\\
c(a+cx)&=0\\
ax(a+cx)&=0\\
cx(a+cx)&=0
\end{align}}\\
\hline
&\\
Proof that $\vec{N}$ is&We know that at least one of a or c is not zero.\\
similar over $\mathbb{C}$ to&If a = 0 then c $\neq$ 0, it must be that x = 0.\\
$\myvec{0&0\\1&0}$&So in this case $\vec{N}=\myvec{0&0\\c&0}$ which is similar to $\myvec{0&0\\1&0}$ as before.\\
&\\
&\parbox{6cm}{\begin{align}
\mbox{If } a &\neq0\\
\mbox{then }x &\neq0\\
\mbox{else }a(a+cx)&=0\\
\implies a&=0\\
\mbox{Thus }a+cx&=0\\
\mbox{Hence }\vec{N}&=\myvec{a&ax\\ \frac{-a}{x}&-a}
\end{align}}\\
 &This is similar to $\myvec{a&a\\-a&-a}$ via $\vec{P}=\myvec{\sqrt{x}&0\\0&\frac{1}{\sqrt{x}}}$.\\
 &And $\myvec{a&a\\-a&-a}$ is similar to $\myvec{0&0\\-a&0}$ via $\vec{P}=\myvec{-1&-1\\1&0}$\\
 &And this finally is similar to $\myvec{0&0\\1&0}$ as before.\\
 &\\
\hline
&\\
Conclusion &Thus either $\vec{N}$ = 0 or $\vec{N}$ is similar over $\mathbb{C}$ to $\myvec{0&0\\1&0}$.\\
&\\
\hline
\caption{Solution summary}
\label{eq:solutions/6/2/11/table:1}
\end{longtable}

\end{enumerate}

\subsection{Coordinates}
\renewcommand{\theequation}{\theenumi}
\renewcommand{\thefigure}{\theenumi}
\begin{enumerate}[label=\thesubsection.\arabic*.,ref=\thesubsection.\theenumi]
\numberwithin{equation}{enumi}
\numberwithin{figure}{enumi}
%
\item Show that the vectors 
\begin{align}
& \vec{\alpha_1} = \myvec{1 & 1 & 0 & 0} & \vec{\alpha_2} = \myvec{0 & 0 & 1 & 1} \\
& \vec{\alpha_3} = \myvec{1 & 0 & 0 & 4} & \vec{\alpha_4} = \myvec{0 & 0 & 0 & 2}
\end{align}
form a basis for $\Re^4$. Find the coordinates of each of the standard basis vectors
in the ordered basis $\myvec{\vec{\alpha_1} &  \vec{\alpha_2} & \vec{\alpha_3} & \vec{\alpha_4}}$
%
\\
\solution
See Table \ref{eq:solutions/6/2/11/table:1}
\onecolumn
\begin{longtable}{|l|l|}
\hline
\multirow{3}{*}{} & \\
Statement&Solution\\
\hline
&\parbox{6cm}{\begin{align}
\mbox{Let }\vec{N}&=\myvec{a&b\\c&d}\\
\mbox{Since }\vec{N}^2&=0
\end{align}}\\
&If $\myvec{a\\c},\myvec{b\\d}$ are linearly independent then $\vec{N}$ is diagonalizable to $\myvec{0&0\\0&0}$.\\
&\parbox{6cm}{\begin{align}
\mbox{If }\vec{P}\vec{N}\vec{P}^{-1}=0\\
\mbox{then }\vec{N}=\vec{P}^{-1}\vec{0}\vec{P}=0
\end{align}}\\
Proof that&So in this case $\vec{N}$ itself is the zero matrix.\\
$\vec{N}=0$&This contradicts the assumption that $\myvec{a\\c},\myvec{b\\d}$ are linearly independent.\\
&$\therefore$ we can assume that $\myvec{a\\c},\myvec{b\\d}$ are linearly dependent if both are\\
&equal to the zero vector\\
&\parbox{6cm}
{\begin{align}
   \mbox{then } \vec{N} &= 0.
\end{align}}\\
%\hline
%\pagebreak
\hline
&\\
&Therefore we can assume at least one vector is non-zero.\\
Assuming $\myvec{b\\d}$ as&Therefore
$\vec{N}=\myvec{a&0\\c&0}$\\
the zero vector&\\
&\parbox{6cm}{\begin{align}
\mbox{So }\vec{N}^2&=0\\
\implies a^2&=0\\
\therefore a&=0\\
\mbox{Thus }\vec{N}&=\myvec{a&0\\c&0}
\end{align}}\\
&In this case $\vec{N}$ is similar to $\vec{N}=\myvec{0&0\\1&0}$ via the matrix $\vec{P}=\myvec{c&0\\0&1}$\\
&\\
\hline
&\\
Assuming $\myvec{a\\c}$ as&Therefore
$\vec{N}=\myvec{0&b\\0&d}$\\
the zero vector&\\
&\parbox{6cm}{\begin{align}
\mbox{Then }\vec{N}^2&=0\\
\implies d^2&=0\\
\therefore d&=0\\
\mbox{Thus }\vec{N}&=\myvec{0&b\\0&0}
\end{align}}\\
&In this case $\vec{N}$ is similar to $\vec{N}=\myvec{0&0\\b&0}$ via the matrix $\vec{P}=\myvec{0&1\\1&0}$,\\
&which is similar to $\myvec{0&0\\1&0}$ as above.\\
&\\
\hline
&\\
Hence& we can assume neither $\myvec{a\\c}$ or $\myvec{b\\d}$ is the zero vector.\\
&\\
%\hline
%\pagebreak
\hline
&\\
Consequences of &Since they are linearly dependent we can assume,\\
linear&\\
independence&\\
&\parbox{6cm}{\begin{align}
\myvec{b\\d}&=x\myvec{a\\c}\\
\therefore \vec{N}&=\myvec{a&ax\\c&cx}\\
\therefore \vec{N}^2&=0\\
\implies a(a+cx)&=0\\
c(a+cx)&=0\\
ax(a+cx)&=0\\
cx(a+cx)&=0
\end{align}}\\
\hline
&\\
Proof that $\vec{N}$ is&We know that at least one of a or c is not zero.\\
similar over $\mathbb{C}$ to&If a = 0 then c $\neq$ 0, it must be that x = 0.\\
$\myvec{0&0\\1&0}$&So in this case $\vec{N}=\myvec{0&0\\c&0}$ which is similar to $\myvec{0&0\\1&0}$ as before.\\
&\\
&\parbox{6cm}{\begin{align}
\mbox{If } a &\neq0\\
\mbox{then }x &\neq0\\
\mbox{else }a(a+cx)&=0\\
\implies a&=0\\
\mbox{Thus }a+cx&=0\\
\mbox{Hence }\vec{N}&=\myvec{a&ax\\ \frac{-a}{x}&-a}
\end{align}}\\
 &This is similar to $\myvec{a&a\\-a&-a}$ via $\vec{P}=\myvec{\sqrt{x}&0\\0&\frac{1}{\sqrt{x}}}$.\\
 &And $\myvec{a&a\\-a&-a}$ is similar to $\myvec{0&0\\-a&0}$ via $\vec{P}=\myvec{-1&-1\\1&0}$\\
 &And this finally is similar to $\myvec{0&0\\1&0}$ as before.\\
 &\\
\hline
&\\
Conclusion &Thus either $\vec{N}$ = 0 or $\vec{N}$ is similar over $\mathbb{C}$ to $\myvec{0&0\\1&0}$.\\
&\\
\hline
\caption{Solution summary}
\label{eq:solutions/6/2/11/table:1}
\end{longtable}

%
\item Find the coordinate matrix of the vector \myvec{1&0&1} in the basis of $C^{3}$ consisting of the vectors \myvec{2i&1&0} , \myvec{2&-1&1} , \myvec{0&1+i&1-i} in that order.
%
\\
\solution
See Table \ref{eq:solutions/6/2/11/table:1}
\onecolumn
\begin{longtable}{|l|l|}
\hline
\multirow{3}{*}{} & \\
Statement&Solution\\
\hline
&\parbox{6cm}{\begin{align}
\mbox{Let }\vec{N}&=\myvec{a&b\\c&d}\\
\mbox{Since }\vec{N}^2&=0
\end{align}}\\
&If $\myvec{a\\c},\myvec{b\\d}$ are linearly independent then $\vec{N}$ is diagonalizable to $\myvec{0&0\\0&0}$.\\
&\parbox{6cm}{\begin{align}
\mbox{If }\vec{P}\vec{N}\vec{P}^{-1}=0\\
\mbox{then }\vec{N}=\vec{P}^{-1}\vec{0}\vec{P}=0
\end{align}}\\
Proof that&So in this case $\vec{N}$ itself is the zero matrix.\\
$\vec{N}=0$&This contradicts the assumption that $\myvec{a\\c},\myvec{b\\d}$ are linearly independent.\\
&$\therefore$ we can assume that $\myvec{a\\c},\myvec{b\\d}$ are linearly dependent if both are\\
&equal to the zero vector\\
&\parbox{6cm}
{\begin{align}
   \mbox{then } \vec{N} &= 0.
\end{align}}\\
%\hline
%\pagebreak
\hline
&\\
&Therefore we can assume at least one vector is non-zero.\\
Assuming $\myvec{b\\d}$ as&Therefore
$\vec{N}=\myvec{a&0\\c&0}$\\
the zero vector&\\
&\parbox{6cm}{\begin{align}
\mbox{So }\vec{N}^2&=0\\
\implies a^2&=0\\
\therefore a&=0\\
\mbox{Thus }\vec{N}&=\myvec{a&0\\c&0}
\end{align}}\\
&In this case $\vec{N}$ is similar to $\vec{N}=\myvec{0&0\\1&0}$ via the matrix $\vec{P}=\myvec{c&0\\0&1}$\\
&\\
\hline
&\\
Assuming $\myvec{a\\c}$ as&Therefore
$\vec{N}=\myvec{0&b\\0&d}$\\
the zero vector&\\
&\parbox{6cm}{\begin{align}
\mbox{Then }\vec{N}^2&=0\\
\implies d^2&=0\\
\therefore d&=0\\
\mbox{Thus }\vec{N}&=\myvec{0&b\\0&0}
\end{align}}\\
&In this case $\vec{N}$ is similar to $\vec{N}=\myvec{0&0\\b&0}$ via the matrix $\vec{P}=\myvec{0&1\\1&0}$,\\
&which is similar to $\myvec{0&0\\1&0}$ as above.\\
&\\
\hline
&\\
Hence& we can assume neither $\myvec{a\\c}$ or $\myvec{b\\d}$ is the zero vector.\\
&\\
%\hline
%\pagebreak
\hline
&\\
Consequences of &Since they are linearly dependent we can assume,\\
linear&\\
independence&\\
&\parbox{6cm}{\begin{align}
\myvec{b\\d}&=x\myvec{a\\c}\\
\therefore \vec{N}&=\myvec{a&ax\\c&cx}\\
\therefore \vec{N}^2&=0\\
\implies a(a+cx)&=0\\
c(a+cx)&=0\\
ax(a+cx)&=0\\
cx(a+cx)&=0
\end{align}}\\
\hline
&\\
Proof that $\vec{N}$ is&We know that at least one of a or c is not zero.\\
similar over $\mathbb{C}$ to&If a = 0 then c $\neq$ 0, it must be that x = 0.\\
$\myvec{0&0\\1&0}$&So in this case $\vec{N}=\myvec{0&0\\c&0}$ which is similar to $\myvec{0&0\\1&0}$ as before.\\
&\\
&\parbox{6cm}{\begin{align}
\mbox{If } a &\neq0\\
\mbox{then }x &\neq0\\
\mbox{else }a(a+cx)&=0\\
\implies a&=0\\
\mbox{Thus }a+cx&=0\\
\mbox{Hence }\vec{N}&=\myvec{a&ax\\ \frac{-a}{x}&-a}
\end{align}}\\
 &This is similar to $\myvec{a&a\\-a&-a}$ via $\vec{P}=\myvec{\sqrt{x}&0\\0&\frac{1}{\sqrt{x}}}$.\\
 &And $\myvec{a&a\\-a&-a}$ is similar to $\myvec{0&0\\-a&0}$ via $\vec{P}=\myvec{-1&-1\\1&0}$\\
 &And this finally is similar to $\myvec{0&0\\1&0}$ as before.\\
 &\\
\hline
&\\
Conclusion &Thus either $\vec{N}$ = 0 or $\vec{N}$ is similar over $\mathbb{C}$ to $\myvec{0&0\\1&0}$.\\
&\\
\hline
\caption{Solution summary}
\label{eq:solutions/6/2/11/table:1}
\end{longtable}

%
\item 
Let $\vec{B} = \myvec{\alpha_1&\alpha_2&\alpha_3}$ be the ordered basis for $R^3$ consisting of 
$$\vec{\alpha_1} = \myvec{1&0&-1}, \vec{\alpha_2} = \myvec{1&1&1}, \vec{\alpha_3} = \myvec{1&0&0}.$$
What are the coordinates of vector $\myvec{a&b&c}$ in the ordered basis $\vec{B}$ ?
%
\\
\solution
See Table \ref{eq:solutions/6/2/11/table:1}
\onecolumn
\begin{longtable}{|l|l|}
\hline
\multirow{3}{*}{} & \\
Statement&Solution\\
\hline
&\parbox{6cm}{\begin{align}
\mbox{Let }\vec{N}&=\myvec{a&b\\c&d}\\
\mbox{Since }\vec{N}^2&=0
\end{align}}\\
&If $\myvec{a\\c},\myvec{b\\d}$ are linearly independent then $\vec{N}$ is diagonalizable to $\myvec{0&0\\0&0}$.\\
&\parbox{6cm}{\begin{align}
\mbox{If }\vec{P}\vec{N}\vec{P}^{-1}=0\\
\mbox{then }\vec{N}=\vec{P}^{-1}\vec{0}\vec{P}=0
\end{align}}\\
Proof that&So in this case $\vec{N}$ itself is the zero matrix.\\
$\vec{N}=0$&This contradicts the assumption that $\myvec{a\\c},\myvec{b\\d}$ are linearly independent.\\
&$\therefore$ we can assume that $\myvec{a\\c},\myvec{b\\d}$ are linearly dependent if both are\\
&equal to the zero vector\\
&\parbox{6cm}
{\begin{align}
   \mbox{then } \vec{N} &= 0.
\end{align}}\\
%\hline
%\pagebreak
\hline
&\\
&Therefore we can assume at least one vector is non-zero.\\
Assuming $\myvec{b\\d}$ as&Therefore
$\vec{N}=\myvec{a&0\\c&0}$\\
the zero vector&\\
&\parbox{6cm}{\begin{align}
\mbox{So }\vec{N}^2&=0\\
\implies a^2&=0\\
\therefore a&=0\\
\mbox{Thus }\vec{N}&=\myvec{a&0\\c&0}
\end{align}}\\
&In this case $\vec{N}$ is similar to $\vec{N}=\myvec{0&0\\1&0}$ via the matrix $\vec{P}=\myvec{c&0\\0&1}$\\
&\\
\hline
&\\
Assuming $\myvec{a\\c}$ as&Therefore
$\vec{N}=\myvec{0&b\\0&d}$\\
the zero vector&\\
&\parbox{6cm}{\begin{align}
\mbox{Then }\vec{N}^2&=0\\
\implies d^2&=0\\
\therefore d&=0\\
\mbox{Thus }\vec{N}&=\myvec{0&b\\0&0}
\end{align}}\\
&In this case $\vec{N}$ is similar to $\vec{N}=\myvec{0&0\\b&0}$ via the matrix $\vec{P}=\myvec{0&1\\1&0}$,\\
&which is similar to $\myvec{0&0\\1&0}$ as above.\\
&\\
\hline
&\\
Hence& we can assume neither $\myvec{a\\c}$ or $\myvec{b\\d}$ is the zero vector.\\
&\\
%\hline
%\pagebreak
\hline
&\\
Consequences of &Since they are linearly dependent we can assume,\\
linear&\\
independence&\\
&\parbox{6cm}{\begin{align}
\myvec{b\\d}&=x\myvec{a\\c}\\
\therefore \vec{N}&=\myvec{a&ax\\c&cx}\\
\therefore \vec{N}^2&=0\\
\implies a(a+cx)&=0\\
c(a+cx)&=0\\
ax(a+cx)&=0\\
cx(a+cx)&=0
\end{align}}\\
\hline
&\\
Proof that $\vec{N}$ is&We know that at least one of a or c is not zero.\\
similar over $\mathbb{C}$ to&If a = 0 then c $\neq$ 0, it must be that x = 0.\\
$\myvec{0&0\\1&0}$&So in this case $\vec{N}=\myvec{0&0\\c&0}$ which is similar to $\myvec{0&0\\1&0}$ as before.\\
&\\
&\parbox{6cm}{\begin{align}
\mbox{If } a &\neq0\\
\mbox{then }x &\neq0\\
\mbox{else }a(a+cx)&=0\\
\implies a&=0\\
\mbox{Thus }a+cx&=0\\
\mbox{Hence }\vec{N}&=\myvec{a&ax\\ \frac{-a}{x}&-a}
\end{align}}\\
 &This is similar to $\myvec{a&a\\-a&-a}$ via $\vec{P}=\myvec{\sqrt{x}&0\\0&\frac{1}{\sqrt{x}}}$.\\
 &And $\myvec{a&a\\-a&-a}$ is similar to $\myvec{0&0\\-a&0}$ via $\vec{P}=\myvec{-1&-1\\1&0}$\\
 &And this finally is similar to $\myvec{0&0\\1&0}$ as before.\\
 &\\
\hline
&\\
Conclusion &Thus either $\vec{N}$ = 0 or $\vec{N}$ is similar over $\mathbb{C}$ to $\myvec{0&0\\1&0}$.\\
&\\
\hline
\caption{Solution summary}
\label{eq:solutions/6/2/11/table:1}
\end{longtable}

\item Let $\vec{W}$ be the subspace of $\vec{C}^3$ spanned by 
$\alpha_1=\myvec{1\\0\\i}$ and $\alpha_2=\myvec{1\\i\\1+i}$.
\begin{enumerate}[label=\emph{\alph*)}]
	\item Show that $\alpha_1$ and $\alpha_2$ form a basis for $\vec{W}$.\\
	\item Show that the vectors $\beta_1=\myvec{1\\1\\0}$ and 
		$\beta_2=\myvec{1\\i\\1+i}$ are in $\vec{W}$ and form another
		basis for $\vec{W}$. \\
	\item What are the coordinates of $\alpha_1$ and $\alpha_2$ in the
		ordered basis $\cbrak{\beta_1, \beta_2} $ for $\vec{W}$.
\end{enumerate}
\solution
See Table \ref{eq:solutions/6/2/11/table:1}
\onecolumn
\begin{longtable}{|l|l|}
\hline
\multirow{3}{*}{} & \\
Statement&Solution\\
\hline
&\parbox{6cm}{\begin{align}
\mbox{Let }\vec{N}&=\myvec{a&b\\c&d}\\
\mbox{Since }\vec{N}^2&=0
\end{align}}\\
&If $\myvec{a\\c},\myvec{b\\d}$ are linearly independent then $\vec{N}$ is diagonalizable to $\myvec{0&0\\0&0}$.\\
&\parbox{6cm}{\begin{align}
\mbox{If }\vec{P}\vec{N}\vec{P}^{-1}=0\\
\mbox{then }\vec{N}=\vec{P}^{-1}\vec{0}\vec{P}=0
\end{align}}\\
Proof that&So in this case $\vec{N}$ itself is the zero matrix.\\
$\vec{N}=0$&This contradicts the assumption that $\myvec{a\\c},\myvec{b\\d}$ are linearly independent.\\
&$\therefore$ we can assume that $\myvec{a\\c},\myvec{b\\d}$ are linearly dependent if both are\\
&equal to the zero vector\\
&\parbox{6cm}
{\begin{align}
   \mbox{then } \vec{N} &= 0.
\end{align}}\\
%\hline
%\pagebreak
\hline
&\\
&Therefore we can assume at least one vector is non-zero.\\
Assuming $\myvec{b\\d}$ as&Therefore
$\vec{N}=\myvec{a&0\\c&0}$\\
the zero vector&\\
&\parbox{6cm}{\begin{align}
\mbox{So }\vec{N}^2&=0\\
\implies a^2&=0\\
\therefore a&=0\\
\mbox{Thus }\vec{N}&=\myvec{a&0\\c&0}
\end{align}}\\
&In this case $\vec{N}$ is similar to $\vec{N}=\myvec{0&0\\1&0}$ via the matrix $\vec{P}=\myvec{c&0\\0&1}$\\
&\\
\hline
&\\
Assuming $\myvec{a\\c}$ as&Therefore
$\vec{N}=\myvec{0&b\\0&d}$\\
the zero vector&\\
&\parbox{6cm}{\begin{align}
\mbox{Then }\vec{N}^2&=0\\
\implies d^2&=0\\
\therefore d&=0\\
\mbox{Thus }\vec{N}&=\myvec{0&b\\0&0}
\end{align}}\\
&In this case $\vec{N}$ is similar to $\vec{N}=\myvec{0&0\\b&0}$ via the matrix $\vec{P}=\myvec{0&1\\1&0}$,\\
&which is similar to $\myvec{0&0\\1&0}$ as above.\\
&\\
\hline
&\\
Hence& we can assume neither $\myvec{a\\c}$ or $\myvec{b\\d}$ is the zero vector.\\
&\\
%\hline
%\pagebreak
\hline
&\\
Consequences of &Since they are linearly dependent we can assume,\\
linear&\\
independence&\\
&\parbox{6cm}{\begin{align}
\myvec{b\\d}&=x\myvec{a\\c}\\
\therefore \vec{N}&=\myvec{a&ax\\c&cx}\\
\therefore \vec{N}^2&=0\\
\implies a(a+cx)&=0\\
c(a+cx)&=0\\
ax(a+cx)&=0\\
cx(a+cx)&=0
\end{align}}\\
\hline
&\\
Proof that $\vec{N}$ is&We know that at least one of a or c is not zero.\\
similar over $\mathbb{C}$ to&If a = 0 then c $\neq$ 0, it must be that x = 0.\\
$\myvec{0&0\\1&0}$&So in this case $\vec{N}=\myvec{0&0\\c&0}$ which is similar to $\myvec{0&0\\1&0}$ as before.\\
&\\
&\parbox{6cm}{\begin{align}
\mbox{If } a &\neq0\\
\mbox{then }x &\neq0\\
\mbox{else }a(a+cx)&=0\\
\implies a&=0\\
\mbox{Thus }a+cx&=0\\
\mbox{Hence }\vec{N}&=\myvec{a&ax\\ \frac{-a}{x}&-a}
\end{align}}\\
 &This is similar to $\myvec{a&a\\-a&-a}$ via $\vec{P}=\myvec{\sqrt{x}&0\\0&\frac{1}{\sqrt{x}}}$.\\
 &And $\myvec{a&a\\-a&-a}$ is similar to $\myvec{0&0\\-a&0}$ via $\vec{P}=\myvec{-1&-1\\1&0}$\\
 &And this finally is similar to $\myvec{0&0\\1&0}$ as before.\\
 &\\
\hline
&\\
Conclusion &Thus either $\vec{N}$ = 0 or $\vec{N}$ is similar over $\mathbb{C}$ to $\myvec{0&0\\1&0}$.\\
&\\
\hline
\caption{Solution summary}
\label{eq:solutions/6/2/11/table:1}
\end{longtable}

%
\item let $\alpha=\myvec{x_1\\x_2}$ and $\beta=\myvec{y_1\\y_2}$ be vectors in $\mathbb{R}^2$ such that\\
$x_1y_1+x_2y_2=0$;\quad $x_1^2+x_2^2=y_1^2+y_2^2=1$.\\
Proove that $\ss=\{\vec{\alpha},\vec{\beta}\}$ is a basis of $\mathbb{R}^2$. Find the coordinates of the vector $\myvec{a,b}$ in the ordered basis $\ss=\{\vec{\alpha},\vec{\beta}\}$. (The conditions on $\alpha$ and $\beta$ say, geometrically, that $\alpha$ and $\beta$ are perpendicular and each has length 1).
%
\\
\solution
See Table \ref{eq:solutions/6/2/11/table:1}
\onecolumn
\begin{longtable}{|l|l|}
\hline
\multirow{3}{*}{} & \\
Statement&Solution\\
\hline
&\parbox{6cm}{\begin{align}
\mbox{Let }\vec{N}&=\myvec{a&b\\c&d}\\
\mbox{Since }\vec{N}^2&=0
\end{align}}\\
&If $\myvec{a\\c},\myvec{b\\d}$ are linearly independent then $\vec{N}$ is diagonalizable to $\myvec{0&0\\0&0}$.\\
&\parbox{6cm}{\begin{align}
\mbox{If }\vec{P}\vec{N}\vec{P}^{-1}=0\\
\mbox{then }\vec{N}=\vec{P}^{-1}\vec{0}\vec{P}=0
\end{align}}\\
Proof that&So in this case $\vec{N}$ itself is the zero matrix.\\
$\vec{N}=0$&This contradicts the assumption that $\myvec{a\\c},\myvec{b\\d}$ are linearly independent.\\
&$\therefore$ we can assume that $\myvec{a\\c},\myvec{b\\d}$ are linearly dependent if both are\\
&equal to the zero vector\\
&\parbox{6cm}
{\begin{align}
   \mbox{then } \vec{N} &= 0.
\end{align}}\\
%\hline
%\pagebreak
\hline
&\\
&Therefore we can assume at least one vector is non-zero.\\
Assuming $\myvec{b\\d}$ as&Therefore
$\vec{N}=\myvec{a&0\\c&0}$\\
the zero vector&\\
&\parbox{6cm}{\begin{align}
\mbox{So }\vec{N}^2&=0\\
\implies a^2&=0\\
\therefore a&=0\\
\mbox{Thus }\vec{N}&=\myvec{a&0\\c&0}
\end{align}}\\
&In this case $\vec{N}$ is similar to $\vec{N}=\myvec{0&0\\1&0}$ via the matrix $\vec{P}=\myvec{c&0\\0&1}$\\
&\\
\hline
&\\
Assuming $\myvec{a\\c}$ as&Therefore
$\vec{N}=\myvec{0&b\\0&d}$\\
the zero vector&\\
&\parbox{6cm}{\begin{align}
\mbox{Then }\vec{N}^2&=0\\
\implies d^2&=0\\
\therefore d&=0\\
\mbox{Thus }\vec{N}&=\myvec{0&b\\0&0}
\end{align}}\\
&In this case $\vec{N}$ is similar to $\vec{N}=\myvec{0&0\\b&0}$ via the matrix $\vec{P}=\myvec{0&1\\1&0}$,\\
&which is similar to $\myvec{0&0\\1&0}$ as above.\\
&\\
\hline
&\\
Hence& we can assume neither $\myvec{a\\c}$ or $\myvec{b\\d}$ is the zero vector.\\
&\\
%\hline
%\pagebreak
\hline
&\\
Consequences of &Since they are linearly dependent we can assume,\\
linear&\\
independence&\\
&\parbox{6cm}{\begin{align}
\myvec{b\\d}&=x\myvec{a\\c}\\
\therefore \vec{N}&=\myvec{a&ax\\c&cx}\\
\therefore \vec{N}^2&=0\\
\implies a(a+cx)&=0\\
c(a+cx)&=0\\
ax(a+cx)&=0\\
cx(a+cx)&=0
\end{align}}\\
\hline
&\\
Proof that $\vec{N}$ is&We know that at least one of a or c is not zero.\\
similar over $\mathbb{C}$ to&If a = 0 then c $\neq$ 0, it must be that x = 0.\\
$\myvec{0&0\\1&0}$&So in this case $\vec{N}=\myvec{0&0\\c&0}$ which is similar to $\myvec{0&0\\1&0}$ as before.\\
&\\
&\parbox{6cm}{\begin{align}
\mbox{If } a &\neq0\\
\mbox{then }x &\neq0\\
\mbox{else }a(a+cx)&=0\\
\implies a&=0\\
\mbox{Thus }a+cx&=0\\
\mbox{Hence }\vec{N}&=\myvec{a&ax\\ \frac{-a}{x}&-a}
\end{align}}\\
 &This is similar to $\myvec{a&a\\-a&-a}$ via $\vec{P}=\myvec{\sqrt{x}&0\\0&\frac{1}{\sqrt{x}}}$.\\
 &And $\myvec{a&a\\-a&-a}$ is similar to $\myvec{0&0\\-a&0}$ via $\vec{P}=\myvec{-1&-1\\1&0}$\\
 &And this finally is similar to $\myvec{0&0\\1&0}$ as before.\\
 &\\
\hline
&\\
Conclusion &Thus either $\vec{N}$ = 0 or $\vec{N}$ is similar over $\mathbb{C}$ to $\myvec{0&0\\1&0}$.\\
&\\
\hline
\caption{Solution summary}
\label{eq:solutions/6/2/11/table:1}
\end{longtable}

%
\item    Let $\vec{V}$ be the real vector space of all polynomial functions from $\mathbb{R}$ to $\mathbb{R}$ of degree 2 or less, i.e, the space of all functions f of the form,
   \begin{align*}
   f(x)=c_0+c_1x+c_2x^2
   \end{align*} 
   Let t be a fixed real number and define
   \begin{align*}
   g_1(x)=1,        g_2(x)=x+t,         g_3(x)=(x+t)^2
   \end{align*}
   Prove that $\beta$=\{g1,g2,g3\} is a basis for V. If 
   \begin{align*}
   f(x)=c_0+c_1x+c_2x^2
   \end{align*}
   what are the coordinates of f in the ordered basis $\beta$
%
\\
\solution
See Table \ref{eq:solutions/6/2/11/table:1}
\onecolumn
\begin{longtable}{|l|l|}
\hline
\multirow{3}{*}{} & \\
Statement&Solution\\
\hline
&\parbox{6cm}{\begin{align}
\mbox{Let }\vec{N}&=\myvec{a&b\\c&d}\\
\mbox{Since }\vec{N}^2&=0
\end{align}}\\
&If $\myvec{a\\c},\myvec{b\\d}$ are linearly independent then $\vec{N}$ is diagonalizable to $\myvec{0&0\\0&0}$.\\
&\parbox{6cm}{\begin{align}
\mbox{If }\vec{P}\vec{N}\vec{P}^{-1}=0\\
\mbox{then }\vec{N}=\vec{P}^{-1}\vec{0}\vec{P}=0
\end{align}}\\
Proof that&So in this case $\vec{N}$ itself is the zero matrix.\\
$\vec{N}=0$&This contradicts the assumption that $\myvec{a\\c},\myvec{b\\d}$ are linearly independent.\\
&$\therefore$ we can assume that $\myvec{a\\c},\myvec{b\\d}$ are linearly dependent if both are\\
&equal to the zero vector\\
&\parbox{6cm}
{\begin{align}
   \mbox{then } \vec{N} &= 0.
\end{align}}\\
%\hline
%\pagebreak
\hline
&\\
&Therefore we can assume at least one vector is non-zero.\\
Assuming $\myvec{b\\d}$ as&Therefore
$\vec{N}=\myvec{a&0\\c&0}$\\
the zero vector&\\
&\parbox{6cm}{\begin{align}
\mbox{So }\vec{N}^2&=0\\
\implies a^2&=0\\
\therefore a&=0\\
\mbox{Thus }\vec{N}&=\myvec{a&0\\c&0}
\end{align}}\\
&In this case $\vec{N}$ is similar to $\vec{N}=\myvec{0&0\\1&0}$ via the matrix $\vec{P}=\myvec{c&0\\0&1}$\\
&\\
\hline
&\\
Assuming $\myvec{a\\c}$ as&Therefore
$\vec{N}=\myvec{0&b\\0&d}$\\
the zero vector&\\
&\parbox{6cm}{\begin{align}
\mbox{Then }\vec{N}^2&=0\\
\implies d^2&=0\\
\therefore d&=0\\
\mbox{Thus }\vec{N}&=\myvec{0&b\\0&0}
\end{align}}\\
&In this case $\vec{N}$ is similar to $\vec{N}=\myvec{0&0\\b&0}$ via the matrix $\vec{P}=\myvec{0&1\\1&0}$,\\
&which is similar to $\myvec{0&0\\1&0}$ as above.\\
&\\
\hline
&\\
Hence& we can assume neither $\myvec{a\\c}$ or $\myvec{b\\d}$ is the zero vector.\\
&\\
%\hline
%\pagebreak
\hline
&\\
Consequences of &Since they are linearly dependent we can assume,\\
linear&\\
independence&\\
&\parbox{6cm}{\begin{align}
\myvec{b\\d}&=x\myvec{a\\c}\\
\therefore \vec{N}&=\myvec{a&ax\\c&cx}\\
\therefore \vec{N}^2&=0\\
\implies a(a+cx)&=0\\
c(a+cx)&=0\\
ax(a+cx)&=0\\
cx(a+cx)&=0
\end{align}}\\
\hline
&\\
Proof that $\vec{N}$ is&We know that at least one of a or c is not zero.\\
similar over $\mathbb{C}$ to&If a = 0 then c $\neq$ 0, it must be that x = 0.\\
$\myvec{0&0\\1&0}$&So in this case $\vec{N}=\myvec{0&0\\c&0}$ which is similar to $\myvec{0&0\\1&0}$ as before.\\
&\\
&\parbox{6cm}{\begin{align}
\mbox{If } a &\neq0\\
\mbox{then }x &\neq0\\
\mbox{else }a(a+cx)&=0\\
\implies a&=0\\
\mbox{Thus }a+cx&=0\\
\mbox{Hence }\vec{N}&=\myvec{a&ax\\ \frac{-a}{x}&-a}
\end{align}}\\
 &This is similar to $\myvec{a&a\\-a&-a}$ via $\vec{P}=\myvec{\sqrt{x}&0\\0&\frac{1}{\sqrt{x}}}$.\\
 &And $\myvec{a&a\\-a&-a}$ is similar to $\myvec{0&0\\-a&0}$ via $\vec{P}=\myvec{-1&-1\\1&0}$\\
 &And this finally is similar to $\myvec{0&0\\1&0}$ as before.\\
 &\\
\hline
&\\
Conclusion &Thus either $\vec{N}$ = 0 or $\vec{N}$ is similar over $\mathbb{C}$ to $\myvec{0&0\\1&0}$.\\
&\\
\hline
\caption{Solution summary}
\label{eq:solutions/6/2/11/table:1}
\end{longtable}

\end{enumerate}

\subsection{Summary of Row Equivalence}
\renewcommand{\theequation}{\theenumi}
\renewcommand{\thefigure}{\theenumi}
\begin{enumerate}[label=\thesubsection.\arabic*.,ref=\thesubsection.\theenumi]
\numberwithin{equation}{enumi}
\numberwithin{figure}{enumi}
%
\item Consider the vectors in $\mathbb{R}^4$ defined by:\\
$\alpha_1=\myvec{-1\\0\\1\\2}$, $\alpha_2=\myvec{3\\4\\-2\\5}$ and $\alpha_3=\myvec{1\\4\\0\\9}$.\\
Find a system of homogeneous linear equations for which the space of solutions is exactly the subspace of $\mathbb{R}^4$ spanned by the given three vectors.\\
%
\\
\solution
See Table \ref{eq:solutions/6/2/11/table:1}
\onecolumn
\begin{longtable}{|l|l|}
\hline
\multirow{3}{*}{} & \\
Statement&Solution\\
\hline
&\parbox{6cm}{\begin{align}
\mbox{Let }\vec{N}&=\myvec{a&b\\c&d}\\
\mbox{Since }\vec{N}^2&=0
\end{align}}\\
&If $\myvec{a\\c},\myvec{b\\d}$ are linearly independent then $\vec{N}$ is diagonalizable to $\myvec{0&0\\0&0}$.\\
&\parbox{6cm}{\begin{align}
\mbox{If }\vec{P}\vec{N}\vec{P}^{-1}=0\\
\mbox{then }\vec{N}=\vec{P}^{-1}\vec{0}\vec{P}=0
\end{align}}\\
Proof that&So in this case $\vec{N}$ itself is the zero matrix.\\
$\vec{N}=0$&This contradicts the assumption that $\myvec{a\\c},\myvec{b\\d}$ are linearly independent.\\
&$\therefore$ we can assume that $\myvec{a\\c},\myvec{b\\d}$ are linearly dependent if both are\\
&equal to the zero vector\\
&\parbox{6cm}
{\begin{align}
   \mbox{then } \vec{N} &= 0.
\end{align}}\\
%\hline
%\pagebreak
\hline
&\\
&Therefore we can assume at least one vector is non-zero.\\
Assuming $\myvec{b\\d}$ as&Therefore
$\vec{N}=\myvec{a&0\\c&0}$\\
the zero vector&\\
&\parbox{6cm}{\begin{align}
\mbox{So }\vec{N}^2&=0\\
\implies a^2&=0\\
\therefore a&=0\\
\mbox{Thus }\vec{N}&=\myvec{a&0\\c&0}
\end{align}}\\
&In this case $\vec{N}$ is similar to $\vec{N}=\myvec{0&0\\1&0}$ via the matrix $\vec{P}=\myvec{c&0\\0&1}$\\
&\\
\hline
&\\
Assuming $\myvec{a\\c}$ as&Therefore
$\vec{N}=\myvec{0&b\\0&d}$\\
the zero vector&\\
&\parbox{6cm}{\begin{align}
\mbox{Then }\vec{N}^2&=0\\
\implies d^2&=0\\
\therefore d&=0\\
\mbox{Thus }\vec{N}&=\myvec{0&b\\0&0}
\end{align}}\\
&In this case $\vec{N}$ is similar to $\vec{N}=\myvec{0&0\\b&0}$ via the matrix $\vec{P}=\myvec{0&1\\1&0}$,\\
&which is similar to $\myvec{0&0\\1&0}$ as above.\\
&\\
\hline
&\\
Hence& we can assume neither $\myvec{a\\c}$ or $\myvec{b\\d}$ is the zero vector.\\
&\\
%\hline
%\pagebreak
\hline
&\\
Consequences of &Since they are linearly dependent we can assume,\\
linear&\\
independence&\\
&\parbox{6cm}{\begin{align}
\myvec{b\\d}&=x\myvec{a\\c}\\
\therefore \vec{N}&=\myvec{a&ax\\c&cx}\\
\therefore \vec{N}^2&=0\\
\implies a(a+cx)&=0\\
c(a+cx)&=0\\
ax(a+cx)&=0\\
cx(a+cx)&=0
\end{align}}\\
\hline
&\\
Proof that $\vec{N}$ is&We know that at least one of a or c is not zero.\\
similar over $\mathbb{C}$ to&If a = 0 then c $\neq$ 0, it must be that x = 0.\\
$\myvec{0&0\\1&0}$&So in this case $\vec{N}=\myvec{0&0\\c&0}$ which is similar to $\myvec{0&0\\1&0}$ as before.\\
&\\
&\parbox{6cm}{\begin{align}
\mbox{If } a &\neq0\\
\mbox{then }x &\neq0\\
\mbox{else }a(a+cx)&=0\\
\implies a&=0\\
\mbox{Thus }a+cx&=0\\
\mbox{Hence }\vec{N}&=\myvec{a&ax\\ \frac{-a}{x}&-a}
\end{align}}\\
 &This is similar to $\myvec{a&a\\-a&-a}$ via $\vec{P}=\myvec{\sqrt{x}&0\\0&\frac{1}{\sqrt{x}}}$.\\
 &And $\myvec{a&a\\-a&-a}$ is similar to $\myvec{0&0\\-a&0}$ via $\vec{P}=\myvec{-1&-1\\1&0}$\\
 &And this finally is similar to $\myvec{0&0\\1&0}$ as before.\\
 &\\
\hline
&\\
Conclusion &Thus either $\vec{N}$ = 0 or $\vec{N}$ is similar over $\mathbb{C}$ to $\myvec{0&0\\1&0}$.\\
&\\
\hline
\caption{Solution summary}
\label{eq:solutions/6/2/11/table:1}
\end{longtable}

%
\item Let \textit{s\textless n} and A an $s\times n$ matrix with entries in the field $\mathbb{F}$. Use Theorem 4 to show that there is a non-zero $\vec{x}$ in $\mathbb{F}^{n\times 1}$ such that $\vec{A}\vec{x}=\vec{0}$. 
%
\\
\solution
See Table \ref{eq:solutions/6/2/11/table:1}
\onecolumn
\begin{longtable}{|l|l|}
\hline
\multirow{3}{*}{} & \\
Statement&Solution\\
\hline
&\parbox{6cm}{\begin{align}
\mbox{Let }\vec{N}&=\myvec{a&b\\c&d}\\
\mbox{Since }\vec{N}^2&=0
\end{align}}\\
&If $\myvec{a\\c},\myvec{b\\d}$ are linearly independent then $\vec{N}$ is diagonalizable to $\myvec{0&0\\0&0}$.\\
&\parbox{6cm}{\begin{align}
\mbox{If }\vec{P}\vec{N}\vec{P}^{-1}=0\\
\mbox{then }\vec{N}=\vec{P}^{-1}\vec{0}\vec{P}=0
\end{align}}\\
Proof that&So in this case $\vec{N}$ itself is the zero matrix.\\
$\vec{N}=0$&This contradicts the assumption that $\myvec{a\\c},\myvec{b\\d}$ are linearly independent.\\
&$\therefore$ we can assume that $\myvec{a\\c},\myvec{b\\d}$ are linearly dependent if both are\\
&equal to the zero vector\\
&\parbox{6cm}
{\begin{align}
   \mbox{then } \vec{N} &= 0.
\end{align}}\\
%\hline
%\pagebreak
\hline
&\\
&Therefore we can assume at least one vector is non-zero.\\
Assuming $\myvec{b\\d}$ as&Therefore
$\vec{N}=\myvec{a&0\\c&0}$\\
the zero vector&\\
&\parbox{6cm}{\begin{align}
\mbox{So }\vec{N}^2&=0\\
\implies a^2&=0\\
\therefore a&=0\\
\mbox{Thus }\vec{N}&=\myvec{a&0\\c&0}
\end{align}}\\
&In this case $\vec{N}$ is similar to $\vec{N}=\myvec{0&0\\1&0}$ via the matrix $\vec{P}=\myvec{c&0\\0&1}$\\
&\\
\hline
&\\
Assuming $\myvec{a\\c}$ as&Therefore
$\vec{N}=\myvec{0&b\\0&d}$\\
the zero vector&\\
&\parbox{6cm}{\begin{align}
\mbox{Then }\vec{N}^2&=0\\
\implies d^2&=0\\
\therefore d&=0\\
\mbox{Thus }\vec{N}&=\myvec{0&b\\0&0}
\end{align}}\\
&In this case $\vec{N}$ is similar to $\vec{N}=\myvec{0&0\\b&0}$ via the matrix $\vec{P}=\myvec{0&1\\1&0}$,\\
&which is similar to $\myvec{0&0\\1&0}$ as above.\\
&\\
\hline
&\\
Hence& we can assume neither $\myvec{a\\c}$ or $\myvec{b\\d}$ is the zero vector.\\
&\\
%\hline
%\pagebreak
\hline
&\\
Consequences of &Since they are linearly dependent we can assume,\\
linear&\\
independence&\\
&\parbox{6cm}{\begin{align}
\myvec{b\\d}&=x\myvec{a\\c}\\
\therefore \vec{N}&=\myvec{a&ax\\c&cx}\\
\therefore \vec{N}^2&=0\\
\implies a(a+cx)&=0\\
c(a+cx)&=0\\
ax(a+cx)&=0\\
cx(a+cx)&=0
\end{align}}\\
\hline
&\\
Proof that $\vec{N}$ is&We know that at least one of a or c is not zero.\\
similar over $\mathbb{C}$ to&If a = 0 then c $\neq$ 0, it must be that x = 0.\\
$\myvec{0&0\\1&0}$&So in this case $\vec{N}=\myvec{0&0\\c&0}$ which is similar to $\myvec{0&0\\1&0}$ as before.\\
&\\
&\parbox{6cm}{\begin{align}
\mbox{If } a &\neq0\\
\mbox{then }x &\neq0\\
\mbox{else }a(a+cx)&=0\\
\implies a&=0\\
\mbox{Thus }a+cx&=0\\
\mbox{Hence }\vec{N}&=\myvec{a&ax\\ \frac{-a}{x}&-a}
\end{align}}\\
 &This is similar to $\myvec{a&a\\-a&-a}$ via $\vec{P}=\myvec{\sqrt{x}&0\\0&\frac{1}{\sqrt{x}}}$.\\
 &And $\myvec{a&a\\-a&-a}$ is similar to $\myvec{0&0\\-a&0}$ via $\vec{P}=\myvec{-1&-1\\1&0}$\\
 &And this finally is similar to $\myvec{0&0\\1&0}$ as before.\\
 &\\
\hline
&\\
Conclusion &Thus either $\vec{N}$ = 0 or $\vec{N}$ is similar over $\mathbb{C}$ to $\myvec{0&0\\1&0}$.\\
&\\
\hline
\caption{Solution summary}
\label{eq:solutions/6/2/11/table:1}
\end{longtable}

%
\item Let
\begin{align} 
\alpha_1=\myvec{1&1&-2&1}^T \\
\alpha_2=\myvec{3&0&4&-1}^T \\
\alpha_3=\myvec{-1&2&5&2}^T
\end{align}
Let
\begin{align}
\alpha=\myvec{4&-5&9&-7}^T \\
\beta=\myvec{3&1&-4&4}^T \\
\gamma=\myvec{-1&1&0&1}^T
\end{align}
\begin{enumerate}
\item Which of the vectors $\alpha$, $\beta$, $\gamma$ are in the subspace of $\mathbb{R}^4$ spanned by $\alpha_i$?
\item Which of the vectors $\alpha$, $\beta$, $\gamma$ are in the subspace of $\mathbb{C}^4$ spanned by $\alpha_i$?
\item Does this suggest a theorem?
\end{enumerate}
%
%
\solution
See Table \ref{eq:solutions/6/2/11/table:1}
\onecolumn
\begin{longtable}{|l|l|}
\hline
\multirow{3}{*}{} & \\
Statement&Solution\\
\hline
&\parbox{6cm}{\begin{align}
\mbox{Let }\vec{N}&=\myvec{a&b\\c&d}\\
\mbox{Since }\vec{N}^2&=0
\end{align}}\\
&If $\myvec{a\\c},\myvec{b\\d}$ are linearly independent then $\vec{N}$ is diagonalizable to $\myvec{0&0\\0&0}$.\\
&\parbox{6cm}{\begin{align}
\mbox{If }\vec{P}\vec{N}\vec{P}^{-1}=0\\
\mbox{then }\vec{N}=\vec{P}^{-1}\vec{0}\vec{P}=0
\end{align}}\\
Proof that&So in this case $\vec{N}$ itself is the zero matrix.\\
$\vec{N}=0$&This contradicts the assumption that $\myvec{a\\c},\myvec{b\\d}$ are linearly independent.\\
&$\therefore$ we can assume that $\myvec{a\\c},\myvec{b\\d}$ are linearly dependent if both are\\
&equal to the zero vector\\
&\parbox{6cm}
{\begin{align}
   \mbox{then } \vec{N} &= 0.
\end{align}}\\
%\hline
%\pagebreak
\hline
&\\
&Therefore we can assume at least one vector is non-zero.\\
Assuming $\myvec{b\\d}$ as&Therefore
$\vec{N}=\myvec{a&0\\c&0}$\\
the zero vector&\\
&\parbox{6cm}{\begin{align}
\mbox{So }\vec{N}^2&=0\\
\implies a^2&=0\\
\therefore a&=0\\
\mbox{Thus }\vec{N}&=\myvec{a&0\\c&0}
\end{align}}\\
&In this case $\vec{N}$ is similar to $\vec{N}=\myvec{0&0\\1&0}$ via the matrix $\vec{P}=\myvec{c&0\\0&1}$\\
&\\
\hline
&\\
Assuming $\myvec{a\\c}$ as&Therefore
$\vec{N}=\myvec{0&b\\0&d}$\\
the zero vector&\\
&\parbox{6cm}{\begin{align}
\mbox{Then }\vec{N}^2&=0\\
\implies d^2&=0\\
\therefore d&=0\\
\mbox{Thus }\vec{N}&=\myvec{0&b\\0&0}
\end{align}}\\
&In this case $\vec{N}$ is similar to $\vec{N}=\myvec{0&0\\b&0}$ via the matrix $\vec{P}=\myvec{0&1\\1&0}$,\\
&which is similar to $\myvec{0&0\\1&0}$ as above.\\
&\\
\hline
&\\
Hence& we can assume neither $\myvec{a\\c}$ or $\myvec{b\\d}$ is the zero vector.\\
&\\
%\hline
%\pagebreak
\hline
&\\
Consequences of &Since they are linearly dependent we can assume,\\
linear&\\
independence&\\
&\parbox{6cm}{\begin{align}
\myvec{b\\d}&=x\myvec{a\\c}\\
\therefore \vec{N}&=\myvec{a&ax\\c&cx}\\
\therefore \vec{N}^2&=0\\
\implies a(a+cx)&=0\\
c(a+cx)&=0\\
ax(a+cx)&=0\\
cx(a+cx)&=0
\end{align}}\\
\hline
&\\
Proof that $\vec{N}$ is&We know that at least one of a or c is not zero.\\
similar over $\mathbb{C}$ to&If a = 0 then c $\neq$ 0, it must be that x = 0.\\
$\myvec{0&0\\1&0}$&So in this case $\vec{N}=\myvec{0&0\\c&0}$ which is similar to $\myvec{0&0\\1&0}$ as before.\\
&\\
&\parbox{6cm}{\begin{align}
\mbox{If } a &\neq0\\
\mbox{then }x &\neq0\\
\mbox{else }a(a+cx)&=0\\
\implies a&=0\\
\mbox{Thus }a+cx&=0\\
\mbox{Hence }\vec{N}&=\myvec{a&ax\\ \frac{-a}{x}&-a}
\end{align}}\\
 &This is similar to $\myvec{a&a\\-a&-a}$ via $\vec{P}=\myvec{\sqrt{x}&0\\0&\frac{1}{\sqrt{x}}}$.\\
 &And $\myvec{a&a\\-a&-a}$ is similar to $\myvec{0&0\\-a&0}$ via $\vec{P}=\myvec{-1&-1\\1&0}$\\
 &And this finally is similar to $\myvec{0&0\\1&0}$ as before.\\
 &\\
\hline
&\\
Conclusion &Thus either $\vec{N}$ = 0 or $\vec{N}$ is similar over $\mathbb{C}$ to $\myvec{0&0\\1&0}$.\\
&\\
\hline
\caption{Solution summary}
\label{eq:solutions/6/2/11/table:1}
\end{longtable}

%
\item In $C^3$, let $\vec{\alpha_1}$ = $(1,0,-i)$, $\vec{\alpha_2}$ = $(1+i,1-i,1)$ , $\vec{\alpha_3}$ = $(i,i,i)$. Prove that these vectors form a basis for $C^3$. What are the coordinates of the vector (a,b,c) in the basis?
%
\\
\solution
See Table \ref{eq:solutions/6/2/11/table:1}
\onecolumn
\begin{longtable}{|l|l|}
\hline
\multirow{3}{*}{} & \\
Statement&Solution\\
\hline
&\parbox{6cm}{\begin{align}
\mbox{Let }\vec{N}&=\myvec{a&b\\c&d}\\
\mbox{Since }\vec{N}^2&=0
\end{align}}\\
&If $\myvec{a\\c},\myvec{b\\d}$ are linearly independent then $\vec{N}$ is diagonalizable to $\myvec{0&0\\0&0}$.\\
&\parbox{6cm}{\begin{align}
\mbox{If }\vec{P}\vec{N}\vec{P}^{-1}=0\\
\mbox{then }\vec{N}=\vec{P}^{-1}\vec{0}\vec{P}=0
\end{align}}\\
Proof that&So in this case $\vec{N}$ itself is the zero matrix.\\
$\vec{N}=0$&This contradicts the assumption that $\myvec{a\\c},\myvec{b\\d}$ are linearly independent.\\
&$\therefore$ we can assume that $\myvec{a\\c},\myvec{b\\d}$ are linearly dependent if both are\\
&equal to the zero vector\\
&\parbox{6cm}
{\begin{align}
   \mbox{then } \vec{N} &= 0.
\end{align}}\\
%\hline
%\pagebreak
\hline
&\\
&Therefore we can assume at least one vector is non-zero.\\
Assuming $\myvec{b\\d}$ as&Therefore
$\vec{N}=\myvec{a&0\\c&0}$\\
the zero vector&\\
&\parbox{6cm}{\begin{align}
\mbox{So }\vec{N}^2&=0\\
\implies a^2&=0\\
\therefore a&=0\\
\mbox{Thus }\vec{N}&=\myvec{a&0\\c&0}
\end{align}}\\
&In this case $\vec{N}$ is similar to $\vec{N}=\myvec{0&0\\1&0}$ via the matrix $\vec{P}=\myvec{c&0\\0&1}$\\
&\\
\hline
&\\
Assuming $\myvec{a\\c}$ as&Therefore
$\vec{N}=\myvec{0&b\\0&d}$\\
the zero vector&\\
&\parbox{6cm}{\begin{align}
\mbox{Then }\vec{N}^2&=0\\
\implies d^2&=0\\
\therefore d&=0\\
\mbox{Thus }\vec{N}&=\myvec{0&b\\0&0}
\end{align}}\\
&In this case $\vec{N}$ is similar to $\vec{N}=\myvec{0&0\\b&0}$ via the matrix $\vec{P}=\myvec{0&1\\1&0}$,\\
&which is similar to $\myvec{0&0\\1&0}$ as above.\\
&\\
\hline
&\\
Hence& we can assume neither $\myvec{a\\c}$ or $\myvec{b\\d}$ is the zero vector.\\
&\\
%\hline
%\pagebreak
\hline
&\\
Consequences of &Since they are linearly dependent we can assume,\\
linear&\\
independence&\\
&\parbox{6cm}{\begin{align}
\myvec{b\\d}&=x\myvec{a\\c}\\
\therefore \vec{N}&=\myvec{a&ax\\c&cx}\\
\therefore \vec{N}^2&=0\\
\implies a(a+cx)&=0\\
c(a+cx)&=0\\
ax(a+cx)&=0\\
cx(a+cx)&=0
\end{align}}\\
\hline
&\\
Proof that $\vec{N}$ is&We know that at least one of a or c is not zero.\\
similar over $\mathbb{C}$ to&If a = 0 then c $\neq$ 0, it must be that x = 0.\\
$\myvec{0&0\\1&0}$&So in this case $\vec{N}=\myvec{0&0\\c&0}$ which is similar to $\myvec{0&0\\1&0}$ as before.\\
&\\
&\parbox{6cm}{\begin{align}
\mbox{If } a &\neq0\\
\mbox{then }x &\neq0\\
\mbox{else }a(a+cx)&=0\\
\implies a&=0\\
\mbox{Thus }a+cx&=0\\
\mbox{Hence }\vec{N}&=\myvec{a&ax\\ \frac{-a}{x}&-a}
\end{align}}\\
 &This is similar to $\myvec{a&a\\-a&-a}$ via $\vec{P}=\myvec{\sqrt{x}&0\\0&\frac{1}{\sqrt{x}}}$.\\
 &And $\myvec{a&a\\-a&-a}$ is similar to $\myvec{0&0\\-a&0}$ via $\vec{P}=\myvec{-1&-1\\1&0}$\\
 &And this finally is similar to $\myvec{0&0\\1&0}$ as before.\\
 &\\
\hline
&\\
Conclusion &Thus either $\vec{N}$ = 0 or $\vec{N}$ is similar over $\mathbb{C}$ to $\myvec{0&0\\1&0}$.\\
&\\
\hline
\caption{Solution summary}
\label{eq:solutions/6/2/11/table:1}
\end{longtable}

%
\item Let $\mathbb{V}$ be a vector space which is spanned by the rows of matrix
\begin{align}
    \vec{A} = \myvec{3&21&0&9&0\\1&7&-1&-2&-1\\2&14&0&6&1\\6&42&-1&13&0}\label{eq:solutions/2/5/6/eq:mat}
\end{align}
\begin{enumerate}[label=\alph*.]
\item Find a basis for $\mathbb{V}$
\item Tell which vectors \myvec{x_1\\x_2\\x_3\\x_4\\x_5} are elements of $\mathbb{V}$ 
\item If \myvec{x_1\\x_2\\x_3\\x_4\\x_5} is in $\mathbb{V}$ ,what are its coordinates in the basis chosen in part(a)?
\end{enumerate}
%
\solution
See Table \ref{eq:solutions/6/2/11/table:1}
\onecolumn
\begin{longtable}{|l|l|}
\hline
\multirow{3}{*}{} & \\
Statement&Solution\\
\hline
&\parbox{6cm}{\begin{align}
\mbox{Let }\vec{N}&=\myvec{a&b\\c&d}\\
\mbox{Since }\vec{N}^2&=0
\end{align}}\\
&If $\myvec{a\\c},\myvec{b\\d}$ are linearly independent then $\vec{N}$ is diagonalizable to $\myvec{0&0\\0&0}$.\\
&\parbox{6cm}{\begin{align}
\mbox{If }\vec{P}\vec{N}\vec{P}^{-1}=0\\
\mbox{then }\vec{N}=\vec{P}^{-1}\vec{0}\vec{P}=0
\end{align}}\\
Proof that&So in this case $\vec{N}$ itself is the zero matrix.\\
$\vec{N}=0$&This contradicts the assumption that $\myvec{a\\c},\myvec{b\\d}$ are linearly independent.\\
&$\therefore$ we can assume that $\myvec{a\\c},\myvec{b\\d}$ are linearly dependent if both are\\
&equal to the zero vector\\
&\parbox{6cm}
{\begin{align}
   \mbox{then } \vec{N} &= 0.
\end{align}}\\
%\hline
%\pagebreak
\hline
&\\
&Therefore we can assume at least one vector is non-zero.\\
Assuming $\myvec{b\\d}$ as&Therefore
$\vec{N}=\myvec{a&0\\c&0}$\\
the zero vector&\\
&\parbox{6cm}{\begin{align}
\mbox{So }\vec{N}^2&=0\\
\implies a^2&=0\\
\therefore a&=0\\
\mbox{Thus }\vec{N}&=\myvec{a&0\\c&0}
\end{align}}\\
&In this case $\vec{N}$ is similar to $\vec{N}=\myvec{0&0\\1&0}$ via the matrix $\vec{P}=\myvec{c&0\\0&1}$\\
&\\
\hline
&\\
Assuming $\myvec{a\\c}$ as&Therefore
$\vec{N}=\myvec{0&b\\0&d}$\\
the zero vector&\\
&\parbox{6cm}{\begin{align}
\mbox{Then }\vec{N}^2&=0\\
\implies d^2&=0\\
\therefore d&=0\\
\mbox{Thus }\vec{N}&=\myvec{0&b\\0&0}
\end{align}}\\
&In this case $\vec{N}$ is similar to $\vec{N}=\myvec{0&0\\b&0}$ via the matrix $\vec{P}=\myvec{0&1\\1&0}$,\\
&which is similar to $\myvec{0&0\\1&0}$ as above.\\
&\\
\hline
&\\
Hence& we can assume neither $\myvec{a\\c}$ or $\myvec{b\\d}$ is the zero vector.\\
&\\
%\hline
%\pagebreak
\hline
&\\
Consequences of &Since they are linearly dependent we can assume,\\
linear&\\
independence&\\
&\parbox{6cm}{\begin{align}
\myvec{b\\d}&=x\myvec{a\\c}\\
\therefore \vec{N}&=\myvec{a&ax\\c&cx}\\
\therefore \vec{N}^2&=0\\
\implies a(a+cx)&=0\\
c(a+cx)&=0\\
ax(a+cx)&=0\\
cx(a+cx)&=0
\end{align}}\\
\hline
&\\
Proof that $\vec{N}$ is&We know that at least one of a or c is not zero.\\
similar over $\mathbb{C}$ to&If a = 0 then c $\neq$ 0, it must be that x = 0.\\
$\myvec{0&0\\1&0}$&So in this case $\vec{N}=\myvec{0&0\\c&0}$ which is similar to $\myvec{0&0\\1&0}$ as before.\\
&\\
&\parbox{6cm}{\begin{align}
\mbox{If } a &\neq0\\
\mbox{then }x &\neq0\\
\mbox{else }a(a+cx)&=0\\
\implies a&=0\\
\mbox{Thus }a+cx&=0\\
\mbox{Hence }\vec{N}&=\myvec{a&ax\\ \frac{-a}{x}&-a}
\end{align}}\\
 &This is similar to $\myvec{a&a\\-a&-a}$ via $\vec{P}=\myvec{\sqrt{x}&0\\0&\frac{1}{\sqrt{x}}}$.\\
 &And $\myvec{a&a\\-a&-a}$ is similar to $\myvec{0&0\\-a&0}$ via $\vec{P}=\myvec{-1&-1\\1&0}$\\
 &And this finally is similar to $\myvec{0&0\\1&0}$ as before.\\
 &\\
\hline
&\\
Conclusion &Thus either $\vec{N}$ = 0 or $\vec{N}$ is similar over $\mathbb{C}$ to $\myvec{0&0\\1&0}$.\\
&\\
\hline
\caption{Solution summary}
\label{eq:solutions/6/2/11/table:1}
\end{longtable}


\end{enumerate}

\section{Linear Transformations}
\subsection{Linear Transformations}
\renewcommand{\theequation}{\theenumi}
\renewcommand{\thefigure}{\theenumi}
\begin{enumerate}[label=\thesubsection.\arabic*.,ref=\thesubsection.\theenumi]
\numberwithin{equation}{enumi}
\numberwithin{figure}{enumi}
%
\item  Find weather given  functions \textbf{T} from $\mathbb{R}^2$ into $\mathbb{R}^2$ are linear transformations or not

\begin{enumerate}
\item 
\begin{align}
    T\myvec{x_1\\x_2}=\myvec{1+x_1\\x_2}
\label{eq:solutions/3/1/1/a/1}
\end{align}
\solution
See Table \ref{eq:solutions/6/2/11/table:1}
\onecolumn
\begin{longtable}{|l|l|}
\hline
\multirow{3}{*}{} & \\
Statement&Solution\\
\hline
&\parbox{6cm}{\begin{align}
\mbox{Let }\vec{N}&=\myvec{a&b\\c&d}\\
\mbox{Since }\vec{N}^2&=0
\end{align}}\\
&If $\myvec{a\\c},\myvec{b\\d}$ are linearly independent then $\vec{N}$ is diagonalizable to $\myvec{0&0\\0&0}$.\\
&\parbox{6cm}{\begin{align}
\mbox{If }\vec{P}\vec{N}\vec{P}^{-1}=0\\
\mbox{then }\vec{N}=\vec{P}^{-1}\vec{0}\vec{P}=0
\end{align}}\\
Proof that&So in this case $\vec{N}$ itself is the zero matrix.\\
$\vec{N}=0$&This contradicts the assumption that $\myvec{a\\c},\myvec{b\\d}$ are linearly independent.\\
&$\therefore$ we can assume that $\myvec{a\\c},\myvec{b\\d}$ are linearly dependent if both are\\
&equal to the zero vector\\
&\parbox{6cm}
{\begin{align}
   \mbox{then } \vec{N} &= 0.
\end{align}}\\
%\hline
%\pagebreak
\hline
&\\
&Therefore we can assume at least one vector is non-zero.\\
Assuming $\myvec{b\\d}$ as&Therefore
$\vec{N}=\myvec{a&0\\c&0}$\\
the zero vector&\\
&\parbox{6cm}{\begin{align}
\mbox{So }\vec{N}^2&=0\\
\implies a^2&=0\\
\therefore a&=0\\
\mbox{Thus }\vec{N}&=\myvec{a&0\\c&0}
\end{align}}\\
&In this case $\vec{N}$ is similar to $\vec{N}=\myvec{0&0\\1&0}$ via the matrix $\vec{P}=\myvec{c&0\\0&1}$\\
&\\
\hline
&\\
Assuming $\myvec{a\\c}$ as&Therefore
$\vec{N}=\myvec{0&b\\0&d}$\\
the zero vector&\\
&\parbox{6cm}{\begin{align}
\mbox{Then }\vec{N}^2&=0\\
\implies d^2&=0\\
\therefore d&=0\\
\mbox{Thus }\vec{N}&=\myvec{0&b\\0&0}
\end{align}}\\
&In this case $\vec{N}$ is similar to $\vec{N}=\myvec{0&0\\b&0}$ via the matrix $\vec{P}=\myvec{0&1\\1&0}$,\\
&which is similar to $\myvec{0&0\\1&0}$ as above.\\
&\\
\hline
&\\
Hence& we can assume neither $\myvec{a\\c}$ or $\myvec{b\\d}$ is the zero vector.\\
&\\
%\hline
%\pagebreak
\hline
&\\
Consequences of &Since they are linearly dependent we can assume,\\
linear&\\
independence&\\
&\parbox{6cm}{\begin{align}
\myvec{b\\d}&=x\myvec{a\\c}\\
\therefore \vec{N}&=\myvec{a&ax\\c&cx}\\
\therefore \vec{N}^2&=0\\
\implies a(a+cx)&=0\\
c(a+cx)&=0\\
ax(a+cx)&=0\\
cx(a+cx)&=0
\end{align}}\\
\hline
&\\
Proof that $\vec{N}$ is&We know that at least one of a or c is not zero.\\
similar over $\mathbb{C}$ to&If a = 0 then c $\neq$ 0, it must be that x = 0.\\
$\myvec{0&0\\1&0}$&So in this case $\vec{N}=\myvec{0&0\\c&0}$ which is similar to $\myvec{0&0\\1&0}$ as before.\\
&\\
&\parbox{6cm}{\begin{align}
\mbox{If } a &\neq0\\
\mbox{then }x &\neq0\\
\mbox{else }a(a+cx)&=0\\
\implies a&=0\\
\mbox{Thus }a+cx&=0\\
\mbox{Hence }\vec{N}&=\myvec{a&ax\\ \frac{-a}{x}&-a}
\end{align}}\\
 &This is similar to $\myvec{a&a\\-a&-a}$ via $\vec{P}=\myvec{\sqrt{x}&0\\0&\frac{1}{\sqrt{x}}}$.\\
 &And $\myvec{a&a\\-a&-a}$ is similar to $\myvec{0&0\\-a&0}$ via $\vec{P}=\myvec{-1&-1\\1&0}$\\
 &And this finally is similar to $\myvec{0&0\\1&0}$ as before.\\
 &\\
\hline
&\\
Conclusion &Thus either $\vec{N}$ = 0 or $\vec{N}$ is similar over $\mathbb{C}$ to $\myvec{0&0\\1&0}$.\\
&\\
\hline
\caption{Solution summary}
\label{eq:solutions/6/2/11/table:1}
\end{longtable}

%
\item \begin{align}
\vec{T}\myvec{x_1\\x_2}= \myvec{x_2\\ x_1} \label{eq:solutions/3/1/1/b/0}
\end{align}
Does function $\vec{T}$ from $\mathbb{R}^2$ into $\mathbb{R}^2$ is Linear Transformation.

\solution
See Table \ref{eq:solutions/6/2/11/table:1}
\onecolumn
\begin{longtable}{|l|l|}
\hline
\multirow{3}{*}{} & \\
Statement&Solution\\
\hline
&\parbox{6cm}{\begin{align}
\mbox{Let }\vec{N}&=\myvec{a&b\\c&d}\\
\mbox{Since }\vec{N}^2&=0
\end{align}}\\
&If $\myvec{a\\c},\myvec{b\\d}$ are linearly independent then $\vec{N}$ is diagonalizable to $\myvec{0&0\\0&0}$.\\
&\parbox{6cm}{\begin{align}
\mbox{If }\vec{P}\vec{N}\vec{P}^{-1}=0\\
\mbox{then }\vec{N}=\vec{P}^{-1}\vec{0}\vec{P}=0
\end{align}}\\
Proof that&So in this case $\vec{N}$ itself is the zero matrix.\\
$\vec{N}=0$&This contradicts the assumption that $\myvec{a\\c},\myvec{b\\d}$ are linearly independent.\\
&$\therefore$ we can assume that $\myvec{a\\c},\myvec{b\\d}$ are linearly dependent if both are\\
&equal to the zero vector\\
&\parbox{6cm}
{\begin{align}
   \mbox{then } \vec{N} &= 0.
\end{align}}\\
%\hline
%\pagebreak
\hline
&\\
&Therefore we can assume at least one vector is non-zero.\\
Assuming $\myvec{b\\d}$ as&Therefore
$\vec{N}=\myvec{a&0\\c&0}$\\
the zero vector&\\
&\parbox{6cm}{\begin{align}
\mbox{So }\vec{N}^2&=0\\
\implies a^2&=0\\
\therefore a&=0\\
\mbox{Thus }\vec{N}&=\myvec{a&0\\c&0}
\end{align}}\\
&In this case $\vec{N}$ is similar to $\vec{N}=\myvec{0&0\\1&0}$ via the matrix $\vec{P}=\myvec{c&0\\0&1}$\\
&\\
\hline
&\\
Assuming $\myvec{a\\c}$ as&Therefore
$\vec{N}=\myvec{0&b\\0&d}$\\
the zero vector&\\
&\parbox{6cm}{\begin{align}
\mbox{Then }\vec{N}^2&=0\\
\implies d^2&=0\\
\therefore d&=0\\
\mbox{Thus }\vec{N}&=\myvec{0&b\\0&0}
\end{align}}\\
&In this case $\vec{N}$ is similar to $\vec{N}=\myvec{0&0\\b&0}$ via the matrix $\vec{P}=\myvec{0&1\\1&0}$,\\
&which is similar to $\myvec{0&0\\1&0}$ as above.\\
&\\
\hline
&\\
Hence& we can assume neither $\myvec{a\\c}$ or $\myvec{b\\d}$ is the zero vector.\\
&\\
%\hline
%\pagebreak
\hline
&\\
Consequences of &Since they are linearly dependent we can assume,\\
linear&\\
independence&\\
&\parbox{6cm}{\begin{align}
\myvec{b\\d}&=x\myvec{a\\c}\\
\therefore \vec{N}&=\myvec{a&ax\\c&cx}\\
\therefore \vec{N}^2&=0\\
\implies a(a+cx)&=0\\
c(a+cx)&=0\\
ax(a+cx)&=0\\
cx(a+cx)&=0
\end{align}}\\
\hline
&\\
Proof that $\vec{N}$ is&We know that at least one of a or c is not zero.\\
similar over $\mathbb{C}$ to&If a = 0 then c $\neq$ 0, it must be that x = 0.\\
$\myvec{0&0\\1&0}$&So in this case $\vec{N}=\myvec{0&0\\c&0}$ which is similar to $\myvec{0&0\\1&0}$ as before.\\
&\\
&\parbox{6cm}{\begin{align}
\mbox{If } a &\neq0\\
\mbox{then }x &\neq0\\
\mbox{else }a(a+cx)&=0\\
\implies a&=0\\
\mbox{Thus }a+cx&=0\\
\mbox{Hence }\vec{N}&=\myvec{a&ax\\ \frac{-a}{x}&-a}
\end{align}}\\
 &This is similar to $\myvec{a&a\\-a&-a}$ via $\vec{P}=\myvec{\sqrt{x}&0\\0&\frac{1}{\sqrt{x}}}$.\\
 &And $\myvec{a&a\\-a&-a}$ is similar to $\myvec{0&0\\-a&0}$ via $\vec{P}=\myvec{-1&-1\\1&0}$\\
 &And this finally is similar to $\myvec{0&0\\1&0}$ as before.\\
 &\\
\hline
&\\
Conclusion &Thus either $\vec{N}$ = 0 or $\vec{N}$ is similar over $\mathbb{C}$ to $\myvec{0&0\\1&0}$.\\
&\\
\hline
\caption{Solution summary}
\label{eq:solutions/6/2/11/table:1}
\end{longtable}

%
\item 
\begin{align}
	\textbf{T}(x_1,x_2) = ({x_1}^2,x_2)
\end{align}

\solution
See Table \ref{eq:solutions/6/2/11/table:1}
\onecolumn
\begin{longtable}{|l|l|}
\hline
\multirow{3}{*}{} & \\
Statement&Solution\\
\hline
&\parbox{6cm}{\begin{align}
\mbox{Let }\vec{N}&=\myvec{a&b\\c&d}\\
\mbox{Since }\vec{N}^2&=0
\end{align}}\\
&If $\myvec{a\\c},\myvec{b\\d}$ are linearly independent then $\vec{N}$ is diagonalizable to $\myvec{0&0\\0&0}$.\\
&\parbox{6cm}{\begin{align}
\mbox{If }\vec{P}\vec{N}\vec{P}^{-1}=0\\
\mbox{then }\vec{N}=\vec{P}^{-1}\vec{0}\vec{P}=0
\end{align}}\\
Proof that&So in this case $\vec{N}$ itself is the zero matrix.\\
$\vec{N}=0$&This contradicts the assumption that $\myvec{a\\c},\myvec{b\\d}$ are linearly independent.\\
&$\therefore$ we can assume that $\myvec{a\\c},\myvec{b\\d}$ are linearly dependent if both are\\
&equal to the zero vector\\
&\parbox{6cm}
{\begin{align}
   \mbox{then } \vec{N} &= 0.
\end{align}}\\
%\hline
%\pagebreak
\hline
&\\
&Therefore we can assume at least one vector is non-zero.\\
Assuming $\myvec{b\\d}$ as&Therefore
$\vec{N}=\myvec{a&0\\c&0}$\\
the zero vector&\\
&\parbox{6cm}{\begin{align}
\mbox{So }\vec{N}^2&=0\\
\implies a^2&=0\\
\therefore a&=0\\
\mbox{Thus }\vec{N}&=\myvec{a&0\\c&0}
\end{align}}\\
&In this case $\vec{N}$ is similar to $\vec{N}=\myvec{0&0\\1&0}$ via the matrix $\vec{P}=\myvec{c&0\\0&1}$\\
&\\
\hline
&\\
Assuming $\myvec{a\\c}$ as&Therefore
$\vec{N}=\myvec{0&b\\0&d}$\\
the zero vector&\\
&\parbox{6cm}{\begin{align}
\mbox{Then }\vec{N}^2&=0\\
\implies d^2&=0\\
\therefore d&=0\\
\mbox{Thus }\vec{N}&=\myvec{0&b\\0&0}
\end{align}}\\
&In this case $\vec{N}$ is similar to $\vec{N}=\myvec{0&0\\b&0}$ via the matrix $\vec{P}=\myvec{0&1\\1&0}$,\\
&which is similar to $\myvec{0&0\\1&0}$ as above.\\
&\\
\hline
&\\
Hence& we can assume neither $\myvec{a\\c}$ or $\myvec{b\\d}$ is the zero vector.\\
&\\
%\hline
%\pagebreak
\hline
&\\
Consequences of &Since they are linearly dependent we can assume,\\
linear&\\
independence&\\
&\parbox{6cm}{\begin{align}
\myvec{b\\d}&=x\myvec{a\\c}\\
\therefore \vec{N}&=\myvec{a&ax\\c&cx}\\
\therefore \vec{N}^2&=0\\
\implies a(a+cx)&=0\\
c(a+cx)&=0\\
ax(a+cx)&=0\\
cx(a+cx)&=0
\end{align}}\\
\hline
&\\
Proof that $\vec{N}$ is&We know that at least one of a or c is not zero.\\
similar over $\mathbb{C}$ to&If a = 0 then c $\neq$ 0, it must be that x = 0.\\
$\myvec{0&0\\1&0}$&So in this case $\vec{N}=\myvec{0&0\\c&0}$ which is similar to $\myvec{0&0\\1&0}$ as before.\\
&\\
&\parbox{6cm}{\begin{align}
\mbox{If } a &\neq0\\
\mbox{then }x &\neq0\\
\mbox{else }a(a+cx)&=0\\
\implies a&=0\\
\mbox{Thus }a+cx&=0\\
\mbox{Hence }\vec{N}&=\myvec{a&ax\\ \frac{-a}{x}&-a}
\end{align}}\\
 &This is similar to $\myvec{a&a\\-a&-a}$ via $\vec{P}=\myvec{\sqrt{x}&0\\0&\frac{1}{\sqrt{x}}}$.\\
 &And $\myvec{a&a\\-a&-a}$ is similar to $\myvec{0&0\\-a&0}$ via $\vec{P}=\myvec{-1&-1\\1&0}$\\
 &And this finally is similar to $\myvec{0&0\\1&0}$ as before.\\
 &\\
\hline
&\\
Conclusion &Thus either $\vec{N}$ = 0 or $\vec{N}$ is similar over $\mathbb{C}$ to $\myvec{0&0\\1&0}$.\\
&\\
\hline
\caption{Solution summary}
\label{eq:solutions/6/2/11/table:1}
\end{longtable}

%
\item Is the following function $\vec{T}$ from $\mathbb{R}^2$ into $\mathbb{R}^2$  is linear transformation?
\begin{align}
\vec{T} 
\myvec{
x_1\\
x_2}
= 
\myvec{\sin (x_1)\\x_2} \nonumber\end{align}
%
%
\solution
See Table \ref{eq:solutions/6/2/11/table:1}
\onecolumn
\begin{longtable}{|l|l|}
\hline
\multirow{3}{*}{} & \\
Statement&Solution\\
\hline
&\parbox{6cm}{\begin{align}
\mbox{Let }\vec{N}&=\myvec{a&b\\c&d}\\
\mbox{Since }\vec{N}^2&=0
\end{align}}\\
&If $\myvec{a\\c},\myvec{b\\d}$ are linearly independent then $\vec{N}$ is diagonalizable to $\myvec{0&0\\0&0}$.\\
&\parbox{6cm}{\begin{align}
\mbox{If }\vec{P}\vec{N}\vec{P}^{-1}=0\\
\mbox{then }\vec{N}=\vec{P}^{-1}\vec{0}\vec{P}=0
\end{align}}\\
Proof that&So in this case $\vec{N}$ itself is the zero matrix.\\
$\vec{N}=0$&This contradicts the assumption that $\myvec{a\\c},\myvec{b\\d}$ are linearly independent.\\
&$\therefore$ we can assume that $\myvec{a\\c},\myvec{b\\d}$ are linearly dependent if both are\\
&equal to the zero vector\\
&\parbox{6cm}
{\begin{align}
   \mbox{then } \vec{N} &= 0.
\end{align}}\\
%\hline
%\pagebreak
\hline
&\\
&Therefore we can assume at least one vector is non-zero.\\
Assuming $\myvec{b\\d}$ as&Therefore
$\vec{N}=\myvec{a&0\\c&0}$\\
the zero vector&\\
&\parbox{6cm}{\begin{align}
\mbox{So }\vec{N}^2&=0\\
\implies a^2&=0\\
\therefore a&=0\\
\mbox{Thus }\vec{N}&=\myvec{a&0\\c&0}
\end{align}}\\
&In this case $\vec{N}$ is similar to $\vec{N}=\myvec{0&0\\1&0}$ via the matrix $\vec{P}=\myvec{c&0\\0&1}$\\
&\\
\hline
&\\
Assuming $\myvec{a\\c}$ as&Therefore
$\vec{N}=\myvec{0&b\\0&d}$\\
the zero vector&\\
&\parbox{6cm}{\begin{align}
\mbox{Then }\vec{N}^2&=0\\
\implies d^2&=0\\
\therefore d&=0\\
\mbox{Thus }\vec{N}&=\myvec{0&b\\0&0}
\end{align}}\\
&In this case $\vec{N}$ is similar to $\vec{N}=\myvec{0&0\\b&0}$ via the matrix $\vec{P}=\myvec{0&1\\1&0}$,\\
&which is similar to $\myvec{0&0\\1&0}$ as above.\\
&\\
\hline
&\\
Hence& we can assume neither $\myvec{a\\c}$ or $\myvec{b\\d}$ is the zero vector.\\
&\\
%\hline
%\pagebreak
\hline
&\\
Consequences of &Since they are linearly dependent we can assume,\\
linear&\\
independence&\\
&\parbox{6cm}{\begin{align}
\myvec{b\\d}&=x\myvec{a\\c}\\
\therefore \vec{N}&=\myvec{a&ax\\c&cx}\\
\therefore \vec{N}^2&=0\\
\implies a(a+cx)&=0\\
c(a+cx)&=0\\
ax(a+cx)&=0\\
cx(a+cx)&=0
\end{align}}\\
\hline
&\\
Proof that $\vec{N}$ is&We know that at least one of a or c is not zero.\\
similar over $\mathbb{C}$ to&If a = 0 then c $\neq$ 0, it must be that x = 0.\\
$\myvec{0&0\\1&0}$&So in this case $\vec{N}=\myvec{0&0\\c&0}$ which is similar to $\myvec{0&0\\1&0}$ as before.\\
&\\
&\parbox{6cm}{\begin{align}
\mbox{If } a &\neq0\\
\mbox{then }x &\neq0\\
\mbox{else }a(a+cx)&=0\\
\implies a&=0\\
\mbox{Thus }a+cx&=0\\
\mbox{Hence }\vec{N}&=\myvec{a&ax\\ \frac{-a}{x}&-a}
\end{align}}\\
 &This is similar to $\myvec{a&a\\-a&-a}$ via $\vec{P}=\myvec{\sqrt{x}&0\\0&\frac{1}{\sqrt{x}}}$.\\
 &And $\myvec{a&a\\-a&-a}$ is similar to $\myvec{0&0\\-a&0}$ via $\vec{P}=\myvec{-1&-1\\1&0}$\\
 &And this finally is similar to $\myvec{0&0\\1&0}$ as before.\\
 &\\
\hline
&\\
Conclusion &Thus either $\vec{N}$ = 0 or $\vec{N}$ is similar over $\mathbb{C}$ to $\myvec{0&0\\1&0}$.\\
&\\
\hline
\caption{Solution summary}
\label{eq:solutions/6/2/11/table:1}
\end{longtable}

%
\item Verify whether $T(x_1, x_2)  = (x_1 - x_2, 0)$ is a linear transformation or not.
%
\solution
See Table \ref{eq:solutions/6/2/11/table:1}
\onecolumn
\begin{longtable}{|l|l|}
\hline
\multirow{3}{*}{} & \\
Statement&Solution\\
\hline
&\parbox{6cm}{\begin{align}
\mbox{Let }\vec{N}&=\myvec{a&b\\c&d}\\
\mbox{Since }\vec{N}^2&=0
\end{align}}\\
&If $\myvec{a\\c},\myvec{b\\d}$ are linearly independent then $\vec{N}$ is diagonalizable to $\myvec{0&0\\0&0}$.\\
&\parbox{6cm}{\begin{align}
\mbox{If }\vec{P}\vec{N}\vec{P}^{-1}=0\\
\mbox{then }\vec{N}=\vec{P}^{-1}\vec{0}\vec{P}=0
\end{align}}\\
Proof that&So in this case $\vec{N}$ itself is the zero matrix.\\
$\vec{N}=0$&This contradicts the assumption that $\myvec{a\\c},\myvec{b\\d}$ are linearly independent.\\
&$\therefore$ we can assume that $\myvec{a\\c},\myvec{b\\d}$ are linearly dependent if both are\\
&equal to the zero vector\\
&\parbox{6cm}
{\begin{align}
   \mbox{then } \vec{N} &= 0.
\end{align}}\\
%\hline
%\pagebreak
\hline
&\\
&Therefore we can assume at least one vector is non-zero.\\
Assuming $\myvec{b\\d}$ as&Therefore
$\vec{N}=\myvec{a&0\\c&0}$\\
the zero vector&\\
&\parbox{6cm}{\begin{align}
\mbox{So }\vec{N}^2&=0\\
\implies a^2&=0\\
\therefore a&=0\\
\mbox{Thus }\vec{N}&=\myvec{a&0\\c&0}
\end{align}}\\
&In this case $\vec{N}$ is similar to $\vec{N}=\myvec{0&0\\1&0}$ via the matrix $\vec{P}=\myvec{c&0\\0&1}$\\
&\\
\hline
&\\
Assuming $\myvec{a\\c}$ as&Therefore
$\vec{N}=\myvec{0&b\\0&d}$\\
the zero vector&\\
&\parbox{6cm}{\begin{align}
\mbox{Then }\vec{N}^2&=0\\
\implies d^2&=0\\
\therefore d&=0\\
\mbox{Thus }\vec{N}&=\myvec{0&b\\0&0}
\end{align}}\\
&In this case $\vec{N}$ is similar to $\vec{N}=\myvec{0&0\\b&0}$ via the matrix $\vec{P}=\myvec{0&1\\1&0}$,\\
&which is similar to $\myvec{0&0\\1&0}$ as above.\\
&\\
\hline
&\\
Hence& we can assume neither $\myvec{a\\c}$ or $\myvec{b\\d}$ is the zero vector.\\
&\\
%\hline
%\pagebreak
\hline
&\\
Consequences of &Since they are linearly dependent we can assume,\\
linear&\\
independence&\\
&\parbox{6cm}{\begin{align}
\myvec{b\\d}&=x\myvec{a\\c}\\
\therefore \vec{N}&=\myvec{a&ax\\c&cx}\\
\therefore \vec{N}^2&=0\\
\implies a(a+cx)&=0\\
c(a+cx)&=0\\
ax(a+cx)&=0\\
cx(a+cx)&=0
\end{align}}\\
\hline
&\\
Proof that $\vec{N}$ is&We know that at least one of a or c is not zero.\\
similar over $\mathbb{C}$ to&If a = 0 then c $\neq$ 0, it must be that x = 0.\\
$\myvec{0&0\\1&0}$&So in this case $\vec{N}=\myvec{0&0\\c&0}$ which is similar to $\myvec{0&0\\1&0}$ as before.\\
&\\
&\parbox{6cm}{\begin{align}
\mbox{If } a &\neq0\\
\mbox{then }x &\neq0\\
\mbox{else }a(a+cx)&=0\\
\implies a&=0\\
\mbox{Thus }a+cx&=0\\
\mbox{Hence }\vec{N}&=\myvec{a&ax\\ \frac{-a}{x}&-a}
\end{align}}\\
 &This is similar to $\myvec{a&a\\-a&-a}$ via $\vec{P}=\myvec{\sqrt{x}&0\\0&\frac{1}{\sqrt{x}}}$.\\
 &And $\myvec{a&a\\-a&-a}$ is similar to $\myvec{0&0\\-a&0}$ via $\vec{P}=\myvec{-1&-1\\1&0}$\\
 &And this finally is similar to $\myvec{0&0\\1&0}$ as before.\\
 &\\
\hline
&\\
Conclusion &Thus either $\vec{N}$ = 0 or $\vec{N}$ is similar over $\mathbb{C}$ to $\myvec{0&0\\1&0}$.\\
&\\
\hline
\caption{Solution summary}
\label{eq:solutions/6/2/11/table:1}
\end{longtable}


\end{enumerate}
%
\item Find the range, rank, null space, and nullity for the zero transformation and the identity transformation on a finite-dimensional space V. 
%
\\
\solution
See Table \ref{eq:solutions/6/2/11/table:1}
\onecolumn
\begin{longtable}{|l|l|}
\hline
\multirow{3}{*}{} & \\
Statement&Solution\\
\hline
&\parbox{6cm}{\begin{align}
\mbox{Let }\vec{N}&=\myvec{a&b\\c&d}\\
\mbox{Since }\vec{N}^2&=0
\end{align}}\\
&If $\myvec{a\\c},\myvec{b\\d}$ are linearly independent then $\vec{N}$ is diagonalizable to $\myvec{0&0\\0&0}$.\\
&\parbox{6cm}{\begin{align}
\mbox{If }\vec{P}\vec{N}\vec{P}^{-1}=0\\
\mbox{then }\vec{N}=\vec{P}^{-1}\vec{0}\vec{P}=0
\end{align}}\\
Proof that&So in this case $\vec{N}$ itself is the zero matrix.\\
$\vec{N}=0$&This contradicts the assumption that $\myvec{a\\c},\myvec{b\\d}$ are linearly independent.\\
&$\therefore$ we can assume that $\myvec{a\\c},\myvec{b\\d}$ are linearly dependent if both are\\
&equal to the zero vector\\
&\parbox{6cm}
{\begin{align}
   \mbox{then } \vec{N} &= 0.
\end{align}}\\
%\hline
%\pagebreak
\hline
&\\
&Therefore we can assume at least one vector is non-zero.\\
Assuming $\myvec{b\\d}$ as&Therefore
$\vec{N}=\myvec{a&0\\c&0}$\\
the zero vector&\\
&\parbox{6cm}{\begin{align}
\mbox{So }\vec{N}^2&=0\\
\implies a^2&=0\\
\therefore a&=0\\
\mbox{Thus }\vec{N}&=\myvec{a&0\\c&0}
\end{align}}\\
&In this case $\vec{N}$ is similar to $\vec{N}=\myvec{0&0\\1&0}$ via the matrix $\vec{P}=\myvec{c&0\\0&1}$\\
&\\
\hline
&\\
Assuming $\myvec{a\\c}$ as&Therefore
$\vec{N}=\myvec{0&b\\0&d}$\\
the zero vector&\\
&\parbox{6cm}{\begin{align}
\mbox{Then }\vec{N}^2&=0\\
\implies d^2&=0\\
\therefore d&=0\\
\mbox{Thus }\vec{N}&=\myvec{0&b\\0&0}
\end{align}}\\
&In this case $\vec{N}$ is similar to $\vec{N}=\myvec{0&0\\b&0}$ via the matrix $\vec{P}=\myvec{0&1\\1&0}$,\\
&which is similar to $\myvec{0&0\\1&0}$ as above.\\
&\\
\hline
&\\
Hence& we can assume neither $\myvec{a\\c}$ or $\myvec{b\\d}$ is the zero vector.\\
&\\
%\hline
%\pagebreak
\hline
&\\
Consequences of &Since they are linearly dependent we can assume,\\
linear&\\
independence&\\
&\parbox{6cm}{\begin{align}
\myvec{b\\d}&=x\myvec{a\\c}\\
\therefore \vec{N}&=\myvec{a&ax\\c&cx}\\
\therefore \vec{N}^2&=0\\
\implies a(a+cx)&=0\\
c(a+cx)&=0\\
ax(a+cx)&=0\\
cx(a+cx)&=0
\end{align}}\\
\hline
&\\
Proof that $\vec{N}$ is&We know that at least one of a or c is not zero.\\
similar over $\mathbb{C}$ to&If a = 0 then c $\neq$ 0, it must be that x = 0.\\
$\myvec{0&0\\1&0}$&So in this case $\vec{N}=\myvec{0&0\\c&0}$ which is similar to $\myvec{0&0\\1&0}$ as before.\\
&\\
&\parbox{6cm}{\begin{align}
\mbox{If } a &\neq0\\
\mbox{then }x &\neq0\\
\mbox{else }a(a+cx)&=0\\
\implies a&=0\\
\mbox{Thus }a+cx&=0\\
\mbox{Hence }\vec{N}&=\myvec{a&ax\\ \frac{-a}{x}&-a}
\end{align}}\\
 &This is similar to $\myvec{a&a\\-a&-a}$ via $\vec{P}=\myvec{\sqrt{x}&0\\0&\frac{1}{\sqrt{x}}}$.\\
 &And $\myvec{a&a\\-a&-a}$ is similar to $\myvec{0&0\\-a&0}$ via $\vec{P}=\myvec{-1&-1\\1&0}$\\
 &And this finally is similar to $\myvec{0&0\\1&0}$ as before.\\
 &\\
\hline
&\\
Conclusion &Thus either $\vec{N}$ = 0 or $\vec{N}$ is similar over $\mathbb{C}$ to $\myvec{0&0\\1&0}$.\\
&\\
\hline
\caption{Solution summary}
\label{eq:solutions/6/2/11/table:1}
\end{longtable}


\item \begin{enumerate}[label=\emph{\alph*)}]
\item
Let $\vec{F}$ be a field and let $\vec{V}$ be the space of polynomial functions $f$ from $\vec{F}$
into $\vec{F}$, given by
\begin{align*}
	f(x)=c_0+c_1x+\cdots +c_nx^n
\end{align*}
Let $\vec{D}$ be a linear differentiation transformation defined as
\begin{align*}
	(\vec{D}f)(x)=\frac{df(x)}{dx}
\end{align*}
Then find the range and null space of 
$\vec{D}$.\\
\item
Let $\vec{R}$ be the field of real numbers and let $\vec{V}$ be the space of all functions from 
$\vec{R}$ into $\vec{R}$ which are continous. Let $\vec{T}$ be linear transformation defined by
\begin{align*}
	(\vec{T}f)(x)=\int_{0}^{x} f(t)\,dt
\end{align*}
Find the range and null space of $\vec{T}$.
\end{enumerate}
%
\solution
See Table \ref{eq:solutions/6/2/11/table:1}
\onecolumn
\begin{longtable}{|l|l|}
\hline
\multirow{3}{*}{} & \\
Statement&Solution\\
\hline
&\parbox{6cm}{\begin{align}
\mbox{Let }\vec{N}&=\myvec{a&b\\c&d}\\
\mbox{Since }\vec{N}^2&=0
\end{align}}\\
&If $\myvec{a\\c},\myvec{b\\d}$ are linearly independent then $\vec{N}$ is diagonalizable to $\myvec{0&0\\0&0}$.\\
&\parbox{6cm}{\begin{align}
\mbox{If }\vec{P}\vec{N}\vec{P}^{-1}=0\\
\mbox{then }\vec{N}=\vec{P}^{-1}\vec{0}\vec{P}=0
\end{align}}\\
Proof that&So in this case $\vec{N}$ itself is the zero matrix.\\
$\vec{N}=0$&This contradicts the assumption that $\myvec{a\\c},\myvec{b\\d}$ are linearly independent.\\
&$\therefore$ we can assume that $\myvec{a\\c},\myvec{b\\d}$ are linearly dependent if both are\\
&equal to the zero vector\\
&\parbox{6cm}
{\begin{align}
   \mbox{then } \vec{N} &= 0.
\end{align}}\\
%\hline
%\pagebreak
\hline
&\\
&Therefore we can assume at least one vector is non-zero.\\
Assuming $\myvec{b\\d}$ as&Therefore
$\vec{N}=\myvec{a&0\\c&0}$\\
the zero vector&\\
&\parbox{6cm}{\begin{align}
\mbox{So }\vec{N}^2&=0\\
\implies a^2&=0\\
\therefore a&=0\\
\mbox{Thus }\vec{N}&=\myvec{a&0\\c&0}
\end{align}}\\
&In this case $\vec{N}$ is similar to $\vec{N}=\myvec{0&0\\1&0}$ via the matrix $\vec{P}=\myvec{c&0\\0&1}$\\
&\\
\hline
&\\
Assuming $\myvec{a\\c}$ as&Therefore
$\vec{N}=\myvec{0&b\\0&d}$\\
the zero vector&\\
&\parbox{6cm}{\begin{align}
\mbox{Then }\vec{N}^2&=0\\
\implies d^2&=0\\
\therefore d&=0\\
\mbox{Thus }\vec{N}&=\myvec{0&b\\0&0}
\end{align}}\\
&In this case $\vec{N}$ is similar to $\vec{N}=\myvec{0&0\\b&0}$ via the matrix $\vec{P}=\myvec{0&1\\1&0}$,\\
&which is similar to $\myvec{0&0\\1&0}$ as above.\\
&\\
\hline
&\\
Hence& we can assume neither $\myvec{a\\c}$ or $\myvec{b\\d}$ is the zero vector.\\
&\\
%\hline
%\pagebreak
\hline
&\\
Consequences of &Since they are linearly dependent we can assume,\\
linear&\\
independence&\\
&\parbox{6cm}{\begin{align}
\myvec{b\\d}&=x\myvec{a\\c}\\
\therefore \vec{N}&=\myvec{a&ax\\c&cx}\\
\therefore \vec{N}^2&=0\\
\implies a(a+cx)&=0\\
c(a+cx)&=0\\
ax(a+cx)&=0\\
cx(a+cx)&=0
\end{align}}\\
\hline
&\\
Proof that $\vec{N}$ is&We know that at least one of a or c is not zero.\\
similar over $\mathbb{C}$ to&If a = 0 then c $\neq$ 0, it must be that x = 0.\\
$\myvec{0&0\\1&0}$&So in this case $\vec{N}=\myvec{0&0\\c&0}$ which is similar to $\myvec{0&0\\1&0}$ as before.\\
&\\
&\parbox{6cm}{\begin{align}
\mbox{If } a &\neq0\\
\mbox{then }x &\neq0\\
\mbox{else }a(a+cx)&=0\\
\implies a&=0\\
\mbox{Thus }a+cx&=0\\
\mbox{Hence }\vec{N}&=\myvec{a&ax\\ \frac{-a}{x}&-a}
\end{align}}\\
 &This is similar to $\myvec{a&a\\-a&-a}$ via $\vec{P}=\myvec{\sqrt{x}&0\\0&\frac{1}{\sqrt{x}}}$.\\
 &And $\myvec{a&a\\-a&-a}$ is similar to $\myvec{0&0\\-a&0}$ via $\vec{P}=\myvec{-1&-1\\1&0}$\\
 &And this finally is similar to $\myvec{0&0\\1&0}$ as before.\\
 &\\
\hline
&\\
Conclusion &Thus either $\vec{N}$ = 0 or $\vec{N}$ is similar over $\mathbb{C}$ to $\myvec{0&0\\1&0}$.\\
&\\
\hline
\caption{Solution summary}
\label{eq:solutions/6/2/11/table:1}
\end{longtable}

%
\item Describe explicitly the linear transformation T from $\mathbb{F}^2$ into $\mathbb{F}^2$ such that  $T(\epsilon_1 )=(a,b),T(\epsilon_2 )=(c,d)$.
%
\solution
See Table \ref{eq:solutions/6/2/11/table:1}
\onecolumn
\begin{longtable}{|l|l|}
\hline
\multirow{3}{*}{} & \\
Statement&Solution\\
\hline
&\parbox{6cm}{\begin{align}
\mbox{Let }\vec{N}&=\myvec{a&b\\c&d}\\
\mbox{Since }\vec{N}^2&=0
\end{align}}\\
&If $\myvec{a\\c},\myvec{b\\d}$ are linearly independent then $\vec{N}$ is diagonalizable to $\myvec{0&0\\0&0}$.\\
&\parbox{6cm}{\begin{align}
\mbox{If }\vec{P}\vec{N}\vec{P}^{-1}=0\\
\mbox{then }\vec{N}=\vec{P}^{-1}\vec{0}\vec{P}=0
\end{align}}\\
Proof that&So in this case $\vec{N}$ itself is the zero matrix.\\
$\vec{N}=0$&This contradicts the assumption that $\myvec{a\\c},\myvec{b\\d}$ are linearly independent.\\
&$\therefore$ we can assume that $\myvec{a\\c},\myvec{b\\d}$ are linearly dependent if both are\\
&equal to the zero vector\\
&\parbox{6cm}
{\begin{align}
   \mbox{then } \vec{N} &= 0.
\end{align}}\\
%\hline
%\pagebreak
\hline
&\\
&Therefore we can assume at least one vector is non-zero.\\
Assuming $\myvec{b\\d}$ as&Therefore
$\vec{N}=\myvec{a&0\\c&0}$\\
the zero vector&\\
&\parbox{6cm}{\begin{align}
\mbox{So }\vec{N}^2&=0\\
\implies a^2&=0\\
\therefore a&=0\\
\mbox{Thus }\vec{N}&=\myvec{a&0\\c&0}
\end{align}}\\
&In this case $\vec{N}$ is similar to $\vec{N}=\myvec{0&0\\1&0}$ via the matrix $\vec{P}=\myvec{c&0\\0&1}$\\
&\\
\hline
&\\
Assuming $\myvec{a\\c}$ as&Therefore
$\vec{N}=\myvec{0&b\\0&d}$\\
the zero vector&\\
&\parbox{6cm}{\begin{align}
\mbox{Then }\vec{N}^2&=0\\
\implies d^2&=0\\
\therefore d&=0\\
\mbox{Thus }\vec{N}&=\myvec{0&b\\0&0}
\end{align}}\\
&In this case $\vec{N}$ is similar to $\vec{N}=\myvec{0&0\\b&0}$ via the matrix $\vec{P}=\myvec{0&1\\1&0}$,\\
&which is similar to $\myvec{0&0\\1&0}$ as above.\\
&\\
\hline
&\\
Hence& we can assume neither $\myvec{a\\c}$ or $\myvec{b\\d}$ is the zero vector.\\
&\\
%\hline
%\pagebreak
\hline
&\\
Consequences of &Since they are linearly dependent we can assume,\\
linear&\\
independence&\\
&\parbox{6cm}{\begin{align}
\myvec{b\\d}&=x\myvec{a\\c}\\
\therefore \vec{N}&=\myvec{a&ax\\c&cx}\\
\therefore \vec{N}^2&=0\\
\implies a(a+cx)&=0\\
c(a+cx)&=0\\
ax(a+cx)&=0\\
cx(a+cx)&=0
\end{align}}\\
\hline
&\\
Proof that $\vec{N}$ is&We know that at least one of a or c is not zero.\\
similar over $\mathbb{C}$ to&If a = 0 then c $\neq$ 0, it must be that x = 0.\\
$\myvec{0&0\\1&0}$&So in this case $\vec{N}=\myvec{0&0\\c&0}$ which is similar to $\myvec{0&0\\1&0}$ as before.\\
&\\
&\parbox{6cm}{\begin{align}
\mbox{If } a &\neq0\\
\mbox{then }x &\neq0\\
\mbox{else }a(a+cx)&=0\\
\implies a&=0\\
\mbox{Thus }a+cx&=0\\
\mbox{Hence }\vec{N}&=\myvec{a&ax\\ \frac{-a}{x}&-a}
\end{align}}\\
 &This is similar to $\myvec{a&a\\-a&-a}$ via $\vec{P}=\myvec{\sqrt{x}&0\\0&\frac{1}{\sqrt{x}}}$.\\
 &And $\myvec{a&a\\-a&-a}$ is similar to $\myvec{0&0\\-a&0}$ via $\vec{P}=\myvec{-1&-1\\1&0}$\\
 &And this finally is similar to $\myvec{0&0\\1&0}$ as before.\\
 &\\
\hline
&\\
Conclusion &Thus either $\vec{N}$ = 0 or $\vec{N}$ is similar over $\mathbb{C}$ to $\myvec{0&0\\1&0}$.\\
&\\
\hline
\caption{Solution summary}
\label{eq:solutions/6/2/11/table:1}
\end{longtable}

%
\item Let $\mathbb{F}$ be a subfield of the complex numbers and let $\mathbb{T}$ be the function from $\mathbb{F}^3$ into $\mathbb{F}^3$ defined by 
\begin{align}
    \mathbb{T}(x_1,x_2,x_3)=\\(x_1-x_2+2x_3,2x_1+x_2,-x_1-2x_2+2x_3)
\end{align}
\begin{enumerate}
\item  Verify that $\mathbb{T}$ is a linear transformation.\\
\item  If $(a,b,c)$ is a vector in  $\mathbb{F}^3$, what are the conditions on $a,b,c$ that the vector be in the range of  $\mathbb{T}$ ?What is the rank of  $\mathbb{T}$?\\
\item  What are the conditions on $a,b,c$ that $(a,b,c)$ be in the null space of  
$\mathbb{T}$? What is the nullity of  $\mathbb{T}$?
\end{enumerate}
\solution
See Table \ref{eq:solutions/6/2/11/table:1}
\onecolumn
\begin{longtable}{|l|l|}
\hline
\multirow{3}{*}{} & \\
Statement&Solution\\
\hline
&\parbox{6cm}{\begin{align}
\mbox{Let }\vec{N}&=\myvec{a&b\\c&d}\\
\mbox{Since }\vec{N}^2&=0
\end{align}}\\
&If $\myvec{a\\c},\myvec{b\\d}$ are linearly independent then $\vec{N}$ is diagonalizable to $\myvec{0&0\\0&0}$.\\
&\parbox{6cm}{\begin{align}
\mbox{If }\vec{P}\vec{N}\vec{P}^{-1}=0\\
\mbox{then }\vec{N}=\vec{P}^{-1}\vec{0}\vec{P}=0
\end{align}}\\
Proof that&So in this case $\vec{N}$ itself is the zero matrix.\\
$\vec{N}=0$&This contradicts the assumption that $\myvec{a\\c},\myvec{b\\d}$ are linearly independent.\\
&$\therefore$ we can assume that $\myvec{a\\c},\myvec{b\\d}$ are linearly dependent if both are\\
&equal to the zero vector\\
&\parbox{6cm}
{\begin{align}
   \mbox{then } \vec{N} &= 0.
\end{align}}\\
%\hline
%\pagebreak
\hline
&\\
&Therefore we can assume at least one vector is non-zero.\\
Assuming $\myvec{b\\d}$ as&Therefore
$\vec{N}=\myvec{a&0\\c&0}$\\
the zero vector&\\
&\parbox{6cm}{\begin{align}
\mbox{So }\vec{N}^2&=0\\
\implies a^2&=0\\
\therefore a&=0\\
\mbox{Thus }\vec{N}&=\myvec{a&0\\c&0}
\end{align}}\\
&In this case $\vec{N}$ is similar to $\vec{N}=\myvec{0&0\\1&0}$ via the matrix $\vec{P}=\myvec{c&0\\0&1}$\\
&\\
\hline
&\\
Assuming $\myvec{a\\c}$ as&Therefore
$\vec{N}=\myvec{0&b\\0&d}$\\
the zero vector&\\
&\parbox{6cm}{\begin{align}
\mbox{Then }\vec{N}^2&=0\\
\implies d^2&=0\\
\therefore d&=0\\
\mbox{Thus }\vec{N}&=\myvec{0&b\\0&0}
\end{align}}\\
&In this case $\vec{N}$ is similar to $\vec{N}=\myvec{0&0\\b&0}$ via the matrix $\vec{P}=\myvec{0&1\\1&0}$,\\
&which is similar to $\myvec{0&0\\1&0}$ as above.\\
&\\
\hline
&\\
Hence& we can assume neither $\myvec{a\\c}$ or $\myvec{b\\d}$ is the zero vector.\\
&\\
%\hline
%\pagebreak
\hline
&\\
Consequences of &Since they are linearly dependent we can assume,\\
linear&\\
independence&\\
&\parbox{6cm}{\begin{align}
\myvec{b\\d}&=x\myvec{a\\c}\\
\therefore \vec{N}&=\myvec{a&ax\\c&cx}\\
\therefore \vec{N}^2&=0\\
\implies a(a+cx)&=0\\
c(a+cx)&=0\\
ax(a+cx)&=0\\
cx(a+cx)&=0
\end{align}}\\
\hline
&\\
Proof that $\vec{N}$ is&We know that at least one of a or c is not zero.\\
similar over $\mathbb{C}$ to&If a = 0 then c $\neq$ 0, it must be that x = 0.\\
$\myvec{0&0\\1&0}$&So in this case $\vec{N}=\myvec{0&0\\c&0}$ which is similar to $\myvec{0&0\\1&0}$ as before.\\
&\\
&\parbox{6cm}{\begin{align}
\mbox{If } a &\neq0\\
\mbox{then }x &\neq0\\
\mbox{else }a(a+cx)&=0\\
\implies a&=0\\
\mbox{Thus }a+cx&=0\\
\mbox{Hence }\vec{N}&=\myvec{a&ax\\ \frac{-a}{x}&-a}
\end{align}}\\
 &This is similar to $\myvec{a&a\\-a&-a}$ via $\vec{P}=\myvec{\sqrt{x}&0\\0&\frac{1}{\sqrt{x}}}$.\\
 &And $\myvec{a&a\\-a&-a}$ is similar to $\myvec{0&0\\-a&0}$ via $\vec{P}=\myvec{-1&-1\\1&0}$\\
 &And this finally is similar to $\myvec{0&0\\1&0}$ as before.\\
 &\\
\hline
&\\
Conclusion &Thus either $\vec{N}$ = 0 or $\vec{N}$ is similar over $\mathbb{C}$ to $\myvec{0&0\\1&0}$.\\
&\\
\hline
\caption{Solution summary}
\label{eq:solutions/6/2/11/table:1}
\end{longtable}

\item Describe explicitly a linear transformation from $R^{3}$ into $R^{3}$ which has as its range the subspace spanned by \myvec{1&0&-1} and \myvec{1&2&2}.  
%
\\
\solution
See Table \ref{eq:solutions/6/2/11/table:1}
\onecolumn
\begin{longtable}{|l|l|}
\hline
\multirow{3}{*}{} & \\
Statement&Solution\\
\hline
&\parbox{6cm}{\begin{align}
\mbox{Let }\vec{N}&=\myvec{a&b\\c&d}\\
\mbox{Since }\vec{N}^2&=0
\end{align}}\\
&If $\myvec{a\\c},\myvec{b\\d}$ are linearly independent then $\vec{N}$ is diagonalizable to $\myvec{0&0\\0&0}$.\\
&\parbox{6cm}{\begin{align}
\mbox{If }\vec{P}\vec{N}\vec{P}^{-1}=0\\
\mbox{then }\vec{N}=\vec{P}^{-1}\vec{0}\vec{P}=0
\end{align}}\\
Proof that&So in this case $\vec{N}$ itself is the zero matrix.\\
$\vec{N}=0$&This contradicts the assumption that $\myvec{a\\c},\myvec{b\\d}$ are linearly independent.\\
&$\therefore$ we can assume that $\myvec{a\\c},\myvec{b\\d}$ are linearly dependent if both are\\
&equal to the zero vector\\
&\parbox{6cm}
{\begin{align}
   \mbox{then } \vec{N} &= 0.
\end{align}}\\
%\hline
%\pagebreak
\hline
&\\
&Therefore we can assume at least one vector is non-zero.\\
Assuming $\myvec{b\\d}$ as&Therefore
$\vec{N}=\myvec{a&0\\c&0}$\\
the zero vector&\\
&\parbox{6cm}{\begin{align}
\mbox{So }\vec{N}^2&=0\\
\implies a^2&=0\\
\therefore a&=0\\
\mbox{Thus }\vec{N}&=\myvec{a&0\\c&0}
\end{align}}\\
&In this case $\vec{N}$ is similar to $\vec{N}=\myvec{0&0\\1&0}$ via the matrix $\vec{P}=\myvec{c&0\\0&1}$\\
&\\
\hline
&\\
Assuming $\myvec{a\\c}$ as&Therefore
$\vec{N}=\myvec{0&b\\0&d}$\\
the zero vector&\\
&\parbox{6cm}{\begin{align}
\mbox{Then }\vec{N}^2&=0\\
\implies d^2&=0\\
\therefore d&=0\\
\mbox{Thus }\vec{N}&=\myvec{0&b\\0&0}
\end{align}}\\
&In this case $\vec{N}$ is similar to $\vec{N}=\myvec{0&0\\b&0}$ via the matrix $\vec{P}=\myvec{0&1\\1&0}$,\\
&which is similar to $\myvec{0&0\\1&0}$ as above.\\
&\\
\hline
&\\
Hence& we can assume neither $\myvec{a\\c}$ or $\myvec{b\\d}$ is the zero vector.\\
&\\
%\hline
%\pagebreak
\hline
&\\
Consequences of &Since they are linearly dependent we can assume,\\
linear&\\
independence&\\
&\parbox{6cm}{\begin{align}
\myvec{b\\d}&=x\myvec{a\\c}\\
\therefore \vec{N}&=\myvec{a&ax\\c&cx}\\
\therefore \vec{N}^2&=0\\
\implies a(a+cx)&=0\\
c(a+cx)&=0\\
ax(a+cx)&=0\\
cx(a+cx)&=0
\end{align}}\\
\hline
&\\
Proof that $\vec{N}$ is&We know that at least one of a or c is not zero.\\
similar over $\mathbb{C}$ to&If a = 0 then c $\neq$ 0, it must be that x = 0.\\
$\myvec{0&0\\1&0}$&So in this case $\vec{N}=\myvec{0&0\\c&0}$ which is similar to $\myvec{0&0\\1&0}$ as before.\\
&\\
&\parbox{6cm}{\begin{align}
\mbox{If } a &\neq0\\
\mbox{then }x &\neq0\\
\mbox{else }a(a+cx)&=0\\
\implies a&=0\\
\mbox{Thus }a+cx&=0\\
\mbox{Hence }\vec{N}&=\myvec{a&ax\\ \frac{-a}{x}&-a}
\end{align}}\\
 &This is similar to $\myvec{a&a\\-a&-a}$ via $\vec{P}=\myvec{\sqrt{x}&0\\0&\frac{1}{\sqrt{x}}}$.\\
 &And $\myvec{a&a\\-a&-a}$ is similar to $\myvec{0&0\\-a&0}$ via $\vec{P}=\myvec{-1&-1\\1&0}$\\
 &And this finally is similar to $\myvec{0&0\\1&0}$ as before.\\
 &\\
\hline
&\\
Conclusion &Thus either $\vec{N}$ = 0 or $\vec{N}$ is similar over $\mathbb{C}$ to $\myvec{0&0\\1&0}$.\\
&\\
\hline
\caption{Solution summary}
\label{eq:solutions/6/2/11/table:1}
\end{longtable}

\item Let $\vec{V}$ be the vector space of all $n \times n$ matrices over the field $\mathbb{F}$, and let $\vec{B}$ be a fixed $n \times n$ matrix. If a transformation $T$ defined as follows,
\begin{align*}
    T(\vec{A}) &= \vec{AB} - \vec{BA}
\end{align*}
Prove that $T$ is a linear transformation from $\vec{V}$ into $\vec{V}$
%
\solution
See Table \ref{eq:solutions/6/2/11/table:1}
\onecolumn
\begin{longtable}{|l|l|}
\hline
\multirow{3}{*}{} & \\
Statement&Solution\\
\hline
&\parbox{6cm}{\begin{align}
\mbox{Let }\vec{N}&=\myvec{a&b\\c&d}\\
\mbox{Since }\vec{N}^2&=0
\end{align}}\\
&If $\myvec{a\\c},\myvec{b\\d}$ are linearly independent then $\vec{N}$ is diagonalizable to $\myvec{0&0\\0&0}$.\\
&\parbox{6cm}{\begin{align}
\mbox{If }\vec{P}\vec{N}\vec{P}^{-1}=0\\
\mbox{then }\vec{N}=\vec{P}^{-1}\vec{0}\vec{P}=0
\end{align}}\\
Proof that&So in this case $\vec{N}$ itself is the zero matrix.\\
$\vec{N}=0$&This contradicts the assumption that $\myvec{a\\c},\myvec{b\\d}$ are linearly independent.\\
&$\therefore$ we can assume that $\myvec{a\\c},\myvec{b\\d}$ are linearly dependent if both are\\
&equal to the zero vector\\
&\parbox{6cm}
{\begin{align}
   \mbox{then } \vec{N} &= 0.
\end{align}}\\
%\hline
%\pagebreak
\hline
&\\
&Therefore we can assume at least one vector is non-zero.\\
Assuming $\myvec{b\\d}$ as&Therefore
$\vec{N}=\myvec{a&0\\c&0}$\\
the zero vector&\\
&\parbox{6cm}{\begin{align}
\mbox{So }\vec{N}^2&=0\\
\implies a^2&=0\\
\therefore a&=0\\
\mbox{Thus }\vec{N}&=\myvec{a&0\\c&0}
\end{align}}\\
&In this case $\vec{N}$ is similar to $\vec{N}=\myvec{0&0\\1&0}$ via the matrix $\vec{P}=\myvec{c&0\\0&1}$\\
&\\
\hline
&\\
Assuming $\myvec{a\\c}$ as&Therefore
$\vec{N}=\myvec{0&b\\0&d}$\\
the zero vector&\\
&\parbox{6cm}{\begin{align}
\mbox{Then }\vec{N}^2&=0\\
\implies d^2&=0\\
\therefore d&=0\\
\mbox{Thus }\vec{N}&=\myvec{0&b\\0&0}
\end{align}}\\
&In this case $\vec{N}$ is similar to $\vec{N}=\myvec{0&0\\b&0}$ via the matrix $\vec{P}=\myvec{0&1\\1&0}$,\\
&which is similar to $\myvec{0&0\\1&0}$ as above.\\
&\\
\hline
&\\
Hence& we can assume neither $\myvec{a\\c}$ or $\myvec{b\\d}$ is the zero vector.\\
&\\
%\hline
%\pagebreak
\hline
&\\
Consequences of &Since they are linearly dependent we can assume,\\
linear&\\
independence&\\
&\parbox{6cm}{\begin{align}
\myvec{b\\d}&=x\myvec{a\\c}\\
\therefore \vec{N}&=\myvec{a&ax\\c&cx}\\
\therefore \vec{N}^2&=0\\
\implies a(a+cx)&=0\\
c(a+cx)&=0\\
ax(a+cx)&=0\\
cx(a+cx)&=0
\end{align}}\\
\hline
&\\
Proof that $\vec{N}$ is&We know that at least one of a or c is not zero.\\
similar over $\mathbb{C}$ to&If a = 0 then c $\neq$ 0, it must be that x = 0.\\
$\myvec{0&0\\1&0}$&So in this case $\vec{N}=\myvec{0&0\\c&0}$ which is similar to $\myvec{0&0\\1&0}$ as before.\\
&\\
&\parbox{6cm}{\begin{align}
\mbox{If } a &\neq0\\
\mbox{then }x &\neq0\\
\mbox{else }a(a+cx)&=0\\
\implies a&=0\\
\mbox{Thus }a+cx&=0\\
\mbox{Hence }\vec{N}&=\myvec{a&ax\\ \frac{-a}{x}&-a}
\end{align}}\\
 &This is similar to $\myvec{a&a\\-a&-a}$ via $\vec{P}=\myvec{\sqrt{x}&0\\0&\frac{1}{\sqrt{x}}}$.\\
 &And $\myvec{a&a\\-a&-a}$ is similar to $\myvec{0&0\\-a&0}$ via $\vec{P}=\myvec{-1&-1\\1&0}$\\
 &And this finally is similar to $\myvec{0&0\\1&0}$ as before.\\
 &\\
\hline
&\\
Conclusion &Thus either $\vec{N}$ = 0 or $\vec{N}$ is similar over $\mathbb{C}$ to $\myvec{0&0\\1&0}$.\\
&\\
\hline
\caption{Solution summary}
\label{eq:solutions/6/2/11/table:1}
\end{longtable}

%
\item Let V be the set of all complex numbers regarded as a vector space over the field of real numbers(usual operations). Find a function from V into V which is a linear transformation on the above vector space, but which is not a linear transformation on $\mathbb{C}$ i.e., which is not complex linear.
%
\\
\solution
See Table \ref{eq:solutions/6/2/11/table:1}
\onecolumn
\begin{longtable}{|l|l|}
\hline
\multirow{3}{*}{} & \\
Statement&Solution\\
\hline
&\parbox{6cm}{\begin{align}
\mbox{Let }\vec{N}&=\myvec{a&b\\c&d}\\
\mbox{Since }\vec{N}^2&=0
\end{align}}\\
&If $\myvec{a\\c},\myvec{b\\d}$ are linearly independent then $\vec{N}$ is diagonalizable to $\myvec{0&0\\0&0}$.\\
&\parbox{6cm}{\begin{align}
\mbox{If }\vec{P}\vec{N}\vec{P}^{-1}=0\\
\mbox{then }\vec{N}=\vec{P}^{-1}\vec{0}\vec{P}=0
\end{align}}\\
Proof that&So in this case $\vec{N}$ itself is the zero matrix.\\
$\vec{N}=0$&This contradicts the assumption that $\myvec{a\\c},\myvec{b\\d}$ are linearly independent.\\
&$\therefore$ we can assume that $\myvec{a\\c},\myvec{b\\d}$ are linearly dependent if both are\\
&equal to the zero vector\\
&\parbox{6cm}
{\begin{align}
   \mbox{then } \vec{N} &= 0.
\end{align}}\\
%\hline
%\pagebreak
\hline
&\\
&Therefore we can assume at least one vector is non-zero.\\
Assuming $\myvec{b\\d}$ as&Therefore
$\vec{N}=\myvec{a&0\\c&0}$\\
the zero vector&\\
&\parbox{6cm}{\begin{align}
\mbox{So }\vec{N}^2&=0\\
\implies a^2&=0\\
\therefore a&=0\\
\mbox{Thus }\vec{N}&=\myvec{a&0\\c&0}
\end{align}}\\
&In this case $\vec{N}$ is similar to $\vec{N}=\myvec{0&0\\1&0}$ via the matrix $\vec{P}=\myvec{c&0\\0&1}$\\
&\\
\hline
&\\
Assuming $\myvec{a\\c}$ as&Therefore
$\vec{N}=\myvec{0&b\\0&d}$\\
the zero vector&\\
&\parbox{6cm}{\begin{align}
\mbox{Then }\vec{N}^2&=0\\
\implies d^2&=0\\
\therefore d&=0\\
\mbox{Thus }\vec{N}&=\myvec{0&b\\0&0}
\end{align}}\\
&In this case $\vec{N}$ is similar to $\vec{N}=\myvec{0&0\\b&0}$ via the matrix $\vec{P}=\myvec{0&1\\1&0}$,\\
&which is similar to $\myvec{0&0\\1&0}$ as above.\\
&\\
\hline
&\\
Hence& we can assume neither $\myvec{a\\c}$ or $\myvec{b\\d}$ is the zero vector.\\
&\\
%\hline
%\pagebreak
\hline
&\\
Consequences of &Since they are linearly dependent we can assume,\\
linear&\\
independence&\\
&\parbox{6cm}{\begin{align}
\myvec{b\\d}&=x\myvec{a\\c}\\
\therefore \vec{N}&=\myvec{a&ax\\c&cx}\\
\therefore \vec{N}^2&=0\\
\implies a(a+cx)&=0\\
c(a+cx)&=0\\
ax(a+cx)&=0\\
cx(a+cx)&=0
\end{align}}\\
\hline
&\\
Proof that $\vec{N}$ is&We know that at least one of a or c is not zero.\\
similar over $\mathbb{C}$ to&If a = 0 then c $\neq$ 0, it must be that x = 0.\\
$\myvec{0&0\\1&0}$&So in this case $\vec{N}=\myvec{0&0\\c&0}$ which is similar to $\myvec{0&0\\1&0}$ as before.\\
&\\
&\parbox{6cm}{\begin{align}
\mbox{If } a &\neq0\\
\mbox{then }x &\neq0\\
\mbox{else }a(a+cx)&=0\\
\implies a&=0\\
\mbox{Thus }a+cx&=0\\
\mbox{Hence }\vec{N}&=\myvec{a&ax\\ \frac{-a}{x}&-a}
\end{align}}\\
 &This is similar to $\myvec{a&a\\-a&-a}$ via $\vec{P}=\myvec{\sqrt{x}&0\\0&\frac{1}{\sqrt{x}}}$.\\
 &And $\myvec{a&a\\-a&-a}$ is similar to $\myvec{0&0\\-a&0}$ via $\vec{P}=\myvec{-1&-1\\1&0}$\\
 &And this finally is similar to $\myvec{0&0\\1&0}$ as before.\\
 &\\
\hline
&\\
Conclusion &Thus either $\vec{N}$ = 0 or $\vec{N}$ is similar over $\mathbb{C}$ to $\myvec{0&0\\1&0}$.\\
&\\
\hline
\caption{Solution summary}
\label{eq:solutions/6/2/11/table:1}
\end{longtable}

%
\item Let $\vec{V}$ be the space of $n \times 1$ matrices over $F$ and let $\vec{W}$ be the space of $m \times 1$ matrices over $F$. Let $\vec{A}$ be a fixed $m \times n$ matrix over $F$ and let $T$ be the linear transformation from $\vec{V}$ into $\vec{W}$ defined by $T(\vec{X}) = \vec{A}\vec{X}$. Prove that $T$ is the zero transformation if and only if $\vec{A}$ is the zero matrix.
%
\solution
See Table \ref{eq:solutions/6/2/11/table:1}
\onecolumn
\begin{longtable}{|l|l|}
\hline
\multirow{3}{*}{} & \\
Statement&Solution\\
\hline
&\parbox{6cm}{\begin{align}
\mbox{Let }\vec{N}&=\myvec{a&b\\c&d}\\
\mbox{Since }\vec{N}^2&=0
\end{align}}\\
&If $\myvec{a\\c},\myvec{b\\d}$ are linearly independent then $\vec{N}$ is diagonalizable to $\myvec{0&0\\0&0}$.\\
&\parbox{6cm}{\begin{align}
\mbox{If }\vec{P}\vec{N}\vec{P}^{-1}=0\\
\mbox{then }\vec{N}=\vec{P}^{-1}\vec{0}\vec{P}=0
\end{align}}\\
Proof that&So in this case $\vec{N}$ itself is the zero matrix.\\
$\vec{N}=0$&This contradicts the assumption that $\myvec{a\\c},\myvec{b\\d}$ are linearly independent.\\
&$\therefore$ we can assume that $\myvec{a\\c},\myvec{b\\d}$ are linearly dependent if both are\\
&equal to the zero vector\\
&\parbox{6cm}
{\begin{align}
   \mbox{then } \vec{N} &= 0.
\end{align}}\\
%\hline
%\pagebreak
\hline
&\\
&Therefore we can assume at least one vector is non-zero.\\
Assuming $\myvec{b\\d}$ as&Therefore
$\vec{N}=\myvec{a&0\\c&0}$\\
the zero vector&\\
&\parbox{6cm}{\begin{align}
\mbox{So }\vec{N}^2&=0\\
\implies a^2&=0\\
\therefore a&=0\\
\mbox{Thus }\vec{N}&=\myvec{a&0\\c&0}
\end{align}}\\
&In this case $\vec{N}$ is similar to $\vec{N}=\myvec{0&0\\1&0}$ via the matrix $\vec{P}=\myvec{c&0\\0&1}$\\
&\\
\hline
&\\
Assuming $\myvec{a\\c}$ as&Therefore
$\vec{N}=\myvec{0&b\\0&d}$\\
the zero vector&\\
&\parbox{6cm}{\begin{align}
\mbox{Then }\vec{N}^2&=0\\
\implies d^2&=0\\
\therefore d&=0\\
\mbox{Thus }\vec{N}&=\myvec{0&b\\0&0}
\end{align}}\\
&In this case $\vec{N}$ is similar to $\vec{N}=\myvec{0&0\\b&0}$ via the matrix $\vec{P}=\myvec{0&1\\1&0}$,\\
&which is similar to $\myvec{0&0\\1&0}$ as above.\\
&\\
\hline
&\\
Hence& we can assume neither $\myvec{a\\c}$ or $\myvec{b\\d}$ is the zero vector.\\
&\\
%\hline
%\pagebreak
\hline
&\\
Consequences of &Since they are linearly dependent we can assume,\\
linear&\\
independence&\\
&\parbox{6cm}{\begin{align}
\myvec{b\\d}&=x\myvec{a\\c}\\
\therefore \vec{N}&=\myvec{a&ax\\c&cx}\\
\therefore \vec{N}^2&=0\\
\implies a(a+cx)&=0\\
c(a+cx)&=0\\
ax(a+cx)&=0\\
cx(a+cx)&=0
\end{align}}\\
\hline
&\\
Proof that $\vec{N}$ is&We know that at least one of a or c is not zero.\\
similar over $\mathbb{C}$ to&If a = 0 then c $\neq$ 0, it must be that x = 0.\\
$\myvec{0&0\\1&0}$&So in this case $\vec{N}=\myvec{0&0\\c&0}$ which is similar to $\myvec{0&0\\1&0}$ as before.\\
&\\
&\parbox{6cm}{\begin{align}
\mbox{If } a &\neq0\\
\mbox{then }x &\neq0\\
\mbox{else }a(a+cx)&=0\\
\implies a&=0\\
\mbox{Thus }a+cx&=0\\
\mbox{Hence }\vec{N}&=\myvec{a&ax\\ \frac{-a}{x}&-a}
\end{align}}\\
 &This is similar to $\myvec{a&a\\-a&-a}$ via $\vec{P}=\myvec{\sqrt{x}&0\\0&\frac{1}{\sqrt{x}}}$.\\
 &And $\myvec{a&a\\-a&-a}$ is similar to $\myvec{0&0\\-a&0}$ via $\vec{P}=\myvec{-1&-1\\1&0}$\\
 &And this finally is similar to $\myvec{0&0\\1&0}$ as before.\\
 &\\
\hline
&\\
Conclusion &Thus either $\vec{N}$ = 0 or $\vec{N}$ is similar over $\mathbb{C}$ to $\myvec{0&0\\1&0}$.\\
&\\
\hline
\caption{Solution summary}
\label{eq:solutions/6/2/11/table:1}
\end{longtable}

%
\item Let $\vec{V}$ be an $n$-dimensional vector space over the field $\vec{F}$ and let $\vec{T}$ be a linear transformation from $\vec{V}$ into $\vec{V}$ such that the range and null space of $\vec{T}$ are identical. Prove that $n$ is even. (Can you give an example of such a linear transformation $\vec{T}$)?
%
\solution
See Table \ref{eq:solutions/6/2/11/table:1}
\onecolumn
\begin{longtable}{|l|l|}
\hline
\multirow{3}{*}{} & \\
Statement&Solution\\
\hline
&\parbox{6cm}{\begin{align}
\mbox{Let }\vec{N}&=\myvec{a&b\\c&d}\\
\mbox{Since }\vec{N}^2&=0
\end{align}}\\
&If $\myvec{a\\c},\myvec{b\\d}$ are linearly independent then $\vec{N}$ is diagonalizable to $\myvec{0&0\\0&0}$.\\
&\parbox{6cm}{\begin{align}
\mbox{If }\vec{P}\vec{N}\vec{P}^{-1}=0\\
\mbox{then }\vec{N}=\vec{P}^{-1}\vec{0}\vec{P}=0
\end{align}}\\
Proof that&So in this case $\vec{N}$ itself is the zero matrix.\\
$\vec{N}=0$&This contradicts the assumption that $\myvec{a\\c},\myvec{b\\d}$ are linearly independent.\\
&$\therefore$ we can assume that $\myvec{a\\c},\myvec{b\\d}$ are linearly dependent if both are\\
&equal to the zero vector\\
&\parbox{6cm}
{\begin{align}
   \mbox{then } \vec{N} &= 0.
\end{align}}\\
%\hline
%\pagebreak
\hline
&\\
&Therefore we can assume at least one vector is non-zero.\\
Assuming $\myvec{b\\d}$ as&Therefore
$\vec{N}=\myvec{a&0\\c&0}$\\
the zero vector&\\
&\parbox{6cm}{\begin{align}
\mbox{So }\vec{N}^2&=0\\
\implies a^2&=0\\
\therefore a&=0\\
\mbox{Thus }\vec{N}&=\myvec{a&0\\c&0}
\end{align}}\\
&In this case $\vec{N}$ is similar to $\vec{N}=\myvec{0&0\\1&0}$ via the matrix $\vec{P}=\myvec{c&0\\0&1}$\\
&\\
\hline
&\\
Assuming $\myvec{a\\c}$ as&Therefore
$\vec{N}=\myvec{0&b\\0&d}$\\
the zero vector&\\
&\parbox{6cm}{\begin{align}
\mbox{Then }\vec{N}^2&=0\\
\implies d^2&=0\\
\therefore d&=0\\
\mbox{Thus }\vec{N}&=\myvec{0&b\\0&0}
\end{align}}\\
&In this case $\vec{N}$ is similar to $\vec{N}=\myvec{0&0\\b&0}$ via the matrix $\vec{P}=\myvec{0&1\\1&0}$,\\
&which is similar to $\myvec{0&0\\1&0}$ as above.\\
&\\
\hline
&\\
Hence& we can assume neither $\myvec{a\\c}$ or $\myvec{b\\d}$ is the zero vector.\\
&\\
%\hline
%\pagebreak
\hline
&\\
Consequences of &Since they are linearly dependent we can assume,\\
linear&\\
independence&\\
&\parbox{6cm}{\begin{align}
\myvec{b\\d}&=x\myvec{a\\c}\\
\therefore \vec{N}&=\myvec{a&ax\\c&cx}\\
\therefore \vec{N}^2&=0\\
\implies a(a+cx)&=0\\
c(a+cx)&=0\\
ax(a+cx)&=0\\
cx(a+cx)&=0
\end{align}}\\
\hline
&\\
Proof that $\vec{N}$ is&We know that at least one of a or c is not zero.\\
similar over $\mathbb{C}$ to&If a = 0 then c $\neq$ 0, it must be that x = 0.\\
$\myvec{0&0\\1&0}$&So in this case $\vec{N}=\myvec{0&0\\c&0}$ which is similar to $\myvec{0&0\\1&0}$ as before.\\
&\\
&\parbox{6cm}{\begin{align}
\mbox{If } a &\neq0\\
\mbox{then }x &\neq0\\
\mbox{else }a(a+cx)&=0\\
\implies a&=0\\
\mbox{Thus }a+cx&=0\\
\mbox{Hence }\vec{N}&=\myvec{a&ax\\ \frac{-a}{x}&-a}
\end{align}}\\
 &This is similar to $\myvec{a&a\\-a&-a}$ via $\vec{P}=\myvec{\sqrt{x}&0\\0&\frac{1}{\sqrt{x}}}$.\\
 &And $\myvec{a&a\\-a&-a}$ is similar to $\myvec{0&0\\-a&0}$ via $\vec{P}=\myvec{-1&-1\\1&0}$\\
 &And this finally is similar to $\myvec{0&0\\1&0}$ as before.\\
 &\\
\hline
&\\
Conclusion &Thus either $\vec{N}$ = 0 or $\vec{N}$ is similar over $\mathbb{C}$ to $\myvec{0&0\\1&0}$.\\
&\\
\hline
\caption{Solution summary}
\label{eq:solutions/6/2/11/table:1}
\end{longtable}

\end{enumerate}

\subsection{The Algebra of Linear Transformations}
\renewcommand{\theequation}{\theenumi}
\renewcommand{\thefigure}{\theenumi}
\begin{enumerate}[label=\thesubsection.\arabic*.,ref=\thesubsection.\theenumi]
\numberwithin{equation}{enumi}
\numberwithin{figure}{enumi}
\numberwithin{table}{enumi}

%
\item Let T be the unique linear operator on $C^{3}$ for which 
   \begin{multline}
    \begin{aligned}
    T\brak{\epsilon_1}=\myvec{1&0&i} , T\brak{\epsilon_2}=\myvec{0&1&1} ,\\ T\brak{\epsilon_3}=\myvec{i&1&0} 
    \end{aligned}
    \end{multline}
    Is T invertible ?
%
\\
\solution
See Table \ref{eq:solutions/6/2/11/table:1}
\onecolumn
\begin{longtable}{|l|l|}
\hline
\multirow{3}{*}{} & \\
Statement&Solution\\
\hline
&\parbox{6cm}{\begin{align}
\mbox{Let }\vec{N}&=\myvec{a&b\\c&d}\\
\mbox{Since }\vec{N}^2&=0
\end{align}}\\
&If $\myvec{a\\c},\myvec{b\\d}$ are linearly independent then $\vec{N}$ is diagonalizable to $\myvec{0&0\\0&0}$.\\
&\parbox{6cm}{\begin{align}
\mbox{If }\vec{P}\vec{N}\vec{P}^{-1}=0\\
\mbox{then }\vec{N}=\vec{P}^{-1}\vec{0}\vec{P}=0
\end{align}}\\
Proof that&So in this case $\vec{N}$ itself is the zero matrix.\\
$\vec{N}=0$&This contradicts the assumption that $\myvec{a\\c},\myvec{b\\d}$ are linearly independent.\\
&$\therefore$ we can assume that $\myvec{a\\c},\myvec{b\\d}$ are linearly dependent if both are\\
&equal to the zero vector\\
&\parbox{6cm}
{\begin{align}
   \mbox{then } \vec{N} &= 0.
\end{align}}\\
%\hline
%\pagebreak
\hline
&\\
&Therefore we can assume at least one vector is non-zero.\\
Assuming $\myvec{b\\d}$ as&Therefore
$\vec{N}=\myvec{a&0\\c&0}$\\
the zero vector&\\
&\parbox{6cm}{\begin{align}
\mbox{So }\vec{N}^2&=0\\
\implies a^2&=0\\
\therefore a&=0\\
\mbox{Thus }\vec{N}&=\myvec{a&0\\c&0}
\end{align}}\\
&In this case $\vec{N}$ is similar to $\vec{N}=\myvec{0&0\\1&0}$ via the matrix $\vec{P}=\myvec{c&0\\0&1}$\\
&\\
\hline
&\\
Assuming $\myvec{a\\c}$ as&Therefore
$\vec{N}=\myvec{0&b\\0&d}$\\
the zero vector&\\
&\parbox{6cm}{\begin{align}
\mbox{Then }\vec{N}^2&=0\\
\implies d^2&=0\\
\therefore d&=0\\
\mbox{Thus }\vec{N}&=\myvec{0&b\\0&0}
\end{align}}\\
&In this case $\vec{N}$ is similar to $\vec{N}=\myvec{0&0\\b&0}$ via the matrix $\vec{P}=\myvec{0&1\\1&0}$,\\
&which is similar to $\myvec{0&0\\1&0}$ as above.\\
&\\
\hline
&\\
Hence& we can assume neither $\myvec{a\\c}$ or $\myvec{b\\d}$ is the zero vector.\\
&\\
%\hline
%\pagebreak
\hline
&\\
Consequences of &Since they are linearly dependent we can assume,\\
linear&\\
independence&\\
&\parbox{6cm}{\begin{align}
\myvec{b\\d}&=x\myvec{a\\c}\\
\therefore \vec{N}&=\myvec{a&ax\\c&cx}\\
\therefore \vec{N}^2&=0\\
\implies a(a+cx)&=0\\
c(a+cx)&=0\\
ax(a+cx)&=0\\
cx(a+cx)&=0
\end{align}}\\
\hline
&\\
Proof that $\vec{N}$ is&We know that at least one of a or c is not zero.\\
similar over $\mathbb{C}$ to&If a = 0 then c $\neq$ 0, it must be that x = 0.\\
$\myvec{0&0\\1&0}$&So in this case $\vec{N}=\myvec{0&0\\c&0}$ which is similar to $\myvec{0&0\\1&0}$ as before.\\
&\\
&\parbox{6cm}{\begin{align}
\mbox{If } a &\neq0\\
\mbox{then }x &\neq0\\
\mbox{else }a(a+cx)&=0\\
\implies a&=0\\
\mbox{Thus }a+cx&=0\\
\mbox{Hence }\vec{N}&=\myvec{a&ax\\ \frac{-a}{x}&-a}
\end{align}}\\
 &This is similar to $\myvec{a&a\\-a&-a}$ via $\vec{P}=\myvec{\sqrt{x}&0\\0&\frac{1}{\sqrt{x}}}$.\\
 &And $\myvec{a&a\\-a&-a}$ is similar to $\myvec{0&0\\-a&0}$ via $\vec{P}=\myvec{-1&-1\\1&0}$\\
 &And this finally is similar to $\myvec{0&0\\1&0}$ as before.\\
 &\\
\hline
&\\
Conclusion &Thus either $\vec{N}$ = 0 or $\vec{N}$ is similar over $\mathbb{C}$ to $\myvec{0&0\\1&0}$.\\
&\\
\hline
\caption{Solution summary}
\label{eq:solutions/6/2/11/table:1}
\end{longtable}

\item Let $\mathbb{C}$ be the complex vector space of $2\times2$ matrices with complex entries. Let
\begin{align}
	\vec{B} = \myvec{1 & -1 \\ -4 & 4}
\end{align}
and let $\vec{T}$ be the linear operator on $\mathbb{C}^{2\times2}$  defined by $\vec{T(A)} = \vec{BA}$. What is the rank of $\vec{T}$? Can you describe $\vec{T^2}$?   
%
\\
\solution
See Table \ref{eq:solutions/6/2/11/table:1}
\onecolumn
\begin{longtable}{|l|l|}
\hline
\multirow{3}{*}{} & \\
Statement&Solution\\
\hline
&\parbox{6cm}{\begin{align}
\mbox{Let }\vec{N}&=\myvec{a&b\\c&d}\\
\mbox{Since }\vec{N}^2&=0
\end{align}}\\
&If $\myvec{a\\c},\myvec{b\\d}$ are linearly independent then $\vec{N}$ is diagonalizable to $\myvec{0&0\\0&0}$.\\
&\parbox{6cm}{\begin{align}
\mbox{If }\vec{P}\vec{N}\vec{P}^{-1}=0\\
\mbox{then }\vec{N}=\vec{P}^{-1}\vec{0}\vec{P}=0
\end{align}}\\
Proof that&So in this case $\vec{N}$ itself is the zero matrix.\\
$\vec{N}=0$&This contradicts the assumption that $\myvec{a\\c},\myvec{b\\d}$ are linearly independent.\\
&$\therefore$ we can assume that $\myvec{a\\c},\myvec{b\\d}$ are linearly dependent if both are\\
&equal to the zero vector\\
&\parbox{6cm}
{\begin{align}
   \mbox{then } \vec{N} &= 0.
\end{align}}\\
%\hline
%\pagebreak
\hline
&\\
&Therefore we can assume at least one vector is non-zero.\\
Assuming $\myvec{b\\d}$ as&Therefore
$\vec{N}=\myvec{a&0\\c&0}$\\
the zero vector&\\
&\parbox{6cm}{\begin{align}
\mbox{So }\vec{N}^2&=0\\
\implies a^2&=0\\
\therefore a&=0\\
\mbox{Thus }\vec{N}&=\myvec{a&0\\c&0}
\end{align}}\\
&In this case $\vec{N}$ is similar to $\vec{N}=\myvec{0&0\\1&0}$ via the matrix $\vec{P}=\myvec{c&0\\0&1}$\\
&\\
\hline
&\\
Assuming $\myvec{a\\c}$ as&Therefore
$\vec{N}=\myvec{0&b\\0&d}$\\
the zero vector&\\
&\parbox{6cm}{\begin{align}
\mbox{Then }\vec{N}^2&=0\\
\implies d^2&=0\\
\therefore d&=0\\
\mbox{Thus }\vec{N}&=\myvec{0&b\\0&0}
\end{align}}\\
&In this case $\vec{N}$ is similar to $\vec{N}=\myvec{0&0\\b&0}$ via the matrix $\vec{P}=\myvec{0&1\\1&0}$,\\
&which is similar to $\myvec{0&0\\1&0}$ as above.\\
&\\
\hline
&\\
Hence& we can assume neither $\myvec{a\\c}$ or $\myvec{b\\d}$ is the zero vector.\\
&\\
%\hline
%\pagebreak
\hline
&\\
Consequences of &Since they are linearly dependent we can assume,\\
linear&\\
independence&\\
&\parbox{6cm}{\begin{align}
\myvec{b\\d}&=x\myvec{a\\c}\\
\therefore \vec{N}&=\myvec{a&ax\\c&cx}\\
\therefore \vec{N}^2&=0\\
\implies a(a+cx)&=0\\
c(a+cx)&=0\\
ax(a+cx)&=0\\
cx(a+cx)&=0
\end{align}}\\
\hline
&\\
Proof that $\vec{N}$ is&We know that at least one of a or c is not zero.\\
similar over $\mathbb{C}$ to&If a = 0 then c $\neq$ 0, it must be that x = 0.\\
$\myvec{0&0\\1&0}$&So in this case $\vec{N}=\myvec{0&0\\c&0}$ which is similar to $\myvec{0&0\\1&0}$ as before.\\
&\\
&\parbox{6cm}{\begin{align}
\mbox{If } a &\neq0\\
\mbox{then }x &\neq0\\
\mbox{else }a(a+cx)&=0\\
\implies a&=0\\
\mbox{Thus }a+cx&=0\\
\mbox{Hence }\vec{N}&=\myvec{a&ax\\ \frac{-a}{x}&-a}
\end{align}}\\
 &This is similar to $\myvec{a&a\\-a&-a}$ via $\vec{P}=\myvec{\sqrt{x}&0\\0&\frac{1}{\sqrt{x}}}$.\\
 &And $\myvec{a&a\\-a&-a}$ is similar to $\myvec{0&0\\-a&0}$ via $\vec{P}=\myvec{-1&-1\\1&0}$\\
 &And this finally is similar to $\myvec{0&0\\1&0}$ as before.\\
 &\\
\hline
&\\
Conclusion &Thus either $\vec{N}$ = 0 or $\vec{N}$ is similar over $\mathbb{C}$ to $\myvec{0&0\\1&0}$.\\
&\\
\hline
\caption{Solution summary}
\label{eq:solutions/6/2/11/table:1}
\end{longtable}

\item Let $\vec{A}$ be an $m \times n$ matrix with entries in $F$ and let $T$ be the linear transformation from $F^{n \times1 }$ into $F^{m \times l}$ defined by $T(\vec{X}) = \vec{A}\vec{X}$. Show that 
\begin{enumerate}
\item
if $m < n$ it may happen that $T$ is onto without being non-singular
\item
if $m>n$ we may have $T$ non-singular but not onto.
\\
\end{enumerate}
%
\solution
See Table \ref{eq:solutions/6/2/11/table:1}
\onecolumn
\begin{longtable}{|l|l|}
\hline
\multirow{3}{*}{} & \\
Statement&Solution\\
\hline
&\parbox{6cm}{\begin{align}
\mbox{Let }\vec{N}&=\myvec{a&b\\c&d}\\
\mbox{Since }\vec{N}^2&=0
\end{align}}\\
&If $\myvec{a\\c},\myvec{b\\d}$ are linearly independent then $\vec{N}$ is diagonalizable to $\myvec{0&0\\0&0}$.\\
&\parbox{6cm}{\begin{align}
\mbox{If }\vec{P}\vec{N}\vec{P}^{-1}=0\\
\mbox{then }\vec{N}=\vec{P}^{-1}\vec{0}\vec{P}=0
\end{align}}\\
Proof that&So in this case $\vec{N}$ itself is the zero matrix.\\
$\vec{N}=0$&This contradicts the assumption that $\myvec{a\\c},\myvec{b\\d}$ are linearly independent.\\
&$\therefore$ we can assume that $\myvec{a\\c},\myvec{b\\d}$ are linearly dependent if both are\\
&equal to the zero vector\\
&\parbox{6cm}
{\begin{align}
   \mbox{then } \vec{N} &= 0.
\end{align}}\\
%\hline
%\pagebreak
\hline
&\\
&Therefore we can assume at least one vector is non-zero.\\
Assuming $\myvec{b\\d}$ as&Therefore
$\vec{N}=\myvec{a&0\\c&0}$\\
the zero vector&\\
&\parbox{6cm}{\begin{align}
\mbox{So }\vec{N}^2&=0\\
\implies a^2&=0\\
\therefore a&=0\\
\mbox{Thus }\vec{N}&=\myvec{a&0\\c&0}
\end{align}}\\
&In this case $\vec{N}$ is similar to $\vec{N}=\myvec{0&0\\1&0}$ via the matrix $\vec{P}=\myvec{c&0\\0&1}$\\
&\\
\hline
&\\
Assuming $\myvec{a\\c}$ as&Therefore
$\vec{N}=\myvec{0&b\\0&d}$\\
the zero vector&\\
&\parbox{6cm}{\begin{align}
\mbox{Then }\vec{N}^2&=0\\
\implies d^2&=0\\
\therefore d&=0\\
\mbox{Thus }\vec{N}&=\myvec{0&b\\0&0}
\end{align}}\\
&In this case $\vec{N}$ is similar to $\vec{N}=\myvec{0&0\\b&0}$ via the matrix $\vec{P}=\myvec{0&1\\1&0}$,\\
&which is similar to $\myvec{0&0\\1&0}$ as above.\\
&\\
\hline
&\\
Hence& we can assume neither $\myvec{a\\c}$ or $\myvec{b\\d}$ is the zero vector.\\
&\\
%\hline
%\pagebreak
\hline
&\\
Consequences of &Since they are linearly dependent we can assume,\\
linear&\\
independence&\\
&\parbox{6cm}{\begin{align}
\myvec{b\\d}&=x\myvec{a\\c}\\
\therefore \vec{N}&=\myvec{a&ax\\c&cx}\\
\therefore \vec{N}^2&=0\\
\implies a(a+cx)&=0\\
c(a+cx)&=0\\
ax(a+cx)&=0\\
cx(a+cx)&=0
\end{align}}\\
\hline
&\\
Proof that $\vec{N}$ is&We know that at least one of a or c is not zero.\\
similar over $\mathbb{C}$ to&If a = 0 then c $\neq$ 0, it must be that x = 0.\\
$\myvec{0&0\\1&0}$&So in this case $\vec{N}=\myvec{0&0\\c&0}$ which is similar to $\myvec{0&0\\1&0}$ as before.\\
&\\
&\parbox{6cm}{\begin{align}
\mbox{If } a &\neq0\\
\mbox{then }x &\neq0\\
\mbox{else }a(a+cx)&=0\\
\implies a&=0\\
\mbox{Thus }a+cx&=0\\
\mbox{Hence }\vec{N}&=\myvec{a&ax\\ \frac{-a}{x}&-a}
\end{align}}\\
 &This is similar to $\myvec{a&a\\-a&-a}$ via $\vec{P}=\myvec{\sqrt{x}&0\\0&\frac{1}{\sqrt{x}}}$.\\
 &And $\myvec{a&a\\-a&-a}$ is similar to $\myvec{0&0\\-a&0}$ via $\vec{P}=\myvec{-1&-1\\1&0}$\\
 &And this finally is similar to $\myvec{0&0\\1&0}$ as before.\\
 &\\
\hline
&\\
Conclusion &Thus either $\vec{N}$ = 0 or $\vec{N}$ is similar over $\mathbb{C}$ to $\myvec{0&0\\1&0}$.\\
&\\
\hline
\caption{Solution summary}
\label{eq:solutions/6/2/11/table:1}
\end{longtable}

\end{enumerate}

\subsection{Isomorphism}
\renewcommand{\theequation}{\theenumi}
\renewcommand{\thefigure}{\theenumi}
\begin{enumerate}[label=\thesubsection.\arabic*.,ref=\thesubsection.\theenumi]
\numberwithin{equation}{enumi}
\numberwithin{figure}{enumi}
\numberwithin{table}{enumi}
%
\item Let $\vec{V}$ be the set of complex numbers and let $\vec{F}$ be the field of real numbers. With the usual operations, $\vec{V}$ is a vector space over $\vec{F}$. Describe explicitly an isomorphism of this space onto $\mathbb{R}^2$.
%
\\
%
\solution
See Table \ref{eq:solutions/6/2/11/table:1}
\onecolumn
\begin{longtable}{|l|l|}
\hline
\multirow{3}{*}{} & \\
Statement&Solution\\
\hline
&\parbox{6cm}{\begin{align}
\mbox{Let }\vec{N}&=\myvec{a&b\\c&d}\\
\mbox{Since }\vec{N}^2&=0
\end{align}}\\
&If $\myvec{a\\c},\myvec{b\\d}$ are linearly independent then $\vec{N}$ is diagonalizable to $\myvec{0&0\\0&0}$.\\
&\parbox{6cm}{\begin{align}
\mbox{If }\vec{P}\vec{N}\vec{P}^{-1}=0\\
\mbox{then }\vec{N}=\vec{P}^{-1}\vec{0}\vec{P}=0
\end{align}}\\
Proof that&So in this case $\vec{N}$ itself is the zero matrix.\\
$\vec{N}=0$&This contradicts the assumption that $\myvec{a\\c},\myvec{b\\d}$ are linearly independent.\\
&$\therefore$ we can assume that $\myvec{a\\c},\myvec{b\\d}$ are linearly dependent if both are\\
&equal to the zero vector\\
&\parbox{6cm}
{\begin{align}
   \mbox{then } \vec{N} &= 0.
\end{align}}\\
%\hline
%\pagebreak
\hline
&\\
&Therefore we can assume at least one vector is non-zero.\\
Assuming $\myvec{b\\d}$ as&Therefore
$\vec{N}=\myvec{a&0\\c&0}$\\
the zero vector&\\
&\parbox{6cm}{\begin{align}
\mbox{So }\vec{N}^2&=0\\
\implies a^2&=0\\
\therefore a&=0\\
\mbox{Thus }\vec{N}&=\myvec{a&0\\c&0}
\end{align}}\\
&In this case $\vec{N}$ is similar to $\vec{N}=\myvec{0&0\\1&0}$ via the matrix $\vec{P}=\myvec{c&0\\0&1}$\\
&\\
\hline
&\\
Assuming $\myvec{a\\c}$ as&Therefore
$\vec{N}=\myvec{0&b\\0&d}$\\
the zero vector&\\
&\parbox{6cm}{\begin{align}
\mbox{Then }\vec{N}^2&=0\\
\implies d^2&=0\\
\therefore d&=0\\
\mbox{Thus }\vec{N}&=\myvec{0&b\\0&0}
\end{align}}\\
&In this case $\vec{N}$ is similar to $\vec{N}=\myvec{0&0\\b&0}$ via the matrix $\vec{P}=\myvec{0&1\\1&0}$,\\
&which is similar to $\myvec{0&0\\1&0}$ as above.\\
&\\
\hline
&\\
Hence& we can assume neither $\myvec{a\\c}$ or $\myvec{b\\d}$ is the zero vector.\\
&\\
%\hline
%\pagebreak
\hline
&\\
Consequences of &Since they are linearly dependent we can assume,\\
linear&\\
independence&\\
&\parbox{6cm}{\begin{align}
\myvec{b\\d}&=x\myvec{a\\c}\\
\therefore \vec{N}&=\myvec{a&ax\\c&cx}\\
\therefore \vec{N}^2&=0\\
\implies a(a+cx)&=0\\
c(a+cx)&=0\\
ax(a+cx)&=0\\
cx(a+cx)&=0
\end{align}}\\
\hline
&\\
Proof that $\vec{N}$ is&We know that at least one of a or c is not zero.\\
similar over $\mathbb{C}$ to&If a = 0 then c $\neq$ 0, it must be that x = 0.\\
$\myvec{0&0\\1&0}$&So in this case $\vec{N}=\myvec{0&0\\c&0}$ which is similar to $\myvec{0&0\\1&0}$ as before.\\
&\\
&\parbox{6cm}{\begin{align}
\mbox{If } a &\neq0\\
\mbox{then }x &\neq0\\
\mbox{else }a(a+cx)&=0\\
\implies a&=0\\
\mbox{Thus }a+cx&=0\\
\mbox{Hence }\vec{N}&=\myvec{a&ax\\ \frac{-a}{x}&-a}
\end{align}}\\
 &This is similar to $\myvec{a&a\\-a&-a}$ via $\vec{P}=\myvec{\sqrt{x}&0\\0&\frac{1}{\sqrt{x}}}$.\\
 &And $\myvec{a&a\\-a&-a}$ is similar to $\myvec{0&0\\-a&0}$ via $\vec{P}=\myvec{-1&-1\\1&0}$\\
 &And this finally is similar to $\myvec{0&0\\1&0}$ as before.\\
 &\\
\hline
&\\
Conclusion &Thus either $\vec{N}$ = 0 or $\vec{N}$ is similar over $\mathbb{C}$ to $\myvec{0&0\\1&0}$.\\
&\\
\hline
\caption{Solution summary}
\label{eq:solutions/6/2/11/table:1}
\end{longtable}

\item Let V be a vector space over the field of complex numbers , and suppose there is an isomorphism T of V onto $C^{3}$. Let $\alpha_1$ , $\alpha_2$ , $\alpha_3$  , $\alpha_4$ be vectors in V such that 
   \begin{multline}
    \begin{aligned}
    T\brak{\alpha_1}=\myvec{1\\0\\i} , T\brak{\alpha_2}=\myvec{-2\\1+i\\0} ,\\ T\brak{\alpha_3}=\myvec{-1\\1\\1} , T\brak{\alpha_4}=\myvec{\sqrt{2}\\i\\3}
    \end{aligned}
    \end{multline}
\begin{enumerate}
\item     Is $\alpha_1$ in the subspace spanned by $\alpha_2$ and $\alpha_3$?
\\
%
\solution
See Table \ref{eq:solutions/6/2/11/table:1}
\onecolumn
\begin{longtable}{|l|l|}
\hline
\multirow{3}{*}{} & \\
Statement&Solution\\
\hline
&\parbox{6cm}{\begin{align}
\mbox{Let }\vec{N}&=\myvec{a&b\\c&d}\\
\mbox{Since }\vec{N}^2&=0
\end{align}}\\
&If $\myvec{a\\c},\myvec{b\\d}$ are linearly independent then $\vec{N}$ is diagonalizable to $\myvec{0&0\\0&0}$.\\
&\parbox{6cm}{\begin{align}
\mbox{If }\vec{P}\vec{N}\vec{P}^{-1}=0\\
\mbox{then }\vec{N}=\vec{P}^{-1}\vec{0}\vec{P}=0
\end{align}}\\
Proof that&So in this case $\vec{N}$ itself is the zero matrix.\\
$\vec{N}=0$&This contradicts the assumption that $\myvec{a\\c},\myvec{b\\d}$ are linearly independent.\\
&$\therefore$ we can assume that $\myvec{a\\c},\myvec{b\\d}$ are linearly dependent if both are\\
&equal to the zero vector\\
&\parbox{6cm}
{\begin{align}
   \mbox{then } \vec{N} &= 0.
\end{align}}\\
%\hline
%\pagebreak
\hline
&\\
&Therefore we can assume at least one vector is non-zero.\\
Assuming $\myvec{b\\d}$ as&Therefore
$\vec{N}=\myvec{a&0\\c&0}$\\
the zero vector&\\
&\parbox{6cm}{\begin{align}
\mbox{So }\vec{N}^2&=0\\
\implies a^2&=0\\
\therefore a&=0\\
\mbox{Thus }\vec{N}&=\myvec{a&0\\c&0}
\end{align}}\\
&In this case $\vec{N}$ is similar to $\vec{N}=\myvec{0&0\\1&0}$ via the matrix $\vec{P}=\myvec{c&0\\0&1}$\\
&\\
\hline
&\\
Assuming $\myvec{a\\c}$ as&Therefore
$\vec{N}=\myvec{0&b\\0&d}$\\
the zero vector&\\
&\parbox{6cm}{\begin{align}
\mbox{Then }\vec{N}^2&=0\\
\implies d^2&=0\\
\therefore d&=0\\
\mbox{Thus }\vec{N}&=\myvec{0&b\\0&0}
\end{align}}\\
&In this case $\vec{N}$ is similar to $\vec{N}=\myvec{0&0\\b&0}$ via the matrix $\vec{P}=\myvec{0&1\\1&0}$,\\
&which is similar to $\myvec{0&0\\1&0}$ as above.\\
&\\
\hline
&\\
Hence& we can assume neither $\myvec{a\\c}$ or $\myvec{b\\d}$ is the zero vector.\\
&\\
%\hline
%\pagebreak
\hline
&\\
Consequences of &Since they are linearly dependent we can assume,\\
linear&\\
independence&\\
&\parbox{6cm}{\begin{align}
\myvec{b\\d}&=x\myvec{a\\c}\\
\therefore \vec{N}&=\myvec{a&ax\\c&cx}\\
\therefore \vec{N}^2&=0\\
\implies a(a+cx)&=0\\
c(a+cx)&=0\\
ax(a+cx)&=0\\
cx(a+cx)&=0
\end{align}}\\
\hline
&\\
Proof that $\vec{N}$ is&We know that at least one of a or c is not zero.\\
similar over $\mathbb{C}$ to&If a = 0 then c $\neq$ 0, it must be that x = 0.\\
$\myvec{0&0\\1&0}$&So in this case $\vec{N}=\myvec{0&0\\c&0}$ which is similar to $\myvec{0&0\\1&0}$ as before.\\
&\\
&\parbox{6cm}{\begin{align}
\mbox{If } a &\neq0\\
\mbox{then }x &\neq0\\
\mbox{else }a(a+cx)&=0\\
\implies a&=0\\
\mbox{Thus }a+cx&=0\\
\mbox{Hence }\vec{N}&=\myvec{a&ax\\ \frac{-a}{x}&-a}
\end{align}}\\
 &This is similar to $\myvec{a&a\\-a&-a}$ via $\vec{P}=\myvec{\sqrt{x}&0\\0&\frac{1}{\sqrt{x}}}$.\\
 &And $\myvec{a&a\\-a&-a}$ is similar to $\myvec{0&0\\-a&0}$ via $\vec{P}=\myvec{-1&-1\\1&0}$\\
 &And this finally is similar to $\myvec{0&0\\1&0}$ as before.\\
 &\\
\hline
&\\
Conclusion &Thus either $\vec{N}$ = 0 or $\vec{N}$ is similar over $\mathbb{C}$ to $\myvec{0&0\\1&0}$.\\
&\\
\hline
\caption{Solution summary}
\label{eq:solutions/6/2/11/table:1}
\end{longtable}

\item Find a basis for the subspace of $V$ spanned by the 4 vectors $\alpha_i$.
%Let $V$ be a vector space over the field of complex numbers and suppose there is an isomorphism T of $V$ onto $C^3$. Let $\alpha_1$, $\alpha_2$, $\alpha_3$ and $\alpha_4$ be the vectors in $V$ such that:
%\begin{align}
%T\alpha_1 = \myvec{1\\0\\i}\\
%T\alpha_2 = \myvec{-2\\1+i\\0}\\
%T\alpha_3 = \myvec{-1\\1\\1}\\
%T\alpha_4 = \myvec{\sqrt{2}\\i\\3}
%\end{align}
%
%
%
\\
\solution
See Table \ref{eq:solutions/6/2/11/table:1}
\onecolumn
\begin{longtable}{|l|l|}
\hline
\multirow{3}{*}{} & \\
Statement&Solution\\
\hline
&\parbox{6cm}{\begin{align}
\mbox{Let }\vec{N}&=\myvec{a&b\\c&d}\\
\mbox{Since }\vec{N}^2&=0
\end{align}}\\
&If $\myvec{a\\c},\myvec{b\\d}$ are linearly independent then $\vec{N}$ is diagonalizable to $\myvec{0&0\\0&0}$.\\
&\parbox{6cm}{\begin{align}
\mbox{If }\vec{P}\vec{N}\vec{P}^{-1}=0\\
\mbox{then }\vec{N}=\vec{P}^{-1}\vec{0}\vec{P}=0
\end{align}}\\
Proof that&So in this case $\vec{N}$ itself is the zero matrix.\\
$\vec{N}=0$&This contradicts the assumption that $\myvec{a\\c},\myvec{b\\d}$ are linearly independent.\\
&$\therefore$ we can assume that $\myvec{a\\c},\myvec{b\\d}$ are linearly dependent if both are\\
&equal to the zero vector\\
&\parbox{6cm}
{\begin{align}
   \mbox{then } \vec{N} &= 0.
\end{align}}\\
%\hline
%\pagebreak
\hline
&\\
&Therefore we can assume at least one vector is non-zero.\\
Assuming $\myvec{b\\d}$ as&Therefore
$\vec{N}=\myvec{a&0\\c&0}$\\
the zero vector&\\
&\parbox{6cm}{\begin{align}
\mbox{So }\vec{N}^2&=0\\
\implies a^2&=0\\
\therefore a&=0\\
\mbox{Thus }\vec{N}&=\myvec{a&0\\c&0}
\end{align}}\\
&In this case $\vec{N}$ is similar to $\vec{N}=\myvec{0&0\\1&0}$ via the matrix $\vec{P}=\myvec{c&0\\0&1}$\\
&\\
\hline
&\\
Assuming $\myvec{a\\c}$ as&Therefore
$\vec{N}=\myvec{0&b\\0&d}$\\
the zero vector&\\
&\parbox{6cm}{\begin{align}
\mbox{Then }\vec{N}^2&=0\\
\implies d^2&=0\\
\therefore d&=0\\
\mbox{Thus }\vec{N}&=\myvec{0&b\\0&0}
\end{align}}\\
&In this case $\vec{N}$ is similar to $\vec{N}=\myvec{0&0\\b&0}$ via the matrix $\vec{P}=\myvec{0&1\\1&0}$,\\
&which is similar to $\myvec{0&0\\1&0}$ as above.\\
&\\
\hline
&\\
Hence& we can assume neither $\myvec{a\\c}$ or $\myvec{b\\d}$ is the zero vector.\\
&\\
%\hline
%\pagebreak
\hline
&\\
Consequences of &Since they are linearly dependent we can assume,\\
linear&\\
independence&\\
&\parbox{6cm}{\begin{align}
\myvec{b\\d}&=x\myvec{a\\c}\\
\therefore \vec{N}&=\myvec{a&ax\\c&cx}\\
\therefore \vec{N}^2&=0\\
\implies a(a+cx)&=0\\
c(a+cx)&=0\\
ax(a+cx)&=0\\
cx(a+cx)&=0
\end{align}}\\
\hline
&\\
Proof that $\vec{N}$ is&We know that at least one of a or c is not zero.\\
similar over $\mathbb{C}$ to&If a = 0 then c $\neq$ 0, it must be that x = 0.\\
$\myvec{0&0\\1&0}$&So in this case $\vec{N}=\myvec{0&0\\c&0}$ which is similar to $\myvec{0&0\\1&0}$ as before.\\
&\\
&\parbox{6cm}{\begin{align}
\mbox{If } a &\neq0\\
\mbox{then }x &\neq0\\
\mbox{else }a(a+cx)&=0\\
\implies a&=0\\
\mbox{Thus }a+cx&=0\\
\mbox{Hence }\vec{N}&=\myvec{a&ax\\ \frac{-a}{x}&-a}
\end{align}}\\
 &This is similar to $\myvec{a&a\\-a&-a}$ via $\vec{P}=\myvec{\sqrt{x}&0\\0&\frac{1}{\sqrt{x}}}$.\\
 &And $\myvec{a&a\\-a&-a}$ is similar to $\myvec{0&0\\-a&0}$ via $\vec{P}=\myvec{-1&-1\\1&0}$\\
 &And this finally is similar to $\myvec{0&0\\1&0}$ as before.\\
 &\\
\hline
&\\
Conclusion &Thus either $\vec{N}$ = 0 or $\vec{N}$ is similar over $\mathbb{C}$ to $\myvec{0&0\\1&0}$.\\
&\\
\hline
\caption{Solution summary}
\label{eq:solutions/6/2/11/table:1}
\end{longtable}

\end{enumerate}
%
\item Let $\mathbb{W}$ be the set of all $2\times2$ complex Hermitian matrices, that is the sset of $2\times2$ complex matrices $\vec{A}$ ssuch that $\vec{A}_{ij}=\vec{\overline{A_{ji}}}$ (the bar denoting complex conjugation). $\mathbb{W}$ is a vector space over the field of real numbers, under the usual operations. Verify that
\begin{align}\label{iso}
\myvec{x\\y\\z\\t} \rightarrow \myvec{t+x&y+iz\\y-iz&t-x}
\end{align}
is an isomorphism of $\mathbb{R}^4$ onto $\mathbb{W}$.
%
\\
\solution
See Table \ref{eq:solutions/6/2/11/table:1}
\onecolumn
\begin{longtable}{|l|l|}
\hline
\multirow{3}{*}{} & \\
Statement&Solution\\
\hline
&\parbox{6cm}{\begin{align}
\mbox{Let }\vec{N}&=\myvec{a&b\\c&d}\\
\mbox{Since }\vec{N}^2&=0
\end{align}}\\
&If $\myvec{a\\c},\myvec{b\\d}$ are linearly independent then $\vec{N}$ is diagonalizable to $\myvec{0&0\\0&0}$.\\
&\parbox{6cm}{\begin{align}
\mbox{If }\vec{P}\vec{N}\vec{P}^{-1}=0\\
\mbox{then }\vec{N}=\vec{P}^{-1}\vec{0}\vec{P}=0
\end{align}}\\
Proof that&So in this case $\vec{N}$ itself is the zero matrix.\\
$\vec{N}=0$&This contradicts the assumption that $\myvec{a\\c},\myvec{b\\d}$ are linearly independent.\\
&$\therefore$ we can assume that $\myvec{a\\c},\myvec{b\\d}$ are linearly dependent if both are\\
&equal to the zero vector\\
&\parbox{6cm}
{\begin{align}
   \mbox{then } \vec{N} &= 0.
\end{align}}\\
%\hline
%\pagebreak
\hline
&\\
&Therefore we can assume at least one vector is non-zero.\\
Assuming $\myvec{b\\d}$ as&Therefore
$\vec{N}=\myvec{a&0\\c&0}$\\
the zero vector&\\
&\parbox{6cm}{\begin{align}
\mbox{So }\vec{N}^2&=0\\
\implies a^2&=0\\
\therefore a&=0\\
\mbox{Thus }\vec{N}&=\myvec{a&0\\c&0}
\end{align}}\\
&In this case $\vec{N}$ is similar to $\vec{N}=\myvec{0&0\\1&0}$ via the matrix $\vec{P}=\myvec{c&0\\0&1}$\\
&\\
\hline
&\\
Assuming $\myvec{a\\c}$ as&Therefore
$\vec{N}=\myvec{0&b\\0&d}$\\
the zero vector&\\
&\parbox{6cm}{\begin{align}
\mbox{Then }\vec{N}^2&=0\\
\implies d^2&=0\\
\therefore d&=0\\
\mbox{Thus }\vec{N}&=\myvec{0&b\\0&0}
\end{align}}\\
&In this case $\vec{N}$ is similar to $\vec{N}=\myvec{0&0\\b&0}$ via the matrix $\vec{P}=\myvec{0&1\\1&0}$,\\
&which is similar to $\myvec{0&0\\1&0}$ as above.\\
&\\
\hline
&\\
Hence& we can assume neither $\myvec{a\\c}$ or $\myvec{b\\d}$ is the zero vector.\\
&\\
%\hline
%\pagebreak
\hline
&\\
Consequences of &Since they are linearly dependent we can assume,\\
linear&\\
independence&\\
&\parbox{6cm}{\begin{align}
\myvec{b\\d}&=x\myvec{a\\c}\\
\therefore \vec{N}&=\myvec{a&ax\\c&cx}\\
\therefore \vec{N}^2&=0\\
\implies a(a+cx)&=0\\
c(a+cx)&=0\\
ax(a+cx)&=0\\
cx(a+cx)&=0
\end{align}}\\
\hline
&\\
Proof that $\vec{N}$ is&We know that at least one of a or c is not zero.\\
similar over $\mathbb{C}$ to&If a = 0 then c $\neq$ 0, it must be that x = 0.\\
$\myvec{0&0\\1&0}$&So in this case $\vec{N}=\myvec{0&0\\c&0}$ which is similar to $\myvec{0&0\\1&0}$ as before.\\
&\\
&\parbox{6cm}{\begin{align}
\mbox{If } a &\neq0\\
\mbox{then }x &\neq0\\
\mbox{else }a(a+cx)&=0\\
\implies a&=0\\
\mbox{Thus }a+cx&=0\\
\mbox{Hence }\vec{N}&=\myvec{a&ax\\ \frac{-a}{x}&-a}
\end{align}}\\
 &This is similar to $\myvec{a&a\\-a&-a}$ via $\vec{P}=\myvec{\sqrt{x}&0\\0&\frac{1}{\sqrt{x}}}$.\\
 &And $\myvec{a&a\\-a&-a}$ is similar to $\myvec{0&0\\-a&0}$ via $\vec{P}=\myvec{-1&-1\\1&0}$\\
 &And this finally is similar to $\myvec{0&0\\1&0}$ as before.\\
 &\\
\hline
&\\
Conclusion &Thus either $\vec{N}$ = 0 or $\vec{N}$ is similar over $\mathbb{C}$ to $\myvec{0&0\\1&0}$.\\
&\\
\hline
\caption{Solution summary}
\label{eq:solutions/6/2/11/table:1}
\end{longtable}

\item Show that $\vec{F^{m\times n}}$ is isomorphic to $\vec{F^{mn}}$.
%
\\
\solution
See Table \ref{eq:solutions/6/2/11/table:1}
\onecolumn
\begin{longtable}{|l|l|}
\hline
\multirow{3}{*}{} & \\
Statement&Solution\\
\hline
&\parbox{6cm}{\begin{align}
\mbox{Let }\vec{N}&=\myvec{a&b\\c&d}\\
\mbox{Since }\vec{N}^2&=0
\end{align}}\\
&If $\myvec{a\\c},\myvec{b\\d}$ are linearly independent then $\vec{N}$ is diagonalizable to $\myvec{0&0\\0&0}$.\\
&\parbox{6cm}{\begin{align}
\mbox{If }\vec{P}\vec{N}\vec{P}^{-1}=0\\
\mbox{then }\vec{N}=\vec{P}^{-1}\vec{0}\vec{P}=0
\end{align}}\\
Proof that&So in this case $\vec{N}$ itself is the zero matrix.\\
$\vec{N}=0$&This contradicts the assumption that $\myvec{a\\c},\myvec{b\\d}$ are linearly independent.\\
&$\therefore$ we can assume that $\myvec{a\\c},\myvec{b\\d}$ are linearly dependent if both are\\
&equal to the zero vector\\
&\parbox{6cm}
{\begin{align}
   \mbox{then } \vec{N} &= 0.
\end{align}}\\
%\hline
%\pagebreak
\hline
&\\
&Therefore we can assume at least one vector is non-zero.\\
Assuming $\myvec{b\\d}$ as&Therefore
$\vec{N}=\myvec{a&0\\c&0}$\\
the zero vector&\\
&\parbox{6cm}{\begin{align}
\mbox{So }\vec{N}^2&=0\\
\implies a^2&=0\\
\therefore a&=0\\
\mbox{Thus }\vec{N}&=\myvec{a&0\\c&0}
\end{align}}\\
&In this case $\vec{N}$ is similar to $\vec{N}=\myvec{0&0\\1&0}$ via the matrix $\vec{P}=\myvec{c&0\\0&1}$\\
&\\
\hline
&\\
Assuming $\myvec{a\\c}$ as&Therefore
$\vec{N}=\myvec{0&b\\0&d}$\\
the zero vector&\\
&\parbox{6cm}{\begin{align}
\mbox{Then }\vec{N}^2&=0\\
\implies d^2&=0\\
\therefore d&=0\\
\mbox{Thus }\vec{N}&=\myvec{0&b\\0&0}
\end{align}}\\
&In this case $\vec{N}$ is similar to $\vec{N}=\myvec{0&0\\b&0}$ via the matrix $\vec{P}=\myvec{0&1\\1&0}$,\\
&which is similar to $\myvec{0&0\\1&0}$ as above.\\
&\\
\hline
&\\
Hence& we can assume neither $\myvec{a\\c}$ or $\myvec{b\\d}$ is the zero vector.\\
&\\
%\hline
%\pagebreak
\hline
&\\
Consequences of &Since they are linearly dependent we can assume,\\
linear&\\
independence&\\
&\parbox{6cm}{\begin{align}
\myvec{b\\d}&=x\myvec{a\\c}\\
\therefore \vec{N}&=\myvec{a&ax\\c&cx}\\
\therefore \vec{N}^2&=0\\
\implies a(a+cx)&=0\\
c(a+cx)&=0\\
ax(a+cx)&=0\\
cx(a+cx)&=0
\end{align}}\\
\hline
&\\
Proof that $\vec{N}$ is&We know that at least one of a or c is not zero.\\
similar over $\mathbb{C}$ to&If a = 0 then c $\neq$ 0, it must be that x = 0.\\
$\myvec{0&0\\1&0}$&So in this case $\vec{N}=\myvec{0&0\\c&0}$ which is similar to $\myvec{0&0\\1&0}$ as before.\\
&\\
&\parbox{6cm}{\begin{align}
\mbox{If } a &\neq0\\
\mbox{then }x &\neq0\\
\mbox{else }a(a+cx)&=0\\
\implies a&=0\\
\mbox{Thus }a+cx&=0\\
\mbox{Hence }\vec{N}&=\myvec{a&ax\\ \frac{-a}{x}&-a}
\end{align}}\\
 &This is similar to $\myvec{a&a\\-a&-a}$ via $\vec{P}=\myvec{\sqrt{x}&0\\0&\frac{1}{\sqrt{x}}}$.\\
 &And $\myvec{a&a\\-a&-a}$ is similar to $\myvec{0&0\\-a&0}$ via $\vec{P}=\myvec{-1&-1\\1&0}$\\
 &And this finally is similar to $\myvec{0&0\\1&0}$ as before.\\
 &\\
\hline
&\\
Conclusion &Thus either $\vec{N}$ = 0 or $\vec{N}$ is similar over $\mathbb{C}$ to $\myvec{0&0\\1&0}$.\\
&\\
\hline
\caption{Solution summary}
\label{eq:solutions/6/2/11/table:1}
\end{longtable}

%\item Let $\vec{V}$ be the set of complex numbers regarded as a vector space over the field of real numbers.We define a function $\mathbb{T}$ from $\vec{V}$ into the space of $2 \times 2$ real matrices,as follows.If $z=x+i y$ with $x$ and $y$ real numbers,then 
%\begin{align}
%\mathbb{T}(z)=\myvec{x+7y&5y\\-10y&x-7y}
%\end{align}
%\begin{enumerate}}
\item Let $\vec{V}$ be the set of complex numbers regarded as a vector space over the field of real numbers. We define a function T from $\vec{V}$ into the space of $2 \times 2$ real matrices, as follows. If $z = x + iy$ with $x$ and $y$ real numbers, then
\begin{align}
    \vec{T}(z) = \myvec{x + 7y & 5y \\ -10y & x-7y} \nonumber
\end{align}
\begin{enumerate}
\item  Verify that T is a one-one (real) linear transformation of $\vec{V}$ into the space of $2 \times 2$ real matrices.
%
\\
\solution
See Table \ref{eq:solutions/6/2/11/table:1}
\onecolumn
\begin{longtable}{|l|l|}
\hline
\multirow{3}{*}{} & \\
Statement&Solution\\
\hline
&\parbox{6cm}{\begin{align}
\mbox{Let }\vec{N}&=\myvec{a&b\\c&d}\\
\mbox{Since }\vec{N}^2&=0
\end{align}}\\
&If $\myvec{a\\c},\myvec{b\\d}$ are linearly independent then $\vec{N}$ is diagonalizable to $\myvec{0&0\\0&0}$.\\
&\parbox{6cm}{\begin{align}
\mbox{If }\vec{P}\vec{N}\vec{P}^{-1}=0\\
\mbox{then }\vec{N}=\vec{P}^{-1}\vec{0}\vec{P}=0
\end{align}}\\
Proof that&So in this case $\vec{N}$ itself is the zero matrix.\\
$\vec{N}=0$&This contradicts the assumption that $\myvec{a\\c},\myvec{b\\d}$ are linearly independent.\\
&$\therefore$ we can assume that $\myvec{a\\c},\myvec{b\\d}$ are linearly dependent if both are\\
&equal to the zero vector\\
&\parbox{6cm}
{\begin{align}
   \mbox{then } \vec{N} &= 0.
\end{align}}\\
%\hline
%\pagebreak
\hline
&\\
&Therefore we can assume at least one vector is non-zero.\\
Assuming $\myvec{b\\d}$ as&Therefore
$\vec{N}=\myvec{a&0\\c&0}$\\
the zero vector&\\
&\parbox{6cm}{\begin{align}
\mbox{So }\vec{N}^2&=0\\
\implies a^2&=0\\
\therefore a&=0\\
\mbox{Thus }\vec{N}&=\myvec{a&0\\c&0}
\end{align}}\\
&In this case $\vec{N}$ is similar to $\vec{N}=\myvec{0&0\\1&0}$ via the matrix $\vec{P}=\myvec{c&0\\0&1}$\\
&\\
\hline
&\\
Assuming $\myvec{a\\c}$ as&Therefore
$\vec{N}=\myvec{0&b\\0&d}$\\
the zero vector&\\
&\parbox{6cm}{\begin{align}
\mbox{Then }\vec{N}^2&=0\\
\implies d^2&=0\\
\therefore d&=0\\
\mbox{Thus }\vec{N}&=\myvec{0&b\\0&0}
\end{align}}\\
&In this case $\vec{N}$ is similar to $\vec{N}=\myvec{0&0\\b&0}$ via the matrix $\vec{P}=\myvec{0&1\\1&0}$,\\
&which is similar to $\myvec{0&0\\1&0}$ as above.\\
&\\
\hline
&\\
Hence& we can assume neither $\myvec{a\\c}$ or $\myvec{b\\d}$ is the zero vector.\\
&\\
%\hline
%\pagebreak
\hline
&\\
Consequences of &Since they are linearly dependent we can assume,\\
linear&\\
independence&\\
&\parbox{6cm}{\begin{align}
\myvec{b\\d}&=x\myvec{a\\c}\\
\therefore \vec{N}&=\myvec{a&ax\\c&cx}\\
\therefore \vec{N}^2&=0\\
\implies a(a+cx)&=0\\
c(a+cx)&=0\\
ax(a+cx)&=0\\
cx(a+cx)&=0
\end{align}}\\
\hline
&\\
Proof that $\vec{N}$ is&We know that at least one of a or c is not zero.\\
similar over $\mathbb{C}$ to&If a = 0 then c $\neq$ 0, it must be that x = 0.\\
$\myvec{0&0\\1&0}$&So in this case $\vec{N}=\myvec{0&0\\c&0}$ which is similar to $\myvec{0&0\\1&0}$ as before.\\
&\\
&\parbox{6cm}{\begin{align}
\mbox{If } a &\neq0\\
\mbox{then }x &\neq0\\
\mbox{else }a(a+cx)&=0\\
\implies a&=0\\
\mbox{Thus }a+cx&=0\\
\mbox{Hence }\vec{N}&=\myvec{a&ax\\ \frac{-a}{x}&-a}
\end{align}}\\
 &This is similar to $\myvec{a&a\\-a&-a}$ via $\vec{P}=\myvec{\sqrt{x}&0\\0&\frac{1}{\sqrt{x}}}$.\\
 &And $\myvec{a&a\\-a&-a}$ is similar to $\myvec{0&0\\-a&0}$ via $\vec{P}=\myvec{-1&-1\\1&0}$\\
 &And this finally is similar to $\myvec{0&0\\1&0}$ as before.\\
 &\\
\hline
&\\
Conclusion &Thus either $\vec{N}$ = 0 or $\vec{N}$ is similar over $\mathbb{C}$ to $\myvec{0&0\\1&0}$.\\
&\\
\hline
\caption{Solution summary}
\label{eq:solutions/6/2/11/table:1}
\end{longtable}

\item How would you describe the range of $\mathbb{T}$?
%
\\
\solution
See Table \ref{eq:solutions/6/2/11/table:1}
\onecolumn
\begin{longtable}{|l|l|}
\hline
\multirow{3}{*}{} & \\
Statement&Solution\\
\hline
&\parbox{6cm}{\begin{align}
\mbox{Let }\vec{N}&=\myvec{a&b\\c&d}\\
\mbox{Since }\vec{N}^2&=0
\end{align}}\\
&If $\myvec{a\\c},\myvec{b\\d}$ are linearly independent then $\vec{N}$ is diagonalizable to $\myvec{0&0\\0&0}$.\\
&\parbox{6cm}{\begin{align}
\mbox{If }\vec{P}\vec{N}\vec{P}^{-1}=0\\
\mbox{then }\vec{N}=\vec{P}^{-1}\vec{0}\vec{P}=0
\end{align}}\\
Proof that&So in this case $\vec{N}$ itself is the zero matrix.\\
$\vec{N}=0$&This contradicts the assumption that $\myvec{a\\c},\myvec{b\\d}$ are linearly independent.\\
&$\therefore$ we can assume that $\myvec{a\\c},\myvec{b\\d}$ are linearly dependent if both are\\
&equal to the zero vector\\
&\parbox{6cm}
{\begin{align}
   \mbox{then } \vec{N} &= 0.
\end{align}}\\
%\hline
%\pagebreak
\hline
&\\
&Therefore we can assume at least one vector is non-zero.\\
Assuming $\myvec{b\\d}$ as&Therefore
$\vec{N}=\myvec{a&0\\c&0}$\\
the zero vector&\\
&\parbox{6cm}{\begin{align}
\mbox{So }\vec{N}^2&=0\\
\implies a^2&=0\\
\therefore a&=0\\
\mbox{Thus }\vec{N}&=\myvec{a&0\\c&0}
\end{align}}\\
&In this case $\vec{N}$ is similar to $\vec{N}=\myvec{0&0\\1&0}$ via the matrix $\vec{P}=\myvec{c&0\\0&1}$\\
&\\
\hline
&\\
Assuming $\myvec{a\\c}$ as&Therefore
$\vec{N}=\myvec{0&b\\0&d}$\\
the zero vector&\\
&\parbox{6cm}{\begin{align}
\mbox{Then }\vec{N}^2&=0\\
\implies d^2&=0\\
\therefore d&=0\\
\mbox{Thus }\vec{N}&=\myvec{0&b\\0&0}
\end{align}}\\
&In this case $\vec{N}$ is similar to $\vec{N}=\myvec{0&0\\b&0}$ via the matrix $\vec{P}=\myvec{0&1\\1&0}$,\\
&which is similar to $\myvec{0&0\\1&0}$ as above.\\
&\\
\hline
&\\
Hence& we can assume neither $\myvec{a\\c}$ or $\myvec{b\\d}$ is the zero vector.\\
&\\
%\hline
%\pagebreak
\hline
&\\
Consequences of &Since they are linearly dependent we can assume,\\
linear&\\
independence&\\
&\parbox{6cm}{\begin{align}
\myvec{b\\d}&=x\myvec{a\\c}\\
\therefore \vec{N}&=\myvec{a&ax\\c&cx}\\
\therefore \vec{N}^2&=0\\
\implies a(a+cx)&=0\\
c(a+cx)&=0\\
ax(a+cx)&=0\\
cx(a+cx)&=0
\end{align}}\\
\hline
&\\
Proof that $\vec{N}$ is&We know that at least one of a or c is not zero.\\
similar over $\mathbb{C}$ to&If a = 0 then c $\neq$ 0, it must be that x = 0.\\
$\myvec{0&0\\1&0}$&So in this case $\vec{N}=\myvec{0&0\\c&0}$ which is similar to $\myvec{0&0\\1&0}$ as before.\\
&\\
&\parbox{6cm}{\begin{align}
\mbox{If } a &\neq0\\
\mbox{then }x &\neq0\\
\mbox{else }a(a+cx)&=0\\
\implies a&=0\\
\mbox{Thus }a+cx&=0\\
\mbox{Hence }\vec{N}&=\myvec{a&ax\\ \frac{-a}{x}&-a}
\end{align}}\\
 &This is similar to $\myvec{a&a\\-a&-a}$ via $\vec{P}=\myvec{\sqrt{x}&0\\0&\frac{1}{\sqrt{x}}}$.\\
 &And $\myvec{a&a\\-a&-a}$ is similar to $\myvec{0&0\\-a&0}$ via $\vec{P}=\myvec{-1&-1\\1&0}$\\
 &And this finally is similar to $\myvec{0&0\\1&0}$ as before.\\
 &\\
\hline
&\\
Conclusion &Thus either $\vec{N}$ = 0 or $\vec{N}$ is similar over $\mathbb{C}$ to $\myvec{0&0\\1&0}$.\\
&\\
\hline
\caption{Solution summary}
\label{eq:solutions/6/2/11/table:1}
\end{longtable}


\end{enumerate}
\item Let $\vec{V}$ and $\vec{W}$ be finite-dimensional vector spaces over the field $\vec{F}$. Prove that $\vec{V}$ and $\vec{W}$ are isomorphic if and only if $\dim{\vec{V}}$ = $\dim{\vec{W}}$
%
\\
\solution
See Table \ref{eq:solutions/6/2/11/table:1}
\onecolumn
\begin{longtable}{|l|l|}
\hline
\multirow{3}{*}{} & \\
Statement&Solution\\
\hline
&\parbox{6cm}{\begin{align}
\mbox{Let }\vec{N}&=\myvec{a&b\\c&d}\\
\mbox{Since }\vec{N}^2&=0
\end{align}}\\
&If $\myvec{a\\c},\myvec{b\\d}$ are linearly independent then $\vec{N}$ is diagonalizable to $\myvec{0&0\\0&0}$.\\
&\parbox{6cm}{\begin{align}
\mbox{If }\vec{P}\vec{N}\vec{P}^{-1}=0\\
\mbox{then }\vec{N}=\vec{P}^{-1}\vec{0}\vec{P}=0
\end{align}}\\
Proof that&So in this case $\vec{N}$ itself is the zero matrix.\\
$\vec{N}=0$&This contradicts the assumption that $\myvec{a\\c},\myvec{b\\d}$ are linearly independent.\\
&$\therefore$ we can assume that $\myvec{a\\c},\myvec{b\\d}$ are linearly dependent if both are\\
&equal to the zero vector\\
&\parbox{6cm}
{\begin{align}
   \mbox{then } \vec{N} &= 0.
\end{align}}\\
%\hline
%\pagebreak
\hline
&\\
&Therefore we can assume at least one vector is non-zero.\\
Assuming $\myvec{b\\d}$ as&Therefore
$\vec{N}=\myvec{a&0\\c&0}$\\
the zero vector&\\
&\parbox{6cm}{\begin{align}
\mbox{So }\vec{N}^2&=0\\
\implies a^2&=0\\
\therefore a&=0\\
\mbox{Thus }\vec{N}&=\myvec{a&0\\c&0}
\end{align}}\\
&In this case $\vec{N}$ is similar to $\vec{N}=\myvec{0&0\\1&0}$ via the matrix $\vec{P}=\myvec{c&0\\0&1}$\\
&\\
\hline
&\\
Assuming $\myvec{a\\c}$ as&Therefore
$\vec{N}=\myvec{0&b\\0&d}$\\
the zero vector&\\
&\parbox{6cm}{\begin{align}
\mbox{Then }\vec{N}^2&=0\\
\implies d^2&=0\\
\therefore d&=0\\
\mbox{Thus }\vec{N}&=\myvec{0&b\\0&0}
\end{align}}\\
&In this case $\vec{N}$ is similar to $\vec{N}=\myvec{0&0\\b&0}$ via the matrix $\vec{P}=\myvec{0&1\\1&0}$,\\
&which is similar to $\myvec{0&0\\1&0}$ as above.\\
&\\
\hline
&\\
Hence& we can assume neither $\myvec{a\\c}$ or $\myvec{b\\d}$ is the zero vector.\\
&\\
%\hline
%\pagebreak
\hline
&\\
Consequences of &Since they are linearly dependent we can assume,\\
linear&\\
independence&\\
&\parbox{6cm}{\begin{align}
\myvec{b\\d}&=x\myvec{a\\c}\\
\therefore \vec{N}&=\myvec{a&ax\\c&cx}\\
\therefore \vec{N}^2&=0\\
\implies a(a+cx)&=0\\
c(a+cx)&=0\\
ax(a+cx)&=0\\
cx(a+cx)&=0
\end{align}}\\
\hline
&\\
Proof that $\vec{N}$ is&We know that at least one of a or c is not zero.\\
similar over $\mathbb{C}$ to&If a = 0 then c $\neq$ 0, it must be that x = 0.\\
$\myvec{0&0\\1&0}$&So in this case $\vec{N}=\myvec{0&0\\c&0}$ which is similar to $\myvec{0&0\\1&0}$ as before.\\
&\\
&\parbox{6cm}{\begin{align}
\mbox{If } a &\neq0\\
\mbox{then }x &\neq0\\
\mbox{else }a(a+cx)&=0\\
\implies a&=0\\
\mbox{Thus }a+cx&=0\\
\mbox{Hence }\vec{N}&=\myvec{a&ax\\ \frac{-a}{x}&-a}
\end{align}}\\
 &This is similar to $\myvec{a&a\\-a&-a}$ via $\vec{P}=\myvec{\sqrt{x}&0\\0&\frac{1}{\sqrt{x}}}$.\\
 &And $\myvec{a&a\\-a&-a}$ is similar to $\myvec{0&0\\-a&0}$ via $\vec{P}=\myvec{-1&-1\\1&0}$\\
 &And this finally is similar to $\myvec{0&0\\1&0}$ as before.\\
 &\\
\hline
&\\
Conclusion &Thus either $\vec{N}$ = 0 or $\vec{N}$ is similar over $\mathbb{C}$ to $\myvec{0&0\\1&0}$.\\
&\\
\hline
\caption{Solution summary}
\label{eq:solutions/6/2/11/table:1}
\end{longtable}

\item Let $V$ and $W$ be finite-dimensional vector spaces over the field $F$ and let $U$ be an isomorphism of $V$ and $W$. Prove that $T \rightarrow UTU^{-1}$ is an isomorphism of $L(V,V)$ onto $L(W,W)$.
%
\\
\solution
See Table \ref{eq:solutions/6/2/11/table:1}
\onecolumn
\begin{longtable}{|l|l|}
\hline
\multirow{3}{*}{} & \\
Statement&Solution\\
\hline
&\parbox{6cm}{\begin{align}
\mbox{Let }\vec{N}&=\myvec{a&b\\c&d}\\
\mbox{Since }\vec{N}^2&=0
\end{align}}\\
&If $\myvec{a\\c},\myvec{b\\d}$ are linearly independent then $\vec{N}$ is diagonalizable to $\myvec{0&0\\0&0}$.\\
&\parbox{6cm}{\begin{align}
\mbox{If }\vec{P}\vec{N}\vec{P}^{-1}=0\\
\mbox{then }\vec{N}=\vec{P}^{-1}\vec{0}\vec{P}=0
\end{align}}\\
Proof that&So in this case $\vec{N}$ itself is the zero matrix.\\
$\vec{N}=0$&This contradicts the assumption that $\myvec{a\\c},\myvec{b\\d}$ are linearly independent.\\
&$\therefore$ we can assume that $\myvec{a\\c},\myvec{b\\d}$ are linearly dependent if both are\\
&equal to the zero vector\\
&\parbox{6cm}
{\begin{align}
   \mbox{then } \vec{N} &= 0.
\end{align}}\\
%\hline
%\pagebreak
\hline
&\\
&Therefore we can assume at least one vector is non-zero.\\
Assuming $\myvec{b\\d}$ as&Therefore
$\vec{N}=\myvec{a&0\\c&0}$\\
the zero vector&\\
&\parbox{6cm}{\begin{align}
\mbox{So }\vec{N}^2&=0\\
\implies a^2&=0\\
\therefore a&=0\\
\mbox{Thus }\vec{N}&=\myvec{a&0\\c&0}
\end{align}}\\
&In this case $\vec{N}$ is similar to $\vec{N}=\myvec{0&0\\1&0}$ via the matrix $\vec{P}=\myvec{c&0\\0&1}$\\
&\\
\hline
&\\
Assuming $\myvec{a\\c}$ as&Therefore
$\vec{N}=\myvec{0&b\\0&d}$\\
the zero vector&\\
&\parbox{6cm}{\begin{align}
\mbox{Then }\vec{N}^2&=0\\
\implies d^2&=0\\
\therefore d&=0\\
\mbox{Thus }\vec{N}&=\myvec{0&b\\0&0}
\end{align}}\\
&In this case $\vec{N}$ is similar to $\vec{N}=\myvec{0&0\\b&0}$ via the matrix $\vec{P}=\myvec{0&1\\1&0}$,\\
&which is similar to $\myvec{0&0\\1&0}$ as above.\\
&\\
\hline
&\\
Hence& we can assume neither $\myvec{a\\c}$ or $\myvec{b\\d}$ is the zero vector.\\
&\\
%\hline
%\pagebreak
\hline
&\\
Consequences of &Since they are linearly dependent we can assume,\\
linear&\\
independence&\\
&\parbox{6cm}{\begin{align}
\myvec{b\\d}&=x\myvec{a\\c}\\
\therefore \vec{N}&=\myvec{a&ax\\c&cx}\\
\therefore \vec{N}^2&=0\\
\implies a(a+cx)&=0\\
c(a+cx)&=0\\
ax(a+cx)&=0\\
cx(a+cx)&=0
\end{align}}\\
\hline
&\\
Proof that $\vec{N}$ is&We know that at least one of a or c is not zero.\\
similar over $\mathbb{C}$ to&If a = 0 then c $\neq$ 0, it must be that x = 0.\\
$\myvec{0&0\\1&0}$&So in this case $\vec{N}=\myvec{0&0\\c&0}$ which is similar to $\myvec{0&0\\1&0}$ as before.\\
&\\
&\parbox{6cm}{\begin{align}
\mbox{If } a &\neq0\\
\mbox{then }x &\neq0\\
\mbox{else }a(a+cx)&=0\\
\implies a&=0\\
\mbox{Thus }a+cx&=0\\
\mbox{Hence }\vec{N}&=\myvec{a&ax\\ \frac{-a}{x}&-a}
\end{align}}\\
 &This is similar to $\myvec{a&a\\-a&-a}$ via $\vec{P}=\myvec{\sqrt{x}&0\\0&\frac{1}{\sqrt{x}}}$.\\
 &And $\myvec{a&a\\-a&-a}$ is similar to $\myvec{0&0\\-a&0}$ via $\vec{P}=\myvec{-1&-1\\1&0}$\\
 &And this finally is similar to $\myvec{0&0\\1&0}$ as before.\\
 &\\
\hline
&\\
Conclusion &Thus either $\vec{N}$ = 0 or $\vec{N}$ is similar over $\mathbb{C}$ to $\myvec{0&0\\1&0}$.\\
&\\
\hline
\caption{Solution summary}
\label{eq:solutions/6/2/11/table:1}
\end{longtable}

%
\item Let $T$ be a linear operator on the finite-dimensional space $\mathbb{V}$. Suppose there is a linear operator $U$ on $\mathbb{V}$ such that $TU = I$. Prove that $T$ is invertible and $U = T^{-1}$. Give an example which shows that this is false when $\mathbb{V}$ is not finite-dimensional.
%
\\
\solution
See Table \ref{eq:solutions/6/2/11/table:1}
\onecolumn
\begin{longtable}{|l|l|}
\hline
\multirow{3}{*}{} & \\
Statement&Solution\\
\hline
&\parbox{6cm}{\begin{align}
\mbox{Let }\vec{N}&=\myvec{a&b\\c&d}\\
\mbox{Since }\vec{N}^2&=0
\end{align}}\\
&If $\myvec{a\\c},\myvec{b\\d}$ are linearly independent then $\vec{N}$ is diagonalizable to $\myvec{0&0\\0&0}$.\\
&\parbox{6cm}{\begin{align}
\mbox{If }\vec{P}\vec{N}\vec{P}^{-1}=0\\
\mbox{then }\vec{N}=\vec{P}^{-1}\vec{0}\vec{P}=0
\end{align}}\\
Proof that&So in this case $\vec{N}$ itself is the zero matrix.\\
$\vec{N}=0$&This contradicts the assumption that $\myvec{a\\c},\myvec{b\\d}$ are linearly independent.\\
&$\therefore$ we can assume that $\myvec{a\\c},\myvec{b\\d}$ are linearly dependent if both are\\
&equal to the zero vector\\
&\parbox{6cm}
{\begin{align}
   \mbox{then } \vec{N} &= 0.
\end{align}}\\
%\hline
%\pagebreak
\hline
&\\
&Therefore we can assume at least one vector is non-zero.\\
Assuming $\myvec{b\\d}$ as&Therefore
$\vec{N}=\myvec{a&0\\c&0}$\\
the zero vector&\\
&\parbox{6cm}{\begin{align}
\mbox{So }\vec{N}^2&=0\\
\implies a^2&=0\\
\therefore a&=0\\
\mbox{Thus }\vec{N}&=\myvec{a&0\\c&0}
\end{align}}\\
&In this case $\vec{N}$ is similar to $\vec{N}=\myvec{0&0\\1&0}$ via the matrix $\vec{P}=\myvec{c&0\\0&1}$\\
&\\
\hline
&\\
Assuming $\myvec{a\\c}$ as&Therefore
$\vec{N}=\myvec{0&b\\0&d}$\\
the zero vector&\\
&\parbox{6cm}{\begin{align}
\mbox{Then }\vec{N}^2&=0\\
\implies d^2&=0\\
\therefore d&=0\\
\mbox{Thus }\vec{N}&=\myvec{0&b\\0&0}
\end{align}}\\
&In this case $\vec{N}$ is similar to $\vec{N}=\myvec{0&0\\b&0}$ via the matrix $\vec{P}=\myvec{0&1\\1&0}$,\\
&which is similar to $\myvec{0&0\\1&0}$ as above.\\
&\\
\hline
&\\
Hence& we can assume neither $\myvec{a\\c}$ or $\myvec{b\\d}$ is the zero vector.\\
&\\
%\hline
%\pagebreak
\hline
&\\
Consequences of &Since they are linearly dependent we can assume,\\
linear&\\
independence&\\
&\parbox{6cm}{\begin{align}
\myvec{b\\d}&=x\myvec{a\\c}\\
\therefore \vec{N}&=\myvec{a&ax\\c&cx}\\
\therefore \vec{N}^2&=0\\
\implies a(a+cx)&=0\\
c(a+cx)&=0\\
ax(a+cx)&=0\\
cx(a+cx)&=0
\end{align}}\\
\hline
&\\
Proof that $\vec{N}$ is&We know that at least one of a or c is not zero.\\
similar over $\mathbb{C}$ to&If a = 0 then c $\neq$ 0, it must be that x = 0.\\
$\myvec{0&0\\1&0}$&So in this case $\vec{N}=\myvec{0&0\\c&0}$ which is similar to $\myvec{0&0\\1&0}$ as before.\\
&\\
&\parbox{6cm}{\begin{align}
\mbox{If } a &\neq0\\
\mbox{then }x &\neq0\\
\mbox{else }a(a+cx)&=0\\
\implies a&=0\\
\mbox{Thus }a+cx&=0\\
\mbox{Hence }\vec{N}&=\myvec{a&ax\\ \frac{-a}{x}&-a}
\end{align}}\\
 &This is similar to $\myvec{a&a\\-a&-a}$ via $\vec{P}=\myvec{\sqrt{x}&0\\0&\frac{1}{\sqrt{x}}}$.\\
 &And $\myvec{a&a\\-a&-a}$ is similar to $\myvec{0&0\\-a&0}$ via $\vec{P}=\myvec{-1&-1\\1&0}$\\
 &And this finally is similar to $\myvec{0&0\\1&0}$ as before.\\
 &\\
\hline
&\\
Conclusion &Thus either $\vec{N}$ = 0 or $\vec{N}$ is similar over $\mathbb{C}$ to $\myvec{0&0\\1&0}$.\\
&\\
\hline
\caption{Solution summary}
\label{eq:solutions/6/2/11/table:1}
\end{longtable}

\end{enumerate}

\subsection{Representation of Transformations by Matrices}
\renewcommand{\theequation}{\theenumi}
\renewcommand{\thefigure}{\theenumi}
\begin{enumerate}[label=\thesubsection.\arabic*.,ref=\thesubsection.\theenumi]
\numberwithin{equation}{enumi}
\numberwithin{figure}{enumi}
%
\item Let T be the linear operator on $\mathbb{C}^{2}$ defined by $T\brak{x_1 , x_2}=\myvec{x_1 , 0}$.
Let $\beta$ be the standard ordered basis for $\mathbb{C}^{2}$ and $\beta^{'}=\{\alpha_1 , \alpha_2\}$
be the ordered basis defined by $\alpha_1 = \myvec{1 , i} , \alpha_2 = \myvec{-i , 2}$.
\begin{enumerate}
\item What is the matrix of $\vec{T}$ relative to the pair $\vec{\beta}$,$\vec{\beta^{'}}$?
%
\\
\solution
See Table \ref{eq:solutions/6/2/11/table:1}
\onecolumn
\begin{longtable}{|l|l|}
\hline
\multirow{3}{*}{} & \\
Statement&Solution\\
\hline
&\parbox{6cm}{\begin{align}
\mbox{Let }\vec{N}&=\myvec{a&b\\c&d}\\
\mbox{Since }\vec{N}^2&=0
\end{align}}\\
&If $\myvec{a\\c},\myvec{b\\d}$ are linearly independent then $\vec{N}$ is diagonalizable to $\myvec{0&0\\0&0}$.\\
&\parbox{6cm}{\begin{align}
\mbox{If }\vec{P}\vec{N}\vec{P}^{-1}=0\\
\mbox{then }\vec{N}=\vec{P}^{-1}\vec{0}\vec{P}=0
\end{align}}\\
Proof that&So in this case $\vec{N}$ itself is the zero matrix.\\
$\vec{N}=0$&This contradicts the assumption that $\myvec{a\\c},\myvec{b\\d}$ are linearly independent.\\
&$\therefore$ we can assume that $\myvec{a\\c},\myvec{b\\d}$ are linearly dependent if both are\\
&equal to the zero vector\\
&\parbox{6cm}
{\begin{align}
   \mbox{then } \vec{N} &= 0.
\end{align}}\\
%\hline
%\pagebreak
\hline
&\\
&Therefore we can assume at least one vector is non-zero.\\
Assuming $\myvec{b\\d}$ as&Therefore
$\vec{N}=\myvec{a&0\\c&0}$\\
the zero vector&\\
&\parbox{6cm}{\begin{align}
\mbox{So }\vec{N}^2&=0\\
\implies a^2&=0\\
\therefore a&=0\\
\mbox{Thus }\vec{N}&=\myvec{a&0\\c&0}
\end{align}}\\
&In this case $\vec{N}$ is similar to $\vec{N}=\myvec{0&0\\1&0}$ via the matrix $\vec{P}=\myvec{c&0\\0&1}$\\
&\\
\hline
&\\
Assuming $\myvec{a\\c}$ as&Therefore
$\vec{N}=\myvec{0&b\\0&d}$\\
the zero vector&\\
&\parbox{6cm}{\begin{align}
\mbox{Then }\vec{N}^2&=0\\
\implies d^2&=0\\
\therefore d&=0\\
\mbox{Thus }\vec{N}&=\myvec{0&b\\0&0}
\end{align}}\\
&In this case $\vec{N}$ is similar to $\vec{N}=\myvec{0&0\\b&0}$ via the matrix $\vec{P}=\myvec{0&1\\1&0}$,\\
&which is similar to $\myvec{0&0\\1&0}$ as above.\\
&\\
\hline
&\\
Hence& we can assume neither $\myvec{a\\c}$ or $\myvec{b\\d}$ is the zero vector.\\
&\\
%\hline
%\pagebreak
\hline
&\\
Consequences of &Since they are linearly dependent we can assume,\\
linear&\\
independence&\\
&\parbox{6cm}{\begin{align}
\myvec{b\\d}&=x\myvec{a\\c}\\
\therefore \vec{N}&=\myvec{a&ax\\c&cx}\\
\therefore \vec{N}^2&=0\\
\implies a(a+cx)&=0\\
c(a+cx)&=0\\
ax(a+cx)&=0\\
cx(a+cx)&=0
\end{align}}\\
\hline
&\\
Proof that $\vec{N}$ is&We know that at least one of a or c is not zero.\\
similar over $\mathbb{C}$ to&If a = 0 then c $\neq$ 0, it must be that x = 0.\\
$\myvec{0&0\\1&0}$&So in this case $\vec{N}=\myvec{0&0\\c&0}$ which is similar to $\myvec{0&0\\1&0}$ as before.\\
&\\
&\parbox{6cm}{\begin{align}
\mbox{If } a &\neq0\\
\mbox{then }x &\neq0\\
\mbox{else }a(a+cx)&=0\\
\implies a&=0\\
\mbox{Thus }a+cx&=0\\
\mbox{Hence }\vec{N}&=\myvec{a&ax\\ \frac{-a}{x}&-a}
\end{align}}\\
 &This is similar to $\myvec{a&a\\-a&-a}$ via $\vec{P}=\myvec{\sqrt{x}&0\\0&\frac{1}{\sqrt{x}}}$.\\
 &And $\myvec{a&a\\-a&-a}$ is similar to $\myvec{0&0\\-a&0}$ via $\vec{P}=\myvec{-1&-1\\1&0}$\\
 &And this finally is similar to $\myvec{0&0\\1&0}$ as before.\\
 &\\
\hline
&\\
Conclusion &Thus either $\vec{N}$ = 0 or $\vec{N}$ is similar over $\mathbb{C}$ to $\myvec{0&0\\1&0}$.\\
&\\
\hline
\caption{Solution summary}
\label{eq:solutions/6/2/11/table:1}
\end{longtable}

\item What is the matrix of T relative to the pair $\beta$ , $\beta^{'}$?
%
\\
\solution
See Table \ref{eq:solutions/6/2/11/table:1}
\onecolumn
\begin{longtable}{|l|l|}
\hline
\multirow{3}{*}{} & \\
Statement&Solution\\
\hline
&\parbox{6cm}{\begin{align}
\mbox{Let }\vec{N}&=\myvec{a&b\\c&d}\\
\mbox{Since }\vec{N}^2&=0
\end{align}}\\
&If $\myvec{a\\c},\myvec{b\\d}$ are linearly independent then $\vec{N}$ is diagonalizable to $\myvec{0&0\\0&0}$.\\
&\parbox{6cm}{\begin{align}
\mbox{If }\vec{P}\vec{N}\vec{P}^{-1}=0\\
\mbox{then }\vec{N}=\vec{P}^{-1}\vec{0}\vec{P}=0
\end{align}}\\
Proof that&So in this case $\vec{N}$ itself is the zero matrix.\\
$\vec{N}=0$&This contradicts the assumption that $\myvec{a\\c},\myvec{b\\d}$ are linearly independent.\\
&$\therefore$ we can assume that $\myvec{a\\c},\myvec{b\\d}$ are linearly dependent if both are\\
&equal to the zero vector\\
&\parbox{6cm}
{\begin{align}
   \mbox{then } \vec{N} &= 0.
\end{align}}\\
%\hline
%\pagebreak
\hline
&\\
&Therefore we can assume at least one vector is non-zero.\\
Assuming $\myvec{b\\d}$ as&Therefore
$\vec{N}=\myvec{a&0\\c&0}$\\
the zero vector&\\
&\parbox{6cm}{\begin{align}
\mbox{So }\vec{N}^2&=0\\
\implies a^2&=0\\
\therefore a&=0\\
\mbox{Thus }\vec{N}&=\myvec{a&0\\c&0}
\end{align}}\\
&In this case $\vec{N}$ is similar to $\vec{N}=\myvec{0&0\\1&0}$ via the matrix $\vec{P}=\myvec{c&0\\0&1}$\\
&\\
\hline
&\\
Assuming $\myvec{a\\c}$ as&Therefore
$\vec{N}=\myvec{0&b\\0&d}$\\
the zero vector&\\
&\parbox{6cm}{\begin{align}
\mbox{Then }\vec{N}^2&=0\\
\implies d^2&=0\\
\therefore d&=0\\
\mbox{Thus }\vec{N}&=\myvec{0&b\\0&0}
\end{align}}\\
&In this case $\vec{N}$ is similar to $\vec{N}=\myvec{0&0\\b&0}$ via the matrix $\vec{P}=\myvec{0&1\\1&0}$,\\
&which is similar to $\myvec{0&0\\1&0}$ as above.\\
&\\
\hline
&\\
Hence& we can assume neither $\myvec{a\\c}$ or $\myvec{b\\d}$ is the zero vector.\\
&\\
%\hline
%\pagebreak
\hline
&\\
Consequences of &Since they are linearly dependent we can assume,\\
linear&\\
independence&\\
&\parbox{6cm}{\begin{align}
\myvec{b\\d}&=x\myvec{a\\c}\\
\therefore \vec{N}&=\myvec{a&ax\\c&cx}\\
\therefore \vec{N}^2&=0\\
\implies a(a+cx)&=0\\
c(a+cx)&=0\\
ax(a+cx)&=0\\
cx(a+cx)&=0
\end{align}}\\
\hline
&\\
Proof that $\vec{N}$ is&We know that at least one of a or c is not zero.\\
similar over $\mathbb{C}$ to&If a = 0 then c $\neq$ 0, it must be that x = 0.\\
$\myvec{0&0\\1&0}$&So in this case $\vec{N}=\myvec{0&0\\c&0}$ which is similar to $\myvec{0&0\\1&0}$ as before.\\
&\\
&\parbox{6cm}{\begin{align}
\mbox{If } a &\neq0\\
\mbox{then }x &\neq0\\
\mbox{else }a(a+cx)&=0\\
\implies a&=0\\
\mbox{Thus }a+cx&=0\\
\mbox{Hence }\vec{N}&=\myvec{a&ax\\ \frac{-a}{x}&-a}
\end{align}}\\
 &This is similar to $\myvec{a&a\\-a&-a}$ via $\vec{P}=\myvec{\sqrt{x}&0\\0&\frac{1}{\sqrt{x}}}$.\\
 &And $\myvec{a&a\\-a&-a}$ is similar to $\myvec{0&0\\-a&0}$ via $\vec{P}=\myvec{-1&-1\\1&0}$\\
 &And this finally is similar to $\myvec{0&0\\1&0}$ as before.\\
 &\\
\hline
&\\
Conclusion &Thus either $\vec{N}$ = 0 or $\vec{N}$ is similar over $\mathbb{C}$ to $\myvec{0&0\\1&0}$.\\
&\\
\hline
\caption{Solution summary}
\label{eq:solutions/6/2/11/table:1}
\end{longtable}

\item What is the matrix of \textbf{T} in the ordered basis $\beta'$
%
\\
\solution
See Table \ref{eq:solutions/6/2/11/table:1}
\onecolumn
\begin{longtable}{|l|l|}
\hline
\multirow{3}{*}{} & \\
Statement&Solution\\
\hline
&\parbox{6cm}{\begin{align}
\mbox{Let }\vec{N}&=\myvec{a&b\\c&d}\\
\mbox{Since }\vec{N}^2&=0
\end{align}}\\
&If $\myvec{a\\c},\myvec{b\\d}$ are linearly independent then $\vec{N}$ is diagonalizable to $\myvec{0&0\\0&0}$.\\
&\parbox{6cm}{\begin{align}
\mbox{If }\vec{P}\vec{N}\vec{P}^{-1}=0\\
\mbox{then }\vec{N}=\vec{P}^{-1}\vec{0}\vec{P}=0
\end{align}}\\
Proof that&So in this case $\vec{N}$ itself is the zero matrix.\\
$\vec{N}=0$&This contradicts the assumption that $\myvec{a\\c},\myvec{b\\d}$ are linearly independent.\\
&$\therefore$ we can assume that $\myvec{a\\c},\myvec{b\\d}$ are linearly dependent if both are\\
&equal to the zero vector\\
&\parbox{6cm}
{\begin{align}
   \mbox{then } \vec{N} &= 0.
\end{align}}\\
%\hline
%\pagebreak
\hline
&\\
&Therefore we can assume at least one vector is non-zero.\\
Assuming $\myvec{b\\d}$ as&Therefore
$\vec{N}=\myvec{a&0\\c&0}$\\
the zero vector&\\
&\parbox{6cm}{\begin{align}
\mbox{So }\vec{N}^2&=0\\
\implies a^2&=0\\
\therefore a&=0\\
\mbox{Thus }\vec{N}&=\myvec{a&0\\c&0}
\end{align}}\\
&In this case $\vec{N}$ is similar to $\vec{N}=\myvec{0&0\\1&0}$ via the matrix $\vec{P}=\myvec{c&0\\0&1}$\\
&\\
\hline
&\\
Assuming $\myvec{a\\c}$ as&Therefore
$\vec{N}=\myvec{0&b\\0&d}$\\
the zero vector&\\
&\parbox{6cm}{\begin{align}
\mbox{Then }\vec{N}^2&=0\\
\implies d^2&=0\\
\therefore d&=0\\
\mbox{Thus }\vec{N}&=\myvec{0&b\\0&0}
\end{align}}\\
&In this case $\vec{N}$ is similar to $\vec{N}=\myvec{0&0\\b&0}$ via the matrix $\vec{P}=\myvec{0&1\\1&0}$,\\
&which is similar to $\myvec{0&0\\1&0}$ as above.\\
&\\
\hline
&\\
Hence& we can assume neither $\myvec{a\\c}$ or $\myvec{b\\d}$ is the zero vector.\\
&\\
%\hline
%\pagebreak
\hline
&\\
Consequences of &Since they are linearly dependent we can assume,\\
linear&\\
independence&\\
&\parbox{6cm}{\begin{align}
\myvec{b\\d}&=x\myvec{a\\c}\\
\therefore \vec{N}&=\myvec{a&ax\\c&cx}\\
\therefore \vec{N}^2&=0\\
\implies a(a+cx)&=0\\
c(a+cx)&=0\\
ax(a+cx)&=0\\
cx(a+cx)&=0
\end{align}}\\
\hline
&\\
Proof that $\vec{N}$ is&We know that at least one of a or c is not zero.\\
similar over $\mathbb{C}$ to&If a = 0 then c $\neq$ 0, it must be that x = 0.\\
$\myvec{0&0\\1&0}$&So in this case $\vec{N}=\myvec{0&0\\c&0}$ which is similar to $\myvec{0&0\\1&0}$ as before.\\
&\\
&\parbox{6cm}{\begin{align}
\mbox{If } a &\neq0\\
\mbox{then }x &\neq0\\
\mbox{else }a(a+cx)&=0\\
\implies a&=0\\
\mbox{Thus }a+cx&=0\\
\mbox{Hence }\vec{N}&=\myvec{a&ax\\ \frac{-a}{x}&-a}
\end{align}}\\
 &This is similar to $\myvec{a&a\\-a&-a}$ via $\vec{P}=\myvec{\sqrt{x}&0\\0&\frac{1}{\sqrt{x}}}$.\\
 &And $\myvec{a&a\\-a&-a}$ is similar to $\myvec{0&0\\-a&0}$ via $\vec{P}=\myvec{-1&-1\\1&0}$\\
 &And this finally is similar to $\myvec{0&0\\1&0}$ as before.\\
 &\\
\hline
&\\
Conclusion &Thus either $\vec{N}$ = 0 or $\vec{N}$ is similar over $\mathbb{C}$ to $\myvec{0&0\\1&0}$.\\
&\\
\hline
\caption{Solution summary}
\label{eq:solutions/6/2/11/table:1}
\end{longtable}

\item What is the matrix of T in the ordered basis $\{\alpha_2,\alpha_1\}$?
%
%
\\
\solution
See Table \ref{eq:solutions/6/2/11/table:1}
\onecolumn
\begin{longtable}{|l|l|}
\hline
\multirow{3}{*}{} & \\
Statement&Solution\\
\hline
&\parbox{6cm}{\begin{align}
\mbox{Let }\vec{N}&=\myvec{a&b\\c&d}\\
\mbox{Since }\vec{N}^2&=0
\end{align}}\\
&If $\myvec{a\\c},\myvec{b\\d}$ are linearly independent then $\vec{N}$ is diagonalizable to $\myvec{0&0\\0&0}$.\\
&\parbox{6cm}{\begin{align}
\mbox{If }\vec{P}\vec{N}\vec{P}^{-1}=0\\
\mbox{then }\vec{N}=\vec{P}^{-1}\vec{0}\vec{P}=0
\end{align}}\\
Proof that&So in this case $\vec{N}$ itself is the zero matrix.\\
$\vec{N}=0$&This contradicts the assumption that $\myvec{a\\c},\myvec{b\\d}$ are linearly independent.\\
&$\therefore$ we can assume that $\myvec{a\\c},\myvec{b\\d}$ are linearly dependent if both are\\
&equal to the zero vector\\
&\parbox{6cm}
{\begin{align}
   \mbox{then } \vec{N} &= 0.
\end{align}}\\
%\hline
%\pagebreak
\hline
&\\
&Therefore we can assume at least one vector is non-zero.\\
Assuming $\myvec{b\\d}$ as&Therefore
$\vec{N}=\myvec{a&0\\c&0}$\\
the zero vector&\\
&\parbox{6cm}{\begin{align}
\mbox{So }\vec{N}^2&=0\\
\implies a^2&=0\\
\therefore a&=0\\
\mbox{Thus }\vec{N}&=\myvec{a&0\\c&0}
\end{align}}\\
&In this case $\vec{N}$ is similar to $\vec{N}=\myvec{0&0\\1&0}$ via the matrix $\vec{P}=\myvec{c&0\\0&1}$\\
&\\
\hline
&\\
Assuming $\myvec{a\\c}$ as&Therefore
$\vec{N}=\myvec{0&b\\0&d}$\\
the zero vector&\\
&\parbox{6cm}{\begin{align}
\mbox{Then }\vec{N}^2&=0\\
\implies d^2&=0\\
\therefore d&=0\\
\mbox{Thus }\vec{N}&=\myvec{0&b\\0&0}
\end{align}}\\
&In this case $\vec{N}$ is similar to $\vec{N}=\myvec{0&0\\b&0}$ via the matrix $\vec{P}=\myvec{0&1\\1&0}$,\\
&which is similar to $\myvec{0&0\\1&0}$ as above.\\
&\\
\hline
&\\
Hence& we can assume neither $\myvec{a\\c}$ or $\myvec{b\\d}$ is the zero vector.\\
&\\
%\hline
%\pagebreak
\hline
&\\
Consequences of &Since they are linearly dependent we can assume,\\
linear&\\
independence&\\
&\parbox{6cm}{\begin{align}
\myvec{b\\d}&=x\myvec{a\\c}\\
\therefore \vec{N}&=\myvec{a&ax\\c&cx}\\
\therefore \vec{N}^2&=0\\
\implies a(a+cx)&=0\\
c(a+cx)&=0\\
ax(a+cx)&=0\\
cx(a+cx)&=0
\end{align}}\\
\hline
&\\
Proof that $\vec{N}$ is&We know that at least one of a or c is not zero.\\
similar over $\mathbb{C}$ to&If a = 0 then c $\neq$ 0, it must be that x = 0.\\
$\myvec{0&0\\1&0}$&So in this case $\vec{N}=\myvec{0&0\\c&0}$ which is similar to $\myvec{0&0\\1&0}$ as before.\\
&\\
&\parbox{6cm}{\begin{align}
\mbox{If } a &\neq0\\
\mbox{then }x &\neq0\\
\mbox{else }a(a+cx)&=0\\
\implies a&=0\\
\mbox{Thus }a+cx&=0\\
\mbox{Hence }\vec{N}&=\myvec{a&ax\\ \frac{-a}{x}&-a}
\end{align}}\\
 &This is similar to $\myvec{a&a\\-a&-a}$ via $\vec{P}=\myvec{\sqrt{x}&0\\0&\frac{1}{\sqrt{x}}}$.\\
 &And $\myvec{a&a\\-a&-a}$ is similar to $\myvec{0&0\\-a&0}$ via $\vec{P}=\myvec{-1&-1\\1&0}$\\
 &And this finally is similar to $\myvec{0&0\\1&0}$ as before.\\
 &\\
\hline
&\\
Conclusion &Thus either $\vec{N}$ = 0 or $\vec{N}$ is similar over $\mathbb{C}$ to $\myvec{0&0\\1&0}$.\\
&\\
\hline
\caption{Solution summary}
\label{eq:solutions/6/2/11/table:1}
\end{longtable}

\end{enumerate}
%
\item    Let T be the linear transformation from $\mathbb{R}^3$ into $\mathbb{R}^2$ defined by,
   \begin{align}
   T\myvec{x_1\\x_2\\x_3}=\myvec{x_1+x_2\\2x_3-x_1}
   \end{align}
\begin{enumerate}

\item    If $\beta$ is the standard ordered basis for $\mathbb{R}^3$ and $\beta^{\prime}$ is the standard ordered basis for $\mathbb{R}^2$, what is the matrix of T relative to the pair $\beta$, $\beta^{'}$
\end{enumerate}

\item Let T be a linear operator on $\vec{F^n}$, let $\vec{A}$ be the matrix of T in the standard ordered basis for $\vec{F^n}$, and let W be the subspace of $\vec{F^n}$ spanned by the column vectors of $\vec{A}$. What does W have to do with T ?
%
\\
\solution
See Table \ref{eq:solutions/6/2/11/table:1}
\onecolumn
\begin{longtable}{|l|l|}
\hline
\multirow{3}{*}{} & \\
Statement&Solution\\
\hline
&\parbox{6cm}{\begin{align}
\mbox{Let }\vec{N}&=\myvec{a&b\\c&d}\\
\mbox{Since }\vec{N}^2&=0
\end{align}}\\
&If $\myvec{a\\c},\myvec{b\\d}$ are linearly independent then $\vec{N}$ is diagonalizable to $\myvec{0&0\\0&0}$.\\
&\parbox{6cm}{\begin{align}
\mbox{If }\vec{P}\vec{N}\vec{P}^{-1}=0\\
\mbox{then }\vec{N}=\vec{P}^{-1}\vec{0}\vec{P}=0
\end{align}}\\
Proof that&So in this case $\vec{N}$ itself is the zero matrix.\\
$\vec{N}=0$&This contradicts the assumption that $\myvec{a\\c},\myvec{b\\d}$ are linearly independent.\\
&$\therefore$ we can assume that $\myvec{a\\c},\myvec{b\\d}$ are linearly dependent if both are\\
&equal to the zero vector\\
&\parbox{6cm}
{\begin{align}
   \mbox{then } \vec{N} &= 0.
\end{align}}\\
%\hline
%\pagebreak
\hline
&\\
&Therefore we can assume at least one vector is non-zero.\\
Assuming $\myvec{b\\d}$ as&Therefore
$\vec{N}=\myvec{a&0\\c&0}$\\
the zero vector&\\
&\parbox{6cm}{\begin{align}
\mbox{So }\vec{N}^2&=0\\
\implies a^2&=0\\
\therefore a&=0\\
\mbox{Thus }\vec{N}&=\myvec{a&0\\c&0}
\end{align}}\\
&In this case $\vec{N}$ is similar to $\vec{N}=\myvec{0&0\\1&0}$ via the matrix $\vec{P}=\myvec{c&0\\0&1}$\\
&\\
\hline
&\\
Assuming $\myvec{a\\c}$ as&Therefore
$\vec{N}=\myvec{0&b\\0&d}$\\
the zero vector&\\
&\parbox{6cm}{\begin{align}
\mbox{Then }\vec{N}^2&=0\\
\implies d^2&=0\\
\therefore d&=0\\
\mbox{Thus }\vec{N}&=\myvec{0&b\\0&0}
\end{align}}\\
&In this case $\vec{N}$ is similar to $\vec{N}=\myvec{0&0\\b&0}$ via the matrix $\vec{P}=\myvec{0&1\\1&0}$,\\
&which is similar to $\myvec{0&0\\1&0}$ as above.\\
&\\
\hline
&\\
Hence& we can assume neither $\myvec{a\\c}$ or $\myvec{b\\d}$ is the zero vector.\\
&\\
%\hline
%\pagebreak
\hline
&\\
Consequences of &Since they are linearly dependent we can assume,\\
linear&\\
independence&\\
&\parbox{6cm}{\begin{align}
\myvec{b\\d}&=x\myvec{a\\c}\\
\therefore \vec{N}&=\myvec{a&ax\\c&cx}\\
\therefore \vec{N}^2&=0\\
\implies a(a+cx)&=0\\
c(a+cx)&=0\\
ax(a+cx)&=0\\
cx(a+cx)&=0
\end{align}}\\
\hline
&\\
Proof that $\vec{N}$ is&We know that at least one of a or c is not zero.\\
similar over $\mathbb{C}$ to&If a = 0 then c $\neq$ 0, it must be that x = 0.\\
$\myvec{0&0\\1&0}$&So in this case $\vec{N}=\myvec{0&0\\c&0}$ which is similar to $\myvec{0&0\\1&0}$ as before.\\
&\\
&\parbox{6cm}{\begin{align}
\mbox{If } a &\neq0\\
\mbox{then }x &\neq0\\
\mbox{else }a(a+cx)&=0\\
\implies a&=0\\
\mbox{Thus }a+cx&=0\\
\mbox{Hence }\vec{N}&=\myvec{a&ax\\ \frac{-a}{x}&-a}
\end{align}}\\
 &This is similar to $\myvec{a&a\\-a&-a}$ via $\vec{P}=\myvec{\sqrt{x}&0\\0&\frac{1}{\sqrt{x}}}$.\\
 &And $\myvec{a&a\\-a&-a}$ is similar to $\myvec{0&0\\-a&0}$ via $\vec{P}=\myvec{-1&-1\\1&0}$\\
 &And this finally is similar to $\myvec{0&0\\1&0}$ as before.\\
 &\\
\hline
&\\
Conclusion &Thus either $\vec{N}$ = 0 or $\vec{N}$ is similar over $\mathbb{C}$ to $\myvec{0&0\\1&0}$.\\
&\\
\hline
\caption{Solution summary}
\label{eq:solutions/6/2/11/table:1}
\end{longtable}

%
\item Let $V$ be a two-dimensional vector space over the field $F$ and let $B$ be an ordered basis for $V$. If $T$ is a linear operator on $V$ and 
\begin{align}
[T]_B = \myvec{a & b\\c & d}
\end{align}
Prove that 
\begin{align}
T^2 - (a+d)T + (ad-bc)I = 0
\end{align}
%
\\
\solution
See Table \ref{eq:solutions/6/2/11/table:1}
\onecolumn
\begin{longtable}{|l|l|}
\hline
\multirow{3}{*}{} & \\
Statement&Solution\\
\hline
&\parbox{6cm}{\begin{align}
\mbox{Let }\vec{N}&=\myvec{a&b\\c&d}\\
\mbox{Since }\vec{N}^2&=0
\end{align}}\\
&If $\myvec{a\\c},\myvec{b\\d}$ are linearly independent then $\vec{N}$ is diagonalizable to $\myvec{0&0\\0&0}$.\\
&\parbox{6cm}{\begin{align}
\mbox{If }\vec{P}\vec{N}\vec{P}^{-1}=0\\
\mbox{then }\vec{N}=\vec{P}^{-1}\vec{0}\vec{P}=0
\end{align}}\\
Proof that&So in this case $\vec{N}$ itself is the zero matrix.\\
$\vec{N}=0$&This contradicts the assumption that $\myvec{a\\c},\myvec{b\\d}$ are linearly independent.\\
&$\therefore$ we can assume that $\myvec{a\\c},\myvec{b\\d}$ are linearly dependent if both are\\
&equal to the zero vector\\
&\parbox{6cm}
{\begin{align}
   \mbox{then } \vec{N} &= 0.
\end{align}}\\
%\hline
%\pagebreak
\hline
&\\
&Therefore we can assume at least one vector is non-zero.\\
Assuming $\myvec{b\\d}$ as&Therefore
$\vec{N}=\myvec{a&0\\c&0}$\\
the zero vector&\\
&\parbox{6cm}{\begin{align}
\mbox{So }\vec{N}^2&=0\\
\implies a^2&=0\\
\therefore a&=0\\
\mbox{Thus }\vec{N}&=\myvec{a&0\\c&0}
\end{align}}\\
&In this case $\vec{N}$ is similar to $\vec{N}=\myvec{0&0\\1&0}$ via the matrix $\vec{P}=\myvec{c&0\\0&1}$\\
&\\
\hline
&\\
Assuming $\myvec{a\\c}$ as&Therefore
$\vec{N}=\myvec{0&b\\0&d}$\\
the zero vector&\\
&\parbox{6cm}{\begin{align}
\mbox{Then }\vec{N}^2&=0\\
\implies d^2&=0\\
\therefore d&=0\\
\mbox{Thus }\vec{N}&=\myvec{0&b\\0&0}
\end{align}}\\
&In this case $\vec{N}$ is similar to $\vec{N}=\myvec{0&0\\b&0}$ via the matrix $\vec{P}=\myvec{0&1\\1&0}$,\\
&which is similar to $\myvec{0&0\\1&0}$ as above.\\
&\\
\hline
&\\
Hence& we can assume neither $\myvec{a\\c}$ or $\myvec{b\\d}$ is the zero vector.\\
&\\
%\hline
%\pagebreak
\hline
&\\
Consequences of &Since they are linearly dependent we can assume,\\
linear&\\
independence&\\
&\parbox{6cm}{\begin{align}
\myvec{b\\d}&=x\myvec{a\\c}\\
\therefore \vec{N}&=\myvec{a&ax\\c&cx}\\
\therefore \vec{N}^2&=0\\
\implies a(a+cx)&=0\\
c(a+cx)&=0\\
ax(a+cx)&=0\\
cx(a+cx)&=0
\end{align}}\\
\hline
&\\
Proof that $\vec{N}$ is&We know that at least one of a or c is not zero.\\
similar over $\mathbb{C}$ to&If a = 0 then c $\neq$ 0, it must be that x = 0.\\
$\myvec{0&0\\1&0}$&So in this case $\vec{N}=\myvec{0&0\\c&0}$ which is similar to $\myvec{0&0\\1&0}$ as before.\\
&\\
&\parbox{6cm}{\begin{align}
\mbox{If } a &\neq0\\
\mbox{then }x &\neq0\\
\mbox{else }a(a+cx)&=0\\
\implies a&=0\\
\mbox{Thus }a+cx&=0\\
\mbox{Hence }\vec{N}&=\myvec{a&ax\\ \frac{-a}{x}&-a}
\end{align}}\\
 &This is similar to $\myvec{a&a\\-a&-a}$ via $\vec{P}=\myvec{\sqrt{x}&0\\0&\frac{1}{\sqrt{x}}}$.\\
 &And $\myvec{a&a\\-a&-a}$ is similar to $\myvec{0&0\\-a&0}$ via $\vec{P}=\myvec{-1&-1\\1&0}$\\
 &And this finally is similar to $\myvec{0&0\\1&0}$ as before.\\
 &\\
\hline
&\\
Conclusion &Thus either $\vec{N}$ = 0 or $\vec{N}$ is similar over $\mathbb{C}$ to $\myvec{0&0\\1&0}$.\\
&\\
\hline
\caption{Solution summary}
\label{eq:solutions/6/2/11/table:1}
\end{longtable}

\item Let $T$ be a linear operator on $\mathbb{R}^3$, the matrix of which in the standard ordered basis is,
\begin{align}
\vec{A} = \myvec{1&2&1\\0&1&1\\-1&3&4} 
\end{align}
Find a basis for the range of $T$ and a basis for the null-space of $T$.
%
\\
\solution
See Table \ref{eq:solutions/6/2/11/table:1}
\onecolumn
\begin{longtable}{|l|l|}
\hline
\multirow{3}{*}{} & \\
Statement&Solution\\
\hline
&\parbox{6cm}{\begin{align}
\mbox{Let }\vec{N}&=\myvec{a&b\\c&d}\\
\mbox{Since }\vec{N}^2&=0
\end{align}}\\
&If $\myvec{a\\c},\myvec{b\\d}$ are linearly independent then $\vec{N}$ is diagonalizable to $\myvec{0&0\\0&0}$.\\
&\parbox{6cm}{\begin{align}
\mbox{If }\vec{P}\vec{N}\vec{P}^{-1}=0\\
\mbox{then }\vec{N}=\vec{P}^{-1}\vec{0}\vec{P}=0
\end{align}}\\
Proof that&So in this case $\vec{N}$ itself is the zero matrix.\\
$\vec{N}=0$&This contradicts the assumption that $\myvec{a\\c},\myvec{b\\d}$ are linearly independent.\\
&$\therefore$ we can assume that $\myvec{a\\c},\myvec{b\\d}$ are linearly dependent if both are\\
&equal to the zero vector\\
&\parbox{6cm}
{\begin{align}
   \mbox{then } \vec{N} &= 0.
\end{align}}\\
%\hline
%\pagebreak
\hline
&\\
&Therefore we can assume at least one vector is non-zero.\\
Assuming $\myvec{b\\d}$ as&Therefore
$\vec{N}=\myvec{a&0\\c&0}$\\
the zero vector&\\
&\parbox{6cm}{\begin{align}
\mbox{So }\vec{N}^2&=0\\
\implies a^2&=0\\
\therefore a&=0\\
\mbox{Thus }\vec{N}&=\myvec{a&0\\c&0}
\end{align}}\\
&In this case $\vec{N}$ is similar to $\vec{N}=\myvec{0&0\\1&0}$ via the matrix $\vec{P}=\myvec{c&0\\0&1}$\\
&\\
\hline
&\\
Assuming $\myvec{a\\c}$ as&Therefore
$\vec{N}=\myvec{0&b\\0&d}$\\
the zero vector&\\
&\parbox{6cm}{\begin{align}
\mbox{Then }\vec{N}^2&=0\\
\implies d^2&=0\\
\therefore d&=0\\
\mbox{Thus }\vec{N}&=\myvec{0&b\\0&0}
\end{align}}\\
&In this case $\vec{N}$ is similar to $\vec{N}=\myvec{0&0\\b&0}$ via the matrix $\vec{P}=\myvec{0&1\\1&0}$,\\
&which is similar to $\myvec{0&0\\1&0}$ as above.\\
&\\
\hline
&\\
Hence& we can assume neither $\myvec{a\\c}$ or $\myvec{b\\d}$ is the zero vector.\\
&\\
%\hline
%\pagebreak
\hline
&\\
Consequences of &Since they are linearly dependent we can assume,\\
linear&\\
independence&\\
&\parbox{6cm}{\begin{align}
\myvec{b\\d}&=x\myvec{a\\c}\\
\therefore \vec{N}&=\myvec{a&ax\\c&cx}\\
\therefore \vec{N}^2&=0\\
\implies a(a+cx)&=0\\
c(a+cx)&=0\\
ax(a+cx)&=0\\
cx(a+cx)&=0
\end{align}}\\
\hline
&\\
Proof that $\vec{N}$ is&We know that at least one of a or c is not zero.\\
similar over $\mathbb{C}$ to&If a = 0 then c $\neq$ 0, it must be that x = 0.\\
$\myvec{0&0\\1&0}$&So in this case $\vec{N}=\myvec{0&0\\c&0}$ which is similar to $\myvec{0&0\\1&0}$ as before.\\
&\\
&\parbox{6cm}{\begin{align}
\mbox{If } a &\neq0\\
\mbox{then }x &\neq0\\
\mbox{else }a(a+cx)&=0\\
\implies a&=0\\
\mbox{Thus }a+cx&=0\\
\mbox{Hence }\vec{N}&=\myvec{a&ax\\ \frac{-a}{x}&-a}
\end{align}}\\
 &This is similar to $\myvec{a&a\\-a&-a}$ via $\vec{P}=\myvec{\sqrt{x}&0\\0&\frac{1}{\sqrt{x}}}$.\\
 &And $\myvec{a&a\\-a&-a}$ is similar to $\myvec{0&0\\-a&0}$ via $\vec{P}=\myvec{-1&-1\\1&0}$\\
 &And this finally is similar to $\myvec{0&0\\1&0}$ as before.\\
 &\\
\hline
&\\
Conclusion &Thus either $\vec{N}$ = 0 or $\vec{N}$ is similar over $\mathbb{C}$ to $\myvec{0&0\\1&0}$.\\
&\\
\hline
\caption{Solution summary}
\label{eq:solutions/6/2/11/table:1}
\end{longtable}

\item Let T be the linear operator on $\mathbb{R}^2$ defined by
\begin{align}
T\brak{x_1,x_2}=\brak{-x_2,x_1}\label{eq:solutions/3/4/6/c/Tx}
\end{align}
\begin{enumerate}
\item What is the matrix of $\vec{T}$ in the standard ordered basis of $\mathbb{R}^2$?
%
\\
\solution
See Table \ref{eq:solutions/6/2/11/table:1}
\onecolumn
\begin{longtable}{|l|l|}
\hline
\multirow{3}{*}{} & \\
Statement&Solution\\
\hline
&\parbox{6cm}{\begin{align}
\mbox{Let }\vec{N}&=\myvec{a&b\\c&d}\\
\mbox{Since }\vec{N}^2&=0
\end{align}}\\
&If $\myvec{a\\c},\myvec{b\\d}$ are linearly independent then $\vec{N}$ is diagonalizable to $\myvec{0&0\\0&0}$.\\
&\parbox{6cm}{\begin{align}
\mbox{If }\vec{P}\vec{N}\vec{P}^{-1}=0\\
\mbox{then }\vec{N}=\vec{P}^{-1}\vec{0}\vec{P}=0
\end{align}}\\
Proof that&So in this case $\vec{N}$ itself is the zero matrix.\\
$\vec{N}=0$&This contradicts the assumption that $\myvec{a\\c},\myvec{b\\d}$ are linearly independent.\\
&$\therefore$ we can assume that $\myvec{a\\c},\myvec{b\\d}$ are linearly dependent if both are\\
&equal to the zero vector\\
&\parbox{6cm}
{\begin{align}
   \mbox{then } \vec{N} &= 0.
\end{align}}\\
%\hline
%\pagebreak
\hline
&\\
&Therefore we can assume at least one vector is non-zero.\\
Assuming $\myvec{b\\d}$ as&Therefore
$\vec{N}=\myvec{a&0\\c&0}$\\
the zero vector&\\
&\parbox{6cm}{\begin{align}
\mbox{So }\vec{N}^2&=0\\
\implies a^2&=0\\
\therefore a&=0\\
\mbox{Thus }\vec{N}&=\myvec{a&0\\c&0}
\end{align}}\\
&In this case $\vec{N}$ is similar to $\vec{N}=\myvec{0&0\\1&0}$ via the matrix $\vec{P}=\myvec{c&0\\0&1}$\\
&\\
\hline
&\\
Assuming $\myvec{a\\c}$ as&Therefore
$\vec{N}=\myvec{0&b\\0&d}$\\
the zero vector&\\
&\parbox{6cm}{\begin{align}
\mbox{Then }\vec{N}^2&=0\\
\implies d^2&=0\\
\therefore d&=0\\
\mbox{Thus }\vec{N}&=\myvec{0&b\\0&0}
\end{align}}\\
&In this case $\vec{N}$ is similar to $\vec{N}=\myvec{0&0\\b&0}$ via the matrix $\vec{P}=\myvec{0&1\\1&0}$,\\
&which is similar to $\myvec{0&0\\1&0}$ as above.\\
&\\
\hline
&\\
Hence& we can assume neither $\myvec{a\\c}$ or $\myvec{b\\d}$ is the zero vector.\\
&\\
%\hline
%\pagebreak
\hline
&\\
Consequences of &Since they are linearly dependent we can assume,\\
linear&\\
independence&\\
&\parbox{6cm}{\begin{align}
\myvec{b\\d}&=x\myvec{a\\c}\\
\therefore \vec{N}&=\myvec{a&ax\\c&cx}\\
\therefore \vec{N}^2&=0\\
\implies a(a+cx)&=0\\
c(a+cx)&=0\\
ax(a+cx)&=0\\
cx(a+cx)&=0
\end{align}}\\
\hline
&\\
Proof that $\vec{N}$ is&We know that at least one of a or c is not zero.\\
similar over $\mathbb{C}$ to&If a = 0 then c $\neq$ 0, it must be that x = 0.\\
$\myvec{0&0\\1&0}$&So in this case $\vec{N}=\myvec{0&0\\c&0}$ which is similar to $\myvec{0&0\\1&0}$ as before.\\
&\\
&\parbox{6cm}{\begin{align}
\mbox{If } a &\neq0\\
\mbox{then }x &\neq0\\
\mbox{else }a(a+cx)&=0\\
\implies a&=0\\
\mbox{Thus }a+cx&=0\\
\mbox{Hence }\vec{N}&=\myvec{a&ax\\ \frac{-a}{x}&-a}
\end{align}}\\
 &This is similar to $\myvec{a&a\\-a&-a}$ via $\vec{P}=\myvec{\sqrt{x}&0\\0&\frac{1}{\sqrt{x}}}$.\\
 &And $\myvec{a&a\\-a&-a}$ is similar to $\myvec{0&0\\-a&0}$ via $\vec{P}=\myvec{-1&-1\\1&0}$\\
 &And this finally is similar to $\myvec{0&0\\1&0}$ as before.\\
 &\\
\hline
&\\
Conclusion &Thus either $\vec{N}$ = 0 or $\vec{N}$ is similar over $\mathbb{C}$ to $\myvec{0&0\\1&0}$.\\
&\\
\hline
\caption{Solution summary}
\label{eq:solutions/6/2/11/table:1}
\end{longtable}

\item  	Let $T$ be the linear operator on $\vec{R}^2$ defined by 
	\begin{align}
		T\myvec{x_1\\ x_2} = \myvec{-x_2\\ x_1} 
	\end{align}
	What is the matrix of $T$ in the ordered basis $\vec{B} = \cbrak{ \vec{ \alpha }_1, \vec{\alpha}_2}$, where $\vec{\alpha}_1 = \myvec{1\\ 2}$ and $\vec{\alpha}_2 = \myvec{1 \\ -1}$ ?
%
\\
\solution
See Table \ref{eq:solutions/6/2/11/table:1}
\onecolumn
\begin{longtable}{|l|l|}
\hline
\multirow{3}{*}{} & \\
Statement&Solution\\
\hline
&\parbox{6cm}{\begin{align}
\mbox{Let }\vec{N}&=\myvec{a&b\\c&d}\\
\mbox{Since }\vec{N}^2&=0
\end{align}}\\
&If $\myvec{a\\c},\myvec{b\\d}$ are linearly independent then $\vec{N}$ is diagonalizable to $\myvec{0&0\\0&0}$.\\
&\parbox{6cm}{\begin{align}
\mbox{If }\vec{P}\vec{N}\vec{P}^{-1}=0\\
\mbox{then }\vec{N}=\vec{P}^{-1}\vec{0}\vec{P}=0
\end{align}}\\
Proof that&So in this case $\vec{N}$ itself is the zero matrix.\\
$\vec{N}=0$&This contradicts the assumption that $\myvec{a\\c},\myvec{b\\d}$ are linearly independent.\\
&$\therefore$ we can assume that $\myvec{a\\c},\myvec{b\\d}$ are linearly dependent if both are\\
&equal to the zero vector\\
&\parbox{6cm}
{\begin{align}
   \mbox{then } \vec{N} &= 0.
\end{align}}\\
%\hline
%\pagebreak
\hline
&\\
&Therefore we can assume at least one vector is non-zero.\\
Assuming $\myvec{b\\d}$ as&Therefore
$\vec{N}=\myvec{a&0\\c&0}$\\
the zero vector&\\
&\parbox{6cm}{\begin{align}
\mbox{So }\vec{N}^2&=0\\
\implies a^2&=0\\
\therefore a&=0\\
\mbox{Thus }\vec{N}&=\myvec{a&0\\c&0}
\end{align}}\\
&In this case $\vec{N}$ is similar to $\vec{N}=\myvec{0&0\\1&0}$ via the matrix $\vec{P}=\myvec{c&0\\0&1}$\\
&\\
\hline
&\\
Assuming $\myvec{a\\c}$ as&Therefore
$\vec{N}=\myvec{0&b\\0&d}$\\
the zero vector&\\
&\parbox{6cm}{\begin{align}
\mbox{Then }\vec{N}^2&=0\\
\implies d^2&=0\\
\therefore d&=0\\
\mbox{Thus }\vec{N}&=\myvec{0&b\\0&0}
\end{align}}\\
&In this case $\vec{N}$ is similar to $\vec{N}=\myvec{0&0\\b&0}$ via the matrix $\vec{P}=\myvec{0&1\\1&0}$,\\
&which is similar to $\myvec{0&0\\1&0}$ as above.\\
&\\
\hline
&\\
Hence& we can assume neither $\myvec{a\\c}$ or $\myvec{b\\d}$ is the zero vector.\\
&\\
%\hline
%\pagebreak
\hline
&\\
Consequences of &Since they are linearly dependent we can assume,\\
linear&\\
independence&\\
&\parbox{6cm}{\begin{align}
\myvec{b\\d}&=x\myvec{a\\c}\\
\therefore \vec{N}&=\myvec{a&ax\\c&cx}\\
\therefore \vec{N}^2&=0\\
\implies a(a+cx)&=0\\
c(a+cx)&=0\\
ax(a+cx)&=0\\
cx(a+cx)&=0
\end{align}}\\
\hline
&\\
Proof that $\vec{N}$ is&We know that at least one of a or c is not zero.\\
similar over $\mathbb{C}$ to&If a = 0 then c $\neq$ 0, it must be that x = 0.\\
$\myvec{0&0\\1&0}$&So in this case $\vec{N}=\myvec{0&0\\c&0}$ which is similar to $\myvec{0&0\\1&0}$ as before.\\
&\\
&\parbox{6cm}{\begin{align}
\mbox{If } a &\neq0\\
\mbox{then }x &\neq0\\
\mbox{else }a(a+cx)&=0\\
\implies a&=0\\
\mbox{Thus }a+cx&=0\\
\mbox{Hence }\vec{N}&=\myvec{a&ax\\ \frac{-a}{x}&-a}
\end{align}}\\
 &This is similar to $\myvec{a&a\\-a&-a}$ via $\vec{P}=\myvec{\sqrt{x}&0\\0&\frac{1}{\sqrt{x}}}$.\\
 &And $\myvec{a&a\\-a&-a}$ is similar to $\myvec{0&0\\-a&0}$ via $\vec{P}=\myvec{-1&-1\\1&0}$\\
 &And this finally is similar to $\myvec{0&0\\1&0}$ as before.\\
 &\\
\hline
&\\
Conclusion &Thus either $\vec{N}$ = 0 or $\vec{N}$ is similar over $\mathbb{C}$ to $\myvec{0&0\\1&0}$.\\
&\\
\hline
\caption{Solution summary}
\label{eq:solutions/6/2/11/table:1}
\end{longtable}

\item Prove that for every real number c, the operator $\brak{T-cI}$ is invertible.
%
\\
\solution
See Table \ref{eq:solutions/6/2/11/table:1}
\onecolumn
\begin{longtable}{|l|l|}
\hline
\multirow{3}{*}{} & \\
Statement&Solution\\
\hline
&\parbox{6cm}{\begin{align}
\mbox{Let }\vec{N}&=\myvec{a&b\\c&d}\\
\mbox{Since }\vec{N}^2&=0
\end{align}}\\
&If $\myvec{a\\c},\myvec{b\\d}$ are linearly independent then $\vec{N}$ is diagonalizable to $\myvec{0&0\\0&0}$.\\
&\parbox{6cm}{\begin{align}
\mbox{If }\vec{P}\vec{N}\vec{P}^{-1}=0\\
\mbox{then }\vec{N}=\vec{P}^{-1}\vec{0}\vec{P}=0
\end{align}}\\
Proof that&So in this case $\vec{N}$ itself is the zero matrix.\\
$\vec{N}=0$&This contradicts the assumption that $\myvec{a\\c},\myvec{b\\d}$ are linearly independent.\\
&$\therefore$ we can assume that $\myvec{a\\c},\myvec{b\\d}$ are linearly dependent if both are\\
&equal to the zero vector\\
&\parbox{6cm}
{\begin{align}
   \mbox{then } \vec{N} &= 0.
\end{align}}\\
%\hline
%\pagebreak
\hline
&\\
&Therefore we can assume at least one vector is non-zero.\\
Assuming $\myvec{b\\d}$ as&Therefore
$\vec{N}=\myvec{a&0\\c&0}$\\
the zero vector&\\
&\parbox{6cm}{\begin{align}
\mbox{So }\vec{N}^2&=0\\
\implies a^2&=0\\
\therefore a&=0\\
\mbox{Thus }\vec{N}&=\myvec{a&0\\c&0}
\end{align}}\\
&In this case $\vec{N}$ is similar to $\vec{N}=\myvec{0&0\\1&0}$ via the matrix $\vec{P}=\myvec{c&0\\0&1}$\\
&\\
\hline
&\\
Assuming $\myvec{a\\c}$ as&Therefore
$\vec{N}=\myvec{0&b\\0&d}$\\
the zero vector&\\
&\parbox{6cm}{\begin{align}
\mbox{Then }\vec{N}^2&=0\\
\implies d^2&=0\\
\therefore d&=0\\
\mbox{Thus }\vec{N}&=\myvec{0&b\\0&0}
\end{align}}\\
&In this case $\vec{N}$ is similar to $\vec{N}=\myvec{0&0\\b&0}$ via the matrix $\vec{P}=\myvec{0&1\\1&0}$,\\
&which is similar to $\myvec{0&0\\1&0}$ as above.\\
&\\
\hline
&\\
Hence& we can assume neither $\myvec{a\\c}$ or $\myvec{b\\d}$ is the zero vector.\\
&\\
%\hline
%\pagebreak
\hline
&\\
Consequences of &Since they are linearly dependent we can assume,\\
linear&\\
independence&\\
&\parbox{6cm}{\begin{align}
\myvec{b\\d}&=x\myvec{a\\c}\\
\therefore \vec{N}&=\myvec{a&ax\\c&cx}\\
\therefore \vec{N}^2&=0\\
\implies a(a+cx)&=0\\
c(a+cx)&=0\\
ax(a+cx)&=0\\
cx(a+cx)&=0
\end{align}}\\
\hline
&\\
Proof that $\vec{N}$ is&We know that at least one of a or c is not zero.\\
similar over $\mathbb{C}$ to&If a = 0 then c $\neq$ 0, it must be that x = 0.\\
$\myvec{0&0\\1&0}$&So in this case $\vec{N}=\myvec{0&0\\c&0}$ which is similar to $\myvec{0&0\\1&0}$ as before.\\
&\\
&\parbox{6cm}{\begin{align}
\mbox{If } a &\neq0\\
\mbox{then }x &\neq0\\
\mbox{else }a(a+cx)&=0\\
\implies a&=0\\
\mbox{Thus }a+cx&=0\\
\mbox{Hence }\vec{N}&=\myvec{a&ax\\ \frac{-a}{x}&-a}
\end{align}}\\
 &This is similar to $\myvec{a&a\\-a&-a}$ via $\vec{P}=\myvec{\sqrt{x}&0\\0&\frac{1}{\sqrt{x}}}$.\\
 &And $\myvec{a&a\\-a&-a}$ is similar to $\myvec{0&0\\-a&0}$ via $\vec{P}=\myvec{-1&-1\\1&0}$\\
 &And this finally is similar to $\myvec{0&0\\1&0}$ as before.\\
 &\\
\hline
&\\
Conclusion &Thus either $\vec{N}$ = 0 or $\vec{N}$ is similar over $\mathbb{C}$ to $\myvec{0&0\\1&0}$.\\
&\\
\hline
\caption{Solution summary}
\label{eq:solutions/6/2/11/table:1}
\end{longtable}

\end{enumerate}
\item Let $\mathbb{T}$ be the linear operator on $\mathbb{R}^3$ defined by 
\begin{align}
    \mathbb{T}(x_1,x_2,x_3)=\\(3x_1+x_3,-2x_1+x_2,-x_1+2x_2+4x_3)\label{eq:solutions/3/4/7/a/1}
\end{align}
What is the matrix of $\mathbb{T}$ in the standard ordered basis of $\mathbb{R}^3$?
%
\\
\solution
See Table \ref{eq:solutions/6/2/11/table:1}
\onecolumn
\begin{longtable}{|l|l|}
\hline
\multirow{3}{*}{} & \\
Statement&Solution\\
\hline
&\parbox{6cm}{\begin{align}
\mbox{Let }\vec{N}&=\myvec{a&b\\c&d}\\
\mbox{Since }\vec{N}^2&=0
\end{align}}\\
&If $\myvec{a\\c},\myvec{b\\d}$ are linearly independent then $\vec{N}$ is diagonalizable to $\myvec{0&0\\0&0}$.\\
&\parbox{6cm}{\begin{align}
\mbox{If }\vec{P}\vec{N}\vec{P}^{-1}=0\\
\mbox{then }\vec{N}=\vec{P}^{-1}\vec{0}\vec{P}=0
\end{align}}\\
Proof that&So in this case $\vec{N}$ itself is the zero matrix.\\
$\vec{N}=0$&This contradicts the assumption that $\myvec{a\\c},\myvec{b\\d}$ are linearly independent.\\
&$\therefore$ we can assume that $\myvec{a\\c},\myvec{b\\d}$ are linearly dependent if both are\\
&equal to the zero vector\\
&\parbox{6cm}
{\begin{align}
   \mbox{then } \vec{N} &= 0.
\end{align}}\\
%\hline
%\pagebreak
\hline
&\\
&Therefore we can assume at least one vector is non-zero.\\
Assuming $\myvec{b\\d}$ as&Therefore
$\vec{N}=\myvec{a&0\\c&0}$\\
the zero vector&\\
&\parbox{6cm}{\begin{align}
\mbox{So }\vec{N}^2&=0\\
\implies a^2&=0\\
\therefore a&=0\\
\mbox{Thus }\vec{N}&=\myvec{a&0\\c&0}
\end{align}}\\
&In this case $\vec{N}$ is similar to $\vec{N}=\myvec{0&0\\1&0}$ via the matrix $\vec{P}=\myvec{c&0\\0&1}$\\
&\\
\hline
&\\
Assuming $\myvec{a\\c}$ as&Therefore
$\vec{N}=\myvec{0&b\\0&d}$\\
the zero vector&\\
&\parbox{6cm}{\begin{align}
\mbox{Then }\vec{N}^2&=0\\
\implies d^2&=0\\
\therefore d&=0\\
\mbox{Thus }\vec{N}&=\myvec{0&b\\0&0}
\end{align}}\\
&In this case $\vec{N}$ is similar to $\vec{N}=\myvec{0&0\\b&0}$ via the matrix $\vec{P}=\myvec{0&1\\1&0}$,\\
&which is similar to $\myvec{0&0\\1&0}$ as above.\\
&\\
\hline
&\\
Hence& we can assume neither $\myvec{a\\c}$ or $\myvec{b\\d}$ is the zero vector.\\
&\\
%\hline
%\pagebreak
\hline
&\\
Consequences of &Since they are linearly dependent we can assume,\\
linear&\\
independence&\\
&\parbox{6cm}{\begin{align}
\myvec{b\\d}&=x\myvec{a\\c}\\
\therefore \vec{N}&=\myvec{a&ax\\c&cx}\\
\therefore \vec{N}^2&=0\\
\implies a(a+cx)&=0\\
c(a+cx)&=0\\
ax(a+cx)&=0\\
cx(a+cx)&=0
\end{align}}\\
\hline
&\\
Proof that $\vec{N}$ is&We know that at least one of a or c is not zero.\\
similar over $\mathbb{C}$ to&If a = 0 then c $\neq$ 0, it must be that x = 0.\\
$\myvec{0&0\\1&0}$&So in this case $\vec{N}=\myvec{0&0\\c&0}$ which is similar to $\myvec{0&0\\1&0}$ as before.\\
&\\
&\parbox{6cm}{\begin{align}
\mbox{If } a &\neq0\\
\mbox{then }x &\neq0\\
\mbox{else }a(a+cx)&=0\\
\implies a&=0\\
\mbox{Thus }a+cx&=0\\
\mbox{Hence }\vec{N}&=\myvec{a&ax\\ \frac{-a}{x}&-a}
\end{align}}\\
 &This is similar to $\myvec{a&a\\-a&-a}$ via $\vec{P}=\myvec{\sqrt{x}&0\\0&\frac{1}{\sqrt{x}}}$.\\
 &And $\myvec{a&a\\-a&-a}$ is similar to $\myvec{0&0\\-a&0}$ via $\vec{P}=\myvec{-1&-1\\1&0}$\\
 &And this finally is similar to $\myvec{0&0\\1&0}$ as before.\\
 &\\
\hline
&\\
Conclusion &Thus either $\vec{N}$ = 0 or $\vec{N}$ is similar over $\mathbb{C}$ to $\myvec{0&0\\1&0}$.\\
&\\
\hline
\caption{Solution summary}
\label{eq:solutions/6/2/11/table:1}
\end{longtable}

\item The linear operator $\vec{T}$ on $\vec{R}^2$ defined by 
\begin{align}
	\vec{T}\myvec{x_1\\x_2}=\myvec{x_1\\0}
\end{align}
is represented by the matrix
\begin{align}
	\vec{A}=\myvec{1&0\\
		       0&0}
\end{align}
\item Let $\theta$ be a real number. Prove that the following two matrices are similar over the field of complex numbers: 
\begin{align}
    \myvec{\cos{\theta} & -\sin{\theta} \\ \sin{\theta} & \cos{\theta}},
    \quad
    \myvec{e^{i\theta} & 0 \\ 0 & e^{-i\theta}} \nonumber
\end{align}
%
\\
\solution
See Table \ref{eq:solutions/6/2/11/table:1}
\onecolumn
\begin{longtable}{|l|l|}
\hline
\multirow{3}{*}{} & \\
Statement&Solution\\
\hline
&\parbox{6cm}{\begin{align}
\mbox{Let }\vec{N}&=\myvec{a&b\\c&d}\\
\mbox{Since }\vec{N}^2&=0
\end{align}}\\
&If $\myvec{a\\c},\myvec{b\\d}$ are linearly independent then $\vec{N}$ is diagonalizable to $\myvec{0&0\\0&0}$.\\
&\parbox{6cm}{\begin{align}
\mbox{If }\vec{P}\vec{N}\vec{P}^{-1}=0\\
\mbox{then }\vec{N}=\vec{P}^{-1}\vec{0}\vec{P}=0
\end{align}}\\
Proof that&So in this case $\vec{N}$ itself is the zero matrix.\\
$\vec{N}=0$&This contradicts the assumption that $\myvec{a\\c},\myvec{b\\d}$ are linearly independent.\\
&$\therefore$ we can assume that $\myvec{a\\c},\myvec{b\\d}$ are linearly dependent if both are\\
&equal to the zero vector\\
&\parbox{6cm}
{\begin{align}
   \mbox{then } \vec{N} &= 0.
\end{align}}\\
%\hline
%\pagebreak
\hline
&\\
&Therefore we can assume at least one vector is non-zero.\\
Assuming $\myvec{b\\d}$ as&Therefore
$\vec{N}=\myvec{a&0\\c&0}$\\
the zero vector&\\
&\parbox{6cm}{\begin{align}
\mbox{So }\vec{N}^2&=0\\
\implies a^2&=0\\
\therefore a&=0\\
\mbox{Thus }\vec{N}&=\myvec{a&0\\c&0}
\end{align}}\\
&In this case $\vec{N}$ is similar to $\vec{N}=\myvec{0&0\\1&0}$ via the matrix $\vec{P}=\myvec{c&0\\0&1}$\\
&\\
\hline
&\\
Assuming $\myvec{a\\c}$ as&Therefore
$\vec{N}=\myvec{0&b\\0&d}$\\
the zero vector&\\
&\parbox{6cm}{\begin{align}
\mbox{Then }\vec{N}^2&=0\\
\implies d^2&=0\\
\therefore d&=0\\
\mbox{Thus }\vec{N}&=\myvec{0&b\\0&0}
\end{align}}\\
&In this case $\vec{N}$ is similar to $\vec{N}=\myvec{0&0\\b&0}$ via the matrix $\vec{P}=\myvec{0&1\\1&0}$,\\
&which is similar to $\myvec{0&0\\1&0}$ as above.\\
&\\
\hline
&\\
Hence& we can assume neither $\myvec{a\\c}$ or $\myvec{b\\d}$ is the zero vector.\\
&\\
%\hline
%\pagebreak
\hline
&\\
Consequences of &Since they are linearly dependent we can assume,\\
linear&\\
independence&\\
&\parbox{6cm}{\begin{align}
\myvec{b\\d}&=x\myvec{a\\c}\\
\therefore \vec{N}&=\myvec{a&ax\\c&cx}\\
\therefore \vec{N}^2&=0\\
\implies a(a+cx)&=0\\
c(a+cx)&=0\\
ax(a+cx)&=0\\
cx(a+cx)&=0
\end{align}}\\
\hline
&\\
Proof that $\vec{N}$ is&We know that at least one of a or c is not zero.\\
similar over $\mathbb{C}$ to&If a = 0 then c $\neq$ 0, it must be that x = 0.\\
$\myvec{0&0\\1&0}$&So in this case $\vec{N}=\myvec{0&0\\c&0}$ which is similar to $\myvec{0&0\\1&0}$ as before.\\
&\\
&\parbox{6cm}{\begin{align}
\mbox{If } a &\neq0\\
\mbox{then }x &\neq0\\
\mbox{else }a(a+cx)&=0\\
\implies a&=0\\
\mbox{Thus }a+cx&=0\\
\mbox{Hence }\vec{N}&=\myvec{a&ax\\ \frac{-a}{x}&-a}
\end{align}}\\
 &This is similar to $\myvec{a&a\\-a&-a}$ via $\vec{P}=\myvec{\sqrt{x}&0\\0&\frac{1}{\sqrt{x}}}$.\\
 &And $\myvec{a&a\\-a&-a}$ is similar to $\myvec{0&0\\-a&0}$ via $\vec{P}=\myvec{-1&-1\\1&0}$\\
 &And this finally is similar to $\myvec{0&0\\1&0}$ as before.\\
 &\\
\hline
&\\
Conclusion &Thus either $\vec{N}$ = 0 or $\vec{N}$ is similar over $\mathbb{C}$ to $\myvec{0&0\\1&0}$.\\
&\\
\hline
\caption{Solution summary}
\label{eq:solutions/6/2/11/table:1}
\end{longtable}

\item Let $\mathbb V$ be finite dimensional vector space over the field $\mathbb F$, and let $\vec{S}$ and $\vec{T}$ be linear operators on $\mathbb V$. When do there exist ordered bases $\mathcal{B}$ and $\mathcal{B}'$ for $\mathbb V$ such that $[S]_\mathcal{B} = [T]_\mathcal{B}'$? Prove that such bases exist if and only if there is an invertible linear operator $\vec{U}$ on $\mathbb V$ such that $\vec{T} = \vec{U}\vec{S}\vec{U}^{-1}$
%
\\
\solution
See Table \ref{eq:solutions/6/2/11/table:1}
\onecolumn
\begin{longtable}{|l|l|}
\hline
\multirow{3}{*}{} & \\
Statement&Solution\\
\hline
&\parbox{6cm}{\begin{align}
\mbox{Let }\vec{N}&=\myvec{a&b\\c&d}\\
\mbox{Since }\vec{N}^2&=0
\end{align}}\\
&If $\myvec{a\\c},\myvec{b\\d}$ are linearly independent then $\vec{N}$ is diagonalizable to $\myvec{0&0\\0&0}$.\\
&\parbox{6cm}{\begin{align}
\mbox{If }\vec{P}\vec{N}\vec{P}^{-1}=0\\
\mbox{then }\vec{N}=\vec{P}^{-1}\vec{0}\vec{P}=0
\end{align}}\\
Proof that&So in this case $\vec{N}$ itself is the zero matrix.\\
$\vec{N}=0$&This contradicts the assumption that $\myvec{a\\c},\myvec{b\\d}$ are linearly independent.\\
&$\therefore$ we can assume that $\myvec{a\\c},\myvec{b\\d}$ are linearly dependent if both are\\
&equal to the zero vector\\
&\parbox{6cm}
{\begin{align}
   \mbox{then } \vec{N} &= 0.
\end{align}}\\
%\hline
%\pagebreak
\hline
&\\
&Therefore we can assume at least one vector is non-zero.\\
Assuming $\myvec{b\\d}$ as&Therefore
$\vec{N}=\myvec{a&0\\c&0}$\\
the zero vector&\\
&\parbox{6cm}{\begin{align}
\mbox{So }\vec{N}^2&=0\\
\implies a^2&=0\\
\therefore a&=0\\
\mbox{Thus }\vec{N}&=\myvec{a&0\\c&0}
\end{align}}\\
&In this case $\vec{N}$ is similar to $\vec{N}=\myvec{0&0\\1&0}$ via the matrix $\vec{P}=\myvec{c&0\\0&1}$\\
&\\
\hline
&\\
Assuming $\myvec{a\\c}$ as&Therefore
$\vec{N}=\myvec{0&b\\0&d}$\\
the zero vector&\\
&\parbox{6cm}{\begin{align}
\mbox{Then }\vec{N}^2&=0\\
\implies d^2&=0\\
\therefore d&=0\\
\mbox{Thus }\vec{N}&=\myvec{0&b\\0&0}
\end{align}}\\
&In this case $\vec{N}$ is similar to $\vec{N}=\myvec{0&0\\b&0}$ via the matrix $\vec{P}=\myvec{0&1\\1&0}$,\\
&which is similar to $\myvec{0&0\\1&0}$ as above.\\
&\\
\hline
&\\
Hence& we can assume neither $\myvec{a\\c}$ or $\myvec{b\\d}$ is the zero vector.\\
&\\
%\hline
%\pagebreak
\hline
&\\
Consequences of &Since they are linearly dependent we can assume,\\
linear&\\
independence&\\
&\parbox{6cm}{\begin{align}
\myvec{b\\d}&=x\myvec{a\\c}\\
\therefore \vec{N}&=\myvec{a&ax\\c&cx}\\
\therefore \vec{N}^2&=0\\
\implies a(a+cx)&=0\\
c(a+cx)&=0\\
ax(a+cx)&=0\\
cx(a+cx)&=0
\end{align}}\\
\hline
&\\
Proof that $\vec{N}$ is&We know that at least one of a or c is not zero.\\
similar over $\mathbb{C}$ to&If a = 0 then c $\neq$ 0, it must be that x = 0.\\
$\myvec{0&0\\1&0}$&So in this case $\vec{N}=\myvec{0&0\\c&0}$ which is similar to $\myvec{0&0\\1&0}$ as before.\\
&\\
&\parbox{6cm}{\begin{align}
\mbox{If } a &\neq0\\
\mbox{then }x &\neq0\\
\mbox{else }a(a+cx)&=0\\
\implies a&=0\\
\mbox{Thus }a+cx&=0\\
\mbox{Hence }\vec{N}&=\myvec{a&ax\\ \frac{-a}{x}&-a}
\end{align}}\\
 &This is similar to $\myvec{a&a\\-a&-a}$ via $\vec{P}=\myvec{\sqrt{x}&0\\0&\frac{1}{\sqrt{x}}}$.\\
 &And $\myvec{a&a\\-a&-a}$ is similar to $\myvec{0&0\\-a&0}$ via $\vec{P}=\myvec{-1&-1\\1&0}$\\
 &And this finally is similar to $\myvec{0&0\\1&0}$ as before.\\
 &\\
\hline
&\\
Conclusion &Thus either $\vec{N}$ = 0 or $\vec{N}$ is similar over $\mathbb{C}$ to $\myvec{0&0\\1&0}$.\\
&\\
\hline
\caption{Solution summary}
\label{eq:solutions/6/2/11/table:1}
\end{longtable}

\item Prove that if $\vec{S}$ is a linear operator on $\vec{R}^2$ such that $\vec{S}^2=\vec{S}$, 
then $\vec{S}=\vec{0}$, or $\vec{S}=\vec{I}$, or there is an ordered basis $\vec{B}$ for $\vec{R}^2$ 
such that $[\vec{S}]_B=\vec{A}$.
%
\\
\solution
See Table \ref{eq:solutions/6/2/11/table:1}
\onecolumn
\begin{longtable}{|l|l|}
\hline
\multirow{3}{*}{} & \\
Statement&Solution\\
\hline
&\parbox{6cm}{\begin{align}
\mbox{Let }\vec{N}&=\myvec{a&b\\c&d}\\
\mbox{Since }\vec{N}^2&=0
\end{align}}\\
&If $\myvec{a\\c},\myvec{b\\d}$ are linearly independent then $\vec{N}$ is diagonalizable to $\myvec{0&0\\0&0}$.\\
&\parbox{6cm}{\begin{align}
\mbox{If }\vec{P}\vec{N}\vec{P}^{-1}=0\\
\mbox{then }\vec{N}=\vec{P}^{-1}\vec{0}\vec{P}=0
\end{align}}\\
Proof that&So in this case $\vec{N}$ itself is the zero matrix.\\
$\vec{N}=0$&This contradicts the assumption that $\myvec{a\\c},\myvec{b\\d}$ are linearly independent.\\
&$\therefore$ we can assume that $\myvec{a\\c},\myvec{b\\d}$ are linearly dependent if both are\\
&equal to the zero vector\\
&\parbox{6cm}
{\begin{align}
   \mbox{then } \vec{N} &= 0.
\end{align}}\\
%\hline
%\pagebreak
\hline
&\\
&Therefore we can assume at least one vector is non-zero.\\
Assuming $\myvec{b\\d}$ as&Therefore
$\vec{N}=\myvec{a&0\\c&0}$\\
the zero vector&\\
&\parbox{6cm}{\begin{align}
\mbox{So }\vec{N}^2&=0\\
\implies a^2&=0\\
\therefore a&=0\\
\mbox{Thus }\vec{N}&=\myvec{a&0\\c&0}
\end{align}}\\
&In this case $\vec{N}$ is similar to $\vec{N}=\myvec{0&0\\1&0}$ via the matrix $\vec{P}=\myvec{c&0\\0&1}$\\
&\\
\hline
&\\
Assuming $\myvec{a\\c}$ as&Therefore
$\vec{N}=\myvec{0&b\\0&d}$\\
the zero vector&\\
&\parbox{6cm}{\begin{align}
\mbox{Then }\vec{N}^2&=0\\
\implies d^2&=0\\
\therefore d&=0\\
\mbox{Thus }\vec{N}&=\myvec{0&b\\0&0}
\end{align}}\\
&In this case $\vec{N}$ is similar to $\vec{N}=\myvec{0&0\\b&0}$ via the matrix $\vec{P}=\myvec{0&1\\1&0}$,\\
&which is similar to $\myvec{0&0\\1&0}$ as above.\\
&\\
\hline
&\\
Hence& we can assume neither $\myvec{a\\c}$ or $\myvec{b\\d}$ is the zero vector.\\
&\\
%\hline
%\pagebreak
\hline
&\\
Consequences of &Since they are linearly dependent we can assume,\\
linear&\\
independence&\\
&\parbox{6cm}{\begin{align}
\myvec{b\\d}&=x\myvec{a\\c}\\
\therefore \vec{N}&=\myvec{a&ax\\c&cx}\\
\therefore \vec{N}^2&=0\\
\implies a(a+cx)&=0\\
c(a+cx)&=0\\
ax(a+cx)&=0\\
cx(a+cx)&=0
\end{align}}\\
\hline
&\\
Proof that $\vec{N}$ is&We know that at least one of a or c is not zero.\\
similar over $\mathbb{C}$ to&If a = 0 then c $\neq$ 0, it must be that x = 0.\\
$\myvec{0&0\\1&0}$&So in this case $\vec{N}=\myvec{0&0\\c&0}$ which is similar to $\myvec{0&0\\1&0}$ as before.\\
&\\
&\parbox{6cm}{\begin{align}
\mbox{If } a &\neq0\\
\mbox{then }x &\neq0\\
\mbox{else }a(a+cx)&=0\\
\implies a&=0\\
\mbox{Thus }a+cx&=0\\
\mbox{Hence }\vec{N}&=\myvec{a&ax\\ \frac{-a}{x}&-a}
\end{align}}\\
 &This is similar to $\myvec{a&a\\-a&-a}$ via $\vec{P}=\myvec{\sqrt{x}&0\\0&\frac{1}{\sqrt{x}}}$.\\
 &And $\myvec{a&a\\-a&-a}$ is similar to $\myvec{0&0\\-a&0}$ via $\vec{P}=\myvec{-1&-1\\1&0}$\\
 &And this finally is similar to $\myvec{0&0\\1&0}$ as before.\\
 &\\
\hline
&\\
Conclusion &Thus either $\vec{N}$ = 0 or $\vec{N}$ is similar over $\mathbb{C}$ to $\myvec{0&0\\1&0}$.\\
&\\
\hline
\caption{Solution summary}
\label{eq:solutions/6/2/11/table:1}
\end{longtable}

%
\item Let $\vec{W}$ be the space of all $n \times 1$ column matrices over a field $\vec{F}$. If $\vec{A}$ is an $n \times n$ matrix over $\vec{F}$, then $\vec{A}$ defines a linear operator $\vec{L_{A}}$ on $\vec{W}$ through left multiplication: $\vec{L_{A}}$(X) = $\vec{A}$X. Prove that every linear operator on $\vec{W}$ is left multiplication by some $n\times n$ matrix, i.e., is $\vec{L_{A}}$ for some $\vec{A}$.

Now suppose $\vec{V}$ is an $n$-dimensional vector space over the field $\vec{F}$, and let $\beta$ be an ordered basis for $\vec{V}$. For each $\alpha$ in $\vec{V}$, define $\vec{U}_{\alpha}$ = $[\alpha]_{\beta}$. Prove that $\vec{U}$ is an isomorphism of $\vec{V}$ onto $\vec{W}$. If $\vec{T}$ is a linear operator on $\vec{V}$, then $\vec{UTU^{-1}}$ is a linear operator on $\vec{W}$. Accordingly, $\vec{UTU^{-1}}$ is left multiplication by some $n \times n$ matrix $\vec{A}$.
What is A?
%
\\
\solution
See Table \ref{eq:solutions/6/2/11/table:1}
\onecolumn
\begin{longtable}{|l|l|}
\hline
\multirow{3}{*}{} & \\
Statement&Solution\\
\hline
&\parbox{6cm}{\begin{align}
\mbox{Let }\vec{N}&=\myvec{a&b\\c&d}\\
\mbox{Since }\vec{N}^2&=0
\end{align}}\\
&If $\myvec{a\\c},\myvec{b\\d}$ are linearly independent then $\vec{N}$ is diagonalizable to $\myvec{0&0\\0&0}$.\\
&\parbox{6cm}{\begin{align}
\mbox{If }\vec{P}\vec{N}\vec{P}^{-1}=0\\
\mbox{then }\vec{N}=\vec{P}^{-1}\vec{0}\vec{P}=0
\end{align}}\\
Proof that&So in this case $\vec{N}$ itself is the zero matrix.\\
$\vec{N}=0$&This contradicts the assumption that $\myvec{a\\c},\myvec{b\\d}$ are linearly independent.\\
&$\therefore$ we can assume that $\myvec{a\\c},\myvec{b\\d}$ are linearly dependent if both are\\
&equal to the zero vector\\
&\parbox{6cm}
{\begin{align}
   \mbox{then } \vec{N} &= 0.
\end{align}}\\
%\hline
%\pagebreak
\hline
&\\
&Therefore we can assume at least one vector is non-zero.\\
Assuming $\myvec{b\\d}$ as&Therefore
$\vec{N}=\myvec{a&0\\c&0}$\\
the zero vector&\\
&\parbox{6cm}{\begin{align}
\mbox{So }\vec{N}^2&=0\\
\implies a^2&=0\\
\therefore a&=0\\
\mbox{Thus }\vec{N}&=\myvec{a&0\\c&0}
\end{align}}\\
&In this case $\vec{N}$ is similar to $\vec{N}=\myvec{0&0\\1&0}$ via the matrix $\vec{P}=\myvec{c&0\\0&1}$\\
&\\
\hline
&\\
Assuming $\myvec{a\\c}$ as&Therefore
$\vec{N}=\myvec{0&b\\0&d}$\\
the zero vector&\\
&\parbox{6cm}{\begin{align}
\mbox{Then }\vec{N}^2&=0\\
\implies d^2&=0\\
\therefore d&=0\\
\mbox{Thus }\vec{N}&=\myvec{0&b\\0&0}
\end{align}}\\
&In this case $\vec{N}$ is similar to $\vec{N}=\myvec{0&0\\b&0}$ via the matrix $\vec{P}=\myvec{0&1\\1&0}$,\\
&which is similar to $\myvec{0&0\\1&0}$ as above.\\
&\\
\hline
&\\
Hence& we can assume neither $\myvec{a\\c}$ or $\myvec{b\\d}$ is the zero vector.\\
&\\
%\hline
%\pagebreak
\hline
&\\
Consequences of &Since they are linearly dependent we can assume,\\
linear&\\
independence&\\
&\parbox{6cm}{\begin{align}
\myvec{b\\d}&=x\myvec{a\\c}\\
\therefore \vec{N}&=\myvec{a&ax\\c&cx}\\
\therefore \vec{N}^2&=0\\
\implies a(a+cx)&=0\\
c(a+cx)&=0\\
ax(a+cx)&=0\\
cx(a+cx)&=0
\end{align}}\\
\hline
&\\
Proof that $\vec{N}$ is&We know that at least one of a or c is not zero.\\
similar over $\mathbb{C}$ to&If a = 0 then c $\neq$ 0, it must be that x = 0.\\
$\myvec{0&0\\1&0}$&So in this case $\vec{N}=\myvec{0&0\\c&0}$ which is similar to $\myvec{0&0\\1&0}$ as before.\\
&\\
&\parbox{6cm}{\begin{align}
\mbox{If } a &\neq0\\
\mbox{then }x &\neq0\\
\mbox{else }a(a+cx)&=0\\
\implies a&=0\\
\mbox{Thus }a+cx&=0\\
\mbox{Hence }\vec{N}&=\myvec{a&ax\\ \frac{-a}{x}&-a}
\end{align}}\\
 &This is similar to $\myvec{a&a\\-a&-a}$ via $\vec{P}=\myvec{\sqrt{x}&0\\0&\frac{1}{\sqrt{x}}}$.\\
 &And $\myvec{a&a\\-a&-a}$ is similar to $\myvec{0&0\\-a&0}$ via $\vec{P}=\myvec{-1&-1\\1&0}$\\
 &And this finally is similar to $\myvec{0&0\\1&0}$ as before.\\
 &\\
\hline
&\\
Conclusion &Thus either $\vec{N}$ = 0 or $\vec{N}$ is similar over $\mathbb{C}$ to $\myvec{0&0\\1&0}$.\\
&\\
\hline
\caption{Solution summary}
\label{eq:solutions/6/2/11/table:1}
\end{longtable}

\item Let V be an n-dimensional vector space over the field F, and let $\mathcal{B} = \cbrak{\alpha_1,\ldots, \alpha_n}$ be an ordered basis for V then there is a unique linear operator T on V such that 
\begin{align}
   &T\alpha_{j} = \alpha_{j+1},  j=1,\ldots,n-1 \\
   &T\alpha_n = 0. 
\end{align}
%
\begin{enumerate}
\item What is the matrix A of T in the ordered basis $\mathcal{B}$?
\\
\solution
See Table \ref{eq:solutions/6/2/11/table:1}
\onecolumn
\begin{longtable}{|l|l|}
\hline
\multirow{3}{*}{} & \\
Statement&Solution\\
\hline
&\parbox{6cm}{\begin{align}
\mbox{Let }\vec{N}&=\myvec{a&b\\c&d}\\
\mbox{Since }\vec{N}^2&=0
\end{align}}\\
&If $\myvec{a\\c},\myvec{b\\d}$ are linearly independent then $\vec{N}$ is diagonalizable to $\myvec{0&0\\0&0}$.\\
&\parbox{6cm}{\begin{align}
\mbox{If }\vec{P}\vec{N}\vec{P}^{-1}=0\\
\mbox{then }\vec{N}=\vec{P}^{-1}\vec{0}\vec{P}=0
\end{align}}\\
Proof that&So in this case $\vec{N}$ itself is the zero matrix.\\
$\vec{N}=0$&This contradicts the assumption that $\myvec{a\\c},\myvec{b\\d}$ are linearly independent.\\
&$\therefore$ we can assume that $\myvec{a\\c},\myvec{b\\d}$ are linearly dependent if both are\\
&equal to the zero vector\\
&\parbox{6cm}
{\begin{align}
   \mbox{then } \vec{N} &= 0.
\end{align}}\\
%\hline
%\pagebreak
\hline
&\\
&Therefore we can assume at least one vector is non-zero.\\
Assuming $\myvec{b\\d}$ as&Therefore
$\vec{N}=\myvec{a&0\\c&0}$\\
the zero vector&\\
&\parbox{6cm}{\begin{align}
\mbox{So }\vec{N}^2&=0\\
\implies a^2&=0\\
\therefore a&=0\\
\mbox{Thus }\vec{N}&=\myvec{a&0\\c&0}
\end{align}}\\
&In this case $\vec{N}$ is similar to $\vec{N}=\myvec{0&0\\1&0}$ via the matrix $\vec{P}=\myvec{c&0\\0&1}$\\
&\\
\hline
&\\
Assuming $\myvec{a\\c}$ as&Therefore
$\vec{N}=\myvec{0&b\\0&d}$\\
the zero vector&\\
&\parbox{6cm}{\begin{align}
\mbox{Then }\vec{N}^2&=0\\
\implies d^2&=0\\
\therefore d&=0\\
\mbox{Thus }\vec{N}&=\myvec{0&b\\0&0}
\end{align}}\\
&In this case $\vec{N}$ is similar to $\vec{N}=\myvec{0&0\\b&0}$ via the matrix $\vec{P}=\myvec{0&1\\1&0}$,\\
&which is similar to $\myvec{0&0\\1&0}$ as above.\\
&\\
\hline
&\\
Hence& we can assume neither $\myvec{a\\c}$ or $\myvec{b\\d}$ is the zero vector.\\
&\\
%\hline
%\pagebreak
\hline
&\\
Consequences of &Since they are linearly dependent we can assume,\\
linear&\\
independence&\\
&\parbox{6cm}{\begin{align}
\myvec{b\\d}&=x\myvec{a\\c}\\
\therefore \vec{N}&=\myvec{a&ax\\c&cx}\\
\therefore \vec{N}^2&=0\\
\implies a(a+cx)&=0\\
c(a+cx)&=0\\
ax(a+cx)&=0\\
cx(a+cx)&=0
\end{align}}\\
\hline
&\\
Proof that $\vec{N}$ is&We know that at least one of a or c is not zero.\\
similar over $\mathbb{C}$ to&If a = 0 then c $\neq$ 0, it must be that x = 0.\\
$\myvec{0&0\\1&0}$&So in this case $\vec{N}=\myvec{0&0\\c&0}$ which is similar to $\myvec{0&0\\1&0}$ as before.\\
&\\
&\parbox{6cm}{\begin{align}
\mbox{If } a &\neq0\\
\mbox{then }x &\neq0\\
\mbox{else }a(a+cx)&=0\\
\implies a&=0\\
\mbox{Thus }a+cx&=0\\
\mbox{Hence }\vec{N}&=\myvec{a&ax\\ \frac{-a}{x}&-a}
\end{align}}\\
 &This is similar to $\myvec{a&a\\-a&-a}$ via $\vec{P}=\myvec{\sqrt{x}&0\\0&\frac{1}{\sqrt{x}}}$.\\
 &And $\myvec{a&a\\-a&-a}$ is similar to $\myvec{0&0\\-a&0}$ via $\vec{P}=\myvec{-1&-1\\1&0}$\\
 &And this finally is similar to $\myvec{0&0\\1&0}$ as before.\\
 &\\
\hline
&\\
Conclusion &Thus either $\vec{N}$ = 0 or $\vec{N}$ is similar over $\mathbb{C}$ to $\myvec{0&0\\1&0}$.\\
&\\
\hline
\caption{Solution summary}
\label{eq:solutions/6/2/11/table:1}
\end{longtable}

\end{enumerate}
\item Let $V$ and $W$ be finite-dimensional vector spaces over the field $F$ and let $T$ be a linear transformation from $V$ into $W$. If
\begin{equation}
	\mathcal{B} = \{\alpha_1, \dots, \alpha_n\} \text{ and } \mathcal{B}' = \{\beta_1, \dots, \beta_m\}
\end{equation}
are ordered bases for $V$ and $W$, respectively, define the linear transformation $E^{p,q}$ as in the proof of Theorem 5: $E^{p,q}(\alpha_i) = \delta_{iq}\beta_p$. Then the $E^{p,q}$, $1 \leq p \leq m$, $1 \leq q \leq n$, form a basis for $L(V,W)$ and so
\begin{equation}
	T = \displaystyle \sum_{p=1}^m \displaystyle \sum_{q=1}^n A_{pq}E^{p,q}
\end{equation} 
for certain scalars $A_{pq}$ (the coordinates of $T$ in this basis for $L(V,W))$. Show that the matrix $A$ with entries $A(p,q) = A_{pq}$ is precisely the matrix of $T$ relative to the pair $\mathcal{B},\mathcal{B}'$.
%
\\
\solution
See Table \ref{eq:solutions/6/2/11/table:1}
\onecolumn
\begin{longtable}{|l|l|}
\hline
\multirow{3}{*}{} & \\
Statement&Solution\\
\hline
&\parbox{6cm}{\begin{align}
\mbox{Let }\vec{N}&=\myvec{a&b\\c&d}\\
\mbox{Since }\vec{N}^2&=0
\end{align}}\\
&If $\myvec{a\\c},\myvec{b\\d}$ are linearly independent then $\vec{N}$ is diagonalizable to $\myvec{0&0\\0&0}$.\\
&\parbox{6cm}{\begin{align}
\mbox{If }\vec{P}\vec{N}\vec{P}^{-1}=0\\
\mbox{then }\vec{N}=\vec{P}^{-1}\vec{0}\vec{P}=0
\end{align}}\\
Proof that&So in this case $\vec{N}$ itself is the zero matrix.\\
$\vec{N}=0$&This contradicts the assumption that $\myvec{a\\c},\myvec{b\\d}$ are linearly independent.\\
&$\therefore$ we can assume that $\myvec{a\\c},\myvec{b\\d}$ are linearly dependent if both are\\
&equal to the zero vector\\
&\parbox{6cm}
{\begin{align}
   \mbox{then } \vec{N} &= 0.
\end{align}}\\
%\hline
%\pagebreak
\hline
&\\
&Therefore we can assume at least one vector is non-zero.\\
Assuming $\myvec{b\\d}$ as&Therefore
$\vec{N}=\myvec{a&0\\c&0}$\\
the zero vector&\\
&\parbox{6cm}{\begin{align}
\mbox{So }\vec{N}^2&=0\\
\implies a^2&=0\\
\therefore a&=0\\
\mbox{Thus }\vec{N}&=\myvec{a&0\\c&0}
\end{align}}\\
&In this case $\vec{N}$ is similar to $\vec{N}=\myvec{0&0\\1&0}$ via the matrix $\vec{P}=\myvec{c&0\\0&1}$\\
&\\
\hline
&\\
Assuming $\myvec{a\\c}$ as&Therefore
$\vec{N}=\myvec{0&b\\0&d}$\\
the zero vector&\\
&\parbox{6cm}{\begin{align}
\mbox{Then }\vec{N}^2&=0\\
\implies d^2&=0\\
\therefore d&=0\\
\mbox{Thus }\vec{N}&=\myvec{0&b\\0&0}
\end{align}}\\
&In this case $\vec{N}$ is similar to $\vec{N}=\myvec{0&0\\b&0}$ via the matrix $\vec{P}=\myvec{0&1\\1&0}$,\\
&which is similar to $\myvec{0&0\\1&0}$ as above.\\
&\\
\hline
&\\
Hence& we can assume neither $\myvec{a\\c}$ or $\myvec{b\\d}$ is the zero vector.\\
&\\
%\hline
%\pagebreak
\hline
&\\
Consequences of &Since they are linearly dependent we can assume,\\
linear&\\
independence&\\
&\parbox{6cm}{\begin{align}
\myvec{b\\d}&=x\myvec{a\\c}\\
\therefore \vec{N}&=\myvec{a&ax\\c&cx}\\
\therefore \vec{N}^2&=0\\
\implies a(a+cx)&=0\\
c(a+cx)&=0\\
ax(a+cx)&=0\\
cx(a+cx)&=0
\end{align}}\\
\hline
&\\
Proof that $\vec{N}$ is&We know that at least one of a or c is not zero.\\
similar over $\mathbb{C}$ to&If a = 0 then c $\neq$ 0, it must be that x = 0.\\
$\myvec{0&0\\1&0}$&So in this case $\vec{N}=\myvec{0&0\\c&0}$ which is similar to $\myvec{0&0\\1&0}$ as before.\\
&\\
&\parbox{6cm}{\begin{align}
\mbox{If } a &\neq0\\
\mbox{then }x &\neq0\\
\mbox{else }a(a+cx)&=0\\
\implies a&=0\\
\mbox{Thus }a+cx&=0\\
\mbox{Hence }\vec{N}&=\myvec{a&ax\\ \frac{-a}{x}&-a}
\end{align}}\\
 &This is similar to $\myvec{a&a\\-a&-a}$ via $\vec{P}=\myvec{\sqrt{x}&0\\0&\frac{1}{\sqrt{x}}}$.\\
 &And $\myvec{a&a\\-a&-a}$ is similar to $\myvec{0&0\\-a&0}$ via $\vec{P}=\myvec{-1&-1\\1&0}$\\
 &And this finally is similar to $\myvec{0&0\\1&0}$ as before.\\
 &\\
\hline
&\\
Conclusion &Thus either $\vec{N}$ = 0 or $\vec{N}$ is similar over $\mathbb{C}$ to $\myvec{0&0\\1&0}$.\\
&\\
\hline
\caption{Solution summary}
\label{eq:solutions/6/2/11/table:1}
\end{longtable}


\end{enumerate}

\subsection{Linear Functionals}
\renewcommand{\theequation}{\theenumi}
\renewcommand{\thefigure}{\theenumi}
\begin{enumerate}[label=\thesubsection.\arabic*.,ref=\thesubsection.\theenumi]
\numberwithin{equation}{enumi}
\numberwithin{figure}{enumi}

\item Let $\mathbb{V}$ be the vector space of all $2 \times 2$ matrices over the field of real numbers, and let
\begin{align}
\vec{B} &= \myvec{2&-2\\-1&1}
\end{align}
Let $\mathbb{W}$ be the subspace of $\mathbb{V}$ consisting of all $\vec{A}$ such that $\vec{AB} = 0$. Let $f$ be a linear functional on $\mathbb{V}$ which is in the annihilator of $\mathbb{W}$. Suppose that $f(\vec{I}) = 0$ and $f(\vec{C}) = 3$, where $\vec{I}$ is the $2 \times 2$ identity matrix and
\begin{align}
\vec{C} &= \myvec{0&0\\0&1}
\end{align}
Find $f(\vec{B})$
%
\\
\solution
%See Table \ref{eq:solutions/6/2/11/table:1}
\onecolumn
\begin{longtable}{|l|l|}
\hline
\multirow{3}{*}{} & \\
Statement&Solution\\
\hline
&\parbox{6cm}{\begin{align}
\mbox{Let }\vec{N}&=\myvec{a&b\\c&d}\\
\mbox{Since }\vec{N}^2&=0
\end{align}}\\
&If $\myvec{a\\c},\myvec{b\\d}$ are linearly independent then $\vec{N}$ is diagonalizable to $\myvec{0&0\\0&0}$.\\
&\parbox{6cm}{\begin{align}
\mbox{If }\vec{P}\vec{N}\vec{P}^{-1}=0\\
\mbox{then }\vec{N}=\vec{P}^{-1}\vec{0}\vec{P}=0
\end{align}}\\
Proof that&So in this case $\vec{N}$ itself is the zero matrix.\\
$\vec{N}=0$&This contradicts the assumption that $\myvec{a\\c},\myvec{b\\d}$ are linearly independent.\\
&$\therefore$ we can assume that $\myvec{a\\c},\myvec{b\\d}$ are linearly dependent if both are\\
&equal to the zero vector\\
&\parbox{6cm}
{\begin{align}
   \mbox{then } \vec{N} &= 0.
\end{align}}\\
%\hline
%\pagebreak
\hline
&\\
&Therefore we can assume at least one vector is non-zero.\\
Assuming $\myvec{b\\d}$ as&Therefore
$\vec{N}=\myvec{a&0\\c&0}$\\
the zero vector&\\
&\parbox{6cm}{\begin{align}
\mbox{So }\vec{N}^2&=0\\
\implies a^2&=0\\
\therefore a&=0\\
\mbox{Thus }\vec{N}&=\myvec{a&0\\c&0}
\end{align}}\\
&In this case $\vec{N}$ is similar to $\vec{N}=\myvec{0&0\\1&0}$ via the matrix $\vec{P}=\myvec{c&0\\0&1}$\\
&\\
\hline
&\\
Assuming $\myvec{a\\c}$ as&Therefore
$\vec{N}=\myvec{0&b\\0&d}$\\
the zero vector&\\
&\parbox{6cm}{\begin{align}
\mbox{Then }\vec{N}^2&=0\\
\implies d^2&=0\\
\therefore d&=0\\
\mbox{Thus }\vec{N}&=\myvec{0&b\\0&0}
\end{align}}\\
&In this case $\vec{N}$ is similar to $\vec{N}=\myvec{0&0\\b&0}$ via the matrix $\vec{P}=\myvec{0&1\\1&0}$,\\
&which is similar to $\myvec{0&0\\1&0}$ as above.\\
&\\
\hline
&\\
Hence& we can assume neither $\myvec{a\\c}$ or $\myvec{b\\d}$ is the zero vector.\\
&\\
%\hline
%\pagebreak
\hline
&\\
Consequences of &Since they are linearly dependent we can assume,\\
linear&\\
independence&\\
&\parbox{6cm}{\begin{align}
\myvec{b\\d}&=x\myvec{a\\c}\\
\therefore \vec{N}&=\myvec{a&ax\\c&cx}\\
\therefore \vec{N}^2&=0\\
\implies a(a+cx)&=0\\
c(a+cx)&=0\\
ax(a+cx)&=0\\
cx(a+cx)&=0
\end{align}}\\
\hline
&\\
Proof that $\vec{N}$ is&We know that at least one of a or c is not zero.\\
similar over $\mathbb{C}$ to&If a = 0 then c $\neq$ 0, it must be that x = 0.\\
$\myvec{0&0\\1&0}$&So in this case $\vec{N}=\myvec{0&0\\c&0}$ which is similar to $\myvec{0&0\\1&0}$ as before.\\
&\\
&\parbox{6cm}{\begin{align}
\mbox{If } a &\neq0\\
\mbox{then }x &\neq0\\
\mbox{else }a(a+cx)&=0\\
\implies a&=0\\
\mbox{Thus }a+cx&=0\\
\mbox{Hence }\vec{N}&=\myvec{a&ax\\ \frac{-a}{x}&-a}
\end{align}}\\
 &This is similar to $\myvec{a&a\\-a&-a}$ via $\vec{P}=\myvec{\sqrt{x}&0\\0&\frac{1}{\sqrt{x}}}$.\\
 &And $\myvec{a&a\\-a&-a}$ is similar to $\myvec{0&0\\-a&0}$ via $\vec{P}=\myvec{-1&-1\\1&0}$\\
 &And this finally is similar to $\myvec{0&0\\1&0}$ as before.\\
 &\\
\hline
&\\
Conclusion &Thus either $\vec{N}$ = 0 or $\vec{N}$ is similar over $\mathbb{C}$ to $\myvec{0&0\\1&0}$.\\
&\\
\hline
\caption{Solution summary}
\label{eq:solutions/6/2/11/table:1}
\end{longtable}

\item Let $\mathbb{F}$ be a field of characteristic zero and let $\vec{V}$ be a finite dimensional vector space over  $\mathbb{F}$.If $\alpha_1,\alpha_2,\hdots,\alpha_m$ are finitely many vectors in $\vec{V}$ , each different from the zero vector, prove that there is a linear functional $f$ on $\vec{V}$ such that
\begin{align}
    f(\alpha_i) \neq 0, i=1,2,\hdots,m
\end{align}
%
\\
\solution
See Table \ref{eq:solutions/6/2/11/table:1}
\onecolumn
\begin{longtable}{|l|l|}
\hline
\multirow{3}{*}{} & \\
Statement&Solution\\
\hline
&\parbox{6cm}{\begin{align}
\mbox{Let }\vec{N}&=\myvec{a&b\\c&d}\\
\mbox{Since }\vec{N}^2&=0
\end{align}}\\
&If $\myvec{a\\c},\myvec{b\\d}$ are linearly independent then $\vec{N}$ is diagonalizable to $\myvec{0&0\\0&0}$.\\
&\parbox{6cm}{\begin{align}
\mbox{If }\vec{P}\vec{N}\vec{P}^{-1}=0\\
\mbox{then }\vec{N}=\vec{P}^{-1}\vec{0}\vec{P}=0
\end{align}}\\
Proof that&So in this case $\vec{N}$ itself is the zero matrix.\\
$\vec{N}=0$&This contradicts the assumption that $\myvec{a\\c},\myvec{b\\d}$ are linearly independent.\\
&$\therefore$ we can assume that $\myvec{a\\c},\myvec{b\\d}$ are linearly dependent if both are\\
&equal to the zero vector\\
&\parbox{6cm}
{\begin{align}
   \mbox{then } \vec{N} &= 0.
\end{align}}\\
%\hline
%\pagebreak
\hline
&\\
&Therefore we can assume at least one vector is non-zero.\\
Assuming $\myvec{b\\d}$ as&Therefore
$\vec{N}=\myvec{a&0\\c&0}$\\
the zero vector&\\
&\parbox{6cm}{\begin{align}
\mbox{So }\vec{N}^2&=0\\
\implies a^2&=0\\
\therefore a&=0\\
\mbox{Thus }\vec{N}&=\myvec{a&0\\c&0}
\end{align}}\\
&In this case $\vec{N}$ is similar to $\vec{N}=\myvec{0&0\\1&0}$ via the matrix $\vec{P}=\myvec{c&0\\0&1}$\\
&\\
\hline
&\\
Assuming $\myvec{a\\c}$ as&Therefore
$\vec{N}=\myvec{0&b\\0&d}$\\
the zero vector&\\
&\parbox{6cm}{\begin{align}
\mbox{Then }\vec{N}^2&=0\\
\implies d^2&=0\\
\therefore d&=0\\
\mbox{Thus }\vec{N}&=\myvec{0&b\\0&0}
\end{align}}\\
&In this case $\vec{N}$ is similar to $\vec{N}=\myvec{0&0\\b&0}$ via the matrix $\vec{P}=\myvec{0&1\\1&0}$,\\
&which is similar to $\myvec{0&0\\1&0}$ as above.\\
&\\
\hline
&\\
Hence& we can assume neither $\myvec{a\\c}$ or $\myvec{b\\d}$ is the zero vector.\\
&\\
%\hline
%\pagebreak
\hline
&\\
Consequences of &Since they are linearly dependent we can assume,\\
linear&\\
independence&\\
&\parbox{6cm}{\begin{align}
\myvec{b\\d}&=x\myvec{a\\c}\\
\therefore \vec{N}&=\myvec{a&ax\\c&cx}\\
\therefore \vec{N}^2&=0\\
\implies a(a+cx)&=0\\
c(a+cx)&=0\\
ax(a+cx)&=0\\
cx(a+cx)&=0
\end{align}}\\
\hline
&\\
Proof that $\vec{N}$ is&We know that at least one of a or c is not zero.\\
similar over $\mathbb{C}$ to&If a = 0 then c $\neq$ 0, it must be that x = 0.\\
$\myvec{0&0\\1&0}$&So in this case $\vec{N}=\myvec{0&0\\c&0}$ which is similar to $\myvec{0&0\\1&0}$ as before.\\
&\\
&\parbox{6cm}{\begin{align}
\mbox{If } a &\neq0\\
\mbox{then }x &\neq0\\
\mbox{else }a(a+cx)&=0\\
\implies a&=0\\
\mbox{Thus }a+cx&=0\\
\mbox{Hence }\vec{N}&=\myvec{a&ax\\ \frac{-a}{x}&-a}
\end{align}}\\
 &This is similar to $\myvec{a&a\\-a&-a}$ via $\vec{P}=\myvec{\sqrt{x}&0\\0&\frac{1}{\sqrt{x}}}$.\\
 &And $\myvec{a&a\\-a&-a}$ is similar to $\myvec{0&0\\-a&0}$ via $\vec{P}=\myvec{-1&-1\\1&0}$\\
 &And this finally is similar to $\myvec{0&0\\1&0}$ as before.\\
 &\\
\hline
&\\
Conclusion &Thus either $\vec{N}$ = 0 or $\vec{N}$ is similar over $\mathbb{C}$ to $\myvec{0&0\\1&0}$.\\
&\\
\hline
\caption{Solution summary}
\label{eq:solutions/6/2/11/table:1}
\end{longtable}

\item Show that the trace functional on $n\times n$ matrices is unique in the following sense. If $W$ is the space of $n \times n$ matrices over the field $F$ and if $f$ is a linear functional on $W$ such that $f(AB) = f(BA)$ for each $A$ and $B$ in $W$, then $f$ is a scalar multiple of the trace function. If, in addition, $f(I)=n$, then $f$ is the trace function.
\\
\solution
See Table \ref{eq:solutions/6/2/11/table:1}
\onecolumn
\begin{longtable}{|l|l|}
\hline
\multirow{3}{*}{} & \\
Statement&Solution\\
\hline
&\parbox{6cm}{\begin{align}
\mbox{Let }\vec{N}&=\myvec{a&b\\c&d}\\
\mbox{Since }\vec{N}^2&=0
\end{align}}\\
&If $\myvec{a\\c},\myvec{b\\d}$ are linearly independent then $\vec{N}$ is diagonalizable to $\myvec{0&0\\0&0}$.\\
&\parbox{6cm}{\begin{align}
\mbox{If }\vec{P}\vec{N}\vec{P}^{-1}=0\\
\mbox{then }\vec{N}=\vec{P}^{-1}\vec{0}\vec{P}=0
\end{align}}\\
Proof that&So in this case $\vec{N}$ itself is the zero matrix.\\
$\vec{N}=0$&This contradicts the assumption that $\myvec{a\\c},\myvec{b\\d}$ are linearly independent.\\
&$\therefore$ we can assume that $\myvec{a\\c},\myvec{b\\d}$ are linearly dependent if both are\\
&equal to the zero vector\\
&\parbox{6cm}
{\begin{align}
   \mbox{then } \vec{N} &= 0.
\end{align}}\\
%\hline
%\pagebreak
\hline
&\\
&Therefore we can assume at least one vector is non-zero.\\
Assuming $\myvec{b\\d}$ as&Therefore
$\vec{N}=\myvec{a&0\\c&0}$\\
the zero vector&\\
&\parbox{6cm}{\begin{align}
\mbox{So }\vec{N}^2&=0\\
\implies a^2&=0\\
\therefore a&=0\\
\mbox{Thus }\vec{N}&=\myvec{a&0\\c&0}
\end{align}}\\
&In this case $\vec{N}$ is similar to $\vec{N}=\myvec{0&0\\1&0}$ via the matrix $\vec{P}=\myvec{c&0\\0&1}$\\
&\\
\hline
&\\
Assuming $\myvec{a\\c}$ as&Therefore
$\vec{N}=\myvec{0&b\\0&d}$\\
the zero vector&\\
&\parbox{6cm}{\begin{align}
\mbox{Then }\vec{N}^2&=0\\
\implies d^2&=0\\
\therefore d&=0\\
\mbox{Thus }\vec{N}&=\myvec{0&b\\0&0}
\end{align}}\\
&In this case $\vec{N}$ is similar to $\vec{N}=\myvec{0&0\\b&0}$ via the matrix $\vec{P}=\myvec{0&1\\1&0}$,\\
&which is similar to $\myvec{0&0\\1&0}$ as above.\\
&\\
\hline
&\\
Hence& we can assume neither $\myvec{a\\c}$ or $\myvec{b\\d}$ is the zero vector.\\
&\\
%\hline
%\pagebreak
\hline
&\\
Consequences of &Since they are linearly dependent we can assume,\\
linear&\\
independence&\\
&\parbox{6cm}{\begin{align}
\myvec{b\\d}&=x\myvec{a\\c}\\
\therefore \vec{N}&=\myvec{a&ax\\c&cx}\\
\therefore \vec{N}^2&=0\\
\implies a(a+cx)&=0\\
c(a+cx)&=0\\
ax(a+cx)&=0\\
cx(a+cx)&=0
\end{align}}\\
\hline
&\\
Proof that $\vec{N}$ is&We know that at least one of a or c is not zero.\\
similar over $\mathbb{C}$ to&If a = 0 then c $\neq$ 0, it must be that x = 0.\\
$\myvec{0&0\\1&0}$&So in this case $\vec{N}=\myvec{0&0\\c&0}$ which is similar to $\myvec{0&0\\1&0}$ as before.\\
&\\
&\parbox{6cm}{\begin{align}
\mbox{If } a &\neq0\\
\mbox{then }x &\neq0\\
\mbox{else }a(a+cx)&=0\\
\implies a&=0\\
\mbox{Thus }a+cx&=0\\
\mbox{Hence }\vec{N}&=\myvec{a&ax\\ \frac{-a}{x}&-a}
\end{align}}\\
 &This is similar to $\myvec{a&a\\-a&-a}$ via $\vec{P}=\myvec{\sqrt{x}&0\\0&\frac{1}{\sqrt{x}}}$.\\
 &And $\myvec{a&a\\-a&-a}$ is similar to $\myvec{0&0\\-a&0}$ via $\vec{P}=\myvec{-1&-1\\1&0}$\\
 &And this finally is similar to $\myvec{0&0\\1&0}$ as before.\\
 &\\
\hline
&\\
Conclusion &Thus either $\vec{N}$ = 0 or $\vec{N}$ is similar over $\mathbb{C}$ to $\myvec{0&0\\1&0}$.\\
&\\
\hline
\caption{Solution summary}
\label{eq:solutions/6/2/11/table:1}
\end{longtable}

\end{enumerate}

\subsection{The Double Dual}
\renewcommand{\theequation}{\theenumi}
\renewcommand{\thefigure}{\theenumi}
\begin{enumerate}[label=\thesubsection.\arabic*.,ref=\thesubsection.\theenumi]
\numberwithin{equation}{enumi}
\numberwithin{figure}{enumi}

\item Use theorem 20 to prove the following.If $\vec{W}$ is a subspace of a finite dimensional vector space $\vec{V}$ and if $\cbrak{g_1,g_2,\hdots,g_r}$ is any basis for $\vec{W}^0$ then
\begin{align}
    \vec{W}=\bigcap_{i=1}^r \vec{N}_{g_i}
\end{align}
%
\\
\solution
See Table \ref{eq:solutions/6/2/11/table:1}
\onecolumn
\begin{longtable}{|l|l|}
\hline
\multirow{3}{*}{} & \\
Statement&Solution\\
\hline
&\parbox{6cm}{\begin{align}
\mbox{Let }\vec{N}&=\myvec{a&b\\c&d}\\
\mbox{Since }\vec{N}^2&=0
\end{align}}\\
&If $\myvec{a\\c},\myvec{b\\d}$ are linearly independent then $\vec{N}$ is diagonalizable to $\myvec{0&0\\0&0}$.\\
&\parbox{6cm}{\begin{align}
\mbox{If }\vec{P}\vec{N}\vec{P}^{-1}=0\\
\mbox{then }\vec{N}=\vec{P}^{-1}\vec{0}\vec{P}=0
\end{align}}\\
Proof that&So in this case $\vec{N}$ itself is the zero matrix.\\
$\vec{N}=0$&This contradicts the assumption that $\myvec{a\\c},\myvec{b\\d}$ are linearly independent.\\
&$\therefore$ we can assume that $\myvec{a\\c},\myvec{b\\d}$ are linearly dependent if both are\\
&equal to the zero vector\\
&\parbox{6cm}
{\begin{align}
   \mbox{then } \vec{N} &= 0.
\end{align}}\\
%\hline
%\pagebreak
\hline
&\\
&Therefore we can assume at least one vector is non-zero.\\
Assuming $\myvec{b\\d}$ as&Therefore
$\vec{N}=\myvec{a&0\\c&0}$\\
the zero vector&\\
&\parbox{6cm}{\begin{align}
\mbox{So }\vec{N}^2&=0\\
\implies a^2&=0\\
\therefore a&=0\\
\mbox{Thus }\vec{N}&=\myvec{a&0\\c&0}
\end{align}}\\
&In this case $\vec{N}$ is similar to $\vec{N}=\myvec{0&0\\1&0}$ via the matrix $\vec{P}=\myvec{c&0\\0&1}$\\
&\\
\hline
&\\
Assuming $\myvec{a\\c}$ as&Therefore
$\vec{N}=\myvec{0&b\\0&d}$\\
the zero vector&\\
&\parbox{6cm}{\begin{align}
\mbox{Then }\vec{N}^2&=0\\
\implies d^2&=0\\
\therefore d&=0\\
\mbox{Thus }\vec{N}&=\myvec{0&b\\0&0}
\end{align}}\\
&In this case $\vec{N}$ is similar to $\vec{N}=\myvec{0&0\\b&0}$ via the matrix $\vec{P}=\myvec{0&1\\1&0}$,\\
&which is similar to $\myvec{0&0\\1&0}$ as above.\\
&\\
\hline
&\\
Hence& we can assume neither $\myvec{a\\c}$ or $\myvec{b\\d}$ is the zero vector.\\
&\\
%\hline
%\pagebreak
\hline
&\\
Consequences of &Since they are linearly dependent we can assume,\\
linear&\\
independence&\\
&\parbox{6cm}{\begin{align}
\myvec{b\\d}&=x\myvec{a\\c}\\
\therefore \vec{N}&=\myvec{a&ax\\c&cx}\\
\therefore \vec{N}^2&=0\\
\implies a(a+cx)&=0\\
c(a+cx)&=0\\
ax(a+cx)&=0\\
cx(a+cx)&=0
\end{align}}\\
\hline
&\\
Proof that $\vec{N}$ is&We know that at least one of a or c is not zero.\\
similar over $\mathbb{C}$ to&If a = 0 then c $\neq$ 0, it must be that x = 0.\\
$\myvec{0&0\\1&0}$&So in this case $\vec{N}=\myvec{0&0\\c&0}$ which is similar to $\myvec{0&0\\1&0}$ as before.\\
&\\
&\parbox{6cm}{\begin{align}
\mbox{If } a &\neq0\\
\mbox{then }x &\neq0\\
\mbox{else }a(a+cx)&=0\\
\implies a&=0\\
\mbox{Thus }a+cx&=0\\
\mbox{Hence }\vec{N}&=\myvec{a&ax\\ \frac{-a}{x}&-a}
\end{align}}\\
 &This is similar to $\myvec{a&a\\-a&-a}$ via $\vec{P}=\myvec{\sqrt{x}&0\\0&\frac{1}{\sqrt{x}}}$.\\
 &And $\myvec{a&a\\-a&-a}$ is similar to $\myvec{0&0\\-a&0}$ via $\vec{P}=\myvec{-1&-1\\1&0}$\\
 &And this finally is similar to $\myvec{0&0\\1&0}$ as before.\\
 &\\
\hline
&\\
Conclusion &Thus either $\vec{N}$ = 0 or $\vec{N}$ is similar over $\mathbb{C}$ to $\myvec{0&0\\1&0}$.\\
&\\
\hline
\caption{Solution summary}
\label{eq:solutions/6/2/11/table:1}
\end{longtable}


\item Let $S$ be a set, $F$ a field, and $V(S;F)$ the space of all functions from $S$ into $F$:
\begin{align}
	\nonumber (f+g)(x) &= f(x) + g(x)\\
	\nonumber (cf)(x) &= cf(x)
\end{align}
Let $W$ be any $n$-dimensional  subspace of $V(S,F)$. Show that there exist points $x_1,x_2, \dots, x_n$ in $S$ and functions $f_1,\dots,f_n$ in $W$ such that $f_i(x_j) = \delta_{ij}$
%
\\
\solution
See Table \ref{eq:solutions/6/2/11/table:1}
\onecolumn
\begin{longtable}{|l|l|}
\hline
\multirow{3}{*}{} & \\
Statement&Solution\\
\hline
&\parbox{6cm}{\begin{align}
\mbox{Let }\vec{N}&=\myvec{a&b\\c&d}\\
\mbox{Since }\vec{N}^2&=0
\end{align}}\\
&If $\myvec{a\\c},\myvec{b\\d}$ are linearly independent then $\vec{N}$ is diagonalizable to $\myvec{0&0\\0&0}$.\\
&\parbox{6cm}{\begin{align}
\mbox{If }\vec{P}\vec{N}\vec{P}^{-1}=0\\
\mbox{then }\vec{N}=\vec{P}^{-1}\vec{0}\vec{P}=0
\end{align}}\\
Proof that&So in this case $\vec{N}$ itself is the zero matrix.\\
$\vec{N}=0$&This contradicts the assumption that $\myvec{a\\c},\myvec{b\\d}$ are linearly independent.\\
&$\therefore$ we can assume that $\myvec{a\\c},\myvec{b\\d}$ are linearly dependent if both are\\
&equal to the zero vector\\
&\parbox{6cm}
{\begin{align}
   \mbox{then } \vec{N} &= 0.
\end{align}}\\
%\hline
%\pagebreak
\hline
&\\
&Therefore we can assume at least one vector is non-zero.\\
Assuming $\myvec{b\\d}$ as&Therefore
$\vec{N}=\myvec{a&0\\c&0}$\\
the zero vector&\\
&\parbox{6cm}{\begin{align}
\mbox{So }\vec{N}^2&=0\\
\implies a^2&=0\\
\therefore a&=0\\
\mbox{Thus }\vec{N}&=\myvec{a&0\\c&0}
\end{align}}\\
&In this case $\vec{N}$ is similar to $\vec{N}=\myvec{0&0\\1&0}$ via the matrix $\vec{P}=\myvec{c&0\\0&1}$\\
&\\
\hline
&\\
Assuming $\myvec{a\\c}$ as&Therefore
$\vec{N}=\myvec{0&b\\0&d}$\\
the zero vector&\\
&\parbox{6cm}{\begin{align}
\mbox{Then }\vec{N}^2&=0\\
\implies d^2&=0\\
\therefore d&=0\\
\mbox{Thus }\vec{N}&=\myvec{0&b\\0&0}
\end{align}}\\
&In this case $\vec{N}$ is similar to $\vec{N}=\myvec{0&0\\b&0}$ via the matrix $\vec{P}=\myvec{0&1\\1&0}$,\\
&which is similar to $\myvec{0&0\\1&0}$ as above.\\
&\\
\hline
&\\
Hence& we can assume neither $\myvec{a\\c}$ or $\myvec{b\\d}$ is the zero vector.\\
&\\
%\hline
%\pagebreak
\hline
&\\
Consequences of &Since they are linearly dependent we can assume,\\
linear&\\
independence&\\
&\parbox{6cm}{\begin{align}
\myvec{b\\d}&=x\myvec{a\\c}\\
\therefore \vec{N}&=\myvec{a&ax\\c&cx}\\
\therefore \vec{N}^2&=0\\
\implies a(a+cx)&=0\\
c(a+cx)&=0\\
ax(a+cx)&=0\\
cx(a+cx)&=0
\end{align}}\\
\hline
&\\
Proof that $\vec{N}$ is&We know that at least one of a or c is not zero.\\
similar over $\mathbb{C}$ to&If a = 0 then c $\neq$ 0, it must be that x = 0.\\
$\myvec{0&0\\1&0}$&So in this case $\vec{N}=\myvec{0&0\\c&0}$ which is similar to $\myvec{0&0\\1&0}$ as before.\\
&\\
&\parbox{6cm}{\begin{align}
\mbox{If } a &\neq0\\
\mbox{then }x &\neq0\\
\mbox{else }a(a+cx)&=0\\
\implies a&=0\\
\mbox{Thus }a+cx&=0\\
\mbox{Hence }\vec{N}&=\myvec{a&ax\\ \frac{-a}{x}&-a}
\end{align}}\\
 &This is similar to $\myvec{a&a\\-a&-a}$ via $\vec{P}=\myvec{\sqrt{x}&0\\0&\frac{1}{\sqrt{x}}}$.\\
 &And $\myvec{a&a\\-a&-a}$ is similar to $\myvec{0&0\\-a&0}$ via $\vec{P}=\myvec{-1&-1\\1&0}$\\
 &And this finally is similar to $\myvec{0&0\\1&0}$ as before.\\
 &\\
\hline
&\\
Conclusion &Thus either $\vec{N}$ = 0 or $\vec{N}$ is similar over $\mathbb{C}$ to $\myvec{0&0\\1&0}$.\\
&\\
\hline
\caption{Solution summary}
\label{eq:solutions/6/2/11/table:1}
\end{longtable}

\end{enumerate}

\subsection{The Transpose of a Linear Transformation}
\renewcommand{\theequation}{\theenumi}
\renewcommand{\thefigure}{\theenumi}
\begin{enumerate}[label=\thesubsection.\arabic*.,ref=\thesubsection.\theenumi]
\numberwithin{equation}{enumi}
\numberwithin{figure}{enumi}
\numberwithin{table}{enumi}

\item Let $\mathbb{F}$ be a field and let $f$ be the linear functional on $\mathbb{F}^2$ defined by,
\begin{align}
f(x_1,x_2) = ax_1+bx_2
\end{align}
\begin{enumerate}
\item For the linear operator $T(x_1,x_2) = (x_1,0)$\\
Let, $g = T^ty$ and find $g(x_1,x_2)$
%
\\
\solution
See Table \ref{eq:solutions/6/2/11/table:1}
\onecolumn
\begin{longtable}{|l|l|}
\hline
\multirow{3}{*}{} & \\
Statement&Solution\\
\hline
&\parbox{6cm}{\begin{align}
\mbox{Let }\vec{N}&=\myvec{a&b\\c&d}\\
\mbox{Since }\vec{N}^2&=0
\end{align}}\\
&If $\myvec{a\\c},\myvec{b\\d}$ are linearly independent then $\vec{N}$ is diagonalizable to $\myvec{0&0\\0&0}$.\\
&\parbox{6cm}{\begin{align}
\mbox{If }\vec{P}\vec{N}\vec{P}^{-1}=0\\
\mbox{then }\vec{N}=\vec{P}^{-1}\vec{0}\vec{P}=0
\end{align}}\\
Proof that&So in this case $\vec{N}$ itself is the zero matrix.\\
$\vec{N}=0$&This contradicts the assumption that $\myvec{a\\c},\myvec{b\\d}$ are linearly independent.\\
&$\therefore$ we can assume that $\myvec{a\\c},\myvec{b\\d}$ are linearly dependent if both are\\
&equal to the zero vector\\
&\parbox{6cm}
{\begin{align}
   \mbox{then } \vec{N} &= 0.
\end{align}}\\
%\hline
%\pagebreak
\hline
&\\
&Therefore we can assume at least one vector is non-zero.\\
Assuming $\myvec{b\\d}$ as&Therefore
$\vec{N}=\myvec{a&0\\c&0}$\\
the zero vector&\\
&\parbox{6cm}{\begin{align}
\mbox{So }\vec{N}^2&=0\\
\implies a^2&=0\\
\therefore a&=0\\
\mbox{Thus }\vec{N}&=\myvec{a&0\\c&0}
\end{align}}\\
&In this case $\vec{N}$ is similar to $\vec{N}=\myvec{0&0\\1&0}$ via the matrix $\vec{P}=\myvec{c&0\\0&1}$\\
&\\
\hline
&\\
Assuming $\myvec{a\\c}$ as&Therefore
$\vec{N}=\myvec{0&b\\0&d}$\\
the zero vector&\\
&\parbox{6cm}{\begin{align}
\mbox{Then }\vec{N}^2&=0\\
\implies d^2&=0\\
\therefore d&=0\\
\mbox{Thus }\vec{N}&=\myvec{0&b\\0&0}
\end{align}}\\
&In this case $\vec{N}$ is similar to $\vec{N}=\myvec{0&0\\b&0}$ via the matrix $\vec{P}=\myvec{0&1\\1&0}$,\\
&which is similar to $\myvec{0&0\\1&0}$ as above.\\
&\\
\hline
&\\
Hence& we can assume neither $\myvec{a\\c}$ or $\myvec{b\\d}$ is the zero vector.\\
&\\
%\hline
%\pagebreak
\hline
&\\
Consequences of &Since they are linearly dependent we can assume,\\
linear&\\
independence&\\
&\parbox{6cm}{\begin{align}
\myvec{b\\d}&=x\myvec{a\\c}\\
\therefore \vec{N}&=\myvec{a&ax\\c&cx}\\
\therefore \vec{N}^2&=0\\
\implies a(a+cx)&=0\\
c(a+cx)&=0\\
ax(a+cx)&=0\\
cx(a+cx)&=0
\end{align}}\\
\hline
&\\
Proof that $\vec{N}$ is&We know that at least one of a or c is not zero.\\
similar over $\mathbb{C}$ to&If a = 0 then c $\neq$ 0, it must be that x = 0.\\
$\myvec{0&0\\1&0}$&So in this case $\vec{N}=\myvec{0&0\\c&0}$ which is similar to $\myvec{0&0\\1&0}$ as before.\\
&\\
&\parbox{6cm}{\begin{align}
\mbox{If } a &\neq0\\
\mbox{then }x &\neq0\\
\mbox{else }a(a+cx)&=0\\
\implies a&=0\\
\mbox{Thus }a+cx&=0\\
\mbox{Hence }\vec{N}&=\myvec{a&ax\\ \frac{-a}{x}&-a}
\end{align}}\\
 &This is similar to $\myvec{a&a\\-a&-a}$ via $\vec{P}=\myvec{\sqrt{x}&0\\0&\frac{1}{\sqrt{x}}}$.\\
 &And $\myvec{a&a\\-a&-a}$ is similar to $\myvec{0&0\\-a&0}$ via $\vec{P}=\myvec{-1&-1\\1&0}$\\
 &And this finally is similar to $\myvec{0&0\\1&0}$ as before.\\
 &\\
\hline
&\\
Conclusion &Thus either $\vec{N}$ = 0 or $\vec{N}$ is similar over $\mathbb{C}$ to $\myvec{0&0\\1&0}$.\\
&\\
\hline
\caption{Solution summary}
\label{eq:solutions/6/2/11/table:1}
\end{longtable}

\item For the linear operator $T(x_1,x_2) = (x_1-x_2,x_1+x_2)$\\
Let, $g = T^ty$ and find $g(x_1,x_2)$
\\
\solution
See Table \ref{eq:solutions/6/2/11/table:1}
\onecolumn
\begin{longtable}{|l|l|}
\hline
\multirow{3}{*}{} & \\
Statement&Solution\\
\hline
&\parbox{6cm}{\begin{align}
\mbox{Let }\vec{N}&=\myvec{a&b\\c&d}\\
\mbox{Since }\vec{N}^2&=0
\end{align}}\\
&If $\myvec{a\\c},\myvec{b\\d}$ are linearly independent then $\vec{N}$ is diagonalizable to $\myvec{0&0\\0&0}$.\\
&\parbox{6cm}{\begin{align}
\mbox{If }\vec{P}\vec{N}\vec{P}^{-1}=0\\
\mbox{then }\vec{N}=\vec{P}^{-1}\vec{0}\vec{P}=0
\end{align}}\\
Proof that&So in this case $\vec{N}$ itself is the zero matrix.\\
$\vec{N}=0$&This contradicts the assumption that $\myvec{a\\c},\myvec{b\\d}$ are linearly independent.\\
&$\therefore$ we can assume that $\myvec{a\\c},\myvec{b\\d}$ are linearly dependent if both are\\
&equal to the zero vector\\
&\parbox{6cm}
{\begin{align}
   \mbox{then } \vec{N} &= 0.
\end{align}}\\
%\hline
%\pagebreak
\hline
&\\
&Therefore we can assume at least one vector is non-zero.\\
Assuming $\myvec{b\\d}$ as&Therefore
$\vec{N}=\myvec{a&0\\c&0}$\\
the zero vector&\\
&\parbox{6cm}{\begin{align}
\mbox{So }\vec{N}^2&=0\\
\implies a^2&=0\\
\therefore a&=0\\
\mbox{Thus }\vec{N}&=\myvec{a&0\\c&0}
\end{align}}\\
&In this case $\vec{N}$ is similar to $\vec{N}=\myvec{0&0\\1&0}$ via the matrix $\vec{P}=\myvec{c&0\\0&1}$\\
&\\
\hline
&\\
Assuming $\myvec{a\\c}$ as&Therefore
$\vec{N}=\myvec{0&b\\0&d}$\\
the zero vector&\\
&\parbox{6cm}{\begin{align}
\mbox{Then }\vec{N}^2&=0\\
\implies d^2&=0\\
\therefore d&=0\\
\mbox{Thus }\vec{N}&=\myvec{0&b\\0&0}
\end{align}}\\
&In this case $\vec{N}$ is similar to $\vec{N}=\myvec{0&0\\b&0}$ via the matrix $\vec{P}=\myvec{0&1\\1&0}$,\\
&which is similar to $\myvec{0&0\\1&0}$ as above.\\
&\\
\hline
&\\
Hence& we can assume neither $\myvec{a\\c}$ or $\myvec{b\\d}$ is the zero vector.\\
&\\
%\hline
%\pagebreak
\hline
&\\
Consequences of &Since they are linearly dependent we can assume,\\
linear&\\
independence&\\
&\parbox{6cm}{\begin{align}
\myvec{b\\d}&=x\myvec{a\\c}\\
\therefore \vec{N}&=\myvec{a&ax\\c&cx}\\
\therefore \vec{N}^2&=0\\
\implies a(a+cx)&=0\\
c(a+cx)&=0\\
ax(a+cx)&=0\\
cx(a+cx)&=0
\end{align}}\\
\hline
&\\
Proof that $\vec{N}$ is&We know that at least one of a or c is not zero.\\
similar over $\mathbb{C}$ to&If a = 0 then c $\neq$ 0, it must be that x = 0.\\
$\myvec{0&0\\1&0}$&So in this case $\vec{N}=\myvec{0&0\\c&0}$ which is similar to $\myvec{0&0\\1&0}$ as before.\\
&\\
&\parbox{6cm}{\begin{align}
\mbox{If } a &\neq0\\
\mbox{then }x &\neq0\\
\mbox{else }a(a+cx)&=0\\
\implies a&=0\\
\mbox{Thus }a+cx&=0\\
\mbox{Hence }\vec{N}&=\myvec{a&ax\\ \frac{-a}{x}&-a}
\end{align}}\\
 &This is similar to $\myvec{a&a\\-a&-a}$ via $\vec{P}=\myvec{\sqrt{x}&0\\0&\frac{1}{\sqrt{x}}}$.\\
 &And $\myvec{a&a\\-a&-a}$ is similar to $\myvec{0&0\\-a&0}$ via $\vec{P}=\myvec{-1&-1\\1&0}$\\
 &And this finally is similar to $\myvec{0&0\\1&0}$ as before.\\
 &\\
\hline
&\\
Conclusion &Thus either $\vec{N}$ = 0 or $\vec{N}$ is similar over $\mathbb{C}$ to $\myvec{0&0\\1&0}$.\\
&\\
\hline
\caption{Solution summary}
\label{eq:solutions/6/2/11/table:1}
\end{longtable}

\end{enumerate}
%
\item Let V be the vector space of all polynomial function over the field of real numbers. Let a and b be fixed real numbers and let f be the linear functional on V defined by  
\begin{align}
    f\brak{p} = \int_{a}^{b} p\brak{x} \,dx
\end{align}
If D is the differentiation operator on V , what is $D^{t}f$ ?
%
\\
\solution
See Table \ref{eq:solutions/6/2/11/table:1}
\onecolumn
\begin{longtable}{|l|l|}
\hline
\multirow{3}{*}{} & \\
Statement&Solution\\
\hline
&\parbox{6cm}{\begin{align}
\mbox{Let }\vec{N}&=\myvec{a&b\\c&d}\\
\mbox{Since }\vec{N}^2&=0
\end{align}}\\
&If $\myvec{a\\c},\myvec{b\\d}$ are linearly independent then $\vec{N}$ is diagonalizable to $\myvec{0&0\\0&0}$.\\
&\parbox{6cm}{\begin{align}
\mbox{If }\vec{P}\vec{N}\vec{P}^{-1}=0\\
\mbox{then }\vec{N}=\vec{P}^{-1}\vec{0}\vec{P}=0
\end{align}}\\
Proof that&So in this case $\vec{N}$ itself is the zero matrix.\\
$\vec{N}=0$&This contradicts the assumption that $\myvec{a\\c},\myvec{b\\d}$ are linearly independent.\\
&$\therefore$ we can assume that $\myvec{a\\c},\myvec{b\\d}$ are linearly dependent if both are\\
&equal to the zero vector\\
&\parbox{6cm}
{\begin{align}
   \mbox{then } \vec{N} &= 0.
\end{align}}\\
%\hline
%\pagebreak
\hline
&\\
&Therefore we can assume at least one vector is non-zero.\\
Assuming $\myvec{b\\d}$ as&Therefore
$\vec{N}=\myvec{a&0\\c&0}$\\
the zero vector&\\
&\parbox{6cm}{\begin{align}
\mbox{So }\vec{N}^2&=0\\
\implies a^2&=0\\
\therefore a&=0\\
\mbox{Thus }\vec{N}&=\myvec{a&0\\c&0}
\end{align}}\\
&In this case $\vec{N}$ is similar to $\vec{N}=\myvec{0&0\\1&0}$ via the matrix $\vec{P}=\myvec{c&0\\0&1}$\\
&\\
\hline
&\\
Assuming $\myvec{a\\c}$ as&Therefore
$\vec{N}=\myvec{0&b\\0&d}$\\
the zero vector&\\
&\parbox{6cm}{\begin{align}
\mbox{Then }\vec{N}^2&=0\\
\implies d^2&=0\\
\therefore d&=0\\
\mbox{Thus }\vec{N}&=\myvec{0&b\\0&0}
\end{align}}\\
&In this case $\vec{N}$ is similar to $\vec{N}=\myvec{0&0\\b&0}$ via the matrix $\vec{P}=\myvec{0&1\\1&0}$,\\
&which is similar to $\myvec{0&0\\1&0}$ as above.\\
&\\
\hline
&\\
Hence& we can assume neither $\myvec{a\\c}$ or $\myvec{b\\d}$ is the zero vector.\\
&\\
%\hline
%\pagebreak
\hline
&\\
Consequences of &Since they are linearly dependent we can assume,\\
linear&\\
independence&\\
&\parbox{6cm}{\begin{align}
\myvec{b\\d}&=x\myvec{a\\c}\\
\therefore \vec{N}&=\myvec{a&ax\\c&cx}\\
\therefore \vec{N}^2&=0\\
\implies a(a+cx)&=0\\
c(a+cx)&=0\\
ax(a+cx)&=0\\
cx(a+cx)&=0
\end{align}}\\
\hline
&\\
Proof that $\vec{N}$ is&We know that at least one of a or c is not zero.\\
similar over $\mathbb{C}$ to&If a = 0 then c $\neq$ 0, it must be that x = 0.\\
$\myvec{0&0\\1&0}$&So in this case $\vec{N}=\myvec{0&0\\c&0}$ which is similar to $\myvec{0&0\\1&0}$ as before.\\
&\\
&\parbox{6cm}{\begin{align}
\mbox{If } a &\neq0\\
\mbox{then }x &\neq0\\
\mbox{else }a(a+cx)&=0\\
\implies a&=0\\
\mbox{Thus }a+cx&=0\\
\mbox{Hence }\vec{N}&=\myvec{a&ax\\ \frac{-a}{x}&-a}
\end{align}}\\
 &This is similar to $\myvec{a&a\\-a&-a}$ via $\vec{P}=\myvec{\sqrt{x}&0\\0&\frac{1}{\sqrt{x}}}$.\\
 &And $\myvec{a&a\\-a&-a}$ is similar to $\myvec{0&0\\-a&0}$ via $\vec{P}=\myvec{-1&-1\\1&0}$\\
 &And this finally is similar to $\myvec{0&0\\1&0}$ as before.\\
 &\\
\hline
&\\
Conclusion &Thus either $\vec{N}$ = 0 or $\vec{N}$ is similar over $\mathbb{C}$ to $\myvec{0&0\\1&0}$.\\
&\\
\hline
\caption{Solution summary}
\label{eq:solutions/6/2/11/table:1}
\end{longtable}

\item Let, $\vec{V}$ be the space of all $n \times n$ matrices over a field $\vec{F}$ and let B be a fixed $n \times n$ matrix. If T is the linear operator on $\vec{V}$ defined by $\vec{T(A)} = \vec{AB} - \vec{BA}$, and if $f$ is the trace function, what is $\vec{T}^tf$? 
%
\\
\solution
See Table \ref{eq:solutions/6/2/11/table:1}
\onecolumn
\begin{longtable}{|l|l|}
\hline
\multirow{3}{*}{} & \\
Statement&Solution\\
\hline
&\parbox{6cm}{\begin{align}
\mbox{Let }\vec{N}&=\myvec{a&b\\c&d}\\
\mbox{Since }\vec{N}^2&=0
\end{align}}\\
&If $\myvec{a\\c},\myvec{b\\d}$ are linearly independent then $\vec{N}$ is diagonalizable to $\myvec{0&0\\0&0}$.\\
&\parbox{6cm}{\begin{align}
\mbox{If }\vec{P}\vec{N}\vec{P}^{-1}=0\\
\mbox{then }\vec{N}=\vec{P}^{-1}\vec{0}\vec{P}=0
\end{align}}\\
Proof that&So in this case $\vec{N}$ itself is the zero matrix.\\
$\vec{N}=0$&This contradicts the assumption that $\myvec{a\\c},\myvec{b\\d}$ are linearly independent.\\
&$\therefore$ we can assume that $\myvec{a\\c},\myvec{b\\d}$ are linearly dependent if both are\\
&equal to the zero vector\\
&\parbox{6cm}
{\begin{align}
   \mbox{then } \vec{N} &= 0.
\end{align}}\\
%\hline
%\pagebreak
\hline
&\\
&Therefore we can assume at least one vector is non-zero.\\
Assuming $\myvec{b\\d}$ as&Therefore
$\vec{N}=\myvec{a&0\\c&0}$\\
the zero vector&\\
&\parbox{6cm}{\begin{align}
\mbox{So }\vec{N}^2&=0\\
\implies a^2&=0\\
\therefore a&=0\\
\mbox{Thus }\vec{N}&=\myvec{a&0\\c&0}
\end{align}}\\
&In this case $\vec{N}$ is similar to $\vec{N}=\myvec{0&0\\1&0}$ via the matrix $\vec{P}=\myvec{c&0\\0&1}$\\
&\\
\hline
&\\
Assuming $\myvec{a\\c}$ as&Therefore
$\vec{N}=\myvec{0&b\\0&d}$\\
the zero vector&\\
&\parbox{6cm}{\begin{align}
\mbox{Then }\vec{N}^2&=0\\
\implies d^2&=0\\
\therefore d&=0\\
\mbox{Thus }\vec{N}&=\myvec{0&b\\0&0}
\end{align}}\\
&In this case $\vec{N}$ is similar to $\vec{N}=\myvec{0&0\\b&0}$ via the matrix $\vec{P}=\myvec{0&1\\1&0}$,\\
&which is similar to $\myvec{0&0\\1&0}$ as above.\\
&\\
\hline
&\\
Hence& we can assume neither $\myvec{a\\c}$ or $\myvec{b\\d}$ is the zero vector.\\
&\\
%\hline
%\pagebreak
\hline
&\\
Consequences of &Since they are linearly dependent we can assume,\\
linear&\\
independence&\\
&\parbox{6cm}{\begin{align}
\myvec{b\\d}&=x\myvec{a\\c}\\
\therefore \vec{N}&=\myvec{a&ax\\c&cx}\\
\therefore \vec{N}^2&=0\\
\implies a(a+cx)&=0\\
c(a+cx)&=0\\
ax(a+cx)&=0\\
cx(a+cx)&=0
\end{align}}\\
\hline
&\\
Proof that $\vec{N}$ is&We know that at least one of a or c is not zero.\\
similar over $\mathbb{C}$ to&If a = 0 then c $\neq$ 0, it must be that x = 0.\\
$\myvec{0&0\\1&0}$&So in this case $\vec{N}=\myvec{0&0\\c&0}$ which is similar to $\myvec{0&0\\1&0}$ as before.\\
&\\
&\parbox{6cm}{\begin{align}
\mbox{If } a &\neq0\\
\mbox{then }x &\neq0\\
\mbox{else }a(a+cx)&=0\\
\implies a&=0\\
\mbox{Thus }a+cx&=0\\
\mbox{Hence }\vec{N}&=\myvec{a&ax\\ \frac{-a}{x}&-a}
\end{align}}\\
 &This is similar to $\myvec{a&a\\-a&-a}$ via $\vec{P}=\myvec{\sqrt{x}&0\\0&\frac{1}{\sqrt{x}}}$.\\
 &And $\myvec{a&a\\-a&-a}$ is similar to $\myvec{0&0\\-a&0}$ via $\vec{P}=\myvec{-1&-1\\1&0}$\\
 &And this finally is similar to $\myvec{0&0\\1&0}$ as before.\\
 &\\
\hline
&\\
Conclusion &Thus either $\vec{N}$ = 0 or $\vec{N}$ is similar over $\mathbb{C}$ to $\myvec{0&0\\1&0}$.\\
&\\
\hline
\caption{Solution summary}
\label{eq:solutions/6/2/11/table:1}
\end{longtable}

\end{enumerate}

\section{Polynomials}
\subsection{The Algebra of Polynomials}
\renewcommand{\theequation}{\theenumi}
\renewcommand{\thefigure}{\theenumi}
\begin{enumerate}[label=\thesubsection.\arabic*.,ref=\thesubsection.\theenumi]
\numberwithin{equation}{enumi}
\numberwithin{figure}{enumi}

\item Let F be a sub field of the complex numbers , and let A be the following 2 $\times$ 2 matrix over F.
\begin{align}
  \vec{A}=\myvec{2&1\\-1&3}  
\end{align}
Compute f\brak{\vec{A}} for the polynomial
\begin{enumerate}
\item 
\begin{align}
    f\brak{x}=x^{2}-x+2
\end{align}
%
\\
\solution
See Table \ref{eq:solutions/6/2/11/table:1}
\onecolumn
\begin{longtable}{|l|l|}
\hline
\multirow{3}{*}{} & \\
Statement&Solution\\
\hline
&\parbox{6cm}{\begin{align}
\mbox{Let }\vec{N}&=\myvec{a&b\\c&d}\\
\mbox{Since }\vec{N}^2&=0
\end{align}}\\
&If $\myvec{a\\c},\myvec{b\\d}$ are linearly independent then $\vec{N}$ is diagonalizable to $\myvec{0&0\\0&0}$.\\
&\parbox{6cm}{\begin{align}
\mbox{If }\vec{P}\vec{N}\vec{P}^{-1}=0\\
\mbox{then }\vec{N}=\vec{P}^{-1}\vec{0}\vec{P}=0
\end{align}}\\
Proof that&So in this case $\vec{N}$ itself is the zero matrix.\\
$\vec{N}=0$&This contradicts the assumption that $\myvec{a\\c},\myvec{b\\d}$ are linearly independent.\\
&$\therefore$ we can assume that $\myvec{a\\c},\myvec{b\\d}$ are linearly dependent if both are\\
&equal to the zero vector\\
&\parbox{6cm}
{\begin{align}
   \mbox{then } \vec{N} &= 0.
\end{align}}\\
%\hline
%\pagebreak
\hline
&\\
&Therefore we can assume at least one vector is non-zero.\\
Assuming $\myvec{b\\d}$ as&Therefore
$\vec{N}=\myvec{a&0\\c&0}$\\
the zero vector&\\
&\parbox{6cm}{\begin{align}
\mbox{So }\vec{N}^2&=0\\
\implies a^2&=0\\
\therefore a&=0\\
\mbox{Thus }\vec{N}&=\myvec{a&0\\c&0}
\end{align}}\\
&In this case $\vec{N}$ is similar to $\vec{N}=\myvec{0&0\\1&0}$ via the matrix $\vec{P}=\myvec{c&0\\0&1}$\\
&\\
\hline
&\\
Assuming $\myvec{a\\c}$ as&Therefore
$\vec{N}=\myvec{0&b\\0&d}$\\
the zero vector&\\
&\parbox{6cm}{\begin{align}
\mbox{Then }\vec{N}^2&=0\\
\implies d^2&=0\\
\therefore d&=0\\
\mbox{Thus }\vec{N}&=\myvec{0&b\\0&0}
\end{align}}\\
&In this case $\vec{N}$ is similar to $\vec{N}=\myvec{0&0\\b&0}$ via the matrix $\vec{P}=\myvec{0&1\\1&0}$,\\
&which is similar to $\myvec{0&0\\1&0}$ as above.\\
&\\
\hline
&\\
Hence& we can assume neither $\myvec{a\\c}$ or $\myvec{b\\d}$ is the zero vector.\\
&\\
%\hline
%\pagebreak
\hline
&\\
Consequences of &Since they are linearly dependent we can assume,\\
linear&\\
independence&\\
&\parbox{6cm}{\begin{align}
\myvec{b\\d}&=x\myvec{a\\c}\\
\therefore \vec{N}&=\myvec{a&ax\\c&cx}\\
\therefore \vec{N}^2&=0\\
\implies a(a+cx)&=0\\
c(a+cx)&=0\\
ax(a+cx)&=0\\
cx(a+cx)&=0
\end{align}}\\
\hline
&\\
Proof that $\vec{N}$ is&We know that at least one of a or c is not zero.\\
similar over $\mathbb{C}$ to&If a = 0 then c $\neq$ 0, it must be that x = 0.\\
$\myvec{0&0\\1&0}$&So in this case $\vec{N}=\myvec{0&0\\c&0}$ which is similar to $\myvec{0&0\\1&0}$ as before.\\
&\\
&\parbox{6cm}{\begin{align}
\mbox{If } a &\neq0\\
\mbox{then }x &\neq0\\
\mbox{else }a(a+cx)&=0\\
\implies a&=0\\
\mbox{Thus }a+cx&=0\\
\mbox{Hence }\vec{N}&=\myvec{a&ax\\ \frac{-a}{x}&-a}
\end{align}}\\
 &This is similar to $\myvec{a&a\\-a&-a}$ via $\vec{P}=\myvec{\sqrt{x}&0\\0&\frac{1}{\sqrt{x}}}$.\\
 &And $\myvec{a&a\\-a&-a}$ is similar to $\myvec{0&0\\-a&0}$ via $\vec{P}=\myvec{-1&-1\\1&0}$\\
 &And this finally is similar to $\myvec{0&0\\1&0}$ as before.\\
 &\\
\hline
&\\
Conclusion &Thus either $\vec{N}$ = 0 or $\vec{N}$ is similar over $\mathbb{C}$ to $\myvec{0&0\\1&0}$.\\
&\\
\hline
\caption{Solution summary}
\label{eq:solutions/6/2/11/table:1}
\end{longtable}

\item
\begin{align}
f = x^3 - 1
\end{align}
\solution
See Table \ref{eq:solutions/6/2/11/table:1}
\onecolumn
\begin{longtable}{|l|l|}
\hline
\multirow{3}{*}{} & \\
Statement&Solution\\
\hline
&\parbox{6cm}{\begin{align}
\mbox{Let }\vec{N}&=\myvec{a&b\\c&d}\\
\mbox{Since }\vec{N}^2&=0
\end{align}}\\
&If $\myvec{a\\c},\myvec{b\\d}$ are linearly independent then $\vec{N}$ is diagonalizable to $\myvec{0&0\\0&0}$.\\
&\parbox{6cm}{\begin{align}
\mbox{If }\vec{P}\vec{N}\vec{P}^{-1}=0\\
\mbox{then }\vec{N}=\vec{P}^{-1}\vec{0}\vec{P}=0
\end{align}}\\
Proof that&So in this case $\vec{N}$ itself is the zero matrix.\\
$\vec{N}=0$&This contradicts the assumption that $\myvec{a\\c},\myvec{b\\d}$ are linearly independent.\\
&$\therefore$ we can assume that $\myvec{a\\c},\myvec{b\\d}$ are linearly dependent if both are\\
&equal to the zero vector\\
&\parbox{6cm}
{\begin{align}
   \mbox{then } \vec{N} &= 0.
\end{align}}\\
%\hline
%\pagebreak
\hline
&\\
&Therefore we can assume at least one vector is non-zero.\\
Assuming $\myvec{b\\d}$ as&Therefore
$\vec{N}=\myvec{a&0\\c&0}$\\
the zero vector&\\
&\parbox{6cm}{\begin{align}
\mbox{So }\vec{N}^2&=0\\
\implies a^2&=0\\
\therefore a&=0\\
\mbox{Thus }\vec{N}&=\myvec{a&0\\c&0}
\end{align}}\\
&In this case $\vec{N}$ is similar to $\vec{N}=\myvec{0&0\\1&0}$ via the matrix $\vec{P}=\myvec{c&0\\0&1}$\\
&\\
\hline
&\\
Assuming $\myvec{a\\c}$ as&Therefore
$\vec{N}=\myvec{0&b\\0&d}$\\
the zero vector&\\
&\parbox{6cm}{\begin{align}
\mbox{Then }\vec{N}^2&=0\\
\implies d^2&=0\\
\therefore d&=0\\
\mbox{Thus }\vec{N}&=\myvec{0&b\\0&0}
\end{align}}\\
&In this case $\vec{N}$ is similar to $\vec{N}=\myvec{0&0\\b&0}$ via the matrix $\vec{P}=\myvec{0&1\\1&0}$,\\
&which is similar to $\myvec{0&0\\1&0}$ as above.\\
&\\
\hline
&\\
Hence& we can assume neither $\myvec{a\\c}$ or $\myvec{b\\d}$ is the zero vector.\\
&\\
%\hline
%\pagebreak
\hline
&\\
Consequences of &Since they are linearly dependent we can assume,\\
linear&\\
independence&\\
&\parbox{6cm}{\begin{align}
\myvec{b\\d}&=x\myvec{a\\c}\\
\therefore \vec{N}&=\myvec{a&ax\\c&cx}\\
\therefore \vec{N}^2&=0\\
\implies a(a+cx)&=0\\
c(a+cx)&=0\\
ax(a+cx)&=0\\
cx(a+cx)&=0
\end{align}}\\
\hline
&\\
Proof that $\vec{N}$ is&We know that at least one of a or c is not zero.\\
similar over $\mathbb{C}$ to&If a = 0 then c $\neq$ 0, it must be that x = 0.\\
$\myvec{0&0\\1&0}$&So in this case $\vec{N}=\myvec{0&0\\c&0}$ which is similar to $\myvec{0&0\\1&0}$ as before.\\
&\\
&\parbox{6cm}{\begin{align}
\mbox{If } a &\neq0\\
\mbox{then }x &\neq0\\
\mbox{else }a(a+cx)&=0\\
\implies a&=0\\
\mbox{Thus }a+cx&=0\\
\mbox{Hence }\vec{N}&=\myvec{a&ax\\ \frac{-a}{x}&-a}
\end{align}}\\
 &This is similar to $\myvec{a&a\\-a&-a}$ via $\vec{P}=\myvec{\sqrt{x}&0\\0&\frac{1}{\sqrt{x}}}$.\\
 &And $\myvec{a&a\\-a&-a}$ is similar to $\myvec{0&0\\-a&0}$ via $\vec{P}=\myvec{-1&-1\\1&0}$\\
 &And this finally is similar to $\myvec{0&0\\1&0}$ as before.\\
 &\\
\hline
&\\
Conclusion &Thus either $\vec{N}$ = 0 or $\vec{N}$ is similar over $\mathbb{C}$ to $\myvec{0&0\\1&0}$.\\
&\\
\hline
\caption{Solution summary}
\label{eq:solutions/6/2/11/table:1}
\end{longtable}

%
\item 
$f\brak{x}=x^{2}-5x+7$
%
\\
\solution
See Table \ref{eq:solutions/6/2/11/table:1}
\onecolumn
\begin{longtable}{|l|l|}
\hline
\multirow{3}{*}{} & \\
Statement&Solution\\
\hline
&\parbox{6cm}{\begin{align}
\mbox{Let }\vec{N}&=\myvec{a&b\\c&d}\\
\mbox{Since }\vec{N}^2&=0
\end{align}}\\
&If $\myvec{a\\c},\myvec{b\\d}$ are linearly independent then $\vec{N}$ is diagonalizable to $\myvec{0&0\\0&0}$.\\
&\parbox{6cm}{\begin{align}
\mbox{If }\vec{P}\vec{N}\vec{P}^{-1}=0\\
\mbox{then }\vec{N}=\vec{P}^{-1}\vec{0}\vec{P}=0
\end{align}}\\
Proof that&So in this case $\vec{N}$ itself is the zero matrix.\\
$\vec{N}=0$&This contradicts the assumption that $\myvec{a\\c},\myvec{b\\d}$ are linearly independent.\\
&$\therefore$ we can assume that $\myvec{a\\c},\myvec{b\\d}$ are linearly dependent if both are\\
&equal to the zero vector\\
&\parbox{6cm}
{\begin{align}
   \mbox{then } \vec{N} &= 0.
\end{align}}\\
%\hline
%\pagebreak
\hline
&\\
&Therefore we can assume at least one vector is non-zero.\\
Assuming $\myvec{b\\d}$ as&Therefore
$\vec{N}=\myvec{a&0\\c&0}$\\
the zero vector&\\
&\parbox{6cm}{\begin{align}
\mbox{So }\vec{N}^2&=0\\
\implies a^2&=0\\
\therefore a&=0\\
\mbox{Thus }\vec{N}&=\myvec{a&0\\c&0}
\end{align}}\\
&In this case $\vec{N}$ is similar to $\vec{N}=\myvec{0&0\\1&0}$ via the matrix $\vec{P}=\myvec{c&0\\0&1}$\\
&\\
\hline
&\\
Assuming $\myvec{a\\c}$ as&Therefore
$\vec{N}=\myvec{0&b\\0&d}$\\
the zero vector&\\
&\parbox{6cm}{\begin{align}
\mbox{Then }\vec{N}^2&=0\\
\implies d^2&=0\\
\therefore d&=0\\
\mbox{Thus }\vec{N}&=\myvec{0&b\\0&0}
\end{align}}\\
&In this case $\vec{N}$ is similar to $\vec{N}=\myvec{0&0\\b&0}$ via the matrix $\vec{P}=\myvec{0&1\\1&0}$,\\
&which is similar to $\myvec{0&0\\1&0}$ as above.\\
&\\
\hline
&\\
Hence& we can assume neither $\myvec{a\\c}$ or $\myvec{b\\d}$ is the zero vector.\\
&\\
%\hline
%\pagebreak
\hline
&\\
Consequences of &Since they are linearly dependent we can assume,\\
linear&\\
independence&\\
&\parbox{6cm}{\begin{align}
\myvec{b\\d}&=x\myvec{a\\c}\\
\therefore \vec{N}&=\myvec{a&ax\\c&cx}\\
\therefore \vec{N}^2&=0\\
\implies a(a+cx)&=0\\
c(a+cx)&=0\\
ax(a+cx)&=0\\
cx(a+cx)&=0
\end{align}}\\
\hline
&\\
Proof that $\vec{N}$ is&We know that at least one of a or c is not zero.\\
similar over $\mathbb{C}$ to&If a = 0 then c $\neq$ 0, it must be that x = 0.\\
$\myvec{0&0\\1&0}$&So in this case $\vec{N}=\myvec{0&0\\c&0}$ which is similar to $\myvec{0&0\\1&0}$ as before.\\
&\\
&\parbox{6cm}{\begin{align}
\mbox{If } a &\neq0\\
\mbox{then }x &\neq0\\
\mbox{else }a(a+cx)&=0\\
\implies a&=0\\
\mbox{Thus }a+cx&=0\\
\mbox{Hence }\vec{N}&=\myvec{a&ax\\ \frac{-a}{x}&-a}
\end{align}}\\
 &This is similar to $\myvec{a&a\\-a&-a}$ via $\vec{P}=\myvec{\sqrt{x}&0\\0&\frac{1}{\sqrt{x}}}$.\\
 &And $\myvec{a&a\\-a&-a}$ is similar to $\myvec{0&0\\-a&0}$ via $\vec{P}=\myvec{-1&-1\\1&0}$\\
 &And this finally is similar to $\myvec{0&0\\1&0}$ as before.\\
 &\\
\hline
&\\
Conclusion &Thus either $\vec{N}$ = 0 or $\vec{N}$ is similar over $\mathbb{C}$ to $\myvec{0&0\\1&0}$.\\
&\\
\hline
\caption{Solution summary}
\label{eq:solutions/6/2/11/table:1}
\end{longtable}

\end{enumerate}
\item Let T be the linear operator on $\mathbb{R}^3$ defined by
\begin{align}
T(x_1,x_2,x_3)=(x_1,x_3,-2x_2-x_3)
\end{align}
Let f be the polynomial over $\mathbb{R}$ defined by f= $-x^3+2$
%
\\
\solution
See Table \ref{eq:solutions/6/2/11/table:1}
\onecolumn
\begin{longtable}{|l|l|}
\hline
\multirow{3}{*}{} & \\
Statement&Solution\\
\hline
&\parbox{6cm}{\begin{align}
\mbox{Let }\vec{N}&=\myvec{a&b\\c&d}\\
\mbox{Since }\vec{N}^2&=0
\end{align}}\\
&If $\myvec{a\\c},\myvec{b\\d}$ are linearly independent then $\vec{N}$ is diagonalizable to $\myvec{0&0\\0&0}$.\\
&\parbox{6cm}{\begin{align}
\mbox{If }\vec{P}\vec{N}\vec{P}^{-1}=0\\
\mbox{then }\vec{N}=\vec{P}^{-1}\vec{0}\vec{P}=0
\end{align}}\\
Proof that&So in this case $\vec{N}$ itself is the zero matrix.\\
$\vec{N}=0$&This contradicts the assumption that $\myvec{a\\c},\myvec{b\\d}$ are linearly independent.\\
&$\therefore$ we can assume that $\myvec{a\\c},\myvec{b\\d}$ are linearly dependent if both are\\
&equal to the zero vector\\
&\parbox{6cm}
{\begin{align}
   \mbox{then } \vec{N} &= 0.
\end{align}}\\
%\hline
%\pagebreak
\hline
&\\
&Therefore we can assume at least one vector is non-zero.\\
Assuming $\myvec{b\\d}$ as&Therefore
$\vec{N}=\myvec{a&0\\c&0}$\\
the zero vector&\\
&\parbox{6cm}{\begin{align}
\mbox{So }\vec{N}^2&=0\\
\implies a^2&=0\\
\therefore a&=0\\
\mbox{Thus }\vec{N}&=\myvec{a&0\\c&0}
\end{align}}\\
&In this case $\vec{N}$ is similar to $\vec{N}=\myvec{0&0\\1&0}$ via the matrix $\vec{P}=\myvec{c&0\\0&1}$\\
&\\
\hline
&\\
Assuming $\myvec{a\\c}$ as&Therefore
$\vec{N}=\myvec{0&b\\0&d}$\\
the zero vector&\\
&\parbox{6cm}{\begin{align}
\mbox{Then }\vec{N}^2&=0\\
\implies d^2&=0\\
\therefore d&=0\\
\mbox{Thus }\vec{N}&=\myvec{0&b\\0&0}
\end{align}}\\
&In this case $\vec{N}$ is similar to $\vec{N}=\myvec{0&0\\b&0}$ via the matrix $\vec{P}=\myvec{0&1\\1&0}$,\\
&which is similar to $\myvec{0&0\\1&0}$ as above.\\
&\\
\hline
&\\
Hence& we can assume neither $\myvec{a\\c}$ or $\myvec{b\\d}$ is the zero vector.\\
&\\
%\hline
%\pagebreak
\hline
&\\
Consequences of &Since they are linearly dependent we can assume,\\
linear&\\
independence&\\
&\parbox{6cm}{\begin{align}
\myvec{b\\d}&=x\myvec{a\\c}\\
\therefore \vec{N}&=\myvec{a&ax\\c&cx}\\
\therefore \vec{N}^2&=0\\
\implies a(a+cx)&=0\\
c(a+cx)&=0\\
ax(a+cx)&=0\\
cx(a+cx)&=0
\end{align}}\\
\hline
&\\
Proof that $\vec{N}$ is&We know that at least one of a or c is not zero.\\
similar over $\mathbb{C}$ to&If a = 0 then c $\neq$ 0, it must be that x = 0.\\
$\myvec{0&0\\1&0}$&So in this case $\vec{N}=\myvec{0&0\\c&0}$ which is similar to $\myvec{0&0\\1&0}$ as before.\\
&\\
&\parbox{6cm}{\begin{align}
\mbox{If } a &\neq0\\
\mbox{then }x &\neq0\\
\mbox{else }a(a+cx)&=0\\
\implies a&=0\\
\mbox{Thus }a+cx&=0\\
\mbox{Hence }\vec{N}&=\myvec{a&ax\\ \frac{-a}{x}&-a}
\end{align}}\\
 &This is similar to $\myvec{a&a\\-a&-a}$ via $\vec{P}=\myvec{\sqrt{x}&0\\0&\frac{1}{\sqrt{x}}}$.\\
 &And $\myvec{a&a\\-a&-a}$ is similar to $\myvec{0&0\\-a&0}$ via $\vec{P}=\myvec{-1&-1\\1&0}$\\
 &And this finally is similar to $\myvec{0&0\\1&0}$ as before.\\
 &\\
\hline
&\\
Conclusion &Thus either $\vec{N}$ = 0 or $\vec{N}$ is similar over $\mathbb{C}$ to $\myvec{0&0\\1&0}$.\\
&\\
\hline
\caption{Solution summary}
\label{eq:solutions/6/2/11/table:1}
\end{longtable}

\item Let $\vec{A}$ be an $n \times n$ diagonal matrix over the field $\vec{F}$, i.e., a matrix satisfying $\vec{A}_{ij} = 0$ for $i \neq j$. Let $f$ be the polynomial over $\vec{F}$ defined by $f = (x - A_{11}) . . . (x - A_{nn})$. What is the matrix $f(\vec{A})$? 
%
\\
\solution
See Table \ref{eq:solutions/6/2/11/table:1}
\onecolumn
\begin{longtable}{|l|l|}
\hline
\multirow{3}{*}{} & \\
Statement&Solution\\
\hline
&\parbox{6cm}{\begin{align}
\mbox{Let }\vec{N}&=\myvec{a&b\\c&d}\\
\mbox{Since }\vec{N}^2&=0
\end{align}}\\
&If $\myvec{a\\c},\myvec{b\\d}$ are linearly independent then $\vec{N}$ is diagonalizable to $\myvec{0&0\\0&0}$.\\
&\parbox{6cm}{\begin{align}
\mbox{If }\vec{P}\vec{N}\vec{P}^{-1}=0\\
\mbox{then }\vec{N}=\vec{P}^{-1}\vec{0}\vec{P}=0
\end{align}}\\
Proof that&So in this case $\vec{N}$ itself is the zero matrix.\\
$\vec{N}=0$&This contradicts the assumption that $\myvec{a\\c},\myvec{b\\d}$ are linearly independent.\\
&$\therefore$ we can assume that $\myvec{a\\c},\myvec{b\\d}$ are linearly dependent if both are\\
&equal to the zero vector\\
&\parbox{6cm}
{\begin{align}
   \mbox{then } \vec{N} &= 0.
\end{align}}\\
%\hline
%\pagebreak
\hline
&\\
&Therefore we can assume at least one vector is non-zero.\\
Assuming $\myvec{b\\d}$ as&Therefore
$\vec{N}=\myvec{a&0\\c&0}$\\
the zero vector&\\
&\parbox{6cm}{\begin{align}
\mbox{So }\vec{N}^2&=0\\
\implies a^2&=0\\
\therefore a&=0\\
\mbox{Thus }\vec{N}&=\myvec{a&0\\c&0}
\end{align}}\\
&In this case $\vec{N}$ is similar to $\vec{N}=\myvec{0&0\\1&0}$ via the matrix $\vec{P}=\myvec{c&0\\0&1}$\\
&\\
\hline
&\\
Assuming $\myvec{a\\c}$ as&Therefore
$\vec{N}=\myvec{0&b\\0&d}$\\
the zero vector&\\
&\parbox{6cm}{\begin{align}
\mbox{Then }\vec{N}^2&=0\\
\implies d^2&=0\\
\therefore d&=0\\
\mbox{Thus }\vec{N}&=\myvec{0&b\\0&0}
\end{align}}\\
&In this case $\vec{N}$ is similar to $\vec{N}=\myvec{0&0\\b&0}$ via the matrix $\vec{P}=\myvec{0&1\\1&0}$,\\
&which is similar to $\myvec{0&0\\1&0}$ as above.\\
&\\
\hline
&\\
Hence& we can assume neither $\myvec{a\\c}$ or $\myvec{b\\d}$ is the zero vector.\\
&\\
%\hline
%\pagebreak
\hline
&\\
Consequences of &Since they are linearly dependent we can assume,\\
linear&\\
independence&\\
&\parbox{6cm}{\begin{align}
\myvec{b\\d}&=x\myvec{a\\c}\\
\therefore \vec{N}&=\myvec{a&ax\\c&cx}\\
\therefore \vec{N}^2&=0\\
\implies a(a+cx)&=0\\
c(a+cx)&=0\\
ax(a+cx)&=0\\
cx(a+cx)&=0
\end{align}}\\
\hline
&\\
Proof that $\vec{N}$ is&We know that at least one of a or c is not zero.\\
similar over $\mathbb{C}$ to&If a = 0 then c $\neq$ 0, it must be that x = 0.\\
$\myvec{0&0\\1&0}$&So in this case $\vec{N}=\myvec{0&0\\c&0}$ which is similar to $\myvec{0&0\\1&0}$ as before.\\
&\\
&\parbox{6cm}{\begin{align}
\mbox{If } a &\neq0\\
\mbox{then }x &\neq0\\
\mbox{else }a(a+cx)&=0\\
\implies a&=0\\
\mbox{Thus }a+cx&=0\\
\mbox{Hence }\vec{N}&=\myvec{a&ax\\ \frac{-a}{x}&-a}
\end{align}}\\
 &This is similar to $\myvec{a&a\\-a&-a}$ via $\vec{P}=\myvec{\sqrt{x}&0\\0&\frac{1}{\sqrt{x}}}$.\\
 &And $\myvec{a&a\\-a&-a}$ is similar to $\myvec{0&0\\-a&0}$ via $\vec{P}=\myvec{-1&-1\\1&0}$\\
 &And this finally is similar to $\myvec{0&0\\1&0}$ as before.\\
 &\\
\hline
&\\
Conclusion &Thus either $\vec{N}$ = 0 or $\vec{N}$ is similar over $\mathbb{C}$ to $\myvec{0&0\\1&0}$.\\
&\\
\hline
\caption{Solution summary}
\label{eq:solutions/6/2/11/table:1}
\end{longtable}

\item If f  and g are independent polynomials over a field $\vec{F}$and h is a non-zero
polynomial over $\vec{F}$, show that fh and gh are independent.
%
\\
\solution
See Table \ref{eq:solutions/6/2/11/table:1}
\onecolumn
\begin{longtable}{|l|l|}
\hline
\multirow{3}{*}{} & \\
Statement&Solution\\
\hline
&\parbox{6cm}{\begin{align}
\mbox{Let }\vec{N}&=\myvec{a&b\\c&d}\\
\mbox{Since }\vec{N}^2&=0
\end{align}}\\
&If $\myvec{a\\c},\myvec{b\\d}$ are linearly independent then $\vec{N}$ is diagonalizable to $\myvec{0&0\\0&0}$.\\
&\parbox{6cm}{\begin{align}
\mbox{If }\vec{P}\vec{N}\vec{P}^{-1}=0\\
\mbox{then }\vec{N}=\vec{P}^{-1}\vec{0}\vec{P}=0
\end{align}}\\
Proof that&So in this case $\vec{N}$ itself is the zero matrix.\\
$\vec{N}=0$&This contradicts the assumption that $\myvec{a\\c},\myvec{b\\d}$ are linearly independent.\\
&$\therefore$ we can assume that $\myvec{a\\c},\myvec{b\\d}$ are linearly dependent if both are\\
&equal to the zero vector\\
&\parbox{6cm}
{\begin{align}
   \mbox{then } \vec{N} &= 0.
\end{align}}\\
%\hline
%\pagebreak
\hline
&\\
&Therefore we can assume at least one vector is non-zero.\\
Assuming $\myvec{b\\d}$ as&Therefore
$\vec{N}=\myvec{a&0\\c&0}$\\
the zero vector&\\
&\parbox{6cm}{\begin{align}
\mbox{So }\vec{N}^2&=0\\
\implies a^2&=0\\
\therefore a&=0\\
\mbox{Thus }\vec{N}&=\myvec{a&0\\c&0}
\end{align}}\\
&In this case $\vec{N}$ is similar to $\vec{N}=\myvec{0&0\\1&0}$ via the matrix $\vec{P}=\myvec{c&0\\0&1}$\\
&\\
\hline
&\\
Assuming $\myvec{a\\c}$ as&Therefore
$\vec{N}=\myvec{0&b\\0&d}$\\
the zero vector&\\
&\parbox{6cm}{\begin{align}
\mbox{Then }\vec{N}^2&=0\\
\implies d^2&=0\\
\therefore d&=0\\
\mbox{Thus }\vec{N}&=\myvec{0&b\\0&0}
\end{align}}\\
&In this case $\vec{N}$ is similar to $\vec{N}=\myvec{0&0\\b&0}$ via the matrix $\vec{P}=\myvec{0&1\\1&0}$,\\
&which is similar to $\myvec{0&0\\1&0}$ as above.\\
&\\
\hline
&\\
Hence& we can assume neither $\myvec{a\\c}$ or $\myvec{b\\d}$ is the zero vector.\\
&\\
%\hline
%\pagebreak
\hline
&\\
Consequences of &Since they are linearly dependent we can assume,\\
linear&\\
independence&\\
&\parbox{6cm}{\begin{align}
\myvec{b\\d}&=x\myvec{a\\c}\\
\therefore \vec{N}&=\myvec{a&ax\\c&cx}\\
\therefore \vec{N}^2&=0\\
\implies a(a+cx)&=0\\
c(a+cx)&=0\\
ax(a+cx)&=0\\
cx(a+cx)&=0
\end{align}}\\
\hline
&\\
Proof that $\vec{N}$ is&We know that at least one of a or c is not zero.\\
similar over $\mathbb{C}$ to&If a = 0 then c $\neq$ 0, it must be that x = 0.\\
$\myvec{0&0\\1&0}$&So in this case $\vec{N}=\myvec{0&0\\c&0}$ which is similar to $\myvec{0&0\\1&0}$ as before.\\
&\\
&\parbox{6cm}{\begin{align}
\mbox{If } a &\neq0\\
\mbox{then }x &\neq0\\
\mbox{else }a(a+cx)&=0\\
\implies a&=0\\
\mbox{Thus }a+cx&=0\\
\mbox{Hence }\vec{N}&=\myvec{a&ax\\ \frac{-a}{x}&-a}
\end{align}}\\
 &This is similar to $\myvec{a&a\\-a&-a}$ via $\vec{P}=\myvec{\sqrt{x}&0\\0&\frac{1}{\sqrt{x}}}$.\\
 &And $\myvec{a&a\\-a&-a}$ is similar to $\myvec{0&0\\-a&0}$ via $\vec{P}=\myvec{-1&-1\\1&0}$\\
 &And this finally is similar to $\myvec{0&0\\1&0}$ as before.\\
 &\\
\hline
&\\
Conclusion &Thus either $\vec{N}$ = 0 or $\vec{N}$ is similar over $\mathbb{C}$ to $\myvec{0&0\\1&0}$.\\
&\\
\hline
\caption{Solution summary}
\label{eq:solutions/6/2/11/table:1}
\end{longtable}

\end{enumerate}

\subsection{Lagrange Interpolation}
\renewcommand{\theequation}{\theenumi}
\renewcommand{\thefigure}{\theenumi}
\begin{enumerate}[label=\thesubsection.\arabic*.,ref=\thesubsection.\theenumi]
\numberwithin{equation}{enumi}
\numberwithin{figure}{enumi}

\item Use the Lagrange interpolation formula to find a polynomial f with real coefficients such that f has degree $\leq$ 3 and f\brak{-1}=-6 , f\brak{0}=2 , f\brak{1}=-2 , f\brak{2}=6.
%
\\
\solution
See Table \ref{eq:solutions/6/2/11/table:1}
\onecolumn
\begin{longtable}{|l|l|}
\hline
\multirow{3}{*}{} & \\
Statement&Solution\\
\hline
&\parbox{6cm}{\begin{align}
\mbox{Let }\vec{N}&=\myvec{a&b\\c&d}\\
\mbox{Since }\vec{N}^2&=0
\end{align}}\\
&If $\myvec{a\\c},\myvec{b\\d}$ are linearly independent then $\vec{N}$ is diagonalizable to $\myvec{0&0\\0&0}$.\\
&\parbox{6cm}{\begin{align}
\mbox{If }\vec{P}\vec{N}\vec{P}^{-1}=0\\
\mbox{then }\vec{N}=\vec{P}^{-1}\vec{0}\vec{P}=0
\end{align}}\\
Proof that&So in this case $\vec{N}$ itself is the zero matrix.\\
$\vec{N}=0$&This contradicts the assumption that $\myvec{a\\c},\myvec{b\\d}$ are linearly independent.\\
&$\therefore$ we can assume that $\myvec{a\\c},\myvec{b\\d}$ are linearly dependent if both are\\
&equal to the zero vector\\
&\parbox{6cm}
{\begin{align}
   \mbox{then } \vec{N} &= 0.
\end{align}}\\
%\hline
%\pagebreak
\hline
&\\
&Therefore we can assume at least one vector is non-zero.\\
Assuming $\myvec{b\\d}$ as&Therefore
$\vec{N}=\myvec{a&0\\c&0}$\\
the zero vector&\\
&\parbox{6cm}{\begin{align}
\mbox{So }\vec{N}^2&=0\\
\implies a^2&=0\\
\therefore a&=0\\
\mbox{Thus }\vec{N}&=\myvec{a&0\\c&0}
\end{align}}\\
&In this case $\vec{N}$ is similar to $\vec{N}=\myvec{0&0\\1&0}$ via the matrix $\vec{P}=\myvec{c&0\\0&1}$\\
&\\
\hline
&\\
Assuming $\myvec{a\\c}$ as&Therefore
$\vec{N}=\myvec{0&b\\0&d}$\\
the zero vector&\\
&\parbox{6cm}{\begin{align}
\mbox{Then }\vec{N}^2&=0\\
\implies d^2&=0\\
\therefore d&=0\\
\mbox{Thus }\vec{N}&=\myvec{0&b\\0&0}
\end{align}}\\
&In this case $\vec{N}$ is similar to $\vec{N}=\myvec{0&0\\b&0}$ via the matrix $\vec{P}=\myvec{0&1\\1&0}$,\\
&which is similar to $\myvec{0&0\\1&0}$ as above.\\
&\\
\hline
&\\
Hence& we can assume neither $\myvec{a\\c}$ or $\myvec{b\\d}$ is the zero vector.\\
&\\
%\hline
%\pagebreak
\hline
&\\
Consequences of &Since they are linearly dependent we can assume,\\
linear&\\
independence&\\
&\parbox{6cm}{\begin{align}
\myvec{b\\d}&=x\myvec{a\\c}\\
\therefore \vec{N}&=\myvec{a&ax\\c&cx}\\
\therefore \vec{N}^2&=0\\
\implies a(a+cx)&=0\\
c(a+cx)&=0\\
ax(a+cx)&=0\\
cx(a+cx)&=0
\end{align}}\\
\hline
&\\
Proof that $\vec{N}$ is&We know that at least one of a or c is not zero.\\
similar over $\mathbb{C}$ to&If a = 0 then c $\neq$ 0, it must be that x = 0.\\
$\myvec{0&0\\1&0}$&So in this case $\vec{N}=\myvec{0&0\\c&0}$ which is similar to $\myvec{0&0\\1&0}$ as before.\\
&\\
&\parbox{6cm}{\begin{align}
\mbox{If } a &\neq0\\
\mbox{then }x &\neq0\\
\mbox{else }a(a+cx)&=0\\
\implies a&=0\\
\mbox{Thus }a+cx&=0\\
\mbox{Hence }\vec{N}&=\myvec{a&ax\\ \frac{-a}{x}&-a}
\end{align}}\\
 &This is similar to $\myvec{a&a\\-a&-a}$ via $\vec{P}=\myvec{\sqrt{x}&0\\0&\frac{1}{\sqrt{x}}}$.\\
 &And $\myvec{a&a\\-a&-a}$ is similar to $\myvec{0&0\\-a&0}$ via $\vec{P}=\myvec{-1&-1\\1&0}$\\
 &And this finally is similar to $\myvec{0&0\\1&0}$ as before.\\
 &\\
\hline
&\\
Conclusion &Thus either $\vec{N}$ = 0 or $\vec{N}$ is similar over $\mathbb{C}$ to $\myvec{0&0\\1&0}$.\\
&\\
\hline
\caption{Solution summary}
\label{eq:solutions/6/2/11/table:1}
\end{longtable}

\item Let $T$ be the linear operator on $R^3$ defined by\\
$T(x_1,x_2,x_3)=(x_1,x_3,-2x_2-x_3)$\\
Let $f$ be the polynomial over $R$ defined by $f=-x^3+2$. Find $f(T)$.
%
\\
\solution
See Table \ref{eq:solutions/6/2/11/table:1}
\onecolumn
\begin{longtable}{|l|l|}
\hline
\multirow{3}{*}{} & \\
Statement&Solution\\
\hline
&\parbox{6cm}{\begin{align}
\mbox{Let }\vec{N}&=\myvec{a&b\\c&d}\\
\mbox{Since }\vec{N}^2&=0
\end{align}}\\
&If $\myvec{a\\c},\myvec{b\\d}$ are linearly independent then $\vec{N}$ is diagonalizable to $\myvec{0&0\\0&0}$.\\
&\parbox{6cm}{\begin{align}
\mbox{If }\vec{P}\vec{N}\vec{P}^{-1}=0\\
\mbox{then }\vec{N}=\vec{P}^{-1}\vec{0}\vec{P}=0
\end{align}}\\
Proof that&So in this case $\vec{N}$ itself is the zero matrix.\\
$\vec{N}=0$&This contradicts the assumption that $\myvec{a\\c},\myvec{b\\d}$ are linearly independent.\\
&$\therefore$ we can assume that $\myvec{a\\c},\myvec{b\\d}$ are linearly dependent if both are\\
&equal to the zero vector\\
&\parbox{6cm}
{\begin{align}
   \mbox{then } \vec{N} &= 0.
\end{align}}\\
%\hline
%\pagebreak
\hline
&\\
&Therefore we can assume at least one vector is non-zero.\\
Assuming $\myvec{b\\d}$ as&Therefore
$\vec{N}=\myvec{a&0\\c&0}$\\
the zero vector&\\
&\parbox{6cm}{\begin{align}
\mbox{So }\vec{N}^2&=0\\
\implies a^2&=0\\
\therefore a&=0\\
\mbox{Thus }\vec{N}&=\myvec{a&0\\c&0}
\end{align}}\\
&In this case $\vec{N}$ is similar to $\vec{N}=\myvec{0&0\\1&0}$ via the matrix $\vec{P}=\myvec{c&0\\0&1}$\\
&\\
\hline
&\\
Assuming $\myvec{a\\c}$ as&Therefore
$\vec{N}=\myvec{0&b\\0&d}$\\
the zero vector&\\
&\parbox{6cm}{\begin{align}
\mbox{Then }\vec{N}^2&=0\\
\implies d^2&=0\\
\therefore d&=0\\
\mbox{Thus }\vec{N}&=\myvec{0&b\\0&0}
\end{align}}\\
&In this case $\vec{N}$ is similar to $\vec{N}=\myvec{0&0\\b&0}$ via the matrix $\vec{P}=\myvec{0&1\\1&0}$,\\
&which is similar to $\myvec{0&0\\1&0}$ as above.\\
&\\
\hline
&\\
Hence& we can assume neither $\myvec{a\\c}$ or $\myvec{b\\d}$ is the zero vector.\\
&\\
%\hline
%\pagebreak
\hline
&\\
Consequences of &Since they are linearly dependent we can assume,\\
linear&\\
independence&\\
&\parbox{6cm}{\begin{align}
\myvec{b\\d}&=x\myvec{a\\c}\\
\therefore \vec{N}&=\myvec{a&ax\\c&cx}\\
\therefore \vec{N}^2&=0\\
\implies a(a+cx)&=0\\
c(a+cx)&=0\\
ax(a+cx)&=0\\
cx(a+cx)&=0
\end{align}}\\
\hline
&\\
Proof that $\vec{N}$ is&We know that at least one of a or c is not zero.\\
similar over $\mathbb{C}$ to&If a = 0 then c $\neq$ 0, it must be that x = 0.\\
$\myvec{0&0\\1&0}$&So in this case $\vec{N}=\myvec{0&0\\c&0}$ which is similar to $\myvec{0&0\\1&0}$ as before.\\
&\\
&\parbox{6cm}{\begin{align}
\mbox{If } a &\neq0\\
\mbox{then }x &\neq0\\
\mbox{else }a(a+cx)&=0\\
\implies a&=0\\
\mbox{Thus }a+cx&=0\\
\mbox{Hence }\vec{N}&=\myvec{a&ax\\ \frac{-a}{x}&-a}
\end{align}}\\
 &This is similar to $\myvec{a&a\\-a&-a}$ via $\vec{P}=\myvec{\sqrt{x}&0\\0&\frac{1}{\sqrt{x}}}$.\\
 &And $\myvec{a&a\\-a&-a}$ is similar to $\myvec{0&0\\-a&0}$ via $\vec{P}=\myvec{-1&-1\\1&0}$\\
 &And this finally is similar to $\myvec{0&0\\1&0}$ as before.\\
 &\\
\hline
&\\
Conclusion &Thus either $\vec{N}$ = 0 or $\vec{N}$ is similar over $\mathbb{C}$ to $\myvec{0&0\\1&0}$.\\
&\\
\hline
\caption{Solution summary}
\label{eq:solutions/6/2/11/table:1}
\end{longtable}

\item Let $\vec{F}$ be the field of real numbers, 
\begin{align}
\vec{A}=\myvec{2&0&0&0\\0&2&0&0\\0&0&3&0\\0&0&0&1}\\
p= (x - 2)(x - 3)(x - 1) 
\end{align}
\begin{enumerate}
\item Let $P_1$, $P_2$, $P_3$ be the Lagrange polynomials for $t_1 = 2$, $t_2 = 3$, $t_3 = 1$. Compute $E_i = P_i(\vec{A})\quad i = 1, 2, 3$. 
%
\\
\solution
See Table \ref{eq:solutions/6/2/11/table:1}
\onecolumn
\begin{longtable}{|l|l|}
\hline
\multirow{3}{*}{} & \\
Statement&Solution\\
\hline
&\parbox{6cm}{\begin{align}
\mbox{Let }\vec{N}&=\myvec{a&b\\c&d}\\
\mbox{Since }\vec{N}^2&=0
\end{align}}\\
&If $\myvec{a\\c},\myvec{b\\d}$ are linearly independent then $\vec{N}$ is diagonalizable to $\myvec{0&0\\0&0}$.\\
&\parbox{6cm}{\begin{align}
\mbox{If }\vec{P}\vec{N}\vec{P}^{-1}=0\\
\mbox{then }\vec{N}=\vec{P}^{-1}\vec{0}\vec{P}=0
\end{align}}\\
Proof that&So in this case $\vec{N}$ itself is the zero matrix.\\
$\vec{N}=0$&This contradicts the assumption that $\myvec{a\\c},\myvec{b\\d}$ are linearly independent.\\
&$\therefore$ we can assume that $\myvec{a\\c},\myvec{b\\d}$ are linearly dependent if both are\\
&equal to the zero vector\\
&\parbox{6cm}
{\begin{align}
   \mbox{then } \vec{N} &= 0.
\end{align}}\\
%\hline
%\pagebreak
\hline
&\\
&Therefore we can assume at least one vector is non-zero.\\
Assuming $\myvec{b\\d}$ as&Therefore
$\vec{N}=\myvec{a&0\\c&0}$\\
the zero vector&\\
&\parbox{6cm}{\begin{align}
\mbox{So }\vec{N}^2&=0\\
\implies a^2&=0\\
\therefore a&=0\\
\mbox{Thus }\vec{N}&=\myvec{a&0\\c&0}
\end{align}}\\
&In this case $\vec{N}$ is similar to $\vec{N}=\myvec{0&0\\1&0}$ via the matrix $\vec{P}=\myvec{c&0\\0&1}$\\
&\\
\hline
&\\
Assuming $\myvec{a\\c}$ as&Therefore
$\vec{N}=\myvec{0&b\\0&d}$\\
the zero vector&\\
&\parbox{6cm}{\begin{align}
\mbox{Then }\vec{N}^2&=0\\
\implies d^2&=0\\
\therefore d&=0\\
\mbox{Thus }\vec{N}&=\myvec{0&b\\0&0}
\end{align}}\\
&In this case $\vec{N}$ is similar to $\vec{N}=\myvec{0&0\\b&0}$ via the matrix $\vec{P}=\myvec{0&1\\1&0}$,\\
&which is similar to $\myvec{0&0\\1&0}$ as above.\\
&\\
\hline
&\\
Hence& we can assume neither $\myvec{a\\c}$ or $\myvec{b\\d}$ is the zero vector.\\
&\\
%\hline
%\pagebreak
\hline
&\\
Consequences of &Since they are linearly dependent we can assume,\\
linear&\\
independence&\\
&\parbox{6cm}{\begin{align}
\myvec{b\\d}&=x\myvec{a\\c}\\
\therefore \vec{N}&=\myvec{a&ax\\c&cx}\\
\therefore \vec{N}^2&=0\\
\implies a(a+cx)&=0\\
c(a+cx)&=0\\
ax(a+cx)&=0\\
cx(a+cx)&=0
\end{align}}\\
\hline
&\\
Proof that $\vec{N}$ is&We know that at least one of a or c is not zero.\\
similar over $\mathbb{C}$ to&If a = 0 then c $\neq$ 0, it must be that x = 0.\\
$\myvec{0&0\\1&0}$&So in this case $\vec{N}=\myvec{0&0\\c&0}$ which is similar to $\myvec{0&0\\1&0}$ as before.\\
&\\
&\parbox{6cm}{\begin{align}
\mbox{If } a &\neq0\\
\mbox{then }x &\neq0\\
\mbox{else }a(a+cx)&=0\\
\implies a&=0\\
\mbox{Thus }a+cx&=0\\
\mbox{Hence }\vec{N}&=\myvec{a&ax\\ \frac{-a}{x}&-a}
\end{align}}\\
 &This is similar to $\myvec{a&a\\-a&-a}$ via $\vec{P}=\myvec{\sqrt{x}&0\\0&\frac{1}{\sqrt{x}}}$.\\
 &And $\myvec{a&a\\-a&-a}$ is similar to $\myvec{0&0\\-a&0}$ via $\vec{P}=\myvec{-1&-1\\1&0}$\\
 &And this finally is similar to $\myvec{0&0\\1&0}$ as before.\\
 &\\
\hline
&\\
Conclusion &Thus either $\vec{N}$ = 0 or $\vec{N}$ is similar over $\mathbb{C}$ to $\myvec{0&0\\1&0}$.\\
&\\
\hline
\caption{Solution summary}
\label{eq:solutions/6/2/11/table:1}
\end{longtable}

\end{enumerate}
\item Let $n$ be a positive integer and $\mathbb{F}$ be a field. Suppose $\vec{A}$ is an $n \times n$ matrix over field $\mathbb{F}$ and $\vec{P}$ is an invertible $n \times n$ matrix over field $\mathbb{F}$. If $f$ is any polynomial over $\mathbb{F}$, prove
that,
\begin{align}
f(\vec{P}^{-1}\vec{A}\vec{P}) &= \vec{P}^{-1}f(\vec{A})\vec{P}
\end{align}
%
\\
\solution
See Table \ref{eq:solutions/6/2/11/table:1}
\onecolumn
\begin{longtable}{|l|l|}
\hline
\multirow{3}{*}{} & \\
Statement&Solution\\
\hline
&\parbox{6cm}{\begin{align}
\mbox{Let }\vec{N}&=\myvec{a&b\\c&d}\\
\mbox{Since }\vec{N}^2&=0
\end{align}}\\
&If $\myvec{a\\c},\myvec{b\\d}$ are linearly independent then $\vec{N}$ is diagonalizable to $\myvec{0&0\\0&0}$.\\
&\parbox{6cm}{\begin{align}
\mbox{If }\vec{P}\vec{N}\vec{P}^{-1}=0\\
\mbox{then }\vec{N}=\vec{P}^{-1}\vec{0}\vec{P}=0
\end{align}}\\
Proof that&So in this case $\vec{N}$ itself is the zero matrix.\\
$\vec{N}=0$&This contradicts the assumption that $\myvec{a\\c},\myvec{b\\d}$ are linearly independent.\\
&$\therefore$ we can assume that $\myvec{a\\c},\myvec{b\\d}$ are linearly dependent if both are\\
&equal to the zero vector\\
&\parbox{6cm}
{\begin{align}
   \mbox{then } \vec{N} &= 0.
\end{align}}\\
%\hline
%\pagebreak
\hline
&\\
&Therefore we can assume at least one vector is non-zero.\\
Assuming $\myvec{b\\d}$ as&Therefore
$\vec{N}=\myvec{a&0\\c&0}$\\
the zero vector&\\
&\parbox{6cm}{\begin{align}
\mbox{So }\vec{N}^2&=0\\
\implies a^2&=0\\
\therefore a&=0\\
\mbox{Thus }\vec{N}&=\myvec{a&0\\c&0}
\end{align}}\\
&In this case $\vec{N}$ is similar to $\vec{N}=\myvec{0&0\\1&0}$ via the matrix $\vec{P}=\myvec{c&0\\0&1}$\\
&\\
\hline
&\\
Assuming $\myvec{a\\c}$ as&Therefore
$\vec{N}=\myvec{0&b\\0&d}$\\
the zero vector&\\
&\parbox{6cm}{\begin{align}
\mbox{Then }\vec{N}^2&=0\\
\implies d^2&=0\\
\therefore d&=0\\
\mbox{Thus }\vec{N}&=\myvec{0&b\\0&0}
\end{align}}\\
&In this case $\vec{N}$ is similar to $\vec{N}=\myvec{0&0\\b&0}$ via the matrix $\vec{P}=\myvec{0&1\\1&0}$,\\
&which is similar to $\myvec{0&0\\1&0}$ as above.\\
&\\
\hline
&\\
Hence& we can assume neither $\myvec{a\\c}$ or $\myvec{b\\d}$ is the zero vector.\\
&\\
%\hline
%\pagebreak
\hline
&\\
Consequences of &Since they are linearly dependent we can assume,\\
linear&\\
independence&\\
&\parbox{6cm}{\begin{align}
\myvec{b\\d}&=x\myvec{a\\c}\\
\therefore \vec{N}&=\myvec{a&ax\\c&cx}\\
\therefore \vec{N}^2&=0\\
\implies a(a+cx)&=0\\
c(a+cx)&=0\\
ax(a+cx)&=0\\
cx(a+cx)&=0
\end{align}}\\
\hline
&\\
Proof that $\vec{N}$ is&We know that at least one of a or c is not zero.\\
similar over $\mathbb{C}$ to&If a = 0 then c $\neq$ 0, it must be that x = 0.\\
$\myvec{0&0\\1&0}$&So in this case $\vec{N}=\myvec{0&0\\c&0}$ which is similar to $\myvec{0&0\\1&0}$ as before.\\
&\\
&\parbox{6cm}{\begin{align}
\mbox{If } a &\neq0\\
\mbox{then }x &\neq0\\
\mbox{else }a(a+cx)&=0\\
\implies a&=0\\
\mbox{Thus }a+cx&=0\\
\mbox{Hence }\vec{N}&=\myvec{a&ax\\ \frac{-a}{x}&-a}
\end{align}}\\
 &This is similar to $\myvec{a&a\\-a&-a}$ via $\vec{P}=\myvec{\sqrt{x}&0\\0&\frac{1}{\sqrt{x}}}$.\\
 &And $\myvec{a&a\\-a&-a}$ is similar to $\myvec{0&0\\-a&0}$ via $\vec{P}=\myvec{-1&-1\\1&0}$\\
 &And this finally is similar to $\myvec{0&0\\1&0}$ as before.\\
 &\\
\hline
&\\
Conclusion &Thus either $\vec{N}$ = 0 or $\vec{N}$ is similar over $\mathbb{C}$ to $\myvec{0&0\\1&0}$.\\
&\\
\hline
\caption{Solution summary}
\label{eq:solutions/6/2/11/table:1}
\end{longtable}

\item Let $\vec{F}$ be a field. We have considered certain special linear functionals on $\vec{F}[x]$
obtained via ‘evaluation at t’:
$L(f) = f(t)$.
Such functionals are not only linear but also have the property that L(fg) =L(f)L(g). Prove that if L is any linear functional on F[x] such that L(fg) =L(f)L(g)for all f and g, then either L = 0 or there is a t in F such that L(f) = f(t) for all f.
%
\\
\solution
See Table \ref{eq:solutions/6/2/11/table:1}
\onecolumn
\begin{longtable}{|l|l|}
\hline
\multirow{3}{*}{} & \\
Statement&Solution\\
\hline
&\parbox{6cm}{\begin{align}
\mbox{Let }\vec{N}&=\myvec{a&b\\c&d}\\
\mbox{Since }\vec{N}^2&=0
\end{align}}\\
&If $\myvec{a\\c},\myvec{b\\d}$ are linearly independent then $\vec{N}$ is diagonalizable to $\myvec{0&0\\0&0}$.\\
&\parbox{6cm}{\begin{align}
\mbox{If }\vec{P}\vec{N}\vec{P}^{-1}=0\\
\mbox{then }\vec{N}=\vec{P}^{-1}\vec{0}\vec{P}=0
\end{align}}\\
Proof that&So in this case $\vec{N}$ itself is the zero matrix.\\
$\vec{N}=0$&This contradicts the assumption that $\myvec{a\\c},\myvec{b\\d}$ are linearly independent.\\
&$\therefore$ we can assume that $\myvec{a\\c},\myvec{b\\d}$ are linearly dependent if both are\\
&equal to the zero vector\\
&\parbox{6cm}
{\begin{align}
   \mbox{then } \vec{N} &= 0.
\end{align}}\\
%\hline
%\pagebreak
\hline
&\\
&Therefore we can assume at least one vector is non-zero.\\
Assuming $\myvec{b\\d}$ as&Therefore
$\vec{N}=\myvec{a&0\\c&0}$\\
the zero vector&\\
&\parbox{6cm}{\begin{align}
\mbox{So }\vec{N}^2&=0\\
\implies a^2&=0\\
\therefore a&=0\\
\mbox{Thus }\vec{N}&=\myvec{a&0\\c&0}
\end{align}}\\
&In this case $\vec{N}$ is similar to $\vec{N}=\myvec{0&0\\1&0}$ via the matrix $\vec{P}=\myvec{c&0\\0&1}$\\
&\\
\hline
&\\
Assuming $\myvec{a\\c}$ as&Therefore
$\vec{N}=\myvec{0&b\\0&d}$\\
the zero vector&\\
&\parbox{6cm}{\begin{align}
\mbox{Then }\vec{N}^2&=0\\
\implies d^2&=0\\
\therefore d&=0\\
\mbox{Thus }\vec{N}&=\myvec{0&b\\0&0}
\end{align}}\\
&In this case $\vec{N}$ is similar to $\vec{N}=\myvec{0&0\\b&0}$ via the matrix $\vec{P}=\myvec{0&1\\1&0}$,\\
&which is similar to $\myvec{0&0\\1&0}$ as above.\\
&\\
\hline
&\\
Hence& we can assume neither $\myvec{a\\c}$ or $\myvec{b\\d}$ is the zero vector.\\
&\\
%\hline
%\pagebreak
\hline
&\\
Consequences of &Since they are linearly dependent we can assume,\\
linear&\\
independence&\\
&\parbox{6cm}{\begin{align}
\myvec{b\\d}&=x\myvec{a\\c}\\
\therefore \vec{N}&=\myvec{a&ax\\c&cx}\\
\therefore \vec{N}^2&=0\\
\implies a(a+cx)&=0\\
c(a+cx)&=0\\
ax(a+cx)&=0\\
cx(a+cx)&=0
\end{align}}\\
\hline
&\\
Proof that $\vec{N}$ is&We know that at least one of a or c is not zero.\\
similar over $\mathbb{C}$ to&If a = 0 then c $\neq$ 0, it must be that x = 0.\\
$\myvec{0&0\\1&0}$&So in this case $\vec{N}=\myvec{0&0\\c&0}$ which is similar to $\myvec{0&0\\1&0}$ as before.\\
&\\
&\parbox{6cm}{\begin{align}
\mbox{If } a &\neq0\\
\mbox{then }x &\neq0\\
\mbox{else }a(a+cx)&=0\\
\implies a&=0\\
\mbox{Thus }a+cx&=0\\
\mbox{Hence }\vec{N}&=\myvec{a&ax\\ \frac{-a}{x}&-a}
\end{align}}\\
 &This is similar to $\myvec{a&a\\-a&-a}$ via $\vec{P}=\myvec{\sqrt{x}&0\\0&\frac{1}{\sqrt{x}}}$.\\
 &And $\myvec{a&a\\-a&-a}$ is similar to $\myvec{0&0\\-a&0}$ via $\vec{P}=\myvec{-1&-1\\1&0}$\\
 &And this finally is similar to $\myvec{0&0\\1&0}$ as before.\\
 &\\
\hline
&\\
Conclusion &Thus either $\vec{N}$ = 0 or $\vec{N}$ is similar over $\mathbb{C}$ to $\myvec{0&0\\1&0}$.\\
&\\
\hline
\caption{Solution summary}
\label{eq:solutions/6/2/11/table:1}
\end{longtable}

\item If $F$ is a field and $h$ is a polynomial over $F$ of degree $\geq 1$,  show that the mapping $f \rightarrow f(h)$ is a one-one linear transformation of $F[x]$ into $F[x]$. Show that this transformation is an isomorphism of $F[x]$ onto $F[x]$ if and only if deg $h = 1$. 
%
\\
\solution
See Table \ref{eq:solutions/6/2/11/table:1}
\onecolumn
\begin{longtable}{|l|l|}
\hline
\multirow{3}{*}{} & \\
Statement&Solution\\
\hline
&\parbox{6cm}{\begin{align}
\mbox{Let }\vec{N}&=\myvec{a&b\\c&d}\\
\mbox{Since }\vec{N}^2&=0
\end{align}}\\
&If $\myvec{a\\c},\myvec{b\\d}$ are linearly independent then $\vec{N}$ is diagonalizable to $\myvec{0&0\\0&0}$.\\
&\parbox{6cm}{\begin{align}
\mbox{If }\vec{P}\vec{N}\vec{P}^{-1}=0\\
\mbox{then }\vec{N}=\vec{P}^{-1}\vec{0}\vec{P}=0
\end{align}}\\
Proof that&So in this case $\vec{N}$ itself is the zero matrix.\\
$\vec{N}=0$&This contradicts the assumption that $\myvec{a\\c},\myvec{b\\d}$ are linearly independent.\\
&$\therefore$ we can assume that $\myvec{a\\c},\myvec{b\\d}$ are linearly dependent if both are\\
&equal to the zero vector\\
&\parbox{6cm}
{\begin{align}
   \mbox{then } \vec{N} &= 0.
\end{align}}\\
%\hline
%\pagebreak
\hline
&\\
&Therefore we can assume at least one vector is non-zero.\\
Assuming $\myvec{b\\d}$ as&Therefore
$\vec{N}=\myvec{a&0\\c&0}$\\
the zero vector&\\
&\parbox{6cm}{\begin{align}
\mbox{So }\vec{N}^2&=0\\
\implies a^2&=0\\
\therefore a&=0\\
\mbox{Thus }\vec{N}&=\myvec{a&0\\c&0}
\end{align}}\\
&In this case $\vec{N}$ is similar to $\vec{N}=\myvec{0&0\\1&0}$ via the matrix $\vec{P}=\myvec{c&0\\0&1}$\\
&\\
\hline
&\\
Assuming $\myvec{a\\c}$ as&Therefore
$\vec{N}=\myvec{0&b\\0&d}$\\
the zero vector&\\
&\parbox{6cm}{\begin{align}
\mbox{Then }\vec{N}^2&=0\\
\implies d^2&=0\\
\therefore d&=0\\
\mbox{Thus }\vec{N}&=\myvec{0&b\\0&0}
\end{align}}\\
&In this case $\vec{N}$ is similar to $\vec{N}=\myvec{0&0\\b&0}$ via the matrix $\vec{P}=\myvec{0&1\\1&0}$,\\
&which is similar to $\myvec{0&0\\1&0}$ as above.\\
&\\
\hline
&\\
Hence& we can assume neither $\myvec{a\\c}$ or $\myvec{b\\d}$ is the zero vector.\\
&\\
%\hline
%\pagebreak
\hline
&\\
Consequences of &Since they are linearly dependent we can assume,\\
linear&\\
independence&\\
&\parbox{6cm}{\begin{align}
\myvec{b\\d}&=x\myvec{a\\c}\\
\therefore \vec{N}&=\myvec{a&ax\\c&cx}\\
\therefore \vec{N}^2&=0\\
\implies a(a+cx)&=0\\
c(a+cx)&=0\\
ax(a+cx)&=0\\
cx(a+cx)&=0
\end{align}}\\
\hline
&\\
Proof that $\vec{N}$ is&We know that at least one of a or c is not zero.\\
similar over $\mathbb{C}$ to&If a = 0 then c $\neq$ 0, it must be that x = 0.\\
$\myvec{0&0\\1&0}$&So in this case $\vec{N}=\myvec{0&0\\c&0}$ which is similar to $\myvec{0&0\\1&0}$ as before.\\
&\\
&\parbox{6cm}{\begin{align}
\mbox{If } a &\neq0\\
\mbox{then }x &\neq0\\
\mbox{else }a(a+cx)&=0\\
\implies a&=0\\
\mbox{Thus }a+cx&=0\\
\mbox{Hence }\vec{N}&=\myvec{a&ax\\ \frac{-a}{x}&-a}
\end{align}}\\
 &This is similar to $\myvec{a&a\\-a&-a}$ via $\vec{P}=\myvec{\sqrt{x}&0\\0&\frac{1}{\sqrt{x}}}$.\\
 &And $\myvec{a&a\\-a&-a}$ is similar to $\myvec{0&0\\-a&0}$ via $\vec{P}=\myvec{-1&-1\\1&0}$\\
 &And this finally is similar to $\myvec{0&0\\1&0}$ as before.\\
 &\\
\hline
&\\
Conclusion &Thus either $\vec{N}$ = 0 or $\vec{N}$ is similar over $\mathbb{C}$ to $\myvec{0&0\\1&0}$.\\
&\\
\hline
\caption{Solution summary}
\label{eq:solutions/6/2/11/table:1}
\end{longtable}

\end{enumerate}

\subsection{Polynomial Ideals}
\renewcommand{\theequation}{\theenumi}
\renewcommand{\thefigure}{\theenumi}
\begin{enumerate}[label=\thesubsection.\arabic*.,ref=\thesubsection.\theenumi]
\numberwithin{equation}{enumi}
\numberwithin{figure}{enumi}

\item 
\begin{enumerate}
\item Find the g.c.d of each of the following pairs of polynomials.
    \begin{align}
    2x^{5}-x^{3}-3x^{2}-6x+4 , x^{4}+x^{3}-x^{2}-2x-2
    \end{align}
%
\\
\solution
See Table \ref{eq:solutions/6/2/11/table:1}
\onecolumn
\begin{longtable}{|l|l|}
\hline
\multirow{3}{*}{} & \\
Statement&Solution\\
\hline
&\parbox{6cm}{\begin{align}
\mbox{Let }\vec{N}&=\myvec{a&b\\c&d}\\
\mbox{Since }\vec{N}^2&=0
\end{align}}\\
&If $\myvec{a\\c},\myvec{b\\d}$ are linearly independent then $\vec{N}$ is diagonalizable to $\myvec{0&0\\0&0}$.\\
&\parbox{6cm}{\begin{align}
\mbox{If }\vec{P}\vec{N}\vec{P}^{-1}=0\\
\mbox{then }\vec{N}=\vec{P}^{-1}\vec{0}\vec{P}=0
\end{align}}\\
Proof that&So in this case $\vec{N}$ itself is the zero matrix.\\
$\vec{N}=0$&This contradicts the assumption that $\myvec{a\\c},\myvec{b\\d}$ are linearly independent.\\
&$\therefore$ we can assume that $\myvec{a\\c},\myvec{b\\d}$ are linearly dependent if both are\\
&equal to the zero vector\\
&\parbox{6cm}
{\begin{align}
   \mbox{then } \vec{N} &= 0.
\end{align}}\\
%\hline
%\pagebreak
\hline
&\\
&Therefore we can assume at least one vector is non-zero.\\
Assuming $\myvec{b\\d}$ as&Therefore
$\vec{N}=\myvec{a&0\\c&0}$\\
the zero vector&\\
&\parbox{6cm}{\begin{align}
\mbox{So }\vec{N}^2&=0\\
\implies a^2&=0\\
\therefore a&=0\\
\mbox{Thus }\vec{N}&=\myvec{a&0\\c&0}
\end{align}}\\
&In this case $\vec{N}$ is similar to $\vec{N}=\myvec{0&0\\1&0}$ via the matrix $\vec{P}=\myvec{c&0\\0&1}$\\
&\\
\hline
&\\
Assuming $\myvec{a\\c}$ as&Therefore
$\vec{N}=\myvec{0&b\\0&d}$\\
the zero vector&\\
&\parbox{6cm}{\begin{align}
\mbox{Then }\vec{N}^2&=0\\
\implies d^2&=0\\
\therefore d&=0\\
\mbox{Thus }\vec{N}&=\myvec{0&b\\0&0}
\end{align}}\\
&In this case $\vec{N}$ is similar to $\vec{N}=\myvec{0&0\\b&0}$ via the matrix $\vec{P}=\myvec{0&1\\1&0}$,\\
&which is similar to $\myvec{0&0\\1&0}$ as above.\\
&\\
\hline
&\\
Hence& we can assume neither $\myvec{a\\c}$ or $\myvec{b\\d}$ is the zero vector.\\
&\\
%\hline
%\pagebreak
\hline
&\\
Consequences of &Since they are linearly dependent we can assume,\\
linear&\\
independence&\\
&\parbox{6cm}{\begin{align}
\myvec{b\\d}&=x\myvec{a\\c}\\
\therefore \vec{N}&=\myvec{a&ax\\c&cx}\\
\therefore \vec{N}^2&=0\\
\implies a(a+cx)&=0\\
c(a+cx)&=0\\
ax(a+cx)&=0\\
cx(a+cx)&=0
\end{align}}\\
\hline
&\\
Proof that $\vec{N}$ is&We know that at least one of a or c is not zero.\\
similar over $\mathbb{C}$ to&If a = 0 then c $\neq$ 0, it must be that x = 0.\\
$\myvec{0&0\\1&0}$&So in this case $\vec{N}=\myvec{0&0\\c&0}$ which is similar to $\myvec{0&0\\1&0}$ as before.\\
&\\
&\parbox{6cm}{\begin{align}
\mbox{If } a &\neq0\\
\mbox{then }x &\neq0\\
\mbox{else }a(a+cx)&=0\\
\implies a&=0\\
\mbox{Thus }a+cx&=0\\
\mbox{Hence }\vec{N}&=\myvec{a&ax\\ \frac{-a}{x}&-a}
\end{align}}\\
 &This is similar to $\myvec{a&a\\-a&-a}$ via $\vec{P}=\myvec{\sqrt{x}&0\\0&\frac{1}{\sqrt{x}}}$.\\
 &And $\myvec{a&a\\-a&-a}$ is similar to $\myvec{0&0\\-a&0}$ via $\vec{P}=\myvec{-1&-1\\1&0}$\\
 &And this finally is similar to $\myvec{0&0\\1&0}$ as before.\\
 &\\
\hline
&\\
Conclusion &Thus either $\vec{N}$ = 0 or $\vec{N}$ is similar over $\mathbb{C}$ to $\myvec{0&0\\1&0}$.\\
&\\
\hline
\caption{Solution summary}
\label{eq:solutions/6/2/11/table:1}
\end{longtable}

\end{enumerate}
\end{enumerate}

\section{Determinants}
\section{Elementary Canonical Forms}
\subsection{Characteristic Values}
\renewcommand{\theequation}{\theenumi}
\renewcommand{\thefigure}{\theenumi}
\begin{enumerate}[label=\thesubsection.\arabic*.,ref=\thesubsection.\theenumi]
\numberwithin{equation}{enumi}
\numberwithin{figure}{enumi}

\item Let T be the linear operator on $\vec{R}^3$ which is represented in the standard ordered basis by the matrix
\begin{align}
\myvec{-9&4&4\\-8&3&4\\-16&8&7}
\label{eq:solutions/6/2/4/Q}
\end{align}
Prove that T is diagonalizable by exhibiting a basis for $\vec{R}^3$, each vector of which is a characteristic vector of T. 
%
%
\\
\solution
See Table \ref{eq:solutions/6/2/11/table:1}
\onecolumn
\begin{longtable}{|l|l|}
\hline
\multirow{3}{*}{} & \\
Statement&Solution\\
\hline
&\parbox{6cm}{\begin{align}
\mbox{Let }\vec{N}&=\myvec{a&b\\c&d}\\
\mbox{Since }\vec{N}^2&=0
\end{align}}\\
&If $\myvec{a\\c},\myvec{b\\d}$ are linearly independent then $\vec{N}$ is diagonalizable to $\myvec{0&0\\0&0}$.\\
&\parbox{6cm}{\begin{align}
\mbox{If }\vec{P}\vec{N}\vec{P}^{-1}=0\\
\mbox{then }\vec{N}=\vec{P}^{-1}\vec{0}\vec{P}=0
\end{align}}\\
Proof that&So in this case $\vec{N}$ itself is the zero matrix.\\
$\vec{N}=0$&This contradicts the assumption that $\myvec{a\\c},\myvec{b\\d}$ are linearly independent.\\
&$\therefore$ we can assume that $\myvec{a\\c},\myvec{b\\d}$ are linearly dependent if both are\\
&equal to the zero vector\\
&\parbox{6cm}
{\begin{align}
   \mbox{then } \vec{N} &= 0.
\end{align}}\\
%\hline
%\pagebreak
\hline
&\\
&Therefore we can assume at least one vector is non-zero.\\
Assuming $\myvec{b\\d}$ as&Therefore
$\vec{N}=\myvec{a&0\\c&0}$\\
the zero vector&\\
&\parbox{6cm}{\begin{align}
\mbox{So }\vec{N}^2&=0\\
\implies a^2&=0\\
\therefore a&=0\\
\mbox{Thus }\vec{N}&=\myvec{a&0\\c&0}
\end{align}}\\
&In this case $\vec{N}$ is similar to $\vec{N}=\myvec{0&0\\1&0}$ via the matrix $\vec{P}=\myvec{c&0\\0&1}$\\
&\\
\hline
&\\
Assuming $\myvec{a\\c}$ as&Therefore
$\vec{N}=\myvec{0&b\\0&d}$\\
the zero vector&\\
&\parbox{6cm}{\begin{align}
\mbox{Then }\vec{N}^2&=0\\
\implies d^2&=0\\
\therefore d&=0\\
\mbox{Thus }\vec{N}&=\myvec{0&b\\0&0}
\end{align}}\\
&In this case $\vec{N}$ is similar to $\vec{N}=\myvec{0&0\\b&0}$ via the matrix $\vec{P}=\myvec{0&1\\1&0}$,\\
&which is similar to $\myvec{0&0\\1&0}$ as above.\\
&\\
\hline
&\\
Hence& we can assume neither $\myvec{a\\c}$ or $\myvec{b\\d}$ is the zero vector.\\
&\\
%\hline
%\pagebreak
\hline
&\\
Consequences of &Since they are linearly dependent we can assume,\\
linear&\\
independence&\\
&\parbox{6cm}{\begin{align}
\myvec{b\\d}&=x\myvec{a\\c}\\
\therefore \vec{N}&=\myvec{a&ax\\c&cx}\\
\therefore \vec{N}^2&=0\\
\implies a(a+cx)&=0\\
c(a+cx)&=0\\
ax(a+cx)&=0\\
cx(a+cx)&=0
\end{align}}\\
\hline
&\\
Proof that $\vec{N}$ is&We know that at least one of a or c is not zero.\\
similar over $\mathbb{C}$ to&If a = 0 then c $\neq$ 0, it must be that x = 0.\\
$\myvec{0&0\\1&0}$&So in this case $\vec{N}=\myvec{0&0\\c&0}$ which is similar to $\myvec{0&0\\1&0}$ as before.\\
&\\
&\parbox{6cm}{\begin{align}
\mbox{If } a &\neq0\\
\mbox{then }x &\neq0\\
\mbox{else }a(a+cx)&=0\\
\implies a&=0\\
\mbox{Thus }a+cx&=0\\
\mbox{Hence }\vec{N}&=\myvec{a&ax\\ \frac{-a}{x}&-a}
\end{align}}\\
 &This is similar to $\myvec{a&a\\-a&-a}$ via $\vec{P}=\myvec{\sqrt{x}&0\\0&\frac{1}{\sqrt{x}}}$.\\
 &And $\myvec{a&a\\-a&-a}$ is similar to $\myvec{0&0\\-a&0}$ via $\vec{P}=\myvec{-1&-1\\1&0}$\\
 &And this finally is similar to $\myvec{0&0\\1&0}$ as before.\\
 &\\
\hline
&\\
Conclusion &Thus either $\vec{N}$ = 0 or $\vec{N}$ is similar over $\mathbb{C}$ to $\myvec{0&0\\1&0}$.\\
&\\
\hline
\caption{Solution summary}
\label{eq:solutions/6/2/11/table:1}
\end{longtable}

\end{enumerate}

\subsection{Annihilating Polynomials}
\renewcommand{\theequation}{\theenumi}
\renewcommand{\thefigure}{\theenumi}
\begin{enumerate}[label=\thesubsection.\arabic*.,ref=\thesubsection.\theenumi]
\numberwithin{equation}{enumi}
\numberwithin{figure}{enumi}


\item Let a, b and c be the elements of a field $\vec{F}$, and let $\vec{A}$ be the following 3$\times$ 3 matrix over $\vec{F}$
\begin{align}
\vec{A} = \myvec{0&0&c\\1&0&b\\0&1&a}
\label{eq:solutions/6/3/2/Q}
\end{align}
Prove that the characteristic polynomial for $\vec{A}$ is $x^3-ax^2-bx-c$ and that this is also minimal polynomial for $\vec{A}$.
%
\solution
See Table \ref{eq:solutions/6/2/11/table:1}
\onecolumn
\begin{longtable}{|l|l|}
\hline
\multirow{3}{*}{} & \\
Statement&Solution\\
\hline
&\parbox{6cm}{\begin{align}
\mbox{Let }\vec{N}&=\myvec{a&b\\c&d}\\
\mbox{Since }\vec{N}^2&=0
\end{align}}\\
&If $\myvec{a\\c},\myvec{b\\d}$ are linearly independent then $\vec{N}$ is diagonalizable to $\myvec{0&0\\0&0}$.\\
&\parbox{6cm}{\begin{align}
\mbox{If }\vec{P}\vec{N}\vec{P}^{-1}=0\\
\mbox{then }\vec{N}=\vec{P}^{-1}\vec{0}\vec{P}=0
\end{align}}\\
Proof that&So in this case $\vec{N}$ itself is the zero matrix.\\
$\vec{N}=0$&This contradicts the assumption that $\myvec{a\\c},\myvec{b\\d}$ are linearly independent.\\
&$\therefore$ we can assume that $\myvec{a\\c},\myvec{b\\d}$ are linearly dependent if both are\\
&equal to the zero vector\\
&\parbox{6cm}
{\begin{align}
   \mbox{then } \vec{N} &= 0.
\end{align}}\\
%\hline
%\pagebreak
\hline
&\\
&Therefore we can assume at least one vector is non-zero.\\
Assuming $\myvec{b\\d}$ as&Therefore
$\vec{N}=\myvec{a&0\\c&0}$\\
the zero vector&\\
&\parbox{6cm}{\begin{align}
\mbox{So }\vec{N}^2&=0\\
\implies a^2&=0\\
\therefore a&=0\\
\mbox{Thus }\vec{N}&=\myvec{a&0\\c&0}
\end{align}}\\
&In this case $\vec{N}$ is similar to $\vec{N}=\myvec{0&0\\1&0}$ via the matrix $\vec{P}=\myvec{c&0\\0&1}$\\
&\\
\hline
&\\
Assuming $\myvec{a\\c}$ as&Therefore
$\vec{N}=\myvec{0&b\\0&d}$\\
the zero vector&\\
&\parbox{6cm}{\begin{align}
\mbox{Then }\vec{N}^2&=0\\
\implies d^2&=0\\
\therefore d&=0\\
\mbox{Thus }\vec{N}&=\myvec{0&b\\0&0}
\end{align}}\\
&In this case $\vec{N}$ is similar to $\vec{N}=\myvec{0&0\\b&0}$ via the matrix $\vec{P}=\myvec{0&1\\1&0}$,\\
&which is similar to $\myvec{0&0\\1&0}$ as above.\\
&\\
\hline
&\\
Hence& we can assume neither $\myvec{a\\c}$ or $\myvec{b\\d}$ is the zero vector.\\
&\\
%\hline
%\pagebreak
\hline
&\\
Consequences of &Since they are linearly dependent we can assume,\\
linear&\\
independence&\\
&\parbox{6cm}{\begin{align}
\myvec{b\\d}&=x\myvec{a\\c}\\
\therefore \vec{N}&=\myvec{a&ax\\c&cx}\\
\therefore \vec{N}^2&=0\\
\implies a(a+cx)&=0\\
c(a+cx)&=0\\
ax(a+cx)&=0\\
cx(a+cx)&=0
\end{align}}\\
\hline
&\\
Proof that $\vec{N}$ is&We know that at least one of a or c is not zero.\\
similar over $\mathbb{C}$ to&If a = 0 then c $\neq$ 0, it must be that x = 0.\\
$\myvec{0&0\\1&0}$&So in this case $\vec{N}=\myvec{0&0\\c&0}$ which is similar to $\myvec{0&0\\1&0}$ as before.\\
&\\
&\parbox{6cm}{\begin{align}
\mbox{If } a &\neq0\\
\mbox{then }x &\neq0\\
\mbox{else }a(a+cx)&=0\\
\implies a&=0\\
\mbox{Thus }a+cx&=0\\
\mbox{Hence }\vec{N}&=\myvec{a&ax\\ \frac{-a}{x}&-a}
\end{align}}\\
 &This is similar to $\myvec{a&a\\-a&-a}$ via $\vec{P}=\myvec{\sqrt{x}&0\\0&\frac{1}{\sqrt{x}}}$.\\
 &And $\myvec{a&a\\-a&-a}$ is similar to $\myvec{0&0\\-a&0}$ via $\vec{P}=\myvec{-1&-1\\1&0}$\\
 &And this finally is similar to $\myvec{0&0\\1&0}$ as before.\\
 &\\
\hline
&\\
Conclusion &Thus either $\vec{N}$ = 0 or $\vec{N}$ is similar over $\mathbb{C}$ to $\myvec{0&0\\1&0}$.\\
&\\
\hline
\caption{Solution summary}
\label{eq:solutions/6/2/11/table:1}
\end{longtable}

\item %
Let $\Vec{A}$ be the $4\times 4$ real matrix
\begin{align}
    \Vec{A}=\myvec{1&1&0&0\\-1&-1&0&0\\-2&-2&2&1\\1&1&-1&0}
\end{align}
Show that the characteristic polynomial for $\Vec{A}$ is $x^2\brak{x-1}^2$ and that it is also the minimal polynomial
%
\\
%
\solution
See Table \ref{eq:solutions/6/2/11/table:1}
\onecolumn
\begin{longtable}{|l|l|}
\hline
\multirow{3}{*}{} & \\
Statement&Solution\\
\hline
&\parbox{6cm}{\begin{align}
\mbox{Let }\vec{N}&=\myvec{a&b\\c&d}\\
\mbox{Since }\vec{N}^2&=0
\end{align}}\\
&If $\myvec{a\\c},\myvec{b\\d}$ are linearly independent then $\vec{N}$ is diagonalizable to $\myvec{0&0\\0&0}$.\\
&\parbox{6cm}{\begin{align}
\mbox{If }\vec{P}\vec{N}\vec{P}^{-1}=0\\
\mbox{then }\vec{N}=\vec{P}^{-1}\vec{0}\vec{P}=0
\end{align}}\\
Proof that&So in this case $\vec{N}$ itself is the zero matrix.\\
$\vec{N}=0$&This contradicts the assumption that $\myvec{a\\c},\myvec{b\\d}$ are linearly independent.\\
&$\therefore$ we can assume that $\myvec{a\\c},\myvec{b\\d}$ are linearly dependent if both are\\
&equal to the zero vector\\
&\parbox{6cm}
{\begin{align}
   \mbox{then } \vec{N} &= 0.
\end{align}}\\
%\hline
%\pagebreak
\hline
&\\
&Therefore we can assume at least one vector is non-zero.\\
Assuming $\myvec{b\\d}$ as&Therefore
$\vec{N}=\myvec{a&0\\c&0}$\\
the zero vector&\\
&\parbox{6cm}{\begin{align}
\mbox{So }\vec{N}^2&=0\\
\implies a^2&=0\\
\therefore a&=0\\
\mbox{Thus }\vec{N}&=\myvec{a&0\\c&0}
\end{align}}\\
&In this case $\vec{N}$ is similar to $\vec{N}=\myvec{0&0\\1&0}$ via the matrix $\vec{P}=\myvec{c&0\\0&1}$\\
&\\
\hline
&\\
Assuming $\myvec{a\\c}$ as&Therefore
$\vec{N}=\myvec{0&b\\0&d}$\\
the zero vector&\\
&\parbox{6cm}{\begin{align}
\mbox{Then }\vec{N}^2&=0\\
\implies d^2&=0\\
\therefore d&=0\\
\mbox{Thus }\vec{N}&=\myvec{0&b\\0&0}
\end{align}}\\
&In this case $\vec{N}$ is similar to $\vec{N}=\myvec{0&0\\b&0}$ via the matrix $\vec{P}=\myvec{0&1\\1&0}$,\\
&which is similar to $\myvec{0&0\\1&0}$ as above.\\
&\\
\hline
&\\
Hence& we can assume neither $\myvec{a\\c}$ or $\myvec{b\\d}$ is the zero vector.\\
&\\
%\hline
%\pagebreak
\hline
&\\
Consequences of &Since they are linearly dependent we can assume,\\
linear&\\
independence&\\
&\parbox{6cm}{\begin{align}
\myvec{b\\d}&=x\myvec{a\\c}\\
\therefore \vec{N}&=\myvec{a&ax\\c&cx}\\
\therefore \vec{N}^2&=0\\
\implies a(a+cx)&=0\\
c(a+cx)&=0\\
ax(a+cx)&=0\\
cx(a+cx)&=0
\end{align}}\\
\hline
&\\
Proof that $\vec{N}$ is&We know that at least one of a or c is not zero.\\
similar over $\mathbb{C}$ to&If a = 0 then c $\neq$ 0, it must be that x = 0.\\
$\myvec{0&0\\1&0}$&So in this case $\vec{N}=\myvec{0&0\\c&0}$ which is similar to $\myvec{0&0\\1&0}$ as before.\\
&\\
&\parbox{6cm}{\begin{align}
\mbox{If } a &\neq0\\
\mbox{then }x &\neq0\\
\mbox{else }a(a+cx)&=0\\
\implies a&=0\\
\mbox{Thus }a+cx&=0\\
\mbox{Hence }\vec{N}&=\myvec{a&ax\\ \frac{-a}{x}&-a}
\end{align}}\\
 &This is similar to $\myvec{a&a\\-a&-a}$ via $\vec{P}=\myvec{\sqrt{x}&0\\0&\frac{1}{\sqrt{x}}}$.\\
 &And $\myvec{a&a\\-a&-a}$ is similar to $\myvec{0&0\\-a&0}$ via $\vec{P}=\myvec{-1&-1\\1&0}$\\
 &And this finally is similar to $\myvec{0&0\\1&0}$ as before.\\
 &\\
\hline
&\\
Conclusion &Thus either $\vec{N}$ = 0 or $\vec{N}$ is similar over $\mathbb{C}$ to $\myvec{0&0\\1&0}$.\\
&\\
\hline
\caption{Solution summary}
\label{eq:solutions/6/2/11/table:1}
\end{longtable}

\twocolumn
\item Let $\vec{A}$ and $\vec{B}$ be $n\times n$ matrices over the 
field $F$.Prove that if $\brak{\vec{I}-\vec{A}\vec{B}}$ is invertibe
\begin{enumerate}
    \item $\brak{\vec{I}-\vec{B}\vec{A}}$ is invertible and
    \item $\brak{\vec{I}-\vec{B}\vec{A}}^{-1}=\vec{I}+\vec{B}\brak{\vec{I}-\vec{A}\vec{B}}^{-1}\vec{A}$
\end{enumerate}
%
\item Let $P$ be the operator on $R^2$ which projects each vector onto the x-axis, parallel to the y-axis: $p\brak{x,y}$ = $\brak{x,0}$. Show that $P$ is linear. What is the minimal polynomial for $P$? 
%
\\
\solution
See Table \ref{eq:solutions/6/2/11/table:1}
\onecolumn
\begin{longtable}{|l|l|}
\hline
\multirow{3}{*}{} & \\
Statement&Solution\\
\hline
&\parbox{6cm}{\begin{align}
\mbox{Let }\vec{N}&=\myvec{a&b\\c&d}\\
\mbox{Since }\vec{N}^2&=0
\end{align}}\\
&If $\myvec{a\\c},\myvec{b\\d}$ are linearly independent then $\vec{N}$ is diagonalizable to $\myvec{0&0\\0&0}$.\\
&\parbox{6cm}{\begin{align}
\mbox{If }\vec{P}\vec{N}\vec{P}^{-1}=0\\
\mbox{then }\vec{N}=\vec{P}^{-1}\vec{0}\vec{P}=0
\end{align}}\\
Proof that&So in this case $\vec{N}$ itself is the zero matrix.\\
$\vec{N}=0$&This contradicts the assumption that $\myvec{a\\c},\myvec{b\\d}$ are linearly independent.\\
&$\therefore$ we can assume that $\myvec{a\\c},\myvec{b\\d}$ are linearly dependent if both are\\
&equal to the zero vector\\
&\parbox{6cm}
{\begin{align}
   \mbox{then } \vec{N} &= 0.
\end{align}}\\
%\hline
%\pagebreak
\hline
&\\
&Therefore we can assume at least one vector is non-zero.\\
Assuming $\myvec{b\\d}$ as&Therefore
$\vec{N}=\myvec{a&0\\c&0}$\\
the zero vector&\\
&\parbox{6cm}{\begin{align}
\mbox{So }\vec{N}^2&=0\\
\implies a^2&=0\\
\therefore a&=0\\
\mbox{Thus }\vec{N}&=\myvec{a&0\\c&0}
\end{align}}\\
&In this case $\vec{N}$ is similar to $\vec{N}=\myvec{0&0\\1&0}$ via the matrix $\vec{P}=\myvec{c&0\\0&1}$\\
&\\
\hline
&\\
Assuming $\myvec{a\\c}$ as&Therefore
$\vec{N}=\myvec{0&b\\0&d}$\\
the zero vector&\\
&\parbox{6cm}{\begin{align}
\mbox{Then }\vec{N}^2&=0\\
\implies d^2&=0\\
\therefore d&=0\\
\mbox{Thus }\vec{N}&=\myvec{0&b\\0&0}
\end{align}}\\
&In this case $\vec{N}$ is similar to $\vec{N}=\myvec{0&0\\b&0}$ via the matrix $\vec{P}=\myvec{0&1\\1&0}$,\\
&which is similar to $\myvec{0&0\\1&0}$ as above.\\
&\\
\hline
&\\
Hence& we can assume neither $\myvec{a\\c}$ or $\myvec{b\\d}$ is the zero vector.\\
&\\
%\hline
%\pagebreak
\hline
&\\
Consequences of &Since they are linearly dependent we can assume,\\
linear&\\
independence&\\
&\parbox{6cm}{\begin{align}
\myvec{b\\d}&=x\myvec{a\\c}\\
\therefore \vec{N}&=\myvec{a&ax\\c&cx}\\
\therefore \vec{N}^2&=0\\
\implies a(a+cx)&=0\\
c(a+cx)&=0\\
ax(a+cx)&=0\\
cx(a+cx)&=0
\end{align}}\\
\hline
&\\
Proof that $\vec{N}$ is&We know that at least one of a or c is not zero.\\
similar over $\mathbb{C}$ to&If a = 0 then c $\neq$ 0, it must be that x = 0.\\
$\myvec{0&0\\1&0}$&So in this case $\vec{N}=\myvec{0&0\\c&0}$ which is similar to $\myvec{0&0\\1&0}$ as before.\\
&\\
&\parbox{6cm}{\begin{align}
\mbox{If } a &\neq0\\
\mbox{then }x &\neq0\\
\mbox{else }a(a+cx)&=0\\
\implies a&=0\\
\mbox{Thus }a+cx&=0\\
\mbox{Hence }\vec{N}&=\myvec{a&ax\\ \frac{-a}{x}&-a}
\end{align}}\\
 &This is similar to $\myvec{a&a\\-a&-a}$ via $\vec{P}=\myvec{\sqrt{x}&0\\0&\frac{1}{\sqrt{x}}}$.\\
 &And $\myvec{a&a\\-a&-a}$ is similar to $\myvec{0&0\\-a&0}$ via $\vec{P}=\myvec{-1&-1\\1&0}$\\
 &And this finally is similar to $\myvec{0&0\\1&0}$ as before.\\
 &\\
\hline
&\\
Conclusion &Thus either $\vec{N}$ = 0 or $\vec{N}$ is similar over $\mathbb{C}$ to $\myvec{0&0\\1&0}$.\\
&\\
\hline
\caption{Solution summary}
\label{eq:solutions/6/2/11/table:1}
\end{longtable}

\twocolumn
\item %
Let $\vec{A}$ be an $n\times n$ matrix with characteristics polynomial\\
\begin{align}
& f = (x-c_1)^{d_1}....(x-c_k)^{d_k} 
\end{align}
Show that
\begin{align}
& c_1d_1+....+c_kd_k = tr(A)
\end{align}
%
%
\\
\solution
See Table \ref{eq:solutions/6/2/11/table:1}
\onecolumn
\begin{longtable}{|l|l|}
\hline
\multirow{3}{*}{} & \\
Statement&Solution\\
\hline
&\parbox{6cm}{\begin{align}
\mbox{Let }\vec{N}&=\myvec{a&b\\c&d}\\
\mbox{Since }\vec{N}^2&=0
\end{align}}\\
&If $\myvec{a\\c},\myvec{b\\d}$ are linearly independent then $\vec{N}$ is diagonalizable to $\myvec{0&0\\0&0}$.\\
&\parbox{6cm}{\begin{align}
\mbox{If }\vec{P}\vec{N}\vec{P}^{-1}=0\\
\mbox{then }\vec{N}=\vec{P}^{-1}\vec{0}\vec{P}=0
\end{align}}\\
Proof that&So in this case $\vec{N}$ itself is the zero matrix.\\
$\vec{N}=0$&This contradicts the assumption that $\myvec{a\\c},\myvec{b\\d}$ are linearly independent.\\
&$\therefore$ we can assume that $\myvec{a\\c},\myvec{b\\d}$ are linearly dependent if both are\\
&equal to the zero vector\\
&\parbox{6cm}
{\begin{align}
   \mbox{then } \vec{N} &= 0.
\end{align}}\\
%\hline
%\pagebreak
\hline
&\\
&Therefore we can assume at least one vector is non-zero.\\
Assuming $\myvec{b\\d}$ as&Therefore
$\vec{N}=\myvec{a&0\\c&0}$\\
the zero vector&\\
&\parbox{6cm}{\begin{align}
\mbox{So }\vec{N}^2&=0\\
\implies a^2&=0\\
\therefore a&=0\\
\mbox{Thus }\vec{N}&=\myvec{a&0\\c&0}
\end{align}}\\
&In this case $\vec{N}$ is similar to $\vec{N}=\myvec{0&0\\1&0}$ via the matrix $\vec{P}=\myvec{c&0\\0&1}$\\
&\\
\hline
&\\
Assuming $\myvec{a\\c}$ as&Therefore
$\vec{N}=\myvec{0&b\\0&d}$\\
the zero vector&\\
&\parbox{6cm}{\begin{align}
\mbox{Then }\vec{N}^2&=0\\
\implies d^2&=0\\
\therefore d&=0\\
\mbox{Thus }\vec{N}&=\myvec{0&b\\0&0}
\end{align}}\\
&In this case $\vec{N}$ is similar to $\vec{N}=\myvec{0&0\\b&0}$ via the matrix $\vec{P}=\myvec{0&1\\1&0}$,\\
&which is similar to $\myvec{0&0\\1&0}$ as above.\\
&\\
\hline
&\\
Hence& we can assume neither $\myvec{a\\c}$ or $\myvec{b\\d}$ is the zero vector.\\
&\\
%\hline
%\pagebreak
\hline
&\\
Consequences of &Since they are linearly dependent we can assume,\\
linear&\\
independence&\\
&\parbox{6cm}{\begin{align}
\myvec{b\\d}&=x\myvec{a\\c}\\
\therefore \vec{N}&=\myvec{a&ax\\c&cx}\\
\therefore \vec{N}^2&=0\\
\implies a(a+cx)&=0\\
c(a+cx)&=0\\
ax(a+cx)&=0\\
cx(a+cx)&=0
\end{align}}\\
\hline
&\\
Proof that $\vec{N}$ is&We know that at least one of a or c is not zero.\\
similar over $\mathbb{C}$ to&If a = 0 then c $\neq$ 0, it must be that x = 0.\\
$\myvec{0&0\\1&0}$&So in this case $\vec{N}=\myvec{0&0\\c&0}$ which is similar to $\myvec{0&0\\1&0}$ as before.\\
&\\
&\parbox{6cm}{\begin{align}
\mbox{If } a &\neq0\\
\mbox{then }x &\neq0\\
\mbox{else }a(a+cx)&=0\\
\implies a&=0\\
\mbox{Thus }a+cx&=0\\
\mbox{Hence }\vec{N}&=\myvec{a&ax\\ \frac{-a}{x}&-a}
\end{align}}\\
 &This is similar to $\myvec{a&a\\-a&-a}$ via $\vec{P}=\myvec{\sqrt{x}&0\\0&\frac{1}{\sqrt{x}}}$.\\
 &And $\myvec{a&a\\-a&-a}$ is similar to $\myvec{0&0\\-a&0}$ via $\vec{P}=\myvec{-1&-1\\1&0}$\\
 &And this finally is similar to $\myvec{0&0\\1&0}$ as before.\\
 &\\
\hline
&\\
Conclusion &Thus either $\vec{N}$ = 0 or $\vec{N}$ is similar over $\mathbb{C}$ to $\myvec{0&0\\1&0}$.\\
&\\
\hline
\caption{Solution summary}
\label{eq:solutions/6/2/11/table:1}
\end{longtable}

\twocolumn
\item 	Let $\vec{A}$ and $\vec{B}$ be $n\times n$ matrices over the field $\mathbf{F}$. If the matrices $\vec{AB}$ and $\vec{BA}$ have the same characteristic values. Do they have the same characteristic polynomial? Do they have the same minimal polynomial? 
%
\\
\solution
See Table \ref{eq:solutions/6/2/11/table:1}
\onecolumn
\begin{longtable}{|l|l|}
\hline
\multirow{3}{*}{} & \\
Statement&Solution\\
\hline
&\parbox{6cm}{\begin{align}
\mbox{Let }\vec{N}&=\myvec{a&b\\c&d}\\
\mbox{Since }\vec{N}^2&=0
\end{align}}\\
&If $\myvec{a\\c},\myvec{b\\d}$ are linearly independent then $\vec{N}$ is diagonalizable to $\myvec{0&0\\0&0}$.\\
&\parbox{6cm}{\begin{align}
\mbox{If }\vec{P}\vec{N}\vec{P}^{-1}=0\\
\mbox{then }\vec{N}=\vec{P}^{-1}\vec{0}\vec{P}=0
\end{align}}\\
Proof that&So in this case $\vec{N}$ itself is the zero matrix.\\
$\vec{N}=0$&This contradicts the assumption that $\myvec{a\\c},\myvec{b\\d}$ are linearly independent.\\
&$\therefore$ we can assume that $\myvec{a\\c},\myvec{b\\d}$ are linearly dependent if both are\\
&equal to the zero vector\\
&\parbox{6cm}
{\begin{align}
   \mbox{then } \vec{N} &= 0.
\end{align}}\\
%\hline
%\pagebreak
\hline
&\\
&Therefore we can assume at least one vector is non-zero.\\
Assuming $\myvec{b\\d}$ as&Therefore
$\vec{N}=\myvec{a&0\\c&0}$\\
the zero vector&\\
&\parbox{6cm}{\begin{align}
\mbox{So }\vec{N}^2&=0\\
\implies a^2&=0\\
\therefore a&=0\\
\mbox{Thus }\vec{N}&=\myvec{a&0\\c&0}
\end{align}}\\
&In this case $\vec{N}$ is similar to $\vec{N}=\myvec{0&0\\1&0}$ via the matrix $\vec{P}=\myvec{c&0\\0&1}$\\
&\\
\hline
&\\
Assuming $\myvec{a\\c}$ as&Therefore
$\vec{N}=\myvec{0&b\\0&d}$\\
the zero vector&\\
&\parbox{6cm}{\begin{align}
\mbox{Then }\vec{N}^2&=0\\
\implies d^2&=0\\
\therefore d&=0\\
\mbox{Thus }\vec{N}&=\myvec{0&b\\0&0}
\end{align}}\\
&In this case $\vec{N}$ is similar to $\vec{N}=\myvec{0&0\\b&0}$ via the matrix $\vec{P}=\myvec{0&1\\1&0}$,\\
&which is similar to $\myvec{0&0\\1&0}$ as above.\\
&\\
\hline
&\\
Hence& we can assume neither $\myvec{a\\c}$ or $\myvec{b\\d}$ is the zero vector.\\
&\\
%\hline
%\pagebreak
\hline
&\\
Consequences of &Since they are linearly dependent we can assume,\\
linear&\\
independence&\\
&\parbox{6cm}{\begin{align}
\myvec{b\\d}&=x\myvec{a\\c}\\
\therefore \vec{N}&=\myvec{a&ax\\c&cx}\\
\therefore \vec{N}^2&=0\\
\implies a(a+cx)&=0\\
c(a+cx)&=0\\
ax(a+cx)&=0\\
cx(a+cx)&=0
\end{align}}\\
\hline
&\\
Proof that $\vec{N}$ is&We know that at least one of a or c is not zero.\\
similar over $\mathbb{C}$ to&If a = 0 then c $\neq$ 0, it must be that x = 0.\\
$\myvec{0&0\\1&0}$&So in this case $\vec{N}=\myvec{0&0\\c&0}$ which is similar to $\myvec{0&0\\1&0}$ as before.\\
&\\
&\parbox{6cm}{\begin{align}
\mbox{If } a &\neq0\\
\mbox{then }x &\neq0\\
\mbox{else }a(a+cx)&=0\\
\implies a&=0\\
\mbox{Thus }a+cx&=0\\
\mbox{Hence }\vec{N}&=\myvec{a&ax\\ \frac{-a}{x}&-a}
\end{align}}\\
 &This is similar to $\myvec{a&a\\-a&-a}$ via $\vec{P}=\myvec{\sqrt{x}&0\\0&\frac{1}{\sqrt{x}}}$.\\
 &And $\myvec{a&a\\-a&-a}$ is similar to $\myvec{0&0\\-a&0}$ via $\vec{P}=\myvec{-1&-1\\1&0}$\\
 &And this finally is similar to $\myvec{0&0\\1&0}$ as before.\\
 &\\
\hline
&\\
Conclusion &Thus either $\vec{N}$ = 0 or $\vec{N}$ is similar over $\mathbb{C}$ to $\myvec{0&0\\1&0}$.\\
&\\
\hline
\caption{Solution summary}
\label{eq:solutions/6/2/11/table:1}
\end{longtable}

%\twocolumn
\end{enumerate}

\subsection{Invariant Subspaces}
\renewcommand{\theequation}{\theenumi}
\renewcommand{\thefigure}{\theenumi}
\begin{enumerate}[label=\thesubsection.\arabic*.,ref=\thesubsection.\theenumi]
\numberwithin{equation}{enumi}
\numberwithin{figure}{enumi}

\item %
Let
\begin{align}
    \Vec{A}=\myvec{0&1&0\\2&-2&2\\2&-3&2}
\end{align}
Is $\vec{A}$ similar over the field of real numbers to a triangular matrix? If so, find such a triangular matrix.
\\
\solution
See Table \ref{eq:solutions/6/2/11/table:1}
\onecolumn
\begin{longtable}{|l|l|}
\hline
\multirow{3}{*}{} & \\
Statement&Solution\\
\hline
&\parbox{6cm}{\begin{align}
\mbox{Let }\vec{N}&=\myvec{a&b\\c&d}\\
\mbox{Since }\vec{N}^2&=0
\end{align}}\\
&If $\myvec{a\\c},\myvec{b\\d}$ are linearly independent then $\vec{N}$ is diagonalizable to $\myvec{0&0\\0&0}$.\\
&\parbox{6cm}{\begin{align}
\mbox{If }\vec{P}\vec{N}\vec{P}^{-1}=0\\
\mbox{then }\vec{N}=\vec{P}^{-1}\vec{0}\vec{P}=0
\end{align}}\\
Proof that&So in this case $\vec{N}$ itself is the zero matrix.\\
$\vec{N}=0$&This contradicts the assumption that $\myvec{a\\c},\myvec{b\\d}$ are linearly independent.\\
&$\therefore$ we can assume that $\myvec{a\\c},\myvec{b\\d}$ are linearly dependent if both are\\
&equal to the zero vector\\
&\parbox{6cm}
{\begin{align}
   \mbox{then } \vec{N} &= 0.
\end{align}}\\
%\hline
%\pagebreak
\hline
&\\
&Therefore we can assume at least one vector is non-zero.\\
Assuming $\myvec{b\\d}$ as&Therefore
$\vec{N}=\myvec{a&0\\c&0}$\\
the zero vector&\\
&\parbox{6cm}{\begin{align}
\mbox{So }\vec{N}^2&=0\\
\implies a^2&=0\\
\therefore a&=0\\
\mbox{Thus }\vec{N}&=\myvec{a&0\\c&0}
\end{align}}\\
&In this case $\vec{N}$ is similar to $\vec{N}=\myvec{0&0\\1&0}$ via the matrix $\vec{P}=\myvec{c&0\\0&1}$\\
&\\
\hline
&\\
Assuming $\myvec{a\\c}$ as&Therefore
$\vec{N}=\myvec{0&b\\0&d}$\\
the zero vector&\\
&\parbox{6cm}{\begin{align}
\mbox{Then }\vec{N}^2&=0\\
\implies d^2&=0\\
\therefore d&=0\\
\mbox{Thus }\vec{N}&=\myvec{0&b\\0&0}
\end{align}}\\
&In this case $\vec{N}$ is similar to $\vec{N}=\myvec{0&0\\b&0}$ via the matrix $\vec{P}=\myvec{0&1\\1&0}$,\\
&which is similar to $\myvec{0&0\\1&0}$ as above.\\
&\\
\hline
&\\
Hence& we can assume neither $\myvec{a\\c}$ or $\myvec{b\\d}$ is the zero vector.\\
&\\
%\hline
%\pagebreak
\hline
&\\
Consequences of &Since they are linearly dependent we can assume,\\
linear&\\
independence&\\
&\parbox{6cm}{\begin{align}
\myvec{b\\d}&=x\myvec{a\\c}\\
\therefore \vec{N}&=\myvec{a&ax\\c&cx}\\
\therefore \vec{N}^2&=0\\
\implies a(a+cx)&=0\\
c(a+cx)&=0\\
ax(a+cx)&=0\\
cx(a+cx)&=0
\end{align}}\\
\hline
&\\
Proof that $\vec{N}$ is&We know that at least one of a or c is not zero.\\
similar over $\mathbb{C}$ to&If a = 0 then c $\neq$ 0, it must be that x = 0.\\
$\myvec{0&0\\1&0}$&So in this case $\vec{N}=\myvec{0&0\\c&0}$ which is similar to $\myvec{0&0\\1&0}$ as before.\\
&\\
&\parbox{6cm}{\begin{align}
\mbox{If } a &\neq0\\
\mbox{then }x &\neq0\\
\mbox{else }a(a+cx)&=0\\
\implies a&=0\\
\mbox{Thus }a+cx&=0\\
\mbox{Hence }\vec{N}&=\myvec{a&ax\\ \frac{-a}{x}&-a}
\end{align}}\\
 &This is similar to $\myvec{a&a\\-a&-a}$ via $\vec{P}=\myvec{\sqrt{x}&0\\0&\frac{1}{\sqrt{x}}}$.\\
 &And $\myvec{a&a\\-a&-a}$ is similar to $\myvec{0&0\\-a&0}$ via $\vec{P}=\myvec{-1&-1\\1&0}$\\
 &And this finally is similar to $\myvec{0&0\\1&0}$ as before.\\
 &\\
\hline
&\\
Conclusion &Thus either $\vec{N}$ = 0 or $\vec{N}$ is similar over $\mathbb{C}$ to $\myvec{0&0\\1&0}$.\\
&\\
\hline
\caption{Solution summary}
\label{eq:solutions/6/2/11/table:1}
\end{longtable}

\twocolumn


\item  Every matrix $A$ such that $A^2$ = $A$ is similar to a diagonal matrix.
%
\\
\solution
See Table \ref{eq:solutions/6/2/11/table:1}
\onecolumn
\begin{longtable}{|l|l|}
\hline
\multirow{3}{*}{} & \\
Statement&Solution\\
\hline
&\parbox{6cm}{\begin{align}
\mbox{Let }\vec{N}&=\myvec{a&b\\c&d}\\
\mbox{Since }\vec{N}^2&=0
\end{align}}\\
&If $\myvec{a\\c},\myvec{b\\d}$ are linearly independent then $\vec{N}$ is diagonalizable to $\myvec{0&0\\0&0}$.\\
&\parbox{6cm}{\begin{align}
\mbox{If }\vec{P}\vec{N}\vec{P}^{-1}=0\\
\mbox{then }\vec{N}=\vec{P}^{-1}\vec{0}\vec{P}=0
\end{align}}\\
Proof that&So in this case $\vec{N}$ itself is the zero matrix.\\
$\vec{N}=0$&This contradicts the assumption that $\myvec{a\\c},\myvec{b\\d}$ are linearly independent.\\
&$\therefore$ we can assume that $\myvec{a\\c},\myvec{b\\d}$ are linearly dependent if both are\\
&equal to the zero vector\\
&\parbox{6cm}
{\begin{align}
   \mbox{then } \vec{N} &= 0.
\end{align}}\\
%\hline
%\pagebreak
\hline
&\\
&Therefore we can assume at least one vector is non-zero.\\
Assuming $\myvec{b\\d}$ as&Therefore
$\vec{N}=\myvec{a&0\\c&0}$\\
the zero vector&\\
&\parbox{6cm}{\begin{align}
\mbox{So }\vec{N}^2&=0\\
\implies a^2&=0\\
\therefore a&=0\\
\mbox{Thus }\vec{N}&=\myvec{a&0\\c&0}
\end{align}}\\
&In this case $\vec{N}$ is similar to $\vec{N}=\myvec{0&0\\1&0}$ via the matrix $\vec{P}=\myvec{c&0\\0&1}$\\
&\\
\hline
&\\
Assuming $\myvec{a\\c}$ as&Therefore
$\vec{N}=\myvec{0&b\\0&d}$\\
the zero vector&\\
&\parbox{6cm}{\begin{align}
\mbox{Then }\vec{N}^2&=0\\
\implies d^2&=0\\
\therefore d&=0\\
\mbox{Thus }\vec{N}&=\myvec{0&b\\0&0}
\end{align}}\\
&In this case $\vec{N}$ is similar to $\vec{N}=\myvec{0&0\\b&0}$ via the matrix $\vec{P}=\myvec{0&1\\1&0}$,\\
&which is similar to $\myvec{0&0\\1&0}$ as above.\\
&\\
\hline
&\\
Hence& we can assume neither $\myvec{a\\c}$ or $\myvec{b\\d}$ is the zero vector.\\
&\\
%\hline
%\pagebreak
\hline
&\\
Consequences of &Since they are linearly dependent we can assume,\\
linear&\\
independence&\\
&\parbox{6cm}{\begin{align}
\myvec{b\\d}&=x\myvec{a\\c}\\
\therefore \vec{N}&=\myvec{a&ax\\c&cx}\\
\therefore \vec{N}^2&=0\\
\implies a(a+cx)&=0\\
c(a+cx)&=0\\
ax(a+cx)&=0\\
cx(a+cx)&=0
\end{align}}\\
\hline
&\\
Proof that $\vec{N}$ is&We know that at least one of a or c is not zero.\\
similar over $\mathbb{C}$ to&If a = 0 then c $\neq$ 0, it must be that x = 0.\\
$\myvec{0&0\\1&0}$&So in this case $\vec{N}=\myvec{0&0\\c&0}$ which is similar to $\myvec{0&0\\1&0}$ as before.\\
&\\
&\parbox{6cm}{\begin{align}
\mbox{If } a &\neq0\\
\mbox{then }x &\neq0\\
\mbox{else }a(a+cx)&=0\\
\implies a&=0\\
\mbox{Thus }a+cx&=0\\
\mbox{Hence }\vec{N}&=\myvec{a&ax\\ \frac{-a}{x}&-a}
\end{align}}\\
 &This is similar to $\myvec{a&a\\-a&-a}$ via $\vec{P}=\myvec{\sqrt{x}&0\\0&\frac{1}{\sqrt{x}}}$.\\
 &And $\myvec{a&a\\-a&-a}$ is similar to $\myvec{0&0\\-a&0}$ via $\vec{P}=\myvec{-1&-1\\1&0}$\\
 &And this finally is similar to $\myvec{0&0\\1&0}$ as before.\\
 &\\
\hline
&\\
Conclusion &Thus either $\vec{N}$ = 0 or $\vec{N}$ is similar over $\mathbb{C}$ to $\myvec{0&0\\1&0}$.\\
&\\
\hline
\caption{Solution summary}
\label{eq:solutions/6/2/11/table:1}
\end{longtable}

\twocolumn
\item True or false? If the triangular matrix $\vec{A}$ is similar to a diagonal matrix, then $\vec{A}$ is already diagonal.
%
\\
\solution
See Table \ref{eq:solutions/6/2/11/table:1}
\onecolumn
\begin{longtable}{|l|l|}
\hline
\multirow{3}{*}{} & \\
Statement&Solution\\
\hline
&\parbox{6cm}{\begin{align}
\mbox{Let }\vec{N}&=\myvec{a&b\\c&d}\\
\mbox{Since }\vec{N}^2&=0
\end{align}}\\
&If $\myvec{a\\c},\myvec{b\\d}$ are linearly independent then $\vec{N}$ is diagonalizable to $\myvec{0&0\\0&0}$.\\
&\parbox{6cm}{\begin{align}
\mbox{If }\vec{P}\vec{N}\vec{P}^{-1}=0\\
\mbox{then }\vec{N}=\vec{P}^{-1}\vec{0}\vec{P}=0
\end{align}}\\
Proof that&So in this case $\vec{N}$ itself is the zero matrix.\\
$\vec{N}=0$&This contradicts the assumption that $\myvec{a\\c},\myvec{b\\d}$ are linearly independent.\\
&$\therefore$ we can assume that $\myvec{a\\c},\myvec{b\\d}$ are linearly dependent if both are\\
&equal to the zero vector\\
&\parbox{6cm}
{\begin{align}
   \mbox{then } \vec{N} &= 0.
\end{align}}\\
%\hline
%\pagebreak
\hline
&\\
&Therefore we can assume at least one vector is non-zero.\\
Assuming $\myvec{b\\d}$ as&Therefore
$\vec{N}=\myvec{a&0\\c&0}$\\
the zero vector&\\
&\parbox{6cm}{\begin{align}
\mbox{So }\vec{N}^2&=0\\
\implies a^2&=0\\
\therefore a&=0\\
\mbox{Thus }\vec{N}&=\myvec{a&0\\c&0}
\end{align}}\\
&In this case $\vec{N}$ is similar to $\vec{N}=\myvec{0&0\\1&0}$ via the matrix $\vec{P}=\myvec{c&0\\0&1}$\\
&\\
\hline
&\\
Assuming $\myvec{a\\c}$ as&Therefore
$\vec{N}=\myvec{0&b\\0&d}$\\
the zero vector&\\
&\parbox{6cm}{\begin{align}
\mbox{Then }\vec{N}^2&=0\\
\implies d^2&=0\\
\therefore d&=0\\
\mbox{Thus }\vec{N}&=\myvec{0&b\\0&0}
\end{align}}\\
&In this case $\vec{N}$ is similar to $\vec{N}=\myvec{0&0\\b&0}$ via the matrix $\vec{P}=\myvec{0&1\\1&0}$,\\
&which is similar to $\myvec{0&0\\1&0}$ as above.\\
&\\
\hline
&\\
Hence& we can assume neither $\myvec{a\\c}$ or $\myvec{b\\d}$ is the zero vector.\\
&\\
%\hline
%\pagebreak
\hline
&\\
Consequences of &Since they are linearly dependent we can assume,\\
linear&\\
independence&\\
&\parbox{6cm}{\begin{align}
\myvec{b\\d}&=x\myvec{a\\c}\\
\therefore \vec{N}&=\myvec{a&ax\\c&cx}\\
\therefore \vec{N}^2&=0\\
\implies a(a+cx)&=0\\
c(a+cx)&=0\\
ax(a+cx)&=0\\
cx(a+cx)&=0
\end{align}}\\
\hline
&\\
Proof that $\vec{N}$ is&We know that at least one of a or c is not zero.\\
similar over $\mathbb{C}$ to&If a = 0 then c $\neq$ 0, it must be that x = 0.\\
$\myvec{0&0\\1&0}$&So in this case $\vec{N}=\myvec{0&0\\c&0}$ which is similar to $\myvec{0&0\\1&0}$ as before.\\
&\\
&\parbox{6cm}{\begin{align}
\mbox{If } a &\neq0\\
\mbox{then }x &\neq0\\
\mbox{else }a(a+cx)&=0\\
\implies a&=0\\
\mbox{Thus }a+cx&=0\\
\mbox{Hence }\vec{N}&=\myvec{a&ax\\ \frac{-a}{x}&-a}
\end{align}}\\
 &This is similar to $\myvec{a&a\\-a&-a}$ via $\vec{P}=\myvec{\sqrt{x}&0\\0&\frac{1}{\sqrt{x}}}$.\\
 &And $\myvec{a&a\\-a&-a}$ is similar to $\myvec{0&0\\-a&0}$ via $\vec{P}=\myvec{-1&-1\\1&0}$\\
 &And this finally is similar to $\myvec{0&0\\1&0}$ as before.\\
 &\\
\hline
&\\
Conclusion &Thus either $\vec{N}$ = 0 or $\vec{N}$ is similar over $\mathbb{C}$ to $\myvec{0&0\\1&0}$.\\
&\\
\hline
\caption{Solution summary}
\label{eq:solutions/6/2/11/table:1}
\end{longtable}

\end{enumerate}

\subsection{Simultaneous Triangulation; Simultaneous Diagonalization}
\renewcommand{\theequation}{\theenumi}
\renewcommand{\thefigure}{\theenumi}
\begin{enumerate}[label=\thesubsection.\arabic*.,ref=\thesubsection.\theenumi]
\numberwithin{equation}{enumi}
\numberwithin{figure}{enumi}

\item Find an invertible matrix $\vec{P}$ such that $\vec{P^{-1}AP}$ and $\vec{P^{-1}BP}$ are both diagonal where $\vec{A}$ and $\vec{B}$ are real matrices.
%
\begin{enumerate}
\item 
\begin{align}
\Vec{A}=\myvec{1&2\\0&2} , \Vec{B}=\myvec{3&-8\\0&-1}
    \end{align}
\solution
See Table \ref{eq:solutions/6/2/11/table:1}
\onecolumn
\begin{longtable}{|l|l|}
\hline
\multirow{3}{*}{} & \\
Statement&Solution\\
\hline
&\parbox{6cm}{\begin{align}
\mbox{Let }\vec{N}&=\myvec{a&b\\c&d}\\
\mbox{Since }\vec{N}^2&=0
\end{align}}\\
&If $\myvec{a\\c},\myvec{b\\d}$ are linearly independent then $\vec{N}$ is diagonalizable to $\myvec{0&0\\0&0}$.\\
&\parbox{6cm}{\begin{align}
\mbox{If }\vec{P}\vec{N}\vec{P}^{-1}=0\\
\mbox{then }\vec{N}=\vec{P}^{-1}\vec{0}\vec{P}=0
\end{align}}\\
Proof that&So in this case $\vec{N}$ itself is the zero matrix.\\
$\vec{N}=0$&This contradicts the assumption that $\myvec{a\\c},\myvec{b\\d}$ are linearly independent.\\
&$\therefore$ we can assume that $\myvec{a\\c},\myvec{b\\d}$ are linearly dependent if both are\\
&equal to the zero vector\\
&\parbox{6cm}
{\begin{align}
   \mbox{then } \vec{N} &= 0.
\end{align}}\\
%\hline
%\pagebreak
\hline
&\\
&Therefore we can assume at least one vector is non-zero.\\
Assuming $\myvec{b\\d}$ as&Therefore
$\vec{N}=\myvec{a&0\\c&0}$\\
the zero vector&\\
&\parbox{6cm}{\begin{align}
\mbox{So }\vec{N}^2&=0\\
\implies a^2&=0\\
\therefore a&=0\\
\mbox{Thus }\vec{N}&=\myvec{a&0\\c&0}
\end{align}}\\
&In this case $\vec{N}$ is similar to $\vec{N}=\myvec{0&0\\1&0}$ via the matrix $\vec{P}=\myvec{c&0\\0&1}$\\
&\\
\hline
&\\
Assuming $\myvec{a\\c}$ as&Therefore
$\vec{N}=\myvec{0&b\\0&d}$\\
the zero vector&\\
&\parbox{6cm}{\begin{align}
\mbox{Then }\vec{N}^2&=0\\
\implies d^2&=0\\
\therefore d&=0\\
\mbox{Thus }\vec{N}&=\myvec{0&b\\0&0}
\end{align}}\\
&In this case $\vec{N}$ is similar to $\vec{N}=\myvec{0&0\\b&0}$ via the matrix $\vec{P}=\myvec{0&1\\1&0}$,\\
&which is similar to $\myvec{0&0\\1&0}$ as above.\\
&\\
\hline
&\\
Hence& we can assume neither $\myvec{a\\c}$ or $\myvec{b\\d}$ is the zero vector.\\
&\\
%\hline
%\pagebreak
\hline
&\\
Consequences of &Since they are linearly dependent we can assume,\\
linear&\\
independence&\\
&\parbox{6cm}{\begin{align}
\myvec{b\\d}&=x\myvec{a\\c}\\
\therefore \vec{N}&=\myvec{a&ax\\c&cx}\\
\therefore \vec{N}^2&=0\\
\implies a(a+cx)&=0\\
c(a+cx)&=0\\
ax(a+cx)&=0\\
cx(a+cx)&=0
\end{align}}\\
\hline
&\\
Proof that $\vec{N}$ is&We know that at least one of a or c is not zero.\\
similar over $\mathbb{C}$ to&If a = 0 then c $\neq$ 0, it must be that x = 0.\\
$\myvec{0&0\\1&0}$&So in this case $\vec{N}=\myvec{0&0\\c&0}$ which is similar to $\myvec{0&0\\1&0}$ as before.\\
&\\
&\parbox{6cm}{\begin{align}
\mbox{If } a &\neq0\\
\mbox{then }x &\neq0\\
\mbox{else }a(a+cx)&=0\\
\implies a&=0\\
\mbox{Thus }a+cx&=0\\
\mbox{Hence }\vec{N}&=\myvec{a&ax\\ \frac{-a}{x}&-a}
\end{align}}\\
 &This is similar to $\myvec{a&a\\-a&-a}$ via $\vec{P}=\myvec{\sqrt{x}&0\\0&\frac{1}{\sqrt{x}}}$.\\
 &And $\myvec{a&a\\-a&-a}$ is similar to $\myvec{0&0\\-a&0}$ via $\vec{P}=\myvec{-1&-1\\1&0}$\\
 &And this finally is similar to $\myvec{0&0\\1&0}$ as before.\\
 &\\
\hline
&\\
Conclusion &Thus either $\vec{N}$ = 0 or $\vec{N}$ is similar over $\mathbb{C}$ to $\myvec{0&0\\1&0}$.\\
&\\
\hline
\caption{Solution summary}
\label{eq:solutions/6/2/11/table:1}
\end{longtable}

\twocolumn
\item %
\begin{align}
    \Vec{A}=\myvec{1&1\\1&1}\\ 
    \vec{B}=\myvec{1&a\\a&1}
\end{align}
%
\\
\solution
See Table \ref{eq:solutions/6/2/11/table:1}
\onecolumn
\begin{longtable}{|l|l|}
\hline
\multirow{3}{*}{} & \\
Statement&Solution\\
\hline
&\parbox{6cm}{\begin{align}
\mbox{Let }\vec{N}&=\myvec{a&b\\c&d}\\
\mbox{Since }\vec{N}^2&=0
\end{align}}\\
&If $\myvec{a\\c},\myvec{b\\d}$ are linearly independent then $\vec{N}$ is diagonalizable to $\myvec{0&0\\0&0}$.\\
&\parbox{6cm}{\begin{align}
\mbox{If }\vec{P}\vec{N}\vec{P}^{-1}=0\\
\mbox{then }\vec{N}=\vec{P}^{-1}\vec{0}\vec{P}=0
\end{align}}\\
Proof that&So in this case $\vec{N}$ itself is the zero matrix.\\
$\vec{N}=0$&This contradicts the assumption that $\myvec{a\\c},\myvec{b\\d}$ are linearly independent.\\
&$\therefore$ we can assume that $\myvec{a\\c},\myvec{b\\d}$ are linearly dependent if both are\\
&equal to the zero vector\\
&\parbox{6cm}
{\begin{align}
   \mbox{then } \vec{N} &= 0.
\end{align}}\\
%\hline
%\pagebreak
\hline
&\\
&Therefore we can assume at least one vector is non-zero.\\
Assuming $\myvec{b\\d}$ as&Therefore
$\vec{N}=\myvec{a&0\\c&0}$\\
the zero vector&\\
&\parbox{6cm}{\begin{align}
\mbox{So }\vec{N}^2&=0\\
\implies a^2&=0\\
\therefore a&=0\\
\mbox{Thus }\vec{N}&=\myvec{a&0\\c&0}
\end{align}}\\
&In this case $\vec{N}$ is similar to $\vec{N}=\myvec{0&0\\1&0}$ via the matrix $\vec{P}=\myvec{c&0\\0&1}$\\
&\\
\hline
&\\
Assuming $\myvec{a\\c}$ as&Therefore
$\vec{N}=\myvec{0&b\\0&d}$\\
the zero vector&\\
&\parbox{6cm}{\begin{align}
\mbox{Then }\vec{N}^2&=0\\
\implies d^2&=0\\
\therefore d&=0\\
\mbox{Thus }\vec{N}&=\myvec{0&b\\0&0}
\end{align}}\\
&In this case $\vec{N}$ is similar to $\vec{N}=\myvec{0&0\\b&0}$ via the matrix $\vec{P}=\myvec{0&1\\1&0}$,\\
&which is similar to $\myvec{0&0\\1&0}$ as above.\\
&\\
\hline
&\\
Hence& we can assume neither $\myvec{a\\c}$ or $\myvec{b\\d}$ is the zero vector.\\
&\\
%\hline
%\pagebreak
\hline
&\\
Consequences of &Since they are linearly dependent we can assume,\\
linear&\\
independence&\\
&\parbox{6cm}{\begin{align}
\myvec{b\\d}&=x\myvec{a\\c}\\
\therefore \vec{N}&=\myvec{a&ax\\c&cx}\\
\therefore \vec{N}^2&=0\\
\implies a(a+cx)&=0\\
c(a+cx)&=0\\
ax(a+cx)&=0\\
cx(a+cx)&=0
\end{align}}\\
\hline
&\\
Proof that $\vec{N}$ is&We know that at least one of a or c is not zero.\\
similar over $\mathbb{C}$ to&If a = 0 then c $\neq$ 0, it must be that x = 0.\\
$\myvec{0&0\\1&0}$&So in this case $\vec{N}=\myvec{0&0\\c&0}$ which is similar to $\myvec{0&0\\1&0}$ as before.\\
&\\
&\parbox{6cm}{\begin{align}
\mbox{If } a &\neq0\\
\mbox{then }x &\neq0\\
\mbox{else }a(a+cx)&=0\\
\implies a&=0\\
\mbox{Thus }a+cx&=0\\
\mbox{Hence }\vec{N}&=\myvec{a&ax\\ \frac{-a}{x}&-a}
\end{align}}\\
 &This is similar to $\myvec{a&a\\-a&-a}$ via $\vec{P}=\myvec{\sqrt{x}&0\\0&\frac{1}{\sqrt{x}}}$.\\
 &And $\myvec{a&a\\-a&-a}$ is similar to $\myvec{0&0\\-a&0}$ via $\vec{P}=\myvec{-1&-1\\1&0}$\\
 &And this finally is similar to $\myvec{0&0\\1&0}$ as before.\\
 &\\
\hline
&\\
Conclusion &Thus either $\vec{N}$ = 0 or $\vec{N}$ is similar over $\mathbb{C}$ to $\myvec{0&0\\1&0}$.\\
&\\
\hline
\caption{Solution summary}
\label{eq:solutions/6/2/11/table:1}
\end{longtable}

\twocolumn
\end{enumerate}
\item %
Let $\vec{A}$,$\vec{B}$,$\vec{C}$,$\vec{D}$ be $n\times n$ complex matrices which commute. Let $\vec{E}$ be the $2n\times 2n$ matrix
\begin{align}
    \vec{E}=\myvec{\vec{A} & \vec{B}\\\vec{C} & \vec{D}}\label{eq:solutions/6/5/4/E}
\end{align}
prove that $\det(\vec{E})=\det(\vec{A}\vec{D}-\vec{B}\vec{C})$
%
\\
\solution
See Table \ref{eq:solutions/6/2/11/table:1}
\onecolumn
\begin{longtable}{|l|l|}
\hline
\multirow{3}{*}{} & \\
Statement&Solution\\
\hline
&\parbox{6cm}{\begin{align}
\mbox{Let }\vec{N}&=\myvec{a&b\\c&d}\\
\mbox{Since }\vec{N}^2&=0
\end{align}}\\
&If $\myvec{a\\c},\myvec{b\\d}$ are linearly independent then $\vec{N}$ is diagonalizable to $\myvec{0&0\\0&0}$.\\
&\parbox{6cm}{\begin{align}
\mbox{If }\vec{P}\vec{N}\vec{P}^{-1}=0\\
\mbox{then }\vec{N}=\vec{P}^{-1}\vec{0}\vec{P}=0
\end{align}}\\
Proof that&So in this case $\vec{N}$ itself is the zero matrix.\\
$\vec{N}=0$&This contradicts the assumption that $\myvec{a\\c},\myvec{b\\d}$ are linearly independent.\\
&$\therefore$ we can assume that $\myvec{a\\c},\myvec{b\\d}$ are linearly dependent if both are\\
&equal to the zero vector\\
&\parbox{6cm}
{\begin{align}
   \mbox{then } \vec{N} &= 0.
\end{align}}\\
%\hline
%\pagebreak
\hline
&\\
&Therefore we can assume at least one vector is non-zero.\\
Assuming $\myvec{b\\d}$ as&Therefore
$\vec{N}=\myvec{a&0\\c&0}$\\
the zero vector&\\
&\parbox{6cm}{\begin{align}
\mbox{So }\vec{N}^2&=0\\
\implies a^2&=0\\
\therefore a&=0\\
\mbox{Thus }\vec{N}&=\myvec{a&0\\c&0}
\end{align}}\\
&In this case $\vec{N}$ is similar to $\vec{N}=\myvec{0&0\\1&0}$ via the matrix $\vec{P}=\myvec{c&0\\0&1}$\\
&\\
\hline
&\\
Assuming $\myvec{a\\c}$ as&Therefore
$\vec{N}=\myvec{0&b\\0&d}$\\
the zero vector&\\
&\parbox{6cm}{\begin{align}
\mbox{Then }\vec{N}^2&=0\\
\implies d^2&=0\\
\therefore d&=0\\
\mbox{Thus }\vec{N}&=\myvec{0&b\\0&0}
\end{align}}\\
&In this case $\vec{N}$ is similar to $\vec{N}=\myvec{0&0\\b&0}$ via the matrix $\vec{P}=\myvec{0&1\\1&0}$,\\
&which is similar to $\myvec{0&0\\1&0}$ as above.\\
&\\
\hline
&\\
Hence& we can assume neither $\myvec{a\\c}$ or $\myvec{b\\d}$ is the zero vector.\\
&\\
%\hline
%\pagebreak
\hline
&\\
Consequences of &Since they are linearly dependent we can assume,\\
linear&\\
independence&\\
&\parbox{6cm}{\begin{align}
\myvec{b\\d}&=x\myvec{a\\c}\\
\therefore \vec{N}&=\myvec{a&ax\\c&cx}\\
\therefore \vec{N}^2&=0\\
\implies a(a+cx)&=0\\
c(a+cx)&=0\\
ax(a+cx)&=0\\
cx(a+cx)&=0
\end{align}}\\
\hline
&\\
Proof that $\vec{N}$ is&We know that at least one of a or c is not zero.\\
similar over $\mathbb{C}$ to&If a = 0 then c $\neq$ 0, it must be that x = 0.\\
$\myvec{0&0\\1&0}$&So in this case $\vec{N}=\myvec{0&0\\c&0}$ which is similar to $\myvec{0&0\\1&0}$ as before.\\
&\\
&\parbox{6cm}{\begin{align}
\mbox{If } a &\neq0\\
\mbox{then }x &\neq0\\
\mbox{else }a(a+cx)&=0\\
\implies a&=0\\
\mbox{Thus }a+cx&=0\\
\mbox{Hence }\vec{N}&=\myvec{a&ax\\ \frac{-a}{x}&-a}
\end{align}}\\
 &This is similar to $\myvec{a&a\\-a&-a}$ via $\vec{P}=\myvec{\sqrt{x}&0\\0&\frac{1}{\sqrt{x}}}$.\\
 &And $\myvec{a&a\\-a&-a}$ is similar to $\myvec{0&0\\-a&0}$ via $\vec{P}=\myvec{-1&-1\\1&0}$\\
 &And this finally is similar to $\myvec{0&0\\1&0}$ as before.\\
 &\\
\hline
&\\
Conclusion &Thus either $\vec{N}$ = 0 or $\vec{N}$ is similar over $\mathbb{C}$ to $\myvec{0&0\\1&0}$.\\
&\\
\hline
\caption{Solution summary}
\label{eq:solutions/6/2/11/table:1}
\end{longtable}

\end{enumerate}

\subsection{Direct Sum Decomposition}
\renewcommand{\theequation}{\theenumi}
\renewcommand{\thefigure}{\theenumi}
\begin{enumerate}[label=\thesubsection.\arabic*.,ref=\thesubsection.\theenumi]
\numberwithin{equation}{enumi}
\numberwithin{figure}{enumi}

\item Let $\mathbb{V}$ be a finite-dimensional vector space and let $\mathbb{W}_1$ be any subspace of $\mathbb{V}$.Prove that there is a subspace $\mathbb{W}_2$ of $\mathbb{V}$ such that $\mathbb{V}$= $\mathbb{W}_1$ $\oplus$ $\mathbb{W}_2$
%
\\
\solution
See Table \ref{eq:solutions/6/2/11/table:1}
\onecolumn
\begin{longtable}{|l|l|}
\hline
\multirow{3}{*}{} & \\
Statement&Solution\\
\hline
&\parbox{6cm}{\begin{align}
\mbox{Let }\vec{N}&=\myvec{a&b\\c&d}\\
\mbox{Since }\vec{N}^2&=0
\end{align}}\\
&If $\myvec{a\\c},\myvec{b\\d}$ are linearly independent then $\vec{N}$ is diagonalizable to $\myvec{0&0\\0&0}$.\\
&\parbox{6cm}{\begin{align}
\mbox{If }\vec{P}\vec{N}\vec{P}^{-1}=0\\
\mbox{then }\vec{N}=\vec{P}^{-1}\vec{0}\vec{P}=0
\end{align}}\\
Proof that&So in this case $\vec{N}$ itself is the zero matrix.\\
$\vec{N}=0$&This contradicts the assumption that $\myvec{a\\c},\myvec{b\\d}$ are linearly independent.\\
&$\therefore$ we can assume that $\myvec{a\\c},\myvec{b\\d}$ are linearly dependent if both are\\
&equal to the zero vector\\
&\parbox{6cm}
{\begin{align}
   \mbox{then } \vec{N} &= 0.
\end{align}}\\
%\hline
%\pagebreak
\hline
&\\
&Therefore we can assume at least one vector is non-zero.\\
Assuming $\myvec{b\\d}$ as&Therefore
$\vec{N}=\myvec{a&0\\c&0}$\\
the zero vector&\\
&\parbox{6cm}{\begin{align}
\mbox{So }\vec{N}^2&=0\\
\implies a^2&=0\\
\therefore a&=0\\
\mbox{Thus }\vec{N}&=\myvec{a&0\\c&0}
\end{align}}\\
&In this case $\vec{N}$ is similar to $\vec{N}=\myvec{0&0\\1&0}$ via the matrix $\vec{P}=\myvec{c&0\\0&1}$\\
&\\
\hline
&\\
Assuming $\myvec{a\\c}$ as&Therefore
$\vec{N}=\myvec{0&b\\0&d}$\\
the zero vector&\\
&\parbox{6cm}{\begin{align}
\mbox{Then }\vec{N}^2&=0\\
\implies d^2&=0\\
\therefore d&=0\\
\mbox{Thus }\vec{N}&=\myvec{0&b\\0&0}
\end{align}}\\
&In this case $\vec{N}$ is similar to $\vec{N}=\myvec{0&0\\b&0}$ via the matrix $\vec{P}=\myvec{0&1\\1&0}$,\\
&which is similar to $\myvec{0&0\\1&0}$ as above.\\
&\\
\hline
&\\
Hence& we can assume neither $\myvec{a\\c}$ or $\myvec{b\\d}$ is the zero vector.\\
&\\
%\hline
%\pagebreak
\hline
&\\
Consequences of &Since they are linearly dependent we can assume,\\
linear&\\
independence&\\
&\parbox{6cm}{\begin{align}
\myvec{b\\d}&=x\myvec{a\\c}\\
\therefore \vec{N}&=\myvec{a&ax\\c&cx}\\
\therefore \vec{N}^2&=0\\
\implies a(a+cx)&=0\\
c(a+cx)&=0\\
ax(a+cx)&=0\\
cx(a+cx)&=0
\end{align}}\\
\hline
&\\
Proof that $\vec{N}$ is&We know that at least one of a or c is not zero.\\
similar over $\mathbb{C}$ to&If a = 0 then c $\neq$ 0, it must be that x = 0.\\
$\myvec{0&0\\1&0}$&So in this case $\vec{N}=\myvec{0&0\\c&0}$ which is similar to $\myvec{0&0\\1&0}$ as before.\\
&\\
&\parbox{6cm}{\begin{align}
\mbox{If } a &\neq0\\
\mbox{then }x &\neq0\\
\mbox{else }a(a+cx)&=0\\
\implies a&=0\\
\mbox{Thus }a+cx&=0\\
\mbox{Hence }\vec{N}&=\myvec{a&ax\\ \frac{-a}{x}&-a}
\end{align}}\\
 &This is similar to $\myvec{a&a\\-a&-a}$ via $\vec{P}=\myvec{\sqrt{x}&0\\0&\frac{1}{\sqrt{x}}}$.\\
 &And $\myvec{a&a\\-a&-a}$ is similar to $\myvec{0&0\\-a&0}$ via $\vec{P}=\myvec{-1&-1\\1&0}$\\
 &And this finally is similar to $\myvec{0&0\\1&0}$ as before.\\
 &\\
\hline
&\\
Conclusion &Thus either $\vec{N}$ = 0 or $\vec{N}$ is similar over $\mathbb{C}$ to $\myvec{0&0\\1&0}$.\\
&\\
\hline
\caption{Solution summary}
\label{eq:solutions/6/2/11/table:1}
\end{longtable}

\item Find a projection $\vec{E}$ which projects $\mathbb{R}^{2}$ onto the subspace spanned by $\brak{1,-1}$ along the subspace spanned by $\brak{1,2}$.
%
\\
\solution
See Table \ref{eq:solutions/6/2/11/table:1}
\onecolumn
\begin{longtable}{|l|l|}
\hline
\multirow{3}{*}{} & \\
Statement&Solution\\
\hline
&\parbox{6cm}{\begin{align}
\mbox{Let }\vec{N}&=\myvec{a&b\\c&d}\\
\mbox{Since }\vec{N}^2&=0
\end{align}}\\
&If $\myvec{a\\c},\myvec{b\\d}$ are linearly independent then $\vec{N}$ is diagonalizable to $\myvec{0&0\\0&0}$.\\
&\parbox{6cm}{\begin{align}
\mbox{If }\vec{P}\vec{N}\vec{P}^{-1}=0\\
\mbox{then }\vec{N}=\vec{P}^{-1}\vec{0}\vec{P}=0
\end{align}}\\
Proof that&So in this case $\vec{N}$ itself is the zero matrix.\\
$\vec{N}=0$&This contradicts the assumption that $\myvec{a\\c},\myvec{b\\d}$ are linearly independent.\\
&$\therefore$ we can assume that $\myvec{a\\c},\myvec{b\\d}$ are linearly dependent if both are\\
&equal to the zero vector\\
&\parbox{6cm}
{\begin{align}
   \mbox{then } \vec{N} &= 0.
\end{align}}\\
%\hline
%\pagebreak
\hline
&\\
&Therefore we can assume at least one vector is non-zero.\\
Assuming $\myvec{b\\d}$ as&Therefore
$\vec{N}=\myvec{a&0\\c&0}$\\
the zero vector&\\
&\parbox{6cm}{\begin{align}
\mbox{So }\vec{N}^2&=0\\
\implies a^2&=0\\
\therefore a&=0\\
\mbox{Thus }\vec{N}&=\myvec{a&0\\c&0}
\end{align}}\\
&In this case $\vec{N}$ is similar to $\vec{N}=\myvec{0&0\\1&0}$ via the matrix $\vec{P}=\myvec{c&0\\0&1}$\\
&\\
\hline
&\\
Assuming $\myvec{a\\c}$ as&Therefore
$\vec{N}=\myvec{0&b\\0&d}$\\
the zero vector&\\
&\parbox{6cm}{\begin{align}
\mbox{Then }\vec{N}^2&=0\\
\implies d^2&=0\\
\therefore d&=0\\
\mbox{Thus }\vec{N}&=\myvec{0&b\\0&0}
\end{align}}\\
&In this case $\vec{N}$ is similar to $\vec{N}=\myvec{0&0\\b&0}$ via the matrix $\vec{P}=\myvec{0&1\\1&0}$,\\
&which is similar to $\myvec{0&0\\1&0}$ as above.\\
&\\
\hline
&\\
Hence& we can assume neither $\myvec{a\\c}$ or $\myvec{b\\d}$ is the zero vector.\\
&\\
%\hline
%\pagebreak
\hline
&\\
Consequences of &Since they are linearly dependent we can assume,\\
linear&\\
independence&\\
&\parbox{6cm}{\begin{align}
\myvec{b\\d}&=x\myvec{a\\c}\\
\therefore \vec{N}&=\myvec{a&ax\\c&cx}\\
\therefore \vec{N}^2&=0\\
\implies a(a+cx)&=0\\
c(a+cx)&=0\\
ax(a+cx)&=0\\
cx(a+cx)&=0
\end{align}}\\
\hline
&\\
Proof that $\vec{N}$ is&We know that at least one of a or c is not zero.\\
similar over $\mathbb{C}$ to&If a = 0 then c $\neq$ 0, it must be that x = 0.\\
$\myvec{0&0\\1&0}$&So in this case $\vec{N}=\myvec{0&0\\c&0}$ which is similar to $\myvec{0&0\\1&0}$ as before.\\
&\\
&\parbox{6cm}{\begin{align}
\mbox{If } a &\neq0\\
\mbox{then }x &\neq0\\
\mbox{else }a(a+cx)&=0\\
\implies a&=0\\
\mbox{Thus }a+cx&=0\\
\mbox{Hence }\vec{N}&=\myvec{a&ax\\ \frac{-a}{x}&-a}
\end{align}}\\
 &This is similar to $\myvec{a&a\\-a&-a}$ via $\vec{P}=\myvec{\sqrt{x}&0\\0&\frac{1}{\sqrt{x}}}$.\\
 &And $\myvec{a&a\\-a&-a}$ is similar to $\myvec{0&0\\-a&0}$ via $\vec{P}=\myvec{-1&-1\\1&0}$\\
 &And this finally is similar to $\myvec{0&0\\1&0}$ as before.\\
 &\\
\hline
&\\
Conclusion &Thus either $\vec{N}$ = 0 or $\vec{N}$ is similar over $\mathbb{C}$ to $\myvec{0&0\\1&0}$.\\
&\\
\hline
\caption{Solution summary}
\label{eq:solutions/6/2/11/table:1}
\end{longtable}

\item %
If $E_1$ and $E_2$ are the projections of independent subspaces, then $E_1$ + $E_2$ is a projection.True or False?
%
\\
\solution
See Table \ref{eq:solutions/6/2/11/table:1}
\onecolumn
\begin{longtable}{|l|l|}
\hline
\multirow{3}{*}{} & \\
Statement&Solution\\
\hline
&\parbox{6cm}{\begin{align}
\mbox{Let }\vec{N}&=\myvec{a&b\\c&d}\\
\mbox{Since }\vec{N}^2&=0
\end{align}}\\
&If $\myvec{a\\c},\myvec{b\\d}$ are linearly independent then $\vec{N}$ is diagonalizable to $\myvec{0&0\\0&0}$.\\
&\parbox{6cm}{\begin{align}
\mbox{If }\vec{P}\vec{N}\vec{P}^{-1}=0\\
\mbox{then }\vec{N}=\vec{P}^{-1}\vec{0}\vec{P}=0
\end{align}}\\
Proof that&So in this case $\vec{N}$ itself is the zero matrix.\\
$\vec{N}=0$&This contradicts the assumption that $\myvec{a\\c},\myvec{b\\d}$ are linearly independent.\\
&$\therefore$ we can assume that $\myvec{a\\c},\myvec{b\\d}$ are linearly dependent if both are\\
&equal to the zero vector\\
&\parbox{6cm}
{\begin{align}
   \mbox{then } \vec{N} &= 0.
\end{align}}\\
%\hline
%\pagebreak
\hline
&\\
&Therefore we can assume at least one vector is non-zero.\\
Assuming $\myvec{b\\d}$ as&Therefore
$\vec{N}=\myvec{a&0\\c&0}$\\
the zero vector&\\
&\parbox{6cm}{\begin{align}
\mbox{So }\vec{N}^2&=0\\
\implies a^2&=0\\
\therefore a&=0\\
\mbox{Thus }\vec{N}&=\myvec{a&0\\c&0}
\end{align}}\\
&In this case $\vec{N}$ is similar to $\vec{N}=\myvec{0&0\\1&0}$ via the matrix $\vec{P}=\myvec{c&0\\0&1}$\\
&\\
\hline
&\\
Assuming $\myvec{a\\c}$ as&Therefore
$\vec{N}=\myvec{0&b\\0&d}$\\
the zero vector&\\
&\parbox{6cm}{\begin{align}
\mbox{Then }\vec{N}^2&=0\\
\implies d^2&=0\\
\therefore d&=0\\
\mbox{Thus }\vec{N}&=\myvec{0&b\\0&0}
\end{align}}\\
&In this case $\vec{N}$ is similar to $\vec{N}=\myvec{0&0\\b&0}$ via the matrix $\vec{P}=\myvec{0&1\\1&0}$,\\
&which is similar to $\myvec{0&0\\1&0}$ as above.\\
&\\
\hline
&\\
Hence& we can assume neither $\myvec{a\\c}$ or $\myvec{b\\d}$ is the zero vector.\\
&\\
%\hline
%\pagebreak
\hline
&\\
Consequences of &Since they are linearly dependent we can assume,\\
linear&\\
independence&\\
&\parbox{6cm}{\begin{align}
\myvec{b\\d}&=x\myvec{a\\c}\\
\therefore \vec{N}&=\myvec{a&ax\\c&cx}\\
\therefore \vec{N}^2&=0\\
\implies a(a+cx)&=0\\
c(a+cx)&=0\\
ax(a+cx)&=0\\
cx(a+cx)&=0
\end{align}}\\
\hline
&\\
Proof that $\vec{N}$ is&We know that at least one of a or c is not zero.\\
similar over $\mathbb{C}$ to&If a = 0 then c $\neq$ 0, it must be that x = 0.\\
$\myvec{0&0\\1&0}$&So in this case $\vec{N}=\myvec{0&0\\c&0}$ which is similar to $\myvec{0&0\\1&0}$ as before.\\
&\\
&\parbox{6cm}{\begin{align}
\mbox{If } a &\neq0\\
\mbox{then }x &\neq0\\
\mbox{else }a(a+cx)&=0\\
\implies a&=0\\
\mbox{Thus }a+cx&=0\\
\mbox{Hence }\vec{N}&=\myvec{a&ax\\ \frac{-a}{x}&-a}
\end{align}}\\
 &This is similar to $\myvec{a&a\\-a&-a}$ via $\vec{P}=\myvec{\sqrt{x}&0\\0&\frac{1}{\sqrt{x}}}$.\\
 &And $\myvec{a&a\\-a&-a}$ is similar to $\myvec{0&0\\-a&0}$ via $\vec{P}=\myvec{-1&-1\\1&0}$\\
 &And this finally is similar to $\myvec{0&0\\1&0}$ as before.\\
 &\\
\hline
&\\
Conclusion &Thus either $\vec{N}$ = 0 or $\vec{N}$ is similar over $\mathbb{C}$ to $\myvec{0&0\\1&0}$.\\
&\\
\hline
\caption{Solution summary}
\label{eq:solutions/6/2/11/table:1}
\end{longtable}

\item If $\vec{E}$ is the projection and f is a polynomial,then $f(\vec{E})=a\vec{I}+\vec{b}\vec{E}$. What are a and $\vec{b}$ in terms of the coefficient vector of f?
%
\\
\solution
See Table \ref{eq:solutions/6/2/11/table:1}
\onecolumn
\begin{longtable}{|l|l|}
\hline
\multirow{3}{*}{} & \\
Statement&Solution\\
\hline
&\parbox{6cm}{\begin{align}
\mbox{Let }\vec{N}&=\myvec{a&b\\c&d}\\
\mbox{Since }\vec{N}^2&=0
\end{align}}\\
&If $\myvec{a\\c},\myvec{b\\d}$ are linearly independent then $\vec{N}$ is diagonalizable to $\myvec{0&0\\0&0}$.\\
&\parbox{6cm}{\begin{align}
\mbox{If }\vec{P}\vec{N}\vec{P}^{-1}=0\\
\mbox{then }\vec{N}=\vec{P}^{-1}\vec{0}\vec{P}=0
\end{align}}\\
Proof that&So in this case $\vec{N}$ itself is the zero matrix.\\
$\vec{N}=0$&This contradicts the assumption that $\myvec{a\\c},\myvec{b\\d}$ are linearly independent.\\
&$\therefore$ we can assume that $\myvec{a\\c},\myvec{b\\d}$ are linearly dependent if both are\\
&equal to the zero vector\\
&\parbox{6cm}
{\begin{align}
   \mbox{then } \vec{N} &= 0.
\end{align}}\\
%\hline
%\pagebreak
\hline
&\\
&Therefore we can assume at least one vector is non-zero.\\
Assuming $\myvec{b\\d}$ as&Therefore
$\vec{N}=\myvec{a&0\\c&0}$\\
the zero vector&\\
&\parbox{6cm}{\begin{align}
\mbox{So }\vec{N}^2&=0\\
\implies a^2&=0\\
\therefore a&=0\\
\mbox{Thus }\vec{N}&=\myvec{a&0\\c&0}
\end{align}}\\
&In this case $\vec{N}$ is similar to $\vec{N}=\myvec{0&0\\1&0}$ via the matrix $\vec{P}=\myvec{c&0\\0&1}$\\
&\\
\hline
&\\
Assuming $\myvec{a\\c}$ as&Therefore
$\vec{N}=\myvec{0&b\\0&d}$\\
the zero vector&\\
&\parbox{6cm}{\begin{align}
\mbox{Then }\vec{N}^2&=0\\
\implies d^2&=0\\
\therefore d&=0\\
\mbox{Thus }\vec{N}&=\myvec{0&b\\0&0}
\end{align}}\\
&In this case $\vec{N}$ is similar to $\vec{N}=\myvec{0&0\\b&0}$ via the matrix $\vec{P}=\myvec{0&1\\1&0}$,\\
&which is similar to $\myvec{0&0\\1&0}$ as above.\\
&\\
\hline
&\\
Hence& we can assume neither $\myvec{a\\c}$ or $\myvec{b\\d}$ is the zero vector.\\
&\\
%\hline
%\pagebreak
\hline
&\\
Consequences of &Since they are linearly dependent we can assume,\\
linear&\\
independence&\\
&\parbox{6cm}{\begin{align}
\myvec{b\\d}&=x\myvec{a\\c}\\
\therefore \vec{N}&=\myvec{a&ax\\c&cx}\\
\therefore \vec{N}^2&=0\\
\implies a(a+cx)&=0\\
c(a+cx)&=0\\
ax(a+cx)&=0\\
cx(a+cx)&=0
\end{align}}\\
\hline
&\\
Proof that $\vec{N}$ is&We know that at least one of a or c is not zero.\\
similar over $\mathbb{C}$ to&If a = 0 then c $\neq$ 0, it must be that x = 0.\\
$\myvec{0&0\\1&0}$&So in this case $\vec{N}=\myvec{0&0\\c&0}$ which is similar to $\myvec{0&0\\1&0}$ as before.\\
&\\
&\parbox{6cm}{\begin{align}
\mbox{If } a &\neq0\\
\mbox{then }x &\neq0\\
\mbox{else }a(a+cx)&=0\\
\implies a&=0\\
\mbox{Thus }a+cx&=0\\
\mbox{Hence }\vec{N}&=\myvec{a&ax\\ \frac{-a}{x}&-a}
\end{align}}\\
 &This is similar to $\myvec{a&a\\-a&-a}$ via $\vec{P}=\myvec{\sqrt{x}&0\\0&\frac{1}{\sqrt{x}}}$.\\
 &And $\myvec{a&a\\-a&-a}$ is similar to $\myvec{0&0\\-a&0}$ via $\vec{P}=\myvec{-1&-1\\1&0}$\\
 &And this finally is similar to $\myvec{0&0\\1&0}$ as before.\\
 &\\
\hline
&\\
Conclusion &Thus either $\vec{N}$ = 0 or $\vec{N}$ is similar over $\mathbb{C}$ to $\myvec{0&0\\1&0}$.\\
&\\
\hline
\caption{Solution summary}
\label{eq:solutions/6/2/11/table:1}
\end{longtable}

\item %
True or False? If a diagonalizable operator has only the characteristic values $0$ and $1$, it is a projection.
%
\\
\solution
See Table \ref{eq:solutions/6/2/11/table:1}
\onecolumn
\begin{longtable}{|l|l|}
\hline
\multirow{3}{*}{} & \\
Statement&Solution\\
\hline
&\parbox{6cm}{\begin{align}
\mbox{Let }\vec{N}&=\myvec{a&b\\c&d}\\
\mbox{Since }\vec{N}^2&=0
\end{align}}\\
&If $\myvec{a\\c},\myvec{b\\d}$ are linearly independent then $\vec{N}$ is diagonalizable to $\myvec{0&0\\0&0}$.\\
&\parbox{6cm}{\begin{align}
\mbox{If }\vec{P}\vec{N}\vec{P}^{-1}=0\\
\mbox{then }\vec{N}=\vec{P}^{-1}\vec{0}\vec{P}=0
\end{align}}\\
Proof that&So in this case $\vec{N}$ itself is the zero matrix.\\
$\vec{N}=0$&This contradicts the assumption that $\myvec{a\\c},\myvec{b\\d}$ are linearly independent.\\
&$\therefore$ we can assume that $\myvec{a\\c},\myvec{b\\d}$ are linearly dependent if both are\\
&equal to the zero vector\\
&\parbox{6cm}
{\begin{align}
   \mbox{then } \vec{N} &= 0.
\end{align}}\\
%\hline
%\pagebreak
\hline
&\\
&Therefore we can assume at least one vector is non-zero.\\
Assuming $\myvec{b\\d}$ as&Therefore
$\vec{N}=\myvec{a&0\\c&0}$\\
the zero vector&\\
&\parbox{6cm}{\begin{align}
\mbox{So }\vec{N}^2&=0\\
\implies a^2&=0\\
\therefore a&=0\\
\mbox{Thus }\vec{N}&=\myvec{a&0\\c&0}
\end{align}}\\
&In this case $\vec{N}$ is similar to $\vec{N}=\myvec{0&0\\1&0}$ via the matrix $\vec{P}=\myvec{c&0\\0&1}$\\
&\\
\hline
&\\
Assuming $\myvec{a\\c}$ as&Therefore
$\vec{N}=\myvec{0&b\\0&d}$\\
the zero vector&\\
&\parbox{6cm}{\begin{align}
\mbox{Then }\vec{N}^2&=0\\
\implies d^2&=0\\
\therefore d&=0\\
\mbox{Thus }\vec{N}&=\myvec{0&b\\0&0}
\end{align}}\\
&In this case $\vec{N}$ is similar to $\vec{N}=\myvec{0&0\\b&0}$ via the matrix $\vec{P}=\myvec{0&1\\1&0}$,\\
&which is similar to $\myvec{0&0\\1&0}$ as above.\\
&\\
\hline
&\\
Hence& we can assume neither $\myvec{a\\c}$ or $\myvec{b\\d}$ is the zero vector.\\
&\\
%\hline
%\pagebreak
\hline
&\\
Consequences of &Since they are linearly dependent we can assume,\\
linear&\\
independence&\\
&\parbox{6cm}{\begin{align}
\myvec{b\\d}&=x\myvec{a\\c}\\
\therefore \vec{N}&=\myvec{a&ax\\c&cx}\\
\therefore \vec{N}^2&=0\\
\implies a(a+cx)&=0\\
c(a+cx)&=0\\
ax(a+cx)&=0\\
cx(a+cx)&=0
\end{align}}\\
\hline
&\\
Proof that $\vec{N}$ is&We know that at least one of a or c is not zero.\\
similar over $\mathbb{C}$ to&If a = 0 then c $\neq$ 0, it must be that x = 0.\\
$\myvec{0&0\\1&0}$&So in this case $\vec{N}=\myvec{0&0\\c&0}$ which is similar to $\myvec{0&0\\1&0}$ as before.\\
&\\
&\parbox{6cm}{\begin{align}
\mbox{If } a &\neq0\\
\mbox{then }x &\neq0\\
\mbox{else }a(a+cx)&=0\\
\implies a&=0\\
\mbox{Thus }a+cx&=0\\
\mbox{Hence }\vec{N}&=\myvec{a&ax\\ \frac{-a}{x}&-a}
\end{align}}\\
 &This is similar to $\myvec{a&a\\-a&-a}$ via $\vec{P}=\myvec{\sqrt{x}&0\\0&\frac{1}{\sqrt{x}}}$.\\
 &And $\myvec{a&a\\-a&-a}$ is similar to $\myvec{0&0\\-a&0}$ via $\vec{P}=\myvec{-1&-1\\1&0}$\\
 &And this finally is similar to $\myvec{0&0\\1&0}$ as before.\\
 &\\
\hline
&\\
Conclusion &Thus either $\vec{N}$ = 0 or $\vec{N}$ is similar over $\mathbb{C}$ to $\myvec{0&0\\1&0}$.\\
&\\
\hline
\caption{Solution summary}
\label{eq:solutions/6/2/11/table:1}
\end{longtable}

\item 	Prove that if $\mathbf{E}$ is the projection on $\mathbf{R}$ along $\mathbf{N}$, then $(\mathbf{I-E})$ is the projection on $\mathbf{N}$ along $\mathbf{R}$ 
%
\\
\solution
See Table \ref{eq:solutions/6/2/11/table:1}
\onecolumn
\begin{longtable}{|l|l|}
\hline
\multirow{3}{*}{} & \\
Statement&Solution\\
\hline
&\parbox{6cm}{\begin{align}
\mbox{Let }\vec{N}&=\myvec{a&b\\c&d}\\
\mbox{Since }\vec{N}^2&=0
\end{align}}\\
&If $\myvec{a\\c},\myvec{b\\d}$ are linearly independent then $\vec{N}$ is diagonalizable to $\myvec{0&0\\0&0}$.\\
&\parbox{6cm}{\begin{align}
\mbox{If }\vec{P}\vec{N}\vec{P}^{-1}=0\\
\mbox{then }\vec{N}=\vec{P}^{-1}\vec{0}\vec{P}=0
\end{align}}\\
Proof that&So in this case $\vec{N}$ itself is the zero matrix.\\
$\vec{N}=0$&This contradicts the assumption that $\myvec{a\\c},\myvec{b\\d}$ are linearly independent.\\
&$\therefore$ we can assume that $\myvec{a\\c},\myvec{b\\d}$ are linearly dependent if both are\\
&equal to the zero vector\\
&\parbox{6cm}
{\begin{align}
   \mbox{then } \vec{N} &= 0.
\end{align}}\\
%\hline
%\pagebreak
\hline
&\\
&Therefore we can assume at least one vector is non-zero.\\
Assuming $\myvec{b\\d}$ as&Therefore
$\vec{N}=\myvec{a&0\\c&0}$\\
the zero vector&\\
&\parbox{6cm}{\begin{align}
\mbox{So }\vec{N}^2&=0\\
\implies a^2&=0\\
\therefore a&=0\\
\mbox{Thus }\vec{N}&=\myvec{a&0\\c&0}
\end{align}}\\
&In this case $\vec{N}$ is similar to $\vec{N}=\myvec{0&0\\1&0}$ via the matrix $\vec{P}=\myvec{c&0\\0&1}$\\
&\\
\hline
&\\
Assuming $\myvec{a\\c}$ as&Therefore
$\vec{N}=\myvec{0&b\\0&d}$\\
the zero vector&\\
&\parbox{6cm}{\begin{align}
\mbox{Then }\vec{N}^2&=0\\
\implies d^2&=0\\
\therefore d&=0\\
\mbox{Thus }\vec{N}&=\myvec{0&b\\0&0}
\end{align}}\\
&In this case $\vec{N}$ is similar to $\vec{N}=\myvec{0&0\\b&0}$ via the matrix $\vec{P}=\myvec{0&1\\1&0}$,\\
&which is similar to $\myvec{0&0\\1&0}$ as above.\\
&\\
\hline
&\\
Hence& we can assume neither $\myvec{a\\c}$ or $\myvec{b\\d}$ is the zero vector.\\
&\\
%\hline
%\pagebreak
\hline
&\\
Consequences of &Since they are linearly dependent we can assume,\\
linear&\\
independence&\\
&\parbox{6cm}{\begin{align}
\myvec{b\\d}&=x\myvec{a\\c}\\
\therefore \vec{N}&=\myvec{a&ax\\c&cx}\\
\therefore \vec{N}^2&=0\\
\implies a(a+cx)&=0\\
c(a+cx)&=0\\
ax(a+cx)&=0\\
cx(a+cx)&=0
\end{align}}\\
\hline
&\\
Proof that $\vec{N}$ is&We know that at least one of a or c is not zero.\\
similar over $\mathbb{C}$ to&If a = 0 then c $\neq$ 0, it must be that x = 0.\\
$\myvec{0&0\\1&0}$&So in this case $\vec{N}=\myvec{0&0\\c&0}$ which is similar to $\myvec{0&0\\1&0}$ as before.\\
&\\
&\parbox{6cm}{\begin{align}
\mbox{If } a &\neq0\\
\mbox{then }x &\neq0\\
\mbox{else }a(a+cx)&=0\\
\implies a&=0\\
\mbox{Thus }a+cx&=0\\
\mbox{Hence }\vec{N}&=\myvec{a&ax\\ \frac{-a}{x}&-a}
\end{align}}\\
 &This is similar to $\myvec{a&a\\-a&-a}$ via $\vec{P}=\myvec{\sqrt{x}&0\\0&\frac{1}{\sqrt{x}}}$.\\
 &And $\myvec{a&a\\-a&-a}$ is similar to $\myvec{0&0\\-a&0}$ via $\vec{P}=\myvec{-1&-1\\1&0}$\\
 &And this finally is similar to $\myvec{0&0\\1&0}$ as before.\\
 &\\
\hline
&\\
Conclusion &Thus either $\vec{N}$ = 0 or $\vec{N}$ is similar over $\mathbb{C}$ to $\myvec{0&0\\1&0}$.\\
&\\
\hline
\caption{Solution summary}
\label{eq:solutions/6/2/11/table:1}
\end{longtable}

\item Let $E_1,...,E_k$ be linear operators on the space $\vec{V}$ such that $E_1+...+E_k =\vec{I}$.\\
a) Prove that if $E_iE_j = 0$ for $i\neq j$,then $E_i^2=E_i$ for each $i$.\\
b) In the case $k=2$, prove the converse of (a). That is, if $E_1+E_2=\vec{I}$ and $E_1^2 = E_1$, $E_2^2= E_2$, then $E_1E_2=0$
%
\\
\solution
See Table \ref{eq:solutions/6/2/11/table:1}
\onecolumn
\begin{longtable}{|l|l|}
\hline
\multirow{3}{*}{} & \\
Statement&Solution\\
\hline
&\parbox{6cm}{\begin{align}
\mbox{Let }\vec{N}&=\myvec{a&b\\c&d}\\
\mbox{Since }\vec{N}^2&=0
\end{align}}\\
&If $\myvec{a\\c},\myvec{b\\d}$ are linearly independent then $\vec{N}$ is diagonalizable to $\myvec{0&0\\0&0}$.\\
&\parbox{6cm}{\begin{align}
\mbox{If }\vec{P}\vec{N}\vec{P}^{-1}=0\\
\mbox{then }\vec{N}=\vec{P}^{-1}\vec{0}\vec{P}=0
\end{align}}\\
Proof that&So in this case $\vec{N}$ itself is the zero matrix.\\
$\vec{N}=0$&This contradicts the assumption that $\myvec{a\\c},\myvec{b\\d}$ are linearly independent.\\
&$\therefore$ we can assume that $\myvec{a\\c},\myvec{b\\d}$ are linearly dependent if both are\\
&equal to the zero vector\\
&\parbox{6cm}
{\begin{align}
   \mbox{then } \vec{N} &= 0.
\end{align}}\\
%\hline
%\pagebreak
\hline
&\\
&Therefore we can assume at least one vector is non-zero.\\
Assuming $\myvec{b\\d}$ as&Therefore
$\vec{N}=\myvec{a&0\\c&0}$\\
the zero vector&\\
&\parbox{6cm}{\begin{align}
\mbox{So }\vec{N}^2&=0\\
\implies a^2&=0\\
\therefore a&=0\\
\mbox{Thus }\vec{N}&=\myvec{a&0\\c&0}
\end{align}}\\
&In this case $\vec{N}$ is similar to $\vec{N}=\myvec{0&0\\1&0}$ via the matrix $\vec{P}=\myvec{c&0\\0&1}$\\
&\\
\hline
&\\
Assuming $\myvec{a\\c}$ as&Therefore
$\vec{N}=\myvec{0&b\\0&d}$\\
the zero vector&\\
&\parbox{6cm}{\begin{align}
\mbox{Then }\vec{N}^2&=0\\
\implies d^2&=0\\
\therefore d&=0\\
\mbox{Thus }\vec{N}&=\myvec{0&b\\0&0}
\end{align}}\\
&In this case $\vec{N}$ is similar to $\vec{N}=\myvec{0&0\\b&0}$ via the matrix $\vec{P}=\myvec{0&1\\1&0}$,\\
&which is similar to $\myvec{0&0\\1&0}$ as above.\\
&\\
\hline
&\\
Hence& we can assume neither $\myvec{a\\c}$ or $\myvec{b\\d}$ is the zero vector.\\
&\\
%\hline
%\pagebreak
\hline
&\\
Consequences of &Since they are linearly dependent we can assume,\\
linear&\\
independence&\\
&\parbox{6cm}{\begin{align}
\myvec{b\\d}&=x\myvec{a\\c}\\
\therefore \vec{N}&=\myvec{a&ax\\c&cx}\\
\therefore \vec{N}^2&=0\\
\implies a(a+cx)&=0\\
c(a+cx)&=0\\
ax(a+cx)&=0\\
cx(a+cx)&=0
\end{align}}\\
\hline
&\\
Proof that $\vec{N}$ is&We know that at least one of a or c is not zero.\\
similar over $\mathbb{C}$ to&If a = 0 then c $\neq$ 0, it must be that x = 0.\\
$\myvec{0&0\\1&0}$&So in this case $\vec{N}=\myvec{0&0\\c&0}$ which is similar to $\myvec{0&0\\1&0}$ as before.\\
&\\
&\parbox{6cm}{\begin{align}
\mbox{If } a &\neq0\\
\mbox{then }x &\neq0\\
\mbox{else }a(a+cx)&=0\\
\implies a&=0\\
\mbox{Thus }a+cx&=0\\
\mbox{Hence }\vec{N}&=\myvec{a&ax\\ \frac{-a}{x}&-a}
\end{align}}\\
 &This is similar to $\myvec{a&a\\-a&-a}$ via $\vec{P}=\myvec{\sqrt{x}&0\\0&\frac{1}{\sqrt{x}}}$.\\
 &And $\myvec{a&a\\-a&-a}$ is similar to $\myvec{0&0\\-a&0}$ via $\vec{P}=\myvec{-1&-1\\1&0}$\\
 &And this finally is similar to $\myvec{0&0\\1&0}$ as before.\\
 &\\
\hline
&\\
Conclusion &Thus either $\vec{N}$ = 0 or $\vec{N}$ is similar over $\mathbb{C}$ to $\myvec{0&0\\1&0}$.\\
&\\
\hline
\caption{Solution summary}
\label{eq:solutions/6/2/11/table:1}
\end{longtable}

\item 	Let $\mathbf{V}$ be a real vector space and E an idempotent linear operator on $\mathbf{V}$, i.e., a projection. Prove that $\myvec{\vec{I} + \vec{E}}$ is
	invertible. Find $\myvec{\vec{I} + \vec{E}}^{-1}.$
%
\\
\solution
See Table \ref{eq:solutions/6/2/11/table:1}
\onecolumn
\begin{longtable}{|l|l|}
\hline
\multirow{3}{*}{} & \\
Statement&Solution\\
\hline
&\parbox{6cm}{\begin{align}
\mbox{Let }\vec{N}&=\myvec{a&b\\c&d}\\
\mbox{Since }\vec{N}^2&=0
\end{align}}\\
&If $\myvec{a\\c},\myvec{b\\d}$ are linearly independent then $\vec{N}$ is diagonalizable to $\myvec{0&0\\0&0}$.\\
&\parbox{6cm}{\begin{align}
\mbox{If }\vec{P}\vec{N}\vec{P}^{-1}=0\\
\mbox{then }\vec{N}=\vec{P}^{-1}\vec{0}\vec{P}=0
\end{align}}\\
Proof that&So in this case $\vec{N}$ itself is the zero matrix.\\
$\vec{N}=0$&This contradicts the assumption that $\myvec{a\\c},\myvec{b\\d}$ are linearly independent.\\
&$\therefore$ we can assume that $\myvec{a\\c},\myvec{b\\d}$ are linearly dependent if both are\\
&equal to the zero vector\\
&\parbox{6cm}
{\begin{align}
   \mbox{then } \vec{N} &= 0.
\end{align}}\\
%\hline
%\pagebreak
\hline
&\\
&Therefore we can assume at least one vector is non-zero.\\
Assuming $\myvec{b\\d}$ as&Therefore
$\vec{N}=\myvec{a&0\\c&0}$\\
the zero vector&\\
&\parbox{6cm}{\begin{align}
\mbox{So }\vec{N}^2&=0\\
\implies a^2&=0\\
\therefore a&=0\\
\mbox{Thus }\vec{N}&=\myvec{a&0\\c&0}
\end{align}}\\
&In this case $\vec{N}$ is similar to $\vec{N}=\myvec{0&0\\1&0}$ via the matrix $\vec{P}=\myvec{c&0\\0&1}$\\
&\\
\hline
&\\
Assuming $\myvec{a\\c}$ as&Therefore
$\vec{N}=\myvec{0&b\\0&d}$\\
the zero vector&\\
&\parbox{6cm}{\begin{align}
\mbox{Then }\vec{N}^2&=0\\
\implies d^2&=0\\
\therefore d&=0\\
\mbox{Thus }\vec{N}&=\myvec{0&b\\0&0}
\end{align}}\\
&In this case $\vec{N}$ is similar to $\vec{N}=\myvec{0&0\\b&0}$ via the matrix $\vec{P}=\myvec{0&1\\1&0}$,\\
&which is similar to $\myvec{0&0\\1&0}$ as above.\\
&\\
\hline
&\\
Hence& we can assume neither $\myvec{a\\c}$ or $\myvec{b\\d}$ is the zero vector.\\
&\\
%\hline
%\pagebreak
\hline
&\\
Consequences of &Since they are linearly dependent we can assume,\\
linear&\\
independence&\\
&\parbox{6cm}{\begin{align}
\myvec{b\\d}&=x\myvec{a\\c}\\
\therefore \vec{N}&=\myvec{a&ax\\c&cx}\\
\therefore \vec{N}^2&=0\\
\implies a(a+cx)&=0\\
c(a+cx)&=0\\
ax(a+cx)&=0\\
cx(a+cx)&=0
\end{align}}\\
\hline
&\\
Proof that $\vec{N}$ is&We know that at least one of a or c is not zero.\\
similar over $\mathbb{C}$ to&If a = 0 then c $\neq$ 0, it must be that x = 0.\\
$\myvec{0&0\\1&0}$&So in this case $\vec{N}=\myvec{0&0\\c&0}$ which is similar to $\myvec{0&0\\1&0}$ as before.\\
&\\
&\parbox{6cm}{\begin{align}
\mbox{If } a &\neq0\\
\mbox{then }x &\neq0\\
\mbox{else }a(a+cx)&=0\\
\implies a&=0\\
\mbox{Thus }a+cx&=0\\
\mbox{Hence }\vec{N}&=\myvec{a&ax\\ \frac{-a}{x}&-a}
\end{align}}\\
 &This is similar to $\myvec{a&a\\-a&-a}$ via $\vec{P}=\myvec{\sqrt{x}&0\\0&\frac{1}{\sqrt{x}}}$.\\
 &And $\myvec{a&a\\-a&-a}$ is similar to $\myvec{0&0\\-a&0}$ via $\vec{P}=\myvec{-1&-1\\1&0}$\\
 &And this finally is similar to $\myvec{0&0\\1&0}$ as before.\\
 &\\
\hline
&\\
Conclusion &Thus either $\vec{N}$ = 0 or $\vec{N}$ is similar over $\mathbb{C}$ to $\myvec{0&0\\1&0}$.\\
&\\
\hline
\caption{Solution summary}
\label{eq:solutions/6/2/11/table:1}
\end{longtable}

\end{enumerate}

\subsection{Invariant Direct Sums}
\renewcommand{\theequation}{\theenumi}
\renewcommand{\thefigure}{\theenumi}
\begin{enumerate}[label=\thesubsection.\arabic*.,ref=\thesubsection.\theenumi]
\numberwithin{equation}{enumi}
\numberwithin{figure}{enumi}

\item Let $E$ be a projection of $V$ and let $T$ be a linear operator on $V$. Prove that the range of $E$ is invariant under $T$ if and only if $ETE$ = $TE$. Prove that both the range and null space of $E$ are invariant under $T$ if and only if $ET$ = $TE$. 
%
\\
\solution
See Table \ref{eq:solutions/6/2/11/table:1}
\onecolumn
\begin{longtable}{|l|l|}
\hline
\multirow{3}{*}{} & \\
Statement&Solution\\
\hline
&\parbox{6cm}{\begin{align}
\mbox{Let }\vec{N}&=\myvec{a&b\\c&d}\\
\mbox{Since }\vec{N}^2&=0
\end{align}}\\
&If $\myvec{a\\c},\myvec{b\\d}$ are linearly independent then $\vec{N}$ is diagonalizable to $\myvec{0&0\\0&0}$.\\
&\parbox{6cm}{\begin{align}
\mbox{If }\vec{P}\vec{N}\vec{P}^{-1}=0\\
\mbox{then }\vec{N}=\vec{P}^{-1}\vec{0}\vec{P}=0
\end{align}}\\
Proof that&So in this case $\vec{N}$ itself is the zero matrix.\\
$\vec{N}=0$&This contradicts the assumption that $\myvec{a\\c},\myvec{b\\d}$ are linearly independent.\\
&$\therefore$ we can assume that $\myvec{a\\c},\myvec{b\\d}$ are linearly dependent if both are\\
&equal to the zero vector\\
&\parbox{6cm}
{\begin{align}
   \mbox{then } \vec{N} &= 0.
\end{align}}\\
%\hline
%\pagebreak
\hline
&\\
&Therefore we can assume at least one vector is non-zero.\\
Assuming $\myvec{b\\d}$ as&Therefore
$\vec{N}=\myvec{a&0\\c&0}$\\
the zero vector&\\
&\parbox{6cm}{\begin{align}
\mbox{So }\vec{N}^2&=0\\
\implies a^2&=0\\
\therefore a&=0\\
\mbox{Thus }\vec{N}&=\myvec{a&0\\c&0}
\end{align}}\\
&In this case $\vec{N}$ is similar to $\vec{N}=\myvec{0&0\\1&0}$ via the matrix $\vec{P}=\myvec{c&0\\0&1}$\\
&\\
\hline
&\\
Assuming $\myvec{a\\c}$ as&Therefore
$\vec{N}=\myvec{0&b\\0&d}$\\
the zero vector&\\
&\parbox{6cm}{\begin{align}
\mbox{Then }\vec{N}^2&=0\\
\implies d^2&=0\\
\therefore d&=0\\
\mbox{Thus }\vec{N}&=\myvec{0&b\\0&0}
\end{align}}\\
&In this case $\vec{N}$ is similar to $\vec{N}=\myvec{0&0\\b&0}$ via the matrix $\vec{P}=\myvec{0&1\\1&0}$,\\
&which is similar to $\myvec{0&0\\1&0}$ as above.\\
&\\
\hline
&\\
Hence& we can assume neither $\myvec{a\\c}$ or $\myvec{b\\d}$ is the zero vector.\\
&\\
%\hline
%\pagebreak
\hline
&\\
Consequences of &Since they are linearly dependent we can assume,\\
linear&\\
independence&\\
&\parbox{6cm}{\begin{align}
\myvec{b\\d}&=x\myvec{a\\c}\\
\therefore \vec{N}&=\myvec{a&ax\\c&cx}\\
\therefore \vec{N}^2&=0\\
\implies a(a+cx)&=0\\
c(a+cx)&=0\\
ax(a+cx)&=0\\
cx(a+cx)&=0
\end{align}}\\
\hline
&\\
Proof that $\vec{N}$ is&We know that at least one of a or c is not zero.\\
similar over $\mathbb{C}$ to&If a = 0 then c $\neq$ 0, it must be that x = 0.\\
$\myvec{0&0\\1&0}$&So in this case $\vec{N}=\myvec{0&0\\c&0}$ which is similar to $\myvec{0&0\\1&0}$ as before.\\
&\\
&\parbox{6cm}{\begin{align}
\mbox{If } a &\neq0\\
\mbox{then }x &\neq0\\
\mbox{else }a(a+cx)&=0\\
\implies a&=0\\
\mbox{Thus }a+cx&=0\\
\mbox{Hence }\vec{N}&=\myvec{a&ax\\ \frac{-a}{x}&-a}
\end{align}}\\
 &This is similar to $\myvec{a&a\\-a&-a}$ via $\vec{P}=\myvec{\sqrt{x}&0\\0&\frac{1}{\sqrt{x}}}$.\\
 &And $\myvec{a&a\\-a&-a}$ is similar to $\myvec{0&0\\-a&0}$ via $\vec{P}=\myvec{-1&-1\\1&0}$\\
 &And this finally is similar to $\myvec{0&0\\1&0}$ as before.\\
 &\\
\hline
&\\
Conclusion &Thus either $\vec{N}$ = 0 or $\vec{N}$ is similar over $\mathbb{C}$ to $\myvec{0&0\\1&0}$.\\
&\\
\hline
\caption{Solution summary}
\label{eq:solutions/6/2/11/table:1}
\end{longtable}

\item Let $T$ be the linear operator on $R^2$, the matrix of which in the standard ordered basis is \\
$\myvec{2&1\\0&2}$\\
Let $W_1$ be the subspace of $R^2$ spanned by the vector $\epsilon_1=(1,0)$
\begin{enumerate}
\item Proove that $W_1$ is invariant under $T$.
\item Prove that there is no subspace $W_2$ which is invariant under $T$ and is complementary to $W_1$:\\
$R^2=W_1\oplus W_2$\\
(Compare with exercise 1 of section 6.5.)
\end{enumerate}
%
\solution
See Table \ref{eq:solutions/6/2/11/table:1}
\onecolumn
\begin{longtable}{|l|l|}
\hline
\multirow{3}{*}{} & \\
Statement&Solution\\
\hline
&\parbox{6cm}{\begin{align}
\mbox{Let }\vec{N}&=\myvec{a&b\\c&d}\\
\mbox{Since }\vec{N}^2&=0
\end{align}}\\
&If $\myvec{a\\c},\myvec{b\\d}$ are linearly independent then $\vec{N}$ is diagonalizable to $\myvec{0&0\\0&0}$.\\
&\parbox{6cm}{\begin{align}
\mbox{If }\vec{P}\vec{N}\vec{P}^{-1}=0\\
\mbox{then }\vec{N}=\vec{P}^{-1}\vec{0}\vec{P}=0
\end{align}}\\
Proof that&So in this case $\vec{N}$ itself is the zero matrix.\\
$\vec{N}=0$&This contradicts the assumption that $\myvec{a\\c},\myvec{b\\d}$ are linearly independent.\\
&$\therefore$ we can assume that $\myvec{a\\c},\myvec{b\\d}$ are linearly dependent if both are\\
&equal to the zero vector\\
&\parbox{6cm}
{\begin{align}
   \mbox{then } \vec{N} &= 0.
\end{align}}\\
%\hline
%\pagebreak
\hline
&\\
&Therefore we can assume at least one vector is non-zero.\\
Assuming $\myvec{b\\d}$ as&Therefore
$\vec{N}=\myvec{a&0\\c&0}$\\
the zero vector&\\
&\parbox{6cm}{\begin{align}
\mbox{So }\vec{N}^2&=0\\
\implies a^2&=0\\
\therefore a&=0\\
\mbox{Thus }\vec{N}&=\myvec{a&0\\c&0}
\end{align}}\\
&In this case $\vec{N}$ is similar to $\vec{N}=\myvec{0&0\\1&0}$ via the matrix $\vec{P}=\myvec{c&0\\0&1}$\\
&\\
\hline
&\\
Assuming $\myvec{a\\c}$ as&Therefore
$\vec{N}=\myvec{0&b\\0&d}$\\
the zero vector&\\
&\parbox{6cm}{\begin{align}
\mbox{Then }\vec{N}^2&=0\\
\implies d^2&=0\\
\therefore d&=0\\
\mbox{Thus }\vec{N}&=\myvec{0&b\\0&0}
\end{align}}\\
&In this case $\vec{N}$ is similar to $\vec{N}=\myvec{0&0\\b&0}$ via the matrix $\vec{P}=\myvec{0&1\\1&0}$,\\
&which is similar to $\myvec{0&0\\1&0}$ as above.\\
&\\
\hline
&\\
Hence& we can assume neither $\myvec{a\\c}$ or $\myvec{b\\d}$ is the zero vector.\\
&\\
%\hline
%\pagebreak
\hline
&\\
Consequences of &Since they are linearly dependent we can assume,\\
linear&\\
independence&\\
&\parbox{6cm}{\begin{align}
\myvec{b\\d}&=x\myvec{a\\c}\\
\therefore \vec{N}&=\myvec{a&ax\\c&cx}\\
\therefore \vec{N}^2&=0\\
\implies a(a+cx)&=0\\
c(a+cx)&=0\\
ax(a+cx)&=0\\
cx(a+cx)&=0
\end{align}}\\
\hline
&\\
Proof that $\vec{N}$ is&We know that at least one of a or c is not zero.\\
similar over $\mathbb{C}$ to&If a = 0 then c $\neq$ 0, it must be that x = 0.\\
$\myvec{0&0\\1&0}$&So in this case $\vec{N}=\myvec{0&0\\c&0}$ which is similar to $\myvec{0&0\\1&0}$ as before.\\
&\\
&\parbox{6cm}{\begin{align}
\mbox{If } a &\neq0\\
\mbox{then }x &\neq0\\
\mbox{else }a(a+cx)&=0\\
\implies a&=0\\
\mbox{Thus }a+cx&=0\\
\mbox{Hence }\vec{N}&=\myvec{a&ax\\ \frac{-a}{x}&-a}
\end{align}}\\
 &This is similar to $\myvec{a&a\\-a&-a}$ via $\vec{P}=\myvec{\sqrt{x}&0\\0&\frac{1}{\sqrt{x}}}$.\\
 &And $\myvec{a&a\\-a&-a}$ is similar to $\myvec{0&0\\-a&0}$ via $\vec{P}=\myvec{-1&-1\\1&0}$\\
 &And this finally is similar to $\myvec{0&0\\1&0}$ as before.\\
 &\\
\hline
&\\
Conclusion &Thus either $\vec{N}$ = 0 or $\vec{N}$ is similar over $\mathbb{C}$ to $\myvec{0&0\\1&0}$.\\
&\\
\hline
\caption{Solution summary}
\label{eq:solutions/6/2/11/table:1}
\end{longtable}

%
\item %
Let $\vec{T}$ be the diagonalizable linear operator on $\mathbb{R}^3$ which we discussed in example 3 of section $6.2$. Use the Lagrange polynomials to write the representing matrix $\vec{A}$ in the form
\begin{align}
    \vec{A}=\vec{E_1}+2\vec{E_2},\quad \vec{E_1}+\vec{E_2}=\vec{I}, \vec{E_1}\vec{E_2}=\vec{0}
\end{align}
%
\solution
See Table \ref{eq:solutions/6/2/11/table:1}
\onecolumn
\begin{longtable}{|l|l|}
\hline
\multirow{3}{*}{} & \\
Statement&Solution\\
\hline
&\parbox{6cm}{\begin{align}
\mbox{Let }\vec{N}&=\myvec{a&b\\c&d}\\
\mbox{Since }\vec{N}^2&=0
\end{align}}\\
&If $\myvec{a\\c},\myvec{b\\d}$ are linearly independent then $\vec{N}$ is diagonalizable to $\myvec{0&0\\0&0}$.\\
&\parbox{6cm}{\begin{align}
\mbox{If }\vec{P}\vec{N}\vec{P}^{-1}=0\\
\mbox{then }\vec{N}=\vec{P}^{-1}\vec{0}\vec{P}=0
\end{align}}\\
Proof that&So in this case $\vec{N}$ itself is the zero matrix.\\
$\vec{N}=0$&This contradicts the assumption that $\myvec{a\\c},\myvec{b\\d}$ are linearly independent.\\
&$\therefore$ we can assume that $\myvec{a\\c},\myvec{b\\d}$ are linearly dependent if both are\\
&equal to the zero vector\\
&\parbox{6cm}
{\begin{align}
   \mbox{then } \vec{N} &= 0.
\end{align}}\\
%\hline
%\pagebreak
\hline
&\\
&Therefore we can assume at least one vector is non-zero.\\
Assuming $\myvec{b\\d}$ as&Therefore
$\vec{N}=\myvec{a&0\\c&0}$\\
the zero vector&\\
&\parbox{6cm}{\begin{align}
\mbox{So }\vec{N}^2&=0\\
\implies a^2&=0\\
\therefore a&=0\\
\mbox{Thus }\vec{N}&=\myvec{a&0\\c&0}
\end{align}}\\
&In this case $\vec{N}$ is similar to $\vec{N}=\myvec{0&0\\1&0}$ via the matrix $\vec{P}=\myvec{c&0\\0&1}$\\
&\\
\hline
&\\
Assuming $\myvec{a\\c}$ as&Therefore
$\vec{N}=\myvec{0&b\\0&d}$\\
the zero vector&\\
&\parbox{6cm}{\begin{align}
\mbox{Then }\vec{N}^2&=0\\
\implies d^2&=0\\
\therefore d&=0\\
\mbox{Thus }\vec{N}&=\myvec{0&b\\0&0}
\end{align}}\\
&In this case $\vec{N}$ is similar to $\vec{N}=\myvec{0&0\\b&0}$ via the matrix $\vec{P}=\myvec{0&1\\1&0}$,\\
&which is similar to $\myvec{0&0\\1&0}$ as above.\\
&\\
\hline
&\\
Hence& we can assume neither $\myvec{a\\c}$ or $\myvec{b\\d}$ is the zero vector.\\
&\\
%\hline
%\pagebreak
\hline
&\\
Consequences of &Since they are linearly dependent we can assume,\\
linear&\\
independence&\\
&\parbox{6cm}{\begin{align}
\myvec{b\\d}&=x\myvec{a\\c}\\
\therefore \vec{N}&=\myvec{a&ax\\c&cx}\\
\therefore \vec{N}^2&=0\\
\implies a(a+cx)&=0\\
c(a+cx)&=0\\
ax(a+cx)&=0\\
cx(a+cx)&=0
\end{align}}\\
\hline
&\\
Proof that $\vec{N}$ is&We know that at least one of a or c is not zero.\\
similar over $\mathbb{C}$ to&If a = 0 then c $\neq$ 0, it must be that x = 0.\\
$\myvec{0&0\\1&0}$&So in this case $\vec{N}=\myvec{0&0\\c&0}$ which is similar to $\myvec{0&0\\1&0}$ as before.\\
&\\
&\parbox{6cm}{\begin{align}
\mbox{If } a &\neq0\\
\mbox{then }x &\neq0\\
\mbox{else }a(a+cx)&=0\\
\implies a&=0\\
\mbox{Thus }a+cx&=0\\
\mbox{Hence }\vec{N}&=\myvec{a&ax\\ \frac{-a}{x}&-a}
\end{align}}\\
 &This is similar to $\myvec{a&a\\-a&-a}$ via $\vec{P}=\myvec{\sqrt{x}&0\\0&\frac{1}{\sqrt{x}}}$.\\
 &And $\myvec{a&a\\-a&-a}$ is similar to $\myvec{0&0\\-a&0}$ via $\vec{P}=\myvec{-1&-1\\1&0}$\\
 &And this finally is similar to $\myvec{0&0\\1&0}$ as before.\\
 &\\
\hline
&\\
Conclusion &Thus either $\vec{N}$ = 0 or $\vec{N}$ is similar over $\mathbb{C}$ to $\myvec{0&0\\1&0}$.\\
&\\
\hline
\caption{Solution summary}
\label{eq:solutions/6/2/11/table:1}
\end{longtable}

%\twocolumn
\end{enumerate}

\subsection{The Primary Decomposition Theorem}
\renewcommand{\theequation}{\theenumi}
\renewcommand{\thefigure}{\theenumi}
\begin{enumerate}[label=\thesubsection.\arabic*.,ref=\thesubsection.\theenumi]
\numberwithin{equation}{enumi}
\numberwithin{figure}{enumi}

\item %
Let $T$ be a linear operator on $R^3$ which is represented in the standard ordered basis by the matrix 
\begin{align}
    \myvec{6&-3&-2\\
           4&-1&-2\\
           10&-5&-3}
\end{align}
Express the minimal polynomial $p$ for $T$ in the form $p$ = $p_1p_2$, where $p_1$ and $p_2$ are monic and irreducible over the field of real numbers. Let $W_i$ be the null space of $p_i\brak{T}$. Find the basis $B_i$ for the spaces $W_1$ and $W_2$. If $T_i$ is the operator induced on $W_i$ by $T_i$, find the matrix of $T_i$ in the basis $B_i$ above.
%
%
\solution
See Table \ref{eq:solutions/6/2/11/table:1}
\onecolumn
\begin{longtable}{|l|l|}
\hline
\multirow{3}{*}{} & \\
Statement&Solution\\
\hline
&\parbox{6cm}{\begin{align}
\mbox{Let }\vec{N}&=\myvec{a&b\\c&d}\\
\mbox{Since }\vec{N}^2&=0
\end{align}}\\
&If $\myvec{a\\c},\myvec{b\\d}$ are linearly independent then $\vec{N}$ is diagonalizable to $\myvec{0&0\\0&0}$.\\
&\parbox{6cm}{\begin{align}
\mbox{If }\vec{P}\vec{N}\vec{P}^{-1}=0\\
\mbox{then }\vec{N}=\vec{P}^{-1}\vec{0}\vec{P}=0
\end{align}}\\
Proof that&So in this case $\vec{N}$ itself is the zero matrix.\\
$\vec{N}=0$&This contradicts the assumption that $\myvec{a\\c},\myvec{b\\d}$ are linearly independent.\\
&$\therefore$ we can assume that $\myvec{a\\c},\myvec{b\\d}$ are linearly dependent if both are\\
&equal to the zero vector\\
&\parbox{6cm}
{\begin{align}
   \mbox{then } \vec{N} &= 0.
\end{align}}\\
%\hline
%\pagebreak
\hline
&\\
&Therefore we can assume at least one vector is non-zero.\\
Assuming $\myvec{b\\d}$ as&Therefore
$\vec{N}=\myvec{a&0\\c&0}$\\
the zero vector&\\
&\parbox{6cm}{\begin{align}
\mbox{So }\vec{N}^2&=0\\
\implies a^2&=0\\
\therefore a&=0\\
\mbox{Thus }\vec{N}&=\myvec{a&0\\c&0}
\end{align}}\\
&In this case $\vec{N}$ is similar to $\vec{N}=\myvec{0&0\\1&0}$ via the matrix $\vec{P}=\myvec{c&0\\0&1}$\\
&\\
\hline
&\\
Assuming $\myvec{a\\c}$ as&Therefore
$\vec{N}=\myvec{0&b\\0&d}$\\
the zero vector&\\
&\parbox{6cm}{\begin{align}
\mbox{Then }\vec{N}^2&=0\\
\implies d^2&=0\\
\therefore d&=0\\
\mbox{Thus }\vec{N}&=\myvec{0&b\\0&0}
\end{align}}\\
&In this case $\vec{N}$ is similar to $\vec{N}=\myvec{0&0\\b&0}$ via the matrix $\vec{P}=\myvec{0&1\\1&0}$,\\
&which is similar to $\myvec{0&0\\1&0}$ as above.\\
&\\
\hline
&\\
Hence& we can assume neither $\myvec{a\\c}$ or $\myvec{b\\d}$ is the zero vector.\\
&\\
%\hline
%\pagebreak
\hline
&\\
Consequences of &Since they are linearly dependent we can assume,\\
linear&\\
independence&\\
&\parbox{6cm}{\begin{align}
\myvec{b\\d}&=x\myvec{a\\c}\\
\therefore \vec{N}&=\myvec{a&ax\\c&cx}\\
\therefore \vec{N}^2&=0\\
\implies a(a+cx)&=0\\
c(a+cx)&=0\\
ax(a+cx)&=0\\
cx(a+cx)&=0
\end{align}}\\
\hline
&\\
Proof that $\vec{N}$ is&We know that at least one of a or c is not zero.\\
similar over $\mathbb{C}$ to&If a = 0 then c $\neq$ 0, it must be that x = 0.\\
$\myvec{0&0\\1&0}$&So in this case $\vec{N}=\myvec{0&0\\c&0}$ which is similar to $\myvec{0&0\\1&0}$ as before.\\
&\\
&\parbox{6cm}{\begin{align}
\mbox{If } a &\neq0\\
\mbox{then }x &\neq0\\
\mbox{else }a(a+cx)&=0\\
\implies a&=0\\
\mbox{Thus }a+cx&=0\\
\mbox{Hence }\vec{N}&=\myvec{a&ax\\ \frac{-a}{x}&-a}
\end{align}}\\
 &This is similar to $\myvec{a&a\\-a&-a}$ via $\vec{P}=\myvec{\sqrt{x}&0\\0&\frac{1}{\sqrt{x}}}$.\\
 &And $\myvec{a&a\\-a&-a}$ is similar to $\myvec{0&0\\-a&0}$ via $\vec{P}=\myvec{-1&-1\\1&0}$\\
 &And this finally is similar to $\myvec{0&0\\1&0}$ as before.\\
 &\\
\hline
&\\
Conclusion &Thus either $\vec{N}$ = 0 or $\vec{N}$ is similar over $\mathbb{C}$ to $\myvec{0&0\\1&0}$.\\
&\\
\hline
\caption{Solution summary}
\label{eq:solutions/6/2/11/table:1}
\end{longtable}

\item Give an example of two $4\times 4$ nilpotent matrices which have the same minimal polynomial(they necessarily have the same characteristic polynomial) but which are not similar. 
%
\solution
See Table \ref{eq:solutions/6/2/11/table:1}
\onecolumn
\begin{longtable}{|l|l|}
\hline
\multirow{3}{*}{} & \\
Statement&Solution\\
\hline
&\parbox{6cm}{\begin{align}
\mbox{Let }\vec{N}&=\myvec{a&b\\c&d}\\
\mbox{Since }\vec{N}^2&=0
\end{align}}\\
&If $\myvec{a\\c},\myvec{b\\d}$ are linearly independent then $\vec{N}$ is diagonalizable to $\myvec{0&0\\0&0}$.\\
&\parbox{6cm}{\begin{align}
\mbox{If }\vec{P}\vec{N}\vec{P}^{-1}=0\\
\mbox{then }\vec{N}=\vec{P}^{-1}\vec{0}\vec{P}=0
\end{align}}\\
Proof that&So in this case $\vec{N}$ itself is the zero matrix.\\
$\vec{N}=0$&This contradicts the assumption that $\myvec{a\\c},\myvec{b\\d}$ are linearly independent.\\
&$\therefore$ we can assume that $\myvec{a\\c},\myvec{b\\d}$ are linearly dependent if both are\\
&equal to the zero vector\\
&\parbox{6cm}
{\begin{align}
   \mbox{then } \vec{N} &= 0.
\end{align}}\\
%\hline
%\pagebreak
\hline
&\\
&Therefore we can assume at least one vector is non-zero.\\
Assuming $\myvec{b\\d}$ as&Therefore
$\vec{N}=\myvec{a&0\\c&0}$\\
the zero vector&\\
&\parbox{6cm}{\begin{align}
\mbox{So }\vec{N}^2&=0\\
\implies a^2&=0\\
\therefore a&=0\\
\mbox{Thus }\vec{N}&=\myvec{a&0\\c&0}
\end{align}}\\
&In this case $\vec{N}$ is similar to $\vec{N}=\myvec{0&0\\1&0}$ via the matrix $\vec{P}=\myvec{c&0\\0&1}$\\
&\\
\hline
&\\
Assuming $\myvec{a\\c}$ as&Therefore
$\vec{N}=\myvec{0&b\\0&d}$\\
the zero vector&\\
&\parbox{6cm}{\begin{align}
\mbox{Then }\vec{N}^2&=0\\
\implies d^2&=0\\
\therefore d&=0\\
\mbox{Thus }\vec{N}&=\myvec{0&b\\0&0}
\end{align}}\\
&In this case $\vec{N}$ is similar to $\vec{N}=\myvec{0&0\\b&0}$ via the matrix $\vec{P}=\myvec{0&1\\1&0}$,\\
&which is similar to $\myvec{0&0\\1&0}$ as above.\\
&\\
\hline
&\\
Hence& we can assume neither $\myvec{a\\c}$ or $\myvec{b\\d}$ is the zero vector.\\
&\\
%\hline
%\pagebreak
\hline
&\\
Consequences of &Since they are linearly dependent we can assume,\\
linear&\\
independence&\\
&\parbox{6cm}{\begin{align}
\myvec{b\\d}&=x\myvec{a\\c}\\
\therefore \vec{N}&=\myvec{a&ax\\c&cx}\\
\therefore \vec{N}^2&=0\\
\implies a(a+cx)&=0\\
c(a+cx)&=0\\
ax(a+cx)&=0\\
cx(a+cx)&=0
\end{align}}\\
\hline
&\\
Proof that $\vec{N}$ is&We know that at least one of a or c is not zero.\\
similar over $\mathbb{C}$ to&If a = 0 then c $\neq$ 0, it must be that x = 0.\\
$\myvec{0&0\\1&0}$&So in this case $\vec{N}=\myvec{0&0\\c&0}$ which is similar to $\myvec{0&0\\1&0}$ as before.\\
&\\
&\parbox{6cm}{\begin{align}
\mbox{If } a &\neq0\\
\mbox{then }x &\neq0\\
\mbox{else }a(a+cx)&=0\\
\implies a&=0\\
\mbox{Thus }a+cx&=0\\
\mbox{Hence }\vec{N}&=\myvec{a&ax\\ \frac{-a}{x}&-a}
\end{align}}\\
 &This is similar to $\myvec{a&a\\-a&-a}$ via $\vec{P}=\myvec{\sqrt{x}&0\\0&\frac{1}{\sqrt{x}}}$.\\
 &And $\myvec{a&a\\-a&-a}$ is similar to $\myvec{0&0\\-a&0}$ via $\vec{P}=\myvec{-1&-1\\1&0}$\\
 &And this finally is similar to $\myvec{0&0\\1&0}$ as before.\\
 &\\
\hline
&\\
Conclusion &Thus either $\vec{N}$ = 0 or $\vec{N}$ is similar over $\mathbb{C}$ to $\myvec{0&0\\1&0}$.\\
&\\
\hline
\caption{Solution summary}
\label{eq:solutions/6/2/11/table:1}
\end{longtable}

\end{enumerate}

\section{The Rational and Jordan Forms}
\subsection{Cyclic Decompositions and the Rational Form}
\renewcommand{\theequation}{\theenumi}
\renewcommand{\thefigure}{\theenumi}
\begin{enumerate}[label=\thesubsection.\arabic*.,ref=\thesubsection.\theenumi]
\numberwithin{equation}{enumi}
\numberwithin{figure}{enumi}

\item Let, $\vec{T}$ be the linear operator on $\mathbb{R}^3$ which is represented in the standard ordered
basis by the matrix
\begin{align}
	\myvec{2 & 0 & 0 \\ 1 & 2 & 0 \\ 0 & 0 & 3}
\end{align}
Let $\vec{W}$ be the null space of $\vec{T-2I}$. Prove that $\vec{W}$ has no complementary $\vec{T}$-invariant
subspace.
%
\\
\solution
See Table \ref{eq:solutions/6/2/11/table:1}
\onecolumn
\begin{longtable}{|l|l|}
\hline
\multirow{3}{*}{} & \\
Statement&Solution\\
\hline
&\parbox{6cm}{\begin{align}
\mbox{Let }\vec{N}&=\myvec{a&b\\c&d}\\
\mbox{Since }\vec{N}^2&=0
\end{align}}\\
&If $\myvec{a\\c},\myvec{b\\d}$ are linearly independent then $\vec{N}$ is diagonalizable to $\myvec{0&0\\0&0}$.\\
&\parbox{6cm}{\begin{align}
\mbox{If }\vec{P}\vec{N}\vec{P}^{-1}=0\\
\mbox{then }\vec{N}=\vec{P}^{-1}\vec{0}\vec{P}=0
\end{align}}\\
Proof that&So in this case $\vec{N}$ itself is the zero matrix.\\
$\vec{N}=0$&This contradicts the assumption that $\myvec{a\\c},\myvec{b\\d}$ are linearly independent.\\
&$\therefore$ we can assume that $\myvec{a\\c},\myvec{b\\d}$ are linearly dependent if both are\\
&equal to the zero vector\\
&\parbox{6cm}
{\begin{align}
   \mbox{then } \vec{N} &= 0.
\end{align}}\\
%\hline
%\pagebreak
\hline
&\\
&Therefore we can assume at least one vector is non-zero.\\
Assuming $\myvec{b\\d}$ as&Therefore
$\vec{N}=\myvec{a&0\\c&0}$\\
the zero vector&\\
&\parbox{6cm}{\begin{align}
\mbox{So }\vec{N}^2&=0\\
\implies a^2&=0\\
\therefore a&=0\\
\mbox{Thus }\vec{N}&=\myvec{a&0\\c&0}
\end{align}}\\
&In this case $\vec{N}$ is similar to $\vec{N}=\myvec{0&0\\1&0}$ via the matrix $\vec{P}=\myvec{c&0\\0&1}$\\
&\\
\hline
&\\
Assuming $\myvec{a\\c}$ as&Therefore
$\vec{N}=\myvec{0&b\\0&d}$\\
the zero vector&\\
&\parbox{6cm}{\begin{align}
\mbox{Then }\vec{N}^2&=0\\
\implies d^2&=0\\
\therefore d&=0\\
\mbox{Thus }\vec{N}&=\myvec{0&b\\0&0}
\end{align}}\\
&In this case $\vec{N}$ is similar to $\vec{N}=\myvec{0&0\\b&0}$ via the matrix $\vec{P}=\myvec{0&1\\1&0}$,\\
&which is similar to $\myvec{0&0\\1&0}$ as above.\\
&\\
\hline
&\\
Hence& we can assume neither $\myvec{a\\c}$ or $\myvec{b\\d}$ is the zero vector.\\
&\\
%\hline
%\pagebreak
\hline
&\\
Consequences of &Since they are linearly dependent we can assume,\\
linear&\\
independence&\\
&\parbox{6cm}{\begin{align}
\myvec{b\\d}&=x\myvec{a\\c}\\
\therefore \vec{N}&=\myvec{a&ax\\c&cx}\\
\therefore \vec{N}^2&=0\\
\implies a(a+cx)&=0\\
c(a+cx)&=0\\
ax(a+cx)&=0\\
cx(a+cx)&=0
\end{align}}\\
\hline
&\\
Proof that $\vec{N}$ is&We know that at least one of a or c is not zero.\\
similar over $\mathbb{C}$ to&If a = 0 then c $\neq$ 0, it must be that x = 0.\\
$\myvec{0&0\\1&0}$&So in this case $\vec{N}=\myvec{0&0\\c&0}$ which is similar to $\myvec{0&0\\1&0}$ as before.\\
&\\
&\parbox{6cm}{\begin{align}
\mbox{If } a &\neq0\\
\mbox{then }x &\neq0\\
\mbox{else }a(a+cx)&=0\\
\implies a&=0\\
\mbox{Thus }a+cx&=0\\
\mbox{Hence }\vec{N}&=\myvec{a&ax\\ \frac{-a}{x}&-a}
\end{align}}\\
 &This is similar to $\myvec{a&a\\-a&-a}$ via $\vec{P}=\myvec{\sqrt{x}&0\\0&\frac{1}{\sqrt{x}}}$.\\
 &And $\myvec{a&a\\-a&-a}$ is similar to $\myvec{0&0\\-a&0}$ via $\vec{P}=\myvec{-1&-1\\1&0}$\\
 &And this finally is similar to $\myvec{0&0\\1&0}$ as before.\\
 &\\
\hline
&\\
Conclusion &Thus either $\vec{N}$ = 0 or $\vec{N}$ is similar over $\mathbb{C}$ to $\myvec{0&0\\1&0}$.\\
&\\
\hline
\caption{Solution summary}
\label{eq:solutions/6/2/11/table:1}
\end{longtable}

\end{enumerate}

\subsection{The Jordan Form}
\renewcommand{\theequation}{\theenumi}
\renewcommand{\thefigure}{\theenumi}
\begin{enumerate}[label=\thesubsection.\arabic*.,ref=\thesubsection.\theenumi]
\numberwithin{equation}{enumi}
\numberwithin{figure}{enumi}

\item If A is a complex $5 \times 5$ matrix with characteristic polynomial $f = (x-2)^3 (x+7)^2$ and minimal polynomial $p = (x-2)^2 (x+7)$, what is the Jordan form for A? 
%
\\
\solution
See Table \ref{eq:solutions/6/2/11/table:1}
\onecolumn
\begin{longtable}{|l|l|}
\hline
\multirow{3}{*}{} & \\
Statement&Solution\\
\hline
&\parbox{6cm}{\begin{align}
\mbox{Let }\vec{N}&=\myvec{a&b\\c&d}\\
\mbox{Since }\vec{N}^2&=0
\end{align}}\\
&If $\myvec{a\\c},\myvec{b\\d}$ are linearly independent then $\vec{N}$ is diagonalizable to $\myvec{0&0\\0&0}$.\\
&\parbox{6cm}{\begin{align}
\mbox{If }\vec{P}\vec{N}\vec{P}^{-1}=0\\
\mbox{then }\vec{N}=\vec{P}^{-1}\vec{0}\vec{P}=0
\end{align}}\\
Proof that&So in this case $\vec{N}$ itself is the zero matrix.\\
$\vec{N}=0$&This contradicts the assumption that $\myvec{a\\c},\myvec{b\\d}$ are linearly independent.\\
&$\therefore$ we can assume that $\myvec{a\\c},\myvec{b\\d}$ are linearly dependent if both are\\
&equal to the zero vector\\
&\parbox{6cm}
{\begin{align}
   \mbox{then } \vec{N} &= 0.
\end{align}}\\
%\hline
%\pagebreak
\hline
&\\
&Therefore we can assume at least one vector is non-zero.\\
Assuming $\myvec{b\\d}$ as&Therefore
$\vec{N}=\myvec{a&0\\c&0}$\\
the zero vector&\\
&\parbox{6cm}{\begin{align}
\mbox{So }\vec{N}^2&=0\\
\implies a^2&=0\\
\therefore a&=0\\
\mbox{Thus }\vec{N}&=\myvec{a&0\\c&0}
\end{align}}\\
&In this case $\vec{N}$ is similar to $\vec{N}=\myvec{0&0\\1&0}$ via the matrix $\vec{P}=\myvec{c&0\\0&1}$\\
&\\
\hline
&\\
Assuming $\myvec{a\\c}$ as&Therefore
$\vec{N}=\myvec{0&b\\0&d}$\\
the zero vector&\\
&\parbox{6cm}{\begin{align}
\mbox{Then }\vec{N}^2&=0\\
\implies d^2&=0\\
\therefore d&=0\\
\mbox{Thus }\vec{N}&=\myvec{0&b\\0&0}
\end{align}}\\
&In this case $\vec{N}$ is similar to $\vec{N}=\myvec{0&0\\b&0}$ via the matrix $\vec{P}=\myvec{0&1\\1&0}$,\\
&which is similar to $\myvec{0&0\\1&0}$ as above.\\
&\\
\hline
&\\
Hence& we can assume neither $\myvec{a\\c}$ or $\myvec{b\\d}$ is the zero vector.\\
&\\
%\hline
%\pagebreak
\hline
&\\
Consequences of &Since they are linearly dependent we can assume,\\
linear&\\
independence&\\
&\parbox{6cm}{\begin{align}
\myvec{b\\d}&=x\myvec{a\\c}\\
\therefore \vec{N}&=\myvec{a&ax\\c&cx}\\
\therefore \vec{N}^2&=0\\
\implies a(a+cx)&=0\\
c(a+cx)&=0\\
ax(a+cx)&=0\\
cx(a+cx)&=0
\end{align}}\\
\hline
&\\
Proof that $\vec{N}$ is&We know that at least one of a or c is not zero.\\
similar over $\mathbb{C}$ to&If a = 0 then c $\neq$ 0, it must be that x = 0.\\
$\myvec{0&0\\1&0}$&So in this case $\vec{N}=\myvec{0&0\\c&0}$ which is similar to $\myvec{0&0\\1&0}$ as before.\\
&\\
&\parbox{6cm}{\begin{align}
\mbox{If } a &\neq0\\
\mbox{then }x &\neq0\\
\mbox{else }a(a+cx)&=0\\
\implies a&=0\\
\mbox{Thus }a+cx&=0\\
\mbox{Hence }\vec{N}&=\myvec{a&ax\\ \frac{-a}{x}&-a}
\end{align}}\\
 &This is similar to $\myvec{a&a\\-a&-a}$ via $\vec{P}=\myvec{\sqrt{x}&0\\0&\frac{1}{\sqrt{x}}}$.\\
 &And $\myvec{a&a\\-a&-a}$ is similar to $\myvec{0&0\\-a&0}$ via $\vec{P}=\myvec{-1&-1\\1&0}$\\
 &And this finally is similar to $\myvec{0&0\\1&0}$ as before.\\
 &\\
\hline
&\\
Conclusion &Thus either $\vec{N}$ = 0 or $\vec{N}$ is similar over $\mathbb{C}$ to $\myvec{0&0\\1&0}$.\\
&\\
\hline
\caption{Solution summary}
\label{eq:solutions/6/2/11/table:1}
\end{longtable}

\item How many possible Jordan forms are there for a $6\times6$ complex matrix with characteristic polynomial $\brak{x+2}^4\brak{x-1}^2$?
%
\\
\solution
See Table \ref{eq:solutions/6/2/11/table:1}
\onecolumn
\begin{longtable}{|l|l|}
\hline
\multirow{3}{*}{} & \\
Statement&Solution\\
\hline
&\parbox{6cm}{\begin{align}
\mbox{Let }\vec{N}&=\myvec{a&b\\c&d}\\
\mbox{Since }\vec{N}^2&=0
\end{align}}\\
&If $\myvec{a\\c},\myvec{b\\d}$ are linearly independent then $\vec{N}$ is diagonalizable to $\myvec{0&0\\0&0}$.\\
&\parbox{6cm}{\begin{align}
\mbox{If }\vec{P}\vec{N}\vec{P}^{-1}=0\\
\mbox{then }\vec{N}=\vec{P}^{-1}\vec{0}\vec{P}=0
\end{align}}\\
Proof that&So in this case $\vec{N}$ itself is the zero matrix.\\
$\vec{N}=0$&This contradicts the assumption that $\myvec{a\\c},\myvec{b\\d}$ are linearly independent.\\
&$\therefore$ we can assume that $\myvec{a\\c},\myvec{b\\d}$ are linearly dependent if both are\\
&equal to the zero vector\\
&\parbox{6cm}
{\begin{align}
   \mbox{then } \vec{N} &= 0.
\end{align}}\\
%\hline
%\pagebreak
\hline
&\\
&Therefore we can assume at least one vector is non-zero.\\
Assuming $\myvec{b\\d}$ as&Therefore
$\vec{N}=\myvec{a&0\\c&0}$\\
the zero vector&\\
&\parbox{6cm}{\begin{align}
\mbox{So }\vec{N}^2&=0\\
\implies a^2&=0\\
\therefore a&=0\\
\mbox{Thus }\vec{N}&=\myvec{a&0\\c&0}
\end{align}}\\
&In this case $\vec{N}$ is similar to $\vec{N}=\myvec{0&0\\1&0}$ via the matrix $\vec{P}=\myvec{c&0\\0&1}$\\
&\\
\hline
&\\
Assuming $\myvec{a\\c}$ as&Therefore
$\vec{N}=\myvec{0&b\\0&d}$\\
the zero vector&\\
&\parbox{6cm}{\begin{align}
\mbox{Then }\vec{N}^2&=0\\
\implies d^2&=0\\
\therefore d&=0\\
\mbox{Thus }\vec{N}&=\myvec{0&b\\0&0}
\end{align}}\\
&In this case $\vec{N}$ is similar to $\vec{N}=\myvec{0&0\\b&0}$ via the matrix $\vec{P}=\myvec{0&1\\1&0}$,\\
&which is similar to $\myvec{0&0\\1&0}$ as above.\\
&\\
\hline
&\\
Hence& we can assume neither $\myvec{a\\c}$ or $\myvec{b\\d}$ is the zero vector.\\
&\\
%\hline
%\pagebreak
\hline
&\\
Consequences of &Since they are linearly dependent we can assume,\\
linear&\\
independence&\\
&\parbox{6cm}{\begin{align}
\myvec{b\\d}&=x\myvec{a\\c}\\
\therefore \vec{N}&=\myvec{a&ax\\c&cx}\\
\therefore \vec{N}^2&=0\\
\implies a(a+cx)&=0\\
c(a+cx)&=0\\
ax(a+cx)&=0\\
cx(a+cx)&=0
\end{align}}\\
\hline
&\\
Proof that $\vec{N}$ is&We know that at least one of a or c is not zero.\\
similar over $\mathbb{C}$ to&If a = 0 then c $\neq$ 0, it must be that x = 0.\\
$\myvec{0&0\\1&0}$&So in this case $\vec{N}=\myvec{0&0\\c&0}$ which is similar to $\myvec{0&0\\1&0}$ as before.\\
&\\
&\parbox{6cm}{\begin{align}
\mbox{If } a &\neq0\\
\mbox{then }x &\neq0\\
\mbox{else }a(a+cx)&=0\\
\implies a&=0\\
\mbox{Thus }a+cx&=0\\
\mbox{Hence }\vec{N}&=\myvec{a&ax\\ \frac{-a}{x}&-a}
\end{align}}\\
 &This is similar to $\myvec{a&a\\-a&-a}$ via $\vec{P}=\myvec{\sqrt{x}&0\\0&\frac{1}{\sqrt{x}}}$.\\
 &And $\myvec{a&a\\-a&-a}$ is similar to $\myvec{0&0\\-a&0}$ via $\vec{P}=\myvec{-1&-1\\1&0}$\\
 &And this finally is similar to $\myvec{0&0\\1&0}$ as before.\\
 &\\
\hline
&\\
Conclusion &Thus either $\vec{N}$ = 0 or $\vec{N}$ is similar over $\mathbb{C}$ to $\myvec{0&0\\1&0}$.\\
&\\
\hline
\caption{Solution summary}
\label{eq:solutions/6/2/11/table:1}
\end{longtable}


\item The differentiation operator on the space of polynomials of degree less than or equal to 3 is represented in the natural ordered basis by the matrix,
\begin{align}
\vec{A} = \myvec{0&1&0&0\\0&0&2&0\\0&0&0&3\\0&0&0&0}
\end{align}
What is the Jordan form of this matrix? ($\mathbb{F}$ a subfield of the complex numbers.) 
%
%
\\
\solution
See Table \ref{eq:solutions/6/2/11/table:1}
\onecolumn
\begin{longtable}{|l|l|}
\hline
\multirow{3}{*}{} & \\
Statement&Solution\\
\hline
&\parbox{6cm}{\begin{align}
\mbox{Let }\vec{N}&=\myvec{a&b\\c&d}\\
\mbox{Since }\vec{N}^2&=0
\end{align}}\\
&If $\myvec{a\\c},\myvec{b\\d}$ are linearly independent then $\vec{N}$ is diagonalizable to $\myvec{0&0\\0&0}$.\\
&\parbox{6cm}{\begin{align}
\mbox{If }\vec{P}\vec{N}\vec{P}^{-1}=0\\
\mbox{then }\vec{N}=\vec{P}^{-1}\vec{0}\vec{P}=0
\end{align}}\\
Proof that&So in this case $\vec{N}$ itself is the zero matrix.\\
$\vec{N}=0$&This contradicts the assumption that $\myvec{a\\c},\myvec{b\\d}$ are linearly independent.\\
&$\therefore$ we can assume that $\myvec{a\\c},\myvec{b\\d}$ are linearly dependent if both are\\
&equal to the zero vector\\
&\parbox{6cm}
{\begin{align}
   \mbox{then } \vec{N} &= 0.
\end{align}}\\
%\hline
%\pagebreak
\hline
&\\
&Therefore we can assume at least one vector is non-zero.\\
Assuming $\myvec{b\\d}$ as&Therefore
$\vec{N}=\myvec{a&0\\c&0}$\\
the zero vector&\\
&\parbox{6cm}{\begin{align}
\mbox{So }\vec{N}^2&=0\\
\implies a^2&=0\\
\therefore a&=0\\
\mbox{Thus }\vec{N}&=\myvec{a&0\\c&0}
\end{align}}\\
&In this case $\vec{N}$ is similar to $\vec{N}=\myvec{0&0\\1&0}$ via the matrix $\vec{P}=\myvec{c&0\\0&1}$\\
&\\
\hline
&\\
Assuming $\myvec{a\\c}$ as&Therefore
$\vec{N}=\myvec{0&b\\0&d}$\\
the zero vector&\\
&\parbox{6cm}{\begin{align}
\mbox{Then }\vec{N}^2&=0\\
\implies d^2&=0\\
\therefore d&=0\\
\mbox{Thus }\vec{N}&=\myvec{0&b\\0&0}
\end{align}}\\
&In this case $\vec{N}$ is similar to $\vec{N}=\myvec{0&0\\b&0}$ via the matrix $\vec{P}=\myvec{0&1\\1&0}$,\\
&which is similar to $\myvec{0&0\\1&0}$ as above.\\
&\\
\hline
&\\
Hence& we can assume neither $\myvec{a\\c}$ or $\myvec{b\\d}$ is the zero vector.\\
&\\
%\hline
%\pagebreak
\hline
&\\
Consequences of &Since they are linearly dependent we can assume,\\
linear&\\
independence&\\
&\parbox{6cm}{\begin{align}
\myvec{b\\d}&=x\myvec{a\\c}\\
\therefore \vec{N}&=\myvec{a&ax\\c&cx}\\
\therefore \vec{N}^2&=0\\
\implies a(a+cx)&=0\\
c(a+cx)&=0\\
ax(a+cx)&=0\\
cx(a+cx)&=0
\end{align}}\\
\hline
&\\
Proof that $\vec{N}$ is&We know that at least one of a or c is not zero.\\
similar over $\mathbb{C}$ to&If a = 0 then c $\neq$ 0, it must be that x = 0.\\
$\myvec{0&0\\1&0}$&So in this case $\vec{N}=\myvec{0&0\\c&0}$ which is similar to $\myvec{0&0\\1&0}$ as before.\\
&\\
&\parbox{6cm}{\begin{align}
\mbox{If } a &\neq0\\
\mbox{then }x &\neq0\\
\mbox{else }a(a+cx)&=0\\
\implies a&=0\\
\mbox{Thus }a+cx&=0\\
\mbox{Hence }\vec{N}&=\myvec{a&ax\\ \frac{-a}{x}&-a}
\end{align}}\\
 &This is similar to $\myvec{a&a\\-a&-a}$ via $\vec{P}=\myvec{\sqrt{x}&0\\0&\frac{1}{\sqrt{x}}}$.\\
 &And $\myvec{a&a\\-a&-a}$ is similar to $\myvec{0&0\\-a&0}$ via $\vec{P}=\myvec{-1&-1\\1&0}$\\
 &And this finally is similar to $\myvec{0&0\\1&0}$ as before.\\
 &\\
\hline
&\\
Conclusion &Thus either $\vec{N}$ = 0 or $\vec{N}$ is similar over $\mathbb{C}$ to $\myvec{0&0\\1&0}$.\\
&\\
\hline
\caption{Solution summary}
\label{eq:solutions/6/2/11/table:1}
\end{longtable}

\item Let $\vec{N_1}$ and $\vec{N_2}$ be $6 \times 6$ nilpotent matrices over the field $\vec{F}$. Suppose that $\vec{N_1}$ and $\vec{N_2}$ have the same minimal polynomial and the same nullity. Prove that $\vec{N_1}$ and $\vec{N_2}$ are similar. Show that this is not true for $7 \times 7$ nilpotent matrices.
%
\\
\solution
See Table \ref{eq:solutions/6/2/11/table:1}
\onecolumn
\begin{longtable}{|l|l|}
\hline
\multirow{3}{*}{} & \\
Statement&Solution\\
\hline
&\parbox{6cm}{\begin{align}
\mbox{Let }\vec{N}&=\myvec{a&b\\c&d}\\
\mbox{Since }\vec{N}^2&=0
\end{align}}\\
&If $\myvec{a\\c},\myvec{b\\d}$ are linearly independent then $\vec{N}$ is diagonalizable to $\myvec{0&0\\0&0}$.\\
&\parbox{6cm}{\begin{align}
\mbox{If }\vec{P}\vec{N}\vec{P}^{-1}=0\\
\mbox{then }\vec{N}=\vec{P}^{-1}\vec{0}\vec{P}=0
\end{align}}\\
Proof that&So in this case $\vec{N}$ itself is the zero matrix.\\
$\vec{N}=0$&This contradicts the assumption that $\myvec{a\\c},\myvec{b\\d}$ are linearly independent.\\
&$\therefore$ we can assume that $\myvec{a\\c},\myvec{b\\d}$ are linearly dependent if both are\\
&equal to the zero vector\\
&\parbox{6cm}
{\begin{align}
   \mbox{then } \vec{N} &= 0.
\end{align}}\\
%\hline
%\pagebreak
\hline
&\\
&Therefore we can assume at least one vector is non-zero.\\
Assuming $\myvec{b\\d}$ as&Therefore
$\vec{N}=\myvec{a&0\\c&0}$\\
the zero vector&\\
&\parbox{6cm}{\begin{align}
\mbox{So }\vec{N}^2&=0\\
\implies a^2&=0\\
\therefore a&=0\\
\mbox{Thus }\vec{N}&=\myvec{a&0\\c&0}
\end{align}}\\
&In this case $\vec{N}$ is similar to $\vec{N}=\myvec{0&0\\1&0}$ via the matrix $\vec{P}=\myvec{c&0\\0&1}$\\
&\\
\hline
&\\
Assuming $\myvec{a\\c}$ as&Therefore
$\vec{N}=\myvec{0&b\\0&d}$\\
the zero vector&\\
&\parbox{6cm}{\begin{align}
\mbox{Then }\vec{N}^2&=0\\
\implies d^2&=0\\
\therefore d&=0\\
\mbox{Thus }\vec{N}&=\myvec{0&b\\0&0}
\end{align}}\\
&In this case $\vec{N}$ is similar to $\vec{N}=\myvec{0&0\\b&0}$ via the matrix $\vec{P}=\myvec{0&1\\1&0}$,\\
&which is similar to $\myvec{0&0\\1&0}$ as above.\\
&\\
\hline
&\\
Hence& we can assume neither $\myvec{a\\c}$ or $\myvec{b\\d}$ is the zero vector.\\
&\\
%\hline
%\pagebreak
\hline
&\\
Consequences of &Since they are linearly dependent we can assume,\\
linear&\\
independence&\\
&\parbox{6cm}{\begin{align}
\myvec{b\\d}&=x\myvec{a\\c}\\
\therefore \vec{N}&=\myvec{a&ax\\c&cx}\\
\therefore \vec{N}^2&=0\\
\implies a(a+cx)&=0\\
c(a+cx)&=0\\
ax(a+cx)&=0\\
cx(a+cx)&=0
\end{align}}\\
\hline
&\\
Proof that $\vec{N}$ is&We know that at least one of a or c is not zero.\\
similar over $\mathbb{C}$ to&If a = 0 then c $\neq$ 0, it must be that x = 0.\\
$\myvec{0&0\\1&0}$&So in this case $\vec{N}=\myvec{0&0\\c&0}$ which is similar to $\myvec{0&0\\1&0}$ as before.\\
&\\
&\parbox{6cm}{\begin{align}
\mbox{If } a &\neq0\\
\mbox{then }x &\neq0\\
\mbox{else }a(a+cx)&=0\\
\implies a&=0\\
\mbox{Thus }a+cx&=0\\
\mbox{Hence }\vec{N}&=\myvec{a&ax\\ \frac{-a}{x}&-a}
\end{align}}\\
 &This is similar to $\myvec{a&a\\-a&-a}$ via $\vec{P}=\myvec{\sqrt{x}&0\\0&\frac{1}{\sqrt{x}}}$.\\
 &And $\myvec{a&a\\-a&-a}$ is similar to $\myvec{0&0\\-a&0}$ via $\vec{P}=\myvec{-1&-1\\1&0}$\\
 &And this finally is similar to $\myvec{0&0\\1&0}$ as before.\\
 &\\
\hline
&\\
Conclusion &Thus either $\vec{N}$ = 0 or $\vec{N}$ is similar over $\mathbb{C}$ to $\myvec{0&0\\1&0}$.\\
&\\
\hline
\caption{Solution summary}
\label{eq:solutions/6/2/11/table:1}
\end{longtable}

\twocolumn
\item If $\vec{N}$ is a nilpotent 3 $\times$ 3 matrix over C, prove that $\vec{A}$ = $\vec{I}$ + $\frac{1}{2}\vec{N}$ - $\frac{1}{8}\vec{N}^{2}$ satisfies $\vec{A}^2$ = $\vec{I}$ + $\vec{N}$, i.e., $\vec{A}$ is a square root of $\vec{I}$ + $\vec{N}$. Use the binomial series for $(1 + t)^\frac{1}{2}$ to obtain a similar formula for a square root of $\vec{I}$ + $\vec{N}$, where $\vec{N}$ is any nilpotent n $\times$ n matrix over C.
%
\\
\solution
See Table \ref{eq:solutions/6/2/11/table:1}
\onecolumn
\begin{longtable}{|l|l|}
\hline
\multirow{3}{*}{} & \\
Statement&Solution\\
\hline
&\parbox{6cm}{\begin{align}
\mbox{Let }\vec{N}&=\myvec{a&b\\c&d}\\
\mbox{Since }\vec{N}^2&=0
\end{align}}\\
&If $\myvec{a\\c},\myvec{b\\d}$ are linearly independent then $\vec{N}$ is diagonalizable to $\myvec{0&0\\0&0}$.\\
&\parbox{6cm}{\begin{align}
\mbox{If }\vec{P}\vec{N}\vec{P}^{-1}=0\\
\mbox{then }\vec{N}=\vec{P}^{-1}\vec{0}\vec{P}=0
\end{align}}\\
Proof that&So in this case $\vec{N}$ itself is the zero matrix.\\
$\vec{N}=0$&This contradicts the assumption that $\myvec{a\\c},\myvec{b\\d}$ are linearly independent.\\
&$\therefore$ we can assume that $\myvec{a\\c},\myvec{b\\d}$ are linearly dependent if both are\\
&equal to the zero vector\\
&\parbox{6cm}
{\begin{align}
   \mbox{then } \vec{N} &= 0.
\end{align}}\\
%\hline
%\pagebreak
\hline
&\\
&Therefore we can assume at least one vector is non-zero.\\
Assuming $\myvec{b\\d}$ as&Therefore
$\vec{N}=\myvec{a&0\\c&0}$\\
the zero vector&\\
&\parbox{6cm}{\begin{align}
\mbox{So }\vec{N}^2&=0\\
\implies a^2&=0\\
\therefore a&=0\\
\mbox{Thus }\vec{N}&=\myvec{a&0\\c&0}
\end{align}}\\
&In this case $\vec{N}$ is similar to $\vec{N}=\myvec{0&0\\1&0}$ via the matrix $\vec{P}=\myvec{c&0\\0&1}$\\
&\\
\hline
&\\
Assuming $\myvec{a\\c}$ as&Therefore
$\vec{N}=\myvec{0&b\\0&d}$\\
the zero vector&\\
&\parbox{6cm}{\begin{align}
\mbox{Then }\vec{N}^2&=0\\
\implies d^2&=0\\
\therefore d&=0\\
\mbox{Thus }\vec{N}&=\myvec{0&b\\0&0}
\end{align}}\\
&In this case $\vec{N}$ is similar to $\vec{N}=\myvec{0&0\\b&0}$ via the matrix $\vec{P}=\myvec{0&1\\1&0}$,\\
&which is similar to $\myvec{0&0\\1&0}$ as above.\\
&\\
\hline
&\\
Hence& we can assume neither $\myvec{a\\c}$ or $\myvec{b\\d}$ is the zero vector.\\
&\\
%\hline
%\pagebreak
\hline
&\\
Consequences of &Since they are linearly dependent we can assume,\\
linear&\\
independence&\\
&\parbox{6cm}{\begin{align}
\myvec{b\\d}&=x\myvec{a\\c}\\
\therefore \vec{N}&=\myvec{a&ax\\c&cx}\\
\therefore \vec{N}^2&=0\\
\implies a(a+cx)&=0\\
c(a+cx)&=0\\
ax(a+cx)&=0\\
cx(a+cx)&=0
\end{align}}\\
\hline
&\\
Proof that $\vec{N}$ is&We know that at least one of a or c is not zero.\\
similar over $\mathbb{C}$ to&If a = 0 then c $\neq$ 0, it must be that x = 0.\\
$\myvec{0&0\\1&0}$&So in this case $\vec{N}=\myvec{0&0\\c&0}$ which is similar to $\myvec{0&0\\1&0}$ as before.\\
&\\
&\parbox{6cm}{\begin{align}
\mbox{If } a &\neq0\\
\mbox{then }x &\neq0\\
\mbox{else }a(a+cx)&=0\\
\implies a&=0\\
\mbox{Thus }a+cx&=0\\
\mbox{Hence }\vec{N}&=\myvec{a&ax\\ \frac{-a}{x}&-a}
\end{align}}\\
 &This is similar to $\myvec{a&a\\-a&-a}$ via $\vec{P}=\myvec{\sqrt{x}&0\\0&\frac{1}{\sqrt{x}}}$.\\
 &And $\myvec{a&a\\-a&-a}$ is similar to $\myvec{0&0\\-a&0}$ via $\vec{P}=\myvec{-1&-1\\1&0}$\\
 &And this finally is similar to $\myvec{0&0\\1&0}$ as before.\\
 &\\
\hline
&\\
Conclusion &Thus either $\vec{N}$ = 0 or $\vec{N}$ is similar over $\mathbb{C}$ to $\myvec{0&0\\1&0}$.\\
&\\
\hline
\caption{Solution summary}
\label{eq:solutions/6/2/11/table:1}
\end{longtable}

\item Use the result of Exercise 15 (that is, if $\vec{A}=\vec{I}+\frac{1}{2}\vec{N}-\frac{1}{8}\vec{N}^2$ then $\vec{A}^2 = \vec{I}+\vec{N}$) to prove that if $c$ is a non-zero complex number and $\vec{N}$ is a nilpotent complex matrix, then $(c\vec{I}+\vec{N})$ has a square root. Now use the Jordon form to prove that every non-singular complex $n \times n$ matrix has a square root.
 %
\\
\solution
See Table \ref{eq:solutions/6/2/11/table:1}
\onecolumn
\begin{longtable}{|l|l|}
\hline
\multirow{3}{*}{} & \\
Statement&Solution\\
\hline
&\parbox{6cm}{\begin{align}
\mbox{Let }\vec{N}&=\myvec{a&b\\c&d}\\
\mbox{Since }\vec{N}^2&=0
\end{align}}\\
&If $\myvec{a\\c},\myvec{b\\d}$ are linearly independent then $\vec{N}$ is diagonalizable to $\myvec{0&0\\0&0}$.\\
&\parbox{6cm}{\begin{align}
\mbox{If }\vec{P}\vec{N}\vec{P}^{-1}=0\\
\mbox{then }\vec{N}=\vec{P}^{-1}\vec{0}\vec{P}=0
\end{align}}\\
Proof that&So in this case $\vec{N}$ itself is the zero matrix.\\
$\vec{N}=0$&This contradicts the assumption that $\myvec{a\\c},\myvec{b\\d}$ are linearly independent.\\
&$\therefore$ we can assume that $\myvec{a\\c},\myvec{b\\d}$ are linearly dependent if both are\\
&equal to the zero vector\\
&\parbox{6cm}
{\begin{align}
   \mbox{then } \vec{N} &= 0.
\end{align}}\\
%\hline
%\pagebreak
\hline
&\\
&Therefore we can assume at least one vector is non-zero.\\
Assuming $\myvec{b\\d}$ as&Therefore
$\vec{N}=\myvec{a&0\\c&0}$\\
the zero vector&\\
&\parbox{6cm}{\begin{align}
\mbox{So }\vec{N}^2&=0\\
\implies a^2&=0\\
\therefore a&=0\\
\mbox{Thus }\vec{N}&=\myvec{a&0\\c&0}
\end{align}}\\
&In this case $\vec{N}$ is similar to $\vec{N}=\myvec{0&0\\1&0}$ via the matrix $\vec{P}=\myvec{c&0\\0&1}$\\
&\\
\hline
&\\
Assuming $\myvec{a\\c}$ as&Therefore
$\vec{N}=\myvec{0&b\\0&d}$\\
the zero vector&\\
&\parbox{6cm}{\begin{align}
\mbox{Then }\vec{N}^2&=0\\
\implies d^2&=0\\
\therefore d&=0\\
\mbox{Thus }\vec{N}&=\myvec{0&b\\0&0}
\end{align}}\\
&In this case $\vec{N}$ is similar to $\vec{N}=\myvec{0&0\\b&0}$ via the matrix $\vec{P}=\myvec{0&1\\1&0}$,\\
&which is similar to $\myvec{0&0\\1&0}$ as above.\\
&\\
\hline
&\\
Hence& we can assume neither $\myvec{a\\c}$ or $\myvec{b\\d}$ is the zero vector.\\
&\\
%\hline
%\pagebreak
\hline
&\\
Consequences of &Since they are linearly dependent we can assume,\\
linear&\\
independence&\\
&\parbox{6cm}{\begin{align}
\myvec{b\\d}&=x\myvec{a\\c}\\
\therefore \vec{N}&=\myvec{a&ax\\c&cx}\\
\therefore \vec{N}^2&=0\\
\implies a(a+cx)&=0\\
c(a+cx)&=0\\
ax(a+cx)&=0\\
cx(a+cx)&=0
\end{align}}\\
\hline
&\\
Proof that $\vec{N}$ is&We know that at least one of a or c is not zero.\\
similar over $\mathbb{C}$ to&If a = 0 then c $\neq$ 0, it must be that x = 0.\\
$\myvec{0&0\\1&0}$&So in this case $\vec{N}=\myvec{0&0\\c&0}$ which is similar to $\myvec{0&0\\1&0}$ as before.\\
&\\
&\parbox{6cm}{\begin{align}
\mbox{If } a &\neq0\\
\mbox{then }x &\neq0\\
\mbox{else }a(a+cx)&=0\\
\implies a&=0\\
\mbox{Thus }a+cx&=0\\
\mbox{Hence }\vec{N}&=\myvec{a&ax\\ \frac{-a}{x}&-a}
\end{align}}\\
 &This is similar to $\myvec{a&a\\-a&-a}$ via $\vec{P}=\myvec{\sqrt{x}&0\\0&\frac{1}{\sqrt{x}}}$.\\
 &And $\myvec{a&a\\-a&-a}$ is similar to $\myvec{0&0\\-a&0}$ via $\vec{P}=\myvec{-1&-1\\1&0}$\\
 &And this finally is similar to $\myvec{0&0\\1&0}$ as before.\\
 &\\
\hline
&\\
Conclusion &Thus either $\vec{N}$ = 0 or $\vec{N}$ is similar over $\mathbb{C}$ to $\myvec{0&0\\1&0}$.\\
&\\
\hline
\caption{Solution summary}
\label{eq:solutions/6/2/11/table:1}
\end{longtable}



\end{enumerate}

\subsection{Computation of Invariant Factors}
\renewcommand{\theequation}{\theenumi}
\renewcommand{\thefigure}{\theenumi}
\begin{enumerate}[label=\thesubsection.\arabic*.,ref=\thesubsection.\theenumi]
\numberwithin{equation}{enumi}
\numberwithin{figure}{enumi}

\item Construct a linear operator $T$ with minimal polynomial $x^2(x-1)^2$ and characteristic polynomial $x^3(x-1)^4$.Describe the primary decomposition of the vector space under $T$ and find the projections on the primary components.Find a basis in which the matrix $T$ is in Jordan form.Also find an explicit direct sum decomposition of the space into $T$cyclic subspaces as in theorem 3 and give the invariant factors.
%
\\
\solution
See Table \ref{eq:solutions/6/2/11/table:1}
\onecolumn
\begin{longtable}{|l|l|}
\hline
\multirow{3}{*}{} & \\
Statement&Solution\\
\hline
&\parbox{6cm}{\begin{align}
\mbox{Let }\vec{N}&=\myvec{a&b\\c&d}\\
\mbox{Since }\vec{N}^2&=0
\end{align}}\\
&If $\myvec{a\\c},\myvec{b\\d}$ are linearly independent then $\vec{N}$ is diagonalizable to $\myvec{0&0\\0&0}$.\\
&\parbox{6cm}{\begin{align}
\mbox{If }\vec{P}\vec{N}\vec{P}^{-1}=0\\
\mbox{then }\vec{N}=\vec{P}^{-1}\vec{0}\vec{P}=0
\end{align}}\\
Proof that&So in this case $\vec{N}$ itself is the zero matrix.\\
$\vec{N}=0$&This contradicts the assumption that $\myvec{a\\c},\myvec{b\\d}$ are linearly independent.\\
&$\therefore$ we can assume that $\myvec{a\\c},\myvec{b\\d}$ are linearly dependent if both are\\
&equal to the zero vector\\
&\parbox{6cm}
{\begin{align}
   \mbox{then } \vec{N} &= 0.
\end{align}}\\
%\hline
%\pagebreak
\hline
&\\
&Therefore we can assume at least one vector is non-zero.\\
Assuming $\myvec{b\\d}$ as&Therefore
$\vec{N}=\myvec{a&0\\c&0}$\\
the zero vector&\\
&\parbox{6cm}{\begin{align}
\mbox{So }\vec{N}^2&=0\\
\implies a^2&=0\\
\therefore a&=0\\
\mbox{Thus }\vec{N}&=\myvec{a&0\\c&0}
\end{align}}\\
&In this case $\vec{N}$ is similar to $\vec{N}=\myvec{0&0\\1&0}$ via the matrix $\vec{P}=\myvec{c&0\\0&1}$\\
&\\
\hline
&\\
Assuming $\myvec{a\\c}$ as&Therefore
$\vec{N}=\myvec{0&b\\0&d}$\\
the zero vector&\\
&\parbox{6cm}{\begin{align}
\mbox{Then }\vec{N}^2&=0\\
\implies d^2&=0\\
\therefore d&=0\\
\mbox{Thus }\vec{N}&=\myvec{0&b\\0&0}
\end{align}}\\
&In this case $\vec{N}$ is similar to $\vec{N}=\myvec{0&0\\b&0}$ via the matrix $\vec{P}=\myvec{0&1\\1&0}$,\\
&which is similar to $\myvec{0&0\\1&0}$ as above.\\
&\\
\hline
&\\
Hence& we can assume neither $\myvec{a\\c}$ or $\myvec{b\\d}$ is the zero vector.\\
&\\
%\hline
%\pagebreak
\hline
&\\
Consequences of &Since they are linearly dependent we can assume,\\
linear&\\
independence&\\
&\parbox{6cm}{\begin{align}
\myvec{b\\d}&=x\myvec{a\\c}\\
\therefore \vec{N}&=\myvec{a&ax\\c&cx}\\
\therefore \vec{N}^2&=0\\
\implies a(a+cx)&=0\\
c(a+cx)&=0\\
ax(a+cx)&=0\\
cx(a+cx)&=0
\end{align}}\\
\hline
&\\
Proof that $\vec{N}$ is&We know that at least one of a or c is not zero.\\
similar over $\mathbb{C}$ to&If a = 0 then c $\neq$ 0, it must be that x = 0.\\
$\myvec{0&0\\1&0}$&So in this case $\vec{N}=\myvec{0&0\\c&0}$ which is similar to $\myvec{0&0\\1&0}$ as before.\\
&\\
&\parbox{6cm}{\begin{align}
\mbox{If } a &\neq0\\
\mbox{then }x &\neq0\\
\mbox{else }a(a+cx)&=0\\
\implies a&=0\\
\mbox{Thus }a+cx&=0\\
\mbox{Hence }\vec{N}&=\myvec{a&ax\\ \frac{-a}{x}&-a}
\end{align}}\\
 &This is similar to $\myvec{a&a\\-a&-a}$ via $\vec{P}=\myvec{\sqrt{x}&0\\0&\frac{1}{\sqrt{x}}}$.\\
 &And $\myvec{a&a\\-a&-a}$ is similar to $\myvec{0&0\\-a&0}$ via $\vec{P}=\myvec{-1&-1\\1&0}$\\
 &And this finally is similar to $\myvec{0&0\\1&0}$ as before.\\
 &\\
\hline
&\\
Conclusion &Thus either $\vec{N}$ = 0 or $\vec{N}$ is similar over $\mathbb{C}$ to $\myvec{0&0\\1&0}$.\\
&\\
\hline
\caption{Solution summary}
\label{eq:solutions/6/2/11/table:1}
\end{longtable}

\twocolumn


\end{enumerate}

\section{Inner Product Spaces}
\subsection{Inner Products}
\renewcommand{\theequation}{\theenumi}
\renewcommand{\thefigure}{\theenumi}
\begin{enumerate}[label=\thesubsection.\arabic*.,ref=\thesubsection.\theenumi]
\numberwithin{equation}{enumi}
\numberwithin{figure}{enumi}

\item Let $( | )$ be the standard inner product on $\mathbb{R}^2$.
\begin{enumerate}
\item Let 
\begin{align}
\vec{\alpha} = \myvec{1\\ 2} \quad \vec{\beta} = \myvec{-1\\ 1}
\end{align}
If $\vec{\gamma}$ is a vector such that $(\alpha^T\gamma)=-1$ and $(\beta^T\gamma)=3$. Find $\vec{\gamma}$
\item Show that for any $\vec{\alpha}$ in $\mathbb{R}^2$ we have
\begin{align}
\vec{\alpha} = (\alpha^T\epsilon_1)\epsilon_1 + (\alpha^T\epsilon_2)\epsilon_2 \label{eq:0}
\end{align} 
\end{enumerate}
%
\solution
See Table \ref{eq:solutions/6/2/11/table:1}
\onecolumn
\begin{longtable}{|l|l|}
\hline
\multirow{3}{*}{} & \\
Statement&Solution\\
\hline
&\parbox{6cm}{\begin{align}
\mbox{Let }\vec{N}&=\myvec{a&b\\c&d}\\
\mbox{Since }\vec{N}^2&=0
\end{align}}\\
&If $\myvec{a\\c},\myvec{b\\d}$ are linearly independent then $\vec{N}$ is diagonalizable to $\myvec{0&0\\0&0}$.\\
&\parbox{6cm}{\begin{align}
\mbox{If }\vec{P}\vec{N}\vec{P}^{-1}=0\\
\mbox{then }\vec{N}=\vec{P}^{-1}\vec{0}\vec{P}=0
\end{align}}\\
Proof that&So in this case $\vec{N}$ itself is the zero matrix.\\
$\vec{N}=0$&This contradicts the assumption that $\myvec{a\\c},\myvec{b\\d}$ are linearly independent.\\
&$\therefore$ we can assume that $\myvec{a\\c},\myvec{b\\d}$ are linearly dependent if both are\\
&equal to the zero vector\\
&\parbox{6cm}
{\begin{align}
   \mbox{then } \vec{N} &= 0.
\end{align}}\\
%\hline
%\pagebreak
\hline
&\\
&Therefore we can assume at least one vector is non-zero.\\
Assuming $\myvec{b\\d}$ as&Therefore
$\vec{N}=\myvec{a&0\\c&0}$\\
the zero vector&\\
&\parbox{6cm}{\begin{align}
\mbox{So }\vec{N}^2&=0\\
\implies a^2&=0\\
\therefore a&=0\\
\mbox{Thus }\vec{N}&=\myvec{a&0\\c&0}
\end{align}}\\
&In this case $\vec{N}$ is similar to $\vec{N}=\myvec{0&0\\1&0}$ via the matrix $\vec{P}=\myvec{c&0\\0&1}$\\
&\\
\hline
&\\
Assuming $\myvec{a\\c}$ as&Therefore
$\vec{N}=\myvec{0&b\\0&d}$\\
the zero vector&\\
&\parbox{6cm}{\begin{align}
\mbox{Then }\vec{N}^2&=0\\
\implies d^2&=0\\
\therefore d&=0\\
\mbox{Thus }\vec{N}&=\myvec{0&b\\0&0}
\end{align}}\\
&In this case $\vec{N}$ is similar to $\vec{N}=\myvec{0&0\\b&0}$ via the matrix $\vec{P}=\myvec{0&1\\1&0}$,\\
&which is similar to $\myvec{0&0\\1&0}$ as above.\\
&\\
\hline
&\\
Hence& we can assume neither $\myvec{a\\c}$ or $\myvec{b\\d}$ is the zero vector.\\
&\\
%\hline
%\pagebreak
\hline
&\\
Consequences of &Since they are linearly dependent we can assume,\\
linear&\\
independence&\\
&\parbox{6cm}{\begin{align}
\myvec{b\\d}&=x\myvec{a\\c}\\
\therefore \vec{N}&=\myvec{a&ax\\c&cx}\\
\therefore \vec{N}^2&=0\\
\implies a(a+cx)&=0\\
c(a+cx)&=0\\
ax(a+cx)&=0\\
cx(a+cx)&=0
\end{align}}\\
\hline
&\\
Proof that $\vec{N}$ is&We know that at least one of a or c is not zero.\\
similar over $\mathbb{C}$ to&If a = 0 then c $\neq$ 0, it must be that x = 0.\\
$\myvec{0&0\\1&0}$&So in this case $\vec{N}=\myvec{0&0\\c&0}$ which is similar to $\myvec{0&0\\1&0}$ as before.\\
&\\
&\parbox{6cm}{\begin{align}
\mbox{If } a &\neq0\\
\mbox{then }x &\neq0\\
\mbox{else }a(a+cx)&=0\\
\implies a&=0\\
\mbox{Thus }a+cx&=0\\
\mbox{Hence }\vec{N}&=\myvec{a&ax\\ \frac{-a}{x}&-a}
\end{align}}\\
 &This is similar to $\myvec{a&a\\-a&-a}$ via $\vec{P}=\myvec{\sqrt{x}&0\\0&\frac{1}{\sqrt{x}}}$.\\
 &And $\myvec{a&a\\-a&-a}$ is similar to $\myvec{0&0\\-a&0}$ via $\vec{P}=\myvec{-1&-1\\1&0}$\\
 &And this finally is similar to $\myvec{0&0\\1&0}$ as before.\\
 &\\
\hline
&\\
Conclusion &Thus either $\vec{N}$ = 0 or $\vec{N}$ is similar over $\mathbb{C}$ to $\myvec{0&0\\1&0}$.\\
&\\
\hline
\caption{Solution summary}
\label{eq:solutions/6/2/11/table:1}
\end{longtable}

\end{enumerate}


\end{document}


