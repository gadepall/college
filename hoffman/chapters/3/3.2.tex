\renewcommand{\theequation}{\theenumi}
\renewcommand{\thefigure}{\theenumi}
\begin{enumerate}[label=\thesubsection.\arabic*.,ref=\thesubsection.\theenumi]
\numberwithin{equation}{enumi}
\numberwithin{figure}{enumi}
\numberwithin{table}{enumi}

\item Let $\vec{T}$ and $\vec{U}$ be the linear operators on $\mathbb{R}^2$ defined by
$\vec{T}$$(x_1,x_2)$ =$(x_2,x_1)$ and $\vec{U}$$(x_1,x_2)$=$(x_1,0)$.
\begin{enumerate}
%
\item Let\textbf{T} and \textbf{U} be the linear operators  on $\mathbb{R}^2$  defined by 
\begin{align}
	\textbf{T}\brak{x_1,x_2} = \brak{x_2,x_1}\intertext{and} \textbf{U}\brak{x_1,x_2} = \brak{x_1,0 }
	\end{align}
	How would you describe T and U geometrically $?$
%
\\
\solution

{Complex Numbers:}
A complex number is a number that can be expressed in the form $a + bi$, where a and b are real numbers, and i represents the imaginary unit, satisfying the equation $i^2 =-1$.The set of complex numbers is denoted by $\mathbb{C}$
\begin{align}
    \mathbb{C}=\{(a,b):a,b \in \mathbb{R}\}
\end{align}
{Rational Numbers:}
A number in the form $\frac{p}{q}$, where both p and q(non-zero) are integers, is called a rational number.The set of rational numbers is dentoed by $\mathbb{Q}$
Let $\mathbb{Q}$ be the set of rational numbers.
\begin{align}
    \mathbb{Q}=\left\{\frac{p}{q}:p \in \mathbb{Z},q \in \mathbb{Z}_{\not=0} \right\}\label{eq:solutions/1/1/7/q}
\end{align}
Let $\mathbb{C}$ be the field of complex numbers and given $\mathbb{F}$ be the subfield of field of complex numbers $\mathbb{C}$ 
Since $\mathbb{F}$ is the subfield , we could say that 
\begin{align}
    0 &\in \mathbb{F} \label{eq:solutions/1/1/7/0}\\
    1 &\in \mathbb{F}
\end{align}
{Closed under addition:}
Here $\mathbb{F}$ is closed under addition since it is subfield
\begin{align}
    1+1=2&\in \mathbb{F}\\
    1+1+1=3&\in \mathbb{F}\\
    \vdots\notag\\
    1+1+\dots+1\text{(p times)}= p &\in \mathbb{F}\label{eq:solutions/1/1/7/p}\\
    1+1+\dots+1\text{(q times)}= q &\in \mathbb{F}\label{eq:solutions/1/1/7/q1}
\end{align}
By using the above property we could say that zero and other positive integers belongs to $\mathbb{F}$.Since $p$ and $q$ are integers we say,
\begin{align}
    p \in \mathbb{Z}\\
    q \in \mathbb{Z}\label{eq:solutions/1/1/7/0}
\end{align}
{Additive Inverse:}
Let $x$ be the positive integer belong $\mathbb{F}$ and by additive inverse we could say, 
\begin{align}
    \forall x &\in \mathbb{F}\label{eq:solutions/1/1/7/1}\\
    (-x) &\in \mathbb{F} \label{eq:solutions/1/1/7/2}
\end{align}
Therefore field $\mathbb{F}$ contains every integers. Let n be a integer then,
\begin{align}
    n \in \mathbb{Z} &\implies n \in \mathbb{F}\\
    \mathbb{Z} &\subseteq \mathbb{F}
\end{align}
Where $\mathbb{Z}$ is subset of $\mathbb{F}$
{Multiplicative Inverse:}
Every element except zero in the subfield $\mathbb{F}$ has an multiplicative inverse. From equation \eqref{eq:solutions/1/1/7/q1}, since q $\in \mathbb{F}$ we could say ,
\begin{align}
    \frac{1}{q} \in \mathbb{F} \quad{\text{and  }} q \not= 0\label{eq:solutions/1/1/7/3}
\end{align}
{Closed under multiplication:}
Also, $\mathbb{F}$ is closed under multiplication and thus,
from equation \eqref{eq:solutions/1/1/7/p} and \eqref{eq:solutions/1/1/7/3} we get , 
\begin{align}
    p\cdot\frac{1}{q} \in \mathbb{F}\\
\implies \frac{p}{q} \in \mathbb{F}\label{eq:solutions/1/1/7/proof}
\end{align}
where , $p \in \mathbb{Z}$ and $q \in \mathbb{Z}_{\not=0}$ (from equation \eqref{eq:solutions/1/1/7/0} and \eqref{eq:solutions/1/1/7/3})
{Conclusion}
From \eqref{eq:solutions/1/1/7/q} and \eqref{eq:solutions/1/1/7/proof} we could say , 
\begin{align}
    \mathbb{Q} \subseteq \mathbb{F}\label{eq:solutions/1/1/7/F}
\end{align}
From equation \eqref{eq:solutions/1/1/7/F} we could say that each subfield of the field of complex number contains every rational number
\begin{center}
    Hence Proved
\end{center}

%
\item Give rules like the ones defining $\vec{T}$ and $\vec{U}$ for each of the transformations
$\vec{U + T}$, $\vec{UT}$, $\vec{TU}$, $\vec{T}^2$, $\vec{U}^2$. $\mathbb{R}^2$ into $\mathbb{R}^2$  is linear transformation?
%
\\
\solution

{Complex Numbers:}
A complex number is a number that can be expressed in the form $a + bi$, where a and b are real numbers, and i represents the imaginary unit, satisfying the equation $i^2 =-1$.The set of complex numbers is denoted by $\mathbb{C}$
\begin{align}
    \mathbb{C}=\{(a,b):a,b \in \mathbb{R}\}
\end{align}
{Rational Numbers:}
A number in the form $\frac{p}{q}$, where both p and q(non-zero) are integers, is called a rational number.The set of rational numbers is dentoed by $\mathbb{Q}$
Let $\mathbb{Q}$ be the set of rational numbers.
\begin{align}
    \mathbb{Q}=\left\{\frac{p}{q}:p \in \mathbb{Z},q \in \mathbb{Z}_{\not=0} \right\}\label{eq:solutions/1/1/7/q}
\end{align}
Let $\mathbb{C}$ be the field of complex numbers and given $\mathbb{F}$ be the subfield of field of complex numbers $\mathbb{C}$ 
Since $\mathbb{F}$ is the subfield , we could say that 
\begin{align}
    0 &\in \mathbb{F} \label{eq:solutions/1/1/7/0}\\
    1 &\in \mathbb{F}
\end{align}
{Closed under addition:}
Here $\mathbb{F}$ is closed under addition since it is subfield
\begin{align}
    1+1=2&\in \mathbb{F}\\
    1+1+1=3&\in \mathbb{F}\\
    \vdots\notag\\
    1+1+\dots+1\text{(p times)}= p &\in \mathbb{F}\label{eq:solutions/1/1/7/p}\\
    1+1+\dots+1\text{(q times)}= q &\in \mathbb{F}\label{eq:solutions/1/1/7/q1}
\end{align}
By using the above property we could say that zero and other positive integers belongs to $\mathbb{F}$.Since $p$ and $q$ are integers we say,
\begin{align}
    p \in \mathbb{Z}\\
    q \in \mathbb{Z}\label{eq:solutions/1/1/7/0}
\end{align}
{Additive Inverse:}
Let $x$ be the positive integer belong $\mathbb{F}$ and by additive inverse we could say, 
\begin{align}
    \forall x &\in \mathbb{F}\label{eq:solutions/1/1/7/1}\\
    (-x) &\in \mathbb{F} \label{eq:solutions/1/1/7/2}
\end{align}
Therefore field $\mathbb{F}$ contains every integers. Let n be a integer then,
\begin{align}
    n \in \mathbb{Z} &\implies n \in \mathbb{F}\\
    \mathbb{Z} &\subseteq \mathbb{F}
\end{align}
Where $\mathbb{Z}$ is subset of $\mathbb{F}$
{Multiplicative Inverse:}
Every element except zero in the subfield $\mathbb{F}$ has an multiplicative inverse. From equation \eqref{eq:solutions/1/1/7/q1}, since q $\in \mathbb{F}$ we could say ,
\begin{align}
    \frac{1}{q} \in \mathbb{F} \quad{\text{and  }} q \not= 0\label{eq:solutions/1/1/7/3}
\end{align}
{Closed under multiplication:}
Also, $\mathbb{F}$ is closed under multiplication and thus,
from equation \eqref{eq:solutions/1/1/7/p} and \eqref{eq:solutions/1/1/7/3} we get , 
\begin{align}
    p\cdot\frac{1}{q} \in \mathbb{F}\\
\implies \frac{p}{q} \in \mathbb{F}\label{eq:solutions/1/1/7/proof}
\end{align}
where , $p \in \mathbb{Z}$ and $q \in \mathbb{Z}_{\not=0}$ (from equation \eqref{eq:solutions/1/1/7/0} and \eqref{eq:solutions/1/1/7/3})
{Conclusion}
From \eqref{eq:solutions/1/1/7/q} and \eqref{eq:solutions/1/1/7/proof} we could say , 
\begin{align}
    \mathbb{Q} \subseteq \mathbb{F}\label{eq:solutions/1/1/7/F}
\end{align}
From equation \eqref{eq:solutions/1/1/7/F} we could say that each subfield of the field of complex number contains every rational number
\begin{center}
    Hence Proved
\end{center}

%
\end{enumerate}
\item Let T be the unique linear operator on $C^{3}$ for which 
   \begin{multline}
    \begin{aligned}
    T\brak{\epsilon_1}=\myvec{1&0&i} , T\brak{\epsilon_2}=\myvec{0&1&1} ,\\ T\brak{\epsilon_3}=\myvec{i&1&0} 
    \end{aligned}
    \end{multline}
    Is T invertible ?
%
\\
\solution

{Complex Numbers:}
A complex number is a number that can be expressed in the form $a + bi$, where a and b are real numbers, and i represents the imaginary unit, satisfying the equation $i^2 =-1$.The set of complex numbers is denoted by $\mathbb{C}$
\begin{align}
    \mathbb{C}=\{(a,b):a,b \in \mathbb{R}\}
\end{align}
{Rational Numbers:}
A number in the form $\frac{p}{q}$, where both p and q(non-zero) are integers, is called a rational number.The set of rational numbers is dentoed by $\mathbb{Q}$
Let $\mathbb{Q}$ be the set of rational numbers.
\begin{align}
    \mathbb{Q}=\left\{\frac{p}{q}:p \in \mathbb{Z},q \in \mathbb{Z}_{\not=0} \right\}\label{eq:solutions/1/1/7/q}
\end{align}
Let $\mathbb{C}$ be the field of complex numbers and given $\mathbb{F}$ be the subfield of field of complex numbers $\mathbb{C}$ 
Since $\mathbb{F}$ is the subfield , we could say that 
\begin{align}
    0 &\in \mathbb{F} \label{eq:solutions/1/1/7/0}\\
    1 &\in \mathbb{F}
\end{align}
{Closed under addition:}
Here $\mathbb{F}$ is closed under addition since it is subfield
\begin{align}
    1+1=2&\in \mathbb{F}\\
    1+1+1=3&\in \mathbb{F}\\
    \vdots\notag\\
    1+1+\dots+1\text{(p times)}= p &\in \mathbb{F}\label{eq:solutions/1/1/7/p}\\
    1+1+\dots+1\text{(q times)}= q &\in \mathbb{F}\label{eq:solutions/1/1/7/q1}
\end{align}
By using the above property we could say that zero and other positive integers belongs to $\mathbb{F}$.Since $p$ and $q$ are integers we say,
\begin{align}
    p \in \mathbb{Z}\\
    q \in \mathbb{Z}\label{eq:solutions/1/1/7/0}
\end{align}
{Additive Inverse:}
Let $x$ be the positive integer belong $\mathbb{F}$ and by additive inverse we could say, 
\begin{align}
    \forall x &\in \mathbb{F}\label{eq:solutions/1/1/7/1}\\
    (-x) &\in \mathbb{F} \label{eq:solutions/1/1/7/2}
\end{align}
Therefore field $\mathbb{F}$ contains every integers. Let n be a integer then,
\begin{align}
    n \in \mathbb{Z} &\implies n \in \mathbb{F}\\
    \mathbb{Z} &\subseteq \mathbb{F}
\end{align}
Where $\mathbb{Z}$ is subset of $\mathbb{F}$
{Multiplicative Inverse:}
Every element except zero in the subfield $\mathbb{F}$ has an multiplicative inverse. From equation \eqref{eq:solutions/1/1/7/q1}, since q $\in \mathbb{F}$ we could say ,
\begin{align}
    \frac{1}{q} \in \mathbb{F} \quad{\text{and  }} q \not= 0\label{eq:solutions/1/1/7/3}
\end{align}
{Closed under multiplication:}
Also, $\mathbb{F}$ is closed under multiplication and thus,
from equation \eqref{eq:solutions/1/1/7/p} and \eqref{eq:solutions/1/1/7/3} we get , 
\begin{align}
    p\cdot\frac{1}{q} \in \mathbb{F}\\
\implies \frac{p}{q} \in \mathbb{F}\label{eq:solutions/1/1/7/proof}
\end{align}
where , $p \in \mathbb{Z}$ and $q \in \mathbb{Z}_{\not=0}$ (from equation \eqref{eq:solutions/1/1/7/0} and \eqref{eq:solutions/1/1/7/3})
{Conclusion}
From \eqref{eq:solutions/1/1/7/q} and \eqref{eq:solutions/1/1/7/proof} we could say , 
\begin{align}
    \mathbb{Q} \subseteq \mathbb{F}\label{eq:solutions/1/1/7/F}
\end{align}
From equation \eqref{eq:solutions/1/1/7/F} we could say that each subfield of the field of complex number contains every rational number
\begin{center}
    Hence Proved
\end{center}

\item For the linear operator $\vec{T}$
\begin{align}
    \vec{T}\myvec{x_1\\x_2\\x_3}=\myvec{3x_1\\x_1-x_2\\2x_1+x_2+x_3}\label{eq:solutions/3/2/4/eq:1}
\end{align}
%
\item Let $\vec{T}$ be a linear operator on $\vec{R}^3$ defined by
\begin{align*}
	\vec{T}\myvec{x_1\\x_2\\x_3}=\myvec{3x_1\\x_1-x_2\\2x_1+x_2+x_3}
\end{align*}
Is $\vec{T}$ invertible? If so, find a rule for $\vec{T}^{-1}$ like the one which defines T.
%
\\
\solution

{Complex Numbers:}
A complex number is a number that can be expressed in the form $a + bi$, where a and b are real numbers, and i represents the imaginary unit, satisfying the equation $i^2 =-1$.The set of complex numbers is denoted by $\mathbb{C}$
\begin{align}
    \mathbb{C}=\{(a,b):a,b \in \mathbb{R}\}
\end{align}
{Rational Numbers:}
A number in the form $\frac{p}{q}$, where both p and q(non-zero) are integers, is called a rational number.The set of rational numbers is dentoed by $\mathbb{Q}$
Let $\mathbb{Q}$ be the set of rational numbers.
\begin{align}
    \mathbb{Q}=\left\{\frac{p}{q}:p \in \mathbb{Z},q \in \mathbb{Z}_{\not=0} \right\}\label{eq:solutions/1/1/7/q}
\end{align}
Let $\mathbb{C}$ be the field of complex numbers and given $\mathbb{F}$ be the subfield of field of complex numbers $\mathbb{C}$ 
Since $\mathbb{F}$ is the subfield , we could say that 
\begin{align}
    0 &\in \mathbb{F} \label{eq:solutions/1/1/7/0}\\
    1 &\in \mathbb{F}
\end{align}
{Closed under addition:}
Here $\mathbb{F}$ is closed under addition since it is subfield
\begin{align}
    1+1=2&\in \mathbb{F}\\
    1+1+1=3&\in \mathbb{F}\\
    \vdots\notag\\
    1+1+\dots+1\text{(p times)}= p &\in \mathbb{F}\label{eq:solutions/1/1/7/p}\\
    1+1+\dots+1\text{(q times)}= q &\in \mathbb{F}\label{eq:solutions/1/1/7/q1}
\end{align}
By using the above property we could say that zero and other positive integers belongs to $\mathbb{F}$.Since $p$ and $q$ are integers we say,
\begin{align}
    p \in \mathbb{Z}\\
    q \in \mathbb{Z}\label{eq:solutions/1/1/7/0}
\end{align}
{Additive Inverse:}
Let $x$ be the positive integer belong $\mathbb{F}$ and by additive inverse we could say, 
\begin{align}
    \forall x &\in \mathbb{F}\label{eq:solutions/1/1/7/1}\\
    (-x) &\in \mathbb{F} \label{eq:solutions/1/1/7/2}
\end{align}
Therefore field $\mathbb{F}$ contains every integers. Let n be a integer then,
\begin{align}
    n \in \mathbb{Z} &\implies n \in \mathbb{F}\\
    \mathbb{Z} &\subseteq \mathbb{F}
\end{align}
Where $\mathbb{Z}$ is subset of $\mathbb{F}$
{Multiplicative Inverse:}
Every element except zero in the subfield $\mathbb{F}$ has an multiplicative inverse. From equation \eqref{eq:solutions/1/1/7/q1}, since q $\in \mathbb{F}$ we could say ,
\begin{align}
    \frac{1}{q} \in \mathbb{F} \quad{\text{and  }} q \not= 0\label{eq:solutions/1/1/7/3}
\end{align}
{Closed under multiplication:}
Also, $\mathbb{F}$ is closed under multiplication and thus,
from equation \eqref{eq:solutions/1/1/7/p} and \eqref{eq:solutions/1/1/7/3} we get , 
\begin{align}
    p\cdot\frac{1}{q} \in \mathbb{F}\\
\implies \frac{p}{q} \in \mathbb{F}\label{eq:solutions/1/1/7/proof}
\end{align}
where , $p \in \mathbb{Z}$ and $q \in \mathbb{Z}_{\not=0}$ (from equation \eqref{eq:solutions/1/1/7/0} and \eqref{eq:solutions/1/1/7/3})
{Conclusion}
From \eqref{eq:solutions/1/1/7/q} and \eqref{eq:solutions/1/1/7/proof} we could say , 
\begin{align}
    \mathbb{Q} \subseteq \mathbb{F}\label{eq:solutions/1/1/7/F}
\end{align}
From equation \eqref{eq:solutions/1/1/7/F} we could say that each subfield of the field of complex number contains every rational number
\begin{center}
    Hence Proved
\end{center}

Prove that
\begin{align}
    \vec(\vec{T}^{2}-I)\vec(\vec{T}-3I)=0
\end{align}
%
\\
\solution

{Complex Numbers:}
A complex number is a number that can be expressed in the form $a + bi$, where a and b are real numbers, and i represents the imaginary unit, satisfying the equation $i^2 =-1$.The set of complex numbers is denoted by $\mathbb{C}$
\begin{align}
    \mathbb{C}=\{(a,b):a,b \in \mathbb{R}\}
\end{align}
{Rational Numbers:}
A number in the form $\frac{p}{q}$, where both p and q(non-zero) are integers, is called a rational number.The set of rational numbers is dentoed by $\mathbb{Q}$
Let $\mathbb{Q}$ be the set of rational numbers.
\begin{align}
    \mathbb{Q}=\left\{\frac{p}{q}:p \in \mathbb{Z},q \in \mathbb{Z}_{\not=0} \right\}\label{eq:solutions/1/1/7/q}
\end{align}
Let $\mathbb{C}$ be the field of complex numbers and given $\mathbb{F}$ be the subfield of field of complex numbers $\mathbb{C}$ 
Since $\mathbb{F}$ is the subfield , we could say that 
\begin{align}
    0 &\in \mathbb{F} \label{eq:solutions/1/1/7/0}\\
    1 &\in \mathbb{F}
\end{align}
{Closed under addition:}
Here $\mathbb{F}$ is closed under addition since it is subfield
\begin{align}
    1+1=2&\in \mathbb{F}\\
    1+1+1=3&\in \mathbb{F}\\
    \vdots\notag\\
    1+1+\dots+1\text{(p times)}= p &\in \mathbb{F}\label{eq:solutions/1/1/7/p}\\
    1+1+\dots+1\text{(q times)}= q &\in \mathbb{F}\label{eq:solutions/1/1/7/q1}
\end{align}
By using the above property we could say that zero and other positive integers belongs to $\mathbb{F}$.Since $p$ and $q$ are integers we say,
\begin{align}
    p \in \mathbb{Z}\\
    q \in \mathbb{Z}\label{eq:solutions/1/1/7/0}
\end{align}
{Additive Inverse:}
Let $x$ be the positive integer belong $\mathbb{F}$ and by additive inverse we could say, 
\begin{align}
    \forall x &\in \mathbb{F}\label{eq:solutions/1/1/7/1}\\
    (-x) &\in \mathbb{F} \label{eq:solutions/1/1/7/2}
\end{align}
Therefore field $\mathbb{F}$ contains every integers. Let n be a integer then,
\begin{align}
    n \in \mathbb{Z} &\implies n \in \mathbb{F}\\
    \mathbb{Z} &\subseteq \mathbb{F}
\end{align}
Where $\mathbb{Z}$ is subset of $\mathbb{F}$
{Multiplicative Inverse:}
Every element except zero in the subfield $\mathbb{F}$ has an multiplicative inverse. From equation \eqref{eq:solutions/1/1/7/q1}, since q $\in \mathbb{F}$ we could say ,
\begin{align}
    \frac{1}{q} \in \mathbb{F} \quad{\text{and  }} q \not= 0\label{eq:solutions/1/1/7/3}
\end{align}
{Closed under multiplication:}
Also, $\mathbb{F}$ is closed under multiplication and thus,
from equation \eqref{eq:solutions/1/1/7/p} and \eqref{eq:solutions/1/1/7/3} we get , 
\begin{align}
    p\cdot\frac{1}{q} \in \mathbb{F}\\
\implies \frac{p}{q} \in \mathbb{F}\label{eq:solutions/1/1/7/proof}
\end{align}
where , $p \in \mathbb{Z}$ and $q \in \mathbb{Z}_{\not=0}$ (from equation \eqref{eq:solutions/1/1/7/0} and \eqref{eq:solutions/1/1/7/3})
{Conclusion}
From \eqref{eq:solutions/1/1/7/q} and \eqref{eq:solutions/1/1/7/proof} we could say , 
\begin{align}
    \mathbb{Q} \subseteq \mathbb{F}\label{eq:solutions/1/1/7/F}
\end{align}
From equation \eqref{eq:solutions/1/1/7/F} we could say that each subfield of the field of complex number contains every rational number
\begin{center}
    Hence Proved
\end{center}

%
\item Let $\mathbb{C}$ be the complex vector space of $2\times2$ matrices with complex entries. Let
\begin{align}
	\vec{B} = \myvec{1 & -1 \\ -4 & 4}
\end{align}
and let $\vec{T}$ be the linear operator on $\mathbb{C}^{2\times2}$  defined by $\vec{T(A)} = \vec{BA}$. What is the rank of $\vec{T}$? Can you describe $\vec{T^2}$?   
%
\\
\solution

{Complex Numbers:}
A complex number is a number that can be expressed in the form $a + bi$, where a and b are real numbers, and i represents the imaginary unit, satisfying the equation $i^2 =-1$.The set of complex numbers is denoted by $\mathbb{C}$
\begin{align}
    \mathbb{C}=\{(a,b):a,b \in \mathbb{R}\}
\end{align}
{Rational Numbers:}
A number in the form $\frac{p}{q}$, where both p and q(non-zero) are integers, is called a rational number.The set of rational numbers is dentoed by $\mathbb{Q}$
Let $\mathbb{Q}$ be the set of rational numbers.
\begin{align}
    \mathbb{Q}=\left\{\frac{p}{q}:p \in \mathbb{Z},q \in \mathbb{Z}_{\not=0} \right\}\label{eq:solutions/1/1/7/q}
\end{align}
Let $\mathbb{C}$ be the field of complex numbers and given $\mathbb{F}$ be the subfield of field of complex numbers $\mathbb{C}$ 
Since $\mathbb{F}$ is the subfield , we could say that 
\begin{align}
    0 &\in \mathbb{F} \label{eq:solutions/1/1/7/0}\\
    1 &\in \mathbb{F}
\end{align}
{Closed under addition:}
Here $\mathbb{F}$ is closed under addition since it is subfield
\begin{align}
    1+1=2&\in \mathbb{F}\\
    1+1+1=3&\in \mathbb{F}\\
    \vdots\notag\\
    1+1+\dots+1\text{(p times)}= p &\in \mathbb{F}\label{eq:solutions/1/1/7/p}\\
    1+1+\dots+1\text{(q times)}= q &\in \mathbb{F}\label{eq:solutions/1/1/7/q1}
\end{align}
By using the above property we could say that zero and other positive integers belongs to $\mathbb{F}$.Since $p$ and $q$ are integers we say,
\begin{align}
    p \in \mathbb{Z}\\
    q \in \mathbb{Z}\label{eq:solutions/1/1/7/0}
\end{align}
{Additive Inverse:}
Let $x$ be the positive integer belong $\mathbb{F}$ and by additive inverse we could say, 
\begin{align}
    \forall x &\in \mathbb{F}\label{eq:solutions/1/1/7/1}\\
    (-x) &\in \mathbb{F} \label{eq:solutions/1/1/7/2}
\end{align}
Therefore field $\mathbb{F}$ contains every integers. Let n be a integer then,
\begin{align}
    n \in \mathbb{Z} &\implies n \in \mathbb{F}\\
    \mathbb{Z} &\subseteq \mathbb{F}
\end{align}
Where $\mathbb{Z}$ is subset of $\mathbb{F}$
{Multiplicative Inverse:}
Every element except zero in the subfield $\mathbb{F}$ has an multiplicative inverse. From equation \eqref{eq:solutions/1/1/7/q1}, since q $\in \mathbb{F}$ we could say ,
\begin{align}
    \frac{1}{q} \in \mathbb{F} \quad{\text{and  }} q \not= 0\label{eq:solutions/1/1/7/3}
\end{align}
{Closed under multiplication:}
Also, $\mathbb{F}$ is closed under multiplication and thus,
from equation \eqref{eq:solutions/1/1/7/p} and \eqref{eq:solutions/1/1/7/3} we get , 
\begin{align}
    p\cdot\frac{1}{q} \in \mathbb{F}\\
\implies \frac{p}{q} \in \mathbb{F}\label{eq:solutions/1/1/7/proof}
\end{align}
where , $p \in \mathbb{Z}$ and $q \in \mathbb{Z}_{\not=0}$ (from equation \eqref{eq:solutions/1/1/7/0} and \eqref{eq:solutions/1/1/7/3})
{Conclusion}
From \eqref{eq:solutions/1/1/7/q} and \eqref{eq:solutions/1/1/7/proof} we could say , 
\begin{align}
    \mathbb{Q} \subseteq \mathbb{F}\label{eq:solutions/1/1/7/F}
\end{align}
From equation \eqref{eq:solutions/1/1/7/F} we could say that each subfield of the field of complex number contains every rational number
\begin{center}
    Hence Proved
\end{center}

%
\item Let T be a linear transformation from $\mathbb{R}^3$ into $\mathbb{R}^2$, and let U be a linear transformation from $\mathbb{R}^2$ into $\mathbb{R}^3$. Prove that the transformation UT is not invertible. Generalize the theorem.
%
\\
\solution

{Complex Numbers:}
A complex number is a number that can be expressed in the form $a + bi$, where a and b are real numbers, and i represents the imaginary unit, satisfying the equation $i^2 =-1$.The set of complex numbers is denoted by $\mathbb{C}$
\begin{align}
    \mathbb{C}=\{(a,b):a,b \in \mathbb{R}\}
\end{align}
{Rational Numbers:}
A number in the form $\frac{p}{q}$, where both p and q(non-zero) are integers, is called a rational number.The set of rational numbers is dentoed by $\mathbb{Q}$
Let $\mathbb{Q}$ be the set of rational numbers.
\begin{align}
    \mathbb{Q}=\left\{\frac{p}{q}:p \in \mathbb{Z},q \in \mathbb{Z}_{\not=0} \right\}\label{eq:solutions/1/1/7/q}
\end{align}
Let $\mathbb{C}$ be the field of complex numbers and given $\mathbb{F}$ be the subfield of field of complex numbers $\mathbb{C}$ 
Since $\mathbb{F}$ is the subfield , we could say that 
\begin{align}
    0 &\in \mathbb{F} \label{eq:solutions/1/1/7/0}\\
    1 &\in \mathbb{F}
\end{align}
{Closed under addition:}
Here $\mathbb{F}$ is closed under addition since it is subfield
\begin{align}
    1+1=2&\in \mathbb{F}\\
    1+1+1=3&\in \mathbb{F}\\
    \vdots\notag\\
    1+1+\dots+1\text{(p times)}= p &\in \mathbb{F}\label{eq:solutions/1/1/7/p}\\
    1+1+\dots+1\text{(q times)}= q &\in \mathbb{F}\label{eq:solutions/1/1/7/q1}
\end{align}
By using the above property we could say that zero and other positive integers belongs to $\mathbb{F}$.Since $p$ and $q$ are integers we say,
\begin{align}
    p \in \mathbb{Z}\\
    q \in \mathbb{Z}\label{eq:solutions/1/1/7/0}
\end{align}
{Additive Inverse:}
Let $x$ be the positive integer belong $\mathbb{F}$ and by additive inverse we could say, 
\begin{align}
    \forall x &\in \mathbb{F}\label{eq:solutions/1/1/7/1}\\
    (-x) &\in \mathbb{F} \label{eq:solutions/1/1/7/2}
\end{align}
Therefore field $\mathbb{F}$ contains every integers. Let n be a integer then,
\begin{align}
    n \in \mathbb{Z} &\implies n \in \mathbb{F}\\
    \mathbb{Z} &\subseteq \mathbb{F}
\end{align}
Where $\mathbb{Z}$ is subset of $\mathbb{F}$
{Multiplicative Inverse:}
Every element except zero in the subfield $\mathbb{F}$ has an multiplicative inverse. From equation \eqref{eq:solutions/1/1/7/q1}, since q $\in \mathbb{F}$ we could say ,
\begin{align}
    \frac{1}{q} \in \mathbb{F} \quad{\text{and  }} q \not= 0\label{eq:solutions/1/1/7/3}
\end{align}
{Closed under multiplication:}
Also, $\mathbb{F}$ is closed under multiplication and thus,
from equation \eqref{eq:solutions/1/1/7/p} and \eqref{eq:solutions/1/1/7/3} we get , 
\begin{align}
    p\cdot\frac{1}{q} \in \mathbb{F}\\
\implies \frac{p}{q} \in \mathbb{F}\label{eq:solutions/1/1/7/proof}
\end{align}
where , $p \in \mathbb{Z}$ and $q \in \mathbb{Z}_{\not=0}$ (from equation \eqref{eq:solutions/1/1/7/0} and \eqref{eq:solutions/1/1/7/3})
{Conclusion}
From \eqref{eq:solutions/1/1/7/q} and \eqref{eq:solutions/1/1/7/proof} we could say , 
\begin{align}
    \mathbb{Q} \subseteq \mathbb{F}\label{eq:solutions/1/1/7/F}
\end{align}
From equation \eqref{eq:solutions/1/1/7/F} we could say that each subfield of the field of complex number contains every rational number
\begin{center}
    Hence Proved
\end{center}

%
\item Find two linear operators $\vec{T}$ and $\vec{U}$ on $\vec{R}^2$ such that $\vec{T}\vec{U}$ = 0 but $\vec{U}\vec{T} \neq 0$
%
\\
\solution

{Complex Numbers:}
A complex number is a number that can be expressed in the form $a + bi$, where a and b are real numbers, and i represents the imaginary unit, satisfying the equation $i^2 =-1$.The set of complex numbers is denoted by $\mathbb{C}$
\begin{align}
    \mathbb{C}=\{(a,b):a,b \in \mathbb{R}\}
\end{align}
{Rational Numbers:}
A number in the form $\frac{p}{q}$, where both p and q(non-zero) are integers, is called a rational number.The set of rational numbers is dentoed by $\mathbb{Q}$
Let $\mathbb{Q}$ be the set of rational numbers.
\begin{align}
    \mathbb{Q}=\left\{\frac{p}{q}:p \in \mathbb{Z},q \in \mathbb{Z}_{\not=0} \right\}\label{eq:solutions/1/1/7/q}
\end{align}
Let $\mathbb{C}$ be the field of complex numbers and given $\mathbb{F}$ be the subfield of field of complex numbers $\mathbb{C}$ 
Since $\mathbb{F}$ is the subfield , we could say that 
\begin{align}
    0 &\in \mathbb{F} \label{eq:solutions/1/1/7/0}\\
    1 &\in \mathbb{F}
\end{align}
{Closed under addition:}
Here $\mathbb{F}$ is closed under addition since it is subfield
\begin{align}
    1+1=2&\in \mathbb{F}\\
    1+1+1=3&\in \mathbb{F}\\
    \vdots\notag\\
    1+1+\dots+1\text{(p times)}= p &\in \mathbb{F}\label{eq:solutions/1/1/7/p}\\
    1+1+\dots+1\text{(q times)}= q &\in \mathbb{F}\label{eq:solutions/1/1/7/q1}
\end{align}
By using the above property we could say that zero and other positive integers belongs to $\mathbb{F}$.Since $p$ and $q$ are integers we say,
\begin{align}
    p \in \mathbb{Z}\\
    q \in \mathbb{Z}\label{eq:solutions/1/1/7/0}
\end{align}
{Additive Inverse:}
Let $x$ be the positive integer belong $\mathbb{F}$ and by additive inverse we could say, 
\begin{align}
    \forall x &\in \mathbb{F}\label{eq:solutions/1/1/7/1}\\
    (-x) &\in \mathbb{F} \label{eq:solutions/1/1/7/2}
\end{align}
Therefore field $\mathbb{F}$ contains every integers. Let n be a integer then,
\begin{align}
    n \in \mathbb{Z} &\implies n \in \mathbb{F}\\
    \mathbb{Z} &\subseteq \mathbb{F}
\end{align}
Where $\mathbb{Z}$ is subset of $\mathbb{F}$
{Multiplicative Inverse:}
Every element except zero in the subfield $\mathbb{F}$ has an multiplicative inverse. From equation \eqref{eq:solutions/1/1/7/q1}, since q $\in \mathbb{F}$ we could say ,
\begin{align}
    \frac{1}{q} \in \mathbb{F} \quad{\text{and  }} q \not= 0\label{eq:solutions/1/1/7/3}
\end{align}
{Closed under multiplication:}
Also, $\mathbb{F}$ is closed under multiplication and thus,
from equation \eqref{eq:solutions/1/1/7/p} and \eqref{eq:solutions/1/1/7/3} we get , 
\begin{align}
    p\cdot\frac{1}{q} \in \mathbb{F}\\
\implies \frac{p}{q} \in \mathbb{F}\label{eq:solutions/1/1/7/proof}
\end{align}
where , $p \in \mathbb{Z}$ and $q \in \mathbb{Z}_{\not=0}$ (from equation \eqref{eq:solutions/1/1/7/0} and \eqref{eq:solutions/1/1/7/3})
{Conclusion}
From \eqref{eq:solutions/1/1/7/q} and \eqref{eq:solutions/1/1/7/proof} we could say , 
\begin{align}
    \mathbb{Q} \subseteq \mathbb{F}\label{eq:solutions/1/1/7/F}
\end{align}
From equation \eqref{eq:solutions/1/1/7/F} we could say that each subfield of the field of complex number contains every rational number
\begin{center}
    Hence Proved
\end{center}

%
\item Let $\vec{V}$ be a vector space over the field $\vec{F}$ and $\vec{T}$ is a linear operator on $\vec{V}$. If $\vec{T}^2=0$, what can you say about the relation of the range of $\vec{T}$ to the null space of $\vec{T}$ ?
Give an example of linear operator $\vec{T}$ on $\vec{R}^2$ such that $\vec{T}^2=0$ but $\vec{T}\ne0$.
%
\\
\solution

{Complex Numbers:}
A complex number is a number that can be expressed in the form $a + bi$, where a and b are real numbers, and i represents the imaginary unit, satisfying the equation $i^2 =-1$.The set of complex numbers is denoted by $\mathbb{C}$
\begin{align}
    \mathbb{C}=\{(a,b):a,b \in \mathbb{R}\}
\end{align}
{Rational Numbers:}
A number in the form $\frac{p}{q}$, where both p and q(non-zero) are integers, is called a rational number.The set of rational numbers is dentoed by $\mathbb{Q}$
Let $\mathbb{Q}$ be the set of rational numbers.
\begin{align}
    \mathbb{Q}=\left\{\frac{p}{q}:p \in \mathbb{Z},q \in \mathbb{Z}_{\not=0} \right\}\label{eq:solutions/1/1/7/q}
\end{align}
Let $\mathbb{C}$ be the field of complex numbers and given $\mathbb{F}$ be the subfield of field of complex numbers $\mathbb{C}$ 
Since $\mathbb{F}$ is the subfield , we could say that 
\begin{align}
    0 &\in \mathbb{F} \label{eq:solutions/1/1/7/0}\\
    1 &\in \mathbb{F}
\end{align}
{Closed under addition:}
Here $\mathbb{F}$ is closed under addition since it is subfield
\begin{align}
    1+1=2&\in \mathbb{F}\\
    1+1+1=3&\in \mathbb{F}\\
    \vdots\notag\\
    1+1+\dots+1\text{(p times)}= p &\in \mathbb{F}\label{eq:solutions/1/1/7/p}\\
    1+1+\dots+1\text{(q times)}= q &\in \mathbb{F}\label{eq:solutions/1/1/7/q1}
\end{align}
By using the above property we could say that zero and other positive integers belongs to $\mathbb{F}$.Since $p$ and $q$ are integers we say,
\begin{align}
    p \in \mathbb{Z}\\
    q \in \mathbb{Z}\label{eq:solutions/1/1/7/0}
\end{align}
{Additive Inverse:}
Let $x$ be the positive integer belong $\mathbb{F}$ and by additive inverse we could say, 
\begin{align}
    \forall x &\in \mathbb{F}\label{eq:solutions/1/1/7/1}\\
    (-x) &\in \mathbb{F} \label{eq:solutions/1/1/7/2}
\end{align}
Therefore field $\mathbb{F}$ contains every integers. Let n be a integer then,
\begin{align}
    n \in \mathbb{Z} &\implies n \in \mathbb{F}\\
    \mathbb{Z} &\subseteq \mathbb{F}
\end{align}
Where $\mathbb{Z}$ is subset of $\mathbb{F}$
{Multiplicative Inverse:}
Every element except zero in the subfield $\mathbb{F}$ has an multiplicative inverse. From equation \eqref{eq:solutions/1/1/7/q1}, since q $\in \mathbb{F}$ we could say ,
\begin{align}
    \frac{1}{q} \in \mathbb{F} \quad{\text{and  }} q \not= 0\label{eq:solutions/1/1/7/3}
\end{align}
{Closed under multiplication:}
Also, $\mathbb{F}$ is closed under multiplication and thus,
from equation \eqref{eq:solutions/1/1/7/p} and \eqref{eq:solutions/1/1/7/3} we get , 
\begin{align}
    p\cdot\frac{1}{q} \in \mathbb{F}\\
\implies \frac{p}{q} \in \mathbb{F}\label{eq:solutions/1/1/7/proof}
\end{align}
where , $p \in \mathbb{Z}$ and $q \in \mathbb{Z}_{\not=0}$ (from equation \eqref{eq:solutions/1/1/7/0} and \eqref{eq:solutions/1/1/7/3})
{Conclusion}
From \eqref{eq:solutions/1/1/7/q} and \eqref{eq:solutions/1/1/7/proof} we could say , 
\begin{align}
    \mathbb{Q} \subseteq \mathbb{F}\label{eq:solutions/1/1/7/F}
\end{align}
From equation \eqref{eq:solutions/1/1/7/F} we could say that each subfield of the field of complex number contains every rational number
\begin{center}
    Hence Proved
\end{center}


%
\item Let $\vec{A}$ be an $m \times n$ matrix with entries in $F$ and let $T$ be the linear transformation from $F^{n \times1 }$ into $F^{m \times l}$ defined by $T(\vec{X}) = \vec{A}\vec{X}$. Show that 
\begin{enumerate}
\item
if $m < n$ it may happen that $T$ is onto without being non-singular
\item
if $m>n$ we may have $T$ non-singular but not onto.
\\
\end{enumerate}
%
\solution

{Complex Numbers:}
A complex number is a number that can be expressed in the form $a + bi$, where a and b are real numbers, and i represents the imaginary unit, satisfying the equation $i^2 =-1$.The set of complex numbers is denoted by $\mathbb{C}$
\begin{align}
    \mathbb{C}=\{(a,b):a,b \in \mathbb{R}\}
\end{align}
{Rational Numbers:}
A number in the form $\frac{p}{q}$, where both p and q(non-zero) are integers, is called a rational number.The set of rational numbers is dentoed by $\mathbb{Q}$
Let $\mathbb{Q}$ be the set of rational numbers.
\begin{align}
    \mathbb{Q}=\left\{\frac{p}{q}:p \in \mathbb{Z},q \in \mathbb{Z}_{\not=0} \right\}\label{eq:solutions/1/1/7/q}
\end{align}
Let $\mathbb{C}$ be the field of complex numbers and given $\mathbb{F}$ be the subfield of field of complex numbers $\mathbb{C}$ 
Since $\mathbb{F}$ is the subfield , we could say that 
\begin{align}
    0 &\in \mathbb{F} \label{eq:solutions/1/1/7/0}\\
    1 &\in \mathbb{F}
\end{align}
{Closed under addition:}
Here $\mathbb{F}$ is closed under addition since it is subfield
\begin{align}
    1+1=2&\in \mathbb{F}\\
    1+1+1=3&\in \mathbb{F}\\
    \vdots\notag\\
    1+1+\dots+1\text{(p times)}= p &\in \mathbb{F}\label{eq:solutions/1/1/7/p}\\
    1+1+\dots+1\text{(q times)}= q &\in \mathbb{F}\label{eq:solutions/1/1/7/q1}
\end{align}
By using the above property we could say that zero and other positive integers belongs to $\mathbb{F}$.Since $p$ and $q$ are integers we say,
\begin{align}
    p \in \mathbb{Z}\\
    q \in \mathbb{Z}\label{eq:solutions/1/1/7/0}
\end{align}
{Additive Inverse:}
Let $x$ be the positive integer belong $\mathbb{F}$ and by additive inverse we could say, 
\begin{align}
    \forall x &\in \mathbb{F}\label{eq:solutions/1/1/7/1}\\
    (-x) &\in \mathbb{F} \label{eq:solutions/1/1/7/2}
\end{align}
Therefore field $\mathbb{F}$ contains every integers. Let n be a integer then,
\begin{align}
    n \in \mathbb{Z} &\implies n \in \mathbb{F}\\
    \mathbb{Z} &\subseteq \mathbb{F}
\end{align}
Where $\mathbb{Z}$ is subset of $\mathbb{F}$
{Multiplicative Inverse:}
Every element except zero in the subfield $\mathbb{F}$ has an multiplicative inverse. From equation \eqref{eq:solutions/1/1/7/q1}, since q $\in \mathbb{F}$ we could say ,
\begin{align}
    \frac{1}{q} \in \mathbb{F} \quad{\text{and  }} q \not= 0\label{eq:solutions/1/1/7/3}
\end{align}
{Closed under multiplication:}
Also, $\mathbb{F}$ is closed under multiplication and thus,
from equation \eqref{eq:solutions/1/1/7/p} and \eqref{eq:solutions/1/1/7/3} we get , 
\begin{align}
    p\cdot\frac{1}{q} \in \mathbb{F}\\
\implies \frac{p}{q} \in \mathbb{F}\label{eq:solutions/1/1/7/proof}
\end{align}
where , $p \in \mathbb{Z}$ and $q \in \mathbb{Z}_{\not=0}$ (from equation \eqref{eq:solutions/1/1/7/0} and \eqref{eq:solutions/1/1/7/3})
{Conclusion}
From \eqref{eq:solutions/1/1/7/q} and \eqref{eq:solutions/1/1/7/proof} we could say , 
\begin{align}
    \mathbb{Q} \subseteq \mathbb{F}\label{eq:solutions/1/1/7/F}
\end{align}
From equation \eqref{eq:solutions/1/1/7/F} we could say that each subfield of the field of complex number contains every rational number
\begin{center}
    Hence Proved
\end{center}

%
\item Let $p,m,n$ be positive integers and $\mathbb{F}$ a field.Let $\vec{V}$ be the space of $m \times n$ matrices over $\mathbb{F}$ and $\vec{W}$ the space of $p \times n$ matrices over $\mathbb{F}$.Let $\vec{B}$ be a fixed $p \times m$ matrix and let $\mathbb{T}$ be the linear transformation from $\vec{V}$ into $\vec{W}$ defined by $\mathbb{T}(\vec{A})=\vec{B}\vec{A}$.Prove that $\mathbb{T}$ is invertible if and only if $p=m$ and $\vec{B}$ is an invertible $m \times m$ matrix. 
%
\solution

{Complex Numbers:}
A complex number is a number that can be expressed in the form $a + bi$, where a and b are real numbers, and i represents the imaginary unit, satisfying the equation $i^2 =-1$.The set of complex numbers is denoted by $\mathbb{C}$
\begin{align}
    \mathbb{C}=\{(a,b):a,b \in \mathbb{R}\}
\end{align}
{Rational Numbers:}
A number in the form $\frac{p}{q}$, where both p and q(non-zero) are integers, is called a rational number.The set of rational numbers is dentoed by $\mathbb{Q}$
Let $\mathbb{Q}$ be the set of rational numbers.
\begin{align}
    \mathbb{Q}=\left\{\frac{p}{q}:p \in \mathbb{Z},q \in \mathbb{Z}_{\not=0} \right\}\label{eq:solutions/1/1/7/q}
\end{align}
Let $\mathbb{C}$ be the field of complex numbers and given $\mathbb{F}$ be the subfield of field of complex numbers $\mathbb{C}$ 
Since $\mathbb{F}$ is the subfield , we could say that 
\begin{align}
    0 &\in \mathbb{F} \label{eq:solutions/1/1/7/0}\\
    1 &\in \mathbb{F}
\end{align}
{Closed under addition:}
Here $\mathbb{F}$ is closed under addition since it is subfield
\begin{align}
    1+1=2&\in \mathbb{F}\\
    1+1+1=3&\in \mathbb{F}\\
    \vdots\notag\\
    1+1+\dots+1\text{(p times)}= p &\in \mathbb{F}\label{eq:solutions/1/1/7/p}\\
    1+1+\dots+1\text{(q times)}= q &\in \mathbb{F}\label{eq:solutions/1/1/7/q1}
\end{align}
By using the above property we could say that zero and other positive integers belongs to $\mathbb{F}$.Since $p$ and $q$ are integers we say,
\begin{align}
    p \in \mathbb{Z}\\
    q \in \mathbb{Z}\label{eq:solutions/1/1/7/0}
\end{align}
{Additive Inverse:}
Let $x$ be the positive integer belong $\mathbb{F}$ and by additive inverse we could say, 
\begin{align}
    \forall x &\in \mathbb{F}\label{eq:solutions/1/1/7/1}\\
    (-x) &\in \mathbb{F} \label{eq:solutions/1/1/7/2}
\end{align}
Therefore field $\mathbb{F}$ contains every integers. Let n be a integer then,
\begin{align}
    n \in \mathbb{Z} &\implies n \in \mathbb{F}\\
    \mathbb{Z} &\subseteq \mathbb{F}
\end{align}
Where $\mathbb{Z}$ is subset of $\mathbb{F}$
{Multiplicative Inverse:}
Every element except zero in the subfield $\mathbb{F}$ has an multiplicative inverse. From equation \eqref{eq:solutions/1/1/7/q1}, since q $\in \mathbb{F}$ we could say ,
\begin{align}
    \frac{1}{q} \in \mathbb{F} \quad{\text{and  }} q \not= 0\label{eq:solutions/1/1/7/3}
\end{align}
{Closed under multiplication:}
Also, $\mathbb{F}$ is closed under multiplication and thus,
from equation \eqref{eq:solutions/1/1/7/p} and \eqref{eq:solutions/1/1/7/3} we get , 
\begin{align}
    p\cdot\frac{1}{q} \in \mathbb{F}\\
\implies \frac{p}{q} \in \mathbb{F}\label{eq:solutions/1/1/7/proof}
\end{align}
where , $p \in \mathbb{Z}$ and $q \in \mathbb{Z}_{\not=0}$ (from equation \eqref{eq:solutions/1/1/7/0} and \eqref{eq:solutions/1/1/7/3})
{Conclusion}
From \eqref{eq:solutions/1/1/7/q} and \eqref{eq:solutions/1/1/7/proof} we could say , 
\begin{align}
    \mathbb{Q} \subseteq \mathbb{F}\label{eq:solutions/1/1/7/F}
\end{align}
From equation \eqref{eq:solutions/1/1/7/F} we could say that each subfield of the field of complex number contains every rational number
\begin{center}
    Hence Proved
\end{center}


\end{enumerate}
