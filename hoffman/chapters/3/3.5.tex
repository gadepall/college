\renewcommand{\theequation}{\theenumi}
\renewcommand{\thefigure}{\theenumi}
\begin{enumerate}[label=\thesubsection.\arabic*.,ref=\thesubsection.\theenumi]
\numberwithin{equation}{enumi}
\numberwithin{figure}{enumi}


\item In $R^3$, let $\alpha_1 = \myvec{1\\0\\1}$, $\alpha_2 =\myvec{0\\1\\-2}$ and 
$\alpha_3 =\myvec{-1\\-1\\0}$. 

\begin{enumerate}
\item if f is a linear functional on $R^3$ such that\\ $f(\alpha_1)=1$, $f(\alpha_2)=-1$,  $f(\alpha_3)=3$,\\ And if $\alpha$=\myvec{a\\b\\c}, find $f(\alpha)$.
%
\\
\solution

{Complex Numbers:}
A complex number is a number that can be expressed in the form $a + bi$, where a and b are real numbers, and i represents the imaginary unit, satisfying the equation $i^2 =-1$.The set of complex numbers is denoted by $\mathbb{C}$
\begin{align}
    \mathbb{C}=\{(a,b):a,b \in \mathbb{R}\}
\end{align}
{Rational Numbers:}
A number in the form $\frac{p}{q}$, where both p and q(non-zero) are integers, is called a rational number.The set of rational numbers is dentoed by $\mathbb{Q}$
Let $\mathbb{Q}$ be the set of rational numbers.
\begin{align}
    \mathbb{Q}=\left\{\frac{p}{q}:p \in \mathbb{Z},q \in \mathbb{Z}_{\not=0} \right\}\label{eq:solutions/1/1/7/q}
\end{align}
Let $\mathbb{C}$ be the field of complex numbers and given $\mathbb{F}$ be the subfield of field of complex numbers $\mathbb{C}$ 
Since $\mathbb{F}$ is the subfield , we could say that 
\begin{align}
    0 &\in \mathbb{F} \label{eq:solutions/1/1/7/0}\\
    1 &\in \mathbb{F}
\end{align}
{Closed under addition:}
Here $\mathbb{F}$ is closed under addition since it is subfield
\begin{align}
    1+1=2&\in \mathbb{F}\\
    1+1+1=3&\in \mathbb{F}\\
    \vdots\notag\\
    1+1+\dots+1\text{(p times)}= p &\in \mathbb{F}\label{eq:solutions/1/1/7/p}\\
    1+1+\dots+1\text{(q times)}= q &\in \mathbb{F}\label{eq:solutions/1/1/7/q1}
\end{align}
By using the above property we could say that zero and other positive integers belongs to $\mathbb{F}$.Since $p$ and $q$ are integers we say,
\begin{align}
    p \in \mathbb{Z}\\
    q \in \mathbb{Z}\label{eq:solutions/1/1/7/0}
\end{align}
{Additive Inverse:}
Let $x$ be the positive integer belong $\mathbb{F}$ and by additive inverse we could say, 
\begin{align}
    \forall x &\in \mathbb{F}\label{eq:solutions/1/1/7/1}\\
    (-x) &\in \mathbb{F} \label{eq:solutions/1/1/7/2}
\end{align}
Therefore field $\mathbb{F}$ contains every integers. Let n be a integer then,
\begin{align}
    n \in \mathbb{Z} &\implies n \in \mathbb{F}\\
    \mathbb{Z} &\subseteq \mathbb{F}
\end{align}
Where $\mathbb{Z}$ is subset of $\mathbb{F}$
{Multiplicative Inverse:}
Every element except zero in the subfield $\mathbb{F}$ has an multiplicative inverse. From equation \eqref{eq:solutions/1/1/7/q1}, since q $\in \mathbb{F}$ we could say ,
\begin{align}
    \frac{1}{q} \in \mathbb{F} \quad{\text{and  }} q \not= 0\label{eq:solutions/1/1/7/3}
\end{align}
{Closed under multiplication:}
Also, $\mathbb{F}$ is closed under multiplication and thus,
from equation \eqref{eq:solutions/1/1/7/p} and \eqref{eq:solutions/1/1/7/3} we get , 
\begin{align}
    p\cdot\frac{1}{q} \in \mathbb{F}\\
\implies \frac{p}{q} \in \mathbb{F}\label{eq:solutions/1/1/7/proof}
\end{align}
where , $p \in \mathbb{Z}$ and $q \in \mathbb{Z}_{\not=0}$ (from equation \eqref{eq:solutions/1/1/7/0} and \eqref{eq:solutions/1/1/7/3})
{Conclusion}
From \eqref{eq:solutions/1/1/7/q} and \eqref{eq:solutions/1/1/7/proof} we could say , 
\begin{align}
    \mathbb{Q} \subseteq \mathbb{F}\label{eq:solutions/1/1/7/F}
\end{align}
From equation \eqref{eq:solutions/1/1/7/F} we could say that each subfield of the field of complex number contains every rational number
\begin{center}
    Hence Proved
\end{center}

\item Describe explicitly a linear functional f on $R^3$ such that $f(\alpha_1)=f(\alpha_2)=0$ but $f(\alpha_3)\neq 0$.
\\
\solution

{Complex Numbers:}
A complex number is a number that can be expressed in the form $a + bi$, where a and b are real numbers, and i represents the imaginary unit, satisfying the equation $i^2 =-1$.The set of complex numbers is denoted by $\mathbb{C}$
\begin{align}
    \mathbb{C}=\{(a,b):a,b \in \mathbb{R}\}
\end{align}
{Rational Numbers:}
A number in the form $\frac{p}{q}$, where both p and q(non-zero) are integers, is called a rational number.The set of rational numbers is dentoed by $\mathbb{Q}$
Let $\mathbb{Q}$ be the set of rational numbers.
\begin{align}
    \mathbb{Q}=\left\{\frac{p}{q}:p \in \mathbb{Z},q \in \mathbb{Z}_{\not=0} \right\}\label{eq:solutions/1/1/7/q}
\end{align}
Let $\mathbb{C}$ be the field of complex numbers and given $\mathbb{F}$ be the subfield of field of complex numbers $\mathbb{C}$ 
Since $\mathbb{F}$ is the subfield , we could say that 
\begin{align}
    0 &\in \mathbb{F} \label{eq:solutions/1/1/7/0}\\
    1 &\in \mathbb{F}
\end{align}
{Closed under addition:}
Here $\mathbb{F}$ is closed under addition since it is subfield
\begin{align}
    1+1=2&\in \mathbb{F}\\
    1+1+1=3&\in \mathbb{F}\\
    \vdots\notag\\
    1+1+\dots+1\text{(p times)}= p &\in \mathbb{F}\label{eq:solutions/1/1/7/p}\\
    1+1+\dots+1\text{(q times)}= q &\in \mathbb{F}\label{eq:solutions/1/1/7/q1}
\end{align}
By using the above property we could say that zero and other positive integers belongs to $\mathbb{F}$.Since $p$ and $q$ are integers we say,
\begin{align}
    p \in \mathbb{Z}\\
    q \in \mathbb{Z}\label{eq:solutions/1/1/7/0}
\end{align}
{Additive Inverse:}
Let $x$ be the positive integer belong $\mathbb{F}$ and by additive inverse we could say, 
\begin{align}
    \forall x &\in \mathbb{F}\label{eq:solutions/1/1/7/1}\\
    (-x) &\in \mathbb{F} \label{eq:solutions/1/1/7/2}
\end{align}
Therefore field $\mathbb{F}$ contains every integers. Let n be a integer then,
\begin{align}
    n \in \mathbb{Z} &\implies n \in \mathbb{F}\\
    \mathbb{Z} &\subseteq \mathbb{F}
\end{align}
Where $\mathbb{Z}$ is subset of $\mathbb{F}$
{Multiplicative Inverse:}
Every element except zero in the subfield $\mathbb{F}$ has an multiplicative inverse. From equation \eqref{eq:solutions/1/1/7/q1}, since q $\in \mathbb{F}$ we could say ,
\begin{align}
    \frac{1}{q} \in \mathbb{F} \quad{\text{and  }} q \not= 0\label{eq:solutions/1/1/7/3}
\end{align}
{Closed under multiplication:}
Also, $\mathbb{F}$ is closed under multiplication and thus,
from equation \eqref{eq:solutions/1/1/7/p} and \eqref{eq:solutions/1/1/7/3} we get , 
\begin{align}
    p\cdot\frac{1}{q} \in \mathbb{F}\\
\implies \frac{p}{q} \in \mathbb{F}\label{eq:solutions/1/1/7/proof}
\end{align}
where , $p \in \mathbb{Z}$ and $q \in \mathbb{Z}_{\not=0}$ (from equation \eqref{eq:solutions/1/1/7/0} and \eqref{eq:solutions/1/1/7/3})
{Conclusion}
From \eqref{eq:solutions/1/1/7/q} and \eqref{eq:solutions/1/1/7/proof} we could say , 
\begin{align}
    \mathbb{Q} \subseteq \mathbb{F}\label{eq:solutions/1/1/7/F}
\end{align}
From equation \eqref{eq:solutions/1/1/7/F} we could say that each subfield of the field of complex number contains every rational number
\begin{center}
    Hence Proved
\end{center}

\end{enumerate}
%
\item 	Let $\vec{B} = \cbrak{\alpha_1, \alpha_2, \alpha_3}$ be the basis for $\vec{C}^3$ defined by
	\begin{align}
		\alpha_1 = \myvec{1\\0\\-1}, \alpha_2 = \myvec{1\\1\\1}, \alpha_3 = \myvec{2\\2\\0}  
	\end{align}
Find the dual basis of $\vec{B}$.
%
\\
\solution

{Complex Numbers:}
A complex number is a number that can be expressed in the form $a + bi$, where a and b are real numbers, and i represents the imaginary unit, satisfying the equation $i^2 =-1$.The set of complex numbers is denoted by $\mathbb{C}$
\begin{align}
    \mathbb{C}=\{(a,b):a,b \in \mathbb{R}\}
\end{align}
{Rational Numbers:}
A number in the form $\frac{p}{q}$, where both p and q(non-zero) are integers, is called a rational number.The set of rational numbers is dentoed by $\mathbb{Q}$
Let $\mathbb{Q}$ be the set of rational numbers.
\begin{align}
    \mathbb{Q}=\left\{\frac{p}{q}:p \in \mathbb{Z},q \in \mathbb{Z}_{\not=0} \right\}\label{eq:solutions/1/1/7/q}
\end{align}
Let $\mathbb{C}$ be the field of complex numbers and given $\mathbb{F}$ be the subfield of field of complex numbers $\mathbb{C}$ 
Since $\mathbb{F}$ is the subfield , we could say that 
\begin{align}
    0 &\in \mathbb{F} \label{eq:solutions/1/1/7/0}\\
    1 &\in \mathbb{F}
\end{align}
{Closed under addition:}
Here $\mathbb{F}$ is closed under addition since it is subfield
\begin{align}
    1+1=2&\in \mathbb{F}\\
    1+1+1=3&\in \mathbb{F}\\
    \vdots\notag\\
    1+1+\dots+1\text{(p times)}= p &\in \mathbb{F}\label{eq:solutions/1/1/7/p}\\
    1+1+\dots+1\text{(q times)}= q &\in \mathbb{F}\label{eq:solutions/1/1/7/q1}
\end{align}
By using the above property we could say that zero and other positive integers belongs to $\mathbb{F}$.Since $p$ and $q$ are integers we say,
\begin{align}
    p \in \mathbb{Z}\\
    q \in \mathbb{Z}\label{eq:solutions/1/1/7/0}
\end{align}
{Additive Inverse:}
Let $x$ be the positive integer belong $\mathbb{F}$ and by additive inverse we could say, 
\begin{align}
    \forall x &\in \mathbb{F}\label{eq:solutions/1/1/7/1}\\
    (-x) &\in \mathbb{F} \label{eq:solutions/1/1/7/2}
\end{align}
Therefore field $\mathbb{F}$ contains every integers. Let n be a integer then,
\begin{align}
    n \in \mathbb{Z} &\implies n \in \mathbb{F}\\
    \mathbb{Z} &\subseteq \mathbb{F}
\end{align}
Where $\mathbb{Z}$ is subset of $\mathbb{F}$
{Multiplicative Inverse:}
Every element except zero in the subfield $\mathbb{F}$ has an multiplicative inverse. From equation \eqref{eq:solutions/1/1/7/q1}, since q $\in \mathbb{F}$ we could say ,
\begin{align}
    \frac{1}{q} \in \mathbb{F} \quad{\text{and  }} q \not= 0\label{eq:solutions/1/1/7/3}
\end{align}
{Closed under multiplication:}
Also, $\mathbb{F}$ is closed under multiplication and thus,
from equation \eqref{eq:solutions/1/1/7/p} and \eqref{eq:solutions/1/1/7/3} we get , 
\begin{align}
    p\cdot\frac{1}{q} \in \mathbb{F}\\
\implies \frac{p}{q} \in \mathbb{F}\label{eq:solutions/1/1/7/proof}
\end{align}
where , $p \in \mathbb{Z}$ and $q \in \mathbb{Z}_{\not=0}$ (from equation \eqref{eq:solutions/1/1/7/0} and \eqref{eq:solutions/1/1/7/3})
{Conclusion}
From \eqref{eq:solutions/1/1/7/q} and \eqref{eq:solutions/1/1/7/proof} we could say , 
\begin{align}
    \mathbb{Q} \subseteq \mathbb{F}\label{eq:solutions/1/1/7/F}
\end{align}
From equation \eqref{eq:solutions/1/1/7/F} we could say that each subfield of the field of complex number contains every rational number
\begin{center}
    Hence Proved
\end{center}

\item If $\vec{A}$ and $\vec{B}$ are $n\times n$ matrices over the field $\vec{F}$, show that trace($\vec{A}\vec{B}$) = trace($\vec{B}\vec{A}$). Now show that similar matrices have the same trace. 
%
\\
\solution

{Complex Numbers:}
A complex number is a number that can be expressed in the form $a + bi$, where a and b are real numbers, and i represents the imaginary unit, satisfying the equation $i^2 =-1$.The set of complex numbers is denoted by $\mathbb{C}$
\begin{align}
    \mathbb{C}=\{(a,b):a,b \in \mathbb{R}\}
\end{align}
{Rational Numbers:}
A number in the form $\frac{p}{q}$, where both p and q(non-zero) are integers, is called a rational number.The set of rational numbers is dentoed by $\mathbb{Q}$
Let $\mathbb{Q}$ be the set of rational numbers.
\begin{align}
    \mathbb{Q}=\left\{\frac{p}{q}:p \in \mathbb{Z},q \in \mathbb{Z}_{\not=0} \right\}\label{eq:solutions/1/1/7/q}
\end{align}
Let $\mathbb{C}$ be the field of complex numbers and given $\mathbb{F}$ be the subfield of field of complex numbers $\mathbb{C}$ 
Since $\mathbb{F}$ is the subfield , we could say that 
\begin{align}
    0 &\in \mathbb{F} \label{eq:solutions/1/1/7/0}\\
    1 &\in \mathbb{F}
\end{align}
{Closed under addition:}
Here $\mathbb{F}$ is closed under addition since it is subfield
\begin{align}
    1+1=2&\in \mathbb{F}\\
    1+1+1=3&\in \mathbb{F}\\
    \vdots\notag\\
    1+1+\dots+1\text{(p times)}= p &\in \mathbb{F}\label{eq:solutions/1/1/7/p}\\
    1+1+\dots+1\text{(q times)}= q &\in \mathbb{F}\label{eq:solutions/1/1/7/q1}
\end{align}
By using the above property we could say that zero and other positive integers belongs to $\mathbb{F}$.Since $p$ and $q$ are integers we say,
\begin{align}
    p \in \mathbb{Z}\\
    q \in \mathbb{Z}\label{eq:solutions/1/1/7/0}
\end{align}
{Additive Inverse:}
Let $x$ be the positive integer belong $\mathbb{F}$ and by additive inverse we could say, 
\begin{align}
    \forall x &\in \mathbb{F}\label{eq:solutions/1/1/7/1}\\
    (-x) &\in \mathbb{F} \label{eq:solutions/1/1/7/2}
\end{align}
Therefore field $\mathbb{F}$ contains every integers. Let n be a integer then,
\begin{align}
    n \in \mathbb{Z} &\implies n \in \mathbb{F}\\
    \mathbb{Z} &\subseteq \mathbb{F}
\end{align}
Where $\mathbb{Z}$ is subset of $\mathbb{F}$
{Multiplicative Inverse:}
Every element except zero in the subfield $\mathbb{F}$ has an multiplicative inverse. From equation \eqref{eq:solutions/1/1/7/q1}, since q $\in \mathbb{F}$ we could say ,
\begin{align}
    \frac{1}{q} \in \mathbb{F} \quad{\text{and  }} q \not= 0\label{eq:solutions/1/1/7/3}
\end{align}
{Closed under multiplication:}
Also, $\mathbb{F}$ is closed under multiplication and thus,
from equation \eqref{eq:solutions/1/1/7/p} and \eqref{eq:solutions/1/1/7/3} we get , 
\begin{align}
    p\cdot\frac{1}{q} \in \mathbb{F}\\
\implies \frac{p}{q} \in \mathbb{F}\label{eq:solutions/1/1/7/proof}
\end{align}
where , $p \in \mathbb{Z}$ and $q \in \mathbb{Z}_{\not=0}$ (from equation \eqref{eq:solutions/1/1/7/0} and \eqref{eq:solutions/1/1/7/3})
{Conclusion}
From \eqref{eq:solutions/1/1/7/q} and \eqref{eq:solutions/1/1/7/proof} we could say , 
\begin{align}
    \mathbb{Q} \subseteq \mathbb{F}\label{eq:solutions/1/1/7/F}
\end{align}
From equation \eqref{eq:solutions/1/1/7/F} we could say that each subfield of the field of complex number contains every rational number
\begin{center}
    Hence Proved
\end{center}

\item Let $\vec{V}$ be the vector space of all polynomial functions p from $\vec{R}$ into $\vec{R}$ which have degree 2 or less:
\begin{align}
    p(x) = c_0 + c_1x + c_2x^2 \nonumber
\end{align}
Define three linear functionals on $\vec{V}$ by
\begin{align}
    f_1(p) = \int_{0}^{1} p(x) \, dx; \: f_2(p) = \int_{0}^{2} p(x) \, dx; \nonumber \\
    f_3(p) = \int_{0}^{-1} p(x) \, dx \nonumber
\end{align}
Show that $\{f_1, f_2,f_3\}$ is a basis for $\vec{V}^*$ by exhibiting the basis for $\vec{V}$ of which it is the dual. 
%
\\
\solution

{Complex Numbers:}
A complex number is a number that can be expressed in the form $a + bi$, where a and b are real numbers, and i represents the imaginary unit, satisfying the equation $i^2 =-1$.The set of complex numbers is denoted by $\mathbb{C}$
\begin{align}
    \mathbb{C}=\{(a,b):a,b \in \mathbb{R}\}
\end{align}
{Rational Numbers:}
A number in the form $\frac{p}{q}$, where both p and q(non-zero) are integers, is called a rational number.The set of rational numbers is dentoed by $\mathbb{Q}$
Let $\mathbb{Q}$ be the set of rational numbers.
\begin{align}
    \mathbb{Q}=\left\{\frac{p}{q}:p \in \mathbb{Z},q \in \mathbb{Z}_{\not=0} \right\}\label{eq:solutions/1/1/7/q}
\end{align}
Let $\mathbb{C}$ be the field of complex numbers and given $\mathbb{F}$ be the subfield of field of complex numbers $\mathbb{C}$ 
Since $\mathbb{F}$ is the subfield , we could say that 
\begin{align}
    0 &\in \mathbb{F} \label{eq:solutions/1/1/7/0}\\
    1 &\in \mathbb{F}
\end{align}
{Closed under addition:}
Here $\mathbb{F}$ is closed under addition since it is subfield
\begin{align}
    1+1=2&\in \mathbb{F}\\
    1+1+1=3&\in \mathbb{F}\\
    \vdots\notag\\
    1+1+\dots+1\text{(p times)}= p &\in \mathbb{F}\label{eq:solutions/1/1/7/p}\\
    1+1+\dots+1\text{(q times)}= q &\in \mathbb{F}\label{eq:solutions/1/1/7/q1}
\end{align}
By using the above property we could say that zero and other positive integers belongs to $\mathbb{F}$.Since $p$ and $q$ are integers we say,
\begin{align}
    p \in \mathbb{Z}\\
    q \in \mathbb{Z}\label{eq:solutions/1/1/7/0}
\end{align}
{Additive Inverse:}
Let $x$ be the positive integer belong $\mathbb{F}$ and by additive inverse we could say, 
\begin{align}
    \forall x &\in \mathbb{F}\label{eq:solutions/1/1/7/1}\\
    (-x) &\in \mathbb{F} \label{eq:solutions/1/1/7/2}
\end{align}
Therefore field $\mathbb{F}$ contains every integers. Let n be a integer then,
\begin{align}
    n \in \mathbb{Z} &\implies n \in \mathbb{F}\\
    \mathbb{Z} &\subseteq \mathbb{F}
\end{align}
Where $\mathbb{Z}$ is subset of $\mathbb{F}$
{Multiplicative Inverse:}
Every element except zero in the subfield $\mathbb{F}$ has an multiplicative inverse. From equation \eqref{eq:solutions/1/1/7/q1}, since q $\in \mathbb{F}$ we could say ,
\begin{align}
    \frac{1}{q} \in \mathbb{F} \quad{\text{and  }} q \not= 0\label{eq:solutions/1/1/7/3}
\end{align}
{Closed under multiplication:}
Also, $\mathbb{F}$ is closed under multiplication and thus,
from equation \eqref{eq:solutions/1/1/7/p} and \eqref{eq:solutions/1/1/7/3} we get , 
\begin{align}
    p\cdot\frac{1}{q} \in \mathbb{F}\\
\implies \frac{p}{q} \in \mathbb{F}\label{eq:solutions/1/1/7/proof}
\end{align}
where , $p \in \mathbb{Z}$ and $q \in \mathbb{Z}_{\not=0}$ (from equation \eqref{eq:solutions/1/1/7/0} and \eqref{eq:solutions/1/1/7/3})
{Conclusion}
From \eqref{eq:solutions/1/1/7/q} and \eqref{eq:solutions/1/1/7/proof} we could say , 
\begin{align}
    \mathbb{Q} \subseteq \mathbb{F}\label{eq:solutions/1/1/7/F}
\end{align}
From equation \eqref{eq:solutions/1/1/7/F} we could say that each subfield of the field of complex number contains every rational number
\begin{center}
    Hence Proved
\end{center}

\item If $\vec{A}$ and $\vec{B}$ are $n\times n$ matrices , show that
\begin{align}
 \vec{A}\vec{B}-\vec{B}\vec{A} = \vec{I}\label{eq:solutions/3/5/5/3}  
\end{align}
is impossible.
%
\\
\solution

{Complex Numbers:}
A complex number is a number that can be expressed in the form $a + bi$, where a and b are real numbers, and i represents the imaginary unit, satisfying the equation $i^2 =-1$.The set of complex numbers is denoted by $\mathbb{C}$
\begin{align}
    \mathbb{C}=\{(a,b):a,b \in \mathbb{R}\}
\end{align}
{Rational Numbers:}
A number in the form $\frac{p}{q}$, where both p and q(non-zero) are integers, is called a rational number.The set of rational numbers is dentoed by $\mathbb{Q}$
Let $\mathbb{Q}$ be the set of rational numbers.
\begin{align}
    \mathbb{Q}=\left\{\frac{p}{q}:p \in \mathbb{Z},q \in \mathbb{Z}_{\not=0} \right\}\label{eq:solutions/1/1/7/q}
\end{align}
Let $\mathbb{C}$ be the field of complex numbers and given $\mathbb{F}$ be the subfield of field of complex numbers $\mathbb{C}$ 
Since $\mathbb{F}$ is the subfield , we could say that 
\begin{align}
    0 &\in \mathbb{F} \label{eq:solutions/1/1/7/0}\\
    1 &\in \mathbb{F}
\end{align}
{Closed under addition:}
Here $\mathbb{F}$ is closed under addition since it is subfield
\begin{align}
    1+1=2&\in \mathbb{F}\\
    1+1+1=3&\in \mathbb{F}\\
    \vdots\notag\\
    1+1+\dots+1\text{(p times)}= p &\in \mathbb{F}\label{eq:solutions/1/1/7/p}\\
    1+1+\dots+1\text{(q times)}= q &\in \mathbb{F}\label{eq:solutions/1/1/7/q1}
\end{align}
By using the above property we could say that zero and other positive integers belongs to $\mathbb{F}$.Since $p$ and $q$ are integers we say,
\begin{align}
    p \in \mathbb{Z}\\
    q \in \mathbb{Z}\label{eq:solutions/1/1/7/0}
\end{align}
{Additive Inverse:}
Let $x$ be the positive integer belong $\mathbb{F}$ and by additive inverse we could say, 
\begin{align}
    \forall x &\in \mathbb{F}\label{eq:solutions/1/1/7/1}\\
    (-x) &\in \mathbb{F} \label{eq:solutions/1/1/7/2}
\end{align}
Therefore field $\mathbb{F}$ contains every integers. Let n be a integer then,
\begin{align}
    n \in \mathbb{Z} &\implies n \in \mathbb{F}\\
    \mathbb{Z} &\subseteq \mathbb{F}
\end{align}
Where $\mathbb{Z}$ is subset of $\mathbb{F}$
{Multiplicative Inverse:}
Every element except zero in the subfield $\mathbb{F}$ has an multiplicative inverse. From equation \eqref{eq:solutions/1/1/7/q1}, since q $\in \mathbb{F}$ we could say ,
\begin{align}
    \frac{1}{q} \in \mathbb{F} \quad{\text{and  }} q \not= 0\label{eq:solutions/1/1/7/3}
\end{align}
{Closed under multiplication:}
Also, $\mathbb{F}$ is closed under multiplication and thus,
from equation \eqref{eq:solutions/1/1/7/p} and \eqref{eq:solutions/1/1/7/3} we get , 
\begin{align}
    p\cdot\frac{1}{q} \in \mathbb{F}\\
\implies \frac{p}{q} \in \mathbb{F}\label{eq:solutions/1/1/7/proof}
\end{align}
where , $p \in \mathbb{Z}$ and $q \in \mathbb{Z}_{\not=0}$ (from equation \eqref{eq:solutions/1/1/7/0} and \eqref{eq:solutions/1/1/7/3})
{Conclusion}
From \eqref{eq:solutions/1/1/7/q} and \eqref{eq:solutions/1/1/7/proof} we could say , 
\begin{align}
    \mathbb{Q} \subseteq \mathbb{F}\label{eq:solutions/1/1/7/F}
\end{align}
From equation \eqref{eq:solutions/1/1/7/F} we could say that each subfield of the field of complex number contains every rational number
\begin{center}
    Hence Proved
\end{center}

\item Let m and n be positive integers and field $\vec{F}$.Let $f_1,$\dots$,f_m$ be linear functions in $F^n$.For $\vec{\alpha}$ in $F^n$ define
\begin{align}
    T\vec{\vec{\alpha}}=(f_1(\vec{\alpha}),\dots,f_m(\vec{\alpha})) \label{eq:solutions/3/5/6/eq:main}
\end{align}
show that T is a linear transformation from $F^n$ into $F^m$.Then show that every linear transformation from $F^n$ into $F^m$ is of the above form ,for some $f_1,$\dots$,f_m$.
%
\\
\solution

{Complex Numbers:}
A complex number is a number that can be expressed in the form $a + bi$, where a and b are real numbers, and i represents the imaginary unit, satisfying the equation $i^2 =-1$.The set of complex numbers is denoted by $\mathbb{C}$
\begin{align}
    \mathbb{C}=\{(a,b):a,b \in \mathbb{R}\}
\end{align}
{Rational Numbers:}
A number in the form $\frac{p}{q}$, where both p and q(non-zero) are integers, is called a rational number.The set of rational numbers is dentoed by $\mathbb{Q}$
Let $\mathbb{Q}$ be the set of rational numbers.
\begin{align}
    \mathbb{Q}=\left\{\frac{p}{q}:p \in \mathbb{Z},q \in \mathbb{Z}_{\not=0} \right\}\label{eq:solutions/1/1/7/q}
\end{align}
Let $\mathbb{C}$ be the field of complex numbers and given $\mathbb{F}$ be the subfield of field of complex numbers $\mathbb{C}$ 
Since $\mathbb{F}$ is the subfield , we could say that 
\begin{align}
    0 &\in \mathbb{F} \label{eq:solutions/1/1/7/0}\\
    1 &\in \mathbb{F}
\end{align}
{Closed under addition:}
Here $\mathbb{F}$ is closed under addition since it is subfield
\begin{align}
    1+1=2&\in \mathbb{F}\\
    1+1+1=3&\in \mathbb{F}\\
    \vdots\notag\\
    1+1+\dots+1\text{(p times)}= p &\in \mathbb{F}\label{eq:solutions/1/1/7/p}\\
    1+1+\dots+1\text{(q times)}= q &\in \mathbb{F}\label{eq:solutions/1/1/7/q1}
\end{align}
By using the above property we could say that zero and other positive integers belongs to $\mathbb{F}$.Since $p$ and $q$ are integers we say,
\begin{align}
    p \in \mathbb{Z}\\
    q \in \mathbb{Z}\label{eq:solutions/1/1/7/0}
\end{align}
{Additive Inverse:}
Let $x$ be the positive integer belong $\mathbb{F}$ and by additive inverse we could say, 
\begin{align}
    \forall x &\in \mathbb{F}\label{eq:solutions/1/1/7/1}\\
    (-x) &\in \mathbb{F} \label{eq:solutions/1/1/7/2}
\end{align}
Therefore field $\mathbb{F}$ contains every integers. Let n be a integer then,
\begin{align}
    n \in \mathbb{Z} &\implies n \in \mathbb{F}\\
    \mathbb{Z} &\subseteq \mathbb{F}
\end{align}
Where $\mathbb{Z}$ is subset of $\mathbb{F}$
{Multiplicative Inverse:}
Every element except zero in the subfield $\mathbb{F}$ has an multiplicative inverse. From equation \eqref{eq:solutions/1/1/7/q1}, since q $\in \mathbb{F}$ we could say ,
\begin{align}
    \frac{1}{q} \in \mathbb{F} \quad{\text{and  }} q \not= 0\label{eq:solutions/1/1/7/3}
\end{align}
{Closed under multiplication:}
Also, $\mathbb{F}$ is closed under multiplication and thus,
from equation \eqref{eq:solutions/1/1/7/p} and \eqref{eq:solutions/1/1/7/3} we get , 
\begin{align}
    p\cdot\frac{1}{q} \in \mathbb{F}\\
\implies \frac{p}{q} \in \mathbb{F}\label{eq:solutions/1/1/7/proof}
\end{align}
where , $p \in \mathbb{Z}$ and $q \in \mathbb{Z}_{\not=0}$ (from equation \eqref{eq:solutions/1/1/7/0} and \eqref{eq:solutions/1/1/7/3})
{Conclusion}
From \eqref{eq:solutions/1/1/7/q} and \eqref{eq:solutions/1/1/7/proof} we could say , 
\begin{align}
    \mathbb{Q} \subseteq \mathbb{F}\label{eq:solutions/1/1/7/F}
\end{align}
From equation \eqref{eq:solutions/1/1/7/F} we could say that each subfield of the field of complex number contains every rational number
\begin{center}
    Hence Proved
\end{center}

\item Let $\alpha_1$ = $(1, 0,-1, 2)$ and $\alpha_2$ = $(2,3, 1,1)$ and let $\vec{W}$ be the subspace of $\mathbb{R}^4$ spanned by $\alpha_1$ and $\alpha_2$. Which linear functionals $\vec{f}$ :
\begin{align}
f(x_1,x_2,x_3,x_4) = c_1x_1 + c_2x_2 + c_3x_3 + c_4x_4 \label{eq:solutions/3/5/7/3} 
\end{align}
are in the annihilator of $\vec{W}$?
%
\\
\solution

{Complex Numbers:}
A complex number is a number that can be expressed in the form $a + bi$, where a and b are real numbers, and i represents the imaginary unit, satisfying the equation $i^2 =-1$.The set of complex numbers is denoted by $\mathbb{C}$
\begin{align}
    \mathbb{C}=\{(a,b):a,b \in \mathbb{R}\}
\end{align}
{Rational Numbers:}
A number in the form $\frac{p}{q}$, where both p and q(non-zero) are integers, is called a rational number.The set of rational numbers is dentoed by $\mathbb{Q}$
Let $\mathbb{Q}$ be the set of rational numbers.
\begin{align}
    \mathbb{Q}=\left\{\frac{p}{q}:p \in \mathbb{Z},q \in \mathbb{Z}_{\not=0} \right\}\label{eq:solutions/1/1/7/q}
\end{align}
Let $\mathbb{C}$ be the field of complex numbers and given $\mathbb{F}$ be the subfield of field of complex numbers $\mathbb{C}$ 
Since $\mathbb{F}$ is the subfield , we could say that 
\begin{align}
    0 &\in \mathbb{F} \label{eq:solutions/1/1/7/0}\\
    1 &\in \mathbb{F}
\end{align}
{Closed under addition:}
Here $\mathbb{F}$ is closed under addition since it is subfield
\begin{align}
    1+1=2&\in \mathbb{F}\\
    1+1+1=3&\in \mathbb{F}\\
    \vdots\notag\\
    1+1+\dots+1\text{(p times)}= p &\in \mathbb{F}\label{eq:solutions/1/1/7/p}\\
    1+1+\dots+1\text{(q times)}= q &\in \mathbb{F}\label{eq:solutions/1/1/7/q1}
\end{align}
By using the above property we could say that zero and other positive integers belongs to $\mathbb{F}$.Since $p$ and $q$ are integers we say,
\begin{align}
    p \in \mathbb{Z}\\
    q \in \mathbb{Z}\label{eq:solutions/1/1/7/0}
\end{align}
{Additive Inverse:}
Let $x$ be the positive integer belong $\mathbb{F}$ and by additive inverse we could say, 
\begin{align}
    \forall x &\in \mathbb{F}\label{eq:solutions/1/1/7/1}\\
    (-x) &\in \mathbb{F} \label{eq:solutions/1/1/7/2}
\end{align}
Therefore field $\mathbb{F}$ contains every integers. Let n be a integer then,
\begin{align}
    n \in \mathbb{Z} &\implies n \in \mathbb{F}\\
    \mathbb{Z} &\subseteq \mathbb{F}
\end{align}
Where $\mathbb{Z}$ is subset of $\mathbb{F}$
{Multiplicative Inverse:}
Every element except zero in the subfield $\mathbb{F}$ has an multiplicative inverse. From equation \eqref{eq:solutions/1/1/7/q1}, since q $\in \mathbb{F}$ we could say ,
\begin{align}
    \frac{1}{q} \in \mathbb{F} \quad{\text{and  }} q \not= 0\label{eq:solutions/1/1/7/3}
\end{align}
{Closed under multiplication:}
Also, $\mathbb{F}$ is closed under multiplication and thus,
from equation \eqref{eq:solutions/1/1/7/p} and \eqref{eq:solutions/1/1/7/3} we get , 
\begin{align}
    p\cdot\frac{1}{q} \in \mathbb{F}\\
\implies \frac{p}{q} \in \mathbb{F}\label{eq:solutions/1/1/7/proof}
\end{align}
where , $p \in \mathbb{Z}$ and $q \in \mathbb{Z}_{\not=0}$ (from equation \eqref{eq:solutions/1/1/7/0} and \eqref{eq:solutions/1/1/7/3})
{Conclusion}
From \eqref{eq:solutions/1/1/7/q} and \eqref{eq:solutions/1/1/7/proof} we could say , 
\begin{align}
    \mathbb{Q} \subseteq \mathbb{F}\label{eq:solutions/1/1/7/F}
\end{align}
From equation \eqref{eq:solutions/1/1/7/F} we could say that each subfield of the field of complex number contains every rational number
\begin{center}
    Hence Proved
\end{center}

\item Let $\vec{W}$ be the subspace of $\mathbb{R}^5$ which is spanned by the vectors  
   \begin{multline}
    \begin{aligned}
    &\alpha_1=\epsilon_1+2\epsilon_2+\epsilon_3,\\ &\alpha_2=\epsilon_2+3\epsilon_3+3\epsilon_4+\epsilon_5,\\&\alpha_3=\epsilon_1+4\epsilon_2+6\epsilon_3+4\epsilon_4+\epsilon_5
    \end{aligned}
    \end{multline}
    Find a basis for $\vec{W^0}$
%
\\
\solution

{Complex Numbers:}
A complex number is a number that can be expressed in the form $a + bi$, where a and b are real numbers, and i represents the imaginary unit, satisfying the equation $i^2 =-1$.The set of complex numbers is denoted by $\mathbb{C}$
\begin{align}
    \mathbb{C}=\{(a,b):a,b \in \mathbb{R}\}
\end{align}
{Rational Numbers:}
A number in the form $\frac{p}{q}$, where both p and q(non-zero) are integers, is called a rational number.The set of rational numbers is dentoed by $\mathbb{Q}$
Let $\mathbb{Q}$ be the set of rational numbers.
\begin{align}
    \mathbb{Q}=\left\{\frac{p}{q}:p \in \mathbb{Z},q \in \mathbb{Z}_{\not=0} \right\}\label{eq:solutions/1/1/7/q}
\end{align}
Let $\mathbb{C}$ be the field of complex numbers and given $\mathbb{F}$ be the subfield of field of complex numbers $\mathbb{C}$ 
Since $\mathbb{F}$ is the subfield , we could say that 
\begin{align}
    0 &\in \mathbb{F} \label{eq:solutions/1/1/7/0}\\
    1 &\in \mathbb{F}
\end{align}
{Closed under addition:}
Here $\mathbb{F}$ is closed under addition since it is subfield
\begin{align}
    1+1=2&\in \mathbb{F}\\
    1+1+1=3&\in \mathbb{F}\\
    \vdots\notag\\
    1+1+\dots+1\text{(p times)}= p &\in \mathbb{F}\label{eq:solutions/1/1/7/p}\\
    1+1+\dots+1\text{(q times)}= q &\in \mathbb{F}\label{eq:solutions/1/1/7/q1}
\end{align}
By using the above property we could say that zero and other positive integers belongs to $\mathbb{F}$.Since $p$ and $q$ are integers we say,
\begin{align}
    p \in \mathbb{Z}\\
    q \in \mathbb{Z}\label{eq:solutions/1/1/7/0}
\end{align}
{Additive Inverse:}
Let $x$ be the positive integer belong $\mathbb{F}$ and by additive inverse we could say, 
\begin{align}
    \forall x &\in \mathbb{F}\label{eq:solutions/1/1/7/1}\\
    (-x) &\in \mathbb{F} \label{eq:solutions/1/1/7/2}
\end{align}
Therefore field $\mathbb{F}$ contains every integers. Let n be a integer then,
\begin{align}
    n \in \mathbb{Z} &\implies n \in \mathbb{F}\\
    \mathbb{Z} &\subseteq \mathbb{F}
\end{align}
Where $\mathbb{Z}$ is subset of $\mathbb{F}$
{Multiplicative Inverse:}
Every element except zero in the subfield $\mathbb{F}$ has an multiplicative inverse. From equation \eqref{eq:solutions/1/1/7/q1}, since q $\in \mathbb{F}$ we could say ,
\begin{align}
    \frac{1}{q} \in \mathbb{F} \quad{\text{and  }} q \not= 0\label{eq:solutions/1/1/7/3}
\end{align}
{Closed under multiplication:}
Also, $\mathbb{F}$ is closed under multiplication and thus,
from equation \eqref{eq:solutions/1/1/7/p} and \eqref{eq:solutions/1/1/7/3} we get , 
\begin{align}
    p\cdot\frac{1}{q} \in \mathbb{F}\\
\implies \frac{p}{q} \in \mathbb{F}\label{eq:solutions/1/1/7/proof}
\end{align}
where , $p \in \mathbb{Z}$ and $q \in \mathbb{Z}_{\not=0}$ (from equation \eqref{eq:solutions/1/1/7/0} and \eqref{eq:solutions/1/1/7/3})
{Conclusion}
From \eqref{eq:solutions/1/1/7/q} and \eqref{eq:solutions/1/1/7/proof} we could say , 
\begin{align}
    \mathbb{Q} \subseteq \mathbb{F}\label{eq:solutions/1/1/7/F}
\end{align}
From equation \eqref{eq:solutions/1/1/7/F} we could say that each subfield of the field of complex number contains every rational number
\begin{center}
    Hence Proved
\end{center}

\item Let $\mathbb{V}$ be the vector space of all $2 \times 2$ matrices over the field of real numbers, and let
\begin{align}
\vec{B} &= \myvec{2&-2\\-1&1}
\end{align}
Let $\mathbb{W}$ be the subspace of $\mathbb{V}$ consisting of all $\vec{A}$ such that $\vec{AB} = 0$. Let $f$ be a linear functional on $\mathbb{V}$ which is in the annihilator of $\mathbb{W}$. Suppose that $f(\vec{I}) = 0$ and $f(\vec{C}) = 3$, where $\vec{I}$ is the $2 \times 2$ identity matrix and
\begin{align}
\vec{C} &= \myvec{0&0\\0&1}
\end{align}
Find $f(\vec{B})$
%
\\
\solution

{Complex Numbers:}
A complex number is a number that can be expressed in the form $a + bi$, where a and b are real numbers, and i represents the imaginary unit, satisfying the equation $i^2 =-1$.The set of complex numbers is denoted by $\mathbb{C}$
\begin{align}
    \mathbb{C}=\{(a,b):a,b \in \mathbb{R}\}
\end{align}
{Rational Numbers:}
A number in the form $\frac{p}{q}$, where both p and q(non-zero) are integers, is called a rational number.The set of rational numbers is dentoed by $\mathbb{Q}$
Let $\mathbb{Q}$ be the set of rational numbers.
\begin{align}
    \mathbb{Q}=\left\{\frac{p}{q}:p \in \mathbb{Z},q \in \mathbb{Z}_{\not=0} \right\}\label{eq:solutions/1/1/7/q}
\end{align}
Let $\mathbb{C}$ be the field of complex numbers and given $\mathbb{F}$ be the subfield of field of complex numbers $\mathbb{C}$ 
Since $\mathbb{F}$ is the subfield , we could say that 
\begin{align}
    0 &\in \mathbb{F} \label{eq:solutions/1/1/7/0}\\
    1 &\in \mathbb{F}
\end{align}
{Closed under addition:}
Here $\mathbb{F}$ is closed under addition since it is subfield
\begin{align}
    1+1=2&\in \mathbb{F}\\
    1+1+1=3&\in \mathbb{F}\\
    \vdots\notag\\
    1+1+\dots+1\text{(p times)}= p &\in \mathbb{F}\label{eq:solutions/1/1/7/p}\\
    1+1+\dots+1\text{(q times)}= q &\in \mathbb{F}\label{eq:solutions/1/1/7/q1}
\end{align}
By using the above property we could say that zero and other positive integers belongs to $\mathbb{F}$.Since $p$ and $q$ are integers we say,
\begin{align}
    p \in \mathbb{Z}\\
    q \in \mathbb{Z}\label{eq:solutions/1/1/7/0}
\end{align}
{Additive Inverse:}
Let $x$ be the positive integer belong $\mathbb{F}$ and by additive inverse we could say, 
\begin{align}
    \forall x &\in \mathbb{F}\label{eq:solutions/1/1/7/1}\\
    (-x) &\in \mathbb{F} \label{eq:solutions/1/1/7/2}
\end{align}
Therefore field $\mathbb{F}$ contains every integers. Let n be a integer then,
\begin{align}
    n \in \mathbb{Z} &\implies n \in \mathbb{F}\\
    \mathbb{Z} &\subseteq \mathbb{F}
\end{align}
Where $\mathbb{Z}$ is subset of $\mathbb{F}$
{Multiplicative Inverse:}
Every element except zero in the subfield $\mathbb{F}$ has an multiplicative inverse. From equation \eqref{eq:solutions/1/1/7/q1}, since q $\in \mathbb{F}$ we could say ,
\begin{align}
    \frac{1}{q} \in \mathbb{F} \quad{\text{and  }} q \not= 0\label{eq:solutions/1/1/7/3}
\end{align}
{Closed under multiplication:}
Also, $\mathbb{F}$ is closed under multiplication and thus,
from equation \eqref{eq:solutions/1/1/7/p} and \eqref{eq:solutions/1/1/7/3} we get , 
\begin{align}
    p\cdot\frac{1}{q} \in \mathbb{F}\\
\implies \frac{p}{q} \in \mathbb{F}\label{eq:solutions/1/1/7/proof}
\end{align}
where , $p \in \mathbb{Z}$ and $q \in \mathbb{Z}_{\not=0}$ (from equation \eqref{eq:solutions/1/1/7/0} and \eqref{eq:solutions/1/1/7/3})
{Conclusion}
From \eqref{eq:solutions/1/1/7/q} and \eqref{eq:solutions/1/1/7/proof} we could say , 
\begin{align}
    \mathbb{Q} \subseteq \mathbb{F}\label{eq:solutions/1/1/7/F}
\end{align}
From equation \eqref{eq:solutions/1/1/7/F} we could say that each subfield of the field of complex number contains every rational number
\begin{center}
    Hence Proved
\end{center}

\item Let $F$ be a subfield of the complex numbers. We define n linear functionals on $F^n(n \ge 2)$ by
\begin{align}
    f_k(x_1,....,x_n) &= \sum_{j=1}^{n}(k-j)x_j, 
    1 \le k \le n.
\end{align}
What is the dimension of the subspace annihilated by $f_1,f_2,...,f_n$ ?
%
\\
\solution

{Complex Numbers:}
A complex number is a number that can be expressed in the form $a + bi$, where a and b are real numbers, and i represents the imaginary unit, satisfying the equation $i^2 =-1$.The set of complex numbers is denoted by $\mathbb{C}$
\begin{align}
    \mathbb{C}=\{(a,b):a,b \in \mathbb{R}\}
\end{align}
{Rational Numbers:}
A number in the form $\frac{p}{q}$, where both p and q(non-zero) are integers, is called a rational number.The set of rational numbers is dentoed by $\mathbb{Q}$
Let $\mathbb{Q}$ be the set of rational numbers.
\begin{align}
    \mathbb{Q}=\left\{\frac{p}{q}:p \in \mathbb{Z},q \in \mathbb{Z}_{\not=0} \right\}\label{eq:solutions/1/1/7/q}
\end{align}
Let $\mathbb{C}$ be the field of complex numbers and given $\mathbb{F}$ be the subfield of field of complex numbers $\mathbb{C}$ 
Since $\mathbb{F}$ is the subfield , we could say that 
\begin{align}
    0 &\in \mathbb{F} \label{eq:solutions/1/1/7/0}\\
    1 &\in \mathbb{F}
\end{align}
{Closed under addition:}
Here $\mathbb{F}$ is closed under addition since it is subfield
\begin{align}
    1+1=2&\in \mathbb{F}\\
    1+1+1=3&\in \mathbb{F}\\
    \vdots\notag\\
    1+1+\dots+1\text{(p times)}= p &\in \mathbb{F}\label{eq:solutions/1/1/7/p}\\
    1+1+\dots+1\text{(q times)}= q &\in \mathbb{F}\label{eq:solutions/1/1/7/q1}
\end{align}
By using the above property we could say that zero and other positive integers belongs to $\mathbb{F}$.Since $p$ and $q$ are integers we say,
\begin{align}
    p \in \mathbb{Z}\\
    q \in \mathbb{Z}\label{eq:solutions/1/1/7/0}
\end{align}
{Additive Inverse:}
Let $x$ be the positive integer belong $\mathbb{F}$ and by additive inverse we could say, 
\begin{align}
    \forall x &\in \mathbb{F}\label{eq:solutions/1/1/7/1}\\
    (-x) &\in \mathbb{F} \label{eq:solutions/1/1/7/2}
\end{align}
Therefore field $\mathbb{F}$ contains every integers. Let n be a integer then,
\begin{align}
    n \in \mathbb{Z} &\implies n \in \mathbb{F}\\
    \mathbb{Z} &\subseteq \mathbb{F}
\end{align}
Where $\mathbb{Z}$ is subset of $\mathbb{F}$
{Multiplicative Inverse:}
Every element except zero in the subfield $\mathbb{F}$ has an multiplicative inverse. From equation \eqref{eq:solutions/1/1/7/q1}, since q $\in \mathbb{F}$ we could say ,
\begin{align}
    \frac{1}{q} \in \mathbb{F} \quad{\text{and  }} q \not= 0\label{eq:solutions/1/1/7/3}
\end{align}
{Closed under multiplication:}
Also, $\mathbb{F}$ is closed under multiplication and thus,
from equation \eqref{eq:solutions/1/1/7/p} and \eqref{eq:solutions/1/1/7/3} we get , 
\begin{align}
    p\cdot\frac{1}{q} \in \mathbb{F}\\
\implies \frac{p}{q} \in \mathbb{F}\label{eq:solutions/1/1/7/proof}
\end{align}
where , $p \in \mathbb{Z}$ and $q \in \mathbb{Z}_{\not=0}$ (from equation \eqref{eq:solutions/1/1/7/0} and \eqref{eq:solutions/1/1/7/3})
{Conclusion}
From \eqref{eq:solutions/1/1/7/q} and \eqref{eq:solutions/1/1/7/proof} we could say , 
\begin{align}
    \mathbb{Q} \subseteq \mathbb{F}\label{eq:solutions/1/1/7/F}
\end{align}
From equation \eqref{eq:solutions/1/1/7/F} we could say that each subfield of the field of complex number contains every rational number
\begin{center}
    Hence Proved
\end{center}

\item Let $W_1$ and $W_2$ be subspaces of a finite-dimensional vector space $\mathbb V$. Prove that
\begin{enumerate}
    \item $(W_1 + W_2)^0 = W_1^0 \cap W_2^0$
    \item $(W_1 \cap W_2)^0 = W_1^0 + W_2^0$
\end{enumerate}
%
\solution

{Complex Numbers:}
A complex number is a number that can be expressed in the form $a + bi$, where a and b are real numbers, and i represents the imaginary unit, satisfying the equation $i^2 =-1$.The set of complex numbers is denoted by $\mathbb{C}$
\begin{align}
    \mathbb{C}=\{(a,b):a,b \in \mathbb{R}\}
\end{align}
{Rational Numbers:}
A number in the form $\frac{p}{q}$, where both p and q(non-zero) are integers, is called a rational number.The set of rational numbers is dentoed by $\mathbb{Q}$
Let $\mathbb{Q}$ be the set of rational numbers.
\begin{align}
    \mathbb{Q}=\left\{\frac{p}{q}:p \in \mathbb{Z},q \in \mathbb{Z}_{\not=0} \right\}\label{eq:solutions/1/1/7/q}
\end{align}
Let $\mathbb{C}$ be the field of complex numbers and given $\mathbb{F}$ be the subfield of field of complex numbers $\mathbb{C}$ 
Since $\mathbb{F}$ is the subfield , we could say that 
\begin{align}
    0 &\in \mathbb{F} \label{eq:solutions/1/1/7/0}\\
    1 &\in \mathbb{F}
\end{align}
{Closed under addition:}
Here $\mathbb{F}$ is closed under addition since it is subfield
\begin{align}
    1+1=2&\in \mathbb{F}\\
    1+1+1=3&\in \mathbb{F}\\
    \vdots\notag\\
    1+1+\dots+1\text{(p times)}= p &\in \mathbb{F}\label{eq:solutions/1/1/7/p}\\
    1+1+\dots+1\text{(q times)}= q &\in \mathbb{F}\label{eq:solutions/1/1/7/q1}
\end{align}
By using the above property we could say that zero and other positive integers belongs to $\mathbb{F}$.Since $p$ and $q$ are integers we say,
\begin{align}
    p \in \mathbb{Z}\\
    q \in \mathbb{Z}\label{eq:solutions/1/1/7/0}
\end{align}
{Additive Inverse:}
Let $x$ be the positive integer belong $\mathbb{F}$ and by additive inverse we could say, 
\begin{align}
    \forall x &\in \mathbb{F}\label{eq:solutions/1/1/7/1}\\
    (-x) &\in \mathbb{F} \label{eq:solutions/1/1/7/2}
\end{align}
Therefore field $\mathbb{F}$ contains every integers. Let n be a integer then,
\begin{align}
    n \in \mathbb{Z} &\implies n \in \mathbb{F}\\
    \mathbb{Z} &\subseteq \mathbb{F}
\end{align}
Where $\mathbb{Z}$ is subset of $\mathbb{F}$
{Multiplicative Inverse:}
Every element except zero in the subfield $\mathbb{F}$ has an multiplicative inverse. From equation \eqref{eq:solutions/1/1/7/q1}, since q $\in \mathbb{F}$ we could say ,
\begin{align}
    \frac{1}{q} \in \mathbb{F} \quad{\text{and  }} q \not= 0\label{eq:solutions/1/1/7/3}
\end{align}
{Closed under multiplication:}
Also, $\mathbb{F}$ is closed under multiplication and thus,
from equation \eqref{eq:solutions/1/1/7/p} and \eqref{eq:solutions/1/1/7/3} we get , 
\begin{align}
    p\cdot\frac{1}{q} \in \mathbb{F}\\
\implies \frac{p}{q} \in \mathbb{F}\label{eq:solutions/1/1/7/proof}
\end{align}
where , $p \in \mathbb{Z}$ and $q \in \mathbb{Z}_{\not=0}$ (from equation \eqref{eq:solutions/1/1/7/0} and \eqref{eq:solutions/1/1/7/3})
{Conclusion}
From \eqref{eq:solutions/1/1/7/q} and \eqref{eq:solutions/1/1/7/proof} we could say , 
\begin{align}
    \mathbb{Q} \subseteq \mathbb{F}\label{eq:solutions/1/1/7/F}
\end{align}
From equation \eqref{eq:solutions/1/1/7/F} we could say that each subfield of the field of complex number contains every rational number
\begin{center}
    Hence Proved
\end{center}

\item Let $\vec{F}$ be a subfield of the field of complex numbers and let $\vec{V}$ be any vector space over $\vec{F}$. Suppose that f and g are linear functionals on $\vec{V}$ such that the function $h$ defined by $h(\alpha) =f(\alpha) g(\alpha)$ is also a linear functional on $\vec{V}$. Prove that either $f=0$ or $g=0$.
%
\\
\solution

{Complex Numbers:}
A complex number is a number that can be expressed in the form $a + bi$, where a and b are real numbers, and i represents the imaginary unit, satisfying the equation $i^2 =-1$.The set of complex numbers is denoted by $\mathbb{C}$
\begin{align}
    \mathbb{C}=\{(a,b):a,b \in \mathbb{R}\}
\end{align}
{Rational Numbers:}
A number in the form $\frac{p}{q}$, where both p and q(non-zero) are integers, is called a rational number.The set of rational numbers is dentoed by $\mathbb{Q}$
Let $\mathbb{Q}$ be the set of rational numbers.
\begin{align}
    \mathbb{Q}=\left\{\frac{p}{q}:p \in \mathbb{Z},q \in \mathbb{Z}_{\not=0} \right\}\label{eq:solutions/1/1/7/q}
\end{align}
Let $\mathbb{C}$ be the field of complex numbers and given $\mathbb{F}$ be the subfield of field of complex numbers $\mathbb{C}$ 
Since $\mathbb{F}$ is the subfield , we could say that 
\begin{align}
    0 &\in \mathbb{F} \label{eq:solutions/1/1/7/0}\\
    1 &\in \mathbb{F}
\end{align}
{Closed under addition:}
Here $\mathbb{F}$ is closed under addition since it is subfield
\begin{align}
    1+1=2&\in \mathbb{F}\\
    1+1+1=3&\in \mathbb{F}\\
    \vdots\notag\\
    1+1+\dots+1\text{(p times)}= p &\in \mathbb{F}\label{eq:solutions/1/1/7/p}\\
    1+1+\dots+1\text{(q times)}= q &\in \mathbb{F}\label{eq:solutions/1/1/7/q1}
\end{align}
By using the above property we could say that zero and other positive integers belongs to $\mathbb{F}$.Since $p$ and $q$ are integers we say,
\begin{align}
    p \in \mathbb{Z}\\
    q \in \mathbb{Z}\label{eq:solutions/1/1/7/0}
\end{align}
{Additive Inverse:}
Let $x$ be the positive integer belong $\mathbb{F}$ and by additive inverse we could say, 
\begin{align}
    \forall x &\in \mathbb{F}\label{eq:solutions/1/1/7/1}\\
    (-x) &\in \mathbb{F} \label{eq:solutions/1/1/7/2}
\end{align}
Therefore field $\mathbb{F}$ contains every integers. Let n be a integer then,
\begin{align}
    n \in \mathbb{Z} &\implies n \in \mathbb{F}\\
    \mathbb{Z} &\subseteq \mathbb{F}
\end{align}
Where $\mathbb{Z}$ is subset of $\mathbb{F}$
{Multiplicative Inverse:}
Every element except zero in the subfield $\mathbb{F}$ has an multiplicative inverse. From equation \eqref{eq:solutions/1/1/7/q1}, since q $\in \mathbb{F}$ we could say ,
\begin{align}
    \frac{1}{q} \in \mathbb{F} \quad{\text{and  }} q \not= 0\label{eq:solutions/1/1/7/3}
\end{align}
{Closed under multiplication:}
Also, $\mathbb{F}$ is closed under multiplication and thus,
from equation \eqref{eq:solutions/1/1/7/p} and \eqref{eq:solutions/1/1/7/3} we get , 
\begin{align}
    p\cdot\frac{1}{q} \in \mathbb{F}\\
\implies \frac{p}{q} \in \mathbb{F}\label{eq:solutions/1/1/7/proof}
\end{align}
where , $p \in \mathbb{Z}$ and $q \in \mathbb{Z}_{\not=0}$ (from equation \eqref{eq:solutions/1/1/7/0} and \eqref{eq:solutions/1/1/7/3})
{Conclusion}
From \eqref{eq:solutions/1/1/7/q} and \eqref{eq:solutions/1/1/7/proof} we could say , 
\begin{align}
    \mathbb{Q} \subseteq \mathbb{F}\label{eq:solutions/1/1/7/F}
\end{align}
From equation \eqref{eq:solutions/1/1/7/F} we could say that each subfield of the field of complex number contains every rational number
\begin{center}
    Hence Proved
\end{center}

\item Let $\mathbb{F}$ be a field of characteristic zero and let $\vec{V}$ be a finite dimensional vector space over  $\mathbb{F}$.If $\alpha_1,\alpha_2,\hdots,\alpha_m$ are finitely many vectors in $\vec{V}$ , each different from the zero vector, prove that there is a linear functional $f$ on $\vec{V}$ such that
\begin{align}
    f(\alpha_i) \neq 0, i=1,2,\hdots,m
\end{align}
%
\\
\solution

{Complex Numbers:}
A complex number is a number that can be expressed in the form $a + bi$, where a and b are real numbers, and i represents the imaginary unit, satisfying the equation $i^2 =-1$.The set of complex numbers is denoted by $\mathbb{C}$
\begin{align}
    \mathbb{C}=\{(a,b):a,b \in \mathbb{R}\}
\end{align}
{Rational Numbers:}
A number in the form $\frac{p}{q}$, where both p and q(non-zero) are integers, is called a rational number.The set of rational numbers is dentoed by $\mathbb{Q}$
Let $\mathbb{Q}$ be the set of rational numbers.
\begin{align}
    \mathbb{Q}=\left\{\frac{p}{q}:p \in \mathbb{Z},q \in \mathbb{Z}_{\not=0} \right\}\label{eq:solutions/1/1/7/q}
\end{align}
Let $\mathbb{C}$ be the field of complex numbers and given $\mathbb{F}$ be the subfield of field of complex numbers $\mathbb{C}$ 
Since $\mathbb{F}$ is the subfield , we could say that 
\begin{align}
    0 &\in \mathbb{F} \label{eq:solutions/1/1/7/0}\\
    1 &\in \mathbb{F}
\end{align}
{Closed under addition:}
Here $\mathbb{F}$ is closed under addition since it is subfield
\begin{align}
    1+1=2&\in \mathbb{F}\\
    1+1+1=3&\in \mathbb{F}\\
    \vdots\notag\\
    1+1+\dots+1\text{(p times)}= p &\in \mathbb{F}\label{eq:solutions/1/1/7/p}\\
    1+1+\dots+1\text{(q times)}= q &\in \mathbb{F}\label{eq:solutions/1/1/7/q1}
\end{align}
By using the above property we could say that zero and other positive integers belongs to $\mathbb{F}$.Since $p$ and $q$ are integers we say,
\begin{align}
    p \in \mathbb{Z}\\
    q \in \mathbb{Z}\label{eq:solutions/1/1/7/0}
\end{align}
{Additive Inverse:}
Let $x$ be the positive integer belong $\mathbb{F}$ and by additive inverse we could say, 
\begin{align}
    \forall x &\in \mathbb{F}\label{eq:solutions/1/1/7/1}\\
    (-x) &\in \mathbb{F} \label{eq:solutions/1/1/7/2}
\end{align}
Therefore field $\mathbb{F}$ contains every integers. Let n be a integer then,
\begin{align}
    n \in \mathbb{Z} &\implies n \in \mathbb{F}\\
    \mathbb{Z} &\subseteq \mathbb{F}
\end{align}
Where $\mathbb{Z}$ is subset of $\mathbb{F}$
{Multiplicative Inverse:}
Every element except zero in the subfield $\mathbb{F}$ has an multiplicative inverse. From equation \eqref{eq:solutions/1/1/7/q1}, since q $\in \mathbb{F}$ we could say ,
\begin{align}
    \frac{1}{q} \in \mathbb{F} \quad{\text{and  }} q \not= 0\label{eq:solutions/1/1/7/3}
\end{align}
{Closed under multiplication:}
Also, $\mathbb{F}$ is closed under multiplication and thus,
from equation \eqref{eq:solutions/1/1/7/p} and \eqref{eq:solutions/1/1/7/3} we get , 
\begin{align}
    p\cdot\frac{1}{q} \in \mathbb{F}\\
\implies \frac{p}{q} \in \mathbb{F}\label{eq:solutions/1/1/7/proof}
\end{align}
where , $p \in \mathbb{Z}$ and $q \in \mathbb{Z}_{\not=0}$ (from equation \eqref{eq:solutions/1/1/7/0} and \eqref{eq:solutions/1/1/7/3})
{Conclusion}
From \eqref{eq:solutions/1/1/7/q} and \eqref{eq:solutions/1/1/7/proof} we could say , 
\begin{align}
    \mathbb{Q} \subseteq \mathbb{F}\label{eq:solutions/1/1/7/F}
\end{align}
From equation \eqref{eq:solutions/1/1/7/F} we could say that each subfield of the field of complex number contains every rational number
\begin{center}
    Hence Proved
\end{center}

\item Similar matrices have the same trace. Thus we can define the trace of a linear operator on a finite-dimensional space to the trace of any matrix which represents the operator in a ordered basis. This is well-defined since all such representing matrices for one operator are similar. 

Now let V be the space of all 2$\times$2 matrices over the field F and let P be a fixed 2$\times$2 matrix. Let T be the linear operator on V defined by $T(A)=PA$. Prove that $tr(T)=2tr(P)$. 
%
\\
\solution

{Complex Numbers:}
A complex number is a number that can be expressed in the form $a + bi$, where a and b are real numbers, and i represents the imaginary unit, satisfying the equation $i^2 =-1$.The set of complex numbers is denoted by $\mathbb{C}$
\begin{align}
    \mathbb{C}=\{(a,b):a,b \in \mathbb{R}\}
\end{align}
{Rational Numbers:}
A number in the form $\frac{p}{q}$, where both p and q(non-zero) are integers, is called a rational number.The set of rational numbers is dentoed by $\mathbb{Q}$
Let $\mathbb{Q}$ be the set of rational numbers.
\begin{align}
    \mathbb{Q}=\left\{\frac{p}{q}:p \in \mathbb{Z},q \in \mathbb{Z}_{\not=0} \right\}\label{eq:solutions/1/1/7/q}
\end{align}
Let $\mathbb{C}$ be the field of complex numbers and given $\mathbb{F}$ be the subfield of field of complex numbers $\mathbb{C}$ 
Since $\mathbb{F}$ is the subfield , we could say that 
\begin{align}
    0 &\in \mathbb{F} \label{eq:solutions/1/1/7/0}\\
    1 &\in \mathbb{F}
\end{align}
{Closed under addition:}
Here $\mathbb{F}$ is closed under addition since it is subfield
\begin{align}
    1+1=2&\in \mathbb{F}\\
    1+1+1=3&\in \mathbb{F}\\
    \vdots\notag\\
    1+1+\dots+1\text{(p times)}= p &\in \mathbb{F}\label{eq:solutions/1/1/7/p}\\
    1+1+\dots+1\text{(q times)}= q &\in \mathbb{F}\label{eq:solutions/1/1/7/q1}
\end{align}
By using the above property we could say that zero and other positive integers belongs to $\mathbb{F}$.Since $p$ and $q$ are integers we say,
\begin{align}
    p \in \mathbb{Z}\\
    q \in \mathbb{Z}\label{eq:solutions/1/1/7/0}
\end{align}
{Additive Inverse:}
Let $x$ be the positive integer belong $\mathbb{F}$ and by additive inverse we could say, 
\begin{align}
    \forall x &\in \mathbb{F}\label{eq:solutions/1/1/7/1}\\
    (-x) &\in \mathbb{F} \label{eq:solutions/1/1/7/2}
\end{align}
Therefore field $\mathbb{F}$ contains every integers. Let n be a integer then,
\begin{align}
    n \in \mathbb{Z} &\implies n \in \mathbb{F}\\
    \mathbb{Z} &\subseteq \mathbb{F}
\end{align}
Where $\mathbb{Z}$ is subset of $\mathbb{F}$
{Multiplicative Inverse:}
Every element except zero in the subfield $\mathbb{F}$ has an multiplicative inverse. From equation \eqref{eq:solutions/1/1/7/q1}, since q $\in \mathbb{F}$ we could say ,
\begin{align}
    \frac{1}{q} \in \mathbb{F} \quad{\text{and  }} q \not= 0\label{eq:solutions/1/1/7/3}
\end{align}
{Closed under multiplication:}
Also, $\mathbb{F}$ is closed under multiplication and thus,
from equation \eqref{eq:solutions/1/1/7/p} and \eqref{eq:solutions/1/1/7/3} we get , 
\begin{align}
    p\cdot\frac{1}{q} \in \mathbb{F}\\
\implies \frac{p}{q} \in \mathbb{F}\label{eq:solutions/1/1/7/proof}
\end{align}
where , $p \in \mathbb{Z}$ and $q \in \mathbb{Z}_{\not=0}$ (from equation \eqref{eq:solutions/1/1/7/0} and \eqref{eq:solutions/1/1/7/3})
{Conclusion}
From \eqref{eq:solutions/1/1/7/q} and \eqref{eq:solutions/1/1/7/proof} we could say , 
\begin{align}
    \mathbb{Q} \subseteq \mathbb{F}\label{eq:solutions/1/1/7/F}
\end{align}
From equation \eqref{eq:solutions/1/1/7/F} we could say that each subfield of the field of complex number contains every rational number
\begin{center}
    Hence Proved
\end{center}

\item Show that the trace functional on $n\times n$ matrices is unique in the following sense. If $W$ is the space of $n \times n$ matrices over the field $F$ and if $f$ is a linear functional on $W$ such that $f(AB) = f(BA)$ for each $A$ and $B$ in $W$, then $f$ is a scalar multiple of the trace function. If, in addition, $f(I)=n$, then $f$ is the trace function.
\\
\solution

{Complex Numbers:}
A complex number is a number that can be expressed in the form $a + bi$, where a and b are real numbers, and i represents the imaginary unit, satisfying the equation $i^2 =-1$.The set of complex numbers is denoted by $\mathbb{C}$
\begin{align}
    \mathbb{C}=\{(a,b):a,b \in \mathbb{R}\}
\end{align}
{Rational Numbers:}
A number in the form $\frac{p}{q}$, where both p and q(non-zero) are integers, is called a rational number.The set of rational numbers is dentoed by $\mathbb{Q}$
Let $\mathbb{Q}$ be the set of rational numbers.
\begin{align}
    \mathbb{Q}=\left\{\frac{p}{q}:p \in \mathbb{Z},q \in \mathbb{Z}_{\not=0} \right\}\label{eq:solutions/1/1/7/q}
\end{align}
Let $\mathbb{C}$ be the field of complex numbers and given $\mathbb{F}$ be the subfield of field of complex numbers $\mathbb{C}$ 
Since $\mathbb{F}$ is the subfield , we could say that 
\begin{align}
    0 &\in \mathbb{F} \label{eq:solutions/1/1/7/0}\\
    1 &\in \mathbb{F}
\end{align}
{Closed under addition:}
Here $\mathbb{F}$ is closed under addition since it is subfield
\begin{align}
    1+1=2&\in \mathbb{F}\\
    1+1+1=3&\in \mathbb{F}\\
    \vdots\notag\\
    1+1+\dots+1\text{(p times)}= p &\in \mathbb{F}\label{eq:solutions/1/1/7/p}\\
    1+1+\dots+1\text{(q times)}= q &\in \mathbb{F}\label{eq:solutions/1/1/7/q1}
\end{align}
By using the above property we could say that zero and other positive integers belongs to $\mathbb{F}$.Since $p$ and $q$ are integers we say,
\begin{align}
    p \in \mathbb{Z}\\
    q \in \mathbb{Z}\label{eq:solutions/1/1/7/0}
\end{align}
{Additive Inverse:}
Let $x$ be the positive integer belong $\mathbb{F}$ and by additive inverse we could say, 
\begin{align}
    \forall x &\in \mathbb{F}\label{eq:solutions/1/1/7/1}\\
    (-x) &\in \mathbb{F} \label{eq:solutions/1/1/7/2}
\end{align}
Therefore field $\mathbb{F}$ contains every integers. Let n be a integer then,
\begin{align}
    n \in \mathbb{Z} &\implies n \in \mathbb{F}\\
    \mathbb{Z} &\subseteq \mathbb{F}
\end{align}
Where $\mathbb{Z}$ is subset of $\mathbb{F}$
{Multiplicative Inverse:}
Every element except zero in the subfield $\mathbb{F}$ has an multiplicative inverse. From equation \eqref{eq:solutions/1/1/7/q1}, since q $\in \mathbb{F}$ we could say ,
\begin{align}
    \frac{1}{q} \in \mathbb{F} \quad{\text{and  }} q \not= 0\label{eq:solutions/1/1/7/3}
\end{align}
{Closed under multiplication:}
Also, $\mathbb{F}$ is closed under multiplication and thus,
from equation \eqref{eq:solutions/1/1/7/p} and \eqref{eq:solutions/1/1/7/3} we get , 
\begin{align}
    p\cdot\frac{1}{q} \in \mathbb{F}\\
\implies \frac{p}{q} \in \mathbb{F}\label{eq:solutions/1/1/7/proof}
\end{align}
where , $p \in \mathbb{Z}$ and $q \in \mathbb{Z}_{\not=0}$ (from equation \eqref{eq:solutions/1/1/7/0} and \eqref{eq:solutions/1/1/7/3})
{Conclusion}
From \eqref{eq:solutions/1/1/7/q} and \eqref{eq:solutions/1/1/7/proof} we could say , 
\begin{align}
    \mathbb{Q} \subseteq \mathbb{F}\label{eq:solutions/1/1/7/F}
\end{align}
From equation \eqref{eq:solutions/1/1/7/F} we could say that each subfield of the field of complex number contains every rational number
\begin{center}
    Hence Proved
\end{center}

\item Let $\vec{W}$ be the space of $n\times n$ matrices over the field $\vec{F}$, and let $\vec{W}_0$ be 
the subspace spanned by the matrices $C$ of the form $C=AB-BA$. Prove that $\vec{W}_0$ is exactly the
subspace of matrices which have trace zero.
%
\\
\solution

{Complex Numbers:}
A complex number is a number that can be expressed in the form $a + bi$, where a and b are real numbers, and i represents the imaginary unit, satisfying the equation $i^2 =-1$.The set of complex numbers is denoted by $\mathbb{C}$
\begin{align}
    \mathbb{C}=\{(a,b):a,b \in \mathbb{R}\}
\end{align}
{Rational Numbers:}
A number in the form $\frac{p}{q}$, where both p and q(non-zero) are integers, is called a rational number.The set of rational numbers is dentoed by $\mathbb{Q}$
Let $\mathbb{Q}$ be the set of rational numbers.
\begin{align}
    \mathbb{Q}=\left\{\frac{p}{q}:p \in \mathbb{Z},q \in \mathbb{Z}_{\not=0} \right\}\label{eq:solutions/1/1/7/q}
\end{align}
Let $\mathbb{C}$ be the field of complex numbers and given $\mathbb{F}$ be the subfield of field of complex numbers $\mathbb{C}$ 
Since $\mathbb{F}$ is the subfield , we could say that 
\begin{align}
    0 &\in \mathbb{F} \label{eq:solutions/1/1/7/0}\\
    1 &\in \mathbb{F}
\end{align}
{Closed under addition:}
Here $\mathbb{F}$ is closed under addition since it is subfield
\begin{align}
    1+1=2&\in \mathbb{F}\\
    1+1+1=3&\in \mathbb{F}\\
    \vdots\notag\\
    1+1+\dots+1\text{(p times)}= p &\in \mathbb{F}\label{eq:solutions/1/1/7/p}\\
    1+1+\dots+1\text{(q times)}= q &\in \mathbb{F}\label{eq:solutions/1/1/7/q1}
\end{align}
By using the above property we could say that zero and other positive integers belongs to $\mathbb{F}$.Since $p$ and $q$ are integers we say,
\begin{align}
    p \in \mathbb{Z}\\
    q \in \mathbb{Z}\label{eq:solutions/1/1/7/0}
\end{align}
{Additive Inverse:}
Let $x$ be the positive integer belong $\mathbb{F}$ and by additive inverse we could say, 
\begin{align}
    \forall x &\in \mathbb{F}\label{eq:solutions/1/1/7/1}\\
    (-x) &\in \mathbb{F} \label{eq:solutions/1/1/7/2}
\end{align}
Therefore field $\mathbb{F}$ contains every integers. Let n be a integer then,
\begin{align}
    n \in \mathbb{Z} &\implies n \in \mathbb{F}\\
    \mathbb{Z} &\subseteq \mathbb{F}
\end{align}
Where $\mathbb{Z}$ is subset of $\mathbb{F}$
{Multiplicative Inverse:}
Every element except zero in the subfield $\mathbb{F}$ has an multiplicative inverse. From equation \eqref{eq:solutions/1/1/7/q1}, since q $\in \mathbb{F}$ we could say ,
\begin{align}
    \frac{1}{q} \in \mathbb{F} \quad{\text{and  }} q \not= 0\label{eq:solutions/1/1/7/3}
\end{align}
{Closed under multiplication:}
Also, $\mathbb{F}$ is closed under multiplication and thus,
from equation \eqref{eq:solutions/1/1/7/p} and \eqref{eq:solutions/1/1/7/3} we get , 
\begin{align}
    p\cdot\frac{1}{q} \in \mathbb{F}\\
\implies \frac{p}{q} \in \mathbb{F}\label{eq:solutions/1/1/7/proof}
\end{align}
where , $p \in \mathbb{Z}$ and $q \in \mathbb{Z}_{\not=0}$ (from equation \eqref{eq:solutions/1/1/7/0} and \eqref{eq:solutions/1/1/7/3})
{Conclusion}
From \eqref{eq:solutions/1/1/7/q} and \eqref{eq:solutions/1/1/7/proof} we could say , 
\begin{align}
    \mathbb{Q} \subseteq \mathbb{F}\label{eq:solutions/1/1/7/F}
\end{align}
From equation \eqref{eq:solutions/1/1/7/F} we could say that each subfield of the field of complex number contains every rational number
\begin{center}
    Hence Proved
\end{center}

\end{enumerate}
