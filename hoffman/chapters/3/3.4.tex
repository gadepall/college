\renewcommand{\theequation}{\theenumi}
\renewcommand{\thefigure}{\theenumi}
\begin{enumerate}[label=\thesubsection.\arabic*.,ref=\thesubsection.\theenumi]
\numberwithin{equation}{enumi}
\numberwithin{figure}{enumi}
%
\item Let T be the linear operator on $\mathbb{C}^{2}$ defined by $T\brak{x_1 , x_2}=\myvec{x_1 , 0}$.
Let $\beta$ be the standard ordered basis for $\mathbb{C}^{2}$ and $\beta^{'}=\{\alpha_1 , \alpha_2\}$
be the ordered basis defined by $\alpha_1 = \myvec{1 , i} , \alpha_2 = \myvec{-i , 2}$.
\begin{enumerate}
\item What is the matrix of $\vec{T}$ relative to the pair $\vec{\beta}$,$\vec{\beta^{'}}$?
%
\\
\solution
See Table \ref{eq:solutions/6/2/11/table:1}
\onecolumn
\begin{longtable}{|l|l|}
\hline
\multirow{3}{*}{} & \\
Statement&Solution\\
\hline
&\parbox{6cm}{\begin{align}
\mbox{Let }\vec{N}&=\myvec{a&b\\c&d}\\
\mbox{Since }\vec{N}^2&=0
\end{align}}\\
&If $\myvec{a\\c},\myvec{b\\d}$ are linearly independent then $\vec{N}$ is diagonalizable to $\myvec{0&0\\0&0}$.\\
&\parbox{6cm}{\begin{align}
\mbox{If }\vec{P}\vec{N}\vec{P}^{-1}=0\\
\mbox{then }\vec{N}=\vec{P}^{-1}\vec{0}\vec{P}=0
\end{align}}\\
Proof that&So in this case $\vec{N}$ itself is the zero matrix.\\
$\vec{N}=0$&This contradicts the assumption that $\myvec{a\\c},\myvec{b\\d}$ are linearly independent.\\
&$\therefore$ we can assume that $\myvec{a\\c},\myvec{b\\d}$ are linearly dependent if both are\\
&equal to the zero vector\\
&\parbox{6cm}
{\begin{align}
   \mbox{then } \vec{N} &= 0.
\end{align}}\\
%\hline
%\pagebreak
\hline
&\\
&Therefore we can assume at least one vector is non-zero.\\
Assuming $\myvec{b\\d}$ as&Therefore
$\vec{N}=\myvec{a&0\\c&0}$\\
the zero vector&\\
&\parbox{6cm}{\begin{align}
\mbox{So }\vec{N}^2&=0\\
\implies a^2&=0\\
\therefore a&=0\\
\mbox{Thus }\vec{N}&=\myvec{a&0\\c&0}
\end{align}}\\
&In this case $\vec{N}$ is similar to $\vec{N}=\myvec{0&0\\1&0}$ via the matrix $\vec{P}=\myvec{c&0\\0&1}$\\
&\\
\hline
&\\
Assuming $\myvec{a\\c}$ as&Therefore
$\vec{N}=\myvec{0&b\\0&d}$\\
the zero vector&\\
&\parbox{6cm}{\begin{align}
\mbox{Then }\vec{N}^2&=0\\
\implies d^2&=0\\
\therefore d&=0\\
\mbox{Thus }\vec{N}&=\myvec{0&b\\0&0}
\end{align}}\\
&In this case $\vec{N}$ is similar to $\vec{N}=\myvec{0&0\\b&0}$ via the matrix $\vec{P}=\myvec{0&1\\1&0}$,\\
&which is similar to $\myvec{0&0\\1&0}$ as above.\\
&\\
\hline
&\\
Hence& we can assume neither $\myvec{a\\c}$ or $\myvec{b\\d}$ is the zero vector.\\
&\\
%\hline
%\pagebreak
\hline
&\\
Consequences of &Since they are linearly dependent we can assume,\\
linear&\\
independence&\\
&\parbox{6cm}{\begin{align}
\myvec{b\\d}&=x\myvec{a\\c}\\
\therefore \vec{N}&=\myvec{a&ax\\c&cx}\\
\therefore \vec{N}^2&=0\\
\implies a(a+cx)&=0\\
c(a+cx)&=0\\
ax(a+cx)&=0\\
cx(a+cx)&=0
\end{align}}\\
\hline
&\\
Proof that $\vec{N}$ is&We know that at least one of a or c is not zero.\\
similar over $\mathbb{C}$ to&If a = 0 then c $\neq$ 0, it must be that x = 0.\\
$\myvec{0&0\\1&0}$&So in this case $\vec{N}=\myvec{0&0\\c&0}$ which is similar to $\myvec{0&0\\1&0}$ as before.\\
&\\
&\parbox{6cm}{\begin{align}
\mbox{If } a &\neq0\\
\mbox{then }x &\neq0\\
\mbox{else }a(a+cx)&=0\\
\implies a&=0\\
\mbox{Thus }a+cx&=0\\
\mbox{Hence }\vec{N}&=\myvec{a&ax\\ \frac{-a}{x}&-a}
\end{align}}\\
 &This is similar to $\myvec{a&a\\-a&-a}$ via $\vec{P}=\myvec{\sqrt{x}&0\\0&\frac{1}{\sqrt{x}}}$.\\
 &And $\myvec{a&a\\-a&-a}$ is similar to $\myvec{0&0\\-a&0}$ via $\vec{P}=\myvec{-1&-1\\1&0}$\\
 &And this finally is similar to $\myvec{0&0\\1&0}$ as before.\\
 &\\
\hline
&\\
Conclusion &Thus either $\vec{N}$ = 0 or $\vec{N}$ is similar over $\mathbb{C}$ to $\myvec{0&0\\1&0}$.\\
&\\
\hline
\caption{Solution summary}
\label{eq:solutions/6/2/11/table:1}
\end{longtable}

\item What is the matrix of T relative to the pair $\beta$ , $\beta^{'}$?
%
\\
\solution
See Table \ref{eq:solutions/6/2/11/table:1}
\onecolumn
\begin{longtable}{|l|l|}
\hline
\multirow{3}{*}{} & \\
Statement&Solution\\
\hline
&\parbox{6cm}{\begin{align}
\mbox{Let }\vec{N}&=\myvec{a&b\\c&d}\\
\mbox{Since }\vec{N}^2&=0
\end{align}}\\
&If $\myvec{a\\c},\myvec{b\\d}$ are linearly independent then $\vec{N}$ is diagonalizable to $\myvec{0&0\\0&0}$.\\
&\parbox{6cm}{\begin{align}
\mbox{If }\vec{P}\vec{N}\vec{P}^{-1}=0\\
\mbox{then }\vec{N}=\vec{P}^{-1}\vec{0}\vec{P}=0
\end{align}}\\
Proof that&So in this case $\vec{N}$ itself is the zero matrix.\\
$\vec{N}=0$&This contradicts the assumption that $\myvec{a\\c},\myvec{b\\d}$ are linearly independent.\\
&$\therefore$ we can assume that $\myvec{a\\c},\myvec{b\\d}$ are linearly dependent if both are\\
&equal to the zero vector\\
&\parbox{6cm}
{\begin{align}
   \mbox{then } \vec{N} &= 0.
\end{align}}\\
%\hline
%\pagebreak
\hline
&\\
&Therefore we can assume at least one vector is non-zero.\\
Assuming $\myvec{b\\d}$ as&Therefore
$\vec{N}=\myvec{a&0\\c&0}$\\
the zero vector&\\
&\parbox{6cm}{\begin{align}
\mbox{So }\vec{N}^2&=0\\
\implies a^2&=0\\
\therefore a&=0\\
\mbox{Thus }\vec{N}&=\myvec{a&0\\c&0}
\end{align}}\\
&In this case $\vec{N}$ is similar to $\vec{N}=\myvec{0&0\\1&0}$ via the matrix $\vec{P}=\myvec{c&0\\0&1}$\\
&\\
\hline
&\\
Assuming $\myvec{a\\c}$ as&Therefore
$\vec{N}=\myvec{0&b\\0&d}$\\
the zero vector&\\
&\parbox{6cm}{\begin{align}
\mbox{Then }\vec{N}^2&=0\\
\implies d^2&=0\\
\therefore d&=0\\
\mbox{Thus }\vec{N}&=\myvec{0&b\\0&0}
\end{align}}\\
&In this case $\vec{N}$ is similar to $\vec{N}=\myvec{0&0\\b&0}$ via the matrix $\vec{P}=\myvec{0&1\\1&0}$,\\
&which is similar to $\myvec{0&0\\1&0}$ as above.\\
&\\
\hline
&\\
Hence& we can assume neither $\myvec{a\\c}$ or $\myvec{b\\d}$ is the zero vector.\\
&\\
%\hline
%\pagebreak
\hline
&\\
Consequences of &Since they are linearly dependent we can assume,\\
linear&\\
independence&\\
&\parbox{6cm}{\begin{align}
\myvec{b\\d}&=x\myvec{a\\c}\\
\therefore \vec{N}&=\myvec{a&ax\\c&cx}\\
\therefore \vec{N}^2&=0\\
\implies a(a+cx)&=0\\
c(a+cx)&=0\\
ax(a+cx)&=0\\
cx(a+cx)&=0
\end{align}}\\
\hline
&\\
Proof that $\vec{N}$ is&We know that at least one of a or c is not zero.\\
similar over $\mathbb{C}$ to&If a = 0 then c $\neq$ 0, it must be that x = 0.\\
$\myvec{0&0\\1&0}$&So in this case $\vec{N}=\myvec{0&0\\c&0}$ which is similar to $\myvec{0&0\\1&0}$ as before.\\
&\\
&\parbox{6cm}{\begin{align}
\mbox{If } a &\neq0\\
\mbox{then }x &\neq0\\
\mbox{else }a(a+cx)&=0\\
\implies a&=0\\
\mbox{Thus }a+cx&=0\\
\mbox{Hence }\vec{N}&=\myvec{a&ax\\ \frac{-a}{x}&-a}
\end{align}}\\
 &This is similar to $\myvec{a&a\\-a&-a}$ via $\vec{P}=\myvec{\sqrt{x}&0\\0&\frac{1}{\sqrt{x}}}$.\\
 &And $\myvec{a&a\\-a&-a}$ is similar to $\myvec{0&0\\-a&0}$ via $\vec{P}=\myvec{-1&-1\\1&0}$\\
 &And this finally is similar to $\myvec{0&0\\1&0}$ as before.\\
 &\\
\hline
&\\
Conclusion &Thus either $\vec{N}$ = 0 or $\vec{N}$ is similar over $\mathbb{C}$ to $\myvec{0&0\\1&0}$.\\
&\\
\hline
\caption{Solution summary}
\label{eq:solutions/6/2/11/table:1}
\end{longtable}

\item What is the matrix of \textbf{T} in the ordered basis $\beta'$
%
\\
\solution
See Table \ref{eq:solutions/6/2/11/table:1}
\onecolumn
\begin{longtable}{|l|l|}
\hline
\multirow{3}{*}{} & \\
Statement&Solution\\
\hline
&\parbox{6cm}{\begin{align}
\mbox{Let }\vec{N}&=\myvec{a&b\\c&d}\\
\mbox{Since }\vec{N}^2&=0
\end{align}}\\
&If $\myvec{a\\c},\myvec{b\\d}$ are linearly independent then $\vec{N}$ is diagonalizable to $\myvec{0&0\\0&0}$.\\
&\parbox{6cm}{\begin{align}
\mbox{If }\vec{P}\vec{N}\vec{P}^{-1}=0\\
\mbox{then }\vec{N}=\vec{P}^{-1}\vec{0}\vec{P}=0
\end{align}}\\
Proof that&So in this case $\vec{N}$ itself is the zero matrix.\\
$\vec{N}=0$&This contradicts the assumption that $\myvec{a\\c},\myvec{b\\d}$ are linearly independent.\\
&$\therefore$ we can assume that $\myvec{a\\c},\myvec{b\\d}$ are linearly dependent if both are\\
&equal to the zero vector\\
&\parbox{6cm}
{\begin{align}
   \mbox{then } \vec{N} &= 0.
\end{align}}\\
%\hline
%\pagebreak
\hline
&\\
&Therefore we can assume at least one vector is non-zero.\\
Assuming $\myvec{b\\d}$ as&Therefore
$\vec{N}=\myvec{a&0\\c&0}$\\
the zero vector&\\
&\parbox{6cm}{\begin{align}
\mbox{So }\vec{N}^2&=0\\
\implies a^2&=0\\
\therefore a&=0\\
\mbox{Thus }\vec{N}&=\myvec{a&0\\c&0}
\end{align}}\\
&In this case $\vec{N}$ is similar to $\vec{N}=\myvec{0&0\\1&0}$ via the matrix $\vec{P}=\myvec{c&0\\0&1}$\\
&\\
\hline
&\\
Assuming $\myvec{a\\c}$ as&Therefore
$\vec{N}=\myvec{0&b\\0&d}$\\
the zero vector&\\
&\parbox{6cm}{\begin{align}
\mbox{Then }\vec{N}^2&=0\\
\implies d^2&=0\\
\therefore d&=0\\
\mbox{Thus }\vec{N}&=\myvec{0&b\\0&0}
\end{align}}\\
&In this case $\vec{N}$ is similar to $\vec{N}=\myvec{0&0\\b&0}$ via the matrix $\vec{P}=\myvec{0&1\\1&0}$,\\
&which is similar to $\myvec{0&0\\1&0}$ as above.\\
&\\
\hline
&\\
Hence& we can assume neither $\myvec{a\\c}$ or $\myvec{b\\d}$ is the zero vector.\\
&\\
%\hline
%\pagebreak
\hline
&\\
Consequences of &Since they are linearly dependent we can assume,\\
linear&\\
independence&\\
&\parbox{6cm}{\begin{align}
\myvec{b\\d}&=x\myvec{a\\c}\\
\therefore \vec{N}&=\myvec{a&ax\\c&cx}\\
\therefore \vec{N}^2&=0\\
\implies a(a+cx)&=0\\
c(a+cx)&=0\\
ax(a+cx)&=0\\
cx(a+cx)&=0
\end{align}}\\
\hline
&\\
Proof that $\vec{N}$ is&We know that at least one of a or c is not zero.\\
similar over $\mathbb{C}$ to&If a = 0 then c $\neq$ 0, it must be that x = 0.\\
$\myvec{0&0\\1&0}$&So in this case $\vec{N}=\myvec{0&0\\c&0}$ which is similar to $\myvec{0&0\\1&0}$ as before.\\
&\\
&\parbox{6cm}{\begin{align}
\mbox{If } a &\neq0\\
\mbox{then }x &\neq0\\
\mbox{else }a(a+cx)&=0\\
\implies a&=0\\
\mbox{Thus }a+cx&=0\\
\mbox{Hence }\vec{N}&=\myvec{a&ax\\ \frac{-a}{x}&-a}
\end{align}}\\
 &This is similar to $\myvec{a&a\\-a&-a}$ via $\vec{P}=\myvec{\sqrt{x}&0\\0&\frac{1}{\sqrt{x}}}$.\\
 &And $\myvec{a&a\\-a&-a}$ is similar to $\myvec{0&0\\-a&0}$ via $\vec{P}=\myvec{-1&-1\\1&0}$\\
 &And this finally is similar to $\myvec{0&0\\1&0}$ as before.\\
 &\\
\hline
&\\
Conclusion &Thus either $\vec{N}$ = 0 or $\vec{N}$ is similar over $\mathbb{C}$ to $\myvec{0&0\\1&0}$.\\
&\\
\hline
\caption{Solution summary}
\label{eq:solutions/6/2/11/table:1}
\end{longtable}

\item What is the matrix of T in the ordered basis $\{\alpha_2,\alpha_1\}$?
%
%
\\
\solution
See Table \ref{eq:solutions/6/2/11/table:1}
\onecolumn
\begin{longtable}{|l|l|}
\hline
\multirow{3}{*}{} & \\
Statement&Solution\\
\hline
&\parbox{6cm}{\begin{align}
\mbox{Let }\vec{N}&=\myvec{a&b\\c&d}\\
\mbox{Since }\vec{N}^2&=0
\end{align}}\\
&If $\myvec{a\\c},\myvec{b\\d}$ are linearly independent then $\vec{N}$ is diagonalizable to $\myvec{0&0\\0&0}$.\\
&\parbox{6cm}{\begin{align}
\mbox{If }\vec{P}\vec{N}\vec{P}^{-1}=0\\
\mbox{then }\vec{N}=\vec{P}^{-1}\vec{0}\vec{P}=0
\end{align}}\\
Proof that&So in this case $\vec{N}$ itself is the zero matrix.\\
$\vec{N}=0$&This contradicts the assumption that $\myvec{a\\c},\myvec{b\\d}$ are linearly independent.\\
&$\therefore$ we can assume that $\myvec{a\\c},\myvec{b\\d}$ are linearly dependent if both are\\
&equal to the zero vector\\
&\parbox{6cm}
{\begin{align}
   \mbox{then } \vec{N} &= 0.
\end{align}}\\
%\hline
%\pagebreak
\hline
&\\
&Therefore we can assume at least one vector is non-zero.\\
Assuming $\myvec{b\\d}$ as&Therefore
$\vec{N}=\myvec{a&0\\c&0}$\\
the zero vector&\\
&\parbox{6cm}{\begin{align}
\mbox{So }\vec{N}^2&=0\\
\implies a^2&=0\\
\therefore a&=0\\
\mbox{Thus }\vec{N}&=\myvec{a&0\\c&0}
\end{align}}\\
&In this case $\vec{N}$ is similar to $\vec{N}=\myvec{0&0\\1&0}$ via the matrix $\vec{P}=\myvec{c&0\\0&1}$\\
&\\
\hline
&\\
Assuming $\myvec{a\\c}$ as&Therefore
$\vec{N}=\myvec{0&b\\0&d}$\\
the zero vector&\\
&\parbox{6cm}{\begin{align}
\mbox{Then }\vec{N}^2&=0\\
\implies d^2&=0\\
\therefore d&=0\\
\mbox{Thus }\vec{N}&=\myvec{0&b\\0&0}
\end{align}}\\
&In this case $\vec{N}$ is similar to $\vec{N}=\myvec{0&0\\b&0}$ via the matrix $\vec{P}=\myvec{0&1\\1&0}$,\\
&which is similar to $\myvec{0&0\\1&0}$ as above.\\
&\\
\hline
&\\
Hence& we can assume neither $\myvec{a\\c}$ or $\myvec{b\\d}$ is the zero vector.\\
&\\
%\hline
%\pagebreak
\hline
&\\
Consequences of &Since they are linearly dependent we can assume,\\
linear&\\
independence&\\
&\parbox{6cm}{\begin{align}
\myvec{b\\d}&=x\myvec{a\\c}\\
\therefore \vec{N}&=\myvec{a&ax\\c&cx}\\
\therefore \vec{N}^2&=0\\
\implies a(a+cx)&=0\\
c(a+cx)&=0\\
ax(a+cx)&=0\\
cx(a+cx)&=0
\end{align}}\\
\hline
&\\
Proof that $\vec{N}$ is&We know that at least one of a or c is not zero.\\
similar over $\mathbb{C}$ to&If a = 0 then c $\neq$ 0, it must be that x = 0.\\
$\myvec{0&0\\1&0}$&So in this case $\vec{N}=\myvec{0&0\\c&0}$ which is similar to $\myvec{0&0\\1&0}$ as before.\\
&\\
&\parbox{6cm}{\begin{align}
\mbox{If } a &\neq0\\
\mbox{then }x &\neq0\\
\mbox{else }a(a+cx)&=0\\
\implies a&=0\\
\mbox{Thus }a+cx&=0\\
\mbox{Hence }\vec{N}&=\myvec{a&ax\\ \frac{-a}{x}&-a}
\end{align}}\\
 &This is similar to $\myvec{a&a\\-a&-a}$ via $\vec{P}=\myvec{\sqrt{x}&0\\0&\frac{1}{\sqrt{x}}}$.\\
 &And $\myvec{a&a\\-a&-a}$ is similar to $\myvec{0&0\\-a&0}$ via $\vec{P}=\myvec{-1&-1\\1&0}$\\
 &And this finally is similar to $\myvec{0&0\\1&0}$ as before.\\
 &\\
\hline
&\\
Conclusion &Thus either $\vec{N}$ = 0 or $\vec{N}$ is similar over $\mathbb{C}$ to $\myvec{0&0\\1&0}$.\\
&\\
\hline
\caption{Solution summary}
\label{eq:solutions/6/2/11/table:1}
\end{longtable}

\end{enumerate}
%
\item    Let T be the linear transformation from $\mathbb{R}^3$ into $\mathbb{R}^2$ defined by,
   \begin{align}
   T\myvec{x_1\\x_2\\x_3}=\myvec{x_1+x_2\\2x_3-x_1}
   \end{align}
\begin{enumerate}

\item    If $\beta$ is the standard ordered basis for $\mathbb{R}^3$ and $\beta^{\prime}$ is the standard ordered basis for $\mathbb{R}^2$, what is the matrix of T relative to the pair $\beta$, $\beta^{'}$
\end{enumerate}

\item Let $V$ be a two-dimensional vector space over the field $F$ and let $B$ be an ordered basis for $V$. If $T$ is a linear operator on $V$ and 
\begin{align}
[T]_B = \myvec{a & b\\c & d}
\end{align}
Prove that 
\begin{align}
T^2 - (a+d)T + (ad-bc)I = 0
\end{align}
%
\\
\solution
See Table \ref{eq:solutions/6/2/11/table:1}
\onecolumn
\begin{longtable}{|l|l|}
\hline
\multirow{3}{*}{} & \\
Statement&Solution\\
\hline
&\parbox{6cm}{\begin{align}
\mbox{Let }\vec{N}&=\myvec{a&b\\c&d}\\
\mbox{Since }\vec{N}^2&=0
\end{align}}\\
&If $\myvec{a\\c},\myvec{b\\d}$ are linearly independent then $\vec{N}$ is diagonalizable to $\myvec{0&0\\0&0}$.\\
&\parbox{6cm}{\begin{align}
\mbox{If }\vec{P}\vec{N}\vec{P}^{-1}=0\\
\mbox{then }\vec{N}=\vec{P}^{-1}\vec{0}\vec{P}=0
\end{align}}\\
Proof that&So in this case $\vec{N}$ itself is the zero matrix.\\
$\vec{N}=0$&This contradicts the assumption that $\myvec{a\\c},\myvec{b\\d}$ are linearly independent.\\
&$\therefore$ we can assume that $\myvec{a\\c},\myvec{b\\d}$ are linearly dependent if both are\\
&equal to the zero vector\\
&\parbox{6cm}
{\begin{align}
   \mbox{then } \vec{N} &= 0.
\end{align}}\\
%\hline
%\pagebreak
\hline
&\\
&Therefore we can assume at least one vector is non-zero.\\
Assuming $\myvec{b\\d}$ as&Therefore
$\vec{N}=\myvec{a&0\\c&0}$\\
the zero vector&\\
&\parbox{6cm}{\begin{align}
\mbox{So }\vec{N}^2&=0\\
\implies a^2&=0\\
\therefore a&=0\\
\mbox{Thus }\vec{N}&=\myvec{a&0\\c&0}
\end{align}}\\
&In this case $\vec{N}$ is similar to $\vec{N}=\myvec{0&0\\1&0}$ via the matrix $\vec{P}=\myvec{c&0\\0&1}$\\
&\\
\hline
&\\
Assuming $\myvec{a\\c}$ as&Therefore
$\vec{N}=\myvec{0&b\\0&d}$\\
the zero vector&\\
&\parbox{6cm}{\begin{align}
\mbox{Then }\vec{N}^2&=0\\
\implies d^2&=0\\
\therefore d&=0\\
\mbox{Thus }\vec{N}&=\myvec{0&b\\0&0}
\end{align}}\\
&In this case $\vec{N}$ is similar to $\vec{N}=\myvec{0&0\\b&0}$ via the matrix $\vec{P}=\myvec{0&1\\1&0}$,\\
&which is similar to $\myvec{0&0\\1&0}$ as above.\\
&\\
\hline
&\\
Hence& we can assume neither $\myvec{a\\c}$ or $\myvec{b\\d}$ is the zero vector.\\
&\\
%\hline
%\pagebreak
\hline
&\\
Consequences of &Since they are linearly dependent we can assume,\\
linear&\\
independence&\\
&\parbox{6cm}{\begin{align}
\myvec{b\\d}&=x\myvec{a\\c}\\
\therefore \vec{N}&=\myvec{a&ax\\c&cx}\\
\therefore \vec{N}^2&=0\\
\implies a(a+cx)&=0\\
c(a+cx)&=0\\
ax(a+cx)&=0\\
cx(a+cx)&=0
\end{align}}\\
\hline
&\\
Proof that $\vec{N}$ is&We know that at least one of a or c is not zero.\\
similar over $\mathbb{C}$ to&If a = 0 then c $\neq$ 0, it must be that x = 0.\\
$\myvec{0&0\\1&0}$&So in this case $\vec{N}=\myvec{0&0\\c&0}$ which is similar to $\myvec{0&0\\1&0}$ as before.\\
&\\
&\parbox{6cm}{\begin{align}
\mbox{If } a &\neq0\\
\mbox{then }x &\neq0\\
\mbox{else }a(a+cx)&=0\\
\implies a&=0\\
\mbox{Thus }a+cx&=0\\
\mbox{Hence }\vec{N}&=\myvec{a&ax\\ \frac{-a}{x}&-a}
\end{align}}\\
 &This is similar to $\myvec{a&a\\-a&-a}$ via $\vec{P}=\myvec{\sqrt{x}&0\\0&\frac{1}{\sqrt{x}}}$.\\
 &And $\myvec{a&a\\-a&-a}$ is similar to $\myvec{0&0\\-a&0}$ via $\vec{P}=\myvec{-1&-1\\1&0}$\\
 &And this finally is similar to $\myvec{0&0\\1&0}$ as before.\\
 &\\
\hline
&\\
Conclusion &Thus either $\vec{N}$ = 0 or $\vec{N}$ is similar over $\mathbb{C}$ to $\myvec{0&0\\1&0}$.\\
&\\
\hline
\caption{Solution summary}
\label{eq:solutions/6/2/11/table:1}
\end{longtable}

\item Let $T$ be a linear operator on $\mathbb{R}^3$, the matrix of which in the standard ordered basis is,
\begin{align}
\vec{A} = \myvec{1&2&1\\0&1&1\\-1&3&4} 
\end{align}
Find a basis for the range of $T$ and a basis for the null-space of $T$.
%
\\
\solution
See Table \ref{eq:solutions/6/2/11/table:1}
\onecolumn
\begin{longtable}{|l|l|}
\hline
\multirow{3}{*}{} & \\
Statement&Solution\\
\hline
&\parbox{6cm}{\begin{align}
\mbox{Let }\vec{N}&=\myvec{a&b\\c&d}\\
\mbox{Since }\vec{N}^2&=0
\end{align}}\\
&If $\myvec{a\\c},\myvec{b\\d}$ are linearly independent then $\vec{N}$ is diagonalizable to $\myvec{0&0\\0&0}$.\\
&\parbox{6cm}{\begin{align}
\mbox{If }\vec{P}\vec{N}\vec{P}^{-1}=0\\
\mbox{then }\vec{N}=\vec{P}^{-1}\vec{0}\vec{P}=0
\end{align}}\\
Proof that&So in this case $\vec{N}$ itself is the zero matrix.\\
$\vec{N}=0$&This contradicts the assumption that $\myvec{a\\c},\myvec{b\\d}$ are linearly independent.\\
&$\therefore$ we can assume that $\myvec{a\\c},\myvec{b\\d}$ are linearly dependent if both are\\
&equal to the zero vector\\
&\parbox{6cm}
{\begin{align}
   \mbox{then } \vec{N} &= 0.
\end{align}}\\
%\hline
%\pagebreak
\hline
&\\
&Therefore we can assume at least one vector is non-zero.\\
Assuming $\myvec{b\\d}$ as&Therefore
$\vec{N}=\myvec{a&0\\c&0}$\\
the zero vector&\\
&\parbox{6cm}{\begin{align}
\mbox{So }\vec{N}^2&=0\\
\implies a^2&=0\\
\therefore a&=0\\
\mbox{Thus }\vec{N}&=\myvec{a&0\\c&0}
\end{align}}\\
&In this case $\vec{N}$ is similar to $\vec{N}=\myvec{0&0\\1&0}$ via the matrix $\vec{P}=\myvec{c&0\\0&1}$\\
&\\
\hline
&\\
Assuming $\myvec{a\\c}$ as&Therefore
$\vec{N}=\myvec{0&b\\0&d}$\\
the zero vector&\\
&\parbox{6cm}{\begin{align}
\mbox{Then }\vec{N}^2&=0\\
\implies d^2&=0\\
\therefore d&=0\\
\mbox{Thus }\vec{N}&=\myvec{0&b\\0&0}
\end{align}}\\
&In this case $\vec{N}$ is similar to $\vec{N}=\myvec{0&0\\b&0}$ via the matrix $\vec{P}=\myvec{0&1\\1&0}$,\\
&which is similar to $\myvec{0&0\\1&0}$ as above.\\
&\\
\hline
&\\
Hence& we can assume neither $\myvec{a\\c}$ or $\myvec{b\\d}$ is the zero vector.\\
&\\
%\hline
%\pagebreak
\hline
&\\
Consequences of &Since they are linearly dependent we can assume,\\
linear&\\
independence&\\
&\parbox{6cm}{\begin{align}
\myvec{b\\d}&=x\myvec{a\\c}\\
\therefore \vec{N}&=\myvec{a&ax\\c&cx}\\
\therefore \vec{N}^2&=0\\
\implies a(a+cx)&=0\\
c(a+cx)&=0\\
ax(a+cx)&=0\\
cx(a+cx)&=0
\end{align}}\\
\hline
&\\
Proof that $\vec{N}$ is&We know that at least one of a or c is not zero.\\
similar over $\mathbb{C}$ to&If a = 0 then c $\neq$ 0, it must be that x = 0.\\
$\myvec{0&0\\1&0}$&So in this case $\vec{N}=\myvec{0&0\\c&0}$ which is similar to $\myvec{0&0\\1&0}$ as before.\\
&\\
&\parbox{6cm}{\begin{align}
\mbox{If } a &\neq0\\
\mbox{then }x &\neq0\\
\mbox{else }a(a+cx)&=0\\
\implies a&=0\\
\mbox{Thus }a+cx&=0\\
\mbox{Hence }\vec{N}&=\myvec{a&ax\\ \frac{-a}{x}&-a}
\end{align}}\\
 &This is similar to $\myvec{a&a\\-a&-a}$ via $\vec{P}=\myvec{\sqrt{x}&0\\0&\frac{1}{\sqrt{x}}}$.\\
 &And $\myvec{a&a\\-a&-a}$ is similar to $\myvec{0&0\\-a&0}$ via $\vec{P}=\myvec{-1&-1\\1&0}$\\
 &And this finally is similar to $\myvec{0&0\\1&0}$ as before.\\
 &\\
\hline
&\\
Conclusion &Thus either $\vec{N}$ = 0 or $\vec{N}$ is similar over $\mathbb{C}$ to $\myvec{0&0\\1&0}$.\\
&\\
\hline
\caption{Solution summary}
\label{eq:solutions/6/2/11/table:1}
\end{longtable}

\item Let T be the linear operator on $\mathbb{R}^2$ defined by
\begin{align}
T\brak{x_1,x_2}=\brak{-x_2,x_1}\label{eq:solutions/3/4/6/c/Tx}
\end{align}
\begin{enumerate}
\item Prove that for every real number c, the operator $\brak{T-cI}$ is invertible.
%
\\
\solution
See Table \ref{eq:solutions/6/2/11/table:1}
\onecolumn
\begin{longtable}{|l|l|}
\hline
\multirow{3}{*}{} & \\
Statement&Solution\\
\hline
&\parbox{6cm}{\begin{align}
\mbox{Let }\vec{N}&=\myvec{a&b\\c&d}\\
\mbox{Since }\vec{N}^2&=0
\end{align}}\\
&If $\myvec{a\\c},\myvec{b\\d}$ are linearly independent then $\vec{N}$ is diagonalizable to $\myvec{0&0\\0&0}$.\\
&\parbox{6cm}{\begin{align}
\mbox{If }\vec{P}\vec{N}\vec{P}^{-1}=0\\
\mbox{then }\vec{N}=\vec{P}^{-1}\vec{0}\vec{P}=0
\end{align}}\\
Proof that&So in this case $\vec{N}$ itself is the zero matrix.\\
$\vec{N}=0$&This contradicts the assumption that $\myvec{a\\c},\myvec{b\\d}$ are linearly independent.\\
&$\therefore$ we can assume that $\myvec{a\\c},\myvec{b\\d}$ are linearly dependent if both are\\
&equal to the zero vector\\
&\parbox{6cm}
{\begin{align}
   \mbox{then } \vec{N} &= 0.
\end{align}}\\
%\hline
%\pagebreak
\hline
&\\
&Therefore we can assume at least one vector is non-zero.\\
Assuming $\myvec{b\\d}$ as&Therefore
$\vec{N}=\myvec{a&0\\c&0}$\\
the zero vector&\\
&\parbox{6cm}{\begin{align}
\mbox{So }\vec{N}^2&=0\\
\implies a^2&=0\\
\therefore a&=0\\
\mbox{Thus }\vec{N}&=\myvec{a&0\\c&0}
\end{align}}\\
&In this case $\vec{N}$ is similar to $\vec{N}=\myvec{0&0\\1&0}$ via the matrix $\vec{P}=\myvec{c&0\\0&1}$\\
&\\
\hline
&\\
Assuming $\myvec{a\\c}$ as&Therefore
$\vec{N}=\myvec{0&b\\0&d}$\\
the zero vector&\\
&\parbox{6cm}{\begin{align}
\mbox{Then }\vec{N}^2&=0\\
\implies d^2&=0\\
\therefore d&=0\\
\mbox{Thus }\vec{N}&=\myvec{0&b\\0&0}
\end{align}}\\
&In this case $\vec{N}$ is similar to $\vec{N}=\myvec{0&0\\b&0}$ via the matrix $\vec{P}=\myvec{0&1\\1&0}$,\\
&which is similar to $\myvec{0&0\\1&0}$ as above.\\
&\\
\hline
&\\
Hence& we can assume neither $\myvec{a\\c}$ or $\myvec{b\\d}$ is the zero vector.\\
&\\
%\hline
%\pagebreak
\hline
&\\
Consequences of &Since they are linearly dependent we can assume,\\
linear&\\
independence&\\
&\parbox{6cm}{\begin{align}
\myvec{b\\d}&=x\myvec{a\\c}\\
\therefore \vec{N}&=\myvec{a&ax\\c&cx}\\
\therefore \vec{N}^2&=0\\
\implies a(a+cx)&=0\\
c(a+cx)&=0\\
ax(a+cx)&=0\\
cx(a+cx)&=0
\end{align}}\\
\hline
&\\
Proof that $\vec{N}$ is&We know that at least one of a or c is not zero.\\
similar over $\mathbb{C}$ to&If a = 0 then c $\neq$ 0, it must be that x = 0.\\
$\myvec{0&0\\1&0}$&So in this case $\vec{N}=\myvec{0&0\\c&0}$ which is similar to $\myvec{0&0\\1&0}$ as before.\\
&\\
&\parbox{6cm}{\begin{align}
\mbox{If } a &\neq0\\
\mbox{then }x &\neq0\\
\mbox{else }a(a+cx)&=0\\
\implies a&=0\\
\mbox{Thus }a+cx&=0\\
\mbox{Hence }\vec{N}&=\myvec{a&ax\\ \frac{-a}{x}&-a}
\end{align}}\\
 &This is similar to $\myvec{a&a\\-a&-a}$ via $\vec{P}=\myvec{\sqrt{x}&0\\0&\frac{1}{\sqrt{x}}}$.\\
 &And $\myvec{a&a\\-a&-a}$ is similar to $\myvec{0&0\\-a&0}$ via $\vec{P}=\myvec{-1&-1\\1&0}$\\
 &And this finally is similar to $\myvec{0&0\\1&0}$ as before.\\
 &\\
\hline
&\\
Conclusion &Thus either $\vec{N}$ = 0 or $\vec{N}$ is similar over $\mathbb{C}$ to $\myvec{0&0\\1&0}$.\\
&\\
\hline
\caption{Solution summary}
\label{eq:solutions/6/2/11/table:1}
\end{longtable}

\end{enumerate}
\item Let $\mathbb{T}$ be the linear operator on $\mathbb{R}^3$ defined by 
\begin{align}
    \mathbb{T}(x_1,x_2,x_3)=\\(3x_1+x_3,-2x_1+x_2,-x_1+2x_2+4x_3)\label{eq:solutions/3/4/7/a/1}
\end{align}
What is the matrix of $\mathbb{T}$ in the standard ordered basis of $\mathbb{R}^3$?
%
\\
\solution
See Table \ref{eq:solutions/6/2/11/table:1}
\onecolumn
\begin{longtable}{|l|l|}
\hline
\multirow{3}{*}{} & \\
Statement&Solution\\
\hline
&\parbox{6cm}{\begin{align}
\mbox{Let }\vec{N}&=\myvec{a&b\\c&d}\\
\mbox{Since }\vec{N}^2&=0
\end{align}}\\
&If $\myvec{a\\c},\myvec{b\\d}$ are linearly independent then $\vec{N}$ is diagonalizable to $\myvec{0&0\\0&0}$.\\
&\parbox{6cm}{\begin{align}
\mbox{If }\vec{P}\vec{N}\vec{P}^{-1}=0\\
\mbox{then }\vec{N}=\vec{P}^{-1}\vec{0}\vec{P}=0
\end{align}}\\
Proof that&So in this case $\vec{N}$ itself is the zero matrix.\\
$\vec{N}=0$&This contradicts the assumption that $\myvec{a\\c},\myvec{b\\d}$ are linearly independent.\\
&$\therefore$ we can assume that $\myvec{a\\c},\myvec{b\\d}$ are linearly dependent if both are\\
&equal to the zero vector\\
&\parbox{6cm}
{\begin{align}
   \mbox{then } \vec{N} &= 0.
\end{align}}\\
%\hline
%\pagebreak
\hline
&\\
&Therefore we can assume at least one vector is non-zero.\\
Assuming $\myvec{b\\d}$ as&Therefore
$\vec{N}=\myvec{a&0\\c&0}$\\
the zero vector&\\
&\parbox{6cm}{\begin{align}
\mbox{So }\vec{N}^2&=0\\
\implies a^2&=0\\
\therefore a&=0\\
\mbox{Thus }\vec{N}&=\myvec{a&0\\c&0}
\end{align}}\\
&In this case $\vec{N}$ is similar to $\vec{N}=\myvec{0&0\\1&0}$ via the matrix $\vec{P}=\myvec{c&0\\0&1}$\\
&\\
\hline
&\\
Assuming $\myvec{a\\c}$ as&Therefore
$\vec{N}=\myvec{0&b\\0&d}$\\
the zero vector&\\
&\parbox{6cm}{\begin{align}
\mbox{Then }\vec{N}^2&=0\\
\implies d^2&=0\\
\therefore d&=0\\
\mbox{Thus }\vec{N}&=\myvec{0&b\\0&0}
\end{align}}\\
&In this case $\vec{N}$ is similar to $\vec{N}=\myvec{0&0\\b&0}$ via the matrix $\vec{P}=\myvec{0&1\\1&0}$,\\
&which is similar to $\myvec{0&0\\1&0}$ as above.\\
&\\
\hline
&\\
Hence& we can assume neither $\myvec{a\\c}$ or $\myvec{b\\d}$ is the zero vector.\\
&\\
%\hline
%\pagebreak
\hline
&\\
Consequences of &Since they are linearly dependent we can assume,\\
linear&\\
independence&\\
&\parbox{6cm}{\begin{align}
\myvec{b\\d}&=x\myvec{a\\c}\\
\therefore \vec{N}&=\myvec{a&ax\\c&cx}\\
\therefore \vec{N}^2&=0\\
\implies a(a+cx)&=0\\
c(a+cx)&=0\\
ax(a+cx)&=0\\
cx(a+cx)&=0
\end{align}}\\
\hline
&\\
Proof that $\vec{N}$ is&We know that at least one of a or c is not zero.\\
similar over $\mathbb{C}$ to&If a = 0 then c $\neq$ 0, it must be that x = 0.\\
$\myvec{0&0\\1&0}$&So in this case $\vec{N}=\myvec{0&0\\c&0}$ which is similar to $\myvec{0&0\\1&0}$ as before.\\
&\\
&\parbox{6cm}{\begin{align}
\mbox{If } a &\neq0\\
\mbox{then }x &\neq0\\
\mbox{else }a(a+cx)&=0\\
\implies a&=0\\
\mbox{Thus }a+cx&=0\\
\mbox{Hence }\vec{N}&=\myvec{a&ax\\ \frac{-a}{x}&-a}
\end{align}}\\
 &This is similar to $\myvec{a&a\\-a&-a}$ via $\vec{P}=\myvec{\sqrt{x}&0\\0&\frac{1}{\sqrt{x}}}$.\\
 &And $\myvec{a&a\\-a&-a}$ is similar to $\myvec{0&0\\-a&0}$ via $\vec{P}=\myvec{-1&-1\\1&0}$\\
 &And this finally is similar to $\myvec{0&0\\1&0}$ as before.\\
 &\\
\hline
&\\
Conclusion &Thus either $\vec{N}$ = 0 or $\vec{N}$ is similar over $\mathbb{C}$ to $\myvec{0&0\\1&0}$.\\
&\\
\hline
\caption{Solution summary}
\label{eq:solutions/6/2/11/table:1}
\end{longtable}

\item The linear operator $\vec{T}$ on $\vec{R}^2$ defined by 
\begin{align}
	\vec{T}\myvec{x_1\\x_2}=\myvec{x_1\\0}
\end{align}
is represented by the matrix
\begin{align}
	\vec{A}=\myvec{1&0\\
		       0&0}
\end{align}
\item Prove that if $\vec{S}$ is a linear operator on $\vec{R}^2$ such that $\vec{S}^2=\vec{S}$, 
then $\vec{S}=\vec{0}$, or $\vec{S}=\vec{I}$, or there is an ordered basis $\vec{B}$ for $\vec{R}^2$ 
such that $[\vec{S}]_B=\vec{A}$.
%
\\
\solution
See Table \ref{eq:solutions/6/2/11/table:1}
\onecolumn
\begin{longtable}{|l|l|}
\hline
\multirow{3}{*}{} & \\
Statement&Solution\\
\hline
&\parbox{6cm}{\begin{align}
\mbox{Let }\vec{N}&=\myvec{a&b\\c&d}\\
\mbox{Since }\vec{N}^2&=0
\end{align}}\\
&If $\myvec{a\\c},\myvec{b\\d}$ are linearly independent then $\vec{N}$ is diagonalizable to $\myvec{0&0\\0&0}$.\\
&\parbox{6cm}{\begin{align}
\mbox{If }\vec{P}\vec{N}\vec{P}^{-1}=0\\
\mbox{then }\vec{N}=\vec{P}^{-1}\vec{0}\vec{P}=0
\end{align}}\\
Proof that&So in this case $\vec{N}$ itself is the zero matrix.\\
$\vec{N}=0$&This contradicts the assumption that $\myvec{a\\c},\myvec{b\\d}$ are linearly independent.\\
&$\therefore$ we can assume that $\myvec{a\\c},\myvec{b\\d}$ are linearly dependent if both are\\
&equal to the zero vector\\
&\parbox{6cm}
{\begin{align}
   \mbox{then } \vec{N} &= 0.
\end{align}}\\
%\hline
%\pagebreak
\hline
&\\
&Therefore we can assume at least one vector is non-zero.\\
Assuming $\myvec{b\\d}$ as&Therefore
$\vec{N}=\myvec{a&0\\c&0}$\\
the zero vector&\\
&\parbox{6cm}{\begin{align}
\mbox{So }\vec{N}^2&=0\\
\implies a^2&=0\\
\therefore a&=0\\
\mbox{Thus }\vec{N}&=\myvec{a&0\\c&0}
\end{align}}\\
&In this case $\vec{N}$ is similar to $\vec{N}=\myvec{0&0\\1&0}$ via the matrix $\vec{P}=\myvec{c&0\\0&1}$\\
&\\
\hline
&\\
Assuming $\myvec{a\\c}$ as&Therefore
$\vec{N}=\myvec{0&b\\0&d}$\\
the zero vector&\\
&\parbox{6cm}{\begin{align}
\mbox{Then }\vec{N}^2&=0\\
\implies d^2&=0\\
\therefore d&=0\\
\mbox{Thus }\vec{N}&=\myvec{0&b\\0&0}
\end{align}}\\
&In this case $\vec{N}$ is similar to $\vec{N}=\myvec{0&0\\b&0}$ via the matrix $\vec{P}=\myvec{0&1\\1&0}$,\\
&which is similar to $\myvec{0&0\\1&0}$ as above.\\
&\\
\hline
&\\
Hence& we can assume neither $\myvec{a\\c}$ or $\myvec{b\\d}$ is the zero vector.\\
&\\
%\hline
%\pagebreak
\hline
&\\
Consequences of &Since they are linearly dependent we can assume,\\
linear&\\
independence&\\
&\parbox{6cm}{\begin{align}
\myvec{b\\d}&=x\myvec{a\\c}\\
\therefore \vec{N}&=\myvec{a&ax\\c&cx}\\
\therefore \vec{N}^2&=0\\
\implies a(a+cx)&=0\\
c(a+cx)&=0\\
ax(a+cx)&=0\\
cx(a+cx)&=0
\end{align}}\\
\hline
&\\
Proof that $\vec{N}$ is&We know that at least one of a or c is not zero.\\
similar over $\mathbb{C}$ to&If a = 0 then c $\neq$ 0, it must be that x = 0.\\
$\myvec{0&0\\1&0}$&So in this case $\vec{N}=\myvec{0&0\\c&0}$ which is similar to $\myvec{0&0\\1&0}$ as before.\\
&\\
&\parbox{6cm}{\begin{align}
\mbox{If } a &\neq0\\
\mbox{then }x &\neq0\\
\mbox{else }a(a+cx)&=0\\
\implies a&=0\\
\mbox{Thus }a+cx&=0\\
\mbox{Hence }\vec{N}&=\myvec{a&ax\\ \frac{-a}{x}&-a}
\end{align}}\\
 &This is similar to $\myvec{a&a\\-a&-a}$ via $\vec{P}=\myvec{\sqrt{x}&0\\0&\frac{1}{\sqrt{x}}}$.\\
 &And $\myvec{a&a\\-a&-a}$ is similar to $\myvec{0&0\\-a&0}$ via $\vec{P}=\myvec{-1&-1\\1&0}$\\
 &And this finally is similar to $\myvec{0&0\\1&0}$ as before.\\
 &\\
\hline
&\\
Conclusion &Thus either $\vec{N}$ = 0 or $\vec{N}$ is similar over $\mathbb{C}$ to $\myvec{0&0\\1&0}$.\\
&\\
\hline
\caption{Solution summary}
\label{eq:solutions/6/2/11/table:1}
\end{longtable}

\item Let V be an n-dimensional vector space over the field F, and let $\mathcal{B} = \cbrak{\alpha_1,\ldots, \alpha_n}$ be an ordered basis for V then there is a unique linear operator T on V such that 
\begin{align}
   &T\alpha_{j} = \alpha_{j+1},  j=1,\ldots,n-1 \\
   &T\alpha_n = 0. 
\end{align}
%
\begin{enumerate}
\item What is the matrix A of T in the ordered basis $\mathcal{B}$?
\\
\solution
See Table \ref{eq:solutions/6/2/11/table:1}
\onecolumn
\begin{longtable}{|l|l|}
\hline
\multirow{3}{*}{} & \\
Statement&Solution\\
\hline
&\parbox{6cm}{\begin{align}
\mbox{Let }\vec{N}&=\myvec{a&b\\c&d}\\
\mbox{Since }\vec{N}^2&=0
\end{align}}\\
&If $\myvec{a\\c},\myvec{b\\d}$ are linearly independent then $\vec{N}$ is diagonalizable to $\myvec{0&0\\0&0}$.\\
&\parbox{6cm}{\begin{align}
\mbox{If }\vec{P}\vec{N}\vec{P}^{-1}=0\\
\mbox{then }\vec{N}=\vec{P}^{-1}\vec{0}\vec{P}=0
\end{align}}\\
Proof that&So in this case $\vec{N}$ itself is the zero matrix.\\
$\vec{N}=0$&This contradicts the assumption that $\myvec{a\\c},\myvec{b\\d}$ are linearly independent.\\
&$\therefore$ we can assume that $\myvec{a\\c},\myvec{b\\d}$ are linearly dependent if both are\\
&equal to the zero vector\\
&\parbox{6cm}
{\begin{align}
   \mbox{then } \vec{N} &= 0.
\end{align}}\\
%\hline
%\pagebreak
\hline
&\\
&Therefore we can assume at least one vector is non-zero.\\
Assuming $\myvec{b\\d}$ as&Therefore
$\vec{N}=\myvec{a&0\\c&0}$\\
the zero vector&\\
&\parbox{6cm}{\begin{align}
\mbox{So }\vec{N}^2&=0\\
\implies a^2&=0\\
\therefore a&=0\\
\mbox{Thus }\vec{N}&=\myvec{a&0\\c&0}
\end{align}}\\
&In this case $\vec{N}$ is similar to $\vec{N}=\myvec{0&0\\1&0}$ via the matrix $\vec{P}=\myvec{c&0\\0&1}$\\
&\\
\hline
&\\
Assuming $\myvec{a\\c}$ as&Therefore
$\vec{N}=\myvec{0&b\\0&d}$\\
the zero vector&\\
&\parbox{6cm}{\begin{align}
\mbox{Then }\vec{N}^2&=0\\
\implies d^2&=0\\
\therefore d&=0\\
\mbox{Thus }\vec{N}&=\myvec{0&b\\0&0}
\end{align}}\\
&In this case $\vec{N}$ is similar to $\vec{N}=\myvec{0&0\\b&0}$ via the matrix $\vec{P}=\myvec{0&1\\1&0}$,\\
&which is similar to $\myvec{0&0\\1&0}$ as above.\\
&\\
\hline
&\\
Hence& we can assume neither $\myvec{a\\c}$ or $\myvec{b\\d}$ is the zero vector.\\
&\\
%\hline
%\pagebreak
\hline
&\\
Consequences of &Since they are linearly dependent we can assume,\\
linear&\\
independence&\\
&\parbox{6cm}{\begin{align}
\myvec{b\\d}&=x\myvec{a\\c}\\
\therefore \vec{N}&=\myvec{a&ax\\c&cx}\\
\therefore \vec{N}^2&=0\\
\implies a(a+cx)&=0\\
c(a+cx)&=0\\
ax(a+cx)&=0\\
cx(a+cx)&=0
\end{align}}\\
\hline
&\\
Proof that $\vec{N}$ is&We know that at least one of a or c is not zero.\\
similar over $\mathbb{C}$ to&If a = 0 then c $\neq$ 0, it must be that x = 0.\\
$\myvec{0&0\\1&0}$&So in this case $\vec{N}=\myvec{0&0\\c&0}$ which is similar to $\myvec{0&0\\1&0}$ as before.\\
&\\
&\parbox{6cm}{\begin{align}
\mbox{If } a &\neq0\\
\mbox{then }x &\neq0\\
\mbox{else }a(a+cx)&=0\\
\implies a&=0\\
\mbox{Thus }a+cx&=0\\
\mbox{Hence }\vec{N}&=\myvec{a&ax\\ \frac{-a}{x}&-a}
\end{align}}\\
 &This is similar to $\myvec{a&a\\-a&-a}$ via $\vec{P}=\myvec{\sqrt{x}&0\\0&\frac{1}{\sqrt{x}}}$.\\
 &And $\myvec{a&a\\-a&-a}$ is similar to $\myvec{0&0\\-a&0}$ via $\vec{P}=\myvec{-1&-1\\1&0}$\\
 &And this finally is similar to $\myvec{0&0\\1&0}$ as before.\\
 &\\
\hline
&\\
Conclusion &Thus either $\vec{N}$ = 0 or $\vec{N}$ is similar over $\mathbb{C}$ to $\myvec{0&0\\1&0}$.\\
&\\
\hline
\caption{Solution summary}
\label{eq:solutions/6/2/11/table:1}
\end{longtable}

\end{enumerate}
\item Let $V$ and $W$ be finite-dimensional vector spaces over the field $F$ and let $T$ be a linear transformation from $V$ into $W$. If
\begin{equation}
	\mathcal{B} = \{\alpha_1, \dots, \alpha_n\} \text{ and } \mathcal{B}' = \{\beta_1, \dots, \beta_m\}
\end{equation}
are ordered bases for $V$ and $W$, respectively, define the linear transformation $E^{p,q}$ as in the proof of Theorem 5: $E^{p,q}(\alpha_i) = \delta_{iq}\beta_p$. Then the $E^{p,q}$, $1 \leq p \leq m$, $1 \leq q \leq n$, form a basis for $L(V,W)$ and so
\begin{equation}
	T = \displaystyle \sum_{p=1}^m \displaystyle \sum_{q=1}^n A_{pq}E^{p,q}
\end{equation} 
for certain scalars $A_{pq}$ (the coordinates of $T$ in this basis for $L(V,W))$. Show that the matrix $A$ with entries $A(p,q) = A_{pq}$ is precisely the matrix of $T$ relative to the pair $\mathcal{B},\mathcal{B}'$.
%
\\
\solution
See Table \ref{eq:solutions/6/2/11/table:1}
\onecolumn
\begin{longtable}{|l|l|}
\hline
\multirow{3}{*}{} & \\
Statement&Solution\\
\hline
&\parbox{6cm}{\begin{align}
\mbox{Let }\vec{N}&=\myvec{a&b\\c&d}\\
\mbox{Since }\vec{N}^2&=0
\end{align}}\\
&If $\myvec{a\\c},\myvec{b\\d}$ are linearly independent then $\vec{N}$ is diagonalizable to $\myvec{0&0\\0&0}$.\\
&\parbox{6cm}{\begin{align}
\mbox{If }\vec{P}\vec{N}\vec{P}^{-1}=0\\
\mbox{then }\vec{N}=\vec{P}^{-1}\vec{0}\vec{P}=0
\end{align}}\\
Proof that&So in this case $\vec{N}$ itself is the zero matrix.\\
$\vec{N}=0$&This contradicts the assumption that $\myvec{a\\c},\myvec{b\\d}$ are linearly independent.\\
&$\therefore$ we can assume that $\myvec{a\\c},\myvec{b\\d}$ are linearly dependent if both are\\
&equal to the zero vector\\
&\parbox{6cm}
{\begin{align}
   \mbox{then } \vec{N} &= 0.
\end{align}}\\
%\hline
%\pagebreak
\hline
&\\
&Therefore we can assume at least one vector is non-zero.\\
Assuming $\myvec{b\\d}$ as&Therefore
$\vec{N}=\myvec{a&0\\c&0}$\\
the zero vector&\\
&\parbox{6cm}{\begin{align}
\mbox{So }\vec{N}^2&=0\\
\implies a^2&=0\\
\therefore a&=0\\
\mbox{Thus }\vec{N}&=\myvec{a&0\\c&0}
\end{align}}\\
&In this case $\vec{N}$ is similar to $\vec{N}=\myvec{0&0\\1&0}$ via the matrix $\vec{P}=\myvec{c&0\\0&1}$\\
&\\
\hline
&\\
Assuming $\myvec{a\\c}$ as&Therefore
$\vec{N}=\myvec{0&b\\0&d}$\\
the zero vector&\\
&\parbox{6cm}{\begin{align}
\mbox{Then }\vec{N}^2&=0\\
\implies d^2&=0\\
\therefore d&=0\\
\mbox{Thus }\vec{N}&=\myvec{0&b\\0&0}
\end{align}}\\
&In this case $\vec{N}$ is similar to $\vec{N}=\myvec{0&0\\b&0}$ via the matrix $\vec{P}=\myvec{0&1\\1&0}$,\\
&which is similar to $\myvec{0&0\\1&0}$ as above.\\
&\\
\hline
&\\
Hence& we can assume neither $\myvec{a\\c}$ or $\myvec{b\\d}$ is the zero vector.\\
&\\
%\hline
%\pagebreak
\hline
&\\
Consequences of &Since they are linearly dependent we can assume,\\
linear&\\
independence&\\
&\parbox{6cm}{\begin{align}
\myvec{b\\d}&=x\myvec{a\\c}\\
\therefore \vec{N}&=\myvec{a&ax\\c&cx}\\
\therefore \vec{N}^2&=0\\
\implies a(a+cx)&=0\\
c(a+cx)&=0\\
ax(a+cx)&=0\\
cx(a+cx)&=0
\end{align}}\\
\hline
&\\
Proof that $\vec{N}$ is&We know that at least one of a or c is not zero.\\
similar over $\mathbb{C}$ to&If a = 0 then c $\neq$ 0, it must be that x = 0.\\
$\myvec{0&0\\1&0}$&So in this case $\vec{N}=\myvec{0&0\\c&0}$ which is similar to $\myvec{0&0\\1&0}$ as before.\\
&\\
&\parbox{6cm}{\begin{align}
\mbox{If } a &\neq0\\
\mbox{then }x &\neq0\\
\mbox{else }a(a+cx)&=0\\
\implies a&=0\\
\mbox{Thus }a+cx&=0\\
\mbox{Hence }\vec{N}&=\myvec{a&ax\\ \frac{-a}{x}&-a}
\end{align}}\\
 &This is similar to $\myvec{a&a\\-a&-a}$ via $\vec{P}=\myvec{\sqrt{x}&0\\0&\frac{1}{\sqrt{x}}}$.\\
 &And $\myvec{a&a\\-a&-a}$ is similar to $\myvec{0&0\\-a&0}$ via $\vec{P}=\myvec{-1&-1\\1&0}$\\
 &And this finally is similar to $\myvec{0&0\\1&0}$ as before.\\
 &\\
\hline
&\\
Conclusion &Thus either $\vec{N}$ = 0 or $\vec{N}$ is similar over $\mathbb{C}$ to $\myvec{0&0\\1&0}$.\\
&\\
\hline
\caption{Solution summary}
\label{eq:solutions/6/2/11/table:1}
\end{longtable}


\end{enumerate}
