\renewcommand{\theequation}{\theenumi}
\renewcommand{\thefigure}{\theenumi}
\begin{enumerate}[label=\thesubsection.\arabic*.,ref=\thesubsection.\theenumi]
\numberwithin{equation}{enumi}
\numberwithin{figure}{enumi}

\item If A is a complex $5 \times 5$ matrix with characteristic polynomial $f = (x-2)^3 (x+7)^2$ and minimal polynomial $p = (x-2)^2 (x+7)$, what is the Jordan form for A? 
%
\\
\solution

{Complex Numbers:}
A complex number is a number that can be expressed in the form $a + bi$, where a and b are real numbers, and i represents the imaginary unit, satisfying the equation $i^2 =-1$.The set of complex numbers is denoted by $\mathbb{C}$
\begin{align}
    \mathbb{C}=\{(a,b):a,b \in \mathbb{R}\}
\end{align}
{Rational Numbers:}
A number in the form $\frac{p}{q}$, where both p and q(non-zero) are integers, is called a rational number.The set of rational numbers is dentoed by $\mathbb{Q}$
Let $\mathbb{Q}$ be the set of rational numbers.
\begin{align}
    \mathbb{Q}=\left\{\frac{p}{q}:p \in \mathbb{Z},q \in \mathbb{Z}_{\not=0} \right\}\label{eq:solutions/1/1/7/q}
\end{align}
Let $\mathbb{C}$ be the field of complex numbers and given $\mathbb{F}$ be the subfield of field of complex numbers $\mathbb{C}$ 
Since $\mathbb{F}$ is the subfield , we could say that 
\begin{align}
    0 &\in \mathbb{F} \label{eq:solutions/1/1/7/0}\\
    1 &\in \mathbb{F}
\end{align}
{Closed under addition:}
Here $\mathbb{F}$ is closed under addition since it is subfield
\begin{align}
    1+1=2&\in \mathbb{F}\\
    1+1+1=3&\in \mathbb{F}\\
    \vdots\notag\\
    1+1+\dots+1\text{(p times)}= p &\in \mathbb{F}\label{eq:solutions/1/1/7/p}\\
    1+1+\dots+1\text{(q times)}= q &\in \mathbb{F}\label{eq:solutions/1/1/7/q1}
\end{align}
By using the above property we could say that zero and other positive integers belongs to $\mathbb{F}$.Since $p$ and $q$ are integers we say,
\begin{align}
    p \in \mathbb{Z}\\
    q \in \mathbb{Z}\label{eq:solutions/1/1/7/0}
\end{align}
{Additive Inverse:}
Let $x$ be the positive integer belong $\mathbb{F}$ and by additive inverse we could say, 
\begin{align}
    \forall x &\in \mathbb{F}\label{eq:solutions/1/1/7/1}\\
    (-x) &\in \mathbb{F} \label{eq:solutions/1/1/7/2}
\end{align}
Therefore field $\mathbb{F}$ contains every integers. Let n be a integer then,
\begin{align}
    n \in \mathbb{Z} &\implies n \in \mathbb{F}\\
    \mathbb{Z} &\subseteq \mathbb{F}
\end{align}
Where $\mathbb{Z}$ is subset of $\mathbb{F}$
{Multiplicative Inverse:}
Every element except zero in the subfield $\mathbb{F}$ has an multiplicative inverse. From equation \eqref{eq:solutions/1/1/7/q1}, since q $\in \mathbb{F}$ we could say ,
\begin{align}
    \frac{1}{q} \in \mathbb{F} \quad{\text{and  }} q \not= 0\label{eq:solutions/1/1/7/3}
\end{align}
{Closed under multiplication:}
Also, $\mathbb{F}$ is closed under multiplication and thus,
from equation \eqref{eq:solutions/1/1/7/p} and \eqref{eq:solutions/1/1/7/3} we get , 
\begin{align}
    p\cdot\frac{1}{q} \in \mathbb{F}\\
\implies \frac{p}{q} \in \mathbb{F}\label{eq:solutions/1/1/7/proof}
\end{align}
where , $p \in \mathbb{Z}$ and $q \in \mathbb{Z}_{\not=0}$ (from equation \eqref{eq:solutions/1/1/7/0} and \eqref{eq:solutions/1/1/7/3})
{Conclusion}
From \eqref{eq:solutions/1/1/7/q} and \eqref{eq:solutions/1/1/7/proof} we could say , 
\begin{align}
    \mathbb{Q} \subseteq \mathbb{F}\label{eq:solutions/1/1/7/F}
\end{align}
From equation \eqref{eq:solutions/1/1/7/F} we could say that each subfield of the field of complex number contains every rational number
\begin{center}
    Hence Proved
\end{center}

\item How many possible Jordan forms are there for a $6\times6$ complex matrix with characteristic polynomial $\brak{x+2}^4\brak{x-1}^2$?
%
\\
\solution

{Complex Numbers:}
A complex number is a number that can be expressed in the form $a + bi$, where a and b are real numbers, and i represents the imaginary unit, satisfying the equation $i^2 =-1$.The set of complex numbers is denoted by $\mathbb{C}$
\begin{align}
    \mathbb{C}=\{(a,b):a,b \in \mathbb{R}\}
\end{align}
{Rational Numbers:}
A number in the form $\frac{p}{q}$, where both p and q(non-zero) are integers, is called a rational number.The set of rational numbers is dentoed by $\mathbb{Q}$
Let $\mathbb{Q}$ be the set of rational numbers.
\begin{align}
    \mathbb{Q}=\left\{\frac{p}{q}:p \in \mathbb{Z},q \in \mathbb{Z}_{\not=0} \right\}\label{eq:solutions/1/1/7/q}
\end{align}
Let $\mathbb{C}$ be the field of complex numbers and given $\mathbb{F}$ be the subfield of field of complex numbers $\mathbb{C}$ 
Since $\mathbb{F}$ is the subfield , we could say that 
\begin{align}
    0 &\in \mathbb{F} \label{eq:solutions/1/1/7/0}\\
    1 &\in \mathbb{F}
\end{align}
{Closed under addition:}
Here $\mathbb{F}$ is closed under addition since it is subfield
\begin{align}
    1+1=2&\in \mathbb{F}\\
    1+1+1=3&\in \mathbb{F}\\
    \vdots\notag\\
    1+1+\dots+1\text{(p times)}= p &\in \mathbb{F}\label{eq:solutions/1/1/7/p}\\
    1+1+\dots+1\text{(q times)}= q &\in \mathbb{F}\label{eq:solutions/1/1/7/q1}
\end{align}
By using the above property we could say that zero and other positive integers belongs to $\mathbb{F}$.Since $p$ and $q$ are integers we say,
\begin{align}
    p \in \mathbb{Z}\\
    q \in \mathbb{Z}\label{eq:solutions/1/1/7/0}
\end{align}
{Additive Inverse:}
Let $x$ be the positive integer belong $\mathbb{F}$ and by additive inverse we could say, 
\begin{align}
    \forall x &\in \mathbb{F}\label{eq:solutions/1/1/7/1}\\
    (-x) &\in \mathbb{F} \label{eq:solutions/1/1/7/2}
\end{align}
Therefore field $\mathbb{F}$ contains every integers. Let n be a integer then,
\begin{align}
    n \in \mathbb{Z} &\implies n \in \mathbb{F}\\
    \mathbb{Z} &\subseteq \mathbb{F}
\end{align}
Where $\mathbb{Z}$ is subset of $\mathbb{F}$
{Multiplicative Inverse:}
Every element except zero in the subfield $\mathbb{F}$ has an multiplicative inverse. From equation \eqref{eq:solutions/1/1/7/q1}, since q $\in \mathbb{F}$ we could say ,
\begin{align}
    \frac{1}{q} \in \mathbb{F} \quad{\text{and  }} q \not= 0\label{eq:solutions/1/1/7/3}
\end{align}
{Closed under multiplication:}
Also, $\mathbb{F}$ is closed under multiplication and thus,
from equation \eqref{eq:solutions/1/1/7/p} and \eqref{eq:solutions/1/1/7/3} we get , 
\begin{align}
    p\cdot\frac{1}{q} \in \mathbb{F}\\
\implies \frac{p}{q} \in \mathbb{F}\label{eq:solutions/1/1/7/proof}
\end{align}
where , $p \in \mathbb{Z}$ and $q \in \mathbb{Z}_{\not=0}$ (from equation \eqref{eq:solutions/1/1/7/0} and \eqref{eq:solutions/1/1/7/3})
{Conclusion}
From \eqref{eq:solutions/1/1/7/q} and \eqref{eq:solutions/1/1/7/proof} we could say , 
\begin{align}
    \mathbb{Q} \subseteq \mathbb{F}\label{eq:solutions/1/1/7/F}
\end{align}
From equation \eqref{eq:solutions/1/1/7/F} we could say that each subfield of the field of complex number contains every rational number
\begin{center}
    Hence Proved
\end{center}


\item The differentiation operator on the space of polynomials of degree less than or equal to 3 is represented in the natural ordered basis by the matrix,
\begin{align}
\vec{A} = \myvec{0&1&0&0\\0&0&2&0\\0&0&0&3\\0&0&0&0}
\end{align}
What is the Jordan form of this matrix? ($\mathbb{F}$ a subfield of the complex numbers.) 
%
%
\\
\solution

{Complex Numbers:}
A complex number is a number that can be expressed in the form $a + bi$, where a and b are real numbers, and i represents the imaginary unit, satisfying the equation $i^2 =-1$.The set of complex numbers is denoted by $\mathbb{C}$
\begin{align}
    \mathbb{C}=\{(a,b):a,b \in \mathbb{R}\}
\end{align}
{Rational Numbers:}
A number in the form $\frac{p}{q}$, where both p and q(non-zero) are integers, is called a rational number.The set of rational numbers is dentoed by $\mathbb{Q}$
Let $\mathbb{Q}$ be the set of rational numbers.
\begin{align}
    \mathbb{Q}=\left\{\frac{p}{q}:p \in \mathbb{Z},q \in \mathbb{Z}_{\not=0} \right\}\label{eq:solutions/1/1/7/q}
\end{align}
Let $\mathbb{C}$ be the field of complex numbers and given $\mathbb{F}$ be the subfield of field of complex numbers $\mathbb{C}$ 
Since $\mathbb{F}$ is the subfield , we could say that 
\begin{align}
    0 &\in \mathbb{F} \label{eq:solutions/1/1/7/0}\\
    1 &\in \mathbb{F}
\end{align}
{Closed under addition:}
Here $\mathbb{F}$ is closed under addition since it is subfield
\begin{align}
    1+1=2&\in \mathbb{F}\\
    1+1+1=3&\in \mathbb{F}\\
    \vdots\notag\\
    1+1+\dots+1\text{(p times)}= p &\in \mathbb{F}\label{eq:solutions/1/1/7/p}\\
    1+1+\dots+1\text{(q times)}= q &\in \mathbb{F}\label{eq:solutions/1/1/7/q1}
\end{align}
By using the above property we could say that zero and other positive integers belongs to $\mathbb{F}$.Since $p$ and $q$ are integers we say,
\begin{align}
    p \in \mathbb{Z}\\
    q \in \mathbb{Z}\label{eq:solutions/1/1/7/0}
\end{align}
{Additive Inverse:}
Let $x$ be the positive integer belong $\mathbb{F}$ and by additive inverse we could say, 
\begin{align}
    \forall x &\in \mathbb{F}\label{eq:solutions/1/1/7/1}\\
    (-x) &\in \mathbb{F} \label{eq:solutions/1/1/7/2}
\end{align}
Therefore field $\mathbb{F}$ contains every integers. Let n be a integer then,
\begin{align}
    n \in \mathbb{Z} &\implies n \in \mathbb{F}\\
    \mathbb{Z} &\subseteq \mathbb{F}
\end{align}
Where $\mathbb{Z}$ is subset of $\mathbb{F}$
{Multiplicative Inverse:}
Every element except zero in the subfield $\mathbb{F}$ has an multiplicative inverse. From equation \eqref{eq:solutions/1/1/7/q1}, since q $\in \mathbb{F}$ we could say ,
\begin{align}
    \frac{1}{q} \in \mathbb{F} \quad{\text{and  }} q \not= 0\label{eq:solutions/1/1/7/3}
\end{align}
{Closed under multiplication:}
Also, $\mathbb{F}$ is closed under multiplication and thus,
from equation \eqref{eq:solutions/1/1/7/p} and \eqref{eq:solutions/1/1/7/3} we get , 
\begin{align}
    p\cdot\frac{1}{q} \in \mathbb{F}\\
\implies \frac{p}{q} \in \mathbb{F}\label{eq:solutions/1/1/7/proof}
\end{align}
where , $p \in \mathbb{Z}$ and $q \in \mathbb{Z}_{\not=0}$ (from equation \eqref{eq:solutions/1/1/7/0} and \eqref{eq:solutions/1/1/7/3})
{Conclusion}
From \eqref{eq:solutions/1/1/7/q} and \eqref{eq:solutions/1/1/7/proof} we could say , 
\begin{align}
    \mathbb{Q} \subseteq \mathbb{F}\label{eq:solutions/1/1/7/F}
\end{align}
From equation \eqref{eq:solutions/1/1/7/F} we could say that each subfield of the field of complex number contains every rational number
\begin{center}
    Hence Proved
\end{center}

\item Let $\vec{N_1}$ and $\vec{N_2}$ be $6 \times 6$ nilpotent matrices over the field $\vec{F}$. Suppose that $\vec{N_1}$ and $\vec{N_2}$ have the same minimal polynomial and the same nullity. Prove that $\vec{N_1}$ and $\vec{N_2}$ are similar. Show that this is not true for $7 \times 7$ nilpotent matrices.
%
\\
\solution

{Complex Numbers:}
A complex number is a number that can be expressed in the form $a + bi$, where a and b are real numbers, and i represents the imaginary unit, satisfying the equation $i^2 =-1$.The set of complex numbers is denoted by $\mathbb{C}$
\begin{align}
    \mathbb{C}=\{(a,b):a,b \in \mathbb{R}\}
\end{align}
{Rational Numbers:}
A number in the form $\frac{p}{q}$, where both p and q(non-zero) are integers, is called a rational number.The set of rational numbers is dentoed by $\mathbb{Q}$
Let $\mathbb{Q}$ be the set of rational numbers.
\begin{align}
    \mathbb{Q}=\left\{\frac{p}{q}:p \in \mathbb{Z},q \in \mathbb{Z}_{\not=0} \right\}\label{eq:solutions/1/1/7/q}
\end{align}
Let $\mathbb{C}$ be the field of complex numbers and given $\mathbb{F}$ be the subfield of field of complex numbers $\mathbb{C}$ 
Since $\mathbb{F}$ is the subfield , we could say that 
\begin{align}
    0 &\in \mathbb{F} \label{eq:solutions/1/1/7/0}\\
    1 &\in \mathbb{F}
\end{align}
{Closed under addition:}
Here $\mathbb{F}$ is closed under addition since it is subfield
\begin{align}
    1+1=2&\in \mathbb{F}\\
    1+1+1=3&\in \mathbb{F}\\
    \vdots\notag\\
    1+1+\dots+1\text{(p times)}= p &\in \mathbb{F}\label{eq:solutions/1/1/7/p}\\
    1+1+\dots+1\text{(q times)}= q &\in \mathbb{F}\label{eq:solutions/1/1/7/q1}
\end{align}
By using the above property we could say that zero and other positive integers belongs to $\mathbb{F}$.Since $p$ and $q$ are integers we say,
\begin{align}
    p \in \mathbb{Z}\\
    q \in \mathbb{Z}\label{eq:solutions/1/1/7/0}
\end{align}
{Additive Inverse:}
Let $x$ be the positive integer belong $\mathbb{F}$ and by additive inverse we could say, 
\begin{align}
    \forall x &\in \mathbb{F}\label{eq:solutions/1/1/7/1}\\
    (-x) &\in \mathbb{F} \label{eq:solutions/1/1/7/2}
\end{align}
Therefore field $\mathbb{F}$ contains every integers. Let n be a integer then,
\begin{align}
    n \in \mathbb{Z} &\implies n \in \mathbb{F}\\
    \mathbb{Z} &\subseteq \mathbb{F}
\end{align}
Where $\mathbb{Z}$ is subset of $\mathbb{F}$
{Multiplicative Inverse:}
Every element except zero in the subfield $\mathbb{F}$ has an multiplicative inverse. From equation \eqref{eq:solutions/1/1/7/q1}, since q $\in \mathbb{F}$ we could say ,
\begin{align}
    \frac{1}{q} \in \mathbb{F} \quad{\text{and  }} q \not= 0\label{eq:solutions/1/1/7/3}
\end{align}
{Closed under multiplication:}
Also, $\mathbb{F}$ is closed under multiplication and thus,
from equation \eqref{eq:solutions/1/1/7/p} and \eqref{eq:solutions/1/1/7/3} we get , 
\begin{align}
    p\cdot\frac{1}{q} \in \mathbb{F}\\
\implies \frac{p}{q} \in \mathbb{F}\label{eq:solutions/1/1/7/proof}
\end{align}
where , $p \in \mathbb{Z}$ and $q \in \mathbb{Z}_{\not=0}$ (from equation \eqref{eq:solutions/1/1/7/0} and \eqref{eq:solutions/1/1/7/3})
{Conclusion}
From \eqref{eq:solutions/1/1/7/q} and \eqref{eq:solutions/1/1/7/proof} we could say , 
\begin{align}
    \mathbb{Q} \subseteq \mathbb{F}\label{eq:solutions/1/1/7/F}
\end{align}
From equation \eqref{eq:solutions/1/1/7/F} we could say that each subfield of the field of complex number contains every rational number
\begin{center}
    Hence Proved
\end{center}

\twocolumn
\item If $\vec{N}$ is a nilpotent 3 $\times$ 3 matrix over C, prove that $\vec{A}$ = $\vec{I}$ + $\frac{1}{2}\vec{N}$ - $\frac{1}{8}\vec{N}^{2}$ satisfies $\vec{A}^2$ = $\vec{I}$ + $\vec{N}$, i.e., $\vec{A}$ is a square root of $\vec{I}$ + $\vec{N}$. Use the binomial series for $(1 + t)^\frac{1}{2}$ to obtain a similar formula for a square root of $\vec{I}$ + $\vec{N}$, where $\vec{N}$ is any nilpotent n $\times$ n matrix over C.
%
\\
\solution

{Complex Numbers:}
A complex number is a number that can be expressed in the form $a + bi$, where a and b are real numbers, and i represents the imaginary unit, satisfying the equation $i^2 =-1$.The set of complex numbers is denoted by $\mathbb{C}$
\begin{align}
    \mathbb{C}=\{(a,b):a,b \in \mathbb{R}\}
\end{align}
{Rational Numbers:}
A number in the form $\frac{p}{q}$, where both p and q(non-zero) are integers, is called a rational number.The set of rational numbers is dentoed by $\mathbb{Q}$
Let $\mathbb{Q}$ be the set of rational numbers.
\begin{align}
    \mathbb{Q}=\left\{\frac{p}{q}:p \in \mathbb{Z},q \in \mathbb{Z}_{\not=0} \right\}\label{eq:solutions/1/1/7/q}
\end{align}
Let $\mathbb{C}$ be the field of complex numbers and given $\mathbb{F}$ be the subfield of field of complex numbers $\mathbb{C}$ 
Since $\mathbb{F}$ is the subfield , we could say that 
\begin{align}
    0 &\in \mathbb{F} \label{eq:solutions/1/1/7/0}\\
    1 &\in \mathbb{F}
\end{align}
{Closed under addition:}
Here $\mathbb{F}$ is closed under addition since it is subfield
\begin{align}
    1+1=2&\in \mathbb{F}\\
    1+1+1=3&\in \mathbb{F}\\
    \vdots\notag\\
    1+1+\dots+1\text{(p times)}= p &\in \mathbb{F}\label{eq:solutions/1/1/7/p}\\
    1+1+\dots+1\text{(q times)}= q &\in \mathbb{F}\label{eq:solutions/1/1/7/q1}
\end{align}
By using the above property we could say that zero and other positive integers belongs to $\mathbb{F}$.Since $p$ and $q$ are integers we say,
\begin{align}
    p \in \mathbb{Z}\\
    q \in \mathbb{Z}\label{eq:solutions/1/1/7/0}
\end{align}
{Additive Inverse:}
Let $x$ be the positive integer belong $\mathbb{F}$ and by additive inverse we could say, 
\begin{align}
    \forall x &\in \mathbb{F}\label{eq:solutions/1/1/7/1}\\
    (-x) &\in \mathbb{F} \label{eq:solutions/1/1/7/2}
\end{align}
Therefore field $\mathbb{F}$ contains every integers. Let n be a integer then,
\begin{align}
    n \in \mathbb{Z} &\implies n \in \mathbb{F}\\
    \mathbb{Z} &\subseteq \mathbb{F}
\end{align}
Where $\mathbb{Z}$ is subset of $\mathbb{F}$
{Multiplicative Inverse:}
Every element except zero in the subfield $\mathbb{F}$ has an multiplicative inverse. From equation \eqref{eq:solutions/1/1/7/q1}, since q $\in \mathbb{F}$ we could say ,
\begin{align}
    \frac{1}{q} \in \mathbb{F} \quad{\text{and  }} q \not= 0\label{eq:solutions/1/1/7/3}
\end{align}
{Closed under multiplication:}
Also, $\mathbb{F}$ is closed under multiplication and thus,
from equation \eqref{eq:solutions/1/1/7/p} and \eqref{eq:solutions/1/1/7/3} we get , 
\begin{align}
    p\cdot\frac{1}{q} \in \mathbb{F}\\
\implies \frac{p}{q} \in \mathbb{F}\label{eq:solutions/1/1/7/proof}
\end{align}
where , $p \in \mathbb{Z}$ and $q \in \mathbb{Z}_{\not=0}$ (from equation \eqref{eq:solutions/1/1/7/0} and \eqref{eq:solutions/1/1/7/3})
{Conclusion}
From \eqref{eq:solutions/1/1/7/q} and \eqref{eq:solutions/1/1/7/proof} we could say , 
\begin{align}
    \mathbb{Q} \subseteq \mathbb{F}\label{eq:solutions/1/1/7/F}
\end{align}
From equation \eqref{eq:solutions/1/1/7/F} we could say that each subfield of the field of complex number contains every rational number
\begin{center}
    Hence Proved
\end{center}

\item Use the result of Exercise 15 (that is, if $\vec{A}=\vec{I}+\frac{1}{2}\vec{N}-\frac{1}{8}\vec{N}^2$ then $\vec{A}^2 = \vec{I}+\vec{N}$) to prove that if $c$ is a non-zero complex number and $\vec{N}$ is a nilpotent complex matrix, then $(c\vec{I}+\vec{N})$ has a square root. Now use the Jordon form to prove that every non-singular complex $n \times n$ matrix has a square root.
 %
\\
\solution

{Complex Numbers:}
A complex number is a number that can be expressed in the form $a + bi$, where a and b are real numbers, and i represents the imaginary unit, satisfying the equation $i^2 =-1$.The set of complex numbers is denoted by $\mathbb{C}$
\begin{align}
    \mathbb{C}=\{(a,b):a,b \in \mathbb{R}\}
\end{align}
{Rational Numbers:}
A number in the form $\frac{p}{q}$, where both p and q(non-zero) are integers, is called a rational number.The set of rational numbers is dentoed by $\mathbb{Q}$
Let $\mathbb{Q}$ be the set of rational numbers.
\begin{align}
    \mathbb{Q}=\left\{\frac{p}{q}:p \in \mathbb{Z},q \in \mathbb{Z}_{\not=0} \right\}\label{eq:solutions/1/1/7/q}
\end{align}
Let $\mathbb{C}$ be the field of complex numbers and given $\mathbb{F}$ be the subfield of field of complex numbers $\mathbb{C}$ 
Since $\mathbb{F}$ is the subfield , we could say that 
\begin{align}
    0 &\in \mathbb{F} \label{eq:solutions/1/1/7/0}\\
    1 &\in \mathbb{F}
\end{align}
{Closed under addition:}
Here $\mathbb{F}$ is closed under addition since it is subfield
\begin{align}
    1+1=2&\in \mathbb{F}\\
    1+1+1=3&\in \mathbb{F}\\
    \vdots\notag\\
    1+1+\dots+1\text{(p times)}= p &\in \mathbb{F}\label{eq:solutions/1/1/7/p}\\
    1+1+\dots+1\text{(q times)}= q &\in \mathbb{F}\label{eq:solutions/1/1/7/q1}
\end{align}
By using the above property we could say that zero and other positive integers belongs to $\mathbb{F}$.Since $p$ and $q$ are integers we say,
\begin{align}
    p \in \mathbb{Z}\\
    q \in \mathbb{Z}\label{eq:solutions/1/1/7/0}
\end{align}
{Additive Inverse:}
Let $x$ be the positive integer belong $\mathbb{F}$ and by additive inverse we could say, 
\begin{align}
    \forall x &\in \mathbb{F}\label{eq:solutions/1/1/7/1}\\
    (-x) &\in \mathbb{F} \label{eq:solutions/1/1/7/2}
\end{align}
Therefore field $\mathbb{F}$ contains every integers. Let n be a integer then,
\begin{align}
    n \in \mathbb{Z} &\implies n \in \mathbb{F}\\
    \mathbb{Z} &\subseteq \mathbb{F}
\end{align}
Where $\mathbb{Z}$ is subset of $\mathbb{F}$
{Multiplicative Inverse:}
Every element except zero in the subfield $\mathbb{F}$ has an multiplicative inverse. From equation \eqref{eq:solutions/1/1/7/q1}, since q $\in \mathbb{F}$ we could say ,
\begin{align}
    \frac{1}{q} \in \mathbb{F} \quad{\text{and  }} q \not= 0\label{eq:solutions/1/1/7/3}
\end{align}
{Closed under multiplication:}
Also, $\mathbb{F}$ is closed under multiplication and thus,
from equation \eqref{eq:solutions/1/1/7/p} and \eqref{eq:solutions/1/1/7/3} we get , 
\begin{align}
    p\cdot\frac{1}{q} \in \mathbb{F}\\
\implies \frac{p}{q} \in \mathbb{F}\label{eq:solutions/1/1/7/proof}
\end{align}
where , $p \in \mathbb{Z}$ and $q \in \mathbb{Z}_{\not=0}$ (from equation \eqref{eq:solutions/1/1/7/0} and \eqref{eq:solutions/1/1/7/3})
{Conclusion}
From \eqref{eq:solutions/1/1/7/q} and \eqref{eq:solutions/1/1/7/proof} we could say , 
\begin{align}
    \mathbb{Q} \subseteq \mathbb{F}\label{eq:solutions/1/1/7/F}
\end{align}
From equation \eqref{eq:solutions/1/1/7/F} we could say that each subfield of the field of complex number contains every rational number
\begin{center}
    Hence Proved
\end{center}



\end{enumerate}
